\section{Условия монотонности и постоянства функций}

\begin{theorem}[условие постоянства функции]
	$f \in C \big( (a, b) \big), \qquad \forall x \in (a, b) \quad \exist f'(x) $
    \begin{equ}1
        \underset{\text{(то есть, } f(x) = \const)}{\forall x_0, x \in (a, b) \quad f(x) = f(x_0)}
    \end{equ}
    \begin{equ}2
        \iff \forall x \in (a, b) \quad f'(x) = 0
    \end{equ}
\end{theorem}

\begin{proof}
    \hfill
    \begin{itemize}
    	\item $\implies$ \\
        $ \eref1 \implies \eref2 $, так как $ c' \equiv 0 $
        \item $\impliedby$ \\
        Пусть $x \ne x_0 $ \\
        По теореме Лагранжа, $ \exist x_1 $ между $x_0$ и $x : f(x) - f(x_0) = \underbrace{f'(x_1)}_{=0}(x - x_0) = 0 $
    \end{itemize}
\end{proof}

\begin{theorem}[условие возрастания и убывания функции]
	$ f \in C \big( [a, b] \big), \qquad \forall x \in (a, b) \quad \exist f'(x) $
    \begin{equ}3
        \forall x_1 < x_2 \in [a, b] \quad f(x_1) \le f(x_2)
    \end{equ}
    \begin{equ}4
        \iff \forall x \in (a, b) \quad f'(x) \ge 0
    \end{equ}
    $ g \in C \big( [a, b] \big), \qquad \forall x \in (a, b) \quad \exist g'(x) $
    \begin{equ}{31}
    	\forall x_1 < x_2 \in [a, b] \quad g(x_1) \ge g(x_2)
    \end{equ}
    \begin{equ}{41}
    	\iff \forall x \in (a, b) \quad g'(x) \le 0
    \end{equ}
\end{theorem}

\begin{proof}
	\hfill
    \begin{itemize}
    	\item $f(x)$
        \begin{itemize}
        	\item $\implies$ \\
            Пусть выполнено \eref3 \\
            Рассмотрим любую точку $x \in (a, b) $ \\
            Возьмём $h > 0 : x + h < b $
            \begin{multline*}
                \eref3 \implies f(x + h) \ge f(x) \iff \frac{f(x + h) - f(x)}h \ge 0 \implies \liml{h \to 0+} \frac{f(x + h) - f(x)}h \ge 0 \iff \\ \iff f'(x) \ge 0
            \end{multline*}
            \item $\impliedby$ \\
            Пусть выполнено \eref4 \\
            Возьмём $ x_1, x_2 \in [a, b] $ \\
            По теореме Лагранжа,
            $$ \exist x_3 \in (x_1, x_2) : f(x_2) - f(x_1) = \underbrace{f'(x_3)}_{\ge 0}\underbrace{(x_2 - x_1)}_{\ge 0} \ge 0 \iff f(x_2) \ge f(x_1) $$
        \end{itemize}
        \item $g(x)$ \\
        Рассмотрим функцию $f(x) \define -g(x)$ \\
        По свойствам производной, $ \forall x \in (a, b) \quad f'(x) = -g'(x) $
        $$ g(x_2) \le g(x_1) \iff -g(x_2) \ge -g(x_1) \iff f(x_2) \ge f(x_1) $$
        По уже доказанному, получаем, что $\eref{31} \iff \eref{41} $
    \end{itemize}
\end{proof}

\begin{theorem}[о строгом возрастании и строгом убывании функции]
	$ f \in C \big( [a, b] \big), \qquad \forall x \in (a, b) \quad \exist f'(x) $
    $$ \forall x_1 < x_2 \in [a, b] \quad f(x_1) < f(x_2) $$
    \begin{mequ}[\iff \empheqlbrace]
        \lbl5 \forall x \in (a, b) \quad f'(x) \ge 0 \\
        \lbl6 \not\exist (\alpha, \beta) \sub (a, b) : \forall x \in (\alpha, \beta) \quad f'(x) = 0
    \end{mequ}
    $ g \in C \big( [a, b] \big), \qquad \forall x \in (a, b) \quad \exist g'(x) $
    $$ \forall x_1 < x_2 \in [a, b] \quad f(x_1) > f(x_2) \iff
    \begin{cases}
        \forall x \in (a, b) \quad g'(x) \le 0 \\
        \not\exist (\alpha_1, \beta_1) \sub (a, b) : \forall x \in (\alpha_1, \beta_1) \quad g'(x) = 0
    \end{cases} $$
\end{theorem}

\begin{proof}
    \hfill
    \begin{itemize}
    	\item $f(x)$
        \begin{itemize}
            \item $\implies$ \\
            Пусть $f$ строго возрастает \\
            По предыдущей теореме, выполнено \eref5 \\
            Пусть \textbf{не} выполнено \eref6, то есть $ \exist (\alpha, \beta) \sub (a, b) : \forall x \in (\alpha, \beta) \quad f'(x) = 0 $ \\
            По теореме об условии постоянства функции, $ \forall x_1 < x_2 \in (\alpha, \beta) \quad f(x_1) = f(x_2) $ -- \contra с предволожением, что $f$ строго возрастает
            \item $\impliedby$ \\
            Пусть выполнены \eref5 и \eref6 \\
            Докажем, что $f$ строго возрастает:
            \begin{equ}{81}
                \eref5 \implies \forall x_1 < x_2 \in [a, b] \quad f(x_1) \le f(x_2)
            \end{equ}
            То есть, $f$ возрастает \\
            Предположим, что $f$ \textbf{не строго} возрастает, то есть
            \begin{equ}8
                \exist x', x'' \in [a, b] : f(x') = f(x'')
            \end{equ}
            \begin{multline*}
                \eref{81}, \eref8 \implies \forall x \in (x', x'') \quad f(x') \le f(x) \le f(x'') \iff \forall x \in (x', x'') \quad f(x) = f(x') \implies \\ \implies \forall x \in (x', x'') \quad f'(x) = 0 \text{ -- } \contra
            \end{multline*}
        \end{itemize}
        \item $g(x)$ \\
        Определим функцию $f(x) \define -g(x) $ \\
        $g$ строго убывает $ \iff f$ строго возрастает \\
        $ f'(x) = 0 \iff g'(x) = 0 $
    \end{itemize}
\end{proof}

\section{Получение некоторых неравенств}

\begin{statement}
	$ \forall x \in [0, \dfrac\pi2] \qquad \sin x \ge \dfrac2\pi x $
\end{statement}

\begin{proof}
	Рассомтрим функцию $ f(x) \define
    \begin{cases}
    	1, \quad x = 0 \\
        \dfrac{\sin x}x, \quad 0 < x \le \dfrac\pi2
    \end{cases} $ \\
    По замечательному пределу для $ \dfrac{\sin x}x $, $ f \in C \big( [0, \dfrac\pi2] \big) $ \\
    Найдём производную $f$:
    $$ f'(x) = \frac{\sin' x \cdot x - \sin x \cdot x'}{x^2} = \frac{x \cos x - \sin x}{x^2} = \frac{\cos x}{x^2} (x - \tg x) $$
    При доказательстве существенного неравенства для $\sin x$ было доказано, что $ \forall x \in (0, \dfrac\pi2) \quad x < \tg x $ \\
    Значит, $ f'(x) = \underbrace{\dfrac{\sin x}{x^2}}_{> 0} \underbrace{(x - \tg x)}_{ < 0} < 0 $ \\
    По теореме о строгом убывании, $f$ строго убывает, то есть $ \forall x \in [0, \dfrac\pi2) \quad f(x) > f(\dfrac\pi2) $
    $$ f(\frac\pi2) = \frac{\sin \frac\pi2}{\frac\pi2} = \frac2\pi $$
\end{proof}

\begin{statement}
	$ \forall x > -1 \qquad \ln (1 + x) \le x $ \\
    Если $x \ne 0$, то $ \ln (1 + x) < x $
\end{statement}

\begin{proof}
	Возьмём $ -1 < a < 0 < b $ \\
    Рассмотрим $ f(x) \define \ln(1 + x) - x, \quad x \in [a, b] $
    $$ f'(x) = \frac1{1 + x} - 1 = -\frac{x}{1 + x} $$
    \begin{itemize}
    	\item Рассмотрим отрезок $[a, 0]$:
        $$ \forall x \in [a, 0] \quad
        \begin{cases}
            -x > 0 \\
            1 + x > 0
        \end{cases} \implies f(x) \text{ строго возрастает на } [a, 0] $$
        То есть, $ \forall x \in [a, 0] \quad
        \begin{cases}
            f(x) \le f(0) = 0 \\
            f(x) < f(x) \text{, если } x \ne 0
        \end{cases} $
        \item Рассмотрим отрезок $[0, b]$:
        $$ \forall x \in [0, b] \quad
        \begin{cases}
        	-x < 0 \\
            1 + x > 0
        \end{cases} \implies f(x) \text{ строго убывает на } [0, b] $$
        То есть, $\forall x \in [0, b] \quad
        \begin{cases}
        	f(x) \le f(0) = 0 \\
            f(x) < f(0) \text{, если } x \ne 0
        \end{cases} $
    \end{itemize}
    Получили, что $ \forall x \in [a, b] \quad f(x) \le 0 \iff \ln(1 + x) \le x $
\end{proof}

\begin{statement}[неравенство Бернулли]
	$ \alpha > 1 $ \\
    $ \forall x > -1 \qquad (1 + x)^\alpha \ge 1 + \alpha x $ \\
    Если $ x \ne 0 $, то $ (1 + x)^\alpha > 1 + \alpha x $
\end{statement}

\begin{proof}
    Рассмотрим $f(x) \define (1 + x)^\alpha - \alpha x $ для $ x \in [a, b] $\footnote{В дальнейшем нам понадобятся \textbf{замкнутые} промежутки, поэтому мы ``покрываем'' ими $(-1, +\infty)$}
    \item Рассотрим $ -1 < a < 0 < b $ \\
    Теперь $ f'(x) = \alpha(1 + x)^{\alpha - 1} - \alpha = \alpha \bigg( (1 + x)^{\alpha - 1} - 1 \bigg) $ \\
    Обозначим $ \alpha - 1 \define \beta > 0 $
    \begin{itemize}
        \item $ -1 < x < 0 $
        $$ 1 + x < 1 $$
        $$ (1 + x)^\beta < 1^\beta $$
        $$ (1 + x)^\beta - 1 < 0 $$
        То есть, $f$ строго убывает на $[a, 0]$ и $f(x) < f(0) = 1 $
        \item $0 < x$
        $$ 1 + x > 1 $$
        $$ (1 + x)^\beta > 1^\beta $$
        $$ (1 + x)^\beta - 1 > 0 $$
        То есть, $f$ строго возрастает на $[a, b]$ и $f(x) > f(0) = 1 $
    \end{itemize}
    Полуичили, что $f(x) \le 1$
\end{proof}

\begin{statement}[неравенство Бернулли]
	$ 0 < \alpha < 1 $ \\
    $ \forall x > -1 \qquad (1 + x)^\alpha \le 1 + \alpha x $ \\
    Если $ x \ne 0 $, то $ (1 + x)^\alpha < 1 + \alpha x $
\end{statement}

\begin{proof}
	Доказательство точно такое же, однако $ \alpha - 1 \define \beta < 0 $ и:
    \begin{itemize}
        \item $ -1 < x < 0 $
        $$ 1 + x < 1 $$
        $$ (1 + x)^\beta > 1^\beta $$
        $$ (1 + x)^\beta - 1 > 0 $$
        То есть, $f$ строго возрастает на $[a, 0]$ и $f(x) > f(0) = 1 $
        \item $0 < x$
        $$ 1 + x > 1 $$
        $$ (1 + x)^\beta > 1^\beta $$
        $$ (1 + x)^\beta - 1 > 0 $$
        То есть, $f$ строго убывает на $[a, b]$ и $f(x) < f(0) = 1 $
    \end{itemize}
    Получили, что $f(x) \le 1 $
\end{proof}

\section{Выпуклые и вогнутые функции}

\begin{definition}
	$ f \in C \big( [a, b] \big) $ \\
    $ f$ выпукла $ \iff \forall x_1, x_2 \in [a, b]$ и $ \forall t_1, t_2 > 0 : t_1 + t_2 = 1 $
    \begin{equ}{11}
        f(t_1x_1 + t_2x_2) \le t_1f(x_1) + t_2f(x_2)
    \end{equ}
    $ g \in C \big( [a, b] \big) $ \\
    $g$ вогнута $ \iff \forall x_1, x_2 \in [a, b]$ и $ \forall t_1, t_2 > 0 : t_1 + t_2 = 1 $
    \begin{equ}{12}
    	g(t_1x_1 + t_2x_2) \ge t_1g(x_1) + t_2g(x_2))
    \end{equ}
\end{definition}

\begin{statement}
	$f$ выпукла $ \implies -f$ вогнута \\
    $g$ вогнута $ \implies -g $ выпукла
\end{statement}

\begin{proof}
	$f$ выпукла $ \iff $ выполнено \eref1 \\
    Домножим \eref1 на $-1$:
    $$ -f(t_1x_1 + t_2x_2) \ge t_1(-f(x_1)) + t_2(-f(x_2)) $$
\end{proof}

\begin{theorem}[о характеристике выпуклых и вогнутых функций в терминах производной]
    \hfill \\
	$f \in C \big( [a, b] \big), \qquad \forall x \in (a, b) \quad \exist f'(x) $ \\
    $f$ выпукла $ \iff f'(x) $ возрастает \\
    $g \in C \big( [a, b] \big), \qquad \forall x \in (a, b) \quad \exist g'(x) $ \\
    $g$ вогнута $ \iff g'(x) $ убывает
\end{theorem}

\begin{proof}
	\hfill
    \begin{itemize}
        \item $f(x)$
        \begin{itemize}
        	\item $\impliedby$ \\
            Пусть $f$ возрастает \\
            НУО положим $ a \le x_1 < x_2 \le b $ \\
            Нужно доказать \eref{11}
            \begin{multline*}
                \eref1 \iff (t_1 + t_2)f(t_1x_1 + t_2x_2) \le t_1f(x_1) + t_2f(x_2) \iff \\ \iff t_1 \bigg( f(t_1x_1 + t_2x_2) - f(x_1) \bigg) \le t_2 \bigg( f(x_2) - f(t_1x_1 + t_2x_2) \bigg)
            \end{multline*}
            Обозначим $ x \define t_1x_1 + t_2x_2 $
            $$ t_1 > 0 \implies x > t_1x_1 + t_2x_1 = x_1 $$
            $$ x < t_1x_2 + t_2x_2 = x_2 $$
            То есть, $ x_1 < x < x_2 $ \\
            По теореме Лагранжа,
            $$ \exist c_1 \in (x_1, x) : f(x) - f(x_1) = f'(c_1) \cdot (x - x_1) $$
            $$ \exist c_2 \in (x_1, x_2) : f(x_2) - f(x) = f'(c_2) \cdot (x_2 - x) $$
            Теперь нужно доказать, что
            \begin{equ}{17}
                t_1f'(c_1)(x - x_1) \le t_2f'(c_2)(x_2 - x)
            \end{equ}
            Если выполнено \eref{17}, то выполнено и \eref{11}
            \begin{equ}{171}
            	t_1 + t_2 \bydef 1 \implies
                \begin{cases}
                	t_1 - 1 = -t_2 \\
                    1 - t_2 = t_1
                \end{cases}
            \end{equ}
            \begin{equ}{172}
                x - x_1 \bydef t_1x_1 + t_2x_2 - x_1 = x_1(t_1 - 1) + t_2x_2 \underset{\eref{171}}= -t_2x_1 + t_2x_2 = t_2(x_2 - x_1)
            \end{equ}
            \begin{equ}{173}
                x_2 - x \bydef x_2 - (t_1x_1 + t_2x_2) = (1 - t_2)x_2 - t_1x_1 = t_1x_2 - t_1x_1 = t_1(x_2 - x_1)
            \end{equ}
            Подставим \eref{172} и \eref{173} в \eref{17}:
            $$ t_1t_2(x_2 - x_1)f'(c_1) \le t_1t_2(x_2 - x_1)f'(c_2) $$
            То есть нужно проверить, что $f'(c_1) \le f'(c_2) $ \\
            При этом, $c_1 \le x \le c_2 \implies c_1 \le c_2 \implies f'(c_1) \le f/(c_2) $
            \item $\implies$ \\
            Пусть $f$ выпукла \\
            Возьмём $ a < x_1 < x_2 < b $ \\
            Возьмём $ x_1 < x < x_2 $ \\
            Положим $ t_1 \define \dfrac{x_2 - x}{x_2 - x_1}, \quad t_2 \define \dfrac{x - x_1}{x_2 - x_1} $
            $$ t_1x_1 + t_2x_2 = \frac{x_2 - x}{x_2 - x_1}x_1 + \frac{x - x_1}{x_2 - x_1}x_2 = \frac{x_1x_2 - xx_1 + xx_2 - x_1x_2}{x_2 - x_1} = x $$
            Подставим в \eref{11}:
            $$ f(x) \le \frac{x_2 - x}{x_2 - x_1}f(x_1) + \frac{x - x_1}{x_2 - x_1}f(x_2) $$
            $$ \underbrace{\bigg( \frac{x_2 - x}{x_2 - x_1} + \frac{x - x_1}{x_2 - x_1} \bigg)}_{= t_1 + t_2 = 1}f(x) \le \frac{x_2 - x}{x_2 - x_1}f(x_1) + \frac{x - x_1}{x_2 - x_1}f(x_2) $$
            $$ \frac{x_2 - x}{x_2 - x_1} \bigg( f(x) - f(x_1) \bigg) \le \frac{x - x_1}{x_2 - x_1} \bigg( f(x_2) - f(x) \bigg) $$
            $$ (x_2 - x) \bigg( f(x) - f(x_1) \bigg) \le (x - x_1) \bigg( f(x_2) - f(x) \bigg) $$
            \begin{equ}{110}
                \frac{f(x) - f(x_1)}{x - x_1} \le \frac{f(x_2) - f(x)}{x_2 - x}
            \end{equ}
            Перейдём к пределу:
            \begin{itemize}
            	\item $$ \liml{x \to x_1 + 0} \frac{f(x) - f(x_1)}{x - x_1} \le \liml{x \to x_1 + 0} \frac{f(x_2) - f(x)}{x_2 - x} $$
                \begin{equ}{111}
                    f'(x_1) \le \frac{f(x_2) - f(x_1)}{x_2 - x_1}
                \end{equ}
                \item $$ \liml{x \to x_2 - 0} \frac{f(x) - f(x_1)}{x - x_1} \le \liml{x \to x_2 - 0} \frac{f(x_2) - f(x)}{x_2 - x} $$
                \begin{equ}{112}
                    \frac{f(x_2) - f(x_1)}{x_2 - x_1} \le f'(x_2)
                \end{equ}
            \end{itemize}
            $$ \eref{111}, \eref{112} \implies f'(x_2) $$
        \end{itemize}
        \item $g(x)$ \\
        Рассмотрим $f \define -g$
    \end{itemize}
\end{proof}

\begin{theorem}[характеристика выпуклых и вогнутых функций в терминах второй производной]
    \hfill \\
	$f \in C \big( [a, b] \big), \qquad \forall x \in (a, b) \quad \exist f''(x) $ \\
    $f$ выпукла $ \iff \forall x \in (a, b) \quad f''(x) \ge 0 $ \\
    $g \in C \big( [a, b] \big), \qquad \forall x \in (a, b) \quad \exist g''(x) $ \\
    $g$ вогнута $ \iff \forall x \in (a, b) \quad g''(x) \le 0 $
\end{theorem}

\begin{proof}
	$f'(x)$ возрастает $ \iff \forall x \in (a, b) \quad (f')'(x) = f''(x) \ge 0 $
\end{proof}

\section{Неравенство Йенсена}

\begin{theorem}
	$ f \in C \big( [a, b] \big), \qquad f $ выпукла, $ \quad \underset{
        \begin{subarray}{c}
        	\forall k = 1...n \quad t_k > 0 \\
            t_1 + ... + t_n = 1
        \end{subarray}}{\forall t_1, ..., t_n,} \qquad \forall x_1, ..., x_n \in [a, b] $
    \begin{equ}{115}
    	f(t_1x_1 + t_2x_2 + ... + t_nx_n) \le t_1f(x_1) + t_2f(x_2) + ... + t_nf(x_n)
    \end{equ}
    $ g \in C \big( [a, b] \big), \qquad g $ вогнута, $ \quad \underset{
        \begin{subarray}{c}
        	\forall k = 1...n \quad t_k > 0 \\
            t_1 + ... + t_n = 1
        \end{subarray}}{\forall t_1, ..., t_n,} \qquad \forall x_1, ..., x_n \in [a, b] $
    \begin{equ}{116}
    	g(t_1x_1 + t_2x_2 + ... + t_nx_n) \ge t_1g(x_1) + t_2g(x_2) + ... + t_ng(x_n)
    \end{equ}
\end{theorem}

\begin{proof}
    \textbf{Индукция}
    \begin{itemize}
        \item \textbf{База.} $n = 2$ -- определение выпуклости
        \item \textbf{Переход.} $n \to n + 1$
        $$ t_1 + t_2 + ... + t_n + t_{n + 1} = 1 $$
        \begin{equ}{117}
            \text{Определим числа } \vawe{t_n} \define t_n + t_{n + 1}, \quad \vawe{x_n} = t_nx_n + t_{n + 1}x_{n + 1}
        \end{equ}
        $$ \eref{117} \implies
        \begin{cases}
            t_1 + ... + \vawe{t_n} = 1 \\
            t_1x_1 + ... + \vawe{t_n}\vawe{x_n} = t_1x_1 + ... + t_nx_n + t_{n + 1}x_{n + 1}
        \end{cases} $$
        В силу индукционного предположения, $ \underbrace{(t_1x_1 + ... + \vawe{t_n}\vawe{x_n})}_{= f(t_nx_1 + ... + t_nx_n + t_{n + 1}x_{n + 1})} \le t_1f(x_1) + ... + \vawe{t_n}f(\vawe{x_n}) $
        $$ \vawe{x_n} = \frac{t_n}{\vawe{t_n}}x_n + \frac{t_{n + 1}}{\vawe{t_n}} $$
        $$ \vawe{t_n}f(\vawe{x_n}) = \vawe{t_n}f \bigg( \frac{t_n}{\vawe{t_n}}x_n + \frac{t_{n + 1}{\vawe{t_n}}}x_{n + 1} \bigg) \le \vawe{t_n} \frac{t_n}{\vawe{t_n}}f(x_n) + \vawe{t_n} \frac{t_{n + 1}}{\vawe{t_n}}f(x_{n + 1}) = t_nf(x_n) + t_{n + 1}f(x_{n + 1}) $$
    \end{itemize}
\end{proof}
