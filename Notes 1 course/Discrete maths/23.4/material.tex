\chapter{Классы P и NP}

\section{Теория сложности алгоритмов}

\begin{definition}
	NP -- класс задач, для которых существует полиномиальный алгоритм на параллельной машине фон-Нейнмана
\end{definition}

\begin{undefthm}{Другими словами}
	NP -- класс всех дискретных задач
\end{undefthm}

\begin{undefthm}{Другими словами}
	NP -- те задачи, которые решаются перебором
\end{undefthm}

\begin{definition}
	P - класс задач, для которых существует полиномиальный алгоритм на машине с ограниченным количеством ячеек памяти и команд
\end{definition}

\begin{definition}
	Длина входа -- количество ячеек памяти, которое требуется, чтобы записать исходные данные задачи
\end{definition}

\begin{definition}
	Задача называется NP-полной, если к ней полиномиально сводится любая задача класса NP
\end{definition}

\begin{definition}
    Говорят, что $ A \xrightarrow{P} B $ (задча $ A $ полиномиально сводится к задаче $ B $), если исходные данные $ A $ можно привести к исходным данным $ B $, а потом результат $ B $ привести к результату $ A $
\end{definition}

\begin{undefthm}{Гипотеза}
	P = NP или P $ \ne $ NP? \\
    Есть ли NP задачи, которые \textbf{не} P и \textbf{не} NP-полные?
\end{undefthm}

\section{Метод ветвей и границ (B\&B)}

\begin{problem}[коммивояжёра]
	$ n $ городов \\
    $ c[i, j] $ -- матрица расстояний между городами \\
    Найти маршрут между всеми городами так, чтобы маршрут был минимальным
\end{problem}

\begin{problem}[о рюкзаке]
	Есть $ n $ предметов \\
    Для каждого предмета известны вес и цена \\
    $ b $ -- максимальный вес, который можно унести \\
    Найти набор предметов, чтобы суммарная цена была максимальной
\end{problem}

\begin{algo}[B\&B]
    \item Строим дерево полного перебора
    \item Придумываем способ оценки частичного решения
    \item Обрезаем те части дерева, где оценка хуже рекорда
    \item Контроль за допустимостью частичного решения
    \item Обходим получившееся дерево, везде считаем оценку, обновляем рекорд
\end{algo}

\begin{eg}
	Матрица расстояний:
    $$
    \begin{matrix}
    	\infty & 13 & 82 & 70 & 63 & 53 \\
        30 & \infty & 18 & 36 & 38 & 5 \\
        8 & 85 & \infty & 82 & 73 & 34 \\
        66 & 69 & 78 & \infty & 59 & 42 \\
        4 & 31 & 91 & 37 & \infty & 56 \\
        6 & 24 & 16 & 28 & 99 & \infty
    \end{matrix} $$
    В качестве оценки будем использовать спуск до нуля (сколько вычитаем -- столько и оценка) \\
    Оценки:
    $$
    \begin{matrix}
    	\infty & 0 & 59 & 35 & 33 & 40 \\
        25 & \infty & 3 & 9 & 16 & 0 \\
        0 & 77 & \infty & 52 & 48 & 26 \\
        24 & 27 & 26 & \infty & 0 & 0 \\
        0 & 27 & 77 & 11 & \infty & 52 \\
        0 & 18 & 0 & 0 & 76 & \infty
    \end{matrix} $$
\end{eg}
