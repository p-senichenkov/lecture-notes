\chapter{Графы}

\section{Критический путь в сетевом графике}

Дан сетевой график $ G = \langle M, N \rangle $. Для каждой дуги задана $ t(u) \ge 0 $. $ N $ -- работы \\
\textbf{Максимальный} путь показывает \textbf{минимальное} время работ (если максимально распараллелить) \\
Можно свести задачу к алгоритму Дейкстры, если все веса домножить на $ -1 $. Но мы так делать не будем

\begin{algo}
	\item Топологическая сортировка
    \item $ \forall i \in M \quad v[i] \define 0 $
    \item
    \begin{algorithm2e}[H]
        \For{$ i \define 1 $ to $ w $}
        {\For{$ u \in N_i^- $}
            {\uIf{$ v[ $end($ u $)$ ] < v[i] + t[u] $}
                {$ v[ $end($ u $)$ ] \define v[i] + t[u] $}}}
    \end{algorithm2e}
    На выходе получаем вектор $ v $, где $ v[i] $ -- максимальная длина пути из $ i_0 $ (начальная фиктивная вершина) в $ i $. Тогда $ v[i_+] $ (конечная фиктивная вершина) -- длина критического пути
\end{algo}

Пусть теперь задано $ W $ -- отношение частичного порядка \\
$ N $ -- множество работ
\begin{algo}
	\item Строим транзитивное замыкание $ W $
    \item $ u \in N \to \pi_i, \qquad i \in 1 : p $
    \item У нас есть разбиения $ D_i = \set{\pi_i, N \setminus \pi_i} $
    \item Строим произведение разбиений $ D_i $. Обозначим его $ D \define \bigcup_j \rho_j $
    \item Строим граф $ G = \langle M_1 \cup M_2, N \cup F \rangle $, где:
    \begin{itemize}
    	\item Каждое $ \pi_i $ даст нам вершину из $ M_1 $
        \item Каждое $ \rho_j $ даст нам вершину из $ M_2 $
        \item $ N $ -- наши изначальные работы
        \item $ F $ -- фиктивные работы
    \end{itemize}
    Процесс построения графа:
    \begin{enumerate}
        \item $ u = (\operatorname{beg} u, \operatorname{end} u) $, где:
        \begin{itemize}
            \item $ \operatorname{beg} u = \pi_i(u) $
            \item $ \operatorname{end} u = \rho_j, \quad u \in \rho_j $
        \end{itemize}
        \item Добавляем фиктивные дуги $ F $, которые будут вести из $ M_2 $ в $ M_1 $ \\
        Каждая дуга будет $ \rho_j \to \pi_i $ так, что $ \pi_i = \bigcap \rho_j $
        \item Удаляем лишние фиктивные дуги по двум правилам:
        \begin{itemize}
        	\item Если фиктивная дуга соединяет две вершины, между которыми есть другой путь
            \item Если из какой-то вершины выходит ровно одна фиктивная дуга, то можно склеить эти вершины \\
            Если входит ровно одна -- аналогично
        \end{itemize}

    \end{enumerate}


\end{algo}
