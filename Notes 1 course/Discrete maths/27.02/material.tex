\chapter{Графы}

\section{Обход в ширину и обход в глубину}

\begin{algorithm}[обход в ширину]
	$ G = \langle M, N \rangle $ \\
    Задан корень $ x \in M $ \\
    $ M = M_0 \cup \underset{\text{фронт}}{M_1} \cup M_2 $ \\
    Начальное состояние: $ M_0 = \O, \quad M_1 = \set{x} $ \\
    Алгоритм работает, пока фронт не пуст
    \begin{enumerate}
    	\item Выбираем $ y \in M_1 $
        \item Просматриваем все рёбра $ e = (y, v) \in N_y $ (выходящие из $y$):
        \begin{enumerate}
            \item $ v \in M_0 $ -- ничего не делаем
            \item $ v \in M_1 $ -- ничего не делаем
            \item $ v \in M_2 $ -- перемещаем $ v $ в $M_1$
        \end{enumerate}
        \item Перемещаем $y$ в $M_0$
    \end{enumerate}
\end{algorithm}

\begin{algorithm}[обход в глубину]
	$ G = \langle M, N \rangle $ \\
    Задан корень $ x \in M $
    \begin{verbatim}
        for u \in V
        {
            color(u) = white
        }
        for u \in V
        {
            if color(u) == white
            {
                DFS(u)
            }
        }
        DFS(u)
        color(u) = gray
        for (для всех дуг e = (u, w))
        {
            if color(w) == white
            {
                DFS(w)
            }
        }
        color($u$) = black
    \end{verbatim}
\end{algorithm}

\begin{theorem}[о связном подграфе]
    В связном графе $ G = \langle M, N \rangle $ можно выделить связный подраф $ \overline{G} = \langle M, N' \rangle, \quad N' \sub N $, такой что:
    \begin{itemize}
    	\item $ |N'| = |M| - 1 $ (Вершины из $0 : k$, рёбра из $ 1 : k $)
        \item $ \operatorname{num}(u) = \max\set{\operatorname{num}(\operatorname{beg}(u)), \operatorname{num}(\operatorname{end}(u))} $
    \end{itemize}
\end{theorem}

\begin{proof}

\end{proof}

\begin{implication}
	Если $ |N| < |M| - 1 $, то граф не связный
\end{implication}

\begin{undefthm}{Задача}
    Задан связный граф $ G = \langle M, N \rangle, \quad l(u), \quad u \in N $. Найти остовное дерево $ \overline{G} = \langle M, N' \rangle : \sum_{u \in N'} l(u) \to \min $
\end{undefthm}

\begin{algorithm}[Краскалла]
	\hfill
    \begin{enumerate}
    	\item Сортируем дуги: $ l(u_1) \le l(u_2) \le ... \le l(u_n), \qquad U_0 \ne \O $
        \item
        \begin{verbatim}
        	for i := 1 to n
                Рассматриваем ребро u_i
                Пытаемся доавить в U_i-1:
                - Если не получили цикла, то U_i = U_i-1 + { u_i }
                - Если получили цикл, то U_i = U_i-1
            end for
        \end{verbatim}

    \end{enumerate}

\end{algorithm}
