\chapter{NP задачи}

\section{Экстремальные задачи}

\begin{definition}
	NP-полные задачи оптимизации называются NP-трудными
\end{definition}

\begin{undefthm}{Классификация приближённых алгоритмов}
	\hfill
    \begin{enumerate}
    	\item Жадные алгоритмы
        \item Алгоритмы с гарантированной оценкой точности
        \item Приближённые алгоритмы, которые выдают одно решение
        \item Метаэвристики:
        \begin{enumerate}
            \item Поиск в локальной окрестности:
            \begin{enumerate}
            	\item Имитиация отжига
            \end{enumerate}
            \item Генетические алгоритмы
            \item Муравбиный поиск
            \item Вероятностные алгоритмы
        \end{enumerate}
    \end{enumerate}
\end{undefthm}

\subsection{Жадные алгоритмы}

Жадный алгоритм для задачи коммивояжёра -- идти в ближайший город

\subsection{Гарантированная оценка точности}

Оценка существует:
$$ \exist k : \forall \text{ решения } A \quad \frac{f_A}{f_{\text{опт}}} \le k $$
Оценки не существует:
$$ \forall k \quad \exist \text{ решение } A : \frac{f_A}{f_{\text{опт}}} > k $$

Следующий алгоритм работает только на неориентированном графе:
\begin{algo}[Эйлера]
	\item Строим кратчайшее остовное дерево $ T $
    \item Удваиваем рёбра дерева. Получаем $ 2T $
    \item Строим эйлеров цикл $ C $
    \item Из $ C $ делаем гамильтонов цикл (маршрут коммивояжёра) $ L $:
    \begin{enumerate}
    	\item Вычёркиваем повторы
    \end{enumerate}
\end{algo}

\begin{theorem}
    Для метрической задачи коммивояжёра, для алгоритма Эйлера верно $ \dfrac{f_A}{f_{\text{опт}}} \le 2 $
\end{theorem}

\begin{proof}
    $$ |T| = \sum_{u \in T} l\set{u}, \qquad |C| = 2|T|, \qquad f_A = |L| \le 2|T|, \qquad f_{\text{опт}} \ge |T| $$
\end{proof}

\begin{algo}[Кристофидиса]
	\item Строим кратчайшее остовное дерево $ T $
    \item
    \begin{enumerate}
    	\item В построенном дереве выделяем вершины нечётной степени. Их чётное число
        \item Находим полное паросочетание на этих вершинах с минимальной суммой. Его добавляем к $ T $
    \end{enumerate}

    \item Строим эйлеров цикл $ C $
    \item Из $ C $ делаем гамильтонов цикл (маршрут коммивояжёра) $ L $:
    \begin{enumerate}
    	\item Вычёркиваем повторы
    \end{enumerate}
\end{algo}

\begin{theorem}
    Алгоритм Кристофидиса на метрической задаче имеет гарантированную оценку $ \dfrac32 $
\end{theorem}

\subsection{Метаэвристики}

\subsubsection{Поиск в локальной окрестности (LS -- Local Search)}

$ X $ -- решение, $ U(X) $ -- окрестность

\begin{definition}
	Окрестность определяется следующим образом: \\
    Задаётся набор операций над $ X $. $ U(X) $ -- все решения, которые можно получить из $ X $ этими операциями
\end{definition}

Остановка:
\begin{itemize}
	\item По числу итераций
    \item По времени
    \item Если целевая функция какое-то колиичество итераций (времени) не уменьшается
    \item Отклонение $ f $ от нижней оценки меньше заданного
\end{itemize}

Можно время от времени переходить в плохое решение: \\
Если встретили плохое решение, подбрасываем монетку ($ \approx 90 \% $ и $ 10 \% $). Если выпало $ 10 \% $, переходим в это решение

\subsubsection{Генетические алгоритмы}

$ X $ -- особь, у которой есть генотип и фенотип \\
Фенотип -- $ f(X) $ -- значение целевой функции \\
Генотип -- \textit{что-то}, например, перестановка \\
Заранее определяем $ n $ -- размер популяции

\begin{algorithm}
	\hfill
    \begin{enumerate}
        \item Разделяем популяцию на пары. Получаем $ \faktor{n}2 $ пар
        \item От каждой пары получаем 2 потомка. Операция -- кроссинговер (кроссовер):
        \begin{enumerate}
            \item ``Комбинируем'' перестановки:
            $$
            \begin{matrix}
                a_1 & ... & a_{\faktor{k}2} & | & a_{\faktor{k}2 + 1} & ... & a_k \\
                b_1 & ... & b_{\faktor{k}2} & | & b_{\faktor{k}2 + 1} & ... & a_k
            \end{matrix} \quad \implies \quad
            \begin{matrix}
                a_1 & ... & a_{\faktor{k}2} & | & b_{\faktor{k}2 + 1} & ... & b_k \\
                b_1 & ... & b_{\faktor{k}2} & | & a_{\faktor{k}2 + 1} & ... & a_k
            \end{matrix} $$
        \end{enumerate}
        \item Отбор $ n $ лучших особей
    \end{enumerate}
\end{algorithm}

\begin{undefthm}{Возможные улучшения}
	\hfill
    \begin{itemize}
    	\item Двухточечный кроссовер
        \item При отборе оставлять несколько плохих особей
        \item Более сильная особь оставляет больше потомства:
        \begin{enumerate}
        	\item Каждой особи приписыаем коэффициент качества
            \item Вместо кроссовера $ \faktor{n}2 $ раз запускаем схему Уолкера и решаем, кто будет размножаться
        \end{enumerate}
    \end{itemize}
\end{undefthm}
