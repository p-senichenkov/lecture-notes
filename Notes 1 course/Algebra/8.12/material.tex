\chapter{Полиномы}

\section{Продолжение чего-то}

\begin{implication}
    $c$ -- корень $P(x)$, $P(c) = P'(c) = ... = P^{(k-1)}(c) = 0$, $P^{(k)}\ne 0$. Тогда $k$ -- показатель кратности корня $c$
\end{implication}

\begin{proof}
	Индукция. \\
    База: $k = 1$, следует из теоремы
    $$k = 1 \quad P(c) = 0, ~ P'(c) \ne 0 \implies c \text{ -- простой корень} $$
    Переход: $k - 1 \to k, ~ k \ge 2$ \\
    Положим $P_1(x) = P'(x)$, $c$ -- корень $P_1(x)$. Докажем, что показатель кратности $c$ для $P_1$ на 1 меньше, чем для $P$. Пусть $P(x) = (x - c)^m Q(x)$, $Q(x) \ndivby x - c$
    \begin{multline*}
        P_1(x) = \bigg((x-c)^mQ(x)\bigg)' = \bigg((x-c)^m\bigg)' \cdot Q(x) + (x-c)^m \cdot Q'(x) = \\ = \underbrace{m(x - c)^{m-1}\underbrace{Q(x)}_{\ndivby x-c}}_{\divby (x-c)^{m-1}, ~ \ndivby (x-c)^m} + (x - c)^m Q'(x) \implies P_1(x) \divby (x-c)^{m-1}, ~ \ndivby (x-c)^m
    \end{multline*}
\end{proof}

\begin{theorem}[Формула Тейлора]
    Пусть $P \in \R[x]$, $\deg P = n$. Тогда
    $$ P(x) = P(c) + \frac{P'(c)}{1!}(x-c) + ... + \frac{P^{(k)}(c)}{k!}(x-c)^k + ... + \frac{P^{(n)}(c)}{n!}(x-c)^n $$
\end{theorem}

\begin{proof}
    \hfill
    \begin{enumerate}
    	\item Докажем, что $\exist d_0, d_1, ..., d_n : P(x) = d_0 + d_1 (x-c) + ... + d_n (x-c)^n $: \\
        Индукция:
        \begin{itemize}
        	\item База: $\deg P = 0 $ или $\deg P = - \infty $, $P(x)$ -- костанта $ \implies P(x) = d_0 $
            \item Переход: $ n - 1 \to n $ \\
            Разделим $P(x)$ на $x-c$ с остатком:
            $$ P(x) = Q(x) \cdot (x - c) + r, \quad \deg Q = n - 1 $$
            По индукционному предположению $ \exist c_i : Q(x) = c_0 + c_1 (x - c) + ... + c_{n-1}(x - c)^{n-1} $
            $$ P(x) = c_0 (x - c) + c_1 (x - c)^2 + ... + c_{n-1} (x - c)^n + r $$
            Подойдёт $d_0 = r$, $d_i = c_{i-1} $, при $ i \ge 1 $
        \end{itemize}
        \item Докажем, что $d_i = \frac{P^{(i)}(c)}{i!} $, $i \ge 1$, $ d_0 = P(c) $
        $$ \bigg( (x-c)^i \bigg)^{(k)} = \left\{ \begin{tabular}{l c}
                                                 	$0$ & $i < k$ \\
                                                    $k(k-1)...1$ & $i = k$ \\
                                                    $i(i-1)...(i-k+1)(x-c)^{i-k}$ & $i > k$
                                                 \end{tabular} \right. $$
        $$ \bigg( (x-c)^i \bigg)^{{k}} \clamp{x=c} = 0 $$
        $$ P^{(k)}(c) = d_0 0 + ... + d_{k-1} 0 + d_k \cdot k! + d_{k+1} 0 + ... $$
        $$ d_k = \frac{P^{(k)}(c)}{k!} $$
    \end{enumerate}
\end{proof}

\section{\S8. Поле частных области целостности}

\begin{egs}
    \hfill
    \begin{enumerate}
    	\item $A = \Z$, $K = \Q$
        \item $A$ -- кольцо многочленов над полем. $K$ -- поле рациональных функций $\frac{P(x)}{Q(x)}$
    \end{enumerate}
\end{egs}

\begin{notation}
	$A$ -- область целостности. $M$ -- множество пар $(a,b)$, где $b \ne 0$ \\
    $\rho$ -- отношение эквивалентности на $M$, заданное правилом $(a,b) \mathrel{\rho} (c,d)$, если $ ad = bc $
\end{notation}

\begin{lemma}
	$ \rho $ -- отношение эквивалентности
\end{lemma}

\begin{proof}
    \hfill
    \begin{itemize}
        \item Рефлексивность: $ ab = ab \implies (a,b) \mathrel\rho (a, b) $
        \item Симметричность: $ (a,b) \mathrel\rho (c,d) \implies ad = bc \implies cb = ad \implies (c,d) \mathrel\rho (a,b) $
        \item Транзитивность:
        $$ \begin{rcases} (a,b) \mathrel\rho (c,d) \\ (c,d) \mathrel\rho (e,f) \end{rcases} \stackrel?= (a,b) \mathrel (e,f) $$
        $$ 0 = (ad - bc)f + (cf -de)b = adf - bde = d(af - be) \underset{\text{обл. цел.}}{\implies} af = be $$
    \end{itemize}
\end{proof}

\begin{definition}
	Пусть $(a,b)$, $(c,d) \in M $. Их суммой и произведением называются пары $(ad + bc, bd)$, $(ac, bd) \in M$
\end{definition}

\begin{lemma}
	Пусть $u, v, u', v' \in M $, $u \mathrel\rho u'$, $v \mathrel\rho v'$. Тогда $(u + v) \mathrel\rho (u' + v')$, $ (uv) \mathrel\rho (u'v')$
\end{lemma}

\begin{proof}
	Отношение $\rho$ транзитивно, значит достаточно проверить, что $v \rho v' \implies (u + v) \mathrel\rho (u + v'), ~ (uv) \mathrel\rho (uv') $ и $u \mathrel u' \implies (u + v) \mathrel\rho (u' + v), ~ (uv) \mathrel\rho (u'v) $ \\
    Пусть $u = (a,b), ~ v = (c,d), ~ v' = (c',d') $, $cd' = c'd$
    $$ (ad + bc)bd' - bd(ad' + bc') = b^2(cd' - dc') = 0 $$
    $$ ac \cdot bd' - bd \cdot ac' = ab(cd' - dc') = 0 $$
\end{proof}

\begin{implication}
    Операции сложения и умножения можно определить на классах эквивалентности
\end{implication}

\begin{theorem}
	Пусть $A$ -- область целостности с единицей. $K$ -- множество классов эквивалентности множества $M$ по отношению $\rho$ с определёнными выше операциями сложения и умножения. Тогда $K$ -- поле
\end{theorem}

\begin{definition}
    $K$ называется полем частных $A$
\end{definition}

\begin{proof}
    $\overline{x}$ -- класс $x$
    \begin{enumerate}
    	\item Ассоциативность сложения:
        $$ x = (a,b), ~ y = (c,d), ~ z = (e,f) $$
        $$ x + y = (ad + bc, bd) $$
        $$ (x + y) + z = (ad + bd, bd) + (e,f) = ((ad + bc) \cdot f + bd \cdot e, bd \cdot f) = (adf + bcf + bde, bdf) $$
        $$ (y + z) = (cf + de, df) $$
        $$ x + (y + z) = (a,b) + (cf + de, df) = (adf + b(cf + de), bdf) = (adf + bcf + bde, bdf) $$
        \item Нейтральный по сложению: $0 = \overline{(0, 1)} $ \\
        Пусть $x = (a,b) $
        $$ x + (0,1) = (a,b) + (0,1) = (a \cdot 1 + b \cdot 0, b \cdot 1) = (a,b) = x $$
        $$ (0,1) + x = (0,1) + (a,b) = (0 \cdot b + 1 \cdot a, 1 \cdot b) = (a,b) = x $$
        Докажем, что $b \ne 0 \implies \overline{(0,b)} = 0 $. То есть, докажем, что $(0,b) \mathrel\rho (0,1) $:
        $$ 0 \cdot 1 = b \cdot 0 \implies (0,b) \mathrel\rho (0,1) \implies \overline{(o,b)} = 0 $$
        Докажем, что $\overline{(a,b)} = 0 \implies a = 0$:
        $$ (a,b) \mathrel\rho (0,1) \implies a \cdot 1 = b \cdot 0 = 0 \implies a = 0$$
        \item Оратный по сложению. Проверим, что $\overline{(-a,b)} = - \overline{(a,b)}$:
        $$ (-a,b) + (a,b) = (-ab + ba, bb) = (0, b^2) = 0 $$
        $$ (a,b) + (-a,b) = (ab + b(-a), bb) = (0, b^2) = 0 $$
        \item Коммутативность сложения, дистрибутивность, ассоциативность умножения, коммутативность умножения -- упражненеия
        \item Нейтральный по умножению: $ 1 = \overline{(1,1)} $ \\
        Пусть $x = (a,b) $
        $$ x \cdot (1,1) = (a,b) \cdot (1,1) = (a \cdot 1 + b \cdot 1) = (a,b) = x \implies \overline{x \cdot (1,1)} = \overline{x} $$
        Докажем, что $ \forall b \ne 0 $ выполнено $ \overline{(b,b)} = 1 $: (не докажем)
        \item Обратный по умножению: \\
        Пусть $ \overline{(a,b)} \ne 0 $. Докажем, что $ \overline{(b,a)} = \overline{(a,b)^{-1}} $:
        $$ \overline{(a,b)} \cdot \overline{(b,a)} = \overline{(ab,ab)} = 1 $$
    \end{enumerate}
\end{proof}

\underline{Переход к стандартным обозначениям}:
Вложим $A$ в $K$ по правилу $a \mapsto \overline{(a,1)}$ \\
Сложение, умножение согласованы:
$$ (a,1) + (b,1) = (a \cdot 1 + 1 \cdot b, 1 \cdot 1) = (a + b, 1) $$
$$ (a,1) \cdot (b,1) = (ab, 1 \cdot 1) = (ab,1) $$
Проверим, что $(a,b) = a : b $, то есть
$$ \overline{(a,b)} \cdot b = a $$
$$ (a,b) \cdot (b, 1) = (ab, b) $$
$$ ab \cdot 1 = ba \implies (ab, b) \mathrel\rho (a,1) \implies (\overline{ab}, b) \mathrel\rho a $$
Класс $\overline{(a,b)} $ записывают как $\frac{a}b $

\section{\S9. Поле рациональных функций}

\begin{definition}
    Пусть $K$ -- поле. Полем рациональных функций над $K$ называется поле частных кольца $K[x]$
\end{definition}

\begin{notation}
	$K(x)$
\end{notation}

\begin{eg}
    $$ x^2 + 1 \in \R[x], ~ \in \R(x) $$
    $$ \frac{x^2 + 1}{x+2} \in \R(x) $$
\end{eg}

\begin{definition}
	Элементы $K(x)$ называются рациональными функциями (над $K$) или рациональными дробями (над $K$)
\end{definition}

