\chapter{Теория групп}

\begin{undefthm}{Напоминание}
	Мы доказываем, что $ |G| = |\Orb(m)| \cdot |\St(m)| $ \\
    $ k \define |\Orb(m)| $
    $$ \Orb(m) = \set{m_1, ..., m_k} $$
\end{undefthm}

\begin{proof}
    Доказываем, что $ g_i $ -- представители всех смежных классов по $ \St(m) $
    \begin{itemize}
    	\item Докажем, что $ g_i, g_j $ принадлежат разным смежным классам: \\
        Пусть они принадлежат одному смежному классу
        $$ g_i^{-1}g_j \in \St(m) $$
        $$ g_i^{-1}g_jm = m $$
        $$ g_jm = g_im $$
        \item Докажем, что $ \forall g \in G \quad \exist g_i : \overline{g} = \overline{g_i} $:
        $$ gm \in \Orb(m) = \set{m_1, ..., m_k} $$
        $ gm = m_i $ для некоторого $ i $
        $$ gm = g_im = m_i $$
        $$ (g_i^{-1}g)m = m \implies g_i^{-1}g \in \St(m) \implies \overline{g_i} = \overline{g} $$
    \end{itemize}
\end{proof}

\begin{lemma}[Бернсамда]
	$ G $ -- конечная группа, $ \quad M $ -- конечное множество, $ \quad G $ действует на $ M $ \\
    Тогда количество орбит равно
    $$ \frac1{|G|} \sum_{g \in G} |\Fix(g)| $$
\end{lemma}

\begin{remark}
	Это средняя мощность фиксатора
\end{remark}

\begin{proof}
	Заметим, что
    $$ \sum_{g \in G} |\Fix(g)| = \sum_{m \in M} |\St(m)| $$
    так как и то, и другое -- это количество пар $ (g, m) $, таких, что $ gm = m $ \\
    Получили, что
    $$ \frac1{|G|} \sum_{g \in G} |\Fix(g)| = \frac1{|G|} \sum_{m \in M} |\St(m)| = \sum_{m \in M} \frac{|\St(m)|}{|G|} \underset{\text{(предыдущая теорема)}}= \underbrace{\sum_{m \in M} \frac1{|\Orb(m)}}_{\define S} \stackrel?= \text{ кол-во орбит} $$
    Пусть есть $ k $ орбит, в них $ a_1, ..., a_k $ элементов
    \begin{multline*}
        S = \underbrace{\overbrace{\bigg( \frac1{a_1} + \frac1{a_1} + ... + \frac1{a_1} \bigg) }^{a_1 \text{ слагаемых}}}_{m \text{ из 1-й орбиты}} + \underbrace{\overbrace{\bigg( \frac1{a_2} + \frac1{a_2} + ... + \frac1{a_2} \bigg) }^{a_2 \text{ слагаемых}}}_{m \text{ из 2-й орбиты}} + \widedots[4em] + \underbrace{\overbrace{\bigg( \frac1{a_k} + \frac1{a_k} + ... + \frac1{a_k} \bigg) }^{a_k \text{ слагаемых}}}_{m \text{ из k-й орбиты}} = \\ = \underbrace{1 + 1 + ... + 1}_k = k
    \end{multline*}
\end{proof}

\section{Прямое произведение групп}

\begin{definition}
	$ (G, *), (H, \cdot) $ -- группы. Прямое произведение (внешнее) -- это множество $ G \times H $ с операцией $ \circ $, определённой как $ (g_1, h_1) \circ (g_2, h_2) = (g_1 * g_2, h_1 \cdot h_2) $
\end{definition}

\begin{remark}
	Аналогично для нескольких групп
\end{remark}

\begin{notation}
	$ G \times H, \qquad G_1 \times ... \times G_k $
\end{notation}

\begin{theorem}
	Прямое произведение групп -- группа
\end{theorem}

\begin{proof}
    Рассмотрим $ G \times H \times ... $
    \begin{itemize}
    	\item Замкнутость относительно операции очевидна
        \item Ассоциативность:
        \begin{multline*}
            \bigg( (g_1, h_1, ...)(g_2, h_2, ...) \bigg) (g_3, h_3, ...) = (g_1g_2, h_1h_2, ...)(g_3, h_3, ...) = (g_1g_2g_3, h_1h_2h_3, ...) = \\ = (g_1, h_1, ...)(g_2g_3, h_2h_3, ...) = (g_1, h_1, ...) \bigg( (g_2, h_2, ...), (g_3, h_3, ...) \bigg)
        \end{multline*}
        \item Нейтральный:
        $$ e_{G \times H \times ...} = (e_G, e_H, ...) $$
        \item Обратный:
        $$ (g, h, ...)^{-1} = (g^{-1}, h^{-1}, ...) $$
    \end{itemize}
\end{proof}

\begin{property}[подгруппа прямого произведения]
    $ G = G_1 \times G_2 \times G_k, \qquad e_i $ -- нейтральный элемент $ G_i $
    $$ H_i \define \set{e_i, ..., e_{i - 1}, \bm{g_i}, e_{i + 1}, ..., e_k} $$
    Тогда
    \begin{enumerate}
        \item[0.] $ H_i \cong G_i $
    	\item $ H_i \vartriangleleft G_i $
        \item $
        \begin{rcases}
        	i \ne j \\
            h_i \in H_i \\
            h_j \in H_j
        \end{rcases} \implies h_ih_j = h_jh_i $
        \item $ H_i \cap H_1 ... H_{i - 1}H_{i + 1} ... H_k = \set{e} $
        \item $ H_1 ... H_k = G $
        \item $ G|H_i \cong G_1 \times ... \times G_{i - 1} \times G_{i + 1} \times G_k $
    \end{enumerate}
\end{property}

\begin{lemma}[порядки элементов в прямом произведении]
	Пусть $ G = H_1 \times H_2 \times ... \times H_k, \qquad h_i \in H_i $ \\
    $ g = (h_1, ..., h_k) $ \\
    Тогда
    \begin{enumerate}
        \item $ \ord_{H_i}(h_i) = \infty $ для некоторого $ i \implies \ord_G(g) = \infty $
    \end{enumerate}

\end{lemma}
