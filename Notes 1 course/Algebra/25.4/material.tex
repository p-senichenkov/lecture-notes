\chapter{Теория групп}

\section{Прямое произведение подгрупп}

\begin{definition}
	$ G $ -- группа, $ H_1, ..., H_k $ -- нормальные подгруппы $ G $ \\
    Говорят, что $ G $ является (внутренним) прямым произведением $ H_1, ..., H_k $, если
    \begin{enumerate}
    	\item Любой элемент $ G $ можно единственным образом представить в виде $ g = h_1h_2...h_k $, где $ h_i \in H_i $
        \item $ \forall \underset{h_i \in H_i, ~ h_j \in H_j}{i \ne j} \quad h_ih_j = h_jh_i $
    \end{enumerate}
\end{definition}

\begin{notation}
	$ G = H_1 \times ... \times H_k $
\end{notation}

\begin{undefthm}{Другое название}
	$ G $ разложена в произведение подгрупп
\end{undefthm}

\begin{lemma}[нормальные подгруппы с единичным пересечением]
    $ G $ -- группа, $ H \vartriangleleft G $, $ K \vartriangleleft G $, $ H \cap K = \set{e} $ \\
    Тогда элементы \widedots
\end{lemma}

\begin{theorem}[прямое произведение нескольких подгрупп]
    $ G $ -- группа, $ H_1 \vartriangleleft G, \widedots[2em], H_k \vartriangleleft G $
    $$ \forall i \quad H_i \cdot ... \cdot H_{i - 1} \cap H_i = \set{e} $$
    $$ H_1H_2...H_k = G $$
    Тогда $ G = H_1 \times ... \times H_k $
\end{theorem}

\begin{proof}
    Докажем, что $ H_i \cap H_j = \set{e} $: \\
    Пусть $ j < i $ \\
    $ H_j \sub H_1...H_j...H_{i - 1} $, т. к. $ h_j = e...eh_je...e $ \\
    $ \implies H_j \cap H_i = \set{e} \implies $ элементы $ H_i $ и $ H_j $ коммутируют \\
    \widedots
\end{proof}

\begin{theorem}[разложение циклической группы в прямое произведение двух подгрупп]
    Пусть $ G $ -- циклическая, $ |G| = m^n $, $ \GCD{m, n} = 1 $ \\
    Тогда $ G $ можно разложить в прямое произведение подгрупп, изоморфных $ \Z_m $ и $ \Z_n $
\end{theorem}

\begin{definition}
    Группа называется примарной, если она изоморфна $ \Z $ или $ \Z_{p^n} $, где $ p \in \Prime $
\end{definition}

\begin{remark}
	Теорема о линейном предстаавлении НОД верна для нескольких чисел:
    $$ \forall a_1, ..., a_k \quad \exist t_1, ..., t_k : t_1a_1 + ... + t_ka_k = \GCD(a_1, ..., a_k) $$
\end{remark}
