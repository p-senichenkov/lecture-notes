\chapter{Полиномы}

\section{Интерполяция}

\begin{tabular}{c || c | c | c | c}
	$x$ & $x_1$ & $x_2$ & ... & $x_n$ \\
    \hline
    $F(x)$ & $y_1$ & $y_2$ & ... & $y_n$
\end{tabular}

\begin{theorem}[Интерполяционная формула Лагранжа]
	Пусть $K$ -- поле. Для любых различных $x_1, ..., x_n \in K$ и любых $y_1, ..., y_n \in K$ существует единственный многочлен $F \in K(x)$, такой что $F(x_i) = y_i ~ \forall i, \quad \deg F \le n - 1 $. \\
    Многочлен $F$ можно найти по формуле
    $$ F(x) = L_1(x)y_1 + L_2(x)y_2 + ... + L_n(x)y_n, $$
    где
    $$ L_i(x) = \frac{(x- x_1)...(x - x_{i-1})(x-x_{i+1})...(x-x_n)}{(x_i-x_1)...(x_i-x_{i-1})(x_i-x_{i+1})...(x_i)-x_n} $$
\end{theorem}

\begin{proof}
	\ \\ 1) Существование и формула \\
    Проверим, что многочлен, заданный формулой, подходит
    $$ \deg L_i = n - 1 \implies \begin{cases} \deg (L_i(x) \cdot y_i = n -1 \\ \text{или} \\ L_i(x) \cdot y_i = 0 \end{cases} $$
    $$ \implies \deg F \le \max \{\deg L_i \} = n - 1 $$
    $$ L_i(x_i) = 1 ~ \forall i, \quad L_i(x_j) = 0 ~ \forall i \ne j$$
    $$ F(x_1) = L_1(x_i)y_1 + ... + L_i(x_i)y_i + ... + L_n(x_i)y_n = 0 \cdot y_1 + ... + 1 \cdot y+i + ... + 0 \cdot y_n = y_i $$
    \ \\
    2) Единственность \ \\
    Пусть $F(x), G(x) : \forall i \begin{cases} F(x_i) = y_i \\ G(x_i) = y_i \end{cases}$
    $$ \deg F \le n - 1, ~ \deg G \le n - 1 $$
    Пусть $ H(x) = F(x) - G(x) \implies \deg H \le n - 1 $
    $$ H(x_i) = y_i - y_i = 0 \implies \text{у } H \text{ есть } n \text{ корней} \implies H = 0 \implies F = G $$
\end{proof}

\begin{eg}
    \begin{tabular}{c || c | c | c}
    	$x$ & 1 & 2 & 3 \\
        \hline
        $F(x)$ & 1 & 4 & 9
    \end{tabular}
    $$ L_1(x) = \frac{(x-2)(x-3)}{(1-2)(1-3)} = \frac12 x^2 - \frac52 x + 3 $$
    $$ L_2(x) = \frac{(x-1)(x-3)}{(2-1)(2-3)} = - x^2 + 4x - 3 $$
    $$ L_3(x) = \frac{(x-1)(x-2)}{(3-1)(3-2)} = \frac12 x^2 - \frac32 x - 1 $$
    Подставляем
    $$ F(x) = (\frac12 x^2 - \frac52x + 3) \cdot 1 + (-x^2 + 4x - 3) \cdot 4 + (\frac12 x^2 - \frac32x + 1) \cdot 9 = ... = x^2 $$
\end{eg}

\begin{algorithm}[Метод интерполяции Ньютона]
	Для данных различных $x_1,...,x_n$ и данных $y_1,...,y_n$ требуется построить многочлен $F(x)$, такой что $F(x_i) = y_i$, и $ \deg F \le n - 1$
000 0   \ \\
    Построим последовательно многочлены $f_1, f_2, ..., f_n$, так что $ \deg f_k(x) \le k - 1$, $f_k(x_1) = y_1, ..., f_k(x_k) = y_k $. То есть, будем строить многочлены, которые решают задачу для первых $k$ корней. На $n$-ом шаге получим $F(x) = f_n(x) $
    \ \\
    В качестве $f_1(x)$ возьмём $ f_1(x) = y_1$. Многочлен $f_k(x)$ определим по формуле
    $$ f_K(x) = f_{k-1}(x) + A_{k-1} \cdot g_{k-1}(x), $$
    где $g_{k-1}(x) = (x-x_1)...(x-x_{k-1}) $, $ A_{k-1} = \frac{y_{k-1}-f_{k-1}(x_k)}{g_{k-1}(x_k)} $
\end{algorithm}

\begin{eg}
	\begin{tabular}{c || c | c | c}
    	$x$ & 1 & 2 & 3 \\
        \hline
        $F(x)$ & 1 & 4 & 9
    \end{tabular}
    $$ f_1(x) = 1 $$
    $$ f_2(x) = 1 + A(x-1) $$
    $$ x = 2, \quad 4 = 1 + A(2-1) \implies A = 3 $$
    $$ f_2(x) = 1 + 3(x-1) = 3x - 2$$
    $$ f_3(x) = (3x - 2) + A(x-1)(x-2) $$
    $$ x = 3, \quad 9 = (3\cdot3 - 2) + A(3 - 1)(3 - 2) \implies A = 1 $$
    $$ f_3(x) = (3x - 2) + 1 (x-1)(x-2) = x^2 $$
    $$ F(x) = x^2 $$
\end{eg}

\begin{theorem}
	Метод интерполяции Ньютона корректно определён и результат его применения -- нужный многочлен
\end{theorem}

\begin{proof}
    $$ g_{k-1} (x_k) = (x_k - x_1)...(x_k-x_{k-1}) \ne 0 \implies A_k \text{ определено} $$
    Докажем по индукции, что $\deg f_k \le k - 1$
    $$ \deg f_1 = \begin{cases} 0 \\ - 1 \end{cases} \le 1 - 1 $$
    $$ \deg f_k \le \max \{ \deg f_{k-1}, \deg (A_k g_{k-1}) \} \le k - 1 $$
    Докажем по индукции, что $f_{k-1} = y_1,...,f_k(x_k) = y_k$
    $$ f_1(x_1) = y_1 $$
    $$ f_k(x_k) = f_{k-1}(x_k) + \frac{g_k - f_{k-1}(x_k)}{g_{k-1}(x_k)}g_{k-1}(x_k) = g_k $$
\end{proof}

\section{Делимость в области целостности}
По учебнику Кострикина

\begin{definition}
	Пусть $A$ -- область целостности, $a,b \in A$. $a$ делится на $b$, если $\exist c \in A: a = bc$
\end{definition}

\begin{notation}
	$$ a \vdots b $$
\end{notation}

\begin{properties}
    ...

\end{properties}
...

$k = m$ и можно так перенумеровать элементы, что $q_i$ и $p_i$ ассоциированы $\forall i$

\begin{eg}
    \begin{enumerate}
    	\item $\Z$ -- факториально \\
        $$6 = 1 \cdot 2 \cdot 3 = 1 \cdot (-2) \cdot (-3) = -1 \cdot (-2) \cdot 3$$
    \end{enumerate}

\end{eg}
