\chapter{Векторные пространства}

\section{Продолжение чего-то}

\begin{proof}
    $ \rk A = \dim $ пространства строк $ \define U $ \\
    Элементарные преобразования строк и перестановка столбцов: $ A \to A' $ \\
    Докажем, что $ \dim U = \dim U' $ \\
    Если строки ЛНЗ, то при элементарных преобразованиях получаются ЛНЗ \\
    Если строки ЛЗ, то получаются ЛЗ \\
    Значит, при элементарных преобразованиях строк, $ \dim $ не меняется
    $$ u_1, ..., u_m, \qquad u_1', ..., u_m' $$
    $$ u_i = (x_1^{(i)}, ..., x_n{(i)}), \qquad u_i' = (x_{\sigma(1)}^{(i)}, ..., x_{\sigma(m)}^{(i)}) $$
    где $ \sigma $ -- перстановка \\
    Рассмотрим ЛК $ \sum c_iu_i, \quad \sum c_iu_i' $:
    $$ c_1u_1 + c_2u_2 + ... = (..., c_1x_k^{(1)} + c_2x_k^{(i)}, ...), \qquad c_1u_1' + c_2u_2' + ... = (..., c_1x_{\sigma(k)}^{(1)} + c_2x_{\sigma(k)}^{(i)}, ...) $$
    $ \sum c_iu_i $ и $ \sum c_iu_i' $ отличаются перестановкой координат \\
    ЛНЗ/ЛЗ наборы соответствуют друг другу \\
    Достаточно доказать утверждение для трапецевидной матрицы
    $$  A' =
    \begin{pmatrix}
        a_{11} & a_{12} & ... & a_{1r} & | & a_{1, r + 1} & ... \\
        0 & a_{22} &  & a_{2r} & | & a_{2, r + 1} & ... \\
         & . &  &  & | & a_{3, r + 1} & ... \\
         &  & . &  & | & ... \\
         &  &  & a_{rr} & | & a_{r, r + 1} & ... \\
        - & - & - & - & - & ... \\
        0 & 0 & ...
    \end{pmatrix} $$
    $$ \rk A' \define r $$
    Рассмотрим минор порядка $ > r $ \\
    Есть нулевая строка, определитель равен 0 \\
    Пусть $ u_i $ -- $i$-я строка матрицы $ A' $
    Докажем, что $ \dim u' = r $: \\
    Достаточно доказать, что $ u_1, u_2, ..., u_r $ ЛНЗ
    $$ (0, 0, ..., 0) = c_1u_1 + c_2u_2 + ... + c_ru_r = (c_1a_{11}, c_1a_{12} + c_2a_{22}, ..., c_1a_{11} + c_2a_{21} + ... + c_1a_{rr}, ...) $$
    $$
    \begin{rcases}
        c_1a_{11} = 0 \\
        a_{11} \ne 0
    \end{rcases} \implies c_1 = 0 $$
    $$ c_1a_{12} + c_2a_{22} = 0 \implies c_2a_{22} = 0 \implies c_2 = 0 $$
\end{proof}

\begin{theorem}[Кронекера-Капелли]
	Система линейных уравнений совместна тогда и только тогда, когда ранг матрицы системы равен рангу расширенной матрицы
\end{theorem}

\begin{proof}
	Приведём матрицу системы к трапецевидной элементарными преобразованиями \\
    строк и перестановкой столбцов
    $$
    \begin{pmatrix}
        a_{11} & a_{12} & ... & ... & a_{1n} & | & b_1 \\
        0 & . & . & . & . & | & . \\
        . & . & . & . & . & | & . \\
        . & . & . & . & . & | & . \\
        0 & . & a_{rr} & . & a_{rn} & | & b_r \\
        0 & . & . & . & 0 & | & b_{r + 1} \\
        . & . & . & . & . & | & .
    \end{pmatrix} $$
    Система совметсна $ \iff b_i = 0 $ при $ i > r $
    $$ \widedots $$
\end{proof}

\chapter{Линейные отображения}

\section{Матрица линейного отображения}

\begin{definition}
	$ U, V $ -- векторные пространства над $ K $ \\
    Отображение $ f : U \to V $ называется линейным, если
    \begin{enumerate}
    	\item $ \forall u_1, u_2 \in U \quad f(u_1 + u_2) = f(u_1) + f(u_2) $
        \item $ \forall u \in U, k \in K \quad f(ku) = kf(u) $
    \end{enumerate}
\end{definition}

\begin{remark}
	Линейное отображение из $U$ в $U$ иногда называют линейным преобразованием
\end{remark}

\begin{props}
	\item $ f : U \to V, \quad g : V \to W $ -- линейны $ \implies g \circ f : U \to W $ -- линейно
    \item $ f : U \to V $ -- линейно, $ U_1 $ -- подпространство $ U \implies f \clamp{U_1} : U_1 \to V $ -- линейно
\end{props}

\begin{definition}
	Пусть $ U, V $ -- конечномерные, $ e_1, ..., e_n $ -- базис $ U $, $ g_1, ..., g_m $ -- базис $ V $, $ f $ -- линейное отображение $ U \to V $ \\
    Матрицей $ f $ в данных базисах называется матрица, в $ i $-м столбце которой записаны координаты $ f(e_i) $ в базисе $ g_1, ..., g_m $, то есть
    $$
    \begin{pmatrix}
        a_11 & ... & a_{1m} \\
        . & . & . \\
        a_{m1} & ... & a_{mn}
    \end{pmatrix} $$
    $$ f(e_i) = \sum_{k = 1}^{m} a_{ki}g_k $$
\end{definition}

\begin{lemma}[матричная запись линейного оторажения]
	$ U, V $ -- конечномерные, $ e_1, ..., e_n $ -- базис $ U $, $ g_1, ..., g_m $ -- базис $ V $, $ f : U \to V $ -- линейное
    \begin{enumerate}
    	\item Пусть $ A $ -- матрица $ f $ в данных базисах, $ u \in U, \quad v \in V $, такие, что $ f(u) = v $, \\
        $ X $ -- столбец координат $ u $ в базисе $ e_1, ..., e_n $ \\
        $ Y $ -- столбец координат $ v $ в базисе $ g_1, ..., g_m $ \\
        Тогда $ Y = AX $
        \begin{proof}
        	Пусть $ X =
            \begin{pmatrix}
            	x_1 \\
                . \\
                . \\
                . \\
                x_n
            \end{pmatrix}, \qquad Y =
            \begin{pmatrix}
            	y_1 \\
                . \\
                . \\
                . \\
                y_m
            \end{pmatrix} $
            \begin{multline*}
                v = f(u) = f(x_1e_1 + ... + x_ne_n) = x_1f(e_1) + ... + x_nf(e_n) = \\ = x_1(a_{11}g_1 + ... + a_{m1}g_m) + ... + x_i(a_{1i}g_1 + ... + a_{mi}g_m) + ... + x_n(a_{1n}g_1 + ... + a_{mn}g_m) = \\ = (a_{11}x_1 + ... + a_{i1}x_i + ... + a_{1m}x_n)g_1 + ... + (a_{mn}x_1 + ... + a_{mi}x_i + ... + a_{mn}x_n)g_m \implies \\ \implies g_1 = y_{11}x_1 + ... + a_{1n}x_n + ... + a_{1n}x_m, \qquad y_m = a_{m1}x_1 + ... + a_{mn}x_n
            \end{multline*}
            $$ AX =
            \begin{pmatrix}
                a_{11} & ... & a_{1n} \\
                . & . & . \\
                a_{m1} & ... & a_{mn}
            \end{pmatrix} \cdot
            \begin{pmatrix}
            	x_1 \\
                . \\
                . \\
                . \\
                x_n
            \end{pmatrix} =
            \begin{pmatrix}
                a_{11} + ... + a_{1n}x_n \\
                \widedots[7em] \\
                a_{mi}x_1 + ... + a_{mn}x_n
            \end{pmatrix} $$
        \end{proof}
        \item Пусть $ A $ -- такая матрица, что $ \forall u, v : f(u) = v $, и их столбцов координат $ X, Y $ выполнено $ Y = AX $, то $ A $ -- матрица $ f $ в этих базисах
        \begin{proof}
        	Аналогично
        \end{proof}
    \end{enumerate}
\end{lemma}

\begin{theorem}[Изменение матрицы при замене базисов]
	$ U, V $ -- конечномерные, $ f : U \to V $ -- линейное, $ e_i, e_i' $ -- базисы $ U $, $ g_i, g_i' $ -- базисы $ V $, $A$ -- матрица $f$ в базисах $ e_i, g_i $, $ A' $ -- матрица $ f $ в базисах $ e_i', g_i' $ \\
    Тогда $ A' = C_{g_i \to g_i'}^{-1} \cdot A \cdot C_{e_i \to e_i'} $
\end{theorem}

\begin{proof}
	Пусть $ u \in U, \quad v \in V, \quad f(u) = v $ \\
    $ X, X' $ -- столбцы коордиинат $ u $ в $ e_i, e_i' $ \\
    $ Y, Y' $ -- столбцы координат $ v $ в $ g_i, g_i' $
\end{proof}

\section{Ядро и образ}

\begin{definition}
	Пусть $ f : U \to V $ -- линейное отображение \\
    Ядром $ f $ называется множество $ \set{u | f(u) = 0 } $
    \begin{notation}
    	$ \ker f $
    \end{notation}
    Образом $ f $ называется множество $ \set{f(u) | u \in U} $
    \begin{notation}
    	$ \Img f $
    \end{notation}
\end{definition}

\begin{props}
	\item $ \ker f $ -- подпространство $ U $
    \item $ \Img f $ -- подпространство $ V $
\end{props}

\begin{definition}
	$ f : U \to V $ называется изоморфизмом, если
    \begin{enumerate}
    	\item $f$ линейно
        \item $f$ -- биекция
    \end{enumerate}
    Если существует изоморфизм $ f : U \to V $, то пространства называются изоморфными
    \begin{notation}
    	$ U \cong V $ ($U \simeq V, U \sim V $)
    \end{notation}
\end{definition}

\begin{props}
	\item
    \begin{itemize}
        \item Если $f$ -- изоморфизм, то $ \exist f^{-1} $ и $ f^{-1} $ -- изоморфизм
        \item Если $ f : U \to V, g : V \to W $ -- изоморфизмы, то $ g \circ f : U \to W $ -- изоморфизм
    \end{itemize}

\end{props}
