\chapter{Векторные пространства}

\section{Продолжение изоморфизма}

\begin{implication}
	Отношение изоморфизма симметрично и транзитивно
\end{implication}

\begin{property}
	$ f : U \to V $ -- линейное отображение. Тогда
    \begin{enumerate}
        \item $ f $ -- инъекция $ \iff \ker f = \set{0} $
        \begin{proof}
        	\hfill
            \begin{itemize}
            	\item $\implies$
                $$
                \begin{rcases}
                	f(0) = 0 \\
                    f \text{ -- инъекция}
                \end{rcases} \implies \forall u \ne 0 \quad f(u) \ne 0 \implies \forall u \ne 0 \quad u \notin \ker f $$
                \item $ \impliedby $ \\
                Пусть $ f(u_1) = f(u_2) \implies f(u_1 - u_2) = f(u_1) - f(u_2) = 0 \implies u_1 - u_2 \in \ker f = set{0} \implies u_1 - u_2 = 0 \implies u_1 = u_2 $
            \end{itemize}
        \end{proof}
        \item $ \ker f = \set{0} \implies f $ -- изоморфизм из $ U $ в $ \Img f $
        \begin{proof}
        	$ f $ -- инъекция (по предыдущему пункту) \\
            $ f $ -- сюръекция (по определению $\Img f $) \\
            Значит, $ f $ -- биекция \\
            $ f $ линейно \\
            Значит, $ f $ -- изоморфизм
        \end{proof}
    \end{enumerate}
\end{property}

\begin{property}
	Пусть $ f : U \to V $ -- изоморфизм. Тогда
    \begin{enumerate}
    	\item $ e_1, ..., e_k $ ЛНЗ $ \iff f(e_1), ..., f(e_k) $ ЛНЗ
        \begin{proof}
            $ f^{-1} $ -- изоморфизм $ \implies $ достаточно доказать $ \impliedby $, то есть, если $ e_1, ..., e_k $ ЛЗ, то $ f(e_1), ..., f(e_k) $ ЛЗ \\
            Пусть $ a_1e_1 + ... + a_ke_k = 0 $, не все $ a_i$ равны 0 \\
            Тогда $ a_1f(e_1) + ... + a_kf(e_k) = f(0) = 0 \implies f(e_i) $ ЛЗ
        \end{proof}
        \item $ e_1, ..., e_k $ -- базис $ U \implies f(e_1), ..., f(e_k) $ -- базис $ V $
        \begin{proof}
        	Базис -- максимальный ЛНЗ. Применим предыдущий пункт
        \end{proof}
        \item $ \dim U = \dim V $
        \begin{proof}
        	Следует из предыдущего пункта
        \end{proof}
    \end{enumerate}
\end{property}

\begin{lemma}[выделение ядра прямым сложением]
	Пусть $ U, V $ -- конечномерны, $ f : U \to V $ линейно \\
    Тогда $ \exist W $ -- подпространство $ U $, такое что:
    \begin{enumerate}
        \item $ W \cong \Img f, \qquad f \clamp{W} \to \Img f $ -- изоморфизм
        \item $ \ker f \oplus W = U $
    \end{enumerate}
\end{lemma}

\begin{proof}
	Пусть $ g_1, ..., g_k \in V, \quad g_1, ..., g_k $ -- базис $ \Img f $
    $$ g_i \in \Img f \implies \exist e_i : f(e_i) = g_i, \quad e_i \in U $$
    Положим $ W = \langle e_1, ..., e_k \rangle $ \\
    Докажем, что $ W $ подходит:
    \begin{enumerate}
        \item Пусть $ f_1 : W \to \Img f, \quad f_1 = f \clamp{W} $. Докажем, что $ f_1 $ -- изоморфизм:
        \begin{itemize}
        	\item Проверим сюръективность: \\
            Пусть $ v \in \Img f \implies \exist a_i : v = a_1g_1 + ... + a_kg_k \implies v = a_1f(e_1) + ... + a_kf(e_k) = f_1(a_1e_1 + ... + a_ke_k) $
            \item Проверим инъективность: \\
            Достаточно проверить, что в 0 переходит только 0 \\
            Пусть $ w \in W, \quad f_1(w) = 0 $
            $$ w = a_1e_1 + ... + a_ke_k $$
            $$ f_1(w) = a_1f(e_1) + ... + a_kf(e_k) = a_1g_1 + ... + a_kg_k \underimp{g_i \text{ ЛНЗ}} \forall i \quad a_i = 0 \implies w = 0 \cdot e_1 + ... + 0 \cdot e_k = 0 $$
        \end{itemize}
        \item Проверим, что $ \ker f + W = U $: \\
        Пусть $ u \in U $ \\
        Пусть $ f(u) = v \in \Img f $ \\
        Пусть $ x \in W : f(x) = v $ (такой $x$ существует, так как $ f \clamp{W} $ -- изоморфизм) \\
        Положим $ y = u - x $ \\
        Тогда $ f(y) = f(u) - f(x) = v - v = 0 \implies y \in \ker f $
        $$
        \begin{rcases}
        	u = y + x \\
            y \in \ker f \\
            x \in W
        \end{rcases} \implies u \in \ker f + W $$
        \item Докажем, что $ U = \ker f \oplus W $: \\
        Достаточно доказать, что $ \underset{
            \begin{subarray}{c}
            	x \in \ker f \\
                y \in W
            \end{subarray}}{x + y = 0} \implies x = y = 0 $
        $$ x \in \ker f \implies f(y) = f(-x) = -f(x) = 0 $$
        $$
        \begin{rcases}
            f \clamp{W} \text{ -- инъекция} \\
            f(y) = 0
        \end{rcases} \implies y = 0 \implies x = 0 $$
    \end{enumerate}
\end{proof}

\begin{theorem}[размерность ядра и образа]
	Пусть $ U $ конечномерно, $ f : U \to V $ линейно \\
    Тогда $ \dim \ker f + \dim \Img f = \dim U $
\end{theorem}

\begin{proof}
    Положим $ W : W \cong \Img f, \quad U = \ker f \oplus W $ \\
    По свойству прямой суммы, $ \dim U = \dim \ker f + \dim W \implies \dim U = \dim \ker f + \dim \Img f $
\end{proof}

\begin{theorem}[каноническая форма матрицы линейного отображения]
	Пусть $ U, V $ конечномерны, $ f : U \to V $ линейно \\
    Тогда существуют базисы $ u, v $, в которых матрица $ f $ имеет вид
    $$
    \begin{pmatrix}
        E & 0 \\
        0 & 0
    \end{pmatrix} =
    \begin{pmatrix}
    	1 & 0 & 0 & ... & 0 \\
        0 & 1 & 0 & ... & 0 \\
        . & . & . & . & 0 \\
        0 & 0 & 0 & 0 & 0
    \end{pmatrix} $$
\end{theorem}

\begin{proof}
    $ U = \ker f \oplus W, \qquad f \clamp{W} $ -- изоморфизм из $ W $ в $ \Img f $ \\
    Пусть $ e_1, ..., e_k $ -- базис $ W, \qquad e_{k + 1}, ..., e_n $ -- базис $ \ker f $ \\
    Тогда $ e_1, ..., e_n $ -- базис $ U $ (по свойству прямой суммы) \\
    $ f(e_1), ..., f(e_k) $ -- базис $ \Img f $ (по свойству изоморфизма) \\
    $ f(e_1), ..., f(e_k) $ ЛНЗ \\
    Положим $ g_1 = f(e_1), ..., g_k = f(e_k) $ \\
    Дополним $ g_1, ..., g_k $ до базиса $ V $ \\
    Пусть $ g_1, ..., g_m $ -- базис $ V $ \\
    Докажем, что базисы $ e_1, ..., e_n $ и $ g_1, ..., g_m $ подходят
    \begin{itemize}
    	\item Пусть $ i \le k $
        $$ f(e_1) = g_i = 0 \cdot g_1 + ... + 1 \cdot g_i + ... + 0 \cdot g_k + ... $$
        \item Пусть $ i > k $
        $$ e_i \in \ker f \implies f(e_i) = 0 = 0 \cdot g_1 + ... + 0 \cdot g_m $$
    \end{itemize}
\end{proof}

\begin{implication}
	Пусть $ A $ -- матрица $ n \times n $ с коэффициентами из поля $ K $ \\
    Тогда $ \exist C, D $ -- обратимые матрицы $ n \times n $, такие, что
    $$ c^{-1} A D =
    \begin{pmatrix}
    	E & 0 \\
        0 & 0
    \end{pmatrix} $$
\end{implication}

\begin{proof}
	Пусть $ U = K^n, \qquad e_1, ..., e_n $ -- базис $ U, \qquad f : A $ -- матрица $ f $ в $ e_1, ..., e_n $ \\
    Пусть $ e_1', ..., e_n', \quad e_1'', ..., e_n'' $ -- базисы $ U $, в которых $ f $ имеет матрицу
    $$
    \begin{pmatrix}
    	E & 0 \\
        0 & 0
    \end{pmatrix} \implies C^{-1} A D =
    \begin{pmatrix}
    	E & 0 \\
        0 & 0
    \end{pmatrix} $$
    где $ C, D $ -- матрицы перехода
\end{proof}

\begin{theorem}[линейное отображение и ранг матрицы]
	Пусть $ U, V $ конечномерны, $ f : U \to V $ линейно, $ A $ -- матрица $ f $ в некоторых базисах \\
    Тогда $ \dim \Img f = \rk A $
\end{theorem}

\begin{proof}
	Пусть $ e_1, ..., e_n $ -- базис $ U $, $ g_1, ..., g_m $ -- базис $ V $ \\
    Пусть $ w_i = f(e_i) $ \\
    Тогда $ \Img f = \langle w_1, ..., w_n \rangle $, т. к.
    $$
    \begin{rcases}
    	\forall v \in \Img f ~ \exist u \in U : f(u) = v \\
        \exist a_i : u = a_1e_1 + ... + a_ke_k
    \end{rcases} \implies v = a_1f(e_1) + ... + a_kf(e_k) = a_1w_1 + ... + a_kw_k $$
    Пусть $ X_j =
    \begin{pmatrix}
        a_{1j} \\
        . \\
        . \\
        . \\
        a_{mj}
    \end{pmatrix} $ -- $j$-й столбец матрицы $ f $ \\
    Тогда $ w_j = a_{1j}g_1 + ... + a_{mj}g_m $
    $$ \rk A = \dim \langle X_1, ..., X_n \rangle, \qquad \dim f = \dim \langle w_1, ..., w_n \rangle $$
    Из любой порождающей системы можно выбрать базис $ \implies \dim \langle w_1, ..., w_n \rangle $ равна максимальному количеству ЛНЗ векторов из $ w_1, ..., w_n $ \\
    Аналогично для $ X_1, ..., X_n $ \\
    Пусть $ v = c_1w_1 + ... + c_nw_n, \quad X $ -- столбец координат базиса \\
    Тогда $ X = c_1X_1 + ... + c_nX_n $
    \begin{multline*}
        v = c_1w_1 + ... + c_nw_n = c_1(a_{11}g_1 + ... + a_{i1}g_1 + ... + a_{m1}g_m) + ... + c_n(a_{1n}g_1 + ... + a_{in}g_i + ... + a_{mn}g_m) = \\ = (c_1a_{11} + ... + c_na_{1n}g_1 + ... + (c_1a_{i1} + ... + c_na_{1n})g_i + ...
    \end{multline*}
    $$ v = 0 \iff x = 0 $$
    $$ c_1w_1 = ... + c_nw_n = 0 \iff c_1x_1 + ... + c_nx_n = 0 $$
\end{proof}

\section{Действия над линейными отображениями}

\begin{definition}
	Пусть $ f, g : U \to V, \quad k $ -- скаляр \\
    Отображением $ f + g $ называется такое отображение, что $ (f + g)(u) = f(u) + g(u) $ \\
    Отображением $ kf $ называется такое отображение, что $ (kf)(u) = k \cdot f(u) $
\end{definition}

\begin{remark}
	$ f + g, ~ kf $ линейны
\end{remark}

\begin{definition}
	Произведением $ f : V \to W $ и $ g : U \to V $ называется $ fg = f \circ g : U \to V $ \\
    В частности, $ f^n = \underbrace{f \circ f \circ ... \circ f}_n : U \to U $
\end{definition}

\begin{remark}
	$ fg, ~ f^n $ линейны
\end{remark}

\begin{lemma}[действия над отображением и матрицей]
	\hfill
    \begin{enumerate}
    	\item Пусть $ U, V $ конечномерны, $ e_i, e_i' $ -- их базисы, $f, g : U \to V $ линейны, $ A, B $ -- матрицы $ f $ и $ g $, $ a, b $ -- скаляры \\
        Тогда $ aA + bB $ -- матрица $ af + bg $
        \begin{proof}
        	Пусть $ u \in U, \quad X $ -- столбец координат $ u $ в $ e_i, \quad Y_1, Y_2 $ -- столбцы координат $ f(u), g(u) $ в $ e_i \implies Y_1 = AX, ~ Y_2 = BX \implies aY_1 + bY_2 = aAX + bBX = (aA + bB)X $
            $$ (af + bg)(u) = af(u) + bg(u) \implies \text{ столбец координат } (af + bg)(u) = aY_1 + bY_2 = (aA + bB)X $$
        \end{proof}
        \item Пусть $ U, V, W $ конечномерны, $ e_i, e_i', e_i'' $ -- их базисы, $ f : V \to W, g : U \to V $ линейны, $ A, B $ -- матрицы $ f, g $ \\
        Тогда $ AB $ -- матрица $ fg $
        \begin{proof}
        	$ u \in U, w \in W : (fg)(u) = w $ \\
            $ X, Z $ -- столбцы координат \\
            Пусть $ v = g(u), \quad Y $ -- толбец координат $ V \implies Y = BX, ~ Z = AY \implies Z = A(BX) = (AB)X $
        \end{proof}
    \end{enumerate}
\end{lemma}

\begin{theorem}[пространство линейных отображений]
	$ U, V $ -- векторные пространства над полем $ K $. Тогда:
    \begin{enumerate}
    	\item Множество линейных отображений образует векторное пространство над $ K $
        \item Если $ \dim U = m, ~ \dim V = n $, то пространство линейных отображений изоморфна пространству матриц размера $ m \times n $, его размерность равна $ mn $
    \end{enumerate}
\end{theorem}
