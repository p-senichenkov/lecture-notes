\chapter{Комплексные числа}

\section{Тригонометрическая форма комплексного числа (продолжение)}

\begin{statement}
    $$ n \in \Z $$
    $$ z = r(\cos \vphi + i \sin \vphi) \implies z^n = r^n (\cos (n\vphi) + i \sin (n \vphi)) $$
\end{statement}

\begin{proof}
    \hfill
    \begin{itemize}
        \item $ n = 0 $
        $$ z^0 = 1 \quad r_0 (\cos (0 \cdot \vphi) + i \sin (0 \cdot \vphi)) = 1 $$
        \item $ n > 0 $
        \begin{multline*}
            z^n = \underbrace{r (\cos \vphi + i \sin \vphi) \cdot r (\cos \vphi + i \sin \vphi)...}_{n \text{ раз}} = \\ = \underbrace{r \cdot r \cdot ... \cdot r}_{n \text{ раз}}(\cos (\underbrace{\vphi + \vphi + ...}_{n \text{ раз}}) + i\sin (\underbrace{\vphi + \vphi + ...}_{n \text{ раз}})) = r^n (\cos (n \vphi) + i \sin (n \vphi))
        \end{multline*}
        \item $ n < 0 $ \\
        Положим $k = -n, ~ k > 0 $
        $$ z^n = \frac1{z^k} = \frac1{r^k(\cos (n \vphi) + i \sin (n \vphi))} = \frac1{r^k}(\cos (-k \vphi) + i \sin (-k \vphi)) = r^n(\cos (n \vphi) + i \sin (n \vphi)) $$
    \end{itemize}
\end{proof}

\begin{eg}
    Найти $(-\sqrt3 + i)^{10} $
    $$ z = -\sqrt3 + i $$
    $$ r = \sqrt{(-\sqrt3)^2 + 1^2} = 2 $$
    $$ \cos \vphi = - \frac{\sqrt3}2 \quad \sin \vphi = \frac12 \quad \vphi = \frac{5\pi}6 $$
    $$ z = 2(\cos \frac{5\pi}6 + i \sin \frac{5\pi}6) $$
    $$ z^{10} = 2^{10} (\cos (\frac{5\pi}6 \cdot 10 + i \sin (\frac{5\pi}6 c\dot 10))) = 1024 (\cos \frac{\pi}3 + i \sin \frac{\pi}3) = 512 + 512 \sqrt3 i $$
\end{eg}

\begin{theorem}[Извлечение корня в тригонометрической форме]
	Пусть $a \in \Co, ~ a \ne 0, ~ n \in \N $. \\
    Тогда уравнение $z^n = a $ имеет $n$ решений. \\
    Если $ a = r(\cos \vphi + i \sin \vphi) $, то решениями уравнения являются числа:
    $$ z_k = r^{\frac1n} (\cos \frac{\vphi + 2 \pi k}n + i \sin \frac{\vphi + 2 \pi k}n), \quad k = 0, 1, ..., n - 1 $$
\end{theorem}

\begin{proof}
	Ищем решение в виде $z = \rho (\cos \psi + i \sin \psi) $. Возведём $z$ в $n$ степень:
    $$ \rho^n (\cos (n \psi) + i \sin (n \psi)) = r(\cos \vphi + i \sin \vphi) $$
    $$ \begin{cases} \rho^n = r \\ n\psi = \vphi + 2 \pi k, ~ k \in \Z \end{cases} \quad \implies \quad \begin{cases} \rho = r^{\frac1n} \\ \psi = \frac{\vphi + 2 \pi k}n, ~ k \in \Z \end{cases} $$
    Проверим, что при $ k = 0, 1, .., n - 1 $ корни различны, и любой корень совпадает с одним из них:\\
    Положим $ z_k = r^{\frac1n} (\cos \frac{\vphi + 2 \pi k}n + i \sin \frac{\vphi + 2 \pi k}n) $
    \begin{multline*}
        z_k = z_l \implies \arg z_k = \arg z_l \iff \frac{\phi + 2 \pi k}n = \frac{\phi + 2 \pi l}n + 2 \pi m \iff \\ \iff \phi + 2 \pi k = \phi + 2 \pi l + 2 \pi m \cdot n \iff k = l + mn \iff k \comp{n} l
    \end{multline*}
\end{proof}

\begin{eg}
    $$ \sqrt[3]{8i} $$
    $$ z^3 = 8i $$
    $$ r = 8 \quad \phi = \frac{\pi}2 $$
    $$ z_k = \sqrt[3]{8}(\cos \frac{\frac{\pi}2 + 2 \pi k}3 + i \sin \frac{\frac{\pi}2 + 2 \pi k}3), \quad k = 0, 1, 2 $$
    \begin{itemize}
    	\item $ k = 0 $
        $$ z_0 = 2 \cdot (\cos \frac{\pi}6 + i \sin \frac{\pi}6) = \sqrt3 + i $$
        \item $ k = 1 $
        $$ z_1 = 2 (\cos \frac{5\pi}6 + i \sin \frac{5\pi}6) = - \sqrt3 + i $$
        \item $ k = 2 $
        $$ z_2 = 2 (\cos \frac{3\pi}2 + i \sin \frac{3\pi}2) = -2i $$
    \end{itemize}

\end{eg}

\begin{definition}
    Число $ \veps \in \Co $ называется корнемм $n$-й степени из единицы, если $ \veps^n = 1 $
\end{definition}

\begin{notation}
    $$ \veps_k = \cos \frac{2 \pi k}n + i \sin \frac{2 \pi k}n $$
\end{notation}

\begin{eg}
	$ n = 4 $
    $$ \veps_0 = 1 $$
    $$ \veps_1 = i $$
    $$ \veps_2 = -1 $$
    $$ \veps_3 = -i $$
\end{eg}

\begin{properties}
    \hfill
    \begin{enumerate}
    	\item Корни $n$-й степени из единицы образуют группу по умножению
        \begin{proof}
            \hfill
            \begin{enumerate}
                \item Ассоциативность -- всегда верно в $\Co$
                \item Нейтральный элемент: 1 -- корень $n$-й степени из 1
                \item Обратный элемент:
                $$ (\frac1x)^n = \frac{1^n}{x^n} = \frac11 = 1 $$
                \item Замкнутость относительно операции:
                $$ (xy)^n = x^ny^n = 1 \cdot 1 = 1 $$
            \end{enumerate}
        \end{proof}
        \item Пусть $a \in \Co, ~ a \ne 0, ~ x$ -- корень $n$-й степени из $a$. Тогда $\veps_0x, \veps_1x, ..., \veps_nx $ -- все корни $n$ степени из $a$
        \begin{proof}
            Докажем, что, если $y = \veps_ix$, то $y$ -- корень $n$-й степени из 1:
            $$ y^n = \veps_i^nx^n = 1 \cdot a $$
            Докажем, что, если $y$ -- корень $n$-й степени из 1, то $y = \veps_ix$, т. е. $\frac{y}x$ -- корень $n$-й степени из 1:
            $$ (\frac{y}x)^n = \frac{y^n}{x^n} = \frac{a}a = 1 $$
        \end{proof}
    \end{enumerate}
\end{properties}

\begin{definition}
	Число $\veps \in \Co$ называется перовобразным корнем $n$-й степени из 1, если $\veps^n = 1, ~ \veps^k \ne 1$ при $1 \le k < n $. Также называется корнем, принадлежащим показателю $n$
\end{definition}

\begin{eg}
	$ n = 4 \quad 1, i, -1, -i $ \\
    $i, -i $ -- первообразные (принадлежат показателю 4) \\
    -1 -- первообразный корень степени 2 \\
    1 -- первообразный корень степени 1
\end{eg}

\begin{props}
    \item $\veps_k$ -- первообразный корень $n$-й степени из 1 $ \iff (k,n) = 1 $
    \begin{proof}
    	Пусть $km = nq + r, \quad 0 \le r \le n $
        $$ \veps_k = (\cos \frac{2 \pi k}n + i \sin \frac{2 \pi k}n)^m = \cos \frac{2 \pi k m}n + i \sin \frac{2 \pi k m}n = \cos \frac{2 \pi r}n + i \sin \frac{2 \pi r}n $$
        \begin{itemize}
        	\item Пусть $(k,n) = 1$. Если $\veps_k^m = 1$, то $mk \divby n \implies m \divby n \implies m \ge n $
            \item Пусть $(k,n) = d \ne 1 \implies \frac{n}d \cdot k \divby n \implies \veps_k^{\frac{n}d} = 1 \implies \veps_k$ -- не первообр.
        \end{itemize}
    \end{proof}
    \item Пусть $\veps_k$ -- первообразный корень. Тогда любой корень $n$-й степени из 1 равен $\veps_k^m$ для некоторого $m$
    \begin{proof}
    	$\veps_k, \veps_k^2, .., \veps_k^n$ -- корни $n$-й степени из 1. Докажем, что они различны: \\
        Пусть $\veps_k^m = \veps_m^l, \quad 1 \le m < l \le n \implies \veps_k^{l-m} = 1, \quad 1 \le l-m < n $
    \end{proof}
\end{props}

\chapter{Полиномы}

\section{\S1. Кольцо многочленов}

Будем рассматривать многочлены как бесконечные последовательность коээфициентов (начиная с какого-то момента -- нули)

\begin{definition}
	Пусть $A$ -- кольцо. Многочленом над $A$ будем называть последовательность $(a_1, a_2, ...)$, в которой только конечное количество членов отлично от нуля.
\end{definition}

\begin{definition}
	Суммой многочленов $(a_0, a_1, ...)$ и $(b_0, b_1, ...)$ называется многочлен $(c_0, c_1, ...)$, где $c_k = a_k + b_k ~ \forall k$
\end{definition}

\begin{definition}
    Произведением многочленов $(a_0, a_1, ...)$ и $(b_0, b_1, ...)$ называется многочлен $(d_0, d_1, ...)$, где $d_k = a_0b_k + a_1b_{k-1} + ... + a_kb_0$, то есть $d_k = \sum_{i + j = k} a_ib_j$
\end{definition}

\begin{notation}
	Множество многочленов над $A$ обозначается $A[x]$
\end{notation}

\begin{theorem}
    \hfill
    \begin{enumerate}
    	\item Сумма и произведение многочленов определены корректно, т. е. в последовательностях $(c_0, c_1, ...)$ и $(d_0, d_1, ...)$ только конечное колическто членов отлично от нуля
        \item $A[x]$ -- кольцо
    \end{enumerate}
\end{theorem}

\begin{proof}
	\hfill
    \begin{enumerate}
        \item Пусть $N,M : \begin{Bmatrix} a_i = 0 \text{ при } i > N \\ b_i = 0 \text{ при } i > M \end{Bmatrix} \implies c_i = 0 \text{ при } i > \max\set{N,M} $
        \item Докажем, что $d_k = 0$ при $k > M + N$:
        $$ d_k = a_0b_k + ... + a_kb_0 = \sum_{i+j=k}a_ib_j$$
        $$ i + j = k \implies i + j > M + N \implies \begin{bmatrix} i > N \\ j > M \end{bmatrix} \implies \begin{bmatrix} a_i = 0 \\ b_j = 0 \end{bmatrix} \implies a_ib_j = 0 $$
     \end{enumerate}
\end{proof}
