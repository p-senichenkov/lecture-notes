\chapter{}

\section{Ещё одну теорему забыл}

\begin{theorem}[Вейерштрасса о прямом умножении]
	$ X $ -- связное топологическое пространство \\
    $ f : X \to \R $ -- непрерывная функция
    $$ f(x_0) = a, \qquad f(x_1) = b, \qquad a \le c \le b $$
    $$ \implies \exist x_2 \in X : f(x_2) = c $$
\end{theorem}

\begin{proof}
	$ X $ -- связное $ \implies f(X) $ -- связное \\
    Какие связные подмножества есть в $ \R $?
\end{proof}

\begin{theorem}
	$ (X, \Omega) $ -- топологическое пространство \\
    Равносильны следующие утверждения:
    \begin{enumerate}
        \item \label{en:1} Компоненты связности $ X $ открыты
        \item \label{en:2} $ X = \bigcup_{i \in I} X_i $ ($ X_i $ -- \textit{компоненты}, топология $ X $ -- топология объединения, контрпример -- $ \Q $)
        \item \label{en:3} У любой точки существует связная окрестность
    \end{enumerate}
\end{theorem}

\begin{proof}
	Упражнение
\end{proof}

\section{Линейная связность}

\begin{definition}
	Пусть в $ X $ -- непрерывная $ f : [0, 1] \to X $
    \begin{itemize}
    	\item $ f(0) = x_0 $ -- начало пути
        \item $ f(1) = x_1 $ -- конец пути
    \end{itemize}
\end{definition}

\begin{definition}
	$ X $ называется линейно свяным, если $ \forall x_0, x_1 \in X \quad \exist $ путь, соединяющий $ x_0 $ и $ x_1 $
\end{definition}

\begin{definition}
    $ A \sub X $ линейно связно, если $ \forall x_0, x_1 \quad \exist \underset{\text{(не выходящий за пределы } A )}{\text{ путь в } A} $ между ними
\end{definition}

\begin{theorem}
	Линейно связное пространство связно
\end{theorem}

\begin{proof}
	Допустим, не связно
    $$ X = U_1 \cup U_2, \qquad U_1 \cap U_2 \ne \O, \qquad x_1 \in U_1, \quad x_2 \in U_2 $$
    Соединим их путём:
    $$ \exist f : [0, 1] \to X :
    \begin{cases}
    	f(0) = x_1 \\
        f(1) = x_2
    \end{cases} $$
    $ f^{-1}(U_1), f^{-1}(U_2) $ разбивают $ [0, 1] \implies [0, 1] $ не связен -- \contra (доказано, что $ [0, 1] $ связен)
\end{proof}

\begin{undefthm}{Верна ли обратная стрелка?}
	Нет
    \begin{eg}
    	$ y = \sin\frac1x $ -- непрерывна на $ (0, +\infty) $
    \end{eg}
\end{undefthm}

\begin{figure}[!ht]
    \begin{tikzpicture}[domain=0.0001:2]
        \draw[->][name path=ox] (-0.2,0) -- (3.2,0);
        \draw[->] (0,-1.2) -- (0,1.2);

        \draw[green!50!black, name path=sin, samples=50] plot function{sin(1 / x)};
    \end{tikzpicture}
    \caption{Связная, но не линейно связная функция}
\end{figure}

\section{Компактность}

\subsection{Понятие компактности}

\begin{definition}
	$ X $ -- топологическое пространство \\
    $ \set{U_i}_{i \in I} $ -- (открытое) покрытие $ X $, если $ \bigcup_{i \in I} U_i = X $ и $ \forall i \quad U_i \in \Omega $ \\
    В этом курсе, покрытие $ \equiv $ открытое покрытие
\end{definition}

Если $ \set{U_i}_{i \in I} $ -- покрытие $ X $ и $ \set{U_{ij}}_{ij \in J \sub I} $ -- покрытие $ X $, то это -- подпокрытие $ \set{U_i}_{i \in I} $

\begin{definition}
	$ X $ называют компактным, если для любого покрытия можно выбрать конечное подпокрытие
\end{definition}

\begin{definition}
	$ A \sub X $ компактно, если $ A $ компактно в индуцированной топологии, то есть
    $$ \underset{(U_i \text{ открыто в } X)}{\forall \set{U_i}_{i \in I}} : \bigcup_{i \in I} U_i \supset A \quad \exist U_{i1}, U_{i2}, ..., U_{in} : \bigcup_{k = 1}^n U_{ik} \supset A $$
\end{definition}

\begin{theorem}
	$ X $ компактно, $ \quad A $ замкнуто в $ X \qquad \implies \quad A $ компактно
\end{theorem}

\begin{proof}
    Рассмотрим любое $ \set{U_i}_{i \in I} $ -- покрытие $ A $ \\
    Рассмотрим $ V \define X \setminus A $ -- открытое \\
    $ \set{U_i, V}_{i \in I} $ -- покрытие $ X $ \\
    $ \implies \exist $ конечное подпокрытие. Выпишем его:
    $$ V, U_{i1}, U_{i2}, ..., U_{in} $$
\end{proof}

\begin{theorem}
	$ f : X \to Y $ непрерывна, $ \qquad A \sub X $ компактно $ \qquad \implies \quad f(A) $ компактно
\end{theorem}

\begin{proof}
    Возьмём $ \set{V_i}_{i \in I} $ -- покрытие $ f(A) $
    $$ V_i \sub Y \implies f^{-1}(V_i) \text{ -- открытое в } X $$
    $ \set{f^{-1}(V_i)}_{i \in I} $ -- покрытие $ A $ \\
    $ \implies f^{-1}(V_{i1}), ..., f^{-1}(V_{in}) $ -- конечное подпокрытие $ A $
\end{proof}

\begin{implication}
	Компактность -- топологическое свойство (т. е. она сохраняется при гомеоморфизме)
\end{implication}

\subsection{Компактность и хаусдорфовость}

\begin{definition}
    $ X $ называется хаусдорфовым, если $ \underset{x \ne y}{\forall x, y \in X} \quad \exist U_x, U_y : U_x \cap U_y = \O $
\end{definition}

\begin{eg}
	$ X $ -- метрическое $ \implies X $ -- хаусдорфово
    $$ U_x \define B \bigg( x, \frac{\rho(x, y)}3 \bigg), \qquad U_y \define B \bigg( y, \frac{\rho(x, y)}3 \bigg) $$
\end{eg}

\begin{theorem}
	$ X $ -- хаусдорфово, $ \quad A $ компактно $ \qquad \implies \quad A $ замкнуто
\end{theorem}

\begin{proof}
    Нужно доказать, что $ X \setminus A $ открыто \\
    Зафиксируем $ x_0 \in X \setminus A $
    $$ \forall a \in A \quad \exist
    \begin{Bmatrix}
        U_{ax_0} \text{ -- окрестность } a \\
        V_{ax_0} \text{ -- окрестность } x_0
    \end{Bmatrix} : U_{ax_0} \cap V_{ax_0} = \O $$
    $ \set{U_{ax_0}}_{a \in A} $ -- покрытие $ A $ \\
    $ \implies \exist U_{ax_01}, ..., U_{ax_0n} $ -- подпокрытие \\
    Возьмём $ V \define \bigcap_{k = 1}^n V_{ax_0k} $ \\
    $ V $ открыто, $ \qquad V \cap A = \O, \qquad x_0 \in V $ \\
    $ \implies X \setminus A $ открыто
\end{proof}

\begin{implication}
	$ X $ -- компактно и хаусдорфово, $ \quad A \sub X $ \\
    $ A $ компактно $ \iff A $ замкнуто
\end{implication}
