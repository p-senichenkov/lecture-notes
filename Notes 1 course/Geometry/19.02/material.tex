\chapter{Метрические пространства}

\begin{lemma}
	$ B(x_0, \veps) $ открытый
\end{lemma}

\begin{proof}
	$$ \forall x_1 \in B(x_0, \veps) $$
	$$ \delta \define \veps - p(x_1, x_0) $$
	$$ B(x_1, \delta) \sub B(x_0, \veps) \text{ -- ?} $$
	$$ \rho(x_1, x_2) < \delta = \veps - \rho(x_0, x_1) $$
	$$ \rho(x_0, x_2) \le \rho(x_0, x_1) + \rho(x_1, x_2) < \veps $$
	$$ x_2 \le B(x_0, \veps) $$
\end{proof}

\begin{theorem}
	Равносильны определения:
	\begin{enumerate}
		\item \label{it:1} Множество внутренних точек $\operatorname{Int} A$
		\item \label{it:2} $ \bigcup_{
			\begin{subarray}{c}
				U \sub A \\
				U \text{ открытое}
			\end{subarray}} U $
		\item \label{it:3} Максимальное открытое подмножество $A$
	\end{enumerate}
\end{theorem}

\begin{proof}
	\hfill
	\begin{itemize}
		\item \eqref{it:2} $ \iff $ \eqref{it:3}
		$$ \bigcup_{i \in I} U_i \text{ открытое, если } \forall i \quad U_i \text{ открытое} $$
		$$ x_0 \in \bigcup_{i \in I} U_i $$
		$$ \exist x_0 \in U_i $$
		$$ \implies \exist \veps : B(x_0, \veps) \sub U_i \sub \bigcup_{i \in I} U_i $$
		Почему это множество внутренних точек?
		$$ x_0 \in \bigcup_{
			\begin{subarray}{c}
				U_i \sub A \\
				U_i \text{ открытое}
			\end{subarray}} U_i \sub A $$
		$$ \exist \veps : B(x_0, \veps) \sub \bigcup_{
			\begin{subarray}{c}
				U_i \sub A \\
				U_i \text{ открытое}
			\end{subarray}} U_i \sub A $$
		Если $x_0$ -- внутренняя для $A$, то $ \exist \veps : B(x_0, \veps) \sub A $
	\end{itemize}
\end{proof}

\begin{theorem}[свойства открытых множеств]
	\hfill
	\begin{enumerate}
		\item $ \bigcup_{i \in I} U_i$ открытое, если $\forall i \quad U_i$ открытое
		\begin{proof}
			$ x_0 \in \bigcup_{i \in I} U_i \implies \exist i : x_0 \in U_i \implies \exist \veps > 0 : B(x_0, \veps) \sub U_i \sub \bigcup_{i \in I} U_i $
		\end{proof}
		\item $ U_1, U_2, ..., U_n$ -- открытые $ \implies \bigcap_{i = 1}^n U_i $ открытое
		\begin{proof}
			$$ x_0 \in \bigcap_{i = 1}^n U_i \implies \forall i \quad x_0 \in M \implies \exist \veps_i : B(x_0, \veps_i) \sub U_i $$
			$$ \veps \define \min\limits_{i = 1 : n}\set{ \veps_i } \implies B(x_0, \veps) \sub B(x_0, \veps_i) \sub U_i $$
			$$ B(x_0, \veps) \sub \bigcap_{i = 1}^n U_i $$
		\end{proof}
		\item $ \O, M $ -- открытые
	\end{enumerate}
\end{theorem}

\begin{definition}
	$F$ называется замкнутым, если $ M \setminus F $ -- открытое
\end{definition}

\begin{theorem}[свойства замкнутых множеств]
	\hfill
	\begin{enumerate}
		\item Если $ \set{F_{i \in I}}$ -- замкнутое $\implies \bigcap_{i \in I} F_i $ -- замкнутое
		\begin{proof}
			$$ U_i \define M \setminus F_i \text{ -- открытое} $$
			$$ \bigcup U_i = \bigcup(M \setminus F_i) = M \setminus \bigcap F_i \implies \bigcap F_i \text{ -- замкнутое} $$
		\end{proof}
		\item Если $F_1, ..., F_n$ -- замкнутые $\implies \bigcup_{i = 1}^n$ -- замкнутое
		\item $ \O, M$ -- замкнутые
	\end{enumerate}
\end{theorem}

\begin{remark}
	Открытое -- \textbf{не} обязательно не замкнутое
\end{remark}

\begin{exmpls}
	\item $M = [0, 1] \cup [2, 3]$; стандартная метрика \\
	$ \underset{=B(\faktor12, 1)}{[0, 1]} $ открыт в $M$ (но не открыт в $\R$) \\
	Аналогично, $[2, 3] = B(2,5; 1) $ -- открытый $ \implies [0, 1] = \overline{[2, 3]} $ -- замкнутый
	\item Дискретная метрика $\rho =
	\begin{cases}
		1, \quad x \ne y \\
		0, \quad x = y
	\end{cases} $ \\
	$ B(x_0, 1) = \set{x_0} $ -- открытое \\
	$ \forall U = \bigcup_{x_0 \in U} \set{x_0} $ -- открытое \\
	Также, (так как все открытые), все замкнутые
\end{exmpls}

\begin{remark}
	Существуют множества, не открытые, и не замкнутые
\end{remark}

\begin{eg}
	$ [a, b) \in \R$; метрика стандартная
\end{eg}


\begin{theorem}[равносильные определения замыкания]
	$M$ -- метрическое пространство, $A \sub M$ \\
	Равносильны определения:
	\begin{enumerate}
		\item \label{it:11} Множество внутренних и граничных точек называется замыканием
		\item \label{it:12} $M \setminus \operatorname{Int} (M \setminus A) $
		\item \label{it:13} $ \bigcap_{
			\begin{subarray}{c}
				F \supset A \\
				F \text{ замкнутое}
			\end{subarray}} F $
		\item \label{it:14} Минимальное замкнутое надмножество $A$
	\end{enumerate}
\end{theorem}

\begin{proof}
	\hfill
	\begin{itemize}
		\item \eqref{it:13} $ \iff $ \eqref{it:14}
		\item \eqref{it:11} $ \iff $ \eqref{it:12}, так как
		$$ \operatorname{Int} (M \setminus A) = \operatorname{Ext} A $$
		\item \eqref{it:12} $ \iff $ \eqref{it:13}, так как
		$$ M \setminus \operatorname{Int} (M \setminus A) = M \setminus \bigcup_{
			\begin{subarray}{c}
				U \subset M \setminus A \\
				U \text{ открытое}
			\end{subarray}} U = \bigcap_{
			\begin{subarray}{c}
				U \subset M \setminus A \\
				U \text{ открытое}
			\end{subarray}} M \setminus U = \bigcap_{
			\begin{subarray}{c}
				F \supset A \\
				F \text{ замкнутое}
			\end{subarray}} F $$
	\end{itemize}
\end{proof}

\chapter{Топологические пространства}

\begin{definition}
	$X$ -- множество, $ \Omega \sub 2^X $ \\
	$(X, \Omega)$ называется топологическим пространством, если
	\begin{enumerate}
		\item $ \forall \set{U_i} \in \Omega \quad \bigcup U_i \in \Omega $
		\item $ U_i, ..., U_n \in \Omega \implies \bigcap_{i = 1}^n U_i \in \Omega $
		\item $\O, X \in \Omega $
	\end{enumerate}
	$\Omega$ называется топологией над $X$ \\
	$U \in \Omega$ называется открытым
\end{definition}

\begin{definition}
	$F$ называется замкнутым, если $X \setminus F $ открытое
\end{definition}

\begin{theorem}
	\hfill
	\begin{enumerate}
		\item $ \set{F_i}$ -- замкнутое $\implies \bigcap_{i \in I} F_i $ -- замкнутое
		\item $F_1, ..., F_n$ -- замкнутое $\implies \bigcup_{i = 1}^n F_i $ -- замкнутое
		\item $\O, X$ -- замкнутые
	\end{enumerate}
\end{theorem}

\begin{note}
	Топологическое пространство можно задавать через замкнутые множества (вместо открытых)
\end{note}

\begin{remark}
	Любое метрическое пространство является топологическим пространством
\end{remark}


\begin{exmpls}
	\item $(M, \rho)$ -- метрическое $\implies$ $M$ -- топологическое
	\item $ \forall X, \quad \Omega = \set{\O, X} $ -- антидискретная топология \\
	$M$ -- топологическое, но \textbf{не} метрическое
	\item $ \forall X, \quad \Omega = 2^X $ -- дискретная топология \\
	Любое подмножество будет открытым (а, следовательно, и замкнутым)
	\item ``Топология Зариского'' или Топология конечных дополнений \\
	Будем называть множество замкнутым, если оно конечное или $X$ ($X$ -- бесконечный) \\
	Открытые -- те, дополнения до которых конечны
	\item Стрелка \\
	$X = \R$ (или $\R_+$, или ...) \\
	Открытыми будут лучи $(a, +\infty) $ или $\O$ или $\R$
	\begin{note}
		Если $(a, +\infty) $ заменить на $[a, +\infty)$, то это не будет топологией (т. к. $\bigcup [\frac1n; \infty] = (0, \infty)$)
	\end{note}
	\item Топология Зариского \\
	$ X = \Co $ (важно, что не $\R$) \\
	Замкнутым будем называть множество корней некоторого многочлена $f(x)$
	$$ F_1 \iff f_1, \qquad F_2 \iff f_2 $$
	$$ F_1 \cup F_2 \iff f_1 \cdot f_2 $$
	$$ F_1 \cap F_2 \iff \GCD{f_1, f_2} $$
\end{exmpls}

\begin{definition}
	$R$ -- коммутативное кольцо, $I \sub R$ \\
	$I$ называется идеалом $F$, если
	\begin{enumerate}
		\item $x, y \in I \implies x + y \in I $
		\item $x \in I; y \in R \implies x \cdot y \in I $
	\end{enumerate}
\end{definition}
