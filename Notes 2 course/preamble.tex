% basics
\usepackage[utf8x]{inputenc}
\usepackage[T2A]{fontenc}
%\usepackage{textcomp}
%\usepackage{url}
\usepackage[a4paper, portrait, margin=.75in]{geometry}
\usepackage[russian]{babel}

% custom \chapter and \section formats
\usepackage{titlesec}

% references in table of contents
\usepackage{hyperref}
\hypersetup{
    colorlinks=true, %set true if you want colored links
    linktoc=all,     %set to all if you want both sections and subsections linked
    linkcolor=black, %choose some color if you want links to stand out
}
\urlstyle{same}      % links will be the same font as other text

\usepackage{graphicx}
%\usepackage{float}
%\usepackage{booktabs}
%\usepackage{enumitem}
%\usepackage{emptypage}
\usepackage{multicol}
\usepackage[dvipsnames]{xcolor}

\usepackage{braket}

% typeset pseudocode (algo2e option -- change env. name algorithm to algorithm2e)
\usepackage[algo2e]{algorithm2e}

% algorithm2e keywords:
\SetKw{Define}{:=}

% sections with separate PDF bookmark form
\usepackage{hypbmsec}

% subequations
\usepackage[ntheorem]{empheq}

\usepackage{amsmath, amsfonts, mathtools, amsthm, amssymb}

% alternative caligraphy font
%\usepackage{mathrsfs}

% diagonal lines through items ("cancelling" 'em)
\usepackage{cancel}

% for \lightning (should be loaded *before* bm)
\usepackage[only,lightning]{stmaryrd}

% https://tex.stackexchange.com/questions/106714/stmaryrd-and-boldsymbol-avoid-warnings
\SetSymbolFont{stmry}{bold}{U}{stmry}{m}{n}

\usepackage{tikz}
\usetikzlibrary{intersections, angles, calc, positioning, arrows.meta, decorations.pathreplacing}

% multiple images in one figure
\usepackage{subcaption}

% figure, wrapped in text
\usepackage{wrapfig}

% fractions like 2/a
\usepackage{faktor}

% xlongequal
\usepackage{extarrows}

% bold font in maths, that preserve spacing (should be loaded *last*)
\usepackage{bm}

%%      math operators
% distance
\DeclareMathOperator{\dist}{dist}

% constant
\DeclareMathOperator{\const}{const}

% matrix rank
\DeclareMathOperator{\rk}{rk}

% image
\DeclareMathOperator{\Img}{Im}

% interior
\DeclareMathOperator{\Int}{Int}

% exterior
\DeclareMathOperator{\Ext}{Ext}

% frame
\DeclareMathOperator{\Fr}{Fr}

% closure
\DeclareMathOperator{\Cl}{Cl}

%%  group and set
\DeclareMathOperator{\Orb}{Orb}
\DeclareMathOperator{\St}{St}
\DeclareMathOperator{\Fix}{Fix}
\DeclareMathOperator{\ord}{ord}

% gradient
\DeclareMathOperator{\grad}{grad}

% differential
\newcommand\di[1][]{\ensuremath{\operatorname{d}^{#1} \phantom{.}}}

% partition diameter
\DeclareMathOperator{\diam}{d}

% sign
\DeclareMathOperator{\sign}{sign}

%%      numerating
% labeled equation
\newenvironment{equ}[1]{\begin{equation} \label{eq:#1}}{\end{equation}}
% multiline equation with *every* line numbered
\newenvironment{mequ}[1][\empheqlbrace]
{
    \setkeys{EmphEqEnv}{align} \setkeys{EmphEqOpt}{left={#1}} \EmphEqMainEnv
}
{\endEmphEqMainEnv}

% variants
\newenvironment{vars}{\left[\begin{array}{l}}{\end{array}\right.}

%%      KaTeX
\newcommand\sub{\ensuremath\subset}

\newcommand\exist{\exists\,}

%%      shorthands
% phi and epsilon
\let\vphi\varphi
\let\veps\varepsilon

% small triangle
\let\vtri\vartriangle

%% over-decorations
\let\ol\overline
\let\vawe\widetilde
\let\hat\widehat
% long \vec (DEPRECATED)
\newcommand*\vect[1]{\overrightarrow{#1}}
% long \vec
\let\vec\overrightarrow

\let\mc\mathcal

% partial derivative
\newcommand*\pder[2]{\ensuremath{\frac{\partial {#1}}{\partial {#2}}}}

% arrows with text below `em
\newcommand\underarr[1]{\ensuremath\xrightarrow[#1]{}}
\newcommand\underimp[1]{\ensuremath\xRightarrow[#1]{}}
\newcommand\underiff[1]{\ensuremath\xLeftrightarrow[#1]{}}

% arrows with text above `em
\newcommand\overarr[1]{\ensuremath\xrightarrow{#1}}

% not-arrows with text below `em
\newcommand\undereq[1]{\ensuremath\xlongequal[#1]{}}

% special cases
\newcommand\limarr[2]{\ensuremath\underarr{{#1} \to {#2}}}
\newcommand\infarr[1]{\ensuremath\limarr{#1}\infty}
\newcommand\zarr[1]{\ensuremath\limarr{#1}0}

% labels and references
\newcommand*\lbl[1]{\label{eq:#1}}
\newcommand*\eref[1]{\eqref{eq:#1}}

% limits
\newcommand\ulim{\ensuremath{\varlimsup\limits}}
\newcommand\dlim{\ensuremath{\varliminf\limits}}

\let\liminf\varliminf
\let\limsup\varlimsup

\newcommand\liml[1]{\lim\limits_{#1}}
\newcommand\limz[1]{\lim\limits_{#1\to0}}
\newcommand\limi[1]{\lim\limits_{#1\to\infty}}

% integrals
\newcommand\uint[2][x]{\ensuremath{\int {#2} \, \operatorname{d} {#1}}}
\newcommand\ufint[2][x]{ \ensuremath{ \int{ \dfrac{\operatorname{d} {#1}}{{#2}} } } }
\newcommand\dint[4][x]{\ensuremath{\int_{#2}^{#3} {#4} \, \operatorname{d} {#1}}}
\newcommand\dfint[4][x]{\ensuremath{\int_{#2}^{#3} \dfrac{\operatorname{d} {#1}}{{#4}}}}

% Riman's integral
\newcommand\Ri{\ensuremath{\mathcal{R}}}

% comparable by modulo
\newcommand*\comp[1]{\ensuremath{\underset{#1}{\equiv}}}

%% definition:
% :=
\newcommand\define\coloneqq
% :{#1}
\newcommand*\definerel[1]{\mathrel{:{#1}}}
% by definition (default is =) (DEPRECATED)
\newcommand*\bydef[1][=]{\ensuremath{\stackrel{\text{def}}{#1}}}
% by definition with NECESSARY text:
\newcommand*\bdef[2][=]{\ensuremath{\stackrel{\text{def }{#2}}{#1}}}
\newcommand*\bdefeq[1]{\ensuremath{\xlongequal{\text{def }{#1}}}}

% sets:
\newcommand\N{\ensuremath{\mathbb{N}}}
\newcommand\R{\ensuremath{\mathbb{R}}}
\newcommand\Z{\ensuremath{\mathbb{Z}}}
\renewcommand\O{\ensuremath{\emptyset}}
\newcommand\Q{\ensuremath{\mathbb{Q}}}
\newcommand\Co{\ensuremath{\mathbb{C}}}     % because /C is already taken
\newcommand\RR{\ensuremath{\overline{\mathbb{R}}}}  % extended R
\newcommand\Prime{\ensuremath{\mathbb{P}}}     % prime numbers
\newcommand\On[1][n]{\ensuremath{\mathbb{O}_{#1}}}  % (0, ..., 0)

% Darboux and Riemann
\newcommand\Par{\ensuremath{\mathtt{P}}}
\newcommand\Teq{\ensuremath{\mathtt{T}}}
\newcommand\Ud{\ensuremath{\mathtt{U}}}
\newcommand\Ld{\ensuremath{\mathtt{L}}}
\newcommand\Sr{\ensuremath{\mathtt{S}}}

% a line of dots (13em by default)
\newcommand*\widedots[1][13em]
{
    \tikz\draw[loosely dotted] (0, 0) -- (#1, 0pt);
}

% sequence
\newcommand\seq[3][\infty]{\ensuremath{\{#2\}_{#3=1}^{#1}}}
\newcommand\seqz[3][\infty]{\ensuremath{\{#2\}_{#3=0}^{#1}}}
\newcommand\seqv[4][\infty]{\ensuremath{\{#2\}_{#3={#4}}^{#1}}}

% column
\newcommand\column[3][bmatrix]{\ensuremath{
        \begin{#1}
            {#2} \\
            \vdots \\
            {#3}
        \end{#1}
    }}

% clamp function
\newcommand\clamp[2][]{\ensuremath{\bigg|_{#2}^{#1}}}

% continuous function
\newcommand\Cont[2][]{\ensuremath{\mathcal{C}^{#1} \bigg( {#2} \bigg) }}

% norm in R^n
\newcommand\norm[1]{\ensuremath{\left\|{#1}\right\|}}

%%      decorations
% divisible by
\DeclareRobustCommand{\divby}{%
  \mathrel{\vbox{\baselineskip.65ex\lineskiplimit0pt\hbox{.}\hbox{.}\hbox{.}}}%
}

% not divisible by
\DeclareRobustCommand{\ndivby}{%
  \mathrel{\not\vbox{\baselineskip.65ex\lineskiplimit0pt\hbox{.}\hbox{.}\hbox{.}}}%
}

% controversy
\newcommand\contra{~\scalebox{1.2}{$\lightning$}~}

% roman numbers
\makeatletter
\newcommand*{\rom}[1]{\expandafter\@slowromancap\romannumeral #1@}
\makeatother

% GCD
\newcommand\GCD[1]{\text{НОД} \left( {#1} \right) }

% SCM
\newcommand\SCM[1]{\text{НОК} \left( {#1} \right) }

% triangle inequality
\newcommand\trile{\ensuremath{\stackrel\vartriangle\le}}
\newcommand\trige{\ensuremath{\stackrel\vartriangle\ge}}

% 1 / 2 or [x] / 2
\newcommand\half[1][1]{\ensuremath{\frac{#1}2}}

\newcommand\TODO[1]{\textcolor{red}{{\Huge TODO:} {#1}} \message{LaTeX Warning: TODO on line \the\inputlineno}}

% "not important"
\newcommand\nimp[1][\text{если не оговорено иное}]{\textcolor{gray}{#1}}

%%     \section
% "math string"
\newcommand\mstring[1]{\texorpdfstring{$ {#1} $}{#1}}
% \\
\newcommand\n{\texorpdfstring\\{}}

%% set package shorthands
\let\tset\set
\let\set\Set

\let\tbraket\braket
\let\braket\Braket

%%     Basov
% Basov's "antlers" -- two cases:
\newcommand\commonantlers[2]{\vspace{-2em}
\begin{figure}[h]
    	\centering
        \begin{tikzpicture}
        	\draw[->, name path=antler-1] (0, 0) -- (2, -0.5);
            \path[name path=vert-1] (1, -0.5) -- (1, 0);
            \node[name intersections={of=antler-1 and vert-1}] at (intersection-1)[above]{#2};

            \draw[->, name path=antler-2] (0, 0) -- (-2, -0.5);
            \path[name path=vert-2] (-1, -0.5) -- (-1, 0);
            \node[name intersections={of=antler-2 and vert-2}] at (intersection-1)[above]{#1};
        \end{tikzpicture}
    \end{figure}\vspace{-2em}}
\newcommand\antlers{\commonantlers{}{}}
\newcommand\antlersimp{\commonantlers{}{\textcolor{red}{!!!}}}

%%     theorems
\usepackage{thmtools, thm-restate}
\usepackage[framemethod=TikZ]{mdframed}
\mdfsetup{skipabove=.5em,skipbelow=0em}

\theoremstyle{definition}

\declaretheoremstyle[
    headfont=\bfseries\sffamily\color{ForestGreen!70!black}, bodyfont=\normalfont,
    mdframed={
        linewidth=2pt,
        rightline=false, topline=false, bottomline=false,
        linecolor=ForestGreen, backgroundcolor=ForestGreen!5,
    }
]{thmgreenbox}

\declaretheoremstyle[
    headfont=\bfseries\sffamily\color{NavyBlue!70!black}, bodyfont=\normalfont,
    mdframed={
        linewidth=2pt,
        rightline=false, topline=false, bottomline=false,
        linecolor=NavyBlue, backgroundcolor=NavyBlue!5,
    }
]{thmbluebox}

\declaretheoremstyle[
    headfont=\bfseries\sffamily\color{NavyBlue!70!black}, bodyfont=\normalfont,
    mdframed={
        linewidth=2pt,
        rightline=false, topline=false, bottomline=false,
        linecolor=NavyBlue
    }
]{thmblueline}

\declaretheoremstyle[
    headfont=\bfseries\sffamily\color{RawSienna!70!black}, bodyfont=\normalfont,
    mdframed={
        linewidth=2pt,
        rightline=false, topline=false, bottomline=false,
        linecolor=RawSienna, backgroundcolor=RawSienna!5,
    }
]{thmredbox}

\declaretheoremstyle[
    headfont=\bfseries\sffamily\color{RawSienna!70!black}, bodyfont=\normalfont,
    mdframed={
        linewidth=2pt,
        rightline=false, topline=false, bottomline=false,
        linecolor=RawSienna, backgroundcolor=RawSienna!2.5
    }
]{thmredline}

\declaretheoremstyle[
    headfont=\bfseries\sffamily\color{RawSienna!70!black}, bodyfont=\normalfont,
    numbered=no,
    mdframed={
        linewidth=2pt,
        rightline=false, topline=false, bottomline=false,
        linecolor=RawSienna, backgroundcolor=RawSienna!1,
    },
    qed=\qedsymbol
]{thmproofbox}

\declaretheorem[style=thmgreenbox, name=Определение]{definition}

\declaretheorem[style=thmbluebox, numbered=no, name=Пример]{eg}
\declaretheorem[style=thmbluebox, numbered=no, name=Примеры]{egs}
\declaretheorem[style=thmbluebox, numbered=no, name=Свойство]{property}
\declaretheorem[style=thmbluebox, numbered=no, name=Свойства]{properties}
\declaretheorem[style=thmbluebox, numbered=no, name=Алгоритм]{algorithm}
\declaretheorem[style=thmbluebox, name=Вопрос]{quest}

\declaretheorem[style=thmredbox, name=Теорема]{theorem}
\declaretheorem[style=thmredbox, name=Лемма]{lemma}
\declaretheorem[style=thmredbox, name=Утверждение]{statement}
\declaretheorem[style=thmredbox, name=Аксиома]{axiom}
\declaretheorem[style=thmredbox, numbered=no, name=Утверждения]{statements}
\declaretheorem[style=thmredbox, numbered=no, name=Следствие]{implication}
\declaretheorem[style=thmredbox, numbered=no, name=Вывод]{corollary}

\declaretheorem[style=thmredline, numbered=no, name=Другая формулировка]{restatebase}

\newenvironment{restate}{\vspace{-.5em}\begin{restatebase}}{\end{restatebase}}

\declaretheorem[style=thmblueline, numbered=no, name=Замечание]{remark}
\declaretheorem[style=thmblueline, numbered=no, name=Примечание]{note}

% Basov's control point. It's green, because it's commonly used inside of algorithm, which is blue
\declaretheorem[style=thmgreenbox, numbered=no, name=Контрольная точка]{control}
\declaretheorem[style=thmgreenbox, numbered=no, name=Ответ]{answerbase}

% numbered theorems
\newenvironment{props}[1][ ]{\begin{properties}[{#1}]\hfill\begin{enumerate}}{\end{enumerate}\end{properties}}
\newenvironment{stmts}[1][ ]{\begin{statements}[{#1}]\hfill\begin{enumerate}}{\end{enumerate}\end{statements}}
\newenvironment{exmpls}[1][ ]{\begin{egs}[{#1}]\hfill\begin{enumerate}}{\end{enumerate}\end{egs}}
\newenvironment{algo}[1][ ]{\begin{algorithm}[{#1}]\hfill\begin{enumerate}}{\end{enumerate}\end{algorithm}}

\newenvironment{iproof}[1][ ]{\begin{proof}[{#1}]\hfill\begin{itemize}}{\end{itemize}\end{proof}}

\declaretheorem[style=thmproofbox, name=Доказательство]{replacementproof}
% proof, moved 0.5em up
\renewenvironment{proof}{\vspace{-.5em}\begin{replacementproof}}{\end{replacementproof}}

\newenvironment{answer}{\vspace{-.5em}\begin{answerbase}}{\end{answerbase}}

% no-style theorems
\newtheorem*{notation}{Обозначение}
\newtheorem*{problem}{Задача}
\newtheorem*{intuition}{Очевидно, что}
\newtheorem*{remind}{Напоминание}

%%    ``undefined theorem'' -- theorem-like environment with custom header

\newmdenv[linewidth=2pt, rightline=false, topline=false, bottomline=false, linecolor=black, backgroundcolor=black!2]{undefinedtheorem}

\newenvironment{undefthm}[1]{\begin{undefinedtheorem}\textbf{#1. }}{\end{undefinedtheorem}}
