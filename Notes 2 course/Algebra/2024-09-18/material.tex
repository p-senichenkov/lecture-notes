\section{Жорданова форма}

\subsection{Существование жордановой формы нильпотентного оператора}

\begin{definition}
	Жордановой лкеткой порядка $ r $ с собств. знач. $ 0 $ назыавется квадратная матрица порядка $ r $ вида
	$$
	\begin{pmatrix}
		0 & - & . & 0 \\
		1 & 0 & . & 0 \\
		. & . & . & . \\
		0 & . & 1 & 0
	\end{pmatrix} $$
\end{definition}

\begin{notation}
	$ J_r(0) $
\end{notation}

\begin{definition}
	Жордановой матрицей с собств. знач. $ 0 $ назвыается матрица вида
	$$
	\begin{pmatrix}
		J_{r_1}(0) & 0 & . & 0 \\
		0 & J_{r_2}(0) & . & 0 \\
		. & . & . & . \\
		0 & . & F_{r_k}(0) & 0
	\end{pmatrix} $$
\end{definition}

\begin{eg}
	$$ J = \left(
	\begin{tabular}{c c c | c c}
		0 & 0 & 0 & 0 & 0 \\
		1 & 0 & 0 & 0 & 0 \\
		0 & 1 & 0 & 0 & 0 \\
		\hline
		0 & 0 & 0 & 0 & 0 \\
		0 & 0 & 0 & 1 & 0 \\
	\end{tabular} \right) $$
	Пусть это матрица оператора в базисах $ e_1, e_2, e_3, e_4, e_5 $
	$$ Je_1 = 0e_1 + 1e_2 + 0e_3 + 0e_4 + 0e_5 = e_2 $$
	$$ Je_2 = e_3, \qquad Je_3 = 0, \qquad Je_4 = e_5, \qquad Je_5 = 0 $$
\end{eg}

\begin{notation}
	$ e_1 \to e_2 \to e_3 \to 0, \qquad e_4 \to e_5 \to 0 $
\end{notation}

\begin{definition}
	$ \mathcal{A} $ -- нильпотентный оператор \\
	Жордановой цепочкой называется такой набор векторов $ e_1, e_2, ..., e_r $, что $ \mathcal{A}(e_i) = e_{i + 1} $ при $ i < r $ и $ \mathcal{A}(e_r) = 0 $
\end{definition}

\begin{notation}
	$ e_1 \to e_2 \to ... \to e_r \to 0 $
\end{notation}

\begin{remark}
	Вектор $ e_r $ является собственным вектором, соотв. $ \lambda = 0 $
\end{remark}

\begin{eg}
	$ \mathcal{A} : \R^3 \to \R^3, \qquad X \mapsto AX $
	$$ A =
	\begin{pmatrix}
		1 & 2 & -3 \\
		1 & 2 & -3 \\
		1 & 2 & -3
	\end{pmatrix}, \qquad A^2 = 0 $$
	\begin{itemize}
		\item $ e_1 \to e_2 \to e_3 \to 0 $ не бывает, т. к. $ \mathcal{A}^2(e_1) = 0 $
		\item Построим цепочку $ e_1 \to e_2 \to 0 $ \\
		Найдём $ \ker \mathcal{A} $
		$$
		\begin{pmatrix}
		1 & 2 & -3 \\
		1 & 2 & -3 \\
		1 & 2 & -3
	\end{pmatrix}
	\begin{pmatrix}
		x \\
		y \\
		z
	\end{pmatrix} = 0 \iff x + 2y - 3z = 0 $$
	$$ \dim \ker \mathcal{A} = 2 $$
	Найдём $ e_1 $, такой что $ e_1 \to e_2 \to 0 $: \\
	Любой вектор за 2 шага перейдёт в 0, т. к. $ \mathcal{A}^2 = 0 $ \\
	Найдём $ e_1 $, который за 1 шаг \textbf{не} перейдёт в 0: \\
	Возьмём $ e_1 =
	\begin{pmatrix}
		1 \\
		0 \\
		0
	\end{pmatrix} $
	$$ e_2 = \mathcal{A}(e_1) =
	\begin{pmatrix}
		1 \\
		1 \\
		1
	\end{pmatrix} $$
	$$
	\begin{pmatrix}
		1 \\
		0 \\
		0
	\end{pmatrix} \to
	\begin{pmatrix}
		1 \\
		1 \\
		1
	\end{pmatrix} \to
	\begin{pmatrix}
		0 \\
		0 \\
		0
	\end{pmatrix} $$
	Найдём $ e_1' $, который перейдёт в 0: \\
	Возьмём $ e_1' $, линейно независимый с $ e_2 $ \\
	Подойдёт $ e_1' =
	\begin{pmatrix}
		-2 \\
		1 \\
		0
	\end{pmatrix} $
	$$ \underset{e_1}{
		\begin{pmatrix}
			1 \\
			0 \\
			0
		\end{pmatrix}} \to \underset{e_2}{
		\begin{pmatrix}
			1 \\
			1 \\
			1
		\end{pmatrix}} \to
	\begin{pmatrix}
		0 \\
		0 \\
		0
	\end{pmatrix}, \qquad \underset{e_1'}{
		\begin{pmatrix}
			-2 \\
			1 \\
			0
		\end{pmatrix}} \to
	\begin{pmatrix}
		0 \\
		0 \\
		0
	\end{pmatrix} $$
	$ e_1, e_2, e_1' $ -- жорданов базис \\
	Жорданова форма:
	$$ \left(
	\begin{tabular}{c c | c}
		0 & 0 & 0 \\
		1 & 0 & 0 \\
		\hline
		0 & 0 & 0
	\end{tabular} \right) $$
	\end{itemize}
\end{eg}

\begin{lemma}[ЛНЗ жордановых цепочек]
	Дано несколько жордановых цепочек:
	$$ e_1^{(1)} \to e_2^{(1)} \to ... \to e_{r_1}^{(1)} \to 0 $$
	$$ \widedots $$
	$$ e_1^{(k)} \to e_2^{(k)} \to ... \to e_{r_k}^{(k)} \to 0 $$
	Если последние векторы цепочек, т. е. $ e_{r_1}^{(1)}, ..., e_{r_k}^{(k)} $ ЛНЗ, то объединение цепочек ЛНЗ
\end{lemma}

\begin{proof}
	\textbf{Индукция} по $ r \define \max\set{r_1, ..., r_k} $
	\begin{itemize}
		\item \textbf{База.} $ r = 1 $ \\
		Все цепочки длины 1 \\
		Все векторы -- последние и, по условию, ЛНЗ
		\item \textbf{Переход.} $ r - 1 \to r $
		$$ \mathcal{A}(e_i^{(j)}) =
		\begin{cases}
			e_{i + 1}^{(j)}, \qquad i < r_j \\
			0, \qquad i = r_j
		\end{cases} $$
		Применим $ s $ раз:
		$$ \mathcal{A}^s(e_i^{(j)}) =
		\begin{cases}
			e_{i + s}^{(j)}, \qquad i + s \le r_j \\
			0, \qquad i + s < r_j
		\end{cases} $$
		Цепочки бывают двух видов: у некоторых длина $ r $, а у некоторых -- меньше (по определению $ r $) \\
		НУО считаем, что цепочки с номерами $ 1, 2, ..., m $ имеют длину $ r $, а остальные -- меньше, т. е.
		$$ r_1 = r_2 = ... = r_m = r, \qquad r_i < r \text{ при } i > m $$
		\textbf{От противного:} пусть набор ЛЗ:
		$$ \sum_{j = 1}^k \sum_{i = 1}^{r_j} a_i^{(j)}e_i^{(j)} = 0, \qquad \text{не все } a_i^{(j)} \text{ равны } 0 $$
		Применим к этому равентсву $ \mathcal{A}^{r- 1} $:
		\begin{itemize}
			\item Если цепочка короче $ r $, то она вся перейдёт в 0
			\item Иначе -- останется только поледний вектор
		\end{itemize}
		То есть,
		$$ e_1^{(j)} \to e_r^{(j)}, \qquad a_1^{(j)}e_1^{(j)} \to a_1^{(j)}e_r^{(j)}, \qquad \text{остальные } \to 0 $$
		Получится сумма:
		$$ \sum_{j = 1}^m a_1^{(j)} e_r^{(j)} $$
		Заметим, что это ЛК последних векторов (которые, по условию, ЛНЗ)
		$$ \implies a_1^{(j)} = 0 \quad \text{при } j \le m $$
		Уберём слагаемые $ 0 \cdot e_1^{(j)} $ при $ j \le m $
		$$ \sum_{j \le m} \sum_{i = 2}^r a_i^{(j)} e_i^{(j)} + \sum_{j > m}a_i^{(j)}e_i^{(j)} = 0 $$
		Это -- ЛК векторов из цепочек длины $ r - 1 $ с теми же последними векторами \\
		Применим \textbf{индукционное предположение}. Вместе с условием, что последние векторы ЛНЗ, получаем, что все они ЛНЗ
	\end{itemize}
\end{proof}

\begin{lemma}[базис из жордановых цепочек]
	$ \mathcal{A} $ -- оператор на $ V $, базис $ e_1, e_2, ..., e_n $ -- базис, являющийся объединением жордановых цепочек (в естественном порядке):
	$$ e_1 \to e_2 \to ... \to e_{r_1} \to 0 $$
	$$ e_{r_1 + 1} \to e_{r_2 + 2} \to ... \to e_{r_1 + r_2} \to 0 $$
	$$ \widedots $$
	$$ e_{r_1 + ... + r_{k - 1} + 1} \to ... \to e_{r_1 + r_2 + ... + r_{k - 1} + r_k} \to 0 $$
	Тогда матрица $ \mathcal{A} $ в этом базисе
	$$ A =
	\begin{pmatrix}
		J_{r_1}(0) & 0 & . & 0 \\
		0 & J_{r_2}(0) & . & 0 \\
		. & . & . & . \\
		0 & . & 0 & J_{r_k}(0)
	\end{pmatrix} $$
\end{lemma}

\begin{proof}
	$$ \mathcal{A}(e_{r_1}) = \mathcal{A}(e_{r_1 + r_2}) = ... = \mathcal{A}(e_{r_1 + ... + r_k}) = 0 $$
	Значит, при $ i = r_1, r_1 + r_2, \widedots[5em], r_1 + ... + r_k $, $ \quad i $-й столбец -- нулевой \\
	При $ i \ne r_1, \widedots[5em], r_1 + ... + r_k $, $ \quad \mathcal{A}(e_i) = e_{i + 1} \implies i $-й столбец:
	$$
	\begin{pmatrix}
		0 \\
		0 \\
		\vdots \\
		0 \\
		1 \\
		0 \\
		\vdots \\
		0
	\end{pmatrix}
	\begin{matrix}
		\ \\
		\ \\
		\ \\
		i \\
		i + 1 \\
		\ \\
		\ \\
		\
	\end{matrix} $$
\end{proof}

\begin{theorem}[жорданова форма нильпотентного оператора]
	Для любого нильпотентного оператора на конечномерном векторном пространстве существует форданов базис
\end{theorem}

\begin{proof}
	Будем доказывать, что существует базис из жордановых цепочек \\
	Положим $ W \define \ker \mathcal{A} $ \\
	Если мы возьмём ЛНЗ векторы из ядра и достроим (слева от них) цепочки, то получим жорданов базис \\
	Положим $ U_i \define \Img \mathcal{A}^i $
	$$ V = U_0 \supset U_1 \supset U_2 \supset ... \supset U_{k - 1} \supset U_k = \set{0} $$
	где $ k $ -- степень нильпотентности \\
	Заметим, что если $ v \in U_t \cap W $, то существует цепочка длины $ t + 1 $ с концом $ v $ \\
	Построим базис $ W $ (такой, чтобы можно было достроть цепочки): \\
	Будем пересекать $ W $ с $ U_i $ \\
	Выберем базис $ W \cap U_{k - 1} $. Он ЛНЗ, значит его можно дополнить до базиса $ W \cap U_{k - 2} $ \\
	В итоге получим базис $ W \cap U_0 = W $ \\
	Получили базис $ e_1, e_2, ... $ пространства $ W $ \\
	Для $ e_i \in W \cap U_t $ построим цепочку длины $ t + 1 $ с концом $ e_i $:
	$$ e_1^{(i)} \to e_2^{(i)} \to ... \to e_{t + 1}^{(i)} = e_i \to 0 $$
	Объединение цепочек -- ЛНЗ (по лемме) \\
	Докажем, что это базис, т. е. что набор порождающий: \\
	Докажем, что если $ \mathcal{A}^s(v) = 0 $, то $ v $ является ЛК векторов цепочек \\
	Докажем \textbf{индукцией} по $ s $:
	\begin{itemize}
		\item \textbf{База.} $ s = 1 $
		$$ \mathcal{A}^1(v) = 0, \qquad v \in W, \qquad e_1, e_2, ... \text{ -- базис } W $$
		\item \textbf{Переход.} $ s \to s + 1 $ \\
		Пусть $ \mathcal{A}^{s + 1}(v) = 0, \quad \mathcal{A}^s(v) \ne 0 $ \\
		Положим $ u = \mathcal{A}^s \implies u \in U_s $
		$$ \underbrace{\overbrace{v \to \cdot \to ... \to u}^s \to 0}_{s + 1} $$
		Значит, $ \mathcal{A}(u) = 0 \implies u \in W $ \\
		Значит, $ u \in U_s \cap W $ \\
		Представим его в виде ЛК базиса $ U_s \cap W $ (того, до которого мы дошли на каком-то очередном шагу дополнения базисов):
		$$ u = \sum_i a_ie_i $$
		$$ \forall e_i \text{ из этого бзиса выбрана цепочка длины хотя бы } s + 1 $$
		$$ e_i = e_{s + t_i}^{(i)} \text{ -- последний вектор цепочки} $$
		Пусть $ e_i' $ -- вектор цепочки, такой что $ \mathcal{A}^s(e_i') = e_i $ (вектор, который на $ s $ шагов раньше)
		$$ \mathcal{A}^s \bigg( \sum a_ie_i' \bigg) = \sum a_ie_i = u $$
		При этом, $ \mathcal{A}^s(v) \bydef u $ \\
		Получили 2 линейных представления $ u $, значит,
		$$ \mathcal{A}^s(v) = \mathcal{A}^s \bigg( \sum a_ie_i' \bigg) \implies \mathcal{A}^s \bigg( v - \sum a_ie_i' \bigg) = 0 $$
		Тогда, \textbf{по индукционному предположению}, $ v - \sum a_ie_i' $ представляется в виде ЛК векторов из цепочек \\
		Значит, $ v $ представляется в виде ЛК векторов цепочек
	\end{itemize}
\end{proof}

\section{Многочлены от оператора}

\begin{notation}
	$ V $ -- векторное пространство над $ K $, $ \quad \mathcal{A} $ -- оператор на $ V $, $ \quad P \in K[x] $
	$$ P(x) = a_nx^n + ... + a_1x + a_0 $$
	Тогда $ P(\mathcal{A}) = a_n\mathcal{A}^n + ... + a_1\mathcal{A} + a_0\mathcal{E} $, т. е. такой опрератор $ \mathcal{B} $, что
	$$ \mathcal{B}(v) = a_n \mathcal{A}^n(v) + ... + a_2\mathcal{A}^2(v) + a_1\mathcal{A}(v) + a_0v $$
\end{notation}

\begin{lemma}[произведение многочленов от оператора]
	$ P, Q $ -- многочлены, $ \mathcal{A} $ -- оператор
	$$ \implies (PQ)(\mathcal{A}) = P(\mathcal{A}) \circ Q(\mathcal{A}) $$
\end{lemma}

\begin{proof}
	Пусть $ P(t) = \sum p_it^i, \quad Q(t) = \sum q_it^i, \quad R(t) = P(t)Q(t) $
	$$ R(t) = \sum p_iq_jt^{i + j} $$
	Положим $ \mathcal{B} = P(\mathcal{A}), \quad \mathcal{C} = Q(\mathcal{A}), \quad \mathcal{D} = R(\mathcal{A}) $ \\
	Нужно доказать, что $ \mathcal{B} \bigg( \mathcal{C}(v) \bigg) = \mathcal{D}(v) \quad \forall v $ \\
	Пусть $ w = \mathcal{C}(v) $
	$$ \implies \mathcal{B}(w) = \sum p_i \mathcal{A}^i(w), \qquad \mathcal{C}(v) = \sum q_j\mathcal{A}^j(v), \qquad \mathcal{D}(v) = \sum p_iq_j \mathcal{A}^{i + j}(v) $$
	\begin{multline*}
		\mathcal{B} \bigg( \mathcal{C}(v) \bigg) = \mathcal{B}(w) = \mathcal{B} \bigg( \sum p_j \mathcal{A}^j(v) \bigg) = \sum q_j \mathcal{B} \bigg( \mathcal{A}^j(v) \bigg) = \sum q_j \bigg( \sum p_i \mathcal{A}^i \big( \mathcal{a^j}(v) \big) \bigg) = \\
		= \sum q_j \bigg( \sum p_i \mathcal{A}^{i _ j} \bigg) = \sum q_jp_i \mathcal{A}^{i + j} = \mathcal{D}(v)
	\end{multline*}
\end{proof}

\begin{implication}
	$ P, Q $ -- многочлены, $ \quad \mathcal{A}, \mathcal{B}, \mathcal{C} $ -- операторы, $ \qquad \mathcal{B} = P(\mathcal{A}), \quad \mathcal{C} = Q(\mathcal{A}) $
	$$ \implies \mathcal{B} \circ \mathcal{C} = \mathcal{C} \circ \mathcal{B} $$
\end{implication}

\begin{proof}
	$ PQ = QP \implies (PQ)(A) = (QP)(A) $
\end{proof}
