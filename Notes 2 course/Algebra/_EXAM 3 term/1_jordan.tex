\begin{notation}
	$ a_1, .., a_n \nin \bigodot \iff $ \textbf{не} все они равны нулю
\end{notation}

\part{Жорданова форма оператора}

\section{Теорема о сумме собственных подпространств и следствие о линейно независимых векторах}

\begin{definition}
	$ V $ "--- векторное пространство, $ \msc{A} $ "--- оператор на $ V $, $ \lambda $ "--- с. ч. \\
	\it{Собственным подпространством}, соответствующим $ \lambda $, называется множество с. в., соответств. $ \lambda $.
\end{definition}

\begin{notation}
	$ V_\lambda $
\end{notation}

\begin{definition}
	$ U $ "--- подпространство $ V $. \\
	$ U $ называется \it{инвариантным} относительно $ \msc{A} $, если
	$$ \forall x \in U \quad \msc{A}x \in U $$
\end{definition}

\begin{statement}
	$ V_\lambda $ "--- инвариантное подпространство.
\end{statement}

\begin{iproof}
	\item Подпространство
	\begin{itemize}
		\item $ u, v \in V_\lambda \implies
		\begin{Bmatrix}
			\msc{A}u = \lambda u \\
			\msc{A}v = \lambda v
		\end{Bmatrix} \implies \msc{A}(u + v) \undereq{\text{линейность}} \msc{A}u + \msc{A}v = \lambda u + \lambda v = \lambda (u + v) \implies u + v \in V_\lambda $
		\item $ u \in V_\lambda, k \in K \implies \msc{A}(ku) \undereq{\text{линейность}} k \msc{A}(u) = k \lambda u = \lambda(ku) \implies ku \in V_\lambda $
	\end{itemize}
	\item Инвариантность
	$$ u \in V_\lambda \implies \msc{A}u = \lambda u \in V_\lambda $$
\end{iproof}

\begin{theorem}[о сумме собственных подпространств]
	$ \lambda_1, \dots, \lambda_k $ "--- \bt{различные} собственные числа \\
	Тогда сумма $ \bm{V_{\lambda_1}} + \dots + \bm{V_{\lambda_k}} $ является \bt{прямой}.
\end{theorem}

\begin{proof}
	\textbf{Индукция} по $ k $
	\begin{itemize}
		\item \textbf{База.} $ k = 1 $. Сумма из одного слагаемого "--- прямая.
		\item \textbf{Переход}. $ k - 1 \to k $ \\
		Пусть $ u_1 + \dots + u_{k - 1} + u_k = 0, \qquad u_i \in V_{\lambda_i} $
		\begin{multline*}
			0 = \msc{A}(\underbrace{u_1 + \dots + u_{k - 1} + u_k}_{= 0 }) - \lambda_k(\underbrace{u_1 + \dots + u_{k - 1} + u_k}_{= 0 }) = \\
			= \lambda_1u_1 + \dots + \lambda_{k - 1}u_{k - 1} + \cancel{\lambda_ku_k} - \lambda_ku_1 - \dots - \lambda_ku_{k - 1} - \cancel{\lambda_ku_k} = \underbrace{(\lambda_1 - \lambda_k)}_{\ne 0}u_1 + \dots + \underbrace{(\lambda_{k - 1} - \lambda_k)}_{\ne 0}u_{k - 1}
		\end{multline*}
		(т. к. по условию собственные числа различны)
		$$ (\lambda_1 - \lambda_k)u_1 \in V_{\lambda_1}, \quad \dots, \quad (\lambda_{k - 1} - \lambda_k)u_{k - 1} \in V_{\lambda_{k - 1}} $$
		По \textbf{индукционному предположению}, $ V_{\lambda_1} \oplus \dots \oplus V_{\lambda_{k - 1}} $ \\
		А мы представили 0 в виде суммы. Значит, все слагаемые нулевые:
		$$ (\lambda_1 - \lambda_k)u_1 = \dots = (\lambda_{k - 1} - \lambda_k)u_{k - 1} = 0 \implies u_1 = \dots = u_{k - 1} = 0 \implies u_k = 0 $$
	\end{itemize}
\end{proof}

\begin{implication}
	$ \lambda_1, \dots, \lambda_k $ "--- \bt{различные} с. ч., $ \qquad u_i \in V_{\lambda_i}, \qquad u_i \ne 0 $

	Тогда $ u_1, \dots, u_k $ \bt{ЛНЗ}.
\end{implication}

\begin{proof}
	Пусть $ a_1u_1 + \dots + a_ku_k = 0 $
	$$ a_1u_1 \in V_{\lambda_1}, \dots, a_ku_k \in V_{\lambda_k} \implies a_1u_1 = \dots = a_ku_k = 0 \implies a_1 = \dots = a_k = 0 $$
\end{proof}

\section(Критерий диагонализуемости в терминах геометрических кратностей){Критерий диагонализуемости в терминах геометрических \\кратностей}

\begin{definition}
	Оператор $ \msc{A} $, действующий на $ V $ называется \it{диагонализуемым}, если его матрица в некотором базисе диагональна.
\end{definition}

\begin{definition}
	$ \msc{A} $ "--- оператор, $ \lambda $ "--- с. ч.
	\begin{itemize}
		\item \it{Геометрической кратностью} $ \lambda $ называется $ \dim V_\lambda $;
		\item \it{Арифметической кратностью} $ \lambda $ называется кратность $ \lambda $ как корня $ \chi_{\msc{A}}(t) $.
	\end{itemize}
\end{definition}

\begin{theorem}[критерий диагонализуемости в терминах геометрической кратности]
	\hfill \\
	(\rom1) $ \msc{A} $ диагонализуем $ \iff $ (\rom2) \bt{сумма геометрических кратностей} всех с.~ч. равна $ \bm{\dim V} $
\end{theorem}

\begin{proof}
	\hfill \\
	$ \msc{A} $ диагонализуем $ \iff $ в нек. базисе $ e_1, \dots, e_n $ матрица $ \msc{A} $ имеет вид $ A = \diagmatrix{a_1}{a_n} $ \\
	$ \iff $ для некоторого базиса $ e_1, \dots, e_n $ выполнено
	$$ \msc{A}e_i = 0 e_1 + \dots + a_ie_i + \dots + 0 e_n = a_ie_i $$
	$ \iff $ ($ \rom1' $) существует базис из с. в. \\
	Докажем, что ($ \rom1' $) $ \iff $ (\rom2): \\
	Пусть $ U = V_{\lambda_1} + \dots + V_{\lambda_k} $
	$$ n \define \dim V, \qquad d_i \define \dim V_{\lambda_i} $$
	\begin{itemize}
		\item (\rom2) $ \implies $ ($ \rom1' $) \\
		Имеем $ d_1 + \dots + d_k = n $
		$$ V_{\lambda_1} \oplus \dots \oplus V_{\lambda_k} \implies \dim U = n \underimp{U \text{ "--- подпр-во } V} U = V $$
		$$ V_{\lambda_1} \oplus \dots \oplus V_{\lambda_k} \implies \text{ объединение базисов } V_\lambda \text{ является базисом } U = V $$
		Эти базисы состоят из с. в. Значит, объединение базисов состоит из с. в. Это и есть базис $ V $.

		\item ($ \rom1' $) $ \implies $ (\rom2) \\
		Существует базис $ V $ из с. в. \\
		Они распределяются по $ V_\lambda $ (но не обязательно для каждого $ V_\lambda $ представлен весь его базис):
		$$ \underbrace{e_1^{(1)}, \dots, e_{t_1}^{(1)}}_{
			\begin{subarray}{c}
				\text{соотв. } \lambda_1 \\
				\in V_{\lambda_1}
			\end{subarray}}, \underbrace{e_1^{(2)}, \dots, e_{t_2}^{(2)}}_{
			\begin{subarray}{c}
				\text{соотв. } \lambda_2 \\
				\in V_{\lambda_2}
			\end{subarray}}, \widedots $$
		$$ \underset{\text{(\as они лежат в большом базисе)}}{e_1^{(i)}, \dots, e_{t_i}^{(i)} \text{ ЛНЗ }} \implies t_i \le d_i \quad \forall i $$
		Сложим все эти неравенства:
		$$
		\begin{rcases}
			d_1 + d_2 + \dots + d_k \ge t_1 + \dots + t_k = n \\
			n \ge \dim U \undereq{U = V_{\lambda_1} \oplus \dots \oplus V_{\lambda_k}} d_1 + \dots + d_k
		\end{rcases} \implies n = d_1 + \dots + d_k $$
	\end{itemize}
\end{proof}

\begin{implication}[достаточное условие диагонализуемости]
	Пусть $ \dim V = n $ \\
	Если у $ \msc{A} $ есть $ \bm n $ \bt{различных} с.~ч., то $ \msc{A} $ диагонализуем
\end{implication}

\begin{proof}
	$ \dim V_{\lambda_i} \ge 1 $
	$$ n \ge \dim(V_{\lambda_1} + \dots + V_{\lambda_k}) \undereq{\oplus} \dim V_{\lambda_1} + \dots + \dim V_{\lambda_k} \ge n $$
	Значит, достигается равенство
\end{proof}

\section{Теорема об арифметической и геометрической кратности.\n Следствие о диагоналиуемом операторе}

\begin{remind}[определитель ступенчатой матрицы]
	$$ M =
	\begin{pmatrix}
		A & B \\
		0 & C
	\end{pmatrix}, \quad A, C \text{ "--- кв. } \implies |M| = |A| \cdot |C| $$
\end{remind}

\begin{theorem}[арифм. и геом. кратности]
	$ \lambda $ "--- с. ч. $ \msc{A} $ \\
	\bt{Геом.} кратность $ \lambda \bm\le $ \bt{арифм.} кратности $ \lambda $
\end{theorem}

\begin{proof}
	Пусть $ n = \dim V, \quad k $ "--- геом. кр. $ \lambda $ \\
	Выберем базис $ e_1, \dots, e_k $ пространства $ V_{\lambda} $ \\
	Дополним его до базиса $ V $: $ e_1, \dots, e_k, \dots., e_n $ \\
	При $ i \le k $ выполнено $ \msc{A} e_i = \lambda e_i = 0 \cdot e_1 + \dots + \lambda e_i + \dots + 0 \cdot e_n $ \\
	Матрица $ \msc{A} $ в базисе $ e_1, \dots, e_n $:
	$$ A =
	\begin{pNiceMatrix}[xdots/line-style=loosely dotted]
		\lambda & \Block[r]{2-2}<\huge>{0} & & 		\Block{3-3}<\Large>{B} & & \\
		\Block[l, b]{2-2}<\huge>{0} & \Ddots & 		& & & \\
		& & \lambda 								& & & \\

		\Block{3-3}<\Large>{0} & & &				\Block{3-3}<\Large>{C} & & \\
		& & &										& & & \\
		& & &										& & &
	\end{pNiceMatrix} $$
	Для некоторых $ B_{k \times n - k} $, $ C_{n - k \times n - k} $
	$$ \chi(t) =
	\begin{vmatrix}
		(\lambda - t)E_k & B \\
		0 & C - tE_{n - k}
	\end{vmatrix} = \det \bigg( (\lambda - t)E_k \bigg) \cdot \det(C - tE_{n - k}) = (\lambda - t)^k \cdot \det(C - tE_{k - n}) $$
\end{proof}

\begin{implication}[критерий диагонализуемости в терминах арифметических и геометрических кратностей]
	Оператор $ \msc{A} $ \bt{диагонализуем} $ \iff $
	\begin{enumerate}
		\item $ \chi_{\msc{A}}(t) $ \bt{раскладывается} на линейные множители
		\item $ \forall $ с. ч. $ \lambda \quad $ \bt{арифм.} кр. $ \bm= $ \bt{геом.} кр.
	\end{enumerate}
\end{implication}

\begin{proof}
	Пусть $ \lambda_i $ "--- с. ч., $ d_i $ "--- геом. кр., $ a_i $ "--- арифм. кр., $ n = \dim C $
	$$ \chi(t) = (t - \lambda_1)^{a_1} \dots(t - \lambda_k)^{a_k} \cdot P(t), $$
	где $ P(t) $ не имеет корней.
	$$ n = \deg \chi(t) \ge a_1 + \dots + a_k \ge d_1 + \dots + d_k $$
	Диагонал. $ \iff n = d_1 + \dots + d_k \iff $ везде достигаются равенства
\end{proof}

\section{Блочные матрицы и инвариантные подпространства. Делители характеристичсекого многочлена}

\begin{definition}
	\it{Блочной матрицей} называется матрица вида
	$$ A =
	\begin{pmatrix}
		A_{11} & A_{12} & \dots & A_{1n} \\
		. & . & . & . \\
		A_{m1} & A_{m2} & \dots & A_{mn}
	\end{pmatrix} $$
	где $ \forall i \quad A_{ix} $ имеют поровну строк и $ \forall j \quad A_{xj} $ имеют поровну столбцов
\end{definition}

\begin{definition}
	\it{Блочно-диагональной матрицей} называется матрица вида
	$$ \diagmatrix{A_1}{A_n} $$
	где $ A_i $ "--- квадратные
\end{definition}

\begin{definition}
	$ U $ "--- инвариантное подпространство оператора $ \msc{A} $ \\
	Через $ \msc{A}\clamp{U} $ обозначим \it{сужение} $ \msc{A} $ на $ U $, т. е.
	$$ \msc{A}\clamp{U} : U \to U, \qquad \msc{A}\clamp{U}(x) = \msc{A}(x) \quad \forall x \in U $$
\end{definition}

\begin{theorem}[блочные матрицы и инвариантные подпространства]
    \hfill \\
	$ \msc{A} $ "--- оператор на конечномерном пространстве $ V $
	\begin{enumerate}
		\item $ U $ "--- инвариантное пространство $ \msc{A}, \qquad e_1, \dots, e_s $ "--- базис $ U, \qquad e_1, \dots, e_s, \dots, e_n $ "--- базис $ V $ \\
		$ A_U, A $ "--- матрицы $ \msc A \clamp U $ и $ \msc{A} $ на $ U, V $ в этих базисах
		$$ \implies A =
		\begin{pmatrix}
			\bm{A_U} & B \\
			\bm0 & C
		\end{pmatrix} \quad \text{для некоторых } B, C $$

		\item $ V = U_1 \oplus \dots \oplus U_k $, где $ U_i $ "--- инвар. для $ \msc{A} $ \\
		$ A_1, \dots, A_k $ "--- матрицы $ \msc{A} $ на $ U_1, \dots, U_k $ в некоторых базисах \\
		$ A $ "--- матрица $ \msc{A} $ на $ V $ в базисе, являющемся объединением базисов $ U_i $ (в естественном порядке: базис $ U_1 $, базис $ U_2 $, \dots)
		$$ \implies A = \diagmatrix{A_1}{A_k} $$
		Так как $ A_1, \dots, A_k $ "--- квадратные, то $ A $ "--- блочно-диагональная
	\end{enumerate}
\end{theorem}

\begin{eproof}
	\item Пусть
	$$ A_U =
	\begin{pNiceMatrix}[xdots/line-style=loosely dotted]
		a_{11} & \Cdots & a_{1s} \\
		\Vdots & & \Vdots \\
		a_{s1} & \Cdots & a_{ss}
	\end{pNiceMatrix} $$
	Возьмём $ 1 \le i \le s $ \\
	Посмотрим, как $ \msc{A} $ действует на $ e_i $:
	$$ \msc{A}(e_i) = a_{1i}e_1 + \dots + a_{si}e_s = a_{1i}e_1 + \dots + a_{si}e_s + \dots + 0 \cdot e_{s + 1} + \dots + 0 \cdot e_n $$
	Получили разложение $ \msc{A}(e_i) $ по базису $ V $, то есть, столбец матрицы оператора в базисе \\
	$ e_1, \dots, e_s, \dots, e_n $:
	$$
	\begin{pmatrix}
		a_{1i} \\
		\vdots \\
		e_{si} \\
		0 \\
		\vdots \\
		0
	\end{pmatrix} \text{ "--- } i \text{-й столбец } A $$
	$$ \implies A =
	\begin{pNiceMatrix}
		a_{11} & 	\Cdots & 	a_{s1} & * \\
		\Vdots &	& 			\Vdots & \Vdots \\
		a_{s1} & 	\Cdots & 	a_{ss} & \\
		0 & 		& 			& 		 \\
		& 			\Ddots & 	& 		 \\
		& 			& 			0 & 	 *
	\end{pNiceMatrix} $$

	\item Пусть $ \dim U_1 = d_1, \quad \dim U_2 = d_2, \quad \dots $ \\
	Рассмотрим столбец матрицы $ A $ с номером $ d_1 + d_2 + d_{i - 1} + t $, где $ 1 \le t \le d_i $ (т. е. $ t $-й столбец $ i $-го набора) \\
	Обозначим элементы базисов:
	$$ U_1 : e_1^{(1)}, \dots, e_{d_1}^{(1)} $$
	$$ U_2 : e_2^{(2)}, \dots, e_{d_2}^{(2)} $$
	$$ \widedots[6em] $$
	В этом столбце записаны координаты вектора $ e_t^{(i)} $ в базисе $ V $ \\
	Разложим его по базису подпространства $ U_i $:
	$$ e_t^{(i)} = a_1e_1^{(i)} + \dots + a_{d_i}e_{d_i}^{(i)} $$
	Дополним нулями:
	$$ \underbrace{0 \cdot e_1^{(1)} + \dots + 0 \cdot d_1^{(1)}}_{d_1} + \underbrace{0 \cdot e_1^{(2)} + \cdots}_{d_2} + \dots + \underbrace{a_1e_1^{(i)} + \dots + a_{d_i}e_d^{(i)}}_{d_i} + 0 \cdot e_1^{(i + 1)} + \dots $$
	Получили разложение $ e_r^{(i)} $ по базису $ V $. \\
	$ (d_1 + d_2 + \dots + d_{i - 1} + t) $-й столбец равен
	$$
	\begin{pmatrix}
		0 & \cdots & 0 & a_1 & \cdots & a_{d_i} & 0 & \cdots & 0
	\end{pmatrix}^T $$
\end{eproof}

\begin{implication}[делители характеристического многочлена]
    \hfill \\
	$ \msc{A} $ "--- оператор на конечномерном пространстве $ V, \qquad \chi(t) $ "--- его характ. многочлен
	\begin{enumerate}
		\item $ U $ "--- инвариантное подпространство, $ \quad \chi_U(t) $ "--- характ. многочлен $ \msc{A}\clamp{U} $
		$$ \implies \chi(t) \divby \chi_U(t) $$
		\item $ V = U_1 \oplus \dots \oplus U_k $, где $ U_i $ "--- инвариатные \\
		$ \chi_i(t) $ "--- характ. многочлен $ \msc{A}\clamp{U_i} $
		$$ \chi(t) = \chi_1(t) \cdots \chi_k(t) $$
	\end{enumerate}
\end{implication}

\begin{proof}
	Рассмотрим базисы как в теореме
	\begin{enumerate}
		\item
		\begin{multline*}
			\chi_{\msc{A}}(t) = \chi_A(t) = \left|
			\begin{pmatrix}
				A_U & B \\
				0 & C
			\end{pmatrix} - t E_n \right| =
			\begin{vmatrix}
				A_U - t E_s & B \\
				0 & C - t E_{n - s}
			\end{vmatrix} = \\
			= |A_U - t E_s | \cdot | C - t E_{n -s} | = \chi_{A_U}(t) \cdot \chi_C(t) = \chi_U(t) \cdot \chi_C(t)
		\end{multline*}
		\item $ \chi_A(t) = |A - tE| =
		\begin{vmatrix}
			A_1 - tE & 0 & . & 0 \\
			0 & A_2 - tE & . & 0 \\
			. & . & . & . \\
			0 & . & 0 & A_k - tE
		\end{vmatrix} = |A_1 - tE| \cdot |A_2 - tE| \cdots = \chi_1(t) \cdot \chi_2(t) \cdots $
	\end{enumerate}
\end{proof}

\section{Ранг блочно-диагональной матрицы}

\begin{lemma}[ранг блочно-диагональной матрицы]
	$ A $ "--- блочно-диагональная
	$$ A = \diagmatrix{A_1}{A_k} $$
	$$ \implies \rk A = \rk A_1 + \dots + \rk A_k $$
\end{lemma}

\begin{proof}
	Воспользуемся тем, что ранг "--- это количество ЛНЗ строк \\
	Пусть для каждой матрицы $ A_i $ выбран набор строк $ s_1^{(i)}, s_2^{(i)}, \dots, s_n^{(i)} $ \\
	Для строки $ s_j^{(i)} $ обозначим через $ \vawe{s_j}^{(i)} $ соответствующую строку матрицы $ A $ \\
	Достаточно доказать, что
	$$ \text{набор } \vawe{s_1}^{(1)}, \dots, \vawe{s_{r_1}}^{(1)}, \vawe{s_1}^{(2)}, \dots, \vawe{s_{r_2}}^{(2)}, \dots \text{ ЛНЗ } \iff \text{ все наборы }
	\begin{cases}
		s_1^{(1)}, \dots, s_{r_1}^{(1)} \\
		s_1^{(2)}, \dots, s_{r_2}^{(2)} \\
		\widedots[4em]
	\end{cases} \text{ ЛНЗ} $$
	\begin{itemize}
		\item $ \implies $ \\
		Докажем \textbf{от противного}: \\
		Предположим, что $ s_1^{(i)}, \dots, s_{r_i}^{(i)} $ ЛЗ \\
		То есть, $ \exist a_1, \dots, a_{r_i} \nin \bigodot $, такие, что $ a_1s_1^{(i)} + \dots + a_{r_i}s_{r_i}^{(i)} = 0 $ \\
		Дополним нулями:
		$$ a_1\vawe{s_1}^{(i)} + \dots + a_{r_i}\vawe{s_{r_i}}^{(i)} = 0 $$
		То есть, $ \vawe{s_1}^{(i)}, \dots, \vawe{s_{r_i}}^{(i)} $ ЛЗ \\
		А значит, и весь набор ЛЗ "--- \contra
		\item $ \impliedby $ \\
		Докажем \textbf{от противного}: \\
		Пусть все наборы $ s_1^{(i)}, \dots, s_{r_i}^{(i)} $ ЛНЗ, а $ \vawe{s_1}^{(1)}, \dots, s_{r_i}^{(1)}, \widedots[4em], \vawe{s_{r_k}}^{(k)} $ ЛЗ, то есть
		$$ \sum_{i, j} a_j^{(i)}\vawe{s_j}^{(i)} = 0, \qquad \text{не все } a_j^{(i)} \text{ равны нулю} $$
		Положим
		$$ T_i \define a_1^{(i)}s_1^{(i)} + \dots + a_{r_i}^{(i)}s_{r_i}^{(i)} $$
		$$ \vawe{T_i} \define a_1^{(i)}\vawe{s_1}^{(i)} + \dots + a_{r_i}^{(i)}\vawe{s_{r_i}}^{(i)} $$
		$$ \vawe{T_1} + \vawe{T_2} + \dots + \vawe{T_k} = 0 \implies \vawe{T_1} = \vawe{T_2} = \dots = 0 $$
		Строки $ \vawe{T_1}, \dots, \vawe{T_k} $ \textbf{не} содержат ненулевые элементы в одном столбце (\ie в нашей записи нет полностью нулевых столбцов)
		$$ \implies T_1 = 0, \quad T_2 = 0, \quad T_k = 0 $$
		$$ \implies \forall i \quad a_1^{(i)}s_1^{(i)} + \dots + a_{r_i}s_{r_i}^{(i)} = 0 $$
		$$ s_1^{(i)}, \dots, s_{r_i}^{(i)} \text{ ЛНЗ } \implies a_1^{(i)} = \dots = a_{r_i}^{(i)} = 0 $$
	\end{itemize}
\end{proof}

\begin{implication}
	$ V = U_1 \oplus \dots \oplus U_k, \qquad U_i $ "--- инвариантно для $ \msc{A} $
	$$ \implies \dim \Img \msc{A} = \dim \Img \msc{A}\clamp{U_1} + \dots + \dim \Img \msc{A}\clamp{U_k} $$
\end{implication}

\section{Жордановы цепочки: линейная независимость, матрица оператора в базисе из цепочек}

\begin{definition}
	Жордановой клеткой порядка $ r $ с собств. знач. $ 0 $ называется квадратная матрица порядка $ r $ вида
	$$ J_r(0) =
	\begin{pNiceMatrix}
		0 & \Cdots & \Cdots & 0 \\
		1 & \Ddots & & \Vdots \\
		\Block[l, b]{2-2}<\Large>{0} & \Ddots & \Ddots & \Vdots \\
		& & 1 & 0
	\end{pNiceMatrix} $$
\end{definition}

\begin{definition}
	Жордановой матрицей с собств. знач. $ 0 $ называется матрица вида
	$$ \diagmatrix{J_{r_1}(0)}{J_{r_k}(0)} $$
\end{definition}

\begin{definition}
	$ \msc{A} $ "--- нильпотентный оператор \\
	Жордановой цепочкой называется такой набор векторов $ e_1, e_2, \dots, e_r $, что $ \msc{A}(e_i) = e_{i + 1} $ при $ i < r $ и $ \msc{A}(e_r) = 0 $
\end{definition}

\begin{notation}
	$ e_1 \to e_2 \to \dots \to e_r \to 0 $
\end{notation}

\begin{lemma}[ЛНЗ жордановых цепочек]
	Дано несколько жордановых цепочек:
	$$ e_1^{(1)} \to e_2^{(1)} \to \dots \to e_{r_1}^{(1)} \to 0 $$
	$$ \widedots $$
	$$ e_1^{(k)} \to e_2^{(k)} \to \dots \to e_{r_k}^{(k)} \to 0 $$
	Если последние векторы цепочек, т. е. $ e_{r_1}^{(1)}, \dots, e_{r_k}^{(k)} $ ЛНЗ, то объединение цепочек ЛНЗ
\end{lemma}

\begin{proof}
	\textbf{Индукция} по $ r \define \max\set{r_1, \dots, r_k} $
	\begin{itemize}
		\item \textbf{База.} $ r = 1 $ \\
		Все цепочки длины 1 \\
		Все векторы "--- последние и, по условию, ЛНЗ
		\item \textbf{Переход.} $ r - 1 \to r $
		$$ \msc{A}(e_i^{(j)}) =
		\begin{cases}
			e_{i + 1}^{(j)}, \qquad i < r_j \\
			0, \qquad i = r_j
		\end{cases} $$
		Применим $ s $ раз:
		$$ \msc{A}^s(e_i^{(j)}) =
		\begin{cases}
			e_{i + s}^{(j)}, \qquad i + s \le r_j \\
			0, \qquad i + s < r_j
		\end{cases} $$
		Цепочки бывают двух видов: у некоторых длина $ r $, а у некоторых "--- меньше (по определению $ r $) \\
		НУО считаем, что цепочки с номерами $ 1, 2, \dots, m $ имеют длину $ r $, а остальные "--- меньше, т. е.
		$$ r_1 = r_2 = \dots = r_m = r, \qquad r_i < r \text{ при } i > m $$
		\textbf{От противного:} пусть набор ЛЗ:
		$$ \sum_{j = 1}^k \sum_{i = 1}^{r_j} a_i^{(j)}e_i^{(j)} = 0, \qquad \text{не все } a_i^{(j)} \text{ равны } 0 $$
		Применим к этому равентсву $ \msc{A}^{r- 1} $:
		\begin{itemize}
			\item Если цепочка короче $ r $, то она вся перейдёт в 0
			\item Иначе "--- останется только поледний вектор
		\end{itemize}
		То есть,
		$$ e_1^{(j)} \to e_r^{(j)}, \qquad a_1^{(j)}e_1^{(j)} \to a_1^{(j)}e_r^{(j)}, \qquad \text{остальные } \to 0 $$
		Получится сумма:
		$$ \sum_{j = 1}^m a_1^{(j)} e_r^{(j)} $$
		Заметим, что это ЛК последних векторов (которые, по условию, ЛНЗ)
		$$ \implies a_1^{(j)} = 0 \quad \text{при } j \le m $$
		Уберём слагаемые $ 0 \cdot e_1^{(j)} $ при $ j \le m $
		$$ \sum_{j \le m} \sum_{i = 2}^r a_i^{(j)} e_i^{(j)} + \sum_{j > m}a_i^{(j)}e_i^{(j)} = 0 $$
		Это "--- ЛК векторов из цепочек длины $ r - 1 $ с теми же последними векторами \\
		Применим \textbf{индукционное предположение}. Вместе с условием, что последние векторы ЛНЗ, получаем, что все они ЛНЗ
	\end{itemize}
\end{proof}

\begin{lemma}[базис из жордановых цепочек]
	$ \msc{A} $ "--- оператор на $ V $. \\
    $ e_1, e_2, \dots, e_n $ "--- базис, являющийся объединением жордановых цепочек (в естественном порядке):
	$$ e_1 \to e_2 \to \dots \to e_{r_1} \to 0 $$
	$$ e_{r_1 + 1} \to e_{r_2 + 2} \to \dots \to e_{r_1 + r_2} \to 0 $$
	$$ \widedots $$
	$$ e_{r_1 + \dots + r_{k - 1} + 1} \to \dots \to e_{r_1 + r_2 + \dots + r_{k - 1} + r_k} \to 0 $$
	Тогда матрица $ \msc{A} $ в этом базисе
	$$ A = \diagmatrix{J_{r_1}(0)}{J_{r_k}(0)} $$
\end{lemma}

\begin{proof}
	$$ \msc{A}(e_{r_1}) = \msc{A}(e_{r_1 + r_2}) = \dots = \msc{A}(e_{r_1 + \dots + r_k}) = 0 $$
	Значит, при $ i = r_1, r_1 + r_2, \widedots[5em], r_1 + \dots + r_k $, $ \quad i $-й столбец "--- нулевой \\
	При $ i \ne r_1, \widedots[5em], r_1 + \dots + r_k $, $ \quad \msc{A}(e_i) = e_{i + 1} \implies i $-й столбец:
	$$
	\begin{pmatrix}
		0 & \cdots & \underset i0 & \underset{i + 1}1 & 0 & \cdots & 0
	\end{pmatrix}^T $$
\end{proof}

\section{Существование жордановой формы нильпотентного оператора}

\begin{theorem}
	Для любого нильпотентного оператора на конечномерном векторном пространстве существует жорданов базис.
\end{theorem}

\begin{proof}
	Будем доказывать, что существует базис из жордановых цепочек \\
	Положим $ W \define \ker \msc{A} $ \\
	Если мы возьмём ЛНЗ векторы из ядра и достроим (слева от них) цепочки, то получим жорданов базис \\
	Положим $ U_i \define \Img \msc{A}^i $
	$$ V = U_0 \supset U_1 \supset U_2 \supset \dots \supset U_{k - 1} \supset U_k = \set{0} $$
	где $ k $ "--- степень нильпотентности $ \msc{A} $ \\
	Заметим, что если $ v \in U_t \cap W $, то существует цепочка длины $ t + 1 $ с концом $ v $ \\
	Построим базис $ W $ (такой, чтобы можно было достроть цепочки): \\
	Будем пересекать $ W $ с $ U_i $ \\
	Выберем базис $ W \cap U_{k - 1} $. Он ЛНЗ, значит его можно дополнить до базиса $ W \cap U_{k - 2} $ \\
	В итоге получим базис $ W \cap U_0 = W $ \\
	Получили базис $ e_1, e_2, \dots $ пространства $ W $ \\
	Для $ e_i \in W \cap U_t $ построим цепочку длины $ t + 1 $ с концом $ e_i $:
	$$ e_1^{(i)} \to e_2^{(i)} \to \dots \to e_{t + 1}^{(i)} = e_i \to 0 $$
	Объединение цепочек "--- ЛНЗ (по лемме) \\
	Докажем, что это базис, \ie что набор порождающий: \\
	Докажем, что если $ \msc{A}^s(v) = 0 $, то $ v $ является ЛК векторов цепочек \\
	Докажем \textbf{индукцией} по $ s $:
	\begin{itemize}
		\item \textbf{База.} $ s = 1 $
		$$ \ker \msc{A}^1 = W, \qquad v \in W, \qquad e_1, e_2, \dots \text{ "--- базис } W $$

		\item \textbf{Переход.} $ s \to s + 1 $ \\
		Пусть $ \msc{A}^{s + 1}(v) = 0, \quad \msc{A}^s(v) \ne 0 $ \\
		Положим $ u = \msc{A}^s v \implies u \in U_s $
		$$ \underbrace{\overbrace{v \to \dots \to \dots \to u}^s \to 0}_{s + 1} $$
		Значит, $ \msc{A}(u) = 0 \implies u \in W $ \\
		Значит, $ u \in U_s \cap W $ \\
		Разложим его по базису $ U_s \cap W $ (тому, до которого мы дошли на каком-то очередном шаге дополнения базисов):
		$$ u = \sum_i a_ie_i $$
		$ \forall e_i $ из этого базиса выбрана цепочка длины хотя бы $ s + 1 $
		$$ e_i = e_{s + t_i}^{(i)} \text{ "--- последний вектор цепочки} $$
		Пусть $ e_i' $ "--- вектор цепочки, такой что $ \msc{A}^s(e_i') = e_i $ (вектор, который на $ s $ шагов раньше)
		$$ \msc{A}^s \bigg( \sum a_ie_i' \bigg) = \sum a_ie_i = u $$
		При этом, $ \msc{A}^s(v) \bydef u $ \\
		Получили 2 линейных представления $ u $, значит,
		$$ \msc{A}^s(v) = \msc{A}^s \bigg( \sum a_ie_i' \bigg) \implies \msc{A}^s \bigg( v - \sum a_ie_i' \bigg) = 0 $$
		Тогда, \textbf{по индукционному предположению}, $ v - \sum a_ie_i' $ представляется в виде ЛК векторов из цепочек \\
		Значит, $ v $ представляется в виде ЛК векторов цепочек
	\end{itemize}
\end{proof}

\section{Многочлен от оператора: произведение многочленов, инвариантность ядра и образа}

\begin{notation}
	$ V $ "--- векторное пространство над $ K $, $ \quad \msc{A} $ "--- оператор на $ V $, $ \quad P \in K[x] $
	$$ P(x) = a_nx^n + \dots + a_1x + a_0 $$
	Тогда $ P(\msc{A}) = a_n\msc{A}^n + \dots + a_1\msc{A} + a_0\msc{E} $, т. е. такой опрератор $ \msc{B} $, что
	$$ \msc{B}(v) = a_n \msc{A}^n(v) + \dots + a_2\msc{A}^2(v) + a_1\msc{A}(v) + a_0v $$
\end{notation}

\begin{lemma}[произведение многочленов от оператора]
	$ P, Q $ "--- многочлены, $ \msc{A} $ "--- оператор
	$$ \implies (PQ)(\msc{A}) = P(\msc{A}) \circ Q(\msc{A}) $$
\end{lemma}

\begin{proof}
	Пусть $ P(t) = \sum p_it^i, \quad Q(t) = \sum q_it^i, \quad R(t) = P(t)Q(t) $
	$$ R(t) = \sum p_iq_jt^{i + j} $$
	Положим $ \msc{B} = P(\msc{A}), \quad \msc{C} = Q(\msc{A}), \quad \msc{D} = R(\msc{A}) $ \\
	Нужно доказать, что $ \msc{B} \bigg( \msc{C}(v) \bigg) = \msc{D}(v) \quad \forall v $
	$$ \msc{B}(v) = \sum p_i \msc{A}^i(v), \qquad \msc{C}(v) = \sum q_j\msc{A}^j(v), \qquad \msc{D}(v) = \sum p_iq_j \msc{A}^{i + j}(v) $$
	$$ \msc{B} \bigg( \msc{C}(v) \bigg) = \msc{B} \bigg( \sum q_j \msc{A}^j(v) \bigg) = \sum q_j \msc{B} \bigg( \msc{A}^j(v) \bigg) = \sum q_j \bigg( \sum p_i \msc{A}^{i + j}(v) \bigg) = \sum q_jp_i \msc{A}^{i + j} = \msc{D}(v) $$
\end{proof}

\begin{implication}
	$ P, Q $ "--- многочлены, $ \quad \msc{A}, \msc{B}, \msc{C} $ "--- операторы, $ \qquad \msc{B} = P(\msc{A}), \quad \msc{C} = Q(\msc{A}) $
	$$ \implies \msc{B} \circ \msc{C} = \msc{C} \circ \msc{B} $$
\end{implication}

\begin{proof}
	$ PQ = QP \implies (PQ)(A) = (QP)(A) $
\end{proof}

\begin{theorem}[ядро и образ многочлена от оператора]
	$ \msc{A} $ "--- оператор на $ V, \quad P $ "--- многочлен, $ \quad \msc{B} = P(\msc{A}) $ \\
	Тогда $ \bm{\ker} \msc{B} $ и $ \bm\Img \msc{B} $ "--- \bt{инвариантные} подпространства относительно $ \msc{A} $
\end{theorem}

\begin{iproof}
	\item $ \ker \msc{B} $
	$$ v \in \ker \msc{B} \implies \msc{B}(v) = 0 \implies \msc{A} \bigg( \msc{B}(v) \bigg) = 0 \underimp{\text{лемма}} \msc{B} \bigg( \msc{A}(v) \bigg) = 0 \implies \msc{A}(v) \in \ker \msc{B} $$
	\item $ \Img \msc{B} $
	$$ v \in \Img \msc{B} \implies v = \msc{B}(w) \implies \msc{A}(v) = \msc{A} \bigg( \msc{B}(w) \bigg) \underset{\text{лемма}}= \msc{B} \bigg( \msc{A}(w) \bigg) $$
\end{iproof}

\section{Свойства аннулятора вектора}

\begin{definition}
	$ \msc{A} $ "--- оператор на $ V $, $ \qquad v \in V $
	\begin{itemize}
		\item \it{Аннулятором} $ v $ называется такой многочлен $ P $, что $ P(\msc{A})(v) = 0 $
		\item \it{Минимальным аннулятором} $ v $ называется многочлен наименьшей степени среди ненулевых аннуляторов
	\end{itemize}
\end{definition}

\begin{props}
	\item $ V $ \bt{конечномерно}.
	\begin{enumerate}
		\item у любого вектора существует ненулевой аннулятор

		\item если $ P_0 $ "--- минимальный аннулятор, то $ \deg P_0 \le \dim V $
	\end{enumerate}

	\item \label{prop:ann:2} $ P_1, \dots, P_k $ "--- аннуляторы $ v $
	$$ \forall \text{ многочл. } Q_1, \dots, Q_k \quad \text{ многочлен } S(t) = Q_1(t)P_1(t) + \dots + Q_k(t)P_k(t) \text{ "--- аннулятор } v $$

	\item $ P_0(t) $ "--- минимальный аннулятор.
	$$ P(t) \text{ "--- аннулятор } \iff P(t) \divby P_0(t) $$

	\item Минимальный аннулятор "--- единственный с точностью до ассоциированности (умножения на обратимый, \ie на константу)
\end{props}

\begin{eproof}
	\item Пусть $ n \define \dim V $ \\
	Докажем, что $ \exist P : \deg P \le n , $ $ P $ "--- аннулятор, $ P \ne 0 $ \\
	Возьмём
	$$ \underbrace{v, \msc{A}(v), \msc{A}^2(v), \dots, \msc{A}^n(v)}_{n + 1 \text{ вектор}} $$
	Они ЛЗ, т. к. их больше, чем размерность пространства. Значит,
	$$ \exist a_i \ne \bigodot : a_0v + a_1\msc{A}(v) + \dots + a_n\msc{A}^n(v) = 0 $$
	Подойдёт $ P(t) = a_nt^n + \dots + a_1t + a_0 $

	\item Пусть $ \msc{B}_i \define P_i(\msc{A}), \qquad \msc{C}_i = Q_i(\msc{A}), \qquad \msc{D} = S(\msc{A}) $
	$$ \msc{D}(v) = \msc{C}_1 \bigg( \underbrace{\msc{B}_1(v)}_{= 0 } \bigg) + \dots + \msc{C}_k \bigg( \underbrace{\msc{B}_k(v)}_{= 0} \bigg) = \msc{C}_1(0) + \dots + \msc{C}_k(0) = 0 $$

	\item Поделим с остатком:
	$$ P(t) = Q(t)P_0(t) + R(t), \qquad \deg R < \deg P_0 $$
	\begin{itemize}
		\item $ \impliedby $
		$$ R(t) = 0, \qquad P(t) = \underbrace{P_0(t)}_{\text{аннулятор}}Q(t) \text{ "--- аннулятор (по (\ref{prop:ann:2}.))} $$
		\item $ \implies $
		$$ R(t) = \underbrace{P(t)}_{\text{аннул.}} - Q(t)\underbrace{P_0(t)}_{\text{аннул.}} \text{ "--- аннулятор (по (\ref{prop:ann:2}.))} $$
	\end{itemize}

	\item $$ \exist P_1, P_2 \text{ "--- мин. аннул. } \implies \underbrace{P_1}_{\text{аннул.}} \divby \underbrace{P_2}_{\text{мин. аннул.}} $$
\end{eproof}

\section{Базис циклического подпространства}

\begin{definition}
	$ \msc{A} $ "--- оператор на $ V $, $ \qquad v \in V $ \\
	\it{Циклическим подпространством}, порождённым $ v $ называется минимальное по включению инвариантное подпространство, содержащее $ v $
\end{definition}

\begin{theorem}[базис циклического подпространства]
	$ k \in \N $ такое, что:
	\begin{enumerate}
		\item $ v, \msc{A}(v), \dots, \msc{A}^{k - 1}(v) $ ЛНЗ
		\item $ v, \msc{A}(v), \dots, \msc{A}^{k - 1}(v), \msc{A}^k(v) $ ЛЗ
	\end{enumerate}
	Тогда первый набор является базисом циклического подпространства, порождённого $ v $
\end{theorem}

\begin{proof}
	Пусть $ U $ "--- циклическое, порождённое $ v $
	$$ U \text{ "--- инвар. } \implies v \in U \implies \msc{A}v \in U \implies \underbrace{\msc{A}^2v}_{= \msc{A}(\msc{A}(v))} \in U \implies \dots \implies v, \msc{A}v, \dots, \msc{A}^{k - 1}v \in U $$
	Они ЛНЗ. Чтобы доказать, что это базис, надо доказать, что они порождают $ U $: \\
	Положим $ W = \braket{v, \msc{A}v, \dots, \msc{A}^{k - 1}v} $ \\
	Докажем, что $ W = U $:
	\begin{itemize}
		\item Докажем, что $ W $ "--- инвар.: \\
		$ \msc{A}^kv $ "--- ЛК $ v, \msc{A}v, \dots, \msc{A}^{k - 1}v $
		$$ w \in W, \qquad w = a_0v + \dots + a_{k - 1}\msc{A}^{k - 1}v $$
		$$ \msc{A}(w) = a_0\msc{A}v + \dots + a_{k - 2}\msc{A}^{k - 1}v + \underbrace{a_{k + 1}\msc{A}^kv}_{\text{ЛК } v, \dots, \msc{A}^{k - 1}v} $$
		Значит, $ w $ является ЛК $ v, \dots, \msc{A}^{k - 1}v $
		\item Докажем, что $ W $ "--- минимальное: \\
		Докажем, что если $ W_1 $ инвариантно и $ v \in W_1 $, то $ W \sub W_1 $:
		\begin{multline*}
			\begin{rcases}
				W_1 \text{ инвар. } \\
				v \in W_1
			\end{rcases} \implies \msc{A}v \in W_1, \qquad
			\begin{rcases}
				W_1 \text{ инвар. } \\
				\msc{A}v \in W_1
			\end{rcases} \implies \msc{A}^2v \in W_1, \quad \dots, \quad \underbrace{\msc{A}^iv}_{\text{порожд.} W} \in W_1 \implies \\
			\implies W_1 \sub W
		\end{multline*}
	\end{itemize}
\end{proof}

\section{Циклическое подпространство и минимальный аннулятор}

\begin{theorem}
	$ V $ "--- конечномерное, \qquad $ \msc{A} $ "--- оператор на $ V $, $ \qquad v \in V $ \\
	$ U $ "--- цикл. подпр-во, порождённое $ v, \qquad \chi $ "--- хар. многочлен $ \msc{A} $ на $ U $

	Тогда $ \chi $ "--- минимальный аннулятор $ v $
\end{theorem}

\begin{proof}
	Пусть $ k $ такое, что
	\begin{enumerate}
		\item $ v, \msc{A}v, \dots, \msc{A}^{k - 1}v $ ЛНЗ
		\item $ v, \msc{A}v, \dots, \msc{A}^{k - 1}v, \msc{A}^kv $ ЛЗ
	\end{enumerate}
	Выберем $ a_i $, не все равные нулю, такие, что
	$$ a_0v + a_1\msc{A}v + \dots + a_{k - 1}\msc{A}^{k - 1}v + a_k\msc{A}^kv = 0 $$
	Значит, $ a_k \ne 0 $ (\as $ v \dots, \msc{A}^{k - 1}v $ ЛНЗ) \\
	Поделим на $ a_k $, не меняя обозначений (\ie $ a_i \define \frac{a_i}{a_k} $):
	$$ \msc{A}^kv + \dots + a_1\msc{A}v + a_0 v = 0 $$
	Положим $ P(t) \define t^k + a_{k - 1}t^{k - 1} + \dots + a_1t + a_0 \implies P(t) $ "--- аннулятор
    \begin{itemize}
        \item Докажем, что $ P(t) $ "--- минимальный. \bt{Пусть это не так}:
        $$ \exist Q'(t) = b_mt^m + \dots + t_0, \qquad Q \ne 0, \qquad Q \text{ "--- аннул. }, \qquad m < k $$
        $$ b_m\msc{A}^mv + \dots + b_0v = 0 $$
        $ b_i \ne \bigodot \implies \msc{A}^mv, \dots, v $ "--- ЛЗ "--- \contra (это был не первый момент линейной зависимости)
        \item Докажем, что $ P(t) = \pm \chi $: \\
        Знаем, что $ v, \dots, \msc{A}^{k - 1}v $ "--- базис $ U $. Образы этих векторов:
		$$ \msc A \big( \msc A^i(v) \big) = \msc A^{i + 1}(v), \quad 0 \le i < k - 1 $$
		$$ \msc A \big( A^{k - 1}(v) \big)(v) = \msc A^k(v) = -a_0v - a_1 \msc A(v) - \dots - a_{k - 1} \msc A^{k - 1}(v) $$
        Значит, матрица $ \msc{A}\clamp{U} $ в этом базисе:
        $$ A =
		\begin{pNiceMatrix}
			0 & 		\Cdots & 	0 & 		-a_0 \\
			1 & 		& 			\Vdots & 	\\
			\Vdots & 	\Ddots & 	& 			\Vdots \\
			0 & 		\Cdots & 	1 & 		-a_{k - 1}
		\end{pNiceMatrix} $$
        $$ \chi_A(t) =
		\begin{vNiceMatrix}[last-col=6]
			-t &	 	\Cdots & & 	0 & 		-a_0 & 				\\
			1 & 	 	\Ddots & & 	\Vdots & 	\Vdots & 			+ \faktor{\rom1}t \\
			\Vdots & 	\Ddots & & 	& 			& 					+ \faktor{\rom2}t \\
			& 		 	& 		 & 	-t & 		-a_{k - 2} & 		\Vdots \\
			0 & 	 	\Cdots & & 	1 & 		-a_{k - 1} - t & 	\Vdots
		\end{vNiceMatrix} \quad =
		\begin{vNiceMatrix}[cell-space-limits=1pt]
			-t & 		\Cdots & 	& 0 & 		-a_1 \\
			\Vdots & 	\Ddots & 	& \Vdots & 	-a_1 - \frac{a_0}t \\
			& 			& 			& & 		-a_2 - \frac{a_1}t - \frac{a_0}{t^2} \\
			& 			& 			& -t & 		\Vdots \\
			0 & 		\Cdots & 	& 0 & 		-t - a_{k - 1} - \frac{a_{k - 2}}t - \dots - \frac{a_1}{t^{k - 2}} - \frac{a_0}{t^{k - 1}}
		\end{vNiceMatrix} $$
        Это будет $ (-1)^kP(t) $
    \end{itemize}
\end{proof}

\section{Минимальный многочлен оператора. Теорема Гамильтона\texorpdfstring{"--~}{--}Кэли и следствие из неё}

\begin{definition}
	Многочлен $ P(t) $ \it{аннулирует} $ \msc{A} $, если $ P(\msc{A}) = 0 $
\end{definition}

\begin{definition}
	\it{Минимальным многочленом} оператора $ \msc{A} $ называется ненулевой многочлен наименьшей степени, аннулирующий $ \msc{A} $
\end{definition}

\begin{properties}
	$ \msc{A} $ "--- оператор на $ V $
	\begin{enumerate}
		\item \label{prop:min_pol:1} $ P_1, \dots, P_k $ аннулируют $ \msc{A} $
		$$ \forall \text{ многочл. } Q_1, \dots, Q_k \quad S(t) = P_1(t)Q_1(t) + \dots + Pk(t)Q_k(t) \text{ аннулирует } \msc{A} $$

		\item $ P_0 $ "--- минимальный многочлен для $ \msc{A} $
		$$ P \text{ аннулирует } \msc{A} \iff P \divby P_0 $$

		\item Минимальный многочлен $ \msc{A} $ единственнен с точностью до ассоциирования

        \item $ e_1, \dots, e_n $ "--- базис $ V, \qquad P_1(t), \dots, P_n(t) $ "--- минимальные аннуляторы для $ e_1, \dots, e_n $ \\
        Тогда $ \NOK(P_1, \dots, P_n) $ является минимальным многочленом для $ A $
    \end{enumerate}
\end{properties}

\begin{eproof}
	\item $ \forall v \quad P_i $ "--- аннулятор $ v \implies S(\msc{A}) $ "--- аннулятор $ v \implies S $ аннулирует $ \msc{A} $

	\item Пусть $ P = P_0Q + R $
	\begin{itemize}
		\item Если $ P \divby P_0 $, то $ P = P_0Q \underimp{\ref{prop:min_pol:1}.} P \text{ аннулирует } \msc{A} $
		\item Если $ P $ аннулирует $ \msc{A} $, то $ R = P - P_0Q $ аннулирует $ \msc{A} \underimp{\ref{prop:min_pol:1}.} R = 0 \implies P \divby P_0 $
	\end{itemize}

	\item Если $ P_1, P_2 $ "--- минимальные многочлены, то $ P_1 \divby P_2 $ и $ P_2 \divby P_1 $.

	\item Пусть $ P = \NOK(P_1, \dots, P_n) $
    \begin{itemize}
        \item Проверим, что $ P $ аннулирует $ A $: \\
        Пусть $ v \in V, \qquad v = a_1e_1 + \dots + a_ne_n $ \\
        Применим $ P $:
        $$ P(\msc{A})(v) = a_1P(\msc{A})e_1 + \dots + a_nP(\msc{A})e_n $$
        $$ P \divby P_i \implies P \text{ "--- аннул. для } e_i \implies P(\msc{A})e_i = 0 $$
        $$ P(\msc{A})(v) = a_1 \cdot 0 + \dots + a_n \cdot 0 = 0 $$
        \item Проверим, что $ P $ минимальный: \\
        Пусть $ Q(t) $ аннулирует $ \msc{A} $
        \begin{multline*}
            \implies Q(\msc{A})v = 0 \quad \forall v \implies Q(\msc{A})e_i = 0 \quad \forall i \underimp{P_i \text{ "--- мин. аннул.}} \\
            \implies Q \divby P_i \quad \forall i \implies Q \divby P \implies \deg Q \ge \deg P
        \end{multline*}
    \end{itemize}
\end{eproof}

\begin{theorem}[Гамильтона"--~Кэли]
	Характеристический многочлен оператора $ \msc{A} $ аннулирует $ \msc{A} $.
\end{theorem}

\begin{proof}
	Нужно доказать, что $ \forall v \quad \chi(\msc{A})v = 0 $ \\
	Докажем, что $ \chi_{\msc{A}} \divby P_0 $, где $ P_0 $ "--- минимальный аннулятор (все аннуляторы делятся на минимальный): \\
	Пусть $ U $ "--- циклическое подпространство, порождённое $ v $ \\
	$ \chi_U $ "--- характеристический многочлен $ \msc{A}\clamp{U} $ (он определён, \as пространство инвариантно) \\
	По следствию о делителях характеристического многочлена, $ \chi \divby \chi_U $ \\
	Знаем, что $ \chi_U $ "--- минимальный аннулятор для $ v $ на $ U $ (по т. о циклическом подпространстве и минимальном аннуляторе)
	$$
	\begin{rcases}
		\chi_U = P_0 \\
		\chi \divby \chi_U
	\end{rcases} \implies \chi \divby P_0 $$
\end{proof}

\begin{implication}
	$ P_0 $ "--- минимальный многочлен $ \msc{A} $ \\
	Тогда $ \chi \divby P $
\end{implication}

\section{Свойства взаимно простых многочленов от оператора}

\begin{properties}
	$ \msc{A} $ "--- оператор на $ V $
	\begin{enumerate}
		\item $ P_1, P_2, \dots, P_k $ "--- попарно взаимно просты, $ \qquad T(t) = P_1(t) \cdots P_k(t), \qquad v \in V, \qquad T $ аннулир. $ v $. \\
		Тогда $ \exist v_1, \dots, v_k : \quad v = v_1 + \dots + v_k $ и $ P_i $ аннулирует $ v_i $

		\item $ P, Q $ взаимно просты, $ \qquad P, Q $ аннуляторы $ v \qquad \implies v = 0 $
	\end{enumerate}
\end{properties}

\begin{eproof}
	\item \textbf{Индукция.}
	\begin{itemize}
		\item \textbf{База.} $ k = 2 $ \\
		$ P, Q $ взаимно просты, $ \qquad v \in V $ \\
		Докажем, что $ \exist v, w : v = u + w, \qquad P(\msc{A})u = 0, \qquad Q(\msc{A})w = 0 $ \\
		Т. к. $ P, Q $ взаимно просты, можно разложить их НОД ($ = 1 $):
		$$ \exist F(t), G(t) : P(t)F(t) + Q(t)G(t) = 1 $$
		Применим к $ \msc{A} $:
		$$ P(\msc{A}) \circ F(\msc{A}) + Q(\msc{A}) \circ G(\msc{A}) = \msc{E} $$
		Применим к $ v $:
		$$ (PF)(\msc{A})v + (QG)(\msc{A})v = v $$
		Положим $ u = (QG)(\msc{A})v, \qquad w = (PF)(\msc{A})v $ \\
		Проверим, что $ P(\msc{A})u = 0 $ (для $ w $ "--- аналогично):
		$$ \bigg( P(\msc A) \circ QG(\msc{A}) \bigg)v = \bigg( PQG \bigg)(\msc{A})v = \bigg( GPQ \bigg)(\msc{A})v = G(\msc{A}) \underbrace{(PQ)(\msc{A})v}_{0 \text{ (это анн. } v)} = 0 $$
		\item \textbf{Переход.} $ k - 1 \to k $
		$$ T = \underbrace{P_1 \dots P_{k - 1}}_{P}\underbrace{P_k}_Q $$
		$$ (PQ)(\msc{A})v = 0 \underimp{\text{\textbf{база}}} \exist u, w : v = u + w, \qquad P(\msc{A})u = 0, \quad Q(\msc{A})w = 0 $$
		По \textbf{индукционному предположению},
		$$ \exist v_1, \dots, v_{k - 1} : P_i \text{ аннул. } v_i, \qquad u = v_1 + \dots + v_{k - 1} $$
		$$ v = v_1 + \dots + v_{k - 1} + \underset{\define v_k}w $$
	\end{itemize}

	\item Пусть $ T $ "--- минимальный аннулятор $ v $
	$$
	\begin{rcases}
		P \divby T \\
		Q \divby T
	\end{rcases} \implies T = \const, \qquad T(t) = c \implies cv = 0 \implies v = 0 $$
\end{eproof}

\section{Разложение пространства в прямую сумму примарных подпространств}

\begin{definition}
	$ K $ "--- поле, $ \qquad V $ "--- векторное пространство над $ K, \qquad \msc{A} $ "--- оператор на $ V $ \\
	$ P(t) $ "--- минимальный многочлен $ \msc{A} $ со старшим коэффициентом, равным 1. \\
	Пространство $ V $ называется \it{примарным} (относительно $ \msc{A} $), если $ P(t) = Q^s(t) $ для некоторого $ Q(t) $, неприводимого над $ K $
\end{definition}

\begin{theorem}
	$ K $ "--- поле, $ \qquad V $ "--- векторное пространство над $ K, \qquad \msc{A} $ "--- оператор на $ V $ \\
	$ P(t) $ "--- минимальный многочлен $ \msc{A} $, он разложен на множители:
	$$ P(t) = P_1(t) \cdots P_k(t), \qquad \text{ где } P_i(t) = Q_i^{s_i}(t), \qquad Q_i \text{ "--- непривод. над } K $$
	Тогда $ \exist $ подпространства $ U_1, \dots, U_k $, такие что
	\begin{enumerate}
		\item все $ U_i $ ивариантны
		\item $ V = U_1 \oplus \dots \oplus U_k $
		\item $ P_i(t) $ "--- минимальный многочлен $ \msc{A} $ на $ U_i \quad \forall i $
	\end{enumerate}
\end{theorem}

\begin{proof}
	Положим $ U_i = \ker P_i(\msc{A}) $. Докажем, что они подойдут:
	\begin{enumerate}
		\item Знаем, что ядро многочлена от оператора инвариантно
		\item
		\begin{enumerate}
			\item Докажем, что $ V = U_1 + \dots + U_k $ \\
			$ P_1, \dots, P_k $ попарно взаимно просты, и $ P_1 \cdots P_k $ аннулирует любой $ v $, значит
			$$ \forall v \quad \exist v_1, \dots, v_k : \quad v = v_1 + \dots + v_k, \qquad P_i \text{ аннул. } v_i \implies v_i \in U_i $$
			\item Докажем, что сумма прямая: \\
			Нужно проверить, что $ U_s \cap \bigg( U_1 + \dots + U_{s - 1} + U_{s + 1} + \dots + U_k \bigg) = \set{0} \quad \forall s $ \\
			НУО проверим, что $ (U_1 + \dots + U_{k - 1}) \cap U_k = \set{0} $ \\
			Возьмём $ v \in (U_1 + \dots + U_{k - 1}) \cap U_k $
			$$ v = v_1 + \dots + v_{k - 1}, \qquad v_i \in U_i, \qquad v \in U_k $$
			По первому свойству, $ P_1 \cdots P_{k - 1} $ аннулирует $ v_1 + \dots + v_{k - 1} = v $ \\
			При этом, $ P_k $ аннулирует $ v $ \\
			Заметим, что $ (P_1 \cdots P_{k - 1}, P_k) = 1 $ \\
			По второму свойству, это означает, что $ v = 0 $.
		\end{enumerate}
		\item
		$$ U_i = \ker P_i(\msc{A}) \implies P_i(\msc{A}) \clamp{U_i} = 0 $$
		$ P_i $ аннулирует $ \msc{A}\clamp{U_i} $ \\
		Значит, $ P_i $ делится на минимальный многочлен $ \msc{A}\clamp{U_i} $ \\
		При этом, $ P_i = Q_i^{s_i} $ \\
		Отсюда минимальный тоже является $ Q_i^{r_i}, \qquad r_i \le s_i $ \\
		Хотим доказать, что $ r_i = s_i $ \\
		Пусть $ T = Q_1^{r_1} \dots Q_k^{r_k} $ \\
		Т. к. у нас прямая сумма, сущестует $ e_1, \dots, e_n $ "--- базис $ V $, он является объединением базисов $ U_i $
		$$ \implies T(\msc{A})e_1 = 0, \quad \dots, \quad T(\msc{A})e_k = 0 $$
		$$ \implies T \text{ аннулирует } \msc{A} \underimp{P \text{ "--- мин. многочл.}} \underbrace{T}_{\prod Q_i^{r_i}} \divby \underbrace{P}_{\prod Q_i^{s_i}}, \quad r_i \ge s_i \implies r_i = s_i $$
	\end{enumerate}
\end{proof}

\section{Корневые подпространства}

\begin{definition}
	$ \lambda $ "--- с. ч. $ \msc{A} $ \\
	Вектор $ v $ называется \it{корневым вектором}, соответствующим $ \lambda $, если для некоторого $ k $ многочлен $ P(t) = (t - \lambda)^k $ является аннулятором $ v $ \\
	Множество корневых векторов называется \it{корневым подпространством}, соотв. $ \lambda $
\end{definition}

\begin{props}
	\item Корневое подпространство инвариантно

	\item $ V $ конечномерно, минимальный многочлен $ \msc{A} $ раскладывается на линейные множители
	$$ P(t) = (\lambda_1 - t)^{s_1} \cdots (\lambda_k - t)^{s_k} $$
	Тогда $ \ker \bigg( (\lambda_i \msc{E} - \msc{A})^{s_i} \bigg) $ "--- корневые подпространства
\end{props}

\begin{eproof}
	\item Пусть $ P(t) = (\lambda - t)^k $ "--- аннул. $ v $, \ie $ P(\msc{A})v = 0 $
	$$ P(\msc{A})(\msc{A}v) = \bigg( P(\msc{A}) \circ \msc{A} \bigg)v = \bigg( \msc{A} \circ P(\msc{A}) \bigg)v = \msc{A} \bigg( \underbrace{P(\msc{A})v}_{= 0} \bigg) = \msc{A}(0) = 0 $$

	\item Пусть $ U_i = \ker \bigg( (\lambda_i \msc{E} - \msc{A})^{s_i} \bigg), \qquad W_i $ "--- корневое подпространство для $ \lambda_i $
	\begin{itemize}
		\item $ U_i \sub W_i $ "--- очевидно ($ v \in U_i \implies (\lambda_i \msc{E} - \msc{A})^{s_i}v = 0, \qquad $ подойдёт $ k = s_i $)
		\item $ W_i \sub U_i $ \\
		Нужно показать, что если вектор аннулируется, то он это сделает не больше чем за $ s_i $ шагов \\
		Пусть $ v \in W_i $ \\
		Пусть $ k $ "--- минимальное число, такое что $ (\lambda_i \msc{E} - \msc{A})^k $ аннулирует $ v $ \\
		Тогда $ (\lambda - t)^k $ "--- минимальный аннулятор $ v $ \\
		При этом, $ P(t) $ "--- аннулятор $ v $
		$$ \implies P(t) \divby (\lambda - t)^k \implies k \le s_i \implies v \in U_i $$
	\end{itemize}
\end{eproof}

\section{Существование жордановой формы}

\begin{definition}
	\it{Жордановой клеткой} порядка $ r $ с с. ч. $ \lambda $ называется матрица порядка $ r $ вида
	$$ J_r(\lambda) =
	\begin{pNiceMatrix}
		\lambda & \Cdots & \Cdots & 0 \\
		1 & \Ddots & & \Vdots \\
		\Vdots & \Ddots & & \Vdots \\
		0 & \Cdots & 1 & \lambda
	\end{pNiceMatrix} $$
\end{definition}

\begin{definition}
	\it{Жордановой матрицей} называется блочно-диагональная матрица вида
	$$ \diagmatrix{J_{1_1}(\lambda_1)}{J_{r_k}(\lambda_k)} \qquad \text{(как } r_i \text{, так и } \lambda_i \text{ могут совпадать)} $$
\end{definition}

\begin{definition}
	\it{Жорданов базис} "--- базис, в котором матрица оператора жорданова
\end{definition}

\begin{theorem}[существование жордановой формы]
	$ K $ "--- поле, $ \qquad V $ "--- векторное пространство над $ K $ \\
	$ \msc{A} $ "--- оператор, $ \qquad \bm{\chi_{\msc{A}}(t)} $ \bt{раскладывается на линейные множители} над $ K $ \\
	Тогда для $ \msc{A} $ существует жорданов базис
\end{theorem}

\begin{iproof}
	\item Докажем для случая, когда минимальный многочлен $ \msc{A} $ имеет вид $ P(t) = (t - \lambda)^r $ \\
	Сведём к случаю нильпотентного оператора: \\
	Положим $ \msc B = \msc{A} - \lambda\msc{E} $ \\
	$ \msc{B}^r = 0, \quad \msc{B} $ "--- нильпотентный \\
	Значит, существует жорданов базис $ \msc{B} $, причём на диагонали жордановой формы стоят нули
	\item Общий случай
	$$ \chi_{\msc{A}} = (-1)^n(t - \lambda_1)^{s_1} \cdots (t - \lambda_m)^{s_m} $$
	По следствию к теореме Гамильтона"--~Кэли минимальный многочлен "--- делитель $ \chi \implies $ минимальный многочлен имеет вид
	$$ P(t) = (t - \lambda_1)^{r_1} \cdots (t - \lambda_m)^{r_m} $$
	Применим теорему о разложении в сумму примарных подпространств: \\
	Пусть $ Q_i \define (t - \lambda_i)^{r_i} $ \\
	По теореме
	$$ V = U_1 \oplus \dots \oplus U_k $$
	$ U_i $ инвариантны \\
	$ Q_i(t) $ "--- минимальный многочлен $ \msc{A} $ на $ U_i $ \\
	К $ U_i $ применяем нильпотентный случай: \\
	Существует жорданов базис $ U_i $ \\
	Матрица $ \msc{A}\clamp{U_i} $ имеет вид
	$$ J_i = \diagmatrix{J_{r_1}(\lambda_i)}{J_{r_k}(\lambda_k)} $$
	Значит, в базисе, полученном объединением базисов $ U_i $ матрица $ \msc{A} $ имеет вид
	$$ J = \diagmatrix{J_1}{J_m} $$
\end{iproof}

\section{Возведение жордановой клетки в степень}

\begin{props}
	\item
	\begin{itemize}
		\item $ \bigg( J_r(0) \bigg)^s =
		\begin{pNiceMatrix}
			0 &			\Cdots & 	& 			& 	& 			0 \\
			\Vdots & 	& 			& 			& 	& 			\Vdots \\
			0 & 		& 			& 			& 	& 			\\
			1 & 		\Ddots & 	& 			& 	& 			\\
			\Vdots & 	\Ddots & 	\Ddots & 	& 	& 			\\
			0 & 		\Cdots & 	1 & 		0 & \Cdots & 	0
		\end{pNiceMatrix} \qquad $ при $ s < r $ \\
		То есть,
		$$ a_{ij} =
		\begin{cases}
			1, \qquad i - j = s \\
			0, \qquad \text{иначе}
		\end{cases} $$

		\item $ \bigg( J_r(0) \bigg)^s = \On[] \qquad $ при $ s \ge r $
	\end{itemize}

	\item Пусть $ \lambda \ne 0, \qquad A = \bigg( J_r(\lambda) \bigg)^s $ \\
	Тогда $ A $ нижнетреугольная
	$$ a_{ij} =
	\begin{cases}
		\lambda^{s - (i - j)}C_s^{i - j}, \qquad 0 \le i - j \le s \\
		0, \qquad \text{иначе}
	\end{cases} $$

	\item $ \rk \bigg( \big( J_r(0) \big)^s \bigg) =
	\begin{cases}
		r - s, \qquad s < r \\
		0, \qquad s \ge r
	\end{cases} $
\end{props}

\begin{eproof}
	\item \bt{Индукция} по $ s $.
	\begin{itemize}
		\item \textbf{База.} $ s = 1 $
		$$ J_1(0) = (0) $$
		\item \textbf{Переход.} $ s \to s + 1 $
		$$ J_r^s(0) = a_{ij}, \qquad J_r(0) = b_{ij}, \qquad J_r^{s + 1}(0) = c_{ij} $$
		\begin{equ}{power_jord:1}
			c_{ij} = a_{i1}b_{1j} + a_{i2}b_{2i} + \dots + a_{in}b_{ni}
		\end{equ}
		Среди $ a_{ij} $ не более одной единицы, остальные нули \\
		Среди $ b_{xj} $ не более одной единицы, остальные нули \\
		Значит, $ c_{ij} = 0 $ или $ c_{ij} = 1 $
		$$ c_{ij} = 1 \iff \exist x :
		\begin{cases}
			a_{ix} = 1 \\
			b_{xi} = 1
		\end{cases} \qquad \underiff{\eref{power_jord:1}} \exist x :
		\begin{cases}
			i - x = s \\
			x - j = 1
		\end{cases} \qquad \iff i - j = s + 1 $$
	\end{itemize}

	\item $ J_r(\lambda) = \lambda \cdot E + J_r(0) $ \\
	Возведём в степень и распишем как бином Ньютона (учитывая, что $ \lambda E $ коммутирует с чем угодно, а значит, можно приводить подобные):
	\begin{multline*}
		\bigg( J_r(\lambda) \bigg)^s = (\lambda E)^s + C_s^1 (\lambda E)^{s - 1} J_r(0) + \dots + C_s^{r - 1}(\lambda E)^{s - r + 1} J_r(0)^{r - 1} + \underbrace{J_r^r(0)}_{= 0}(\dots) \undereq{\text{св-во } 1a} \\
		= \lambda^s E + C_s^1 \lambda^{s - 1}J_r(0) + \dots + C_s^{r - 1}\lambda^{s - r + 1}J_r^{r - 1}(0) = \\
		=
		\begin{pmatrix}
			\lambda^s & . & 0 \\
			. & . & . \\
			0 & . & \lambda^s
		\end{pmatrix} +
		\begin{pmatrix}
			0 & . & . & 0 \\
			\lambda^{s - 1}C_s^1 & . & . & 0 \\
			. & . & . & . \\
			0 & . & \lambda^{s - 1}C_s^1 & 0
		\end{pmatrix} + \dots +
		\begin{pmatrix}
			0 & . & 0 \\
			. & . & . \\
			\lambda^{s - r + 1}C_s^{r - 1} & . & 0
		\end{pmatrix}
	\end{multline*}
\end{eproof}

\section{Количество жордановых блоков и ранг. Следствие о единственности жорадновой формы}

\begin{theorem}[количество клеток и ранг]
	$ J $ "--- жорданова матрица \\
	Тогда количество клеток вида $ J_r(\lambda) $ равно
	$$ \rk \bigg( J - \lambda E \bigg)^{r - 1} - 2 \rk \bigg( J - \lambda E \bigg)^r + \rk \bigg( J - \lambda E \bigg)^{r + 1} $$
\end{theorem}

\begin{proof}
	Положим $ f(s) \define \rk (J - \lambda E)^s $
	\begin{multline*}
		(J - \lambda E)^s = \left\lgroup \diagmatrix{J_{r_1}(\lambda_1)}{J_{r_k}(\lambda_k)} - \lambda E \right\rgroup^s = \diagmatrix{J_{r_1}(\lambda_1 - \lambda)}{J_{r_k}(\lambda_k - \lambda)}^s = \\
		= \diagmatrix{J_{r_1}^s(\lambda_1 - \lambda)}{J_{r_k}^s(\lambda_k - \lambda)}
	\end{multline*}
	Какое-то из $ \lambda_i $ совпало с $ \lambda $
	$$ f(s) = \sum_{i = 1}^k \rk \bigg( J_{r_i}^s(\lambda_i - \lambda) \bigg) $$
	\begin{itemize}
		\item Если $ \lambda \ne \lambda_i $, то $ \rk \bigg( J_{r_i}^s(\lambda_i - \lambda) \bigg) = r_i \quad \forall s $
		$$ f(s) - f(s + 1) = \sum \bigg\lgroup \rk \bigg( J_{r_i}^s(\lambda_i - \lambda) \bigg) - \rk \bigg( J_{r_i}^{s + 1}(\lambda - \lambda_i) \bigg) \bigg\rgroup $$
		То есть, если $ \lambda_i \ne \lambda_j $, то $ i $-е слагаемое равно $ r_i - r_j = 0 $
		\item Если $ \lambda_i = \lambda, \quad r_i \le s $, то $ i $-е слагаемое равно $ 0 - 0 = 0 $
		\item Если $ \lambda_i = \lambda, \quad r_i > s $, то $ i $-е слагаемое равно $ (r_i - s) - \bigg( r_i - (s + 1) \bigg) = 1 $
	\end{itemize}
	$ f(s + 1) - f(s) $ "--- количество клеток, для которых $ \lambda_i = \lambda, \quad r_i > s $ \\
	$ \bigg( f(s + 1) - f(s) \bigg) - \bigg( f(s) - f(s - 1) \bigg) $ "--- количество клеток размера $ s $ \\
	Это равно $ f(s + 1) - 2f(s) + f(s - 1) $
\end{proof}

\begin{implication}[единственность жордановой формы]
	$ J, J' $ "--- жордановы матрицы $ \msc{A} $ в некоторых базисах \\
	Тогда $ J, J' $ совпадают с точностью до перестановки жордановых клеток
\end{implication}

\begin{proof}
	Ранг не зависит от выбора базиса, значит, количество клеток каждого вида совпадает.
\end{proof}

\section(Минимальный многочлен оператора, у которого известна жорданова форма){Минимальный многочлен оператора, у которого известна \\жорданова форма}

\begin{theorem}
	$ J $ "--- жорданова матрица, $ \qquad \lambda_1, \dots, \lambda_k $ "--- с. ч. $ J $ \\
	$ r_i $ "--- максимальный размер жордановой клетки, соотв. $ \lambda_i $ \\
	Тогда минимальный многочлен равен $ (t - \lambda_1)^{r_1} \cdots (t - \lambda_k)^{r_k} $
\end{theorem}

\begin{proof}
	Пусть $ e_1, \dots, e_n $ "--- жорданов базис \\
	$ P_i $ "--- минимальный аннулятор $ e_i $ \\
	Тогда минимальный многочлен равен $ \NOK(P_1, \dots, P_n) $ \\
	Пусть $ e_i $ соответствует $ j $-му столбцу клетки $ J_r(\lambda) $
	$$ (\msc{A} - \lambda \msc{E})^{r - i}(e_i) = 0, \qquad (\msc{A} - \lambda \msc{E})^{r - i - 1}(e_i) \ne 0 $$
	$$ \implies P_i(t) = (t - \lambda)^{r - i} $$
	Минимальный многочлен "--- это НОК многочленов вида $ (t - \lambda_i)^s, \quad s \le r_i $ \\
	Среди них есть $ (t - \lambda_1)^{r_1}, \dots, (t - \lambda_k)^{r_k} $ \\
	Значит, среди них есть $ P_i $, а остальные "--- не делители
	$$ \implies \text{НОК} = (t - \lambda_1)^{r_1} \cdots (t - \lambda_k)^{r_k} $$
\end{proof}

\section{Комплексификация вещественного векторного пространства. Продолжение операторов}

\begin{definition}
	$ V $ "--- векторное пространство над $ \R $ \\
	\it{Комплексификация} $ V $ "--- это множество $ \hat{V} $, состоящее из пар $ (u, v) $ с операцией $ \Co \times \hat{V} \to \hat{V} $, заданной равенством
	$$ (a + bi) \cdot (u, v) = \big( au - bv, av + bu \big) $$
	и операцией $ \hat{V} \times \hat{V} \to \hat V $, заданной равенством
	$$ (u_1, v_1) + (u_2, v_2) = \big( u_1 + u_2, v_1 + v_2 \big) $$
\end{definition}

\begin{definition}
	$ w = (u, v) $ \\
	$ (u, -v) $ называется сопряжённым к $ w $
\end{definition}

\begin{notation}
	$ \ol{w} $
\end{notation}

\begin{theorem}
	$ \hat{V} $ "--- векторное пространство над $ \Co $
\end{theorem}

\begin{proof}
	\hfill
	\begin{enumerate}
		\item $ \hat{V} $ "--- абелева группа по сложению
		\item $ 1 \cdot w = w $
		\item Ассоциативность умножения
		\item Две дистрибутивности умножения
	\end{enumerate}
	Всё проверяется подстановкой
\end{proof}

\begin{notation}
	Пару $ (u, v) $ будем обозначать $ u + vi $
\end{notation}

\begin{notation}
	Множество пар $ (u, 0) $ отождествим с $ V $
\end{notation}

\begin{theorem}[базис комплексификации]
	Пусть $ e_1, \dots, e_n $ "--- базис $ V $ \\
	Тогда $ e_1 = e_1 + 0 \cdot i, \dots, e_n = e_n + 0 \cdot i $ "--- базис $ \hat{V} $
\end{theorem}

\begin{iproof}
	\item Докажем, что система является порождающей: \\
	Пусть $ w \in \hat{V} $ \\
	Разложим $ u $ и $ v $ по базису $ e_1, \dots, e_n $ в $ V $:
	$$ u = a_1e_1 + \dots + a_ne_n, \qquad v = b_1e_1 + \dots + b_ne_n, \qquad a_s, b_s \in \R $$
	$$ w = (\underbrace{a_1 + b_1i}_{\in \Co})e_1 + \dots + (\underbrace{a_n + b_ni}_{\in \Co})e_n $$
	\item Докажем ЛНЗ: \\
	Пусть $ c_1e_1 + \dots + c_ne_n = 0, \quad c_s \in \Co, \quad c_s = a_s + b_si, \quad a_s, b_s \in \R $
	$$ (a_1 + b_1i)(e_1 + 0i) + \dots + (a_n + b_ni)(e_n + 0i) = 0 $$
	Разделим вещественную и мнимую части:
	$$ \bigg( (a_1e_1 - b_10) + \dots + (a_ne_n - b_n0) \bigg) + \bigg( (a_10 + b_1e_1) + \dots + (a_n0 + b_ne_n) \bigg) i = 0 + 0i $$
	Значит, каждое большое слагаемое равно нулю:
	$$
	\begin{cases}
		a_1e_1 + \dots + a_ne_n = 0 \\
		b_1e_1 + \dots + b_ne_n = 0
	\end{cases} \underimp{e_1, \dots, e_n \text{ ЛНЗ в } V}
	\begin{cases}
		a_1 = \dots = a_n = 0 \\
		b_1 = \dots = b_n = 0
	\end{cases} $$
\end{iproof}

\begin{implication}
	$ \dim_\Co\hat{V} = \dim_\R V $
\end{implication}

\begin{props}[сопряжённых векторов]
	\item $ \ol{\ol{w}} = w $

	\item $ \ol{w_1 + w_2} = \ol{w_1} + \ol{w_2}, \qquad \ol{z \cdot w} = \ol{z} \cdot \ol{w} $

	\item $ w_1, \dots, w_n $ ЛНЗ $ \iff \ol{w_1}, \dots, \ol{w_n} $ ЛНЗ
\end{props}

\begin{eproof}
	\item $ w = u + vi, \qquad \ol{w} = u - vi, \qquad \ol{\ol{w}} = u - (-v)i = u + vi = w $

	\item Первое равенство "--- упражнение. Проверим второе: \\
	Пусть $ z = a + bi, \quad w = u + vi $
	$$ \ol{(a + bi)(u + vi)} = \ol{(au - bv) + (av + bu)i} = (au - bv) - (av + bu)i $$
	$$ \ol{(a + bi)} \cdot \ol{(u + vi)} = (a - b_i)(u - v_i) = \bigg( \underbrace{au - (-b)(-v)}_{au - bv} \bigg) + \bigg( \underbrace{a(-v) + (-b)u}_{-(av + bu)} \bigg) $$

	\item Достаточно доказать в одну сторону ($ \implies $), дальше сошлёмся на первое свойство \\
	Пусть $ \ol{w_1}, \dots, \ol{w_n} $ ЛЗ, то есть
	$$ c_1, \dots, c_n \in \Co : \quad c_1\ol{w_1} + \dots + c_n\ol{w_n} = 0, \qquad c_i \nin \bigodot $$
	$$ 0 = \ol0 = \ol{c_1\ol{w_1} + \dots + c_n\ol{w_n}} \undereq{\text{2 св-во}} \ol{c_1}\ol{\ol{w_1}} + \dots + \ol{c_n}\ol{\ol{w_n}} \undereq{\text{1 св-во}} \ol{c_1}w_1 + \dots + \ol{c_n}w_n $$
	$$ c_i \nin \bigodot \implies \ol{c_i} \nin \bigodot $$
	$ w_1, \dots, w_n $ ЛЗ "--- \contra
\end{eproof}

\begin{definition}
	$ \msc{A} $ "--- оператор на $ V $ \\
	\it{Продолжением} $ \msc{A} $ на $ \hat{V} $ называется отображение $ \hat{\msc{A}} : \hat{V} \to \hat{V} $, заданное равенством $ \hat{\msc{A}}(u + vi) = \msc{A}(u) + \msc{A}(v)i $
\end{definition}

\begin{property}
	$ \hat{\msc{A}} $ линейно
\end{property}

\section{Каноническая матрица оператора в вещественном пространстве}

\begin{notation}
	$ P(t) = c_kt^k + c_{k - 1}t^{k - 1} + \dots + c_0, \qquad c_s \in \Co $ \\
	Тогда $ \ol{P}(t) = \ol{c_k}t^k + \ol{c_{k - 1}}t^{k - 1} + \dots + \ol{c_0} $ "--- \it{сопряжённый} к $ P $
\end{notation}

\begin{lemma}[применение операторов к сопряжённым векторам]
	$ \msc{A} $ "--- оператор на $ V $. Тогда
	\begin{enumerate}
		\item $ \hat{\msc{A}}(\ol{w}) = \ol{\hat{\msc{A}}(w)} $

		\item $ P(\hat{\msc{A}})(w_1) = w_2 \implies \ol{P}(\hat{\msc{A}})(\ol{w_1}) = \ol{w_2} $

		\item Если $ P(t) $ аннулирует $ w $, то $ \ol{P}(t) $ аннулирует $ \ol{w} $

		\item Если $ w $ "--- корневой вектор, соответствующий $ \lambda $, то $ \ol{w} $ "--- корневой вектор, соответствующий $ \ol{\lambda} $

		\item Если $ w_1, \dots, w_n $ "--- (жорданов) базис корневого подпространства, соответствующего $ \lambda $, то $ \ol{w}_1, \dots, \\
		\ol{w}_n $ "--- (жорданов) базис корневого подпространства, соответствующего $ \ol{\lambda} $ \nimp[(если один жорданов, то и второй жорданов)]
	\end{enumerate}
\end{lemma}

\begin{eproof}
	\item Пусть $ w = u + iv, \qquad \ol{w} = u - iv $
	$$ \hat{\msc{A}}(w) = \msc{A}u + \msc{A}vi, \qquad \msc{A}(\ol{w}) = \msc{A}u + \msc{A}(-v)i = \msc{A}u - \msc{A}vi $$

	\item Из первого свойства $ \hat{\msc{A}}^{(s)}(\ol{w_1}) = \ol{\msc{A}^{(s)}(w_1)} $ \\
	Пусть $ P(t) = c_kt^k + \dots + c_0 $
	$$ w_2 = c_kP(\hat{\msc{A}})(w_1) + \dots + c_0w_1 $$
	$$ \ol{w_2} = \ol{c_k}\ol{P}(\hat{\msc{A}})(\ol{w_1}) + \dots + c_0\ol{w_1} = \ol{P}(\hat{\msc{A}})(\ol{w_1}) $$

	\item $ P(\hat{\msc{A}})(w) = 0 \implies \ol{P}(\hat{\msc{A}})(\ol{w}) \undereq{\text{2 св-во}} \ol{P(\hat{\msc{A}})(w)} = \ol0 = 0 $

	\item $ P(t) = (t - \lambda)^k $ аннулирует $ w $ для некоторого $ k $ \\
	$ \implies \ol{P}(t) $ аннулирует $ \ol{w} $ (из 3 св-ва)
	$$ \ol{P}(t) = (t - \ol{\lambda})^k $$

	\item
	\begin{itemize}
		\item ЛНЗ доказана
		\item Докажем, что это порождающая система: \\
		Пусть $ \ol{w} $ принадлежит пространству, соответстсвующему $ \ol\lambda \implies w $ принадлежит пространству, соотв. $ \lambda $ \\
		Разложим по базису:
		$$ \exist c_i : \quad w = c_1e_1 + \dots + c_ne_n $$
		$$ \implies \ol{w} = \ol{c_1}\ol{e_1} + \dots + \ol{c_n}\ol{e_n} $$
		\item Докажем, что сопряжённый к жорданову базису жорданов:
		$$ \hat{\msc{A} - \lambda \msc{E}} = \hat{\msc{A}} - \lambda \hat{\msc{E}} $$
		$$ (\hat{\msc{A}} - \lambda \hat{\msc{E}})e_i = e_{i - 1} \implies (\hat{\msc{A}} - \ol{\lambda}\msc{E})\ol{e_i} = \ol{e_{i + 1}} $$
	\end{itemize}
\end{eproof}

\begin{theorem}
	Пусть $ V $ - конечномерное векторное пространство над $ \R $, $ \quad \msc{A} $ "--- оператор на $ V $ \\
	Тогда существует базис $ V $, в котором матрица $ \msc{A} $ является блочно-диагональной, и каждый блок "--- либо жорданова клетка, либо имеет вид
	$$
	\begin{pNiceMatrix}
		a &  	 b & 	  \Cdots & 	& 	& 	& 	 0 \\
		-b & 	 a & 	  & 	   	& 	& 	& 	 \Vdots \\
		1 &  	 0 & 	  \Ddots & 	& 	& 	& 	 \\
		0 &  	 1 & 	  & 	   	& 	& 	& 	 \\
		\Vdots & & 	 	  \Ddots & 	& 	& 	& 	 \\
		& 	 	 & 	 	  & 		1 & 0 & a &  b \\
		0 &  	 \Cdots & & 		0 & 1 & -b & a
	\end{pNiceMatrix} $$
\end{theorem}

\begin{proof}
	Пусть минимальный многочлен $ \msc{A} $ равен
	$$ P(t) = (t - a_1)^{m_1} \cdots (t^2 + p_1t + q_1)^{s_1} \cdots, $$
	где $ t^2 + p_it + q_i $ не имеют вещественных корней \\
	Разложим $ V $ в прямую сумму примарных подпространств \\
	Достаточно доказать для одного подпространства \\
	Для подпространства, соответствующего $ (t - a)^m $ есть базис, в котором матрица $ \msc{A} $ жорданова \\
	Рассмотрим подпространство, соответствующее $ (t^2 + pt + q)^s $: \\
	Пусть $ \lambda, \ol\lambda $ "--- комплексные корни $ t^2 + pt  q $
	$$ (t^2 + pt + q)^s = (t - \lambda)^s(t - \ol\lambda)^s $$
	Пусть $ P_1 = (t^2 + pt + q)^s $ \\
	Знаем, что $ P_1(\msc{A}) = 0 $ на корневом подпространстве $ U $ \\
	Тогда $ P_1(\hat{\msc{A}}) = 0 $ на $ \hat{U} $
	$$ \hat{U} = \hat{W_1} + \hat{W_2}, \qquad \hat{W_1}, \hat{W_2} \text{ "--- корневые подпространства для } \lambda, \ol\lambda $$
	Существует жорданов базис $ w_1, \dots, w_k $ для $ \hat{W_1} $ \\
	Тогда $ \ol{w_1}, \dots, \ol{w_k} $ "--- жорданов базис для $ \hat{W_2} $ \\
	$ w_1, \dots, w_k, \ol{w_1}, \dots, \ol{w_k} $ "--- базис $ \hat{U} $ \\
	Пусть $ w_i = u_i + iv_i $ \\
	Докажем, что $ u_1, v_1, u_2, v_2, \dots, u_k, v_k $ "--- базис $ U $: \\
	Следующие элементарные преобразования переводят базис $ \hat U $ в базис $ \hat U $, а значит, и в базис $ U $:
	$$ u_i \pm iv_i \quad \to \quad (u_i + iv_i) + (u_i - iv_i), ~ u_i - iv_i \quad \to \quad u_i, ~ u_i - v_i \quad \to \quad u_i, ~ (u_i - iv_i) - u_i \quad \to \quad u_i, v_i $$
	Проверим, что в этом базисе получается правильная жорданова матрица: \\
	Рассмотрим жордановы цепочки
	$$ w_1, \dots, w_{r_1}, w_{r_1 + 1}, \dots, w_{r_1 + r_2}, \widedots[4em] $$
	Докажем, что им соответствуют клетки размера $ 2r_1, 2r_2, \dots $: \\
	Рассмотрим первую цепочку:
	$$ \hat{\msc{A}}(u_m + iv_m) =
	\begin{cases}
		\lambda(u_m + iv_m) + (u_m + iv_m), \qquad m < r_1 \\
		\lambda(u_r + iv_r), \qquad m = r
	\end{cases} $$
	Пусть $ \lambda = a + bi $ \\
	При $ m < r $,
	$$ \msc{A}(u_m) + \msc{A}(v_m)i = (au_m - bv_m) + (bu_m + av_m)i = \underbrace{(au_m - bv_m)}_{\msc{A}(u_m)} + \underbrace{(bu_m + av_m)}_{\msc{A}(v_m)}i $$
	При $ m = r $,
	$$ \msc{A}(u_m) = (au_m - bv_m) + (bu_m + av_m)i = (au_m - bv_m) + (bu_m + av_m)i $$
\end{proof}
