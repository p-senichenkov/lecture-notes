\part{Линейные отображения в евклидовых и унитарных пространствах}

\section{Изоморфизм векторного пространства и двойственного к не\hyph{}му}

\begin{definition}
	$ V $ -- векторное пространство над полем $ K $ \\
	Линейным функционалом на $ V $ называется линейное отображение $ V \to K $
\end{definition}

\begin{property}
	Линейные функционалы пространства $ V $ над $ K $ образуют веторное пространство над $ K $
\end{property}

\begin{proof}
	Очевидно
\end{proof}

\begin{definition}
	Пространство функционалов называется двойственным или сопряжённым
\end{definition}

\begin{notation}
	$ V^* $
\end{notation}

\begin{theorem}[изоморфизм пространства и двойственного к нему]
	\hfill
	\begin{enumerate}
		\item $ V $ -- конечномерное пространство над $ K $
		$$ V^* \simeq V $$

		\item $ V $ -- евклидово пространство \\
		Для любого $ v \in V $ определим $ y_v \in V^* $ как $ y_v(x) = (x, v) $ \\
		Тогда отображение $ v \mapsto y_v $ является изоморфизмом
	\end{enumerate}
\end{theorem}

\begin{eproof}
	\item Пусть $ n \define \dim V $ \\
	Достаточно доказать, что $ \dim V^* = n $ (тогда можно будет построить изоморфизм из базиса в базис) \\
	Зафиксируем базис: \\
	Пусть $ e_1, ..., e_n $ -- базис $ V $ \\
	Пусть $ \vphi : V^* \to K^n $ такое, что $ \vphi(y) = \bigg( \underbrace{y(e_1)}_{\in K}, ..., \underbrace{y(e_n)}_{\in K} \bigg) $ \\
	Мы знаем, что протранства одной размерности изоморфны, так что $ K^n \simeq V $ \\
	Докажем, что это изоморфизм:
	\begin{itemize}
		\item Линейность:
		\begin{itemize}
			\item Надо проверить, что $ \vphi(y_1 + y_2) \iseq \vphi(y_1) + \vphi(y_2) $
			\begin{multline*}
				\vphi(y_1 + y_2) = \bigg( (y_1 + y_2)(e_1), \widedots[5em], (y_1 + y_2)(e_n) \bigg) = \\
				= \bigg( y_1(e_1) + y_2(e_1), \widedots[5em], y_1(e_n) + y_2(e_n) \bigg) \undereq{\text{сложение в } K^n \text{ покомпонентно}} \\
				= \bigg( y_1(e_1), ..., y_1(e_n) \bigg) + \bigg( y_2(e_1), ..., y_2(e_n) \bigg) = \vphi(y_1) + \vphi(y_2)
			\end{multline*}
			\item Надо проверить, что $ \vphi(ky) \iseq k\vphi(y) $
			\begin{multline*}
				\vphi(ky) = \bigg( (ky)(e_1), \widedots[3em], (ky)(e_n) \bigg) = \bigg( ky(e_1), \widedots[3em], ky(e_n) \bigg) = \\
				\undereq{\text{умножение в } K^n \text{ покомпонентно}} k \bigg( y(e_1), \widedots[3em], y(e_n) \bigg) = k\vphi(y)
			\end{multline*}
		\end{itemize}
		\item Биективность: \\
		Пусть $ a \in K^n, \qquad a = (a_1, ..., a_n), \quad a_i \in K $
		$$ \exist ! y \in V^* : \quad \vphi(y) = a $$
		так как
		$$ \exist ! y \in V^* : y(e_1) = a_1, \widedots[4em], y(e_n) = a_n $$
	\end{itemize}

	\item
	\begin{itemize}
		\item Проверим, что $ y_v \in V^* $, т. е. что $ y_v $ линейно:
		\begin{itemize}
			\item $ y_v(x_1 + x_2) = (x_1 + x_2, v) \undereq{\text{скалярное произведение линейно по первой координате}} (x_1, v) + (x_2, v) = y_v(x_1) + y_v(x_2) $
			\item $ y_v(kx) = (kx, v) = k(x, v) = ky_v(x) $
		\end{itemize}
		\item Пусть $ \vphi(v) = y_v $. Докажем, что $ \vphi $ -- изоморфизм $ V \to V^* $:
		\begin{itemize}
			\item Линейность:
			\begin{itemize}
				\item $ \vphi(u + v) \iseq \vphi(u) + \vphi(v) $
				\begin{multline*}
					\vphi(u + v) \iseq \vphi(u) + \vphi(v) \quad \iff \quad y_{u + v} \iseq y_u + y_v \quad \iff \\
					\iff \quad y_{u + v}(x) \iseq y_u(x) + y_v(x) \quad \forall x \quad \iff \\
					\iff \quad (x, u + v) \undereq{\text{лин. скалярного произв.}} (x, u) + (x, v)
				\end{multline*}
				\begin{remark}
					В унитарном пространстве не будет этого равенства
				\end{remark}

				\item $ \vphi(kv) \iseq k\vphi(v) $
				\begin{multline*}
					\vphi(kv) \iseq k\vphi(v) \quad \iff \quad y_{kv} \iseq ky_v \quad \iff \quad y_{kv}(x) \iseq ky_v(x) \quad \forall x \quad \iff \\
					(x, kv) \undereq{\text{лин. скалярного произв.}} k(x, v)
				\end{multline*}
			\end{itemize}

			\item Инъективность: \\
			Пусть $ \vphi(v) = 0 $. Тогда
			$$ y_v = 0 \quad \implies \quad y_v(x) = 0 \quad \forall x \quad \implies (x, v) = 0 \quad \forall x \quad \implies v = 0 $$
			Вместе с тем, что $ \dim V = \dim V^* $, это даёт биективность
		\end{itemize}
	\end{itemize}
\end{eproof}

\begin{definition}
	Изоморфизм из пункта 2 называется каноническим изоморфизмом из $ V $ в $ V^* $
\end{definition}

\section{Дважды двойственное пространство}

\begin{theorem}[дважды двойственное пространство]
	$ V $ -- векторное пространство над $ K $ \\
	Для любого $ x \in V $ обозначим через $ z_x $ отображение $ V^* \to K $, заданное формулой $ z_x(\underset{\in V^*}{y}) = \underbrace{y(x)}_{\in K} $ \\
	Тогда:
	\begin{enumerate}
		\item $ \forall x \in K \quad z_x \in (V^*)^* $, т. е. $ z_x $ -- линейный функционал на $ V^* $

		\item отображение $ \vphi : V \to (V^*)^* $, заданное формулой $ \vphi(x) = z_x $ является линейным

		\item если $ V $ конечномерно, то $ \vphi $ -- изоморфизм
	\end{enumerate}
\end{theorem}

\begin{eproof}
	\item
	\begin{itemize}
		\item $ z_x(y_1 + y_2) \iseq z_x(y_1) + z_x(y_2) $
		$$ z_x(y_1 + y_2) = (y_1 + y_2)(x) $$
		$$ z_x(y_1) + z_x(y_2) = y_1(x) + y_2(x) $$
		\item $ z_x(ky) \iseq kz_x(y) $
		$$ z_x(ky) = (ky)(x) = ky(x) = kz_x(y) $$
	\end{itemize}

	\item
	\begin{itemize}
		\item $ \vphi(x_1 + x_2) \iseq \vphi(x_1) + \vphi(x_2) $
		$$ z_{x_1 + x_2} \iseq z_{x_1} + z_{x_2} $$
		$$ \forall y \quad z_{x_1 + x_2}(y) = z_{x_1}(y) + z_{x_2}(y) $$
		$$ y(x_1 + x_2) \iseq y(x_1) + y(x_2) $$
		Это верно, так как $ y $ линейно
		\item $ \vphi(kx) \iseq k\vphi(x) $
		$$ z_{kx} \iseq kz_x $$
		$$ \forall y \quad z_{kx}(y) \iseq kz_x(y) $$
		$$ y(kx) \iseq ky(x) $$
		Это верно, так как $ y $ линейно
	\end{itemize}

	\item Размерности равны, так что достаточно доказать инъективность: \\
	$ \vphi $ инъективно $ \iff \vphi(x) = 0 $ только при $ x = 0 \quad \iff z_x $ -- нулевое отображение только при $ x = 0 \quad \iff z_x(y) = 0 \quad \forall y \quad \iff y(x) = 0 \quad \forall y $ \\
	Нужно проверить, что $ \forall x \ne 0 \quad \exist $ линейное отображение $ y : \quad y(x) \ne 0 $ \\
	Дополним до базиса: \\
	Пусть $ x, e_2, ..., e_n $ -- базис $ V $ \\
	Определим $ y : y(x) = 1, \quad y(e_i) = 0 $
	$$ y(\alpha x + \beta_2e_2 + ... + \beta_ne_n) = \alpha $$
	Оно линейно, $ y(x) \ne 0 $
\end{eproof}

\section{Двойственный базис. Матрица перехода для двойственного базиса}

\begin{lemma}
	$ V $ -- конечномерное векторное пространство, $ \quad e_1, ..., e_n $ -- базис $ V $ \\
	$ f_1, ..., f_n \in V^* $ такие, что $ f_i(e_i) = 1, \quad f_i(e_j) = 0 $ при $ i \ne j \quad $ \nimp[(здесь существование не утверждается, но понятно, что их всегда можно построить)] \\
	Тогда $ f_1, .., f_n $ -- базис $ V^* $
\end{lemma}

\begin{proof}
	Знаем, что $ \dim V = \dim V^* $ \\
	Достаточно доказать ЛНЗ: \\
	Возьмём ЛК: \\
	Пусть $ a_1, ..., a_n \in K $ такие, что $ f = a_1f_1 + ... + a_nf_n $ -- нулевой функционал
	$$ 0 = f(e_i) = a_1\underbrace{f_1(e_i)}_0 + ... + a_i\underbrace{f_i(e_i)}_1 + ... + a_n\underbrace{f_n(e_i)}_0 = a_i \quad \forall i $$
\end{proof}

\begin{definition}
	$ e_1, ..., e_n $ -- базис $ V, \qquad f_1, ..., f_n $ -- базис $ V^*, \qquad f_i(e_i) = 1, \quad f_i(e_j) = 0 $ при $ i \ne j $ \\
	Тогда $ f_1, ..., f_n $ называется двойственным базисом к $ e_1, ..., e_n $
\end{definition}

\begin{remind}
	$ e_i, e_i' $ -- базисы $ V $ \\
	Матрицей перехода от $ e_i $ к $ e_i' $ называется такая матрица $ C $, что в $ i $-м столбце записаны координаты $ e_i' $ в $ e_1, ..., e_n $ \\
	Пусть $ X, X' $ -- координаты $ c $ в $ e_i, e_i' $. Тогда $ X = CX' $
\end{remind}

\begin{theorem}
	$ e_i, e_i' $ -- базисы $ V, \qquad C $ -- матрица перехода от $ e_i $ к $ e_i' $ \\
	$ f_i, f_i' $ -- соответствующие двойственные базисы \\
	Тогда:
	\begin{enumerate}
		\item Матрица перехода от $ f_i $ к $ f_i' $ равна $ (C^{-1})^T \nimp[~ = ~ (C^T)^{-1}] $

		\item Пусть $ Y, Y' $ -- строки координат $ y \in V^* $ в базисах $ f_i, f_i' $ \\
		Тогда $ Y' = YC $
	\end{enumerate}
\end{theorem}

\begin{eproof}
	\item Пусть $ D = (d_{ij}) $ -- матрица перехода от $ f_i $ к $ f_i' $
	$$ U = (u_{ij}), \qquad U = D^TC $$
	Докажем, что $ U = E $
	$$ e_i' = c_{1i}e_1 + c_{2i}e_2 + ..., \qquad f_j' = d_{1j}f_1 + d_{2j}f_2 + ... $$
	Применим одно к другому:
	\begin{multline*}
		\begin{rcases}
			1, \quad i = j \\
			0, \quad i \ne j
		\end{rcases} = f_j'(e_i') = d_{1j}f_1(c_{1i}e_1 + c_{2i}e_2 + ...) + d_{2j}f_2(c_{1i}e_i + c_{2i}e_2 + ...) + \widedots[5em] = \\
		= d_{1j}c_{1i} \cdot 1 + d_{1j}c_{2i} \cdot 0 + \widedots[3em] + d_{2j}c_{1i} \cdot 0 + d_{2j}c_{2i} \cdot 1 + \widedots[3em] = d_{1j}c_{1i} + d_{2j}c_{2i} + \widedots[3em]
	\end{multline*}
	$ d $ -- этой $ j $-я строка $ D^T, \quad c $ -- $ i $-й столбец $ C $ \\
	Значит, $ f_j'(e_i') = u_{ji} $

	\item $ (C^{-1})^T $ -- матрица перехода от $ f_i $ к $ f_i' $ \\
	$ Y^T, Y'^T $ -- столбцы координат $ y $ \\
	$ Y^T = (C^{-1})^TY'^T $ -- транспонированный
	$$ Y = Y'C^{-1} \implies YC = Y' $$
\end{eproof}

\section{Собственные числа самосопряжённого оператора. Лемма об эрмитовой матрице}

\begin{definition}
	$ \mc{A} $ -- оператор в евклидовом или унитарном пространстве \\
	$ \mc{B} $ называется сопряжённым к $ \mc{A} $, если $ (\mc{A}x, y) = (x, \mc{B}y) \quad \forall x, y $
\end{definition}

\begin{notation}
	$ \mc{A}^* $
\end{notation}

\begin{props}
	\item $ (\mc{A}^*)^* = \mc{A} $
	\item Пусть $ A, A^* $ -- матрицы $ \mc{A}, \mc{A}^* $ в некотором ОНБ \\
	Тогда
	\begin{itemize}
		\item $ A^* = A^T $ в евклидовом пространстве
		\item $ A^* = \ol{A}^T $ в унитарном пространстве
	\end{itemize}
\end{props}

\begin{definition}
	Оператор в веклидовом или унитарном пространстве назвыается
	\begin{itemize}
		\item нормальным, если $ \mc{A}^*\mc{A} = \mc{A}\mc{A}^* $
		\item ортогональным (унитарным), если $ \mc{A}\mc{A}^* = \mc{A}^*\mc{A} = \mc{E} $
		\item самосопряжённым, если $ \mc{A}^* = \mc{A} $
	\end{itemize}
\end{definition}

\begin{definition}
	Квадратная матрица называется
	\begin{itemize}
		\item симметричной (симметрической), если $ A = A^T $
		\item эрмитовой, если $ A = \ol{A}^T $
	\end{itemize}
\end{definition}

\begin{property}
	$ \mc{A} $ -- оператор в евлидовом/унитарном пространстве, $ \quad A $ -- его матрица \bt{в ОНБ} \\
	Тогда \\
	$ \mc{A} $ самосопряжённый $ \iff A $ симметрична/эрмитова
\end{property}

\begin{theorem}
	$ \mc{A} $ -- нормальный оператор в унитарном пространстве \\
	Тогда
	\begin{enumerate}
		\item если $ \lambda $ -- с. ч. $ \mc{A} $, то $ \ol\lambda $ -- с. ч. $ \mc{A}^* $
		\item с. в. $ \mc{A}^* $, соответствующие разным с. ч. ортогональны
		\item Существует ОНБ, состоящий из с. в. $ \mc{A} \quad $ \nimp[$ \implies $ он диагонализуем]
	\end{enumerate}
\end{theorem}

\begin{lemma}
	$ \mc{A} $ -- самосопряжённый оператор на унитарном пространстве \\
	Тогда $ (\mc{A}x, x) \in \R \quad \forall x $
\end{lemma}

\begin{proof}
	$$ (\mc{A}x, x) \undereq{\text{самосопр.}} (x, \mc{A}^*x) = (x, \mc{A}x) $$
	$$ (\mc{A}x, x) \undereq{\text{полуторалинейность}} \ol{(x, \mc{A}x)} $$
	$$ \implies (x, \mc{A}x) \in \R \implies (\mc{A}x, x) \in \R $$
\end{proof}

\begin{definition}
	Самосопряжённый оператор назвыается положительно определённым, если
	$$ (\mc{A}x, x) > 0 \quad \forall x \ne 0 $$
\end{definition}

\begin{notation}
	$ a_i \in \bigodot \quad \iff a_1 = \dots = a_n = 0 $
\end{notation}

\begin{theorem}[о собственных числах самосопряжённого оператора]
	\hfill \\
	$ \mc{A} $ -- оператор на унитарном пространстве
	\begin{enumerate}
		\item $ \mc{A} $ -- нормальный \\
		$ \mc{A} $ самоспряжённый $ \quad \iff \quad $ все с. ч. $ \mc{A} $ вещественные

		\item $ \mc{A} $ -- самосопряжённый \\
		$ \mc{A} $ положительно определён $ \quad \iff \quad $ все с. ч. положительны
	\end{enumerate}
\end{theorem}

\begin{eproof}
	\item Знаем, что существует ОНБ из с. в. \\
	Пусть $ \lambda_i $ -- с. ч. \\
	$ A, A^* $ -- матрицы $ \mc{A} $ и $ \ol{\mc{A}^*} $ в этом базисе $ \quad \implies A^* = A^T $ \\
	$ \mc{A} $ -- самосопряжённый $ \iff A = A^* \iff A = \ol{A^T} \iff $
	$$
	\begin{pmatrix}
		\lambda_1 & & 0 \\
		. & . & . \\
		0 & . & \lambda_n
	\end{pmatrix} =
	\begin{pmatrix}
		\ol{\lambda_1} & & 0 \\
		. & . & . \\
		0 & . & \ol{\lambda_n}
	\end{pmatrix}^T \quad \iff \lambda_i = \ol{\lambda_i} \quad \forall i \quad \iff \lambda_i \in \R $$

	\item Пусть $ e_i $ -- ОНБ из с. в., $ \qquad \lambda_i $ -- с. ч., $ \qquad \lambda_i \in \R $ (т. к. самосопр. -- частный случай нормального) \\
	Пусть $ x = a_1e_1 + ... + a_ne_n $
	\begin{multline*}
		(\mc{A}x, x) = (a_1\lambda_1e_1 + ... + a_n\lambda_ne_n, \quad a_1e_1 + ... + a_ne_n) = \sum \lambda_ia_i\ol{a_j}\underbrace{(e_i, e_j)}_{0 \text{ или } 1} = \\
		= \sum \lambda_ia_i\ol{a_i} = \sum \lambda_i |a_i|^2 \nimp[\in \R]
	\end{multline*}
	\begin{itemize}
		\item Если $ \lambda_i > 0 \quad \forall i $, то $ \sum \underbrace{\lambda_i}_{> 0} |a_i|^2 \ge 0 $ \\
		Равенство достигается только при $ |a_i|^2 \in \bigodot $, то есть $ a_i \in \bigodot $. Значит, $ x = 0 $
		\item Пусть не все $ \lambda_i > 0, \qquad \lambda_{i_0} \le 0 $ \\
		Для $ x = e_{i_0} \quad x \ne 0, \quad (\mc{A}x, x) = \lambda_{i_0} \le 0 $ -- \contra
	\end{itemize}
\end{eproof}

\section{Ортогональность собственных векторов. Самосопряжённый оператор на \texorpdfstring{$ \R^n $}{R\textasciicircum{}n}}

\begin{lemma}\label{lemma:ermit}
	$ A $ -- эрмитова матрица \\
	Тогда все корни $ \chi_A(t) $ вещественны
\end{lemma}

\begin{proof}
	$ A $ -- матрица порядка $ n $ \\
	Определим оператор $ \mc{A} : \Co^n $ как $ X \mapsto AX $ \\
	Тогда $ A $ -- матрица $ \mc{A} $ в стандартном базисе \\
	$ A $ -- эрмитова; станд. базис является ОНБ $ \quad \implies \mc{A} $ -- самосопряжённый \\
	Все с. ч. $ \mc{A} $ вещественны, это и есть корни $ \chi_A(t) $
\end{proof}

\begin{lemma}[ортогональность с. в.]
	$ \mc{A} $ самосопряжённый на $ \R^n, \qquad \mu, \lambda $ -- различные с. ч., $ \quad x, y $ -- соостветсвующие с. в. \\
	Тогда $ (x, y) = 0 $
\end{lemma}

\begin{remark}
	Доказательство для $ \Co^n $ здесь не пожходит
\end{remark}

\begin{proof}
	$$ \lambda(x, y) \undereq{\text{линейность}} (\lambda x, y) \undereq{\text{с. в.}} (\mc{A}x, y) \bdefeq{\mc{A}^*} (x, \mc{A}^*y) \undereq{\text{самоспр.}} (x, \mc{A}y) \undereq{\text{с. в.}} (x, \mu x) \undereq{\text{линейность}} \mu(x, y) $$
\end{proof}

\begin{theorem}
	$ \mc{A} $ -- самосопряжённый оператор на $ \R^n $ \\
	Тогда
	\begin{enumerate}
		\item $ \chi_A(t) $ раскладывается на линейные множители над $ \R $
		\item Существует ОНБ $ \R^n $, состоящий из с. в. $ \mc{A} $
	\end{enumerate}
\end{theorem}

\begin{eproof}
	\item Разложим $ \chi_A(t) $ на линейные множиетли над $ \Co $:
	$$ \chi_A(t) = (-1)^n(t - \lambda_1)...(t - \lambda_n), \qquad \lambda_i \in \Co $$
	Пусть $ A $ -- матрица $ \mc{A} $ в стандартном базисе $ \quad \implies A = A^T \quad \underimp{A \text{ вещ.}} A = \ol{A^T} \quad \implies $ \\
	$ \implies A $ эрмитова $ \quad \underimp{\text{лемма \ref{lemma:ermit}}} \lambda_i \in \R \quad \forall i $

	\item $ \chi_A(t) $ раскладывается на линейные множители над $ \R $ \\
	$ \mc{A} $ диагонализуем над $ \Co $. Есть базис из с. в. в $ \Co^n $ \\
	$ \implies $ есть базис из с. в. в $ \R^n $ (т. к. $ \R $ -- поле, там неоткуда взяться комплексным числам, а коэффициенты изначально были вещественные, \textit{надо нормально это записать}) \\
	$ \implies \dim U_{\lambda_i} = r_i $ \\
	Выберем ОНБ в каждом подпространстве $ U_{\lambda_i} $ \\
	По лемме об ортогональный с. в. объединение этих базисов -- ОНБ $ \R^n $
\end{eproof}

\section{Корень из самосопряжённого оператора. Полярное разложение}

\begin{theorem}[корень из самосопряжённого оператора]
	\hfill \\
	$ \mc{A} $ -- положительно определённый самосопряжённый \\
	Тогда существует положиетльно определённый самосопряжённый $ \mc{B} : \quad \mc{A} = \mc{B}^2 $ \nimp[в смысле композиции]
\end{theorem}

\begin{proof}
	$ \mc{A} $ -- самосопряжённый $ \implies \mc{A} $ -- нормальный $ \implies \exist $ ОНБ из с. в. $ \mc{A} $ \\
	Пусть $ e_1, ..., e_n $ -- ОНБ из с. в., $ \qquad \lambda_1, ... \lambda_n $ -- с. ч. \\
	$ \mc{A} $ -- самоспряжённый $ \implies \lambda_i \in \R $ \\
	$ \mc{A} $ -- полож. опред. $ \implies \lambda_i > 0 $ \\
	Определим $ \mc{B} $ как $ \mc{B}(e_i) = \sqrt{\lambda_i}e_i $ \\
	Проверим, что он подойдёт: \\
	Рассмотрим матрицу $ \mc{B} $ в базисе $ e_1, ..., e_n $:
	$$ B =
	\begin{pmatrix}
		\sqrt{\lambda_1} & . & . \\
		. & . & . \\
		. & . & \sqrt{\lambda_n}
	\end{pmatrix} $$
	Она эрмитова $ \implies \mc{B} $ самоспряжённый \\
	$ \sqrt{\lambda_i} > 0 \implies \mc{B} $ положиетльно определён
	$$ \mc{B} \big( \mc{B}(e_i) \big) = \mc{B}(\sqrt{\lambda_i}e_i) = \lambda_ie_i = \mc{A}(e_i) \quad \forall i \quad \implies \mc{B}^2 = \mc{A} $$
\end{proof}

\begin{lemma}
	$ \mc{A} $ невырожденный \\
	Тогда $ \mc{A}\mc{A}^* $ - самосопряжённый положительно определённый
\end{lemma}

\begin{proof}
	$$ \bigg( \mc{A}\mc{A}^* \bigg)^* = \bigg( \mc{A}^* \bigg)^* \mc{A}^* = \mc{A}\mc{A}^* $$
	$$ \bigg( \mc{A}^*\mc{A}x, x \bigg) = \bigg( \mc{A}^*(\mc{A}x), x \bigg) = \bigg( \mc{A}x, (\mc{A}^*)^*x \bigg) = (\mc{A}x, \mc{A}x) \underset{\mc{A}x \ne 0 \text{, т. к. } \mc{A} \text{ невырожд.}}> 0 $$
\end{proof}

\begin{theorem}[полярное разложение оператора]
	\hfill \\
	$ \mc{A} $ -- невырожденный \nimp[(обратимый)] оператор на унитарном пространстве \\
	Тогда $ \exist \mc{U}, \mc{B} $ такие, что:
	\begin{enumerate}
		\item $ \mc{U} $ унитарный
		\item $ \mc{B} $ -- самосопряжённый положительно определённый
		\item $ \mc{A} = \mc{U}\mc{B} $
	\end{enumerate}
\end{theorem}

\begin{proof}
	$ \mc{A}\mc{A}^* $ самосопряжённый положительно определённый (по лемме). Значит
	$$ \exist \mc{B} : \quad \mc{B}^2 = \mc{A}^*\mc{A}, \qquad \mc{B} \text{ полож. опр. самосопряж.} $$
	$ \mc{A}^*, \mc{A} $ невырожденные $ \implies \mc{A}^*\mc{A} $ невырожденный $ \implies \mc{B} $ невырожденный $ \implies \exist \mc{B}^{-1} $ \\
	Положим $ \mc{U} = \mc{A}\mc{B}^{-1} $ \\
	Докажем, что эти $ \mc{U}, \mc{B} $ подойдут: \\
	Осталось проверить только унитарность, т. е. что $ \mc{U}^* \iseq \mc{U} $
	$$ \mc{U}^* = \bigg( \mc{A}\mc{B}^{-1} \bigg)^* = \bigg( \mc{B}^{-1} \bigg)^* \mc{A}^* \undereq{\text{видно из матрицы}} \bigg( \mc{B}^* \bigg)^{-1}\mc{A}^* \undereq{\mc{B} \text{ самоспр.}} \mc{B}^{-1} \mc{A}^* $$
	$$ \mc{U}^*\mc{U} = \bigg( \mc{B}^{-1}\mc{A}^* \bigg) \bigg( \mc{A}\mc{B}^{-1} \bigg) = \mc{B}^{-1} \bigg( \mc{A}^*\mc{A} \bigg)\mc{B}^{-1} = \mc{B}^{-1}\mc{B}^2\mc{B}^{-1} = \mc{E} $$
\end{proof}

\begin{implication}[перестановка сомножителей]
	$ \mc{A} $ -- невырожденный оператор \\
	Тогда $ \exist $ уинтарный $ \mc{U} $ и самосопряжённый положиетльно определённый $ \mc{B} $ такие, что $ \mc{A} = \mc{B}\mc{U} $
\end{implication}

\begin{proof}
	Применим теорему у $ \mc{A}^* $: \\
	$ \mc{A}^* = \mc{U}_1\mc{B}, \qquad \mc{U}_1 $ -- унитарный, $ \quad \mc{B} $ -- самосопряжённый пол. опред.
	$$ \mc{A} = \bigg( \mc{A}^* \bigg)^* = \bigg( \mc{B}\mc{U}_1 \bigg)^* = \mc{U}_1^* \mc{B}^* = \mc{U}_1^*\mc{B} $$
	Подойдёт $ \mc{U} = \mc{U}_1 $
\end{proof}

\section{Квадратичные формы: ортогональное преобразование, преобразование двух форм}

$ A $ -- симметрическая матрица, $ \quad A = (a_{ij}) $ \\
Соответствующая квадратичная форма $ f(x_1, ..., x_n) = \sum a_{ij}x_ix_j $ \\
Матричная запись: $ f(x_1, ..., x_n) = X^TAX $ \\
Линейное преобразование $ X = CY \quad \implies \quad A \mapsto C^TAC $ \\
Неособое преобразование: $ C $ обратима \\
Диагональный (канонический) вид -- если $ A $ -- диагональна

\begin{theorem}[ортогональное преобразование квадратичной формы]
	\hfill
	\begin{enumerate}
		\item Вещественная квадратичная форма может быть приведена к диагональному виду ортогональным преобразованием

		\item Если $ C $ -- ортогональная матрица, $ C^TAC $ -- диагональная, то на диагонали матрицы $ C^TAC $ записаны с. ч. матрицы $ A $
	\end{enumerate}
\end{theorem}

\begin{eg}
	$$ \underbrace{2x_1^2 + x_2^@ - 4x_1x_2 - 4x_2x_3}_{Q} + 12x_1 - 8x_2 + 8x_3 + 6 = 0 $$
	Приведём $ Q $ к диагональному виду:
	$$ C =
	\begin{pmatrix}
		\faktor23 & \faktor23 & \faktor13 \\
		-\faktor23 & \faktor13 & \faktor23 \\
		\faktor13 & -\faktor23 & \faktor23
	\end{pmatrix} $$
	С. ч.: $ \lambda_1 = 4, \quad \lambda_2 = 1, \quad \lambda_3 = -2 $
	$$ 4y_1^2 + 1y_2^2 - 2y_3^2 + 12 \bigg( \frac23 y_1 + \frac23 y_2 + \frac13 y_3 \bigg) - 8 $$
	$$ 4y_1^2 + 16y_1 = 4(y_1 + 2)^2 - 16 $$
	\TODO{Этот пример есть в Боревиче. Надо посмотреть}
\end{eg}

\begin{eproof}
	\item $ \mc{A} $ -- оператор на $ \R^n $, матрица в стандартном базисе \TODO{что-то} \\
	$ \mc{A} $ самосопряжённый $ \implies \exist $ ОНБ $ T_1, ..., T_n $ из с. в. \\
	Пусть $ \lambda_1, ..., \lambda_n $ -- с. ч. \\
	Матрица $ \mc{A} $ в $ T_1, ..., T_n $ является диагональной \\
	Матрица $ \mc{A} $ в $ T_1, ..., T_n $ равна $ C^{-1}AC $, где $ C $ -- матрица перехода \\
	$ C $ состоит из столбцов $ T_i $, т. к. это матрица перехода от стандартного базиса к $ T_i $ \\
	Значит, $ C $ -- ортогональная матрица \\
	$ C^{-1}AC = C^TAC $, т. к. $ C $ ортогональна

	\item Пусть $ B = C^TAC, $ она диагональна, $ \qquad \mu_1, ..., \mu_n $ -- числа на диагонали \\
	$ S_1, ..., S_n $ -- столбцы $ B \quad \implies B = C^{-1}AC \implies B $ -- матрица $ \mc{A} $ в ОНБ $ S_1, ..., S_n \implies \mc{A}S_i = \mu_iS_i \implies \mu_i $ -- с. ч.
\end{eproof}

\begin{theorem}[преобразование двух форм]
	\hfill \\
	$ f(x_1, ..., x_n), \quad g(x_1, ..., x_n) $ -- вещественные квадратичные формы, $ \qquad f $ положительно определена \\
	Тогда существует неособое преобразование, при котором обе формы приводятся к диагональному виду
\end{theorem}

\begin{proof}
	Композиция неособенных преобразований -- неособенное преобразование, так что можно сделать несколько шагов:
	\begin{enumerate}
		\item Приведём $ f $ к диагональному виду $ f_1 $:
		$$ f_1(y_1, ..., y_n) = \lambda_1y_1^2 + ... + \lambda_ny_n^2, \qquad \lambda_i > 0 $$
		\item Избавимся от $ \lambda $:
		$$ z_i = \sqrt{\lambda_i}y_i $$
		$$ f_2(z_1, ..., z_n) = z_1^2 + ... + z_n^2 $$
		При этом, $ g_2 $ тоже как-то изменилась:
		$$ g_2(z_1, ..., z_n) $$
		Нужно доказать, что форму $ f_2 = z_1^2 + ... + z_n^2 $ и любую форму $ g_2 $ можно одновеременно привести к диагоналному виду
		\item Приведём $ g_2 $ к диагональному виду ортогональным перобразованием $ C $ \\
		Матрица $ f_2 $ равна $ E $
		$$ E \to C^TEC = C^TC = E $$
		Значит, $ f $ приведена к диагональному виду
	\end{enumerate}
\end{proof}
