\part{Кольца и поля}

\section{Идеал кольца. Примеры колец главных идеалов. Определения простого и максимального идеала}

\begin{definition}
	$ A $ "--- коммутативное ассоциативное кольцо, $ \qquad I \sub A $ \\
	$ I $ называется \it{идеалом}, если:
	\begin{enumerate}
		\item $ I $ "--- подгруппа по сложению
		\item $ a \in I, ~ t \in A \quad \implies ta \in I $
	\end{enumerate}
\end{definition}

\begin{definition}
	$ A $ "--- ассоциативное коммутативное кольцо с единицей, $ \qquad S \sub A $ \\
	Идеалом, \it{порождённым} $ S $ называется минимальный по включению идеал, содержащий $ S $
\end{definition}

\begin{notation}
	$ \braket{S} $
\end{notation}

\begin{property}
	Идеал $ \braket{S} $ существует и единственный \\
	Он состоит из элементов вида $ t_1s_1 + \dots + t_ks_k, \qquad s_i \in S, \quad t_i \in A $
\end{property}

\begin{noproof}
\end{noproof}

\begin{definition}
	Идеал, порождённый одним элементом, называется \it{главным}
\end{definition}

\begin{definition}
	Если все идеалы главные, то $ A $ называется \it{кольцом главных идеалов}
\end{definition}

\begin{theorem}[примеры колец главных идеалов]
	\hfill
	\begin{enumerate}
		\item $ \Z $ "--- кольцо главных идеалов

		\item $ K $ "--- поле $ \implies K[x] $ "--- кольцо главных идеалов
	\end{enumerate}
\end{theorem}

\begin{eproof}
	\item $ I $ "--- идеал
	\begin{itemize}
		\item Если $ I = \set{0} $, то $ I = \braket{0} $ "--- главный
		\item Пусть $ I \ne \set{0}, \qquad a $ "--- наименьшее положительное число из $ I $ \\
		Докажем, что $ I = \braket{a} $: \\
		$ \braket{a} $ "--- множество чисел, делящихся на $ a $ \\
		\bt{Допустим}, что это не весь идеал, \ie $ \exist b : b \in I, \quad b \ndivby a $ \\
		Поделим с остатком:
		$$ b = aq + r, \qquad a < r < a $$
		$$ r = b - aq = \underbrace{b}_{\in I} + (-q)\underbrace{a}_{\in I} \in I \quad \contra $$
	\end{itemize}

	\item Аналогично, $ I = \braket{p} $, где $ p $ "--- многочлен наименьшей степени, лежащий в $ I $
\end{eproof}

\begin{definition}
	$ A $ "--- коммутативное ассоциативное кольцо, $ \qquad I $ "--- идеал \\
	$ I $ называется \it{простым}, если
	$$ \forall a, b \in A \quad ab \in I \implies a \in I \quad \text{ или } \quad b \in I $$
\end{definition}

\begin{definition}
	$ A $ "--- коммутативное ассоциативное кольцо, $ \qquad I $ "--- идеал \\
	$ I $ называется \it{максимальным}, если не существует такого идеала $ J $, что $ I \sub J, ~ J \ne I, ~ J \ne A $
\end{definition}

\section{Построение факторкольца. Факторкольцо по простому идеалу}

\begin{definition}
	$ A $ "--- коммутативное ассоциативное кольцо с единицей, $ \qquad I $ "--- идеал \\
	Элементы $ a $ и $ b $ называются \it{сравнимыми по модулю} $ I $, если $ a - b \in I $
\end{definition}

\begin{notation}
	$ a \equiv b \pmod I \qquad a \comp{I} b $
\end{notation}

\begin{property}
	$ \comp{I} $ является отношением эквивалентности
\end{property}

\begin{proof}
	\hfill
	\begin{enumerate}
		\item Рефлексивность:
		$$ a - a = 0 \in I $$
		\item Симметричность:
		$$ a - b \in I \implies b - a = -(a - b) \in I $$
		\item Транзитивность:
		$$ a - b \in I, b - c \in I \implies a - c = (a - b) + (b - c) \in I $$
	\end{enumerate}
\end{proof}

\begin{definition}
	$ A $ "--- коммутативное кольцо, $ \qquad I $ "--- идеал \\
	На множестве классов эквивалентности по отношению $ \comp{I} $ введём операции сложения и умножения:
	$$ \ol{x} + \ol{y} = \ol{x + y}, \qquad \ol{x} \cdot \ol{y} = \ol{xy} $$
\end{definition}

\begin{theorem}[факторкольцо]
	$ A $ "--- коммутативное ассоциативное кольцо, $ \qquad I $ "--- идеал \\
	Тогда
	\begin{enumerate}
		\item операции сложения и умножения на классах эквивалентности определены корректно, то есть не зависят от выбора представителей

		\item множество классов эквивалентности является ассоциативным коммутативным кольцом. Если в $ A $ была единица, то и в кольце классов эквивалентности будет единица
	\end{enumerate}
\end{theorem}

\begin{eproof}
	\item
	$$
	\begin{rcases}
		x_1, x_2 \text{ в одном классе} \\
		y_1, y_2 \text{ в одном классе}
	\end{rcases} \stackrel?\implies
	\begin{cases}
		x_1 + y_1 \text{ и } x_2 + y_2 \text{ в одном классе} \\
		x_1y_1 \text{ и } x_2y_2 \text{ в одном классе}
	\end{cases} $$
	Пусть $ x \define x_1 - x_2, \quad y \define y_1 - y_2 \quad \implies x, y \in I $
	$$ (x_1 + y_1) - (x_2 + y_2) = x + y \in I $$
	$$ x_1y_1 - x_2y_2 = (x + x_2)(y + y_2) - x_2y_2 = xy + y_2x + x_2y \in I $$

	\item $ \faktor{A}I $ "--- абелева группа (по т. о факторгруппе) \\
	Нужно доказать, что $ (\ol{x} + \ol{y})\ol{z} = \ol{x}\ol{z} + \ol{y}\ol{z} $ \\
	Выберем $ x \in \ol{x}, \quad y \in \ol{y}, \quad z \in \ol{z} $
	$$ (\ol{x} + \ol{y})\ol{z} = \ol{x}\ol{z} + \ol{y}\ol{z} \quad \impliedby \quad \ol{(x + y)z} = \ol{xz + yz} $$
	Остальное "--- аналогично \\
	Если $ A $ "--- кольцо с единицей, то $ \ol1 $ "--- единица в $ \faktor{A}I $
\end{eproof}

\begin{definition}
	Кольцо классов эквивалентности называется \it{факторкольцом} по идеалу $ I $
\end{definition}

\begin{notation}
	$ \faktor{A}{I} $
\end{notation}

\begin{theorem}[факторкольцо по простому идеалу]
	$ A $ "--- коммутативное ассоциативное кольцо, $ I $ "--- идеал.
	$$ I \text{ простой } \iff \faktor AI \text{ "--- область целостности} $$
\end{theorem}

\begin{proof}
	Пусть $ X \in \faktor{A}I, \quad x \in X $ \\
	Тогда $ X = 0 \quad \iff \quad \ol{x} = \ol0 \quad \iff \quad x \comp{I} 0 \quad \iff \quad x - 0 \in I \quad \iff \quad x \in I $
	\begin{itemize}
		\item $ \implies $

		Пусть $ X, Y \in \faktor{A}I, \qquad XY = \ol0 $ \\
		Пусть $ x \in X, \quad y \in Y \implies \ol{xy} = \ol0 \implies xy \in I \underimp{I \text{ простой}}
		\begin{vars}
			x \in I \implies X = \ol0 \\
			y \in I \implies Y = \ol0
		\end{vars} $

		\item $ \impliedby $

		Пусть $ xy \in I \implies \ol{xy} = \ol0 \implies \ol{x} \cdot \ol{y} = \ol0 \underimp{\text{обл. цел.}}
		\begin{vars}
			\ol{x} = 0 \implies x \in I \\
			\ol{y} = 0 \implies y \in I
		\end{vars} $
	\end{itemize}
\end{proof}

\section(Факторкольцо по максимальному идеалу. Факторкольцо кольца многочленов над полем){Факторкольцо по максимальному идеалу. Факторкольцо \\кольца многочленов над полем}

\begin{theorem}[факторкольцо по максимальному идеалу]
	\hfill \\
	$ A $ "--- коммутативное ассоциативное кольцо с единицей, $ \qquad I $ "--- идеал
	$$ I \text{ максимальный } \iff \faktor AI \text{ "--- поле} $$
\end{theorem}

\begin{iproof}
	\item $ \implies $

	$ \faktor{A}I $ "--- коммутативное ассоциативное кольцо с единицей \\
	Осталось доказать, что $ \forall X \in \faktor{A}I, ~ X \ne \ol0 \quad \exist X^{-1} $
	$$ \ol0 = I \implies X \ne I $$
	Пусть $ x \in X $ \\
	Пусть $ J \define \braket{x, I} $
	$$ J \supset I, ~ J \ne I \underimp{I \text{ "--- макс.}} J = A \implies 1 \in J $$
	$$ 1 \in \braket{I, x} \implies 1 = \underbrace{a_1s_1 + \dots + a_ks_k}_{\in I} + bx \text{ для некоторых } s_i \in I, \quad a_i, b \in A $$
	$$ \implies 1 \comp I bx \implies \ol1 = \ol b \cdot \ol x = \ol b X \implies \ol b = X^{-1} $$

	\item $ \impliedby $

	Пусть $ J $ "--- идеал, $ I \sub J, ~ I \ne J $ \\
	Докажем, что $ J = A $: \\
	Пусть $ x \in J \setminus I $
	$$ \ol x \in \faktor AI, \qquad \ol x \ne \ol0 \quad \implies \exist Y : \ol x Y = \ol 1 $$
	Пусть $ \ol y \in Y \implies \ol x \cdot \ol y = \ol 1 \implies xy - 1 \in I $
	$$
	\begin{rcases}
		x \in J \\
		xy - 1 \in I
	\end{rcases} \implies 1 = \underbrace{xy}_{\in J} - \underbrace{(xy - 1)}_{\in I} \in J \implies J = A $$
\end{iproof}

\begin{theorem}[факторкольцо кольца многочленов]
	$ K $ "--- поле, $ \qquad A = K[x], \qquad P(x) \in A $ \\
	$ I = \braket{P(x)} $ \nimp[(это не условие, а обозначение "--- известно, что все идеалы такие)], $ \qquad B = \faktor AI $

	$ P $ неприводим $ \iff \faktor AI $ "--- поле
\end{theorem}

\begin{proof}
	Правая часть равносильна тому, что $ I $ максимальный
	\begin{itemize}
		\item $ \implies $ \\
		Пусть $ I \sub J, \quad Q(x) $ "--- такой, что $ J = \braket{Q(x)} $
		\begin{multline*}
			\braket{P(x)} \sub \braket{Q(x)} \implies P(x) \divby Q(x) \underimp{P \text{ неприводимый}} \\
			\implies
			\begin{vars}
				Q(x) = cP(x), \quad c \in K, \quad c \ne 0 \implies J = I \\
				Q(x) = c, \quad c \in K, \quad c \ne 0 \implies J = A
			\end{vars} \implies I \text{ максимальный}
		\end{multline*}
		\item $ \impliedby $ \\
		\bt{Пусть} $ P $ приводим
		$$ \implies \exist Q(x) : \quad P(x) \divby Q(x), \qquad Q(x) \ne cP(x), \quad Q(x) \ne c $$
		$$ \implies \braket{P(x)} \subsetneq \braket{Q(x)} \subsetneq A \implies I \text{ не максимальный } $$
	\end{itemize}
\end{proof}

\section{Гомоморфизм колец. Теорема о гомоморфизме}

\begin{definition}
	$ (A, +_A, \cdot_A), ~ (B, +_B, \cdot_B) $ "--- кольца \\
	Отображение $ f : A \to B $ называется \it{гомоморфизмом}, если
	$$ f(x +_A y) = f(x) +_B f(y) $$
	$$ f(x \cdot_A y) = f(x) \cdot_B f(y) $$
\end{definition}

\begin{definition}
	Отображение $ f : A \to B $ называется \it{изоморфизмом}, если $ f $ "--- гомоморфизм и биекция.
\end{definition}

\begin{definition}
	Если существует изоморфизм из $ A $ в $ B $, то $ A $ и $ B $ называются \it{изоморфными}.
\end{definition}

\begin{notation}
	$ A \simeq B $
\end{notation}

\begin{definition}
	$ A $, $ B $ "--- кольцо, $ \qquad f : A \to B $ "--- гомоморфизм

	Ядро: $ \set{x \in A | f(x) = 0}, \qquad $ образ: $ \set{f(x) | x \in A} $.
\end{definition}

\begin{notation}
	$ \ker f, \qquad \Img f $.
\end{notation}

\begin{properties}
	$ A, B $ "--- коммутативные, $ \qquad f : A \to B $ - гомоморфизм
	\begin{enumerate}
		\item $ f(0) = 0 $

		\item $ \ker f $ "--- идеал

		\item $ \Img f $ "--- подкольцо $ B $
	\end{enumerate}
\end{properties}

\begin{eproof}
	\item Следует из аналогичного свойства для гомоморфизма групп

	\item $ \ker f \ne \O $, \as $ 0_A \in \ker f $
	\begin{itemize}
		\item $ x, y \in \ker f \implies f(x + y) = \underbrace{f(x)}_0 + \underbrace{f(y)}_0 = 0 + 0 = 0 $
		\item $ \underbrace{f(0)}_0 = f \big( x + (-x) \big) = \underbrace{f(x)}_0 + f(x) \implies f(-x) = 0 $
		\item $ a \in A \qquad f(ax) = f(a)f(x) = f(a) \cdot 0 = 0 $
	\end{itemize}

	\item $ \Img f \sub B $ \\
	Нужно преверить, что $ \Img f $ замкнут относительно операции \\
	Для сложения "--- можно сослаться на группы \\
	Для умножения:
	$$ x, y \in \Img f \implies a, b \in A : \quad f(a) = x, \quad f(b) = y $$
	$$ \implies xy = f(a)f(b) = f(ab) \in \Img f $$
\end{eproof}

\begin{theorem}[о гомоморфизме колец]
	$ A, B $ "--- коммутативные ассоциативные кольца \\
	$ \qquad f : A \to B $ "--- гомоморфизм

	$$ \faktor{A}{\ker f} \simeq \Img f $$
\end{theorem}

\begin{proof}
	Определим $ \vphi : \faktor{A}{\ker f} \to \Img f: \quad \vphi(X) = f(x) $ для некоторого $ x \in X $.
	\begin{itemize}
		\item Корректность: \\
		Пусть $ x, x' \in X $ \\
		Проверим, что $ f(x') = f(x) $
		$$ \ol x = \ol{x'} \implies x \comp{\ker f} x' \implies x - x' \in \ker f \implies f(x) = f \big( x' + (x - x') \big) = f(x') + \underbrace{f(x - x')}_{0 ~ (x - x' \in \ker f)} $$
		\item Гомоморфизм:
		$$ X, Y \in \faktor{A}{\ker f}, \qquad x \in X, \quad y \in Y $$
		$$ X = \ol x, \quad Y = \ol y, \qquad X + Y = \ol{x + y}, \quad XY = \ol{xy} $$
		$$ \vphi(X + Y) = \vphi(\ol{x + y}) = f(x + y) \undereq{f \text{ гомоморф.}} f(x) + f(y) = \vphi(\ol x) + \vphi(\ol y) = \vphi(\ol x + \ol y) $$
		Для умножения "--- то же самое
		\item Сюръективность: \\
		Пусть $ b \in \Img f \quad \implies \quad \exist x \in A : \quad f(x) = b \quad \implies \quad \vphi(\ol x) = b $
		\item Инъективность: \\
		Пусть $ \vphi(X) = \vphi(Y), \quad x \in X, ~ y \in Y $
		$$ \implies f(x) = f(y) \implies f(x - y) = 0 \implies x - y \in \ker f \implies \ol x \comp{\ker f} y \implies \ol x = \ol y \implies X = Y $$
	\end{itemize}
\end{proof}

\section{Характеристика кольца и поля. Классификация простых полей}

\begin{definition}
	$ A $ "--- кольцо \\
	\it{Характеристикой} $ A $ называется называется наименьшее $ n \in \N $ такое, что
	$$ \underbrace{a + a + \dots + a}_n = 0 \quad \forall a \in A $$
	Если такого $ n $ не существует, то характеристика равна нулю
\end{definition}

\begin{notation}
	$ \chara A $
\end{notation}

\begin{property}
	Если $ A $ кольцо с единицей, то $ \chara A $ "--- наименьшее $ n \in \N $ такое, что
	$$ \underbrace{1 + 1 + \dots + 1}_n = 0 $$
\end{property}

\begin{proof}
	Нужно доказать, что
	$$ \underbrace{a + a + \dots + a}_n = 0 \quad \forall a \in A \qquad \iff \qquad \underbrace{1 + 1 + \dots + 1}_n = 0 $$
	\begin{itemize}
		\item $ \implies $ \\
		Подставим $ a = 1 $
		\item $ \impliedby $
		$$ a + a + \dots + a = a(1 + \dots + 1) = a \cdot 0 = 0 $$
	\end{itemize}
\end{proof}

\begin{property}
	$ A $ "--- поле \\
	Тогда $ \chara A = 0 $ или $ \chara A \in \Prime $
\end{property}

\begin{proof}
	Пусть \bt{это не так} и $ \chara A $ "--- составное
	$$ \chara A = n = mk, \qquad 1 < m, \quad k < n $$
	$$ 0 = \underbrace{1 + \dots + 1}_n = (\underbrace{1 + \dots + 1}_m)(\underbrace{1 + \dots + 1}_k) \implies
	\begin{vars}
		\underbrace{1 + \dots + 1}_m = 0 \\
		\underbrace{1 + \dots + 1}_k = 0
	\end{vars} $$
	Получили противоречие с минимальностью $ n $
\end{proof}

\begin{definition}
	$ L $ "--- поле, $ \qquad K \sub L, \qquad K $ является полем с теми же операциями \\
	Тогда $ K $ называется \it{подполем} $ L, \qquad L $ называется \it{расширением} $ K $.
\end{definition}

\begin{definition}
	Поле $ K $ называется \it{простым}, если оно не содержит подполей, отличных от $ K $
\end{definition}

\begin{theorem}[классификация простых полей]
	\hfill
	\begin{enumerate}
		\item Поля $ \Q $ и $ \Z_p $ при $ p \in \Prime $ "--- простые

		\item Любое простое поле изоморфно $ \Q $ или $ \Z_p $ для некоторого $ p \in \Prime $
	\end{enumerate}
\end{theorem}

\begin{eproof}
	\item
	\begin{itemize}
		\item $ \Q $ \\
		Пусть $ \Q $ \bt{не простое}, и $ K $ "--- подполе $ \Q \quad \implies 0, 1 \in K $
		$$ \underbrace{1 + 1 + \dots + 1}_n \in K \quad \forall n \quad \implies \N \sub K $$
		Если $ n \in K $, то $ -n \in K \quad \implies \Z \sub K $ \\
		Если $ n \in K, ~ n \ne 0 $, то $ \frac1n \in K \quad \implies \frac1n \in K \quad \forall n \in \N $
		$$ m \in \Z, ~ n \in \N \implies \frac mn = m \cdot \frac1n \in K \quad \implies \Q = K $$
		\item $ \Z_p $ \\
		Аналогично, пусть $ K $ "--- подполе $ \Z_p $
		$$ \ol1 \in K $$
		$$ \underbrace{\ol1 + \ol1 + \dots + \ol1}_n \in K \quad \forall n \quad \implies \ol n \in K \quad \forall n \quad \implies \Z_p = K $$
	\end{itemize}

	\item Пусть $ K $ "--- поле \\
	Докажем, что $ K $ содержит подполе, изоморфное $ \Q $ или $ \Z_p $ \\
	Возьмём $ A $ "--- минимальное подкольцо $ K $, содержащее 1 \\
	Докажем, что $ A \simeq \Z $ (взяв все частные из $ A $, получим множество дробей) или $ A \simeq \Z_p $: \\
	Пусть $ f : \Z \to A $ такое, что
	$$ f(n) \define
	\begin{cases}
		\underbrace{1 + 1 + \dots + 1}_n, \qquad n > 0 \\
		-(\underbrace{1 + 1 + \dots + 1}_n), \qquad n < 0 \\
		0, \qquad n = 0
	\end{cases} $$
	Докажем, что $ f $ "--- гомоморфизм:
	\begin{itemize}
		\item Докажем, что $ f(n) + f(k) = f(n + k) $: \\
		Кольцо "--- это группа по сложению. Умножение $ n $ единиц "--- это возведение в $ n $ степень. Знаем, что $ 1^n * 1^k = 1^{n + k} $, где $ * $ "--- это $ + $
		\item $ f(nk) = f(n) \cdot f(k) $:
		\begin{itemize}
			\item $ n, k > 0 $
			$$ (\underbrace{1 + \dots + 1}_n)(\underbrace{1 + \dots + 1}_k) = \underbrace{1 \cdot 1 + \dots + 1 \cdot 1}_{nk} = \underbrace{1 + \dots + 1}_{nk} $$
			\item $ n = 0 $
			$$ f(0) = f(0) f(k) $$
			\item $ n > 0, ~ k < 0 $ \\
			Положим $ k_1 \define -k $
			$$ f \bigg( n(-k_1) \bigg) = f(n) f(-k_1) \quad \impliedby \quad -f(nk_1) = f(n) \bigg( -f(k_1) \bigg) $$
		\end{itemize}
	\end{itemize}
	По теореме о гомоморфизме $ \Img f \simeq \faktor\Z{\ker f} $ \\
	$ \Img f $ "--- подкольцо $ A \quad \implies \Img f = A $ (из минимальности $ A $) \\
	$ \ker f $ "--- идеал $ \implies \ker f = \braket m $
	\begin{itemize}
		\item $ m = 0 $
		$$ \ker f = \set0 \implies \faktor\Z{\ker f} = \faktor\Z{\set0} \simeq \Z $$
		\item $ m \ne 0 $
		$$ \Img f \simeq \faktor\Z{\braket m} \simeq \Z_m $$
		$ \Img f $ "--- подкольцо поля $ K \implies \Img f $ "--- область целостности \\
		$ \implies \braket m $ "--- простой идеал $ \implies m \in \Prime $
	\end{itemize}
\end{eproof}

\section{Степень расширения. Мультипликативность степени, следствия}

\begin{lemma}[корректность]
	$ K $ "--- поле, $ \qquad L $ "--- расширение $ K $ \\
	Тогда $ L $ является векторным пространством над $ K $
\end{lemma}

\begin{iproof}
	\item Операции:
	\begin{itemize}
		\item $ l_1 + l_2, \quad l_1, l_2 \in L $
		\item $ kl, \quad k \in K, ~ l \in L $
	\end{itemize}
	$ k, l $ "--- элементы $ L $, для них операции определены
	\item $ L $ "--- абелева группа по сложению:
	$$ (k_1k_2)l = k_1(k_2l) $$
\end{iproof}

\begin{definition}
	$ L $ "--- расширение $ K $ \\
	Степенью расширения $ L $ над $ K $ называется $ \dim_K L $
\end{definition}

\begin{notation}
	$ |L : K|, \qquad (L : K), \qquad [L : K] $
\end{notation}

\begin{theorem}[мультипликативность степени]
	$ K \sub M \sub L $ "--- поля с общими операциями \\
	Тогда $ |L : K| = |L : M| \cdot |M : K| $
\end{theorem}

\begin{iproof}
	\item Докажем, что если $ e_1, \dots, e_r \in M $ ЛНЗ над $ K $ и $ f_1, \dots, f_s \in L $ ЛНЗ над $ M $, то $ g_{ij} \define e_if_j $ ЛНЗ над $ K $: \\
	Пусть $ a_{ij} \in K : \sum a_{ij}g_{ji} = 0 $
	$$ a_{11}e_1f_1 + a_{12}e_1f_2 + \dots + a_{21}e_2f_1 + a_{22}e_2f_2 + \widedots[4em] = 0 $$
	Сгруппируем по элементам $ f $:
	$$ \bigg( a_{11}e_1f_1 + a_{21}e_2f_1 + \dots \bigg) + \bigg( a_{12}e_1f_2 + a_{22}e_2f_2 + \dots \bigg) + \widedots[4em] = 0 $$
	$$ \underbrace{(a_{11}e_1 + a_{21}e_2 + \dots)}_{\in M}f_1 + \underbrace{(a_{12}e_1 + a_{22}e_2 + \dots)}_{\in M}f_2 + \widedots[4em] = 0 $$
	Пусть $ b_j \define a_{1j}e_1 + a_{2j}e_2 + \dots + a_{rj}e_r $ \\
	Тогда $ b_j \in M, \quad b_1f_1 + \dots b_sf_s = 0 $ \\
	$ f_1, \dots f_s $ ЛНЗ над $ M \implies b_1 = b_2 = \dots = b_s = 0 $
	$$ a_{1j}e_1 + \dots + a_{rj}e_r = b_j = 0 $$
	$ e_1, \dots, e_r $ ЛНЗ над $ K \implies a_{ij} = 0 \quad \forall i, j $

	\item Конечный случай \\
	Пусть $ e_1, \dots, e_r $ "--- базис $ M $ над $ K, \quad f_1, \dots, f_s $ "--- базис $ L $ над $ M $ \\
	Докажем, что $ g_{ij} \define e_if_j $ "--- базис $ L $ над $ K $: \\
	ЛНЗ уже доказана. Осталось доказать, что любой элемент порождается $ g_{ij} $: \\
	Пусть $ c \in L \quad \implies \exist b_i \in M : \quad c = b_1f_1 + \dots + b_sf_s $
	$$ b_j \in M, \quad e_i \text{ порожд. } M \text{ над } K \implies \forall j \quad \exist a_{ij} : \quad b_j = a_{1j}e_1 + \dots + a_{rj}e_r $$
	$$ \implies c = \sum a_{ij}e_if_j = \sum a_{ij}g_{ij} $$

	\item Бесконечный случай \\
	Нужно доказать, что $ \forall N \quad \exist N $ ЛНЗ элементов $ L $ над $ K $ (\ie существует сколь угодно большая ЛНЗ система) \\
	Можно выбрать $ e_1, \dots e_N $ ЛНЗ, или $ f_1, \dots f_N $ ЛНЗ \\
	Тогда $ e_if_j $ ЛНЗ над $ K $
\end{iproof}

\begin{implication}
	$ L $ "--- конечное расширение над $ K, \qquad K \sub M \sub L $ \\
	Тогда $ |M : K| $ и $ |L : M| $ "--- делители $ |L : K| $
\end{implication}

\begin{implication}
	$ L $ "--- конечное расширение $ K, \qquad |L : K| $ "--- простое число
	$$ \implies \not\exist M : \quad K \sub M \sub L, \quad M \ne K, ~ M \ne L $$
\end{implication}

\begin{implication}
	$ K \sub M \sub L $

	\begin{itemize}
		\item если $ |M : K| = |L : K| $, то $ M = L $
		\item если $ |L : M| = |L : K| $, то $ M = K $
	\end{itemize}
\end{implication}

\begin{implication}
	$ K \sub M \sub L, \qquad L $ бесконечно над $ K $ \\
	Тогда $ M $ бесконечно над $ K $ или $ L $ бесконечно над $ M $
\end{implication}

\section{Минимальный многочлен алгебраического элемента. Алгебраичность конечного расширения}

\begin{definition}
	$ L $ "--- расширение $ K, \qquad \alpha \in L $ \\
	$ \alpha $ называется \it{алгебраическим} над $ K $, если $ \exist P(x) \in K[x] $ такой, что $ P(\alpha) = 0, \quad P(x) $ "--- не нулевой. \\
	Если такого $ P(x) $ не существует, то $ \alpha $ называется \it{трансцендентным}.
\end{definition}

\begin{definition}
	$ \alpha $ "--- алгебраическое над $ K, \qquad P(x) \in K[x], \qquad P(\alpha) = 0 $. \\
	Тогда говорят, что $ P(x) $ \it{аннулирует} $ \alpha $.

	\it{Минимальным многочленом} $ \alpha $ над $ K $ называется ненулевой аннулирующий многочлен наименьшей степени со старшим коэффициентом, равным 1
\end{definition}

\begin{definition}
	\it{Алгебраическим числом} называется комплексное число, алгебраическое над $ \Q $
\end{definition}

\begin{properties}[минимального многочлена]
	$ K $ "--- поле, $ \qquad L $ "--- расширение $ K, \qquad \alpha \in L, \qquad \alpha $ алг. над $ K $
	\begin{enumerate}
		\item \label{en:prop:min_poly:1} $ P(x) $ "--- минимальный для $ \alpha $.
		$$ F(\alpha) = 0 \quad \iff \quad F(x) \divby P(x) $$

		\item Минимальный многочлен для $ \alpha $ единственен

		\item Минимальный многочлен неприводим над $ K $

		\item $ P(x) $ неприводим над $ K, \quad P(x) \ne 0, \quad P(\alpha) = 0 $
		$$ \implies P(x) \text{ "--- минимальный для } \alpha $$
	\end{enumerate}
\end{properties}

\begin{eproof}
	\item
	$$ F(x) = P(x)Q(x) + R(x), \qquad \deg R < \deg P $$
	\begin{itemize}
		\item $ \impliedby $
		$$ F(x) \divby P(x) \implies R(x) = 0 $$
		$$ F(x) = P(x)Q(x) $$
		Подставим $ \alpha $:
		$$ F(\alpha) = \underbrace{P(\alpha)}_0Q(x) = 0 $$
		\item $ \implies $
		$$ \underbrace{P(\alpha)}_0Q(\alpha) + R(\alpha) = 0 $$
		$$ R(\alpha) = 0 \implies R \text{ "--- нулевой} $$
	\end{itemize}

	\item Пусть $ P_1 P_2 $ "--- минимальные
	$$ \underimp{\text{св-во \ref{en:prop:min_poly:1}}}
	\begin{cases}
		P_1(x) \divby P_2(x) \\
		P_2(x) \divby P_1(x)
	\end{cases} \quad \implies P_1(x) = P_2(x) $$

	\item Пусть $ P(x) = S(x)T(x), \quad 0 < \deg S, \deg T < \deg P $
	$$ 0 = P(\alpha) = \underbrace{S(\alpha)}_{\in L}\underbrace{T(\alpha)}_{\in L} \underimp{L \text{ "--- поле}}
	\begin{vars}
		S(\alpha) = 0 \\
		T(\alpha) = 0
	\end{vars} \quad \contra \quad \text{миним. } \deg P $$

	\item
	$$
	\begin{rcases}
		P(x) \divby \text{ миним.} \\
		P(x) \text{ "--- непривод.}
	\end{rcases} \implies P(x) \text{ "--- миним.} $$
\end{eproof}

\begin{definition}
	Расширение $ L $ над $ K $ называется алгебраическим, если любой элемент $ L $ является \it{алгебраическим} над $ K $. \\
	Иначе "--- \it{трансцендентным}.
\end{definition}

\begin{theorem}
	Конечное расширение полей является алгебраическим
\end{theorem}

\begin{proof}
	Пусть $ L $ "--- конечное расширение $ K, \quad n \define |L : K|, \quad \alpha \in L $. \\
	Докажем, что $ \alpha $ "--- алгебраическое: \\
	Элементы $ \underbrace{1, \alpha, \dots, \alpha^{n - 1}, \alpha^n}_{n + 1} \in L $ ЛЗ над $ K $, \ie
	$$ \exist k_0, k_1, \dots, k_{n - 1} k_n \in K \nin \bigodot : \quad k_0 \cdot 1 + k_1 \alpha + \dots + k_{n - 1}\alpha^{n - 1} + k_n\alpha^n = 0 $$
	Пусть $ P(x) = k_0 + k_1x + \dots + k_{n - 1}x^{n - 1} + k_nx^n $. \\
	Тогда $ P(x) \in K[x], \quad P(x) $ "--- ненулевой, $ \quad P(\alpha) = 0 \quad \implies \alpha $ "--- алгебраичсекое.
\end{proof}

\section{Строение простого алгебраического расширения. Следствия}

\begin{definition}
	$ L $ "--- поле, $ \qquad K $ "--- подполе $ L, \qquad \alpha_1, \dots \alpha_n \in L $ \\
	Через $ K(\alpha_1, \dots, \alpha_n) $ будем обозначать наименьшее подполе $ L $, содержащее $ K $ и $ \alpha_1, \dots, \alpha_n $. \\
	Если $ M = K(\alpha_1, \dots \alpha_n) $, то говорят, что $ M $ получено из $ K $ \it{присоединением} $ \alpha_1, \dots, \alpha_n $. \\
	Поле, полученное из $ K $ присоединением одного элемента, называется \it{простым расширением} $ K $.
\end{definition}

\begin{theorem}[строение простого алгебраического расширения]
	$ L $ "--- поле, $ \qquad K $ "--- подполе $ L $ \\
	$ \alpha \in L, \qquad \alpha $ алг. над $ K, \qquad P(x) $ "--- минимальный многочлен для $ \alpha $ над $ K $
	\begin{enumerate}
		\item $ \faktor{K[x]}{\braket{P(x)}} \simeq K(\alpha), \qquad \ol{F(x)} \mapsto F(\alpha) $ является изоморфизмом.
		\item $ K(\alpha) $ конечно над $ K, \qquad |K(\alpha) : K| = \deg P, \qquad 1, \alpha, \dots, \alpha^{n - 1} $ образуют базис $ K(\alpha) $ над $ K $.
	\end{enumerate}
\end{theorem}

\begin{proof}
	Определим $ f : K[x] \to K(\alpha) $ как $ f(F) \define F(\alpha) $ ($ x \mapsto \alpha $), \ie
	$$ f(c_0 + c_1x + \dots c_kx^k) = c_0 + c_1\alpha + \dots + c_k\alpha^k, \qquad c_i \in K $$
	\begin{itemize}
		\item Проверим, что $ f $ "--- гомоморфизм:
		$$ f(F + G) = (F + G)(\alpha) = F(\alpha) + G(\alpha) = f(F) + f(G) $$
		$$ f(FG) = (FG)(\alpha) = F(\alpha)G(\alpha) = f(F)f(G) $$

		\item Найдём $ \ker f $:
		$$ F(x) \in \ker f \iff f(F) = 0 \iff F(\alpha) = 0 \iff F(x) \divby P(x) $$
		$$ \implies \ker f = \braket{P(x)} $$

		\item Применим теорему о гомоморфизме:
		$$ \Img f \simeq \faktor{K[x]}{\ker f} $$
		Изоморфизм $ \vphi(\ol F) = f(F) = F(\alpha) $ \\
		Получили изоморфизм $ \faktor{K[x]}{\braket{P(x)}} \to \Img f $

		\item Проверим, что $ \Img f \iseq K(\alpha) $:

		$ K(\alpha) $ "--- минимальное, содержащее $ K $ и $ \alpha $. \\
		$ \Img f $ тоже содержит $ K $ и $ \alpha $. \\
		Значит, $ \Img f = K(\alpha) $.

		\item Проверим, что $ 1, \alpha, \dots, \alpha^{\deg P - 1} $ "--- базис:

		Пусть $ n \define \deg P $
		\begin{itemize}
			\item ЛНЗ:

			\bt{Пусть} ЛЗ:
			$$ a_0 \cdot 1 + a_1 \alpha + \dots + a_{n - 1}\alpha^{n - 1} = 0, \quad a_i \in K $$
			Пусть $ F(x) \define a_0 + a_1x + \dots + a_{n - 1}x^{n - 1} \implies F(\alpha) = 0 $
			$$ \implies F(x) \divby P(x) \underimp{F(x) \text{ "--- ненулевой}} \deg F \ge \deg P = n \quad \contra $$

			\item Порождающий:
			$$ K(\alpha) = \Img f $$
			Пусть $ u \in K(\alpha) \implies \exist F \in K[x] : \quad f(F) = u \quad \implies F(\alpha) = u $ \\
			Делим с остатком:
			$$ F(x) = Q(x)P(x) + R(x), \qquad \deg R < \deg P $$
			$$ \implies \deg R \le n + 1 $$
			$$ F(\alpha) = Q(\alpha)\underbrace{P(\alpha)}_0 + R(\alpha) = R(\alpha) $$
			$$ R(x) = a_0 + a_1x + \dots a_{n - 1}x^{n - 1} \implies F(\alpha) = a_0 + a_1\alpha + \dots a_{n - 1}\alpha^{n - 1} $$
		\end{itemize}
	\end{itemize}
\end{proof}

\begin{implication}
	$ \alpha $ "--- алгебраический над $ K, \qquad F, G \in K[x], \qquad G(\alpha) \ne 0, \qquad \beta = \frac{F(\alpha)}{G(\alpha)} $ \\
	Тогда $ \beta $ "--- алгебраический над $ K $
\end{implication}

\begin{proof}
	Существует поле $ K(\beta) $. \\
	При этом $ \beta \in K(\alpha) $.
	$$ K \sub K(\beta) \sub K(\alpha) $$
	Применим одно из следствий из теоремы о мультипликативности расширения: \\
	$ K(\alpha) $ над $ K $ конечно $ \implies K(\beta) $ над $ K $ конечно \\
	$ \implies $ все элементы $ K(\beta) $ алгебраичны над $ K $
\end{proof}

\begin{implication}
	$ \alpha_1, \dots, \alpha_n $ алгебраичны над $ K $ \\
	Тогда $ K(\alpha_1, \dots, \alpha_n) $ конечно над $ K $
\end{implication}

\begin{proof}
	Обозначим $ K_0 \define K, \quad K_i \define K(\alpha_1, \dots, \alpha_i) $.
	$$ K \sub K_1 \sub K_2 \sub \dots \sub K_n $$
	Достаточно доказать, что $ K_{i + 1} $ кончено над $ K_i $: \\
	Пусть $ P(x) $ "--- ненулевой многочлен, аннулирующий $ \alpha_{i + 1} $ над $ K_i $. \\
	Выполнено $ K_i[x] \sub K[x] $, следовательно, $ P(x) \in K_i[x] $. \\
	Получаем, что $ P(x) $ аннулирует $ \alpha_{i + 1} $ над $ K_i $, и $ \alpha_{i + 1} $ алгебраичен над $ K_i $. \\
	Расширение, полученное присоединением одного алгебраического элемента, конечно.
\end{proof}

\begin{implication}
	$ \alpha, \beta $ алгебраичны над $ K $ \\
	$ \implies \alpha + \beta, \quad \alpha - \beta, \quad \alpha \beta, \quad \faktor\alpha\beta $ алгебраичны над $ K $
\end{implication}

\begin{proof}
	Все эти элементы принадлежат конечному расширению $ K(\alpha, \beta) $ над $ K $, следовательно, они являются алгебраическими.
\end{proof}

\section{Существование простого расширения. Эквивалентные расширения}

\begin{theorem}[существование простого расширения]
	$ K $ "--- поле, $ \qquad P(x) \in K[x] $ "--- неприводимый. \\
	Тогда существует расширение $ L $ поля $ K $ такое, что $ P(x) $ имеет в $ L $ корень $ \alpha $ и $ L = K(\alpha) $.
\end{theorem}

\begin{proof}
	Рассмотрим множество формальных сумм вида
	$$ a_0 + a_1X + a_2X^2 + \dots + a_nX^n, \quad a_i \in K $$
	Введём отношение эквивалентности: \\
	Если
	$$ s = a_0 + a_1X + \dots, \qquad t = b_0 + b_1X + \dots $$
	$$ S(x) \define a_0 + a_1x + \dots, \qquad T(x) \define b_0 + b_1x + \dots $$
	и $ S(x) - T(x) \divby P(x) $, то $ s \sim t $. \\
	Определим на множестве классов элквивалентности сложение и умножение: \\
	Если
	$$ s = a_0 + a_1X + \dots, \qquad t = b_0 + b_1X + \dots, \qquad u = c_0 + c_1X + \dots $$
	$$ S(x) = a_0 + a_1x + \dots, \qquad T(x) = b_0 + b_1x + \dots, \qquad U(x) = c_0 + c_1x + \dots $$
	и $ S(x)T(x) - U(x) \divby P(x) $, то положим $ u \define st $. \\
	Сложение "--- аналогично. \\
	Получается поле, изоморфное $ \faktor{K[x]}{\braket{P(x)}} $ \\
	Изоморфизм: $ \ol{a_0 + a_1X + \dots} \mapsto \ol{a_0 + a_1x + \dots} $ \\
	$ \ol X $ подойдёт в качестве $ \alpha $ (\as $ P(x) \mapsto \ol{P(x)} = 0 $).
\end{proof}

\begin{definition}
	Расширения $ L_1, L_2 $ поля $ K $ называются \it{эквивалентными} \nimp[(относительно $ K $)], если $ L_1 \simeq L_2 $ и существует изоморфизм $ f : L_1 \to L_2 $ такой, что $ f\clamp{K} = \operatorname{id} $.
\end{definition}

\begin{theorem}[эквивалентные простые расширения]
	$ \alpha, \beta $ "--- алгебраические над $ K $, их минимальные многочлены совпадают. \\
	Тогда $ K(\alpha) $ и $ K(\beta) $ эквивалентны, причём существует изоморфизм $ f : K(\alpha) \to K(\beta) $ такой, что
	$$ f\clamp{K} = \operatorname{id}, \quad f(\alpha) = \beta $$
\end{theorem}

\begin{proof}
	Пусть $ P(x) $ "--- минимальный многочлен для $ \alpha $ и $ \beta, \quad n \define \deg P $. \\
	Элементы $ K(\alpha) $ "--- это $ u_0 + u_1\alpha + \dots + u_{n - 1}\alpha^{n - 1} $. \\
	Положим
	$$ f(u_0 + u_1 \alpha + \dots + u_{n - 1}\alpha^{n - 1}) \define u_0 + u_1\beta + \dots + u_{n - 1}\beta^{n - 1} $$
	Пусть
	$$ s = u_0 + u_1\alpha + \dots, \qquad t = v_0 + v_1\alpha + \dots $$
	$$ S(x) = u_0 + u_1 x + \dots, \qquad T(x) = v_0 + v_1x + \dots $$
	Пусть $ R(x) = w_0 + w_1x + \dots + w_{n - 1}x^{n - 1} $ "--- такой, что $ S(x)T(x) - R(x) \divby P(x) $
	$$ r = w_0 + w_1\alpha + \dots + w_{n - 1}\alpha^{n - 1} $$
	Тогда $ s = S(\alpha), \quad t = T(\alpha), \quad r = R(\alpha) $
	$$ f(s) = S(\beta), \qquad f(t) = T(\beta), \qquad f(r) = R(\beta) $$
	$$ st = S(\alpha) T(\alpha) \undereq{ST - R \divby P} R(\alpha) = r^2 $$
	$$ f(ST) = f(r) = R(\beta) $$
	$$ f(s)f(t) = S(\beta)T(\beta) = R(\beta) $$
	Сложение "--- аналогично. \\
	Биективность:
	\begin{itemize}
		\item Инъективность:
		$$ u_0 + u_1\alpha + \dots \to 0 $$
		$$ u_0 + u_1\beta + \dots = 0 $$
		$$ \implies u_i = 0 $$
		\item Сюръективность: \\
		Любой элемент $ K(\beta) $ "--- это $ u_0 + u_1\beta + \dots $
	\end{itemize}
\end{proof}

\section{Поле разложения многочлена: существование, эквивалентность}

\begin{definition}
	$ K $ "--- поле, $ \qquad P(x) \in K[x] $. \\
	\it{Полем разложения} $ P(x) $ называется такое расширение $ L $ поля $ K $, что в $ L $ многочлен $ P(x) $ раскладывается на линейные множители
	$$ P(x) = a(x - \alpha_1)(x - \alpha_2) \cdots (x - \alpha_n), \qquad a \in K, \quad \alpha_i \in L $$
	и $ L = K(\alpha_1, \dots, \alpha_n) $.
\end{definition}

\begin{theorem}[существование поля разложения]
	$ K $ "--- поле, $ \qquad P(x) \in K[x] $.
	\begin{enumerate}
		\item существует поле разложения;
		\item любое поле разложения является конечным расширением $ K $.
	\end{enumerate}
\end{theorem}

\begin{proof}
	Будем считать, что старший коэффициент $ P $ равен 1. \\
	\bt{Индукцией} по $ n $ (не фиксируя $ K $) докажем, что для любого $ n $ выполнено утверждение:
	\begin{quote}
		Для любого $ K $, для любого многочлена степени не выше $ n $ существует поле $ M $, в котором $ P(x) $ раскладывается на линейные множители
	\end{quote}
	\begin{itemize}
		\item \bt{База}. $ n = 1 $ \\
		$ P(x) $ "--- линейный, есть корень в $ K, \quad M = K $
		\item \bt{Переход} к $ n $: \\
		Разложим $ P(x) $ на неприводимые над $ K $:
		$$ P(x) = P_1(x)\dots P_k(x) $$
		Присоединим корень $ \alpha $ многочлена $ P_1(x) $, получим $ K(\alpha) $. \\
		$ K(\alpha) $ "--- поле, в нём верна теорема Безу:
		$$ P_1(x) \divby x - \alpha \text{ в } K(\alpha)[x] $$
		$$ P_1(x) = (x - \alpha)Q(x) $$
		$$ P(x) = (x- \alpha)\underbrace{Q(x)P_2(x)\dots P_k(x)}_{H(x)} = (x - \alpha)H(x) $$
		Применим \bt{предположение индукции} к $ K(\alpha) $ и $ H(x) $: \\
		Существует $ M $, в котором $ H(x) $ раскладывается на линейные множиетели, $ K(\alpha) \sub M $. \\
		Это $ M $ подходит для $ K $ и $ P(x) $. \\
		Поле разложения "--- минимальное подполе $ M $, содержащее $ K $ и $ \alpha_1, \dots, \alpha_n $, \ie $ L = K(\alpha_1, \dots, \alpha_n) $.
	\end{itemize}
\end{proof}

\begin{theorem}[эквивалентность полей разложения многочлена]
	$ K $ "--- поле, $ \qquad P \in K[x] $ \\
	$ L, M $ "--- поля разложения.
	\begin{enumerate}
		\item $ L $ и $ M $ эквивалентны над $ K $;
		\item можно выбрать такие $ \alpha_i \in L, \quad \beta_i \in M $ такие, что
		$$ P(x) = \underset{\in K}a(x - \alpha_1) \cdots (x - \alpha_n), \qquad P(x) = \underset{\in K}b(x - \beta_1) \cdots (x - \beta_n) $$
		для которых существует изоморфизм $ f : L \to M, \quad f(\alpha_i) = \beta_i, \quad f\clamp K = \operatorname{id} $
	\end{enumerate}
\end{theorem}

\begin{proof}
	Строим последовательно $ \alpha_1, \dots, \alpha_s, \quad \beta_1, \dots, \beta_s $.
	$$ f_s : K(\alpha_1, \dots, \alpha_s) \to K(\beta_1, \dots, \beta_s) : \quad f(\alpha_i) = \beta_i $$
	Пусть построены $ \alpha_1, \dots, \alpha_s, \quad \beta_1, \dots, \beta_s, \quad f_s $. \\
	Положим $ L' = K(\alpha_1, \dots, \alpha_s), \quad M' = K(\beta_1, \dots, \beta_s) $ \\
	(на первом шаге считаем, что $ L' = M' = K, \quad f_0 = \operatorname{id} $) \\
	Разложение $ P(x) $ на неприводимые над $ L' $
	$$ P(x) = (x - \alpha_1) \cdots (x - \alpha_s) Q_1(x)Q_2(x) \cdots $$
	$$ f_s = L' \to M' $$
	$$ P(x) = f_s(P(x)) = (x - \beta_1) \cdots (x - \beta_s)R_1(x)R_2(x) \cdots, \qquad R_i(x) = f_s(Q_i(x)) $$
	$ R_i(x) $ неприводимы
	\begin{itemize}
		\item Если $ Q_i(x) $ "--- линейный, обозначим его корень $ \alpha_{s + 1}, \quad \beta_{s + 1} \define f(\alpha_{s + 1}) $
		$$ f_s \big( Q_i(x) \big) = (x - \beta_{s + 1}) $$
		$$ f_{s + 1} \define f_s $$
		\item Если нет линейных, то положим $ \alpha_{s + 1} $ "--- корень $ Q_1(x), \quad \beta_{s + 1} $ "--- корень $ R_1(x) $
		$$ L'(\alpha_{s + 1}) \simeq \faktor{L'(x)}{\braket{Q_1(x)}} \simeq \faktor{M'(x)}{\braket{R_1(x)}} \simeq M'(\beta_{s + 1}) $$
		Отображение
		$$ \vphi : \faktor{L'[x]}{\braket{Q_1(x)}} \to \faktor{M'[x]}{\braket{R_1(x)}} : \qquad \vphi \big( \ol H(x) \big) = \ol{f_{s + 1} \big( H(x) \big)} $$
		является изоморфизмом. Существуют изоморфизмы
		$$ \vphi_1 : \faktor{L'[x]}{\braket{Q_1(x)}} \to L'(\alpha_{s + 1}) : \quad \vphi_1(\ol x) = \alpha_{s + 1} $$
		$$ \vphi_2 : \faktor{M'[x]}{\braket{R_1(x)}} \to M'(\beta_{s + 1}) : \quad \vphi_2(\ol x) = \beta_{s + 1} $$
		Отображение $ \vphi_1^{-1} \circ \vphi \circ \vphi_2 $ подойдёт в качестве $ f_{s + 1} $.
	\end{itemize}
\end{proof}

\section{Свойства корней из единицы. Существование примитивного корня}

\begin{definition}
	$ K $ "--- поле, $ \qquad \veps \in K, \qquad n \in \N $ \\
	$ \veps $ называется \it{корнем} $ n $-й степени \it{из единицы}, если $ \veps^n = 1 $. \\
	$ \veps $ "--- \it{примитивный} корень степени $ n $, если $ \veps^n = 1, \quad \veps^k \ne 1 $ при $ 1 \le k < n $
\end{definition}

\begin{props}
	\item Корни $ n $-й степени из 1 образуют абелеву группу по умножению

	\item $ \chara K = p \in \Prime \ne 0, \qquad n = p^mh, \quad h \ndivby p, \qquad \veps $ "--- корень $ n $-й степени из 1. \\
	Тогда $ \veps $ "--- корень $ h $-й степени из 1.
\end{props}

\begin{eproof}
	\item Пусть $ U $ "--- множество корней $ n $-й степени.
	\begin{itemize}
		\item $ \veps_1, \veps_2 \in U \implies (\veps_1\veps_2)^n = \veps_1^n\veps_2^n = 1 \cdot 1 = 1 \implies \veps_1\veps_2 \in U $
		\item $ \veps \in U \implies \bigg( \frac1\veps \bigg)^n = \frac{1^n}{\veps^n} = \frac11 = 1 \implies \veps^{-1} \in U $
	\end{itemize}

	\item Докажем, что если $ \veps^{ps} = 1 $, то $ \veps^s = 1 $:
	$$ C_p^i = \frac{p!}{(p - i)! \cdot i!} \quad \divby p \text{ при } 1 \le i \le p - 1 \text{ в } \Z $$
	(\as $ p! \divby p, \quad (p - i)! \cdot i! \ndivby p $)
	$$ \chara K = p \implies C_p^i = 0 \text{ при } 1 \le i \le p $$
	$$ (\veps^s - 1)^p = (\veps^s)^p + 0 \cdot (\veps^s)^{p - 1} \cdot (-1) + \dots + 0 \cdot \veps^s \cdot (-1)^{p - 1} + (-1)^p = \veps^{sp} - 1 = 1 - 1 = 0 \underimp{\text{обл. цел.}} \veps^s = 1 $$
\end{eproof}

\begin{theorem}[существование примитивного корня]
	$ K $ "--- поле, $ \qquad h \in \N $ \\
	$ x^h - 1 $ раскладывается в $ K $ на линейные множители, $ \qquad h \ndivby \chara K $

	\begin{enumerate}
		\item в $ K $ есть $ h $ различных корней $ n $-й степени из единицы;
		\item существует примитивный корень $ h $-й степени из единицы;
		\item группа корней $ h $-й степени является циклической и порождается любым примитивным корнем.
	\end{enumerate}
\end{theorem}

\begin{eproof}
	\item $ P(x) = x^h - 1 $ имеет $ h $ корней с учётом кратности \\
	$ P'(x) = hx^{h - 1} $: единственный корень "--- 0 "--- не является корнем $ p(x) $
	\item $ U $ "--- группа корней $ h $-й степени из единицы, $ \quad |U| = h $

	Нужно доказать, что $ \exist \veps \in U : \quad \ord \veps = h $

	Пусть $ h = p_1^{a_1} \cdot \dots \cdot p_k^{a_k}, \quad p_i \in \Prime $

	Докажем, что $ \exist x_1, \dots, x_k \in U : \quad \ord (x_i) = p_i^{a_i} $:

	Докажем для $ i = 1 $ (остальное "--- аналогично):
	$$ x_1 : \ord x_1 \iseq p_1^{a_1} $$
	Докажем, что $ \exist y : \quad \ord y \divby p_1^{a_1} $:

	\bt{Пусть} $ \forall y \in U \quad \ord y \ndivby p_1^{a_1} $
	$$
	\begin{rcases}
		p_1^{a_1}p_2^{a_2} \dots p_k^{a_k} \divby \ord y \\
		\ord y \ndivby p_1^{a_1}
	\end{rcases} \implies \underbrace{p_1^{a_1 - 1}p_2^{a_2} \dots p_k^{a_k}}_{h'} \divby \ord y $$
	$$ h' \divby \ord y \implies y^{h'} = 1 \quad \forall y \in U $$
	$ y $ "--- корень кногочлена $ x^{h'} - 1 \quad \forall y \in U $

	У него $ h > h' $ корней "--- \contra
	$$ \ord y = p_1^{a_1} \cdot t \implies \ord(y^t) = p_1^{a_1} $$
	Подойдёт $ x_1 = y^t $. Аналогично $ x_i $

	Докажем, что для $ \veps = x_1x_2 \dots x_k $ выполнено $ \ord \veps = h $:

	Положим $ b_i \define p_1^{a_1} \dots p_i^{a_i - 1} \dots p_k^{a_k} $

	$ x_i^{b_i} \ne 1 $ \as $ b_i \ndivby \ord x_i $
	$$ x_i^{b_i} - 1, \qquad j \ne i $$
	$ x_j^{b_i} = 1 $ при $ i \ne j $
	$$ \veps^{b_i} = \underbrace{x_1^{b_i}}_1 \dots \underbrace{x_i^{b_i}}_{\ne 1} \dots \underbrace{x_k^{b_i}}_1 \ne 1 $$
	$$ h \divby \ord \veps, \qquad b_i \ndivby \veps \quad \forall i \quad \implies \ord \veps = h $$
	\item $ \veps $ "--- примитивный

	$ 1, \veps, \veps^2, \dots, \veps^{h - 1} $ различны $ \implies 1, \veps, \dots, \veps^{h - 1} $ "--- все элементы $ U $ ($ \veps^i = \veps^j \implies \veps^{i - j} = 1 $)
\end{eproof}

\section{Количество примитивных корней. Многочлен деления круга}

\begin{lemma}[количество примитивных корней]
	$ K $ "--- поле, $ \qquad h \in \N, \quad h \ndivby \chara K $ \\
	$ x^h - 1 $ раскладывается на линейные множители

	Тогда в $ K $ есть $ \vphi(h) $ примитивных корней из единицы.
\end{lemma}

\begin{proof}
	$ \veps $ "--- примитивный корень \\
	Все корни: $ \veps^0 = 1, \quad \veps^1 = \veps, \quad \veps^2, \quad \dots, \quad \veps^{n - 1} $

	Докажем, что $ \veps^s $ примитивный $ \iff \NOD(s, h) = 1 $:
	\begin{itemize}
		\item Пусть $ \NOD(s, h) = 1, \quad (\veps^s)^k = 1 \implies \veps^{sk} = 1 \implies sk \divby h 	\implies k \divby h $
		$$ \ord \veps^s = h $$
		\item Пусть $ \NOD(s, h) = d \ne 1 $
		$$ (\veps^s)^{\frac hd} = \veps^{\frac{sh}d} = (\underset{= 1}{\veps^h})^{\frac sd} = 1 \implies \ord \veps^s = \frac hd \implies \veps^s \text{ не примитивный} $$
	\end{itemize}
\end{proof}

\begin{definition}
	$ K $ "--- поле, $ \qquad h \in \N, \quad h \ndivby \chara K, \qquad x^h - 1 $ раскладывается на лин. множит. \\
	$ \veps_1, \dots, \veps_{\vphi(h)} $ "--- все примитивные корни степени $ h $

	\it{Многочлен} деления круга (круговой многочлен) "--- это
	$$ \Phi_h(x) = (x - \veps_1)(x - \veps_2) \cdots (x - \veps_{\vphi(h)}) $$
\end{definition}

\begin{theorem}[многочлен деления круга]
	$ K $ "--- поле, $ \qquad h \in \N, \quad h \ndivby \chara K $ \\
	$ x^h - 1 $ раскладывается на линейные множители

	\begin{enumerate}
		\item $ {\displaystyle x^h - 1 = \prod_{d | h} \Phi_d(x)} $;
		\item если $ K = \Co $, то коэффициенты $ \Phi_h(x) $ "--- целые числа.
	\end{enumerate}
\end{theorem}

\begin{eproof}
	\item
	$$ x^h - 1 = \prod_{\veps \in U}(x - \veps), \qquad U \text{ "--- группа корней $ h $-й степени из 1} $$
	$$ \prod_{d | h} \Phi_d(x) \iseq \prod_{\veps \in U}(x - \veps) $$
	\begin{itemize}
		\item Пусть $ x - \veps $ входит в $ \Phi_d(x) \implies \veps^d = 1, \quad \veps^k \ne 1 $ при $ k < d \quad \implies \veps^h = 1 $ (\as $ h \divby d $), $ \quad \veps \in U $
		\item Пусть $ x - \veps $ входит в правую часть \\
		Тогда $ \veps \in U $

		Пусть $ \ord \veps = d \implies x - \veps $ входит в $ \Phi_d(x) $, не входит в $ \Phi_{\vawe d}(x), \quad \vawe d \ne d $
	\end{itemize}
	\item \bt{Индукция.}
	\begin{itemize}
		\item \bt{База} "--- проверено в примерах.
		\item \bt{Переход} к $ h $ от меньших чисел.
		$$ \Phi_h(x) = \frac{x^h - 1}{\Phi_{d_1}(x) \cdots \Phi_{d_k}(x)}, \qquad d_1, \dots, d_k \text{ "--- делители } h \text{, не равные } h $$
		Знаменатель "--- многочлен с целыми коэффициентами, старший коэффициент равен 1. При делении на такой многочлен получаем целые коэффициенты.
	\end{itemize}
\end{eproof}

\section{Строение конечного поля. Единственность}

\begin{theorem}[строение конечного поля]
	$ K $ "--- конечное поле

	Существуют $ p \in \Prime, \quad n \in \N $ такие, что
	\begin{enumerate}
		\item $ K $ содержит простое поле $ F \simeq \Z_p $;
		\item $ \chara K = p $;
		\item $ |K| = p^n $;
		\item $ K $ является полем разложения многочлена $ x^{p^n} - x $ над $ F $.
	\end{enumerate}
\end{theorem}

\begin{eproof}
	\item $ F $ "--- минимальное подполе $ \implies F $ простое $ \implies $ оно изоморфно $ \Q $ или $ \Z_p $ \\
	$ \Q $ бесконечно, значит $ \exist p : \quad F \simeq \Z_p $
	\item $ \chara k = \min\set{n | \underbrace{1 + \dots + 1}_n = 0} $
	$$ 1, \quad 1 + 1, \quad 1 + 1 + 1, \quad \dots \in F \implies \chara k = \chara F = p $$
	\item $ K $ конечно над $ F $, \as есть $ \le |K| $ ЛНЗ над $ F $ элементов. \\
	Пусть $ n = |K : F| $ \\
	$ e_1, \dots, e_n $ "--- базис $ K $ над $ F $. \\
	$ \implies $ элементы $ K $ имеют вид $ a_1e_1 + a_2e_2 + \dots a_ne_n \in F $ \\
	$ |K| $ "--- количество наборов $ a_1, \dots, a_n \in F \implies |K| = p^n $
	\item Пусть $ U = K^* $ \nimp[(группа ненулевых элементов $ K $ по умножению)]
	$$ \implies |U| = p^n - 1 $$
	$$ \forall x \in U \quad p^n - 1 \divby \ord x \implies x^{p^n - 1} = 1 \quad \forall x \in K \setminus \set{0} \implies x^{p^n} - x = 0 \quad \forall x \in K $$
	$ \implies $ все элементы $ K $ "--- корни $ x^{p^n} - x = 0 $
	$$ \deg(x^{p^n} - x) = p^n = |K| \implies \text{ других корней нет} $$
\end{eproof}

\begin{implication}[единственность]
	Любые два конечных поля с одинаковым числом элементов изоморфны.
\end{implication}

\begin{proof}
	$ |K| = p^n \implies K $ изоморфно полю разложения $ x^{p^n} - x $ над $ \Z_p $.
\end{proof}

\section{Существование поля с данным количеством элементов}

\begin{theorem}
	Для любых $ p \in \Prime, \quad n \in \N $ существует поле из $ p^n $ элементов
\end{theorem}

\begin{proof}
	Пусть $ L $ "--- поле разложения $ P(x) = x^{p^n} - x $ над $ \Z_p $, $ \quad K $ "--- подмножество $ L $, состоящее из корней $ P(x) $.
	$$ P'(x) = \underset0{p^n}x^{p^n - 1} - 1 = -1 \text{ не имеет корней } \implies \text{ у } P, P' \text{ нет общих корней} $$
	$ \implies $ у $ P $ нет кратных корней $ \implies $ у $ P $ ровно $ p^n $ корней $ \implies |K| = p^n $

	Докажем, что $ K $ "--- поле: \\
	$ K $ "--- подмножество поля. Достаточно доказать, то $ 0, 1 \in K, \quad K $ замкнуто относительно $ =, \cdot $, взятия обратного по $ +, \cdot $:
	\begin{itemize}
		\item $ P(0) = 0^{P^n} - 0 = 0, \qquad P(1) = 1 - 1 \quad \implies 0, 1 \in K $
		\item $ x, y \in K \implies x^{p^n} - x = 0, \quad y^{p^n} - y = 0 $
		\begin{multline*}
			(x = y)^{p^n} = \bigg( (x = y)^p \bigg)^{p^n - 1} = (x^p + y^p)^{x^{p^{n - 1}}} = \bigg( (x^p + y^p)^p \bigg)^{p^{n - p}} = \\
			= (x^{p^2} + y^{p^2})^{p^{n - p}} = \dots = x^{p^n} + y^{p^n} = x + y
		\end{multline*}
		$$ (x + y)^{p^n} - (x + y) = 0 $$
		\item $ P(-x) = (-x)^{p^n} - (-x) = -(x^{p^n} - x) = \dots = -P(x) $
		$$ x \in K \implies p(x) = 0 \implies p (-x) = 0 \implies -x \in K $$
		\item $ x, y \in K \implies x^{p^n} = x, \quad y^{p^n} = y $
		$$ P(xy) = (xy)^{p^n} - xy = x^{p^n}y^{p^n} - xy = 0 \implies xy \in K $$
		\item $ x \in K $
		$$ x^{p^n} = x \implies \bigg( \frac1x \bigg)^{p^n} = \frac1{x^{p^n}} = \frac1x $$
	\end{itemize}
\end{proof}
