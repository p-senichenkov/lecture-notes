\chapter{Теория функции комплексной переменной}

\section{Основные понятия}

Косплексная плоскость $ \Co $ является метрическим пространством, поскольку $ |z| = \sqrt{x^2 + y^2} = \norm{(x, y)}_{R^2} $, если $ z = x + i y $. Поэтому определны понятия точки сгущения множества и предела функции:

\begin{definition}
	$ c $ является \it{точкой сгущения} множетсва $ E \sub \Co $, если
	$$ \forall \delta > 0 \quad \exist z \in E \setminus \set{c} : \quad |z - c| < \delta $$
\end{definition}

\begin{definition}
	$$ f(z) \underarr{z \to c} A \in \Co \iff \forall \veps > 0 \quad \exist \delta > 0 : \quad \forall z \in E \setminus \set{c} : |z - c| < \delta \qquad |f(z) - c| < \veps $$
\end{definition}

Для $ w = u + iv $ полагаем $ \Re w = u, \quad \Im w = v, \quad \ol w = u - iv $. Множеству $ x + i \cdot 0 $ соспоставим $ \R $. Полагаем, что $ i \cdot 0 = 0 $, и $ \R \sub \Co $.

\begin{statement}
	Если $ u(z) = \Re f(z), \quad v(z) = \Im f(z) $, то
	$$ f(z) \underarr{z \to c} A \iff
	\begin{cases}
		u(z) \underarr{z \to c} \Re A \\
		v(z) \underarr{z \to c} \Im A
	\end{cases} $$
	Из свойств функций нескольких переменных получаем:
	\begin{equ}3
		f(z) \underarr{z \to c} A \in \Co \implies \exist \delta > 0, ~ M > 0 : \quad \forall z \in E \setminus \set{c} : |z - c| < \delta \qquad |f(z)| \le M
	\end{equ}
\end{statement}

\begin{stmts}
	\item \label{en:stmt:lim:1} $ f(z) \underarr{z \to c} A, \quad A \ne 0 $
	$$ \implies \exist \delta > 0 : \quad \forall z \in E \setminus \set{c} : |z - c| < \delta \qquad |f(z)| > \frac{|A|}2 $$

	\item $ f(z) \underarr{z \to c} \quad \implies \quad kf(z) \underarr{z \to c} kA $

	\item $ f(z) \underarr{z \to c}, \quad g(z) \underarr{z \to c} \to B $
	$$ \implies f(z) + g(z) \underarr{z \to c} A + B $$

	\item $ f(z) \underarr{z \to c} A, \qquad g(z) \underarr{z \to c} B $
	$$ \implies f(z)g(z) \underarr{z \to c} AB $$

	\item $ f(z) \underarr{z \to c} A, \quad A \ne 0, \quad f(z) \ne 0 $ при $ z \in E \setminus \set c $
	$$ \implies \frac1{f(z)} \underarr{z \to c} \frac1A $$

	\item $ f(z) \underarr{z \to c} A, \qquad f(z) \ne 0, \quad A \ne 0, \quad g(z) \underarr{z \to c} B $
	$$ \implies \frac{g(z)}{f(z)} \underarr{z \to c} \frac BA $$
\end{stmts}

\begin{eproof}
	\item Положим $ \veps \define \frac{|A|}2 > 0 $. Тогда
	$$ \exist \delta > 0 : \quad \forall z \in E \setminus \set c : |z - c| < \delta \qquad |f(z) - A| < \frac{|A|}2 $$
	При таких $ z $ выполнено
	$$ |f(z)| = |f(z) - A + A| \trige |A| - |f(z) - A| > |A| - \frac12 |A| = \frac12 |A| $$

	\item Следует из линейности и аддитивности предела вещественных функций.

	\item Аналогично.

	\item
	$$ \exist \delta' > 0, ~ M' > 0, ~ \delta'' > 0, ~ M'' > 0 : \quad \forall z \in E \setminus \set c \quad
	\begin{cases}
		|z - c| < \delta' \implies |f(z)| < M' \\
		|z - c| < \delta'' \implies |g(z)| < M''
	\end{cases} $$
	Возьмём $ \veps > 0 $. Тогда
	$$ \exist \delta_\circ', ~ \delta_\circ'' : \quad \forall z \in E \setminus \set c \quad
	\begin{cases}
		|z - c| < \delta_\circ' \implies |f(z) - A| < \veps \\
		|z - c| < \delta_\circ'' \implies |g(z) - B| < \veps
	\end{cases} $$
	Пусть $ |z - c| < \delta_\circ = \min\set{\delta', \delta'', \delta_\circ', \delta_\circ''} $. Тогда
	\begin{multline*}
		|f(z)g(z) - AB| = \big| \big( f(z) - A \big)g(z) + A \big( g(z) - B \big) \big| \trile |f(z) - A| \cdot |g(z)| + |A| \cdot |g(z) - B| < \\
		< \veps \cdot M'' + |A| \cdot \veps = \veps(M'' + |A|)
	\end{multline*}

	\item Возьмём $ \delta_1 $ из \ref{en:stmt:lim:1}. Выберем $ \forall \veps > 0 $. Тогда
	\begin{equ}5
		\exist \delta_2 > 0 : \quad \forall z \in E \setminus \set c \quad |z - c| < \delta_2 \implies |f(z) - A| < \veps
	\end{equ}
	Выберем $ \delta_3 \define \min\set{\delta_1, \delta_2} $. Тогда
	$$ \ref{en:stmt:lim:1}., \eref5 \implies \bigg| \frac1{f(z)} - \frac1A \bigg| = \frac{|A - f(z)|}{|A| \cdot |f(z)|} < \frac\veps{|A| \cdot \frac{|A|}2} = \frac{2\veps}{|A|^2} $$
	При $ z \in E \setminus \set c, \quad |z - c| < \delta_3 $ это то, что требовалось доказать.

	\item
	$$ \frac{g(z)}{f(z)} = g(z) \cdot \frac1{f(z)} \to B \cdot \frac1A = \frac BA $$
\end{eproof}

\begin{definition}
	$ f(z) \underarr{z \to c} \infty $, если
	$$ \forall L > 0 \quad \exist \delta > 0 : \quad \forall z \in E \setminus \set c \quad |z - c| < \delta \implies |f(z)| > L $$
\end{definition}

Непрерывность в точке и на множестве определны, поскольку $ \Co $ является метрическим пространством.

Функции $ f : E \to \Co, \quad E \sub \Co $ сопоставим функцию $ f^* : E^* \to \Co $: \\
Для $ z = x + iy, \quad z \in E $ положим $ f^*(x, y) \define f(z) $.

Если $ f(z) = u(z) + iv(z), \quad u, v : E \to \R $, то далее знак $ * $ в обозначениях $ u^*(x, y), ~ v^*(x, y) $ будем пропускать, поэтому верно равенство
$$ f(z) = u(x, y) + iv(x, y) $$

\begin{definition}
	Открытое связное множество в $ \Co $ будем называть областью.
\end{definition}

\section{Частные производные}

Пусть $ E \sub \Co $ "--- область, $ \quad z_\circ = x_\circ + iy_\circ \in E, \quad f(z) = u(x, y) + iv(x, y) $. \\
Положим
$$ f_x'(z_\circ) \define u_x'(x_\circ, y_\circ) + iv_x'(x_\circ, y_\circ) $$
$$ f_y'(z_\circ) \define u_y'(x_\circ, y_\circ) + iv_y'(x_\circ, y_\circ) $$
$$ f_z'(z_\circ) \define \frac12 \bigg( f_x'(z_\circ) - if_y'(z_\circ) \bigg) $$
$$ f_{\ol z}'(z_\circ) \define \frac12 \bigg( f_x'(z_\circ) + if_y'(z_\circ) \bigg) $$

\begin{eg}
	$ f(z) \equiv z, \qquad f : \Co \to \Co, \qquad z = x + iy $
	$$ u(x, y) \equiv x, \qquad v(x, y) \equiv y $$
	$$ z_x' = 1, \qquad z_y' = i, \qquad z_z' = \frac12(1 - i \cdot i) = 1, \qquad z_{\ol z}' = \frac12(1 + i \cdot i) = 0 $$
\end{eg}

\begin{definition}
	Пусть $ E \sub \Co $ "--- область, $ \qquad z_\circ \in E, \qquad f : E \to \Co, \qquad f(z) = u(x, y) + iv(x, y) $ \\
	$ u, v : E^* \to \R $

	Будем говорить, что $ f $ дифференцируема в точке $ z_\circ $, если $ f^*(x, y) $ дифференцируема в точке $ (x_\circ, y_\circ) $ в следующем смысле: \\
	функции $ u(x, y) $ и $ v(x, y) $ дифференцируемы в точке $ (x_\circ, y_\circ) $.
\end{definition}

\section{Формула для дифференцируемой функции}

Пусть $ \sigma \define s + it, \qquad E \sub \Co $ "--- область, $ \qquad z_\circ \in E, \quad z_\circ + \sigma \in E, \qquad z_\circ \leftrightarrow (x_\circ, y_\circ) $ \\
Предположим, что $ f(z) $ дифференцируема в точке $ z_\circ $.

Тогда
\begin{equ}{11}
	f(z_\circ + \sigma) - f(z_\circ) = f^*(x_\circ + s, y_\circ + t) - f^*(x_\circ, y_\circ) = \bigg( u(x_\circ + s, y_\circ + t) - u(x_\circ, y_\circ) \bigg) + i \bigg( v(x_\circ + s, y_\circ + t) - v(x_\circ, y_\circ) \bigg)
\end{equ}

В силу дифференцируемости $ f $
$$ u(x_\circ + s, y_\circ + t) - u(x_\circ, y_\circ) = u_x'(x_\circ, y_\circ)s + u_y'(x_\circ, y_\circ)t + r_1(s, t), \qquad \frac{|r_1(s, t)|}{\sqrt{s^2 + t^2}} = \frac{|r_1(s, t)|}{|\sigma|} \undereq{\sigma \to 0} 0 $$
$$ v(x_\circ + s, y_\circ + t) - v(x_\circ, y_\circ) = v_x'(x_\circ, y_\circ)s + v_y'(x_\circ, y_\circ)t + r_2(s, t), \qquad \frac{|r_2(s, t)|}{|\sigma|} \undereq{\sigma \to 0} 0 $$

Отсюда
\begin{multline*}
	\eref{11} = \bigg( u_x'(x_\circ, y_\circ)s + u_y'(x_\circ, y_\circ)t + r_1(s, t) \bigg) + i \bigg( v_x'(x_\circ, y_\circ)s + v_y'(x_\circ,y_\circ)t + r_2(s, t) \bigg) = \\
	= \bigg( u_x'(x_\circ, y_\circ) + iv_x'(x_\circ, y_\circ) \bigg)s + \bigg( u_y'(x_\circ, y_\circ) + iv_y'(x_\circ, y_\circ) \bigg)t + r_1(s, t) + ir_2(s, t) = \\
	= f_x'(z_\circ)s + f_y'(z_\circ)t + r_1(s, t) + ir_2(s, t)
\end{multline*}

Положим $ \rho(\sigma) \define r_1(s, t) + ir_2(s, t) $. Тогда
\begin{equ}{14}
	\frac{|\rho(\sigma)|}{|\sigma|} = \frac{\sqrt{r_1^2(s, t) + r_2^2(s, t)}}{|\sigma|} = \sqrt{\bigg( \frac{r_1(s, t)}{|\sigma|} \bigg)^2 + \bigg( \frac{r_2(s, t)}{|\sigma|} \bigg)^2} \underarr{\sigma \to 0} 0
\end{equ}

Понятно, что
$$ s = \frac12 (\sigma + \ol \sigma), \qquad t = \frac1{2i}(\sigma - \ol \sigma) = -\frac i2(\ol \sigma - \sigma) $$

Поэтому
\begin{multline}\lbl{15}
	f(z_\circ + \sigma) - f(z_\circ) = f_x'(z_\circ) \cdot \frac12 (\sigma + \ol \sigma) + \frac i2 f_y'(z_\circ)(\ol \sigma - \sigma) + \rho(\sigma) = \\
	= \frac12 \bigg( f_x'(z_\circ) - if_y'(z_\circ) \bigg) \sigma + \frac12 \bigg( f_x'(z_\circ) + if_y'(z_\circ) \bigg) \ol \sigma + \rho(\sigma) = f_z'(z_\circ) \sigma + f_{\ol z}(z_\circ) \ol \sigma + \rho(\sigma)
\end{multline}
где для $ \rho(\sigma) $ выполнено \eref{14}.

\begin{definition}
	$ E \sub \Co $ "--- область, $ \qquad f(z) = u(x, y) + iv(x, y) : E \to \Co, \qquad u, v : E* \to \R $

	Будем говорить, что $ f \in \Cont[1] E $, если $ u \in \Cont[1]{E^*} $ и $ v \in \Cont[1]{E^*} $
\end{definition}

\begin{statement}
	$ f \in \Cont[1] E $

	Тогда $ f $ дифференцируема в $ \forall z \in E $.
\end{statement}

\begin{proof}
	По определению $ u, v \in \Cont[1]{E^*} $, поэтому по достаточному условию дифференциуемости функции $ u, v $ дифференцируемы для $ \forall (x, y) \in E $. \\
	Тогда, по определению, $ f^*(x, y) $ дифференцируема $ \forall (x, y) \in E $. \\
	Значит, $ f $ дифференцируема для $ \forall z \in E $.
\end{proof}

\begin{definition}
	$ E \sub \Co $ "--- область, $ \qquad f : E \to \Co $

	Функцию $ f $ будем называть аналитической, если
	\begin{enumerate}
		\item $ f \in \Cont[1]E $;
		\item $ \forall z \in E \quad f_{\ol z}' = 0 $.
	\end{enumerate}
\end{definition}

\begin{remark}
	По предыдущему утверждению $ f(z) $ дифференцируема $ \forall z \in E $, поэтому для $ \forall z \in E $ определены $ f_x'(z), f_y'(z), f_z'(z), f_{\ol z}'(z) $.
\end{remark}

\begin{notation}
	Мноеество всех функций, аналитических в $ E $, будем обозначать $ A(E) $.
\end{notation}

\subsection{Свойства частных производных}

\begin{properties}
	$ E \sub \Co $ "--- область, $ \qquad z \in E, \qquad f, g $ дифференцируемы в $ z, \qquad \lambda $ "--- любой из символов $ x, y, z, \ol z $.
	\begin{enumerate}
		\item $ \bigg( cf(z) \bigg)_\lambda' = cf_\lambda'(z) $
		\item $ \bigg( f(z) + g(z) \bigg)_\lambda' = f_\lambda'(z) + g_\lambda'(z) $
		\item $ \bigg( f(z)g(z) \bigg)_\lambda' = f_\lambda'(z)g(z) + f(z)g_\lambda'(z) $
		\item $ f(z) \ne 0 $
		$$ \bigg( \frac1{f(z)} \bigg)_\lambda' = -\frac{f_\lambda'(z)}{f^2(z)} $$
		\item $ f(z) \ne 0 $
		$$ \bigg( \frac{g(z)}{f(z)} \bigg)_\lambda' = \frac{g_\lambda'(z)f(z) - g(z)f_\lambda'(z)}{f^2(z)} $$
	\end{enumerate}
\end{properties}

\begin{proof}
	Доказательства проводятся проверкой возникающих тождеств. Докажем для примера 4 при $ \lambda = x $ и при $ \lambda = \ol z $.

	Пусть $ f(z) = u(x, y) + iv(x, y), \quad f_x'(z) = u_x' + iv_x' $ (далее не будем писать вргументы).
	$$ \frac1f = \frac1{u + iv} = \frac{u - iv}{u^2 + v^2} = \frac u{u^2 + v^2} - i \frac v{u^2 + v^2} $$
	\begin{multline*}
		\bigg( \frac1f \bigg)_x' = \bigg( \frac u{u^2 + v^2} \bigg)_x' - i \bigg( \frac v{u^2 + v^2} \bigg)_x' = \frac{u_x'(u^2 + v^2) - 2u(uu_x' + vv_x')}{(u^2 + v^2)^2} - i \frac{v_x'(u^2 + v^2) - 2v(uu_x' + vv_x')}{(u^2 + v^2)^2} = \\
		= \frac{u_x'v^2 - 2uvv_x' - u^2ux' - i(v_x'u^2 - 2uvu_x' - v^2v_x')}{(u^2 + v^2)^2} = \frac{(u_x' + iv_x')(v^2 - u^2) - 2uv(v_x' - iu_x')}{(u^2 + v^2)^2} = \\
		= \frac{f_x'(v^2 - u^2) + 2uvi(u_x' + iv_x')}{(u^2 + v^2)^2} = f_x' \cdot \frac{v^2 - u^2 + 2uvi}{(u^2 + v^2)^2} = f_x' \cdot \frac{(v + iu)^2}{(u^2 + v^2)^2} = -f_x' \cdot \frac{(u - iv)^2}{(u^2 + v^2)^2} = \\
		= -f_x' \cdot \frac{((u - iv)^2)}{(u - iv)^2(u + iv)^2} = -f_x' \cdot \frac1{(u + iv)^2} = -\frac{f_x'}{f^2}
	\end{multline*}
	$$ \bigg( \frac1f \bigg)_{\ol z}' = \frac12 \bigg\lgroup \bigg( \frac1f \bigg)_x' + i \bigg( \frac1f \bigg)_y' \bigg\rgroup = \frac12 \bigg( -\frac{f_x'}{f^2} - i \frac{f_y'}{f^2} \bigg) = -\frac12 \cdot \frac1{f^2} (f_x' + if_y') = -\frac{f_{\ol z}'}{f^2} $$
\end{proof}

\subsection{Первые свойства аналитических функций}

\begin{properties}
	$ f, g \in A(E), \qquad c \in E $
	\begin{enumerate}
		\item $ cf \in A(E) $
		\item $ f + g \in A(E) $
		\item $ fg \in A(E) $
		\item $ f(z) \ne 0 $
		$$ \implies \frac1f \in A(E) $$
		\item $ f(z) \ne 0 $
		$$ \implies \frac gf \in A(E) $$
	\end{enumerate}
\end{properties}

\begin{proof}
	Следует из свойств частных производных, например, 4:
	$$ \bigg( \frac1{f(z)} \bigg)_{\ol z}' = -\frac{f_{\ol z}'(z)}{f^2(z)} = \frac 0{f^2(z)} = 0 $$
\end{proof}

\subsection{Первые примеры аналитических функций}

\begin{exmpls}
	\item $ f(z) \equiv c, \qquad c \in \Co $
	$$ c_x' = c_y' \equiv 0 \quad \implies c_{\ol z}' \equiv 0 $$

	\item $ f(z) \equiv z $ \\
	Уже проверено, что $ z_{\ol z}' \equiv 0 $

	\item Пользуясь свойствами аналитических функций 1., 2., 3. и предыдущими примерами, получаем
	$$ z^2 \in A(\Co), \quad z^3 \in A(\Co), \quad \dots, \quad z^n \in A(\Co) $$
	$$ P(z) = c_0 + c_1z + \dots + c_nz^n \in A(\Co) $$

	\item Для $ z = x + iy $ положим $ e^z \define e^y \cos y + ie^x \sin y $. Тогда
	$$ (e^z)_x' = (e^x \cos y)_x' + i(e^x \sin y)_x' = e^x \cos y + ie^x \sin y $$
	$$ (e^z)_y' = (e^x \cos y)_y' + i(e^x \sin y)_y' = -e^x \sin y + ie^x \cos y $$
	$$ (e^z)_{\ol z}' = \frac12 \bigg( (e^z)_x' + i(e^z)_y' \bigg) = \frac12 \bigg( e^x\cos y + ie^x \sin y + i(-e^x \sin y + ie^x \cos y) \bigg) = 0 $$

	\item Пусть $ P(z) = c_0 + c_1z + \dots + c_nz^n, \quad Q(z) = b_0 + b_1z + \dots + b_mz^m, \quad m \ge 1, \quad \alpha_1, \dots, \alpha_k \in \Co $ "--- все различные корни уравнения $ Q(z) = 0, \quad k \le m $.

	Тогда по примеру 3. и свойству 5.
	$$ \frac{P(z)}{Q(z)} \in A \big( \Co \setminus \bigcup_{j = 1}^k \set{\alpha_j} \big) $$

	\item Пусть $ D = \Co \setminus (-\infty, 0], \qquad $ для $ z \in D $ пусть $ \vphi $ "--- аргумент $ z, \quad -\pi < \vphi < \pi $. \\
	Положим $ \ln z \define \ln|z| + i \vphi $ для $ z \in D $.

	Если $ z = x + iy, \quad |z| > 0, \quad z \in D $, то $ \vphi $ может быть определён разными формулами при $ x > 0 $, при $ y \ge 0 $ или при $ y \le 0 $. \\
	Например, при $ x > 0 \quad \vphi = \arctg \frac yx $ и тогда
	$$ \ln z = \ln \sqrt{x^2 + y^2} = i \arctg \frac yx = \frac12 \ln(x^2 + y^2) = i \arctg \frac yx $$
	Тогда
	$$ (\ln z)_x' = \bigg( \frac12 \ln(x^2 + y^2) \bigg)_x' + i \bigg( \arctg \frac yx)_x' = \frac x{x^2 + y^2} + i \bigg( -\frac y{x^2} \cdot \frac1{1 + (\frac yx)^2} \bigg) = \frac x{x^2 + y^2} - i \frac y{x^2 + y^2} $$
	$$ (\ln z)_y' = \bigg( \frac12 \ln(x^2 + y^2) \bigg)_y' + i \bigg( \arctg \frac yx \bigg)_y' = \frac y{x^2 + y^2} + i \bigg( \frac1x \cdot \frac1{1 + (\frac yx)^2} \bigg) = \frac y{x^2 + y^2} + i \frac x{x^2 + y^2} $$
	$$ (\ln z)_{\ol z}' = \frac12 \bigg( (\ln z)_x' + i(\ln z)_y' \bigg) = \frac12 \bigg( \frac x{x^2 + y^2} - i \frac y{x^2 + y^2} + i \big( \frac y{x^2 + y^2} + i \frac x{x^2 + y^2} \big) \bigg) = 0 $$
	Аналогично, $ (\ln z)_{\ol z}' = 0 $ при $ y \ge 0 $ и при $ y \le 0 $. Получаем
	$$ \ln z \in A(D) $$
\end{exmpls}

\subsection{Ещё одно свойство аналитических функций}

\begin{statement}
	$ f \in A(E), \qquad E $ "--- область, $ \qquad z \in E, \qquad \sigma \in \Co, \quad z + \sigma \in \Co $

	Тогда
	\begin{equ}{16}
		f(z + \sigma) - f(z) = f_z'(z)\sigma + \rho(\sigma), \qquad \frac{|\rho(\sigma)|}{|\sigma|} \underarr{\sigma \to 0} 0
	\end{equ}
\end{statement}

\begin{proof}
	Из \eref{15} и того, что $ f \in A(E) $ следует, что
	$$ f(z - \sigma) - f(z) = f_z'(z)\sigma + f_{\ol z}'(z)\ol \sigma + \rho(\sigma) = f_z'(z)\sigma + \rho(\sigma) $$
	где выполнено \eref{14}.
\end{proof}

\subsection{Эквивалентные определения аналитических функций}

\begin{theorem}
	$ E \sub \Co $ "--- область, $ \qquad f \in \Cont[1] E, \qquad f(z) = u(x, y) + iv(x, y) $

	Следующие условия эквивалентны:
	\begin{enumerate}
		\item $ f_{\ol z}'(z) = 0 \quad \forall z \in E $
		\item $ f(z + \sigma) = f_z'(z)\sigma + \rho(\sigma) \quad \forall z \in E, \qquad \frac{|\rho(\sigma)|}{|\sigma|} \underarr{\sigma \to 0} 0 $
		\item $ \forall z = x + iy $ выполнены уравнения Коши"--~Римана:
		\begin{equ}{18}
			\begin{rcases}
				u_x'(x, y) = v_y'(x, y) \\
				u_y'(x, y) = -v_x'(x, y)
			\end{rcases}
		\end{equ}
		\item
		\begin{equ}{19}
			\forall z \in E \quad \exist \limz\sigma \frac{f(z + \sigma) - f(z)}\sigma \in \Co
		\end{equ}
		Предел в \eref{19} называется комплексной призводной функции $ f $ в точке $ z $ и обозначается $ f'(z) $.
	\end{enumerate}
\end{theorem}

\begin{iproof}
	\item Из \eref{16} следует, что 1. $ \implies $ 2.
	\item Если выполнено 2., то
	$$ \frac{f(z + \sigma) - f(z)}\sigma = f_z'(z) + \frac{\rho(\sigma)}\sigma \underarr{\sigma \to 0} f_z'(z), $$
	поэтому 2. $ \implies $ 4., при этом получаем равенство
	\begin{equ}{20}
		f'(z) = f_z'(z)
	\end{equ}
	\item Предположим, что выполнено 4. \\
	Положим
	$$ \frac{f(z + \sigma) - f(z)}\sigma \define f'(z) = \delta(\sigma) $$
	$$ \implies f(z + \sigma) - f(z) = f'(z) \sigma + \sigma \delta(\sigma) $$
	Положим $ \rho_\circ(\sigma) \define \sigma\delta(\sigma) $. Тогда
	$$ 4. \implies \frac{|\rho_\circ(\sigma)|}{|\sigma|} = |\delta(\sigma)| \underarr{\sigma \to 0} 0 $$
	Запишем для $ f $ формулу \eref{15}:
	$$ f(z + \sigma) - f(z) = f_z'(z)\sigma + f_{\ol z}'(z) \ol \sigma + \rho(\sigma), \qquad \frac{|\rho(\sigma)|}{|\sigma|} \underarr{\sigma \to 0} 0 $$
	Вычитая из неё предыдущую формулу, получаем
	$$ f_z'(z) \sigma + f_z' (z) \ol \sigma + \rho(\sigma) - f'(z)\sigma - \rho_\circ(\sigma) = 0 $$
	Делим на $ \sigma $:
	$$ f_z'(z) - f'(z) + f_{\ol z}'(z) \frac{\ol \sigma}\sigma + \frac{\rho(\sigma)}\sigma - \frac{\rho_\circ(\sigma)}\sigma = 0 $$
	или
	$$ f_[\ol z]'(z) \frac{\ol \sigma}\sigma = f'(z) - f_z'(z) + \frac{\rho_\circ(\sigma)}\sigma - \frac{\rho(\sigma)}\sigma $$
	$$ f'(z) - f_z'(z) + \frac{\rho_\circ(\sigma)}\sigma - \frac{\rho(\sigma)}\sigma \underarr{\sigma \to 0} f'(z) - f_z'(z) $$
	Следовательно, существует $ \limz\sigma f_{\ol z}'(z) \frac{\ol \sigma}\sigma \fed A $.

	Если $ \sigma = s > 0 $, то $ \ol \sigma = \sigma $ и
	$$ a = \liml{\sigma \to 0^+} f_{\ol z}'(z) \cdot 1 = f_{\ol z}'(z) $$

	Если положить $ \sigma = it, \quad t > 0 $, то $ \ol \sigma = -it $, и
	$$ A = \liml{t\ to 0^+} f_{\ol z}' \cdot \frac{-it}{it} = -f_{\ol z}' $$
	$$ \implies f_{\ol z}' = A = 0 $$
	То есть, 4. $ \implies $ 1. и
	$$ f_z'(z) = f'(z) $$

	\item Далее,
	$$ f_x'(z) = u_x'(x, y) + iv_x'(x, y), \qquad f_y'(z) = u_y'(x, y) + iv_y'(x, y) $$
	\begin{multline*}
		f_{\ol z}' = \frac12 \bigg( f_x'(z) + if_y'(z) \bigg) = \frac12 \bigg( \big( u_x'(x, y) + iv_x'(x, y) \big) + i \big( u_y'(x, y) + iv_y'(x, y) \big) \bigg) = \\
		= \frac12 \bigg( \big( u_x'(x, y) - v_y'(x, y) \big) + i \big( v_x'(x, y) + u_y'(x, y) \big) \bigg)
	\end{multline*}
	Отсюда 1. $ \iff $ 3.
\end{iproof}

\begin{implication}
	$ f \in A(E) $
	$$ \implies f'(z) = f_x'(z), \qquad z \in E $$
\end{implication}

\begin{proof}
	Имеем
	$$ f_z' = \frac12 \big( f_x' - if_y' \big), \qquad f_{\ol z}' = \frac12 \big( f_x' + if_y' \big) $$
	$$ \implies f_x' = f_z' + f_{\ol z}' $$
	$$ f \in A(E) \implies f_z' = f_z' + 0 = f_z' = f' $$
\end{proof}
