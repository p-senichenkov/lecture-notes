\part{Функциональные последовательности и ряды}

\section{Определение равномерной сходимости функциональной последовательности и функционального ряда}

\begin{definition}
	$ f_n(x) $ \bt{равномерно} на $ X $ сходится (стремится) к $ f(x) $, если
	$$ \forall \veps > 0 \quad \exist N : \quad \forall n > N \quad \forall x \in X \quad |f_n(x) - f(x)| < \veps $$
\end{definition}

\begin{notation}
	$ f_n(x) \uniarr{x \in X} f(x) $
\end{notation}

\begin{definition}
	Будем говорить, что функциональный ряд \bt{равномерно} сходится на $ X $ к сумме $ S(x) $, если
	$$ S_n(x) \uniarr{x \in X} S(x) $$
	Тогда ряду приписывается значение:
	$$ \sum_{n = 1}^\infty v_n(x) \define S(x) $$
\end{definition}

\section{Критерий Коши равномерной сходимости функциональной последовательности и функционального ряда}

\begin{theorem}
	Для того, чтобы функциональная последовательность \bt{равномерно} сходилась на $ X $ к некоторой функции $ f $, \bt{необходимо и достаточно}, чтобы
	$$ \forall \veps > 0 \quad \exist N : \quad \forall n > N, ~ m > N \quad \forall x \in X \quad |f_n(x) - f_m(x)| < \veps $$
\end{theorem}

\begin{iproof}
	\item Необходимость \\
	Пусть $ f_n(x) \uniarr{x \in X} f(x) $ \\
	В таком случае, по определению равномерной сходиомости,
	$$ \forall \veps > 0 \quad \exist N : \quad \forall n > N \quad \forall x \in X \quad |f_n(x) - f(x)| < \half[\veps] $$
	Возьмём произвольный $ x \in X $
	$$ |f_m(x) - f_n(x)| \undereq{\pm f(x)} \bigg| \bigg( f_m(x) - f(x) \bigg) + \bigg( f(x) - f_n(x) \bigg) \bigg| \trile |f_m(x) - f(x)| + |f(x) - f_n(x)| < \half[\veps] + \half[\veps] = \veps $$
	\item Достаточность \\
	Фиксируем $ x \in X $ \\
	Получаем \bt{числовую} последовательность $ \seq{f_n(x)}n $ \\
	По критерию Коши для числовых последовательностей, она имеет конечный предел:
	$$ \exist \limi{n} f_n(x) \in \R $$
	То есть, любая точка из $ X $ является точкой сходимости:
	$$ E_0 = X $$
	Получается, что $ f_n(x) $ поточечно сходится к $ f(x) $ на $ X $:
	\begin{equ}{7'}
		f_n(x) \infarr{n} f(x)
	\end{equ}
	При фиксированных $ x $ и $ \veps $ имеем
	$$ |f_m(x) - f_n(x)| < \veps $$
	Фиксируем $ \forall m > N $ и переходим к пределу по $ n $:
	$$ \limi{n} |f_m(x) - f_n(x)| \le \veps \quad \underimp{\eref{7'}} |f_m(x) - f(x)| \le \veps $$
	Т. к. мы брали $ \forall m > N $ и $ \forall x \in X $, это и есть определение равномерной сходимости
\end{iproof}

\begin{theorem}
	Имеется ряд
	$$ \sum_{n = 1}^\infty v_n(x), \qquad x \in X $$
	Для того чтобы он \bt{равномерно} сходился, \bt{необходимо и достаточно}, чтобы
	$$ \forall \veps > 0 \quad \exist N : \quad \forall m > n > N \quad \forall x \in X \quad \bigg| \sum_{k = n + 1}^m v_k(x) \bigg| < \veps $$
\end{theorem}

\begin{proof}
	По определению равномерная сходимость функционального ряда означает, что равномерно сходится последовательность $ \seq{S_n(x)}n $ \\
	Применяя к ней критерий Коши, получаем, что для её равномерной сходимости необходимо и достаточно, чтобы
	$$ \forall \veps > 0 \quad \exist N : \quad \forall m > n > N \quad \forall x \in X \quad |S_n(x) - S_m(x)| < \veps $$
	Это и есть условие теоремы.
\end{proof}

\section{Признак Вейерштрасса равномерной сходимости функциональных рядов}

\begin{theorem}
	Имеется ряд
	\begin{equ}{cauchy_seq:14}
		\sum_{n = 1}^\infty v_n(x)
	\end{equ}
	\begin{equ}{cauchy_seq:15}
		\exist c_n : \quad \sum_{n = 1}^\infty c_n \text{ сходится}
	\end{equ}
	\begin{equ}{cauchy_seq:16}
		|v_n(x)| \le c_n \quad \forall x \in X
	\end{equ}
	Тогда ряд \eref{cauchy_seq:14} сходится \bt{равномерно}
\end{theorem}

\begin{proof}
	Вспомним критерий Коши для числовых рядов: \\
	Возьмём $ \forall \veps > 0 $
	\begin{equ}{cauchy_seq:17}
		\eref{cauchy_seq:15} \implies \exist N : \quad \forall m > n > N \quad \sum_{k = n + 1}^m c_k < \veps
	\end{equ}
	\begin{note}
		Мы не ставим модуль, поскольку $ c_k $ положительные
	\end{note}
	Зафиксируем эти $ m, n, N $ и возьмём $ \forall x \in X $
	$$ \eref{cauchy_seq:17} \implies \bigg| \sum_{k = n + 1}^m v_k(x) \bigg| \trile \sum_{k = n + 1}^m |v_k(x)|\underset{\eref{cauchy_seq:16}}\le \sum_{k = n + 1}^m c_k \underset{\eref{cauchy_seq:17}}< \veps $$
	Применяем критерий Коши для функционального ряда
\end{proof}

\section{Признак Дирихле равномерной сходимости функциональных рядов}

\begin{theorem}
	Имеется ряд
	\begin{equ}{dirichle_series:19}
		\sum_{n = 1}^\infty b_n(x)v_n(x)
	\end{equ}
	\begin{equ}{dirichle_series:20}
		\bm{b_n} \text{ \bt{монотонна} по } n \quad \forall \text{ фиксированного } x \in X
	\end{equ}
	\begin{equ}{dirichle_series:21}
		\bm{b_n(x)} \uniarr{x \in X} 0
	\end{equ}
	\begin{equ}{dirichle_series:22}
		\exist c > 0 : \quad \forall n \quad \forall x \in X \quad \bigg| \bm{\sum_{k = 1}^n v_k(x)} \bigg| \bm\le c
	\end{equ}
	Тогда ряд \eref{dirichle_series:19} \bt{равномерно} сходится на $ X $
\end{theorem}

\begin{proof}
	Возьмём $ \forall \veps > 0 $
	\begin{equ}{dirichle_series:23}
		\eref{dirichle_series:21} \implies \exist N : \quad \forall n > N \quad \forall x \in X \quad |b_n(x)| < \veps
	\end{equ}
	\begin{equ}{dirichle_series:24}
		\forall m > n \ge 1 \quad \bigg| \sum_{k = n + 1}^m v_k(x) \bigg| = \bigg| \sum_{k = 1}^m v_k(x) - \sum_{k = 1}^n v_k(x) \bigg| \trile \sum_{k = 1}^m |v_k(x)| - \sum_{k = 1}^n |v_k(x)| \underset{\eref{dirichle_series:22}}\le c + c = 2c
	\end{equ}
	Рассмотрим сумму
	$$ \sum_{k = n + 1}^m b_k(x) v_k(x) $$
	Определим
	$$ V_n(x) \definerel\equiv 0 $$
	$$ V_{n + 1}(x) \define v_{n + 1}(x) $$
	$$ \widedots[6em] $$
	$$ V_l(x) = v_{n + 1}(x) + \dots + v_l(x), \qquad n + 1 < l \le m $$
	Тогда $ v_k(x) = V_k(x) - V_{k - 1}(x), \quad k \ge n + 1 $ \\
	Перепишем нашу сумму:
	\begin{multline}\lbl{dirichle_series:25}
		\sum_{k = n + 1}^m b_k(x)v_k(x) = \sum_{k = n + 1}^m b_k(x) \bigg( V_k(x) - V_{k - 1}(x) \bigg) = \\
		= \sum_{k = n + 1}^m b_k(x) V_k(x) - \sum_{k = n + 1}^m b_k(x) V_{k - 1}(x) \undereq{
			\begin{subarray}c
				k \define k + 1 \\
				\text{во второй сумме}
			\end{subarray}} \sum_{k = n + 1}^m b_k(x) V_k(x) - \sum_{k = n}^{m - 1}b_{k + 1}(x) V_k(x) \undereq{V_n \bydef 0} \\
		= b_m(x)V_m(x) + \sum_{k = n + 1}^{m - 1} \bigg( b_k(x) - b_{k + 1}(x) \bigg)V_k(x)
	\end{multline}
	\begin{equ}{dirichle_series:26}
		\eref{dirichle_series:24} \implies |V_k(x)| \le 2c \quad \forall k
	\end{equ}
	Возьмём $ N $ из \eref{dirichle_series:23}, $ m > n > N $ и $ \forall x \in X $
	\begin{multline*}
		\eref{dirichle_series:25} \implies \bigg| \sum_{k = n + 1}^m b_k(x)v_k(x) \bigg| \le |b_m(x)| \cdot |V_m(x)| + \sum_{k = n + 1}^{m - 1} |b_k(x) - b_{k + 1}(x)| \cdot |V_k(x)| \underset{\eref{dirichle_series:26}}< \\
		< \veps \cdot 2c + 2c \sum_{k = n + 1}^{m - 1} |b_k(x) - b_{k + 1}(x)| \undereq{\eref{dirichle_series:20}} 2c\veps + 2c \bigg| \sum_{k = n + 1}^{m - 1} \bigg( b_k(x) - b_{k + 1}(x) \bigg) \bigg| = 2c\veps + 2c|b_{n + 1}(x) - b_m(x)| \le \\
		\le 2c\veps + 2c \bigg( \underbrace{|b_{n + 1}(x)|}_{< \veps} + \underbrace{|b_m(x)|}_{< \veps} \bigg) < 6c\veps
	\end{multline*}
	Можно применить критерий Коши
\end{proof}

\section{Признак Абеля равномерной сходимости функциональных рядов}

\begin{theorem}
	Имеется ряд
	\begin{equ}{abel_series:27}
		\sum_{n = 1}^\infty b_n(x)v_n(x)
	\end{equ}
	$$ \bm{b_n(x)} \text{ \bt{монотонна} по } n \quad \forall x \in X $$
	$$ \exist c > 0 : \quad |\bm{b_n(x)}| \bm\le c \quad \forall n \quad \forall x \in X $$
	\begin{equ}{abel_series:30}
		\bm{\sum_{n = 1}^\infty v_n(x)} \text{ равномерно \bt{сходится} на } X
	\end{equ}
	$$ \implies \text{ ряд \eref{abel_series:27} равномерно сходится} $$
\end{theorem}

\begin{proof}
	Применим необходимую часть критерия Коши к условию \eref{abel_series:30}:
	\begin{equ}{abel_series:32}
		\forall \veps > 0 \quad \exist N : \quad \forall m > n > N \quad \forall x \in X \quad \bigg| \sum_{k = n + 1}^m v_k(x) \bigg| < \veps
	\end{equ}
	Возьмём какое-нибудь $ m_0 > n > N $ \\
	Соотношение \eref{abel_series:32} действует при $ m = n + 1, ..., m_0 $ \\
	Определим функции $ V_k(x) $ так же, как в доказательстве признака Дирихле \\
	Там было доказано, что
	$$ \sum_{k = 1}^{m_0} b_k(s)v_k(x) = \sum_{k = n + 1}^{m_0 - 1}V_k(x) \bigg( b_k(x) - b_{k + 1}(x) \bigg) + b_{m_0}(x)V_{m_0}(x) $$
	\begin{multline*}
		\implies \bigg| \sum_{k = n + 1}^{m_0} b_k(x)v_k(x) \bigg| \trile \bigg| \sum_{k = n + 1}^{m_0 - 1} V_k(x) \bigg( b_k(x) - b_{k + 1}(x) \bigg) \bigg| + |b_{m_0}(x)| \cdot |V_{m_0}(x)| \le \\
		\le \sum_{k = n + 1}^{m_0 - 1} |V_k(x)| \cdot \bigg| b_k(x) - b_{k + 1} \bigg| + |b_{m_0}(x)V_{m_0}(x)| \underset{\eref{abel_series:32}}\le \underbrace{|b_{m_0}(x)|}_{\le c} \veps + \veps \sum_{k = n + 1}^{m_0 - 1} |b_k(x) - b_{k + 1}(x)| \le \\
		\le c\veps + \veps \bigg| \sum_{k = n + 1}^{m_0 - 1} \bigg( b_k(x) - b_{k + 1} \bigg) \bigg| = c\veps + \veps |b_{n + 1}(x) - b_{m_0}(x)| \le 3c\veps
	\end{multline*}
\end{proof}

\section{Теорема о переходе к пределу в равномерно сходящейся\n функциональной последовательности}

\begin{theorem}
	$ X, \diam(x, y) $ "--- метрическое пространство, $ \qquad x_0 \in X $ "--- точка сгущения $ X $ \\
	$ \seq{f_n(x)}n, \qquad f_n : X \setminus \set{x_0} \to \R, \qquad f : X \setminus \set{x_0} \to \R $
	\begin{equ}{lim_seq:41}
		\bm{f_n(x)} \uniarr{x \in X \setminus \set{x_0}} \bm{f(x)}
	\end{equ}
	\begin{equ}{lim_seq:42}
		\forall x \in X \setminus \set{x_0} \quad \exist \liml{x \to x_0} f_n(x) = a_n
	\end{equ}
	\begin{mequ}[\implies \empheqlbrace]
		\lbl{lim_seq:43} \exist \limi{n} a_n = A \in \R \\
		\lbl{lim_seq:45} \exist \bm{\liml{x \to x_0} f(x) = A}
	\end{mequ}
\end{theorem}

\begin{iproof}
	\item \eref{lim_seq:43}

	Применим критерий Коши к \eref{lim_seq:41}:
	$$ \implies \forall \veps > 0 \quad \exist N : \quad \forall m, n > N \quad \forall x \in X \quad |f_m(x) - f_n(x)| < \veps $$
	Зафиксируем всё, кроме $ x $, а $ x $ устремим к $ x_0 $:
	$$ \implies \liml{x \to x_0} |f_m(x) - f_n(x)| \le \veps \quad \underimp{\eref{lim_seq:42}} \quad |a_m - a_n| \le \veps \quad \implies \quad \exist \limi{n} a_n = A \in \R $$

	\item \eref{lim_seq:45}

	Возьмём $ \forall \veps > 0 $ \\
	Выберем $ N_1 $ такое, что
	\begin{equ}{lim_seq:49}
		\forall n > N_1 \quad \forall x \in X \setminus \set{x_0} \quad |f_n(x) - f(x)| < \veps
	\end{equ}
	Выберем $ N_2 $ такое, что
	\begin{equ}{lim_seq:410}
		\forall n > N_2 \quad |a_n - A| < \veps
	\end{equ}
	Выберем $ N_0 \define \max\set{N_1, N_2} + 1 $ \\
	Выберем $ \delta > 0 $ такое, что
	\begin{equ}{lim_seq:411}
		\forall y \in X \setminus \set{x_0} \quad \nimp[\bigg(] \diam(y, x_0) < \delta \implies |f_{N_0}(y) - a_{N_0}| < \veps \nimp[\bigg)]
	\end{equ}
	$$ f(y) - A \undereq{
		\begin{subarray}{c}
			\pm f_{N_0}(y) \\
			\pm a_{N_0}
		\end{subarray}} f(y) - f_{N_0}(y) + f_{N_0}(y) - a_{N_0} + a_{N_0} - A $$
	$$ \implies |f(y) - A| \trile \underbrace{|f(y) - f_{N_0}(y)|}_{\eref{lim_seq:49}} + \underbrace{|f_{N_0}(y) - a_{N_0}|}_{\eref{lim_seq:411}} + \underbrace{|a_{N_0} - A|}_{\eref{lim_seq:410}} < 3\veps \implies \eref{lim_seq:45} $$
\end{iproof}

\section{Теорема о непрерывности в точке предела равномерно сходящейся функциональной последовательности и суммы равномерно сходящегося функционального ряда}

\begin{implication}[о непрерывности в точке]
	$ x_0 \in X, \qquad f_n : X \to \R, \qquad f_n(x) $ \bt{непрерывна в} $ \bm{x_0} \quad \forall n $
	\begin{equ}{lim_seq:412}
		f_n(x) \uniarr{x \in X} f(x)
	\end{equ}
	$$ \implies f(x) \bt{ непрерывна в } \bm{x_0} \quad \forall n $$
\end{implication}

\begin{proof}
	Непрерывность $ f_n $ означает, что
	$$ \liml{x \to x_0} f_n(x) = \underbrace{f_n(x_0)}_{a_n} \in \R $$
	То есть, выполнено второе условие из теоремы.
	$$ \eref{lim_seq:412} \implies f_n(x_0) \infarr{n} f(x_0) $$
	$$ \implies  \exist \limi{n} a_n = \limi{n} f_n(x_0) = f(x_0) $$
	$$ \exist \liml{x \to x_0} f(x) = \limi{n} a_n = f(x_0) $$
\end{proof}

\begin{implication}
	$ X $ всюду плотно, $ \qquad \bm{f_n \in \Cont{X}}, \qquad f_n(x) \uniarr{x \in X} f(x) $
	$$ \implies \bm{f \in \Cont{X}} $$
\end{implication}

\begin{theorem}
	$ X $ "--- метрическое пространство, $ \qquad \seq{v_n(x)}n, \qquad x_\circ $ "--- т. сг., $ \qquad $ имеется ряд
	\begin{equ}{lim_series:1}
		\sum_{n = 1}^\infty v_n(x) = S(x)
	\end{equ}
	\begin{enumerate}
		\item $ \eref{lim_series:1} $ сходится \bt{равномерно} на $ X \setminus \set{x_0}, \qquad \forall n \quad \bm{\exist \liml{x \to x_0} v_n(x)} = c_n $
		$$ \implies
		\begin{cases}
			\sum_{n = 1}^\infty c_n \text{ сходится } \\
			\exist \liml{x \to x_0} S(x) = \sum_{n = 1}^\infty c_n
		\end{cases} $$

		\item \eref{lim_series:1} сходится равномерно на всём $ X, \qquad v_n $ \bt{непр. в} $ \bm{x_0} \quad \forall n $
		$$ \implies S(x) \bt{ непр. в } \bm{x_0} $$

		\item $ X $ всюду плотно, $ \qquad \eref{lim_series:1} $ сходится равномерно на всём $ X, \qquad \bm{v_n \in \Cont{X}} \quad \forall n $
		$$ \implies \bm{S \in \Cont{X}} $$
	\end{enumerate}
\end{theorem}

\begin{proof}
	$ S_n(x) = v_1(x) + \dots + v_n(x) $ \\
	Равномерная сходимость ряда \eref{lim_series:1} по определению означает, что
	$$ S_n(x) \uniarr{x \in X \setminus \set{x_0}} S(x) $$
	Применимы теоремы для функциональных последовательностей.
\end{proof}

% hyphenation here doesn't work fine
\section{Интегрирование равномерно сходящейся функциональной\n последовательности и равномерно сходящегося функцио\tpst{-\\}{}нального ряда}

\begin{theorem}
	$ \seq{f_n(x)}n, \qquad f_n \in \Cont{[a, b]}, \qquad f_n(x) \uniarr{x \in [a, b]} f(x) $
	\begin{remark}
		В таком случае $ f \in \Cont{[a, b]} $, а значит $ f \in \Ri{[a, b]} $
	\end{remark}
	$$ \implies \dint{a}b{f_n(x)} \infarr{n} \dint{a}b{f(x)} $$
\end{theorem}

\begin{proof}
	Равномерная сходимость означает, что
	$$ \forall \veps > 0 \quad \exist N : \quad \forall n > N \quad \forall x \in [a, b] \quad |f_n(x) - f(x)| < \veps $$
	$$ \bigg| \dint{a}b{f_n(x)} - \dint{a}b{f(x)} \bigg| = \bigg| \dint{a}b{\bigg( f_n(x) - f(x) \bigg)} \bigg| \le \dint{a}b{\big| f_n(x) - f(x) \big|} < \dint{a}b\veps = \veps(b - a) $$
	Это и есть определение сходимости требуемой числовой последовательности
\end{proof}

\begin{theorem}
	$ \seq{v_n(x)}n, \qquad v_n \in \Cont{[a, b]}, \qquad \sum_{n = 1}^\infty v_n(x) $ равномерно сходится на $ [a, b] $
	$$ \implies \dint{a}b{\sum_{n = 1}^\infty v_n(x)} = \sum_{n = 1}^\infty \dint{a}b{v_n(x)} $$
\end{theorem}

\begin{proof}
	Обозначим
	$$ c_n \define \dint{a}b{v_n(x)}, \qquad B_n \define c_1 + \dots + c_n, \qquad S_n(x) \define v_1(x) + \dots + v_n(x), \qquad S(x) \define \sum_{n = 1}^\infty v_n(x) $$
	По определению равномерной сходимости ряда,
	$$ S_n(x) \uniarr{x \in [a, b]} S(x) $$
	Отсюда, по только что доказанной теореме,
	$$
	\begin{rcases}
		\dint{a}b{S_n(x)} \infarr{n} \dint{a}b{S(x)} \\
		\dint{a}b{S_n(x)} = \dint{a}b{\bigg( v_1(x) + \dots + v_n(x) \bigg)} = B_n
	\end{rcases} \implies B_n \infarr{n} \dint{a}b{S(x)} $$
\end{proof}

\section{Дифференцирование равномерно сходящейся функциональной последовательности и равномерно сходящегося функционального ряда}

\begin{theorem}
	$ \seq{f_n(x)}n, \qquad \forall n \quad \forall x \in [a, b] \quad \exist f_n'(x) $
	\begin{equ}{deriv_seq:11}
		\exist \vphi(x) : [a, b]  \to \R : \quad f_n'(x) \uniarr{x \in [a, b]} \vphi(x)
	\end{equ}
	\begin{equ}{deriv_seq:12}
		\exist x_0 \in [a, b] : \quad \exist \limi{n} f_n(x_0) \in \R
	\end{equ}
	\begin{mequ}[{\implies \exist f : [a, b] \to \R : \quad \empheqlbrace}]
		&f_n(x) \uniarr{x \in [a, b]} f(x) \\
		\lbl{deriv_seq:14} &\forall x \in [a, b] \quad \exist \bm{f'(x) = \vphi(x)}
	\end{mequ}
\end{theorem}

\begin{iproof}
	\item Возьмём $ m \ne n $ \\
	Определим функции:
	$$ P_{mn}(x) \define f_m(x) - f_n(x) $$
	Так как $ f_n $ дифференцируемы,
	\begin{equ}{deriv_seq:16}
		\exist P_{mn}'(x) = f_m'(x) - f_n'(x)
	\end{equ}
	Значит, к $ P_{mn} $ можно применить теорему Лагранжа:
	\begin{multline}\lbl{deriv_seq:17}
		\forall x \in [a, b] : x \ne x_0 \quad \exist c \in (x \between x_0) : \quad P_{mn}(x) - P_{mn}(x_0) = P_{mn}'(c)(x - x_0) = \\
		= \bigg( f_m'(c) - f_n'(c) \bigg)(x - x_0)
	\end{multline}
	К функциональной последовательности производных применим необходимую часть критерия Коши:
	\begin{equ}{deriv_seq:18}
		\forall \veps > 0 \quad \exist N_1 : \quad \forall m > n > N_1 \quad \forall x \in [a, b] \quad |f_m'(x) - f_n'(x)| < \veps
	\end{equ}
	\begin{multline}\lbl{deriv_seq:19}
		\underimp{\eref{deriv_seq:17}} \forall m > n > N_1 \quad \forall x \in [a, b], x \ne x_0 \quad |P_{mn}(x) - P_{mn}(x_0)| = \\
		= |f_m'(c) - f_n'(c)| \cdot |x - x_0| < \veps (b - a)
	\end{multline}
	По условию \eref{deriv_seq:12} мы можем применить критерий Коши к $ f_n(x_0) $:
	$$ \exist N_2 : \quad \forall m > n > N_2 \quad |f_m(x_0) - f_n(x_0)| < \veps $$
	Применим обозначение $ P_{mn} $:
	\begin{equ}{deriv_seq:20}
		\forall m > n > N_2 \quad |P_{mn}(x_0)| < \veps
	\end{equ}
	Пусть $ N \define \max\set{N_1, N_2} $. При $ m > n > N $ действуют и \eref{deriv_seq:19}, и \eref{deriv_seq:20}:
	\begin{multline*}
		\forall x \in [a, b], x \ne x_0 \quad |P_{mn}(x)| = |P_{mn}(x) - P_{mn}(x_0) + P_{mn}(x_0)| \trile |P_{mn}(x) - P_{mn}(x_0)| + |P_{mn}(x_0)| < \\
		< (b - a)\veps + \veps = (b - a + 1)\veps
	\end{multline*}
	При $ x = x_0 $ это тоже верно (т. к. у нас есть \eref{deriv_seq:20}), т. е.
	$$ \forall x \in [a, b] \quad |f_m(x) - f_n(x)| < (b - a + 1)\veps $$
	Значит, к функциональной последовательности $ f_n(x) $ можно применить критерий Коши:
	\begin{equ}{deriv_seq:13}
		\implies f_n(x) \uniarr{x \in [a, b]} f(x)
	\end{equ}
	$$ f_n \in \Cont{[a, b]} \implies f \in \Cont{[a, b]} $$
	\item Фиксируем произвольный $ x \in [a, b] $ \\
	Рассмотрим
	$$ g_n : [a, b] \setminus \set{x} : \quad g_n(y) \define \frac{f_n(y) - f_n(x)}{y - x}, \qquad g : [a, b] \setminus \set{x} : \quad g(y) \define \frac{f(y) - f(x)}{y - x} $$
	\begin{multline}\lbl{deriv_seq:24}
		g_m(y) - g_n(y) = \frac{f_m(y) - f_m(x) - (f_n(y) - f_n(x))}{y - x} = \frac{\big( f_m(y) - f_n(y) \big) - \big( f_m(x) - f_n(x) \big)}{y - x} = \\
		= \frac{P_{mn}(y) - P_{mn}(x)}{y - x}
	\end{multline}
	Применим теорему Лагранжа:
	$$ \exist c_1 \in (y \between x) : \quad P_{mn}(y) - P_{mn}(x) = P_{mn}'(c_1)(y - x) $$
	Подставим в \eref{deriv_seq:24}:
	\begin{equ}{deriv_seq:26}
		g_m(y) - g_n(y) = P_{mn}'(c_1)
	\end{equ}
	$$ P_{mn}'(c_1) \undereq{\eref{deriv_seq:16}} f_m'(c_1) - f_n'(c_1) $$
	$$ \eref{deriv_seq:18}, \eref{deriv_seq:26} \implies \forall y \in [a, b] \setminus \set{x} \quad \forall m > n > N_1 \quad |g_m(y) - g_n(y)| < \veps|y - x| \le \veps(b - a) $$
	Применим критерий Коши:
	\begin{equ}{deriv_seq:28}
		\exist h : [a, b] \setminus \set{x} : \quad g_n(y) \uniarr{y \in [a, b] \setminus \set{x}} h(y)
	\end{equ}
	Зафиксируем $ y \in [a, b] \setminus \set{x} $ и рассмотрим числовую последовательность:
	\begin{equ}{deriv_seq:29}
		\implies g_n(y) \infarr{n} h(y)
	\end{equ}
	$$
	\eref{deriv_seq:13} \implies
	\begin{cases}
		f_n(y) \infarr{n} f(y) \\
		f_n(x) \infarr{n} f(x)
	\end{cases} $$
	$$ \underimp{\operatorname{def} g_n} g_n(y) \infarr{n} \frac{f(y) - f(x)}{y - x} $$
	$$ \underimp{\eref{deriv_seq:29}} h(y) = \frac{f(y) - f(x)}{y - x} $$
	$$ \underimp{\eref{deriv_seq:28}} \frac{f_n(y) - f_n(x)}{y - x} \uniarr{y \in [a, b] \setminus \set{x}} \frac{f(y) - f(x)}{y - x} $$
	$$ \underimp{\operatorname{def} g_n, g} g_n(y) \uniarr{y \in [a, b] \setminus \set{x}} g(y) $$
	$$ \exist f_n' \bdef[\iff]{g_n} \forall n \quad \exist \liml{y \to x} g_n(y) = f_n'(x) $$
	К последним двум выражениям можно применить теорему о предельном переходе в функциональной последовательности:
	$$ \exist \liml{y \to x} g(y) = A, \qquad \exist \limi{n} f_n'(x), \qquad A = \limi{n} f_n'(x) $$
	$$ \bdefimp{g} \exist \liml{y \to x} \frac{f(y) - f(x)}{y - x} = f'(x) = A $$
	$$
	\begin{rcases}
		\implies \exist f'(x) = \limi{n} f_n'(x) \\
		\eref{deriv_seq:11} \implies \text{ для фиксированного } x \in [a, b] \quad f_n'(x) \infarr{n} \vphi(x)
	\end{rcases} \implies \vphi(x) = f'(x) $$
\end{iproof}

У этой теоремы имеется вариант для функциональных рядов:

\begin{theorem}
	$ \seq{v_n(x)}, \qquad v_n \in \Cont{[a, b]}, \qquad \forall x \in [a, b] \quad \exist v_n'(x) $
	\begin{equ}{deriv_seq:40}
		\sum_{n = 1}^\infty v_n'(x) \text{ равномерно сходится на } [a, b]
	\end{equ}
	\begin{equ}{deriv_seq:41}
		\exist x_0 \in [a, b] : \quad \sum_{n = 1}^\infty v_n(x_0) \text{ сходится}
	\end{equ}
	$$ \implies \forall x \in [a, b] \quad \exist \bigg( \sum_{n = 1}^\infty v_n(x) \bigg)' = \sum_{n = 1}^\infty v_n' $$
\end{theorem}

\begin{proof}
	Рассмотрим частичные суммы:
	$$ S_n(x) = v_1(x) + \dots + v_n(x) \quad \forall x \in [a, b] \quad \forall n $$
	$$ \exist S_n'(x) = v_1'(x) + \dots + v_n'(x) $$
	$$ \eref{deriv_seq:40} \implies \exist \vphi(x) : \quad S_n'(x) \rightrightarrows \vphi(x) $$
	$$ \eref{deriv_seq:41} \implies S_n(x_0) \infarr{n} A \in \R $$
	К функциональной последовательности частичных сумм можно применить только что доказанную теорему
\end{proof}

\section{Пример Ван дер Вардена}

\begin{theorem}
	$ \exist f \in \Cont\R : \quad \forall x \in \R \quad \not\exist f'(x) $
\end{theorem}

\begin{proof}
	Рассмотрим функцию $ \vphi(x) \define 1 - |x - 1| $ при $ x \in [0, 2] $ (\autoref{tikz:van_der_varden:1.a}) \\
	При $ x \in [2k, 2k + 2] $, где $ k \ne 0 \in \Z $, полагаем $ \vphi(x) \define \vphi(x - 2k) $ (\autoref{tikz:van_der_varden:1.b})
	$$ f(x) \define \sum_{n = 0}^\infty \bigg( \frac34 \bigg)^n \vphi(4^n x) $$
	\begin{itemize}
		\item Проверим непрерывность $ f(x) $: \\
		Воспользуемся признаком Вейерштрасса:
		\begin{intuition}
			$ \vphi(4^n x) \in \Cont\R $
		\end{intuition}
		При этом, $ 0 \le \vphi(x) \le 1 $
		$$ \implies \bigg( \frac34 \bigg)^n |\vphi(4^n x) | = \bigg( \frac34 \bigg)^n \vphi(4^n x) \le \bigg( \frac34 \bigg)^n $$
		$$ \sum_{n = 0}^\infty \bigg( \frac34 \bigg)^n = \frac1{1 - \frac34} = 4 $$
		Значит, ряд $ f(x) $ равномерно сходится на $ \R $ \\
		Значит, и его сумма непрерывна
		\item Докажем, что производной не существует: \\
		Доказывать будем \bt{от противного}. Пусть есть точка, в которой существует производная:
		\begin{equ}{van_der_varden:44}
			\exist x \in \R : \quad \exist f'(x)
		\end{equ}
		Это эквивалентно тому, что $ f $ дифференцируема в этой точке:
		\begin{equ}{van_der_varden:45}
			f(y) - f(x) = f'(x)(y - x) + r(y),
		\end{equ}
		где $ \dfrac{|r(y)|}{|y - x|} \underarr{y \in x} 0 $, то есть
		\begin{equ}{van_der_varden:47}
			\exist \delta > 0 : \quad \forall y \in [x - \delta, x + \delta] \quad \frac{|r(y)|}{|y - x|} \le 1
		\end{equ}
		\begin{multline*}
			\underimp{\eref{van_der_varden:45}} \forall y \in [x - \delta, x + \delta] \quad |f(y) - f(x)| \le |f'(x)| \cdot |y - x| + |r(y)| \le \\
			\le \bigg( \underbrace{|f'(x)| + 1 \bigg)}_{\fed A}|y - x| = A|y - x|
		\end{multline*}
		Рассмотрим $ x - \delta \le y_1 \le x \le y_2 \le x + \delta $
		\begin{equ}{van_der_varden:48}
			\implies
			\begin{cases}
				|f(y_2) - f(x)| \le A(y_2 - x) \\
				|f(x) - f(y_1)| \le A(x - y_1)
			\end{cases}
		\end{equ}
		\begin{multline*}
			\implies |f(y_2) - f(y_1)| = \bigg| \bigg( f(y_2) - f(x) \bigg) + \bigg( f(x) - f(y_1) \bigg) \bigg| \trile \\
			\le |f(y_2) - f(x)| + |f(x) - f(y_1)| \le A(y_2 - x) + A(x - y_1) = A(y_2 - y_1)
		\end{multline*}
		Выберем $ m \ge 1 $ так, что
		$$ 4^m > \frac1\delta $$
		Рассмотрим число $ 4^mx $ \\
		Так как это какое-то конкретное вещественное число, то
		$$ \exist k \in \Z : \quad k \le 4^mx < k + 1 $$
		$$ \implies \underbrace{k \cdot 4^{-m}}_{\fed a_m} \le x < \underbrace{(k + 1) \cdot 4^{-m}}_{\fed b_m} $$
		\begin{equ}{van_der_varden:416}
			b_m - a_m = 4^{-m}
		\end{equ}
		$$ 4^nx = 4^{n - m} \cdot 4^mx $$
		Рассмотрим числа $ 4^na_m $ и $ 4^nb_m $
		\begin{itemize}
			\item Пусть $ n > m $
			$$ 4^na_m = 4^{n - m} \cdot 4^ma_m = 4^{n - m} \cdot k $$
			$$ 4^nb_m = 4^{n - m} \cdot 4^mb_m = 4^{n - m}(k + 1) = \underbrace{4^{n - m}k}_{4^na_m} + \underbrace{4^{n - m}}_{\text{чётное}} $$
			\begin{equ}{van_der_varden:418}
				\implies \vphi(4^nb_m) = \vphi(4^na_m + \text{ чётное}) = \vphi(4^na_m)
			\end{equ}
			(т. к. у функциии $ \vphi $ период 2)

			\item Пусть $ n = m $
			$$ 4^ma_m = k, \qquad 4^mb_m = k + 1 $$
			\begin{equ}{van_der_varden:419}
				\vphi(4^mb_m) - \vphi(4^ma_m) = \vphi(k + 1) - \vphi(k)
			\end{equ}
			Посмотрев на график $ \vphi $, видим, что для всякого целого $ k $
			\begin{equ}{van_der_varden:420}
				|\vphi(k + 1) - \vphi(k)| = 1
			\end{equ}

			\item Пусть $ 0 < n < m $
			\begin{equ}{van_der_varden:421}
				4^nb_m - 4^na_n = 4^{n - m}4^m(b_m - a_m) = 4^{n - m}
			\end{equ}
			Запишем свойство $ \vphi(x) $, которое видно из графика, но можно доказать и аналитически:
			\begin{equ}{van_der_varden:422}
				\forall y_1, y_2 \in \R \quad |\vphi(y_2) - \vphi(y_1)| \le |y_2 - y_1|
			\end{equ}
			Рассмотрим выражение
			\begin{multline*}
				f(b_m) - f(a_m) \bdefeq{f} \sum_{n = 0}^\infty \bigg( \frac34 \bigg)^n \vphi(4^n b_m) - \sum_{n = 0}^\infty \bigg( \frac34 \bigg)^n \vphi(4^nb_m) = \\
				= \sum_{n = 0}^{m - 1} \bigg( \frac34 \bigg)^n \bigg( \vphi(4^nb_m) - \vphi(4^na_m) \bigg) + \bigg( \frac34 \bigg)^m \bigg( \vphi(4^mb_m) - \vphi(4^ma_m) \bigg) + \\
				+ \sum_{n = m + 1}^\infty \bigg( \frac34 \bigg)^n \bigg( \underbrace{\vphi(4^nb_m) - \vphi(4^na_m)}_{= 0 \text{ по } \eref{van_der_varden:418}} \bigg) = \\
				= \bigg( \frac34 \bigg)^m \bigg( \vphi(4^mb_m) - \vphi(4^ma_m) \bigg) + \sum_{n = 0}^{m - 1} \bigg( \frac34 \bigg)^n \bigg( \vphi(4^nb_m) - \vphi(4^na_m) \bigg)
			\end{multline*}
			\begin{multline}\lbl{van_der_varden:424}
				\implies |f(b_m) - f(a_m)| \ge \\
				\ge \bigg( \frac34 \bigg)^m |\vphi(4^mb_m) - \vphi(4^ma_m)| - \sum_{n = 0}^{m - 1} \bigg( \frac34 \bigg)^n |\vphi(4^nb_m) - \vphi(4^na_m)| \underset{\eref{van_der_varden:419}, \eref{van_der_varden:422}}\ge \\
				\ge \bigg( \frac34 \bigg)^m |\vphi(k + 1) - \vphi(k)| - \sum_{n = 0}^{m - 1} |4^nb_m - 4^na_m| \undereq{\eref{van_der_varden:420}, \eref{van_der_varden:421}} \bigg( \frac34 \bigg)^m \cdot 1 - \sum_{n = 0}^{m - 1} \bigg( \frac34 \bigg)^n \cdot 4^{n - m} = \\
				= \bigg( \frac34 \bigg)^m - 4^{-m} \sum_{n = 0}^{m - 1} 3^n \undereq{\text{геом. прогр.}} \bigg( \frac34 \bigg)^m - 4^{-m} \cdot \frac{3^m - 1}{3 - 1} > \half \cdot \bigg( \frac34 \bigg)^m
			\end{multline}
			При этом, $ a_m \le x < b_m $, поэтому
			$$ |f(b_m) - f(a_m)| \underset{\eref{van_der_varden:48}}\le A(b_m - a_m) \undereq{\eref{van_der_varden:416}} A \cdot 4^{-m} $$
			$$ \underimp{\eref{van_der_varden:424}} 4^{-m} > \half \cdot 3^m \cdot 4^{-m} \quad \implies \quad \half \cdot 3^m < A $$
			При этом, $ A $ не зависит от $ m $, а условие на $ m $ позволяет нам брать произвольно большие $ m $, в том числе такое, что
			$$ \half \cdot 3^m > A \text{ "--- } \contra $$
		\end{itemize}
	\end{itemize}
\end{proof}

\begin{figure}[!ht]
	\begin{subcaptionblock}{0.249\textwidth}
		\begin{tikzpicture}[>=Stealth]
			\draw[->] (-0.5, 0) -- (2.5, 0) node[right]{$ x $};
			\draw[->] (0, -0.5) -- (0, 1.5) node[above]{$ y $};

			\node[anchor=north east] at (0, 0) {0};
			\foreach \x in {1, ..., 2} \draw (\x, 0.05) -- (\x, -0.05) node[below]{\x};
			\foreach \y in {1} \draw (0.05, \y) -- (-0.05, \y) node[left]{\y};

			\draw[blue] (0, 0) -- (1, 1) -- (2, 0);
		\end{tikzpicture}
		\subcaption{Шаг 1}
		\label{tikz:van_der_varden:1.a}
	\end{subcaptionblock}
	\begin{subcaptionblock}{0.499\textwidth}
		\begin{tikzpicture}[>=Stealth]
			\draw[->] (-1, 0) -- (4.5, 0) node[right]{$ x $};
			\draw[->] (0, -0.5) -- (0, 1.5) node[above]{$ y $};

			\node[anchor=north east] at (0, 0) {0};
			\foreach \x in {1, ..., 4} \draw (\x, 0.05) -- (\x, -0.05) node[below]{\x};
			\foreach \y in {1} \draw (0.05, \y) -- (-0.05, \y) node[left]{\y};

			\draw[blue] (-0.5, 0.5) -- (0, 0) -- (1, 1) -- (2, 0) -- (3, 1) -- (4, 0) -- (4.5, 0.5);
		\end{tikzpicture}
		\subcaption{Шаг 2}
		\label{tikz:van_der_varden:1.b}
	\end{subcaptionblock}
\end{figure}

\section{Определение равномерной сходимости семейства функций; критерий Коши равномерной сходимости семейства функций}

\begin{definition}
	$ y_0 $ "--- т. сг. $ Y, \qquad f : E \times Y \to \R, \qquad f_0 : E \to \R $ \\
	Будем говорить, что семейство функций \bt{равномерно} сходится к $ f_0 $ при $ y \to y_0 $, если
	$$ \forall \veps > 0 \quad \exist \text{окрест. } U(y_0) : \quad \forall x \in E \quad \quad \forall y \in \bigg( U(y_0) \cap Y \bigg) \setminus \set{y_0} \quad |f(x, y) - f_0(x)| < \veps $$
\end{definition}

\begin{notation}
	$ f(x, y) \uniarr[y \to y_0]{x \in E} f_0(x) $
\end{notation}

\begin{theorem}[Критерий Коши]
	$ f : E \times Y \to \R, \qquad y_0 $ "--- т. сг. $ Y $ \\
	Для того, чтобы семейство функций равномерно сходилось к некоторой функции $ f_0 $ \bt{необходимо и достаточно}, чтобы
	$$ \forall \veps > 0 \quad \exist \text{окрест. } U(y_0) : \quad \forall y_1, y_2 \in \bigg( U(y_0) \cap Y \bigg) \setminus \set{y_0} \quad \forall x \in E \quad |f(x, y_2) - f(x, y_1)| < \veps $$
\end{theorem}

\begin{iproof}
	\item Необходимость: \\
	Пусть семейство функций $ f : E \times Y \to \R $ равномерно сходится к $ f_0 $ при $ y \to y_0 $ \\
	По определению это означает, что
	$$ \forall \veps > 0 \quad \exist \text{окр. } U(y_0) : \quad \forall y_1, y_2 \in \bigg( U(y_0) \cap Y \bigg) \setminus \set{y_0} \quad |f(x, y_{1,2}) - f_0(x)| < \half[\veps] $$
	$$ \implies |f(x, y_2) - f(x, y_1)| \trile |f(x, y_2) - f_0(x)| + |f_0(x) - f(x, y_1)| < \half[\veps] + \half[\veps] = \veps $$
	\item Достаточность \\
	Фиксируем $ x \in E $ \\
	Применяя критерий Коши к функции одного аругмента $ f(x, y) $, получаем, что $ \exist \liml{y \to y_0} f(x, y) \define f_0(x) $ \\
	Возьмём $ \forall \veps > 0 $, выберем окрестность $ U(y_0) $ \\
	Возьмём $ \forall y_1, y_2 \in \big( U(y_0), \cap Y \big) \setminus \set{x_0} $ и зафиксируем $ y_1 $
	$$ \liml{y_2 \to y_0} |f(x, y_2) - f(x, y_1)| \le \veps $$
	$$ |f_0(x) - f(x, y_1)| \bdefeq{f_0} \bigg| \liml{y_2 \to y_0} \bigg( f(x, y_2) - f(x, y_1) \bigg) \bigg| \undereq{\text{непр. } | \cdot |} \liml{y_2 \to y_0} | f(x, y_2) - f(x, y_1)| \le \veps $$
	Значит, $ f(x, y) $ равномерно сходится к $ f_0(x) $ при $ y \to y_0 $
\end{iproof}

\section{Теорема о переходе к пределу в равномерно сходящемся семействе функций}

\begin{theorem}
	$ f : E \times Y \to \R, \quad Y \sub \R^{n \ge 1}, \quad y_\circ \in \R^n $ "--- т. сг. $ Y, \quad E $ "--- метр. пр-во, $ \qquad x_\circ \in E $ "--- т. сг. $ E $
	$$ f(x, y) \uniarr[y \to y_0]{x \in E} f_0(x), \qquad \forall y \in Y \quad \exist \liml{x \to x_0} f(x, y) = \vphi(y) $$
	Тогда $ \exist \liml{y \to y_0} \vphi(y) $ и $ \exist \liml{x \to x_0} f_0(x) $ и справедливо
	$$ \bm{\liml{y \to y_0} \vphi(y) = \liml{x \to x_0} f_0(x)} $$
\end{theorem}

\begin{proof}
	Возьмём любую последовательность $ \seq{y_n}n, \quad y_n \in Y, \quad y_n \infarr{n} y_0 $ \\
	Положим $ f_n(x) \define f(x, y_n) $
	$$ f(x, y) \uniarr[y \to y_0]{x \in E} f_0(x) \quad \implies \quad f_n(x) \uniarr{x \in E} f_0(x) $$
	При этом, по условию теоремы для любого $ n $ имеем
	$$ \vphi(y_n) = \liml{x \to x_0} f(x, y_n) \bdefeq{f_n} \liml{x \to x_0}f_n(x) $$
	Значит, можно применить аналогичную теорему для функциональной последовательности:
	$$ \exist \limi{n}\vphi(y_n) \in \R, \qquad \exist \liml{x \to x_0} f_0(x), \qquad \limi{n}\vphi(y_n) = \liml{x \to x_0}f_0(x) $$
	В силу произвольности $ \seq{y_n}n $ последнее утверждение доказывает теорему.
\end{proof}

\section{Непрерывность предельной функции равномерно сходящегося семейства функций}

\begin{theorem}
	$ E, d $ "--- метрическое пространство, $ \qquad x_0 \in E $ "--- т. сг., $ \qquad y_0 $ "--- т. сг. $ Y \sub \R^n $ \\
	$ f(x, y) \uniarr[y \to y_0]{x \in E} f_0(x), \qquad \forall y \in Y \quad f(x, y) $ \bt{непр. в} $ \bm{x_0} $ \\
	Тогда $ f_0(x) $ \bt{непр. в} $ \bm{x_0} $
\end{theorem}

\begin{proof}
	Применим предыдущую теорему: \\
	По условию имеем $ \exist \liml{x \to x_0} f(x, y) \define \vphi(y) \quad \forall y \in Y $, при этом $ \vphi(y) = f(x_0, y) $ \\
	По предыдущей теореме $ \exist \liml{y \to y_0} \vphi(y) $ и $ \exist \liml{x \to x_0} f_0(x) $ и тогда
	$$ \liml{y \to y_0} f(x_0, y) = \liml{y \to y_0} \vphi(y) = \liml{x \to x_0} f_0(x) $$
	Но $ \liml{y \to y_0} f(x_0, y) = f_0(x_0) $, что и даёт непрерывность $ f_0 $ в $ x_0 $
\end{proof}

\begin{implication}
	$ f : E \times Y \to \R, \qquad f(x, y) \uniarr[y \to y_0]{x \in E} f_0(x), \qquad \forall y \in Y \quad \bm{f(x, y) \in \Cont{E}} $
	$$ \implies \bm{f_0 \in \Cont{E}} $$
\end{implication}

\section{Теорема о непрерывности интеграла, зависящего от параметра}

\begin{theorem}
	$ y_0 $ "--- т. сг. $ Y, \qquad f(x, y) \uniarr[y \to y_0]{x \in [a, b]} f_0(x), \qquad f(x, y) \in \Cont{[a, b]} \quad \forall y \in Y $ \\
	Тогда $ f_0 \in \Cont{[a, b]} $ и
	$$ I(y) \define \dint{a}b{f(x, y)} \underarr{y \to y_0} \dint{a}b{f_0(x)} $$
\end{theorem}

\begin{proof}
	Непрерывность $ f_0 $ следует из следствия к предыдущей теореме, поэтому интеграл в правой части определён \\
	По определению равномерной сходимости,
	$$ \forall \veps > 0 \quad \exist U(y_0) : \quad \forall y \in \bigg( U(y_0) \cap Y \bigg) \setminus \set{y_0} \quad |f(x, y) - f_0(x)| < \veps $$
	При таких $ y $ имеем
	$$ |I(t) - \dint{a}b{f_0(x)}| = \bigg| \dint{a}b{\bigg( f(x, y) - f_0(x) \bigg)} \bigg| \le \dint{a}b{|f(x, y) - f_0(x)|} \le \dint{a}b\veps = \veps(b - a) $$
\end{proof}

\begin{implication}
	$ Y = [p, q], \qquad f : [a, b] \times Y \to \R, \qquad f \in \Cont{[a, b] \times Y} $
	$$ \implies I(y) \in \Cont{[p, q]} $$
\end{implication}

\section{Теорема о производной интеграла, зависящего от параметра}

\begin{theorem}
	$ f : [a, b] \times [p, q] \to \R, \qquad \bm{f \in} \Cont{[a, b] \times [p, q]} $
	$$ \foral (x, y) \in [a, b] \times [p, q] \quad \exist f_y'(x, y), \qquad \bm{f_y'(x, y) \in} \Cont{[a, b] \times [p, q]} $$
	$$ \implies \quad \forall y \in [p, q] \quad \exist \bm{I'(y) =} \dint{a}b{f_y'(x, y)} $$
\end{theorem}

\begin{proof}
	Поскольку $ f_y' $ непрерывна, к ней применима терема Кантора:
	$$ \forall \veps > 0 \quad \exist \delta > 0 : \quad \forall (x_1, y_1), (x_2, y_2) : \sqrt{(x_2 - x_1)^2 + (y_2 - y_1)^2} < \delta \qquad |f_y'(x_2, y_2) - f_y'(x_1, y_1)| < \veps $$
	Пусть $ 0 < |h| < \delta $, тогда
	$$ \exist c \in (y \between y + h) : \quad f(x, y + h) - f(x, y) = f_y'(x, c)h $$
	\begin{equ}{par_int:18}
		f(x, y + h) - f(x, y) = f_y'(x, y)h + \bigg( f_y'(x, c) - f_y'(x, y) \bigg)h \define f_y'(x, y)h + r_h(x, y)h, \qquad |r_h(x, y)| < \veps
	\end{equ}
	\begin{multline*}
		I(y + h) - I(y) \bydef \dint{a}b{\bigg( f(x, y + h) - f(x, y) \bigg)} \undereq{\eref{par_int:18}} \\
		= \dint{a}b{f_y'(x, y)h} + \dint{a}b{r_h(x, y)h} = h\dint{a}b{f'(x, y)} + h\dint{a}b{r_h(x, y)}
	\end{multline*}
	$$ \bigg| h\dint{a}b{r_h(x, y)}\bigg| \le |h| \dint{a}b{|r_h(x, y)|} \le |h| \dint{a}b\veps = |h|\veps(b - a) $$
	Отсюда следует, что $ I(y) $ дифференцируема в $ y $ и выполнено утверждение теоремы
\end{proof}

\section{Теорема об интегрировании по параметру интеграла, зависящего от параметра}

\begin{theorem}
	$ \bm{f \in} \Cont{[a, b] \times [p, q]}, \qquad I(y) \define \dint{a}b{f(x, y)}, \quad K(x) \define \dint[y]{p}q{f(x, y)} $
	$$ \implies \dint[y]pq{I(y)} \bm= \dint{a}b{K(x)} $$
\end{theorem}

\begin{proof}
	По теореме о непрерывности интеграла, $ I(y) \in \Cont{[a, b]} $ \\
	Положим
	$$ \vphi(y_0) \define \dint[y]{p}{y_0}{I(y)}, \qquad v(y) \define \dint{a}b{l(x, y_0)}, \qquad l(x, y_0) \define \dint[y]{p}{y_0}{f(x, y)} $$
	$ \vphi \in \Cont{[p, q]} $, поскольку $ I(y) \in \Cont{[p, q]} $ \\
	Поскольку $ f \in \Cont{[a, b] \times [p, q]} $, то она ограничена (по первой теореме Вейерштрасса), \ie
	$$ \exist M : \quad \forall (x, y) \in [a, b] \times [p, q] \quad |f(x, y)| \le M $$
	Поэтому при $ y_1, y_2 \in [p, q] $ имеем
	\begin{multline}\lbl{par_int:24}
		|f(x, y_2) - l(x, y_1)| = \bigg| \dint[y]{p}{y_2}{f(x, y)} - \dint[y]{p}{y_1}{f(x, y)} \bigg| = \bigg| \dint[y]{y_1}{y_2}{f(x, y)} \bigg| \le \\
		\le \bigg| \dint[y]{y_1}{y_2}{|f(x, y)|} \bigg| \le |M(y_2 - y_1)| = M|y_2 - y_1|
	\end{multline}
	При фиксированном $ y_0 $ функция $ l(x, y_0) \in \Cont{[a, b]} $, поэтому, с учётом \eref{par_int:24} имеем
	$$ l(x, y_0) \in \Cont{[a, b] \times [p, q]} $$
	По определению $ l $, при фиксированном $ x $ получаем
	$$ l_{y_0}'(x, y) = f(x, y_0) \quad \implies \quad l_{y_0}'(x, y) \in \Cont{[a, b] \times [p, q]} $$
	$$ \implies \exist v'(y_0), \qquad v'(y_0) = \dint{a}b{l_{y_0}'(x, y_0)} = \dint{a}b{f(x, y_0)} = I(y_0) $$
	По определению $ \vphi $,
	$$ \exist \vphi'(y_0), \qquad \vphi'(y_0) = I(y_0) $$
	Из последних двух выражений следует, что
	$$ v'(y_0) = \vphi'(y_0), \qquad y_0 \in [p, q] $$
	Подставляя $ p $ вместо $ y_0 $ получаем
	$$ \vphi(p) = \dint[y]{p}p{I(y)} = 0, \qquad v(p) = \dint{a}b{l(x, p)} $$
	$$ f(x, p) = \dint[y]{p}p{f(x, y)} = 0 \quad \implies \quad v(p) = 0 $$
	\begin{multline*}
		\implies \dint[y]pq{I(y)} = \vphi(q) = \vphi(q) - \vphi(p) = \dint[y_0]pq{\vphi'(y_0)} = \dint[y_0]{p}q{v'(y_0)} = \\
		= v(q) - v(p) = v(q) = \dint{a}b{l(x, q)} = \dint{a}b{K(x)}
	\end{multline*}
\end{proof}

\section{Равномерная сходимость несобственного интеграла, зависящего от параметра; критерий Коши равномерной сходимости несобственного интеграла от параметра}

Пусть $ Y \sub \R^{n \ge 1}, \qquad f : [a, \infty) \times Y \to \R $ "--- семейство функций, $ \qquad f \in \Cont{[a, \infty) \times Y} $

Определим функцию $ F : Y \times [a, \infty) $:
$$ F(y, A) \define \dint{a}A{f(x, y)}, \qquad y \in Y, \quad A > a $$
Пусть
$$ \forall y \in Y \quad \exist \limi{A} F(y, A) \fed F_0(y) $$

\begin{definition}
	Будем говорить, что несобственный интеграл $ \dint{a}\infty{f(x, y)} $ равномерно сходится при $ y \in Y $, если
	$$ F(y, A) \uniarr[A \to \infty]{y \in Y} F_0(y) $$
\end{definition}

Применяя критерий Коши для семейства функций, получаем следующее утверждение:

\begin{theorem}
	Для того, чтобы несобственнный интеграл $ \dint{a}\infty{f(x, y)} $, зависящий от параметра, равномерно сходился при $ y \in Y $, \bt{необходимо и достаточно}, чтобы
	$$ \forall \veps > 0 \quad \exist L > a : \quad \forall A_1, A_2 > L \quad \forall y \in Y \quad \bigg| \dint{A_1}{A_2}{f(x, y)} \bigg| < \veps $$
\end{theorem}

\begin{proof}
	Заметим, что
	$$ F(y, A_2) - F(y, A_1) = \dint{a}{A_2}{f(x, y)} - \dint{a}{A_1}{f(x, y)} = \dint{A_1}{A_2}{f(x, y)} $$
\end{proof}

\section{Признак Вейерштрасса равномерной сходимости несобственного интеграла от параметра}

\begin{theorem}
	$ f \in \Cont{[a, \infty) \times Y} $
	\begin{equ}{par_int:35}
		\forall y \in Y \quad \bm{|f(x, y)| \le g(x)}
	\end{equ}
	$$ \dint{a}\infty{g(x)} \bm{< \infty} $$
	Тогда несобственный интеграл $ \dint{a}\infty{f(x, y)} $ \bt{сходится равномерно} при $ y \in Y $
\end{theorem}

\begin{proof}
	Возьмём $ \forall \veps > 0 $, выберем $ L $ так, чтобы $ \dint{L}\infty{g(x)} < \veps $. Тогда
	$$ \forall y \in Y \quad \forall A_1, A_2 > L \quad \bigg| \dint{A_1}{A_2}{f(x, y)} \bigg| \le \bigg| \dint{A_1}{A_2}{|f(x, y)|} \bigg| \underset{\eref{par_int:35}}\le \bigg| \dint{A_1}{A_2}{g(x)} \bigg| \le \dint{L}\infty{g(x)} < \veps $$
	По предыдущей теореме
	$$ \dint{a}A{f(x, y)} \uniarr[A \to \infty]{y \in Y} \dint{a}\infty{f(x, y)} $$
\end{proof}

\section{Признак Абеля равномерной сходимости несобственного интеграла от параметра}

\begin{theorem}
	$ f : [a, \infty) \times Y, \qquad Y \sub \R^n, \qquad f \in \Cont{[a, \infty) \times Y}, \qquad g : [a, \infty) \times Y \to \R $
	\begin{equ}{abel_par_int:1}
		\exist M : \quad \bm{|g(x, y)| \le M} \quad \forall x \in [a, \infty)
	\end{equ}
	$ \bm{g(x, y)} $ \bt{монотонна} по $ x $ при $ \forall y \in Y $
	\begin{equ}{abel_par_int:3}
		\dint{a}\infty{f(x, y)} \text{ \bt{равномерно сх.} при } y \in Y
	\end{equ}
	$$ \implies \dint{a}\infty{f(x, y)g(x, y)} \text{ \bt{равномерно сх.} при } y \in Y $$
\end{theorem}

\begin{proof}
	Возьмём $ \forall \veps > 0 $ и воспользуемся критерием Коши для несобственных интегралов (условие \eref{abel_par_int:3}):
	\begin{equ}{abel_par_int:4}
		\exist A > a : \quad \forall A_2 > A_1 > A \quad \forall y \in Y \quad \bigg| \dint{A_1}{A_2}{f(x, y)} \bigg| < \veps
	\end{equ}
	Применим вторую теорему о среднем:
	$$ \dint{A_1}{A_2}{f(x, y)g(x, y)} = g(A, y) \dint{A_1}c{f(x, y)} + g(A_2, y)\dint{c}{A_2}{f(x, y)} $$
	\begin{multline*}
		\implies \bigg| \dint{A_1}{A_2}{f(x, y)g(x, y)} \bigg| \le |g(A, y)| \cdot \bigg| \dint{A_1}c{F(x, y)} \bigg| + |g(A_2, y)| \cdot \bigg| \dint{c}{A_2}{f(x, y)}\bigg| \underset{\eref{abel_par_int:1}, \eref{abel_par_int:4}}\le \\
		\le M \cdot \veps + M \cdot \veps = 2M\veps
	\end{multline*}
\end{proof}

\section{Признак Дирихле равномерной сходимости несобственного интеграла от параметра}

\begin{theorem}
	$ f \in \Cont{[a, \infty) \times Y}, \qquad g : [a, \infty) \times Y \to \R, \qquad \bm g $ \bt{монотонна} по $ x $ при $ \forall y \in Y $
	\begin{equ}{dirichle_par_int:7}
		g(x, y) \uniarr[x \to \infty]{y \in Y} 0
	\end{equ}
	\begin{equ}{dirichle_par_int:8}
		\exist L > 0 : \quad \forall A > a \quad \forall y \in Y \quad \bigg| \dint{a}A{f(x, y)} \bigg| \bm{\le L}
	\end{equ}
	$$ \implies \dint{a}\infty{f(x, y)g(x, y)} \text{ \bt{сходится равномерно} при } y \in Y $$
\end{theorem}

\begin{proof}
	\begin{multline}\lbl{dirichle_par_int:10}
		\forall A_2 > A_1 > a \quad \bigg| \dint{A_1}{A_2}{f(x, y)} \bigg| = \bigg| \dint{a}{A_2}{f(x, y)} - \dint{a}{A_1}{f(x, y)} \bigg| \trile \\
		\le \bigg| \dint{a}{A_2}{f(x, y)} \bigg| - \bigg| \dint{a}{A_1}{f(x, y)} \bigg| \underset{\eref{dirichle_par_int:8}}\le 2L
	\end{multline}
	По определению равномерной сходимости,
	\begin{equ}{dirichle_par_int:11}
		\eref{dirichle_par_int:7} \implies \exist A > a : \quad \forall A_1 > A \quad \forall y \in Y \quad |g(A_1, y)| < \veps
	\end{equ}
	Возьмём $ A_2 > A_1 > A $ \\
	Воспользуемся второй теоремой о среднем:
	$$ \exist c \in (A_1, A_2) : \quad \dint{A_1}{A_2}{f(x, y)g(x, y)} = g(A_1, y)\dint{A_1}c{f(x, y)} + g(A_2, y)\dint{c}{A_2}{f(x, y)} $$
	\begin{multline*}
		\implies \bigg| \dint{A_1}{A_2}{f(x, y)g(x, y)} \bigg| \le |g(A_1, y)| \cdot \bigg| \dint{A_1}c{F(x, y)} \bigg| + |g(A_2, y)| \cdot \bigg| \dint{c}{A_2}{f(x, y)} \bigg| \underset{\eref{dirichle_par_int:10}, \eref{dirichle_par_int:11}}\le \\
		\le 2L\veps + 2L\veps = 4L\veps
	\end{multline*}
\end{proof}

\section{Предел несобственного интеграла, зависящего от параметра}

\begin{theorem}
	$ Y \sub \R^n, \qquad y_0 $ "--- т. сг. $ Y $ \nimp[(не обязательно $ \in Y $)], $ \qquad f \in \Cont{[a, \infty) \times Y} $
	\begin{equ}{uni_int:14}
		\bm{I(y)} \define \dint{a}\infty{f(x, y)} \text{ \bt{равномерно сходится} при } y \in Y
	\end{equ}
	\begin{equ}{uni_int:15}
		\forall x \in [a, \infty) \quad \exist \vphi(x) : \quad \bm{f(x, y)} \uniarr[y \to y_0]{x \in [a, \infty)} \bm{\vphi(x)}
	\end{equ}
	$$ \implies \dint{a}\infty{\vphi(x)} \bt{ сходится}, \qquad \bm{I(y) \underarr{y \to y_0}} \dint{a}\infty{\vphi(x)} $$
\end{theorem}

\begin{proof}
	Рассмотрим функции:
	$$ F(A, y) \define \dint{a}{A}{f(x, y)}, \qquad \Phi(A) \define \dint{a}{A}{\vphi(x)} $$
	$ \eref{uni_int:15} \implies \vphi \in \Cont{[a, \infty)}, $ значит, функция $ \Phi $ корректно определена
	\begin{equ}{uni_int:14'}
		\eref{uni_int:14} \iff F(A, y) \uniarr[A \to \infty]{y \in Y} I(y)
	\end{equ}
	Можно применить теорему о пределе интеграла от параметра:
	$$ \eref{uni_int:15} \implies \forall A > a \quad \dint{a}{A}{f(x, y)} \underarr{y \to y_0} \dint{a}{A}{\vphi(x)} $$
	В обозначениях $ F, \Phi $ это означает, что
	\begin{equ}{uni_int:19'}
		F(A, y) \underarr{y \to y_0} \Phi(A)
	\end{equ}
	Можно применить теорему о предельном переходе в функциональном семействе:
	$$ \eref{uni_int:14'}, \eref{uni_int:19'} \implies \exist \liml{y \to y_0} I(y), \qquad \exist \liml{A \to \infty} \Phi(A), \qquad \limi{A} \Phi(A) = \liml{y \to y_0} I(y) $$
	При этом,
	$$ \dint{a}\infty{\vphi(x)} \bdefeq{\Phi} \limi{A} \Phi(A) $$
\end{proof}

\section{Определённый интеграл от интеграла, зависящего от параметра}

\begin{theorem}
	$ f \in \Cont{[a, \infty) \times [p, q]} $
	$$ I(y) \define \dint{a}\infty{f(x, y)} \text{ \bt{равномерно сх.} при } y \in [p, q] $$
	По последнему следствию, $ I(y) \in \Cont{[p, q]} $, и  можно рассматривать $ \dint[y]pq{I(y)} $
	$$ K(y) \define \dint[y]{p}{q}{f(x, y)} $$
	$ K(y) $ "--- собственный интеграл от параметра, значит $ k \in \Cont{[a, \infty)} $
	$$ \implies \dint{a}\infty{K(x)} \bt{ сходится}, \qquad \dint[y]{p}{q}{I(y)} \bm= \dint{a}\infty{K(x)}, $$
	или
	$$ \dint[y]{p}q{\bigg( \dint{a}\infty{f(x, y)} \bigg)} \bm= \dint{a}\infty{ \bigg( \dint[y]pq{f(x, y)} \bigg)} $$
\end{theorem}

\begin{proof}
	Рассмотрим функцию
	$$ F(A, y) = \dint{a}A{f(x, y)} $$
	$ F \in \Cont{[a, A] \times [p, q]} $, значит, по теореме об интегрировании \nimp[``собственного''] интеграла от параметра,
	$$ \dint[y]pq{F(A, y)} \bdefeq{F} \dint[y]pq{ \bigg( \dint{a}{A}{f(x, y)} \bigg)} = \dint{a}{A}{ \bigg( \dint[y]pq{f(x, y)} \bigg)} \bdefeq{K} \dint{a}{A}{K(x)} $$
	По условию,
	$$ F(y) \uniarr[A \to \infty]{y \in [p, q]} I(y) \underimp{\text{т. о переходе к пределу}} \dint[y]pq{F(y)} \infarr{A} \dint[y]pq{I(y)} $$
	$$ \implies \exist \limi{A} \dint{a}A{K(x)} \bydef \dint{a}\infty{K(x)} $$
\end{proof}

\section{Производная несобственного интеграла, зависящего от параметра}

\begin{theorem}
	$ f \in \Cont{[a, \infty) \times [p, q]}, \qquad \forall x \in [a, \infty) \quad \forall y \in [p, q] \quad \exist \bm{f_y'(x, y) \in} \Cont{[a, \infty) \times [p, q]} $
	$$ \forall y \in [p, q] \quad \bt{ сходится } I(y) \define \dint{a}\infty{f(x, y)} $$
	\begin{equ}{der_int:38}
		\dint{a}\infty{f_y'(x, y)} \text{ \bt{равномерно сходится} при } y \in [p, q]
	\end{equ}
	$$ \implies \forall y \in [p, q] \quad \exist \bm{I'(y) =} \dint{a}\infty{f_y'(x, y)} $$
\end{theorem}

\begin{proof}
	Зафиксируем $ y_0 \in [p, q] $ \\
	Обозначим $ Y \define [p, q] \setminus \set{y_0} $

	Рассмотрим функции
	$$ F(A, y) \define \dint{a}A{f(x, y)}, \qquad G(A, y) \define \frac{F(A, y) - F(A, y_0)}{y - y_0} $$

	$ f $ удовлетворяет требованиям, которые накладывались на функцию в теореме о производной интеграла от параметра:
	\begin{equ}{der_int:312}
		\exist \liml{y \to y_0} G(A, y) \bydef \liml{y \to y_0} \frac{F(A, y) - F(A, y_0)}{y - y_0} \bdefeq{F_y'} F_y'(A, y_0) \undereq{\text{упомятнутая теорема}} \dint{a}A{f_y'(x, y_0)}
	\end{equ}
	(для любого фиксированного $ A > a $)

	\begin{statement}
		\begin{equ}{der_int:314}
			\exist \Phi(y), \quad y \in Y : \quad G(A, y) \uniarr[A \to \infty]{y \in Y} \Phi(y)
		\end{equ}
	\end{statement}
	\begin{proof}
		Применим критерий Коши к условию \eref{der_int:38}:
		\begin{equ}{der_int:315}
			\forall \veps > 0 \quad \exist A > a : \quad \forall A_2 > A_1 > A \quad \forall y \in [p, q] \quad \bigg| \dint{A_1}{A_2}{f_y'(x, y)} \bigg| < \veps
		\end{equ}
		\begin{multline}\lbl{der_int:316}
			G(A_2, y) - G(A_1, y) \bdefeq{G} \frac{F(A_2, y) - F(A_2, y_0)}{y - y_0} - \frac{F(A_1, y) - F(A_1, y_0)}{y - y_0} = \\
			= \frac{\bigg( F(A_2, y) - F(A_1, y) \bigg) - \bigg( F(A_2, y_0) - F(A_1, y_0) \bigg)}{y - y_0} \bdefeq{F} \frac{\dint{A_1}{A_2}{f(x, y)} - \dint{A_1}{A_2}{f(x, y_0)}}{y - y_0}
		\end{multline}
		Определим (учитывая, что $ A_1, A_2 $ фиксированы)
		$$ V(y) \define \dint{A_1}{A_2}{F(x, y)} $$
		\begin{equ}{der_int:317}
			\forall y \in [p, q] \quad \exist V'(y), \qquad V'(y) = \dint{A_1}{A_2}{f_y'(x, y)}
		\end{equ}
		По теореме Лагранжа
		\begin{equ}{der_int:318}
			\exist c \in (y, \between y_0) : \quad V(y) - V(y_0) = V'(c)(y - y_0)
		\end{equ}
		$$ G(A_2, y) - G(A_1, y) \xlongequal[\eref{der_int:316}]{\operatorname{def} V} \frac{V(y) - V(y_0)}{y - y_0} \undereq{\eref{der_int:318}} \frac{V'(c)(y - y_0)}{y - y_0} = V'(c) \undereq{\eref{der_int:317}} \dint{A_1}{A_2}{f_y'(x, c)} $$
		$$ \implies |G(A_2, y) - G(A_1, y)| = \bigg| \dint{A_1}{A_2}{f_y'(x, c)} \bigg| \underset{\eref{der_int:315}}< \veps $$
	\end{proof}

	Применим теорему о предельном переходе в функциональном семействе:
	\begin{equ}{der_int:323}
		\eref{der_int:312}, \eref{der_int:314} \implies
		\begin{cases}
			\exist \liml{y \to y_0} \Phi(y) = \dint{a}\infty{f_y'(x, y_0)} \\
			\exist \limi{A} \dint{a}A{f_y'(x, y_0)} = \dint{a}\infty{f_y'(x, y_0)}
		\end{cases}
	\end{equ}
	$$ G(A, y) \xrightarrow[A \to \infty]{\operatorname{def} G, F} \frac{I(y) - I(y_0)}{y - y_0} $$
	То есть,
	$$ \Phi(y) = \frac{I(y) - I(y_0)}{y - y_0} $$
	Вместе с \eref{der_int:323}, получаем утверждение теоремы
\end{proof}

\section{Несобственный интеграл по параметру от несобственного интеграла от параметра}

\begin{theorem}
	$ f \in \Cont{[a, \infty) \times [p, \infty)} $
	$$ I(y) \define \dint{a}\infty{f(x, y)}, \qquad K(x) = \dint[y]p\infty{f(x, y)} $$
	\bt{Существует} по крайней мере \bt{один из интегралов}:
	$$ \dint[y]p\infty{I(y)}, \qquad \dint{a}\infty{K(y)} $$
	Тогда \bt{существует и второй}, и справедливо равенство:
	$$ \dint[y]p\infty{I(y)} \bm= \dint{a}\infty{K(x)} $$
	то есть,
	$$ \dint[y]p\infty{\bigg( \dint{a}\infty{f(x, y)} \bigg)} \bm= \dint{a}\infty{\bigg( \dint[y]p\infty{f(x, y)} \bigg)} $$
\end{theorem}

\begin{remark}
	Важность этой теоремы заключается в том, что в ней не требуется равномерная сходимость
\end{remark}

\begin{proof}
	Будет доказано в четвёртом семестре
\end{proof}

\section{Вычисление интеграла Дирихле}

\begin{theorem}
	$$ \dint0\infty{\frac{\sin x}x} = \half[\pi] $$
\end{theorem}

\begin{proof}
	$$ f(x, y) \define \frac{\sin x}x, \qquad x \in [0, \infty), \quad y \in [0, \infty) $$
	Интеграл, не зависящий от $ y $ равномерно сходится:
	$$ \dint0\infty{f(x, y)} = \dint0\infty{\frac{\sin x}x} \text{ равн. сх. при } y \in [0, \infty) $$
	$$ g(x, y) \define e^{-xy} \text{ монот. по } x \text{ при } y \in [0, \infty) $$
	$$ 0 < e^{-xy} \le 1 $$
	Применяем признак Абеля:
	$$ I(y) \define \dint0\infty{\frac{\sin x}xe^{-xy}} \text{ равн. сх. при } y \in [0, \infty) $$
	$$ h(x, y) \define \frac{\sin x}xe^{-xy} \in \Cont{[0, \infty) \times [0, \infty)} $$
	Применим частный случай перехода к пределу к последним двум выражениям:
	$$ I(0) = \liml{y \to +0} I(y) $$
	Возьмём $ y \ge \delta > 0 $
	$$ \bigg| \frac{\sin x}xe^{-xy} \bigg| \le e^{-\delta x} $$
	$$ \implies h_y'(x, y) = -\sin xe^{-xy} \quad \implies |h_y'(x, y)| \le e^{-xy} \le e^{-\delta x} $$
	Применим теорему о производной несобственного интеграла от параметра:
	$$ \dint0\infty{\frac{\sin x}xe^{-xy}} \text{ сходится}, \qquad \dint0\infty{h_y'(x, y)} \text{ равномерно сходится} $$
	То есть, $ \forall A > \delta \quad \exist I'(y) $ при $ y \in [\delta, A] $
	$$ \implies \forall y \ge \delta \quad \exist I'(y), \qquad I'(y) = \dint0\infty{h_y'(x, y)} = -\dint0\infty{\sin xe^{-xy}} $$
	Проинтегрируем по частям:
	\begin{multline*}
		-\dint0\infty{\sin xe^{-xy}} = \dint0\infty{e^{-xy}(\cos x)'} = e^{-xy}\cos \clamp[\infty]0 - \dint0\infty{\cos x(e^{-xy})_x'} = \\
		= -1 + y\dint0\infty{e^{-xy}\cos x} \undereq{\text{по частям}} -1 + y\dint0\infty{e^{-xy}(\sin x)'} = \\
		= -1 + y \bigg( \underbrace{e^{-xy}\sin x \clamp[\infty]0}_{=0} - \dint0\infty{\sin x(e^{-xy})'} \bigg) = -1 + y^2\dint0\infty{e^{-xy}\sin x}
	\end{multline*}
	$$ \implies -(1 + y^2)\dint0\infty{e^{-xy}\sin x} = -1 \quad \implies \quad -\dint0\infty{e^{-xy}\sin x} = -\frac1{1 + y^2} $$
	\begin{equ}{dirichle_int:412}
		\iff I'(y) = -\frac1{1 + y^2}
	\end{equ}
	Применим формулу Ньютона"--~Лейбница:
	\begin{equ}{dirichle_int:413}
		I(B) - I(\delta) = \dint[y]\delta{B}{I'(y)} \undereq{\eref{dirichle_int:412}} -\dfint[y]\delta{B}{1 + y^2} = -\arctg y\clamp[B]\delta = \arctg \delta - \arctg B
	\end{equ}
	При $ B \to \infty $
	$$ |I(B)| \le \dint0\infty{e^{-Bx}} = \frac1B \infarr{B} 0 $$
	$$ \eref{dirichle_int:413} \implies 0 - I(\delta) = \arctg \delta - \half[\pi] \quad \implies \quad \implies I(\delta) = \half[\pi] - \arctg \delta $$
	$$ \dint0\infty{\sin x} = I(0) = \liml{\delta \to +0} I(\delta) = \lim \bigg( \half[\pi] - \arctg \delta \bigg) = \half[\pi] $$
\end{proof}

\section{Вычисление интеграла Эйлера\tpst{"--~}{--}Пуассона}

\begin{theorem}
	$$ E \define \dint0\infty{e^{-x^2}} = \frac{\sqrt\pi}2 $$
\end{theorem}

\begin{proof}
	Рассмотрим $ f(x, y) \define e^{-x^2y - y}, \quad x \ge 0, \quad y \ge 0 $ \\
	Понятно, что $ f(x, y) \in \Cont{[0, \infty) \times [0, \infty)} $

	Пусть
	$$ I(y) \define \dint0\infty{f(x, y)}, \qquad K(x) \define \dint[y]0\infty{f(x, y)} $$
	$$ I(y) = \dint0\infty{e^{-x^2y - y}} \undereq{
		\begin{subarray}c
			x_1 \define x\sqrt y \\
			\di x = \frac1{\sqrt y}\di x_1
		\end{subarray}} \dint[x_1]0\infty{e^{-x_1^2} \cdot \frac1{\sqrt y} \cdot e^{-y}} = \frac1{\sqrt y} e^{-y} \dint[x_1]0\infty{e^{-x_1^2}} = \frac1{\sqrt y}e^{-y}E $$
	$$ \dint[y]0\infty{I(y)} = E \dint[y]0\infty{\frac1{\sqrt y}e^{-y}} \undereq{
		\begin{subarray}c
			y \define y_1^2 \\
			\di y = 2y_1\di y_1
		\end{subarray}} E \cdot \dint[y_1]0\infty{\frac1{y_1}e^{-y_1^2} \cdot 2y_1} = 2E \cdot \dint[y_1]0\infty{e^{-y_1^2}} = 2E^2 $$
	Применим теорему об интегрировании несобственного интеграла по параметру:
	\begin{equ}{euler_puasson:53}
		\implies 2E^2 = \dint[y]0\infty{I(y)} = \dint0\infty{K(x)}
	\end{equ}
	$$ K(x) = \dint[y]0\infty{e^{-x^2y - y}} = \dint[y]0\infty{e^{-(x^2 + 1)y}} \undereq{t = (x^2 + 1)y} \frac1{1 + x^2} \cdot \dint[t]0\infty{e^{-t}} = \frac1{1 + x^2} $$
	$$ \underimp{\eref{euler_puasson:53}} 2E^2 = \dfint0\infty{1 + x^2} = \frac\pi2 $$
	При этом, $ E > 0 $.
\end{proof}

\section{Числовые и функциональные ряды с комплексными слагаемыми; абсолютная и равномерная сходимость; признак Вейерштрасса равномерной сходимости}

\begin{definition}
	Если $ c_n = a_n + ib_n, \quad n \ge 1, \quad a_n, b_n \in \R $, то \it{рядом с комплексными слагаемыми} называется символ
	\begin{equ}{comp_series:1}
		\sum_{n = 1}^\infty c_n
	\end{equ}
\end{definition}

\begin{definition}
	Ряд \eref{comp_series:1} по определению \it{сходится}, если сходятся ряды $ \sum a_n, ~ \sum b_n $, при этом
	$$ \sum_{n = 1}^\infty c_n \define \sum_{n = 1}^\infty a_n + i \sum_{n = 1}^\infty b_n $$
\end{definition}

\begin{definition}
	Ряд \eref{comp_series:1} называют \it{абсолютно сходящимся}, если сходится ряд $ \sum |c_n| $.
\end{definition}

\begin{remind}
	$ |a + bi| = \sqrt{a^2 + b^2} $
\end{remind}

\begin{statement}
	Чтобы ряд \eref{comp_series:1} абсолютно сходился, необходимо и достаточно, чтобы абсолютно сходились ряды $ \sum a_n $ и $ \sum b_n $.
\end{statement}

\begin{iproof}
	\item Поскольку $ |a_n| \le |c_n| $ и $ |b_n| \le |c_n| $, то абсолютная сходимость ряда \eref{comp_series:1} влечёт абсолютную сходимость рядов $ \sum a_n $ и $ \sum b_n $.

	\item Поскольку $ |c_n| \le |a_n| + |b_n| $, то абсолютная сходимость рядов $ \sum a_n $ и $ b_n $ влечёт абсолютную сходимость ряда \eref{comp_series:1}.
\end{iproof}

\begin{implication}
	Если ряд \eref{comp_series:1} абсолютно сходится, то он сходится.
\end{implication}

\begin{proof}
	Следует из факта, что абсолютно сходящийся вещественный ряд сходится.
\end{proof}

\begin{definition}
	$ E \ne \O \sub \Co, \qquad u_n : E \to \R, \quad v_n : E \to \R, \quad n = 1, 2, \dots $

	\it{Функциональным комплекснозначным рядом} будем называть символ
	$$ \sum_{n = 1}^\infty w_n(x), \qquad w_n(x) = u_n(x) + iv_n(x), \quad x \in E $$
\end{definition}

\begin{definition}
	Ряд $ \sum w_n $ будем называть \it{равномерно сходящимся} на $ E $, если равномерно сходятся на $ E $ ряды $ \sum u_n $ и $ \sum v_n $.
\end{definition}

\begin{theorem}[критерий Коши]
	Для того чтобы ряд $ \sum \gamma_n $ \bt{равномерно} сходился на $ E $, \bt{необходимо и достаточно}, чтобы
	$$ \forall \veps > 0 \quad \exist N : \quad \forall m > n > N \quad \forall x \in E \quad |\gamma_{n + 1}(x) + \dots + \gamma_m(x)| < \veps $$
\end{theorem}

\begin{proof}
	Можно применить критерий Коши для комплексной функциональной последовательности
\end{proof}

\begin{theorem}[признак Вейерштрасса]
	$ \seq{a_n}n, \quad a_n > 0 : \quad \bm{|\gamma_n(x)| \le a_n} \quad \forall n \quad \forall x \in E $ \\
	$ \bm{\sum a_n} $ \bt{сходится}
	\begin{equ}{comp_series:22}
		\sum_{n = 1}^\infty a_n < \infty
	\end{equ}
	$$ \implies \sum_{n = 1}^\infty \gamma_n \bt{ сходится равномерно} $$
\end{theorem}

\begin{proof}
	Возьмём $ \forall \veps > 0 $
	$$ \eref{comp_series:22} \implies \exist N : \quad \forall m > n > N \quad a_{n + 1} + \dots + a_n < \veps $$
	$$ |\gamma_{n + 1}(x) + \dots + \gamma_n(x)| \le |\gamma_{n + 1}(x)| + \dots + |\gamma_m(x)| \underset{a_n > 0}\le a_{n + 1} + \dots + a_m < \veps $$
	По критерию Коши получаем равномерную сходимость.
\end{proof}

\section{Степенные ряды; лемма Абеля}

$$ E = \Co, \qquad \seqz{c_n}n, \qquad c_n \in \Co, \qquad z_0 \in \Co $$
Положим $ \gamma_0(z) \define c_0, \quad \gamma_n(z) \define c_n(z - z_0)^n $
\begin{equ}{comp_power_series:32}
	c_0 + \sum_{n = 1}^\infty c_n(z - z_0)^n
\end{equ}
Такое выражение будем называть \it{комплексным степенным рядом} с центром $ z_0 $.

\begin{remark}
	$ \seq{c_n}n, \qquad c_n \to c \in \Co $
	$$ \implies \exist M : \quad |c_n| \le M \quad \forall n $$
\end{remark}

\begin{proof}
	Положим $ c_n = a_n + ib_n, \quad c = a + ib $
	$$ a_n \to a, \qquad b_n \to b $$
	Дальше применяем теорему из первого семестра
\end{proof}

\begin{remark}[необходимый признак сходимости комплексных числовых рядов]
	$$ \sum_{n = 1}^\infty \gamma_n \text{ сх. } \implies \gamma_n \infarr{n} 0 $$
\end{remark}

\begin{proof}
	$ c_n = \gamma_1 + \dots + \gamma_n $
	$$
	\begin{rcases}
		c_n \to c \\
		c_{n - 1} \to c
	\end{rcases} \implies \underbrace{c_n - c_{n - 1}}_{\gamma_n} \to c - c = 0 $$
\end{proof}

\begin{implication}
	$ \exist M : \quad |\gamma_n| \le M \quad \forall n $
\end{implication}

\begin{lemma}[Абеля]
	$ \exist z_1 \ne z_0 : \quad \eref{comp_power_series:32} \text{ сходится при } z_1, \qquad R \define |z_1 - z_0| $
	$$ \implies \eref{comp_power_series:32} \text{ сх. } \quad \forall z : |z - z_0| < R $$
	\begin{equ}{comp_power_series:34}
		\implies \forall 0 < r < R \quad \eref{comp_power_series:32} \text{ равн. сх. при } |z - z_0| \le r
	\end{equ}
\end{lemma}

\begin{proof}
	Докажем \eref{comp_power_series:34}: \\
	Обозначим $ 0 < q \define \frac r R < 1 $ \\
	Сходимость при $ z_1 $, по необходимому признаку, означает, что
	$$ c_n(z_1 - z_0)^n \infarr{n} 0 $$
	Тогда, по следствию,
	\begin{equ}{comp_power_series:37}
		\exist M : \quad |c_n(z_1 - z_0)^n| \le M \quad \iff \quad |c_n| \cdot |z_1 - z_0|^n \le M \overset{\operatorname{def} R}\iff |c_n| \le \frac{M}{R^n}
	\end{equ}
	$$ |c_n(z - z_0)^n| = |c_n| \cdot |z - z_0|^n \underset{\eref{comp_power_series:37}, \operatorname{def} r}\le \frac{M}{R^n} \cdot r^n = Mq^n $$
	$$ \sum_{n = 1}^\infty Mq^n = \frac{Mq}{1 - q} $$
	Можно применить признак Вейерштрасса, тем самым доказывая \eref{comp_power_series:34}
	$$ \eref{comp_power_series:34} \implies \eref{comp_power_series:32} \text{ сх. абс. при } |z - z_0| < R $$
\end{proof}

\section{Определение радиуса сходимости и круга сходимости степенного ряда}

\begin{definition}
	\hfill
	\begin{enumerate}
		\item Пусть \eref{comp_power_series:32} сходится только при $ z = z_0 $ \\
		Будем полагать радиус сходимости $ R \define 0 $, круг сходимости $ \B \define \O $
		\item \eref{comp_power_series:32} сходится при всех $ z $ \\
		Полагаем $ R \define +\infty, \quad \B \define \Co $
		\item $ \exist z_1 \ne z_0 : \quad \eref{comp_power_series:32} \text{ сх. в } z_1, \qquad \exist z_2 : \quad \eref{comp_power_series:32} \text{ расх. в } z_2 $
		$$ R \define \sup\set{r \mid r = |z_* - z_0|, \quad \eref{comp_power_series:32} \text{ сх. в } z_*}, \qquad \B \define \set{z_0 \mid |z - z_0| < R} $$
	\end{enumerate}
\end{definition}

Положим $ r_1 \define |z_1 - z_0|, \quad r_2 \define |z_2 - z_1| $ \\
По определению $ R $
$$ R \ge r_1 > 0 $$
Возьмём $ z_3 : \quad r_3 \define |z_3 - z_0| > r_2 $ \\
Если бы \eref{comp_power_series:32} сходился при $ z_3 $, можно было бы применить к $ z_3 $ лемму Абеля. Тогда бы \eref{comp_power_series:32} сходился в $ z_2 $ "--- \contra \\
То есть, в $ z_3 $ ряд расходится \\
Значит, $ R \le r_2, \quad r_1 < r_2 $

\section{Свойства круга сходимости}

Рассматриваем только случай, когда $ 0 < R < \infty $

\begin{theorem}
	$$ \eref{comp_power_series:32} \text{ сх. } \quad \forall z \in \B $$
	$$ \eref{comp_power_series:32} \text{ расх. } \quad \forall z_2 \in \Co \setminus \ol{B} $$
\end{theorem}

\begin{iproof}
	\item Возьмём $ r \define |z - z_0| < R $ \\
	По определению $ R $
	$$ \exist z_* : \quad |z_* - z_0| > R, \quad \eref{comp_power_series:32} \text{ сх. в } z_* $$
	По лемме Абеля \eref{comp_power_series:32} сх. в $ z $
	\item Возьмём $ \rho \define |\hat z - z_0| > R $ \\
	Если ряд сходится, то $ \rho $ больше супремума, что невозможно
\end{iproof}

\section{Вычисление радиуса сходимости}

\begin{theorem}
	Определим
	\begin{equ}{comp_power_series:314}
		t \define \ulim_{n \to \infty} \sqrt[n]{c_n}
	\end{equ}
	\begin{enumerate}
		\item $ R = 0 $, если $ t = +\infty $
		\item $ R = +\infty $, если $ t = 0 $
		\item $ R = \frac1t $ иначе
	\end{enumerate}
\end{theorem}

\begin{proof}
	Будем рассматривать только последний случай \\
	Определим $ R_0 \define \frac1t $
	\begin{itemize}
		\item Возьмём $ z_2 : \quad |z_2 - z_0| > R_0 $ \\
		Обозначим $ \veps \define |z_2 - z_0| - R_0 > 0 $ \\
		Определим
		$$ \delta \define \frac{\veps t^2}{1 + \veps t} $$
		По определению верхнего предела
		$$ \exist \seq{n_k}k : \quad \sqrt[n_k]{c_{n_k}} > t - \delta \quad \iff \quad |c_{n_k}| > (t - \veps)^{n_k} $$
		$$ \implies |c_{n_k}(z_2 - z_0)^{n_k}| = |c_{n_k}| \cdot |z_2 - z_0|^{n_k} > (t - \delta)^{n_k} \cdot (R_0 + \veps)^{n_k} = \bigg( (t - \delta)(R_0 + \veps) \bigg)^{n_k} $$
		$$ (t - \delta)(R_0 + \veps) = \bigg( t - \frac{\veps t^2}{1 + \veps t} \bigg)\bigg( \frac1t + \veps \bigg) = \frac{t + \veps t^2 - \veps t^2}{1 + \veps t} \cdot \frac{1 + \veps t}t = 1 $$
		$$ \implies |c_{n_k}(z_2 - z_0)^{n_k}| \ge 1 $$
		По второму замечанию ряд в $ z_2 $ расходится
		\item Возьмём $ z_1 : \quad |z_1 - z_0| < R_0 $ \\
		Пусть
		$$ \veps_0 \define R_0 - |z_1 - z_0|, \quad \delta_0 \define \half \cdot \frac{\veps_0t^2}{1 - \veps_0t} $$
		По свойствам верхнего предела
		$$ \exist N : \quad \forall n > N \quad \sqrt[n]{|c_n|} < t + \delta_0 \quad \iff \quad |c_n| < (t + \delta_0)^n $$
		$$ \implies \forall n > N \quad |c_n(z_1 - z_0)^n| = |c_n| \cdot |z_1 - z_0|^n < (t + \delta_0)^n \cdot (R_0 - \veps_0)^n = \bigg( (t - \delta_0)(R_0 - \veps_0) \bigg)^n $$
		$$ (t + \delta_0)(R_0 - \veps_0) \bdefeq{\delta} \bigg( t + \half \cdot \frac{\veps_0t^2}{1 - \veps_0t} \bigg) \bigg( \frac1t - \veps_0 \bigg) = \frac{t - \veps_0t^2 + \half\veps_0t^2}{1 - \veps_0t} \cdot \frac{1 - \veps_0t}{t} = 1 - \half\veps_0t $$
		$$ 0 < q \define 1 -\half \veps_0t < 1 \quad \implies \quad |c_n(z_1 - z_0)^n| < q^n < 1 $$
		Значит, ряд сходится при $ z_1 $
	\end{itemize}
\end{proof}

\begin{theorem}
	$ c_n \ne 0 \quad \forall n, \qquad \exist \limi{n} \frac{|c_n|}{|c_{n + 1}|} $ \\
	Тогда этот предел и равен радиусу сходимости
\end{theorem}

\begin{proof}
	Аналогично.
\end{proof}

\section{Интервал сходимости вещественного степенного ряда, его\n свойства}

\begin{definition}
	\begin{equ}{real_power_series:1}
		S(x) \define \sum_{n = 1}^\infty a_n(x - x_0)^n
	\end{equ}
	Будем называть $ S(x) $ вещественным степенным рядом, если $ x_0 \in \R, \quad a_n \in \R, ~ n \ge 1, \quad x \in \R $
\end{definition}

Можно найти радиус сходимости и круг сходимости соответствующего комплексного степенного ряда:

\begin{enumerate}
	\item $ R = 0, \quad \B = \O $ \\
	Ряд сходится только при $ x = x_0 $
	\item $ R = \infty, \quad \B = \Co $ \\
	Ряд сходится при любых $ z \in \Co $, а значит, и при любых $ x \in \R $
	\item $ 0 < R < \infty, \quad \B \ne \O, \Co $ \\
	Пусть $ I \define \B \cap \R $
	$$ I = (x_0 - R, x_0 + R) $$
	\begin{itemize}
		\item $ x \in I \implies x \in \B \implies $ ряд сходится в $ x $
		\item $ x_1 \nin \ol I \implies x_1 \nin \ol\B \implies $ ряд расходится в $ x_1 $
	\end{itemize}
	Пусть есть $ 0 < r < R $ \\
	Рассмотрим промежуток $ [x_0 - r, x_0 + r] \sub \B \quad \implies $ ряд сходится равномерно на $ [x_0 - r, x_0 + r] $
\end{enumerate}

При доказательстве теоремы о радиусе сходимости для комплексных рядов мы пользовались признаком Вейерштрасса. Если перейти к вещественным рядам, то при $ x \in [x_0 - r, x_0 + r] $ будет равномерно сходится ряд
$$ \sum_{n = 1}^\infty |a_n(x - x_0)^n| $$

\section{Теорема Абеля о вещественном степенном ряде}

Бывает, что при $ r = R $ ряд сходится

\begin{theorem}[Абеля]
	Ряд $ S(x) $ сходится при $ x_0 - R $ или при $ x_0 + R $
	$$ S(x) \define \sum_{n = 1}^\infty a_n(x - x_0)^n $$
	Тогда ряд сходится равномерно на $ [x_0 - R, x_0] $ или $ [x_0, x_0 + R] $, и
	$$ \implies
	\begin{vars}
		S \in \Cont{[x_0 - R, x_0]} \\
		S \in \Cont{[x_0, x_0 + R]}
	\end{vars} $$
	Если ряд сходится и при $ x_0 - R $, и при $ x_0 + R $, то верны оба утверждения
\end{theorem}

\begin{proof}
	Докажем для $ [x_0 - R, x_0] $: \\
	Так как $ x_0 - R - x_0 = -R, \quad \sum_{n = 1}^\infty a_n(-R)^n $ сходится.
	Пусть $ x_0 - R < x < x_0 $
	$$ \sum_{n = 1}^\infty a_n(x - x_0)^n = \sum_{n = 1}^\infty a_n(-R)^n \cdot \bigg( \frac{x - x_0}{-R} \bigg)^n = \sum_{n = 1}^\infty (-R)^n \cdot \bigg( \frac{x_0 - x}R \bigg)^n $$
	Положим
	$$ u_n(x) \define a_n(-R)^n, \qquad v_n(x) \define \bigg( \frac{x_0 - x}R \bigg)^n $$
	Тогда $ \sum u_n(x) $ равномерно сходится на $ [x_0 - R, x_0] $ (т.~к. он не зависит от $ x $)
	$$ 0 \le v_n(x) \le 1, \qquad v_n(x) \text{ монотонна по } n \quad \forall x \in [x_0 - R, x_0] $$
	По признаку Абеля, последние два утверждения влекут, что
	$$ S(x) \bdefeq{u_n, v_n} \sum_{n = 1}^\infty u_n(x)v_n(x) \text{ равномерно сходится при } x \in [x_0 - R, x_0] $$
	$$ a_n(x - x_0)^n \in \Cont{[x_0 - R, x_0]} $$
	Можно применить следствие о непрерывности ряда непрерывных функций
\end{proof}

\section{Производная вещественного степенного ряда}

\begin{theorem}
	Имеется вещественный степенной ряд
	$$ S(x) \define a_0 + \sum_{n = 1}^\infty a_n(x - x_0)^n, \qquad R > 0 $$
	$$ T(x) \define a_1 + \sum_{n = 2}^\infty na_n(x - x_0)^{n - 1}, \qquad R_0 \text{ "--- его радиус сх.} $$
	$$ \implies R_0 = R $$
\end{theorem}

\begin{remark}
	$$ \sum_{n = 1}^\infty na_n(x - x_0)^n = (x - x_0)\sum_{n = 1}^\infty na_n(x - x_0)^{n - 1}, \qquad x \ne x_0 $$
	Ряды слева и справа сходится или расходятся одновременно, так как они различаются умножением на ненулевую константу
\end{remark}

\begin{proof}
	$$ t = \ulim_{n \to \infty} \sqrt[n]{|a_n|}, \qquad t_0 = \ulim_{n \to \infty}\sqrt[n]{|na_n|} $$
	Видно, что $ t_0 \ge t $
	$$ R \bydef \frac1t, \qquad R_0 \bydef \frac1{t_0} $$
	$$ \implies R_0 \le R $$
	Нужно доказать, что они совпадают \\
	Возьмём $ x $ такой, что $ |x - x_0| \fed r < R $ \\
	Докажем, что при таком $ x $ будет сходится ряд $ T(x) $: \\
	Возьмём $ r < \rho < R, \quad q \define \frac r\rho, \quad 0 < q < 1 $ \\
	Докажем, что $ T(x) $ абсолютно сходится (из этого будет следовать, что он сходится):
	\begin{equ}{real_power_series:17}
		\sum_{n = 1}^\infty n|a_n|r^n = \sum_{n = 1}^\infty |a_n|\rho^n \cdot \bigg( n\frac{r^n}{\rho^n} \bigg) \bdefeq q \sum |a_n|\rho^n \cdot (nq^n)
	\end{equ}
	Рассмотрим $ \vphi(x) \define xq^x, \quad x \ge 0 $ \\
	Понятно, что $ \vphi(0) = 0, \qquad \vphi(x) \infarr x 0 $ \\
	Найдём её максимум:
	$$ \vphi'(x) = q^x + x\ln q q^x $$
	$$ q^{x_0} + x_0\ln qq^{x_0} = 0 $$
	$$ x_0 = -\frac1{\ln q} = \frac1{\ln \frac1q} \fed M $$
	$$ \implies nq^n \le M \quad \forall n $$
	$$ \underimp{\eref{real_power_series:17}} \sum_{n = 1}^\infty |a_n|\rho^n(nq^n) \le \sum |a_n|\rho^n \cdot M \underset{S(x) \text{ сх.}}< \infty $$
	$$ \implies [x_0 - r, x_0 + r] \sub (x_0 - R_0, x_0 + R_0) $$
	$$ \underimp{\text{в силу произвольности } r} (x_0 - R, x_0 + R) \sub (x_0 - R_0, x_0 + R_0) \quad \implies R \le R_0 $$
\end{proof}

\begin{implication}
	Обозначим $ u_n(x) \define a_n(x - x_0)^n $ \\
	Тогда $ u_n'(x) = na_n(x - x_0)^{n - 1} $ \\
	Если взять $ \forall 0 < r < R $, то ряд $ T(x) $ сходится равномерно при $ x \in [x_0 - r, x_0 + r] $ \\
	Ряд $ S(x) $ сходится равномерно там же
	$$ \implies \forall x \in [x_0 - r, x_0 + r] \quad \exist S'(x) = T(x) $$
	Это верно при $ \forall x \in \B $ (т. к. можно обозначить $ |x - x_0| \fed r < R $)
\end{implication}

\section{Старшие производные вещественного степенного ряда; степенной ряд как ряд Тейлора своей суммы}

Рассмотрим ряд $ T(x) $ как первоначальный ряд. \\
По теореме получаем, что радиус сходимости $ T'(x) $ будет таким же, то есть,

\begin{implication}
	$$ 2a_2 + \sum_{n = 3}^\infty n(n - 1)a_n(x - x_0)^{n - 2} = \bigg( S'(x) \bigg)' = S''(x) $$
\end{implication}

Это можно продолжать. Получаем следующую теорему:

\begin{theorem}
	\begin{equ}{higher_der:1}
		\forall m \quad \forall x \in I \quad \exist S^{(m)}(x) = \sum_{n = 1}^\infty \bigg( a_n(x - x_0)^n \bigg)^{(m)}
	\end{equ}
\end{theorem}

Заметим, что справедливы равенства:
$$ \big( (x - x_0)^n \big)' = n(x - x_0)^{n - 1}, \quad \big( (x - x_0)^n \big)'' = n(n - 1)(x - x_0)^{n - 2}, \quad \dots $$
$$
\begin{cases}
	\nder[m]{\big( (x - x_0)^n \big)} = n(n - 1)\cdots(n - m + 1)(x - x_0)^{n - m}, \qquad m < n \\
	\nder{\big( (x - x_0)^n \big)} = n! \\
	\nder[n + k]{\big( (x - x_0)^n \big)} = 0, \qquad k \ge 1
\end{cases} $$
Отсюда следует, что
$$ \nder[m]{\big( (x - x_0)^n \big)}\clamp{x = x_0} =
\begin{cases}
	0, \qquad n \ne m \\
	n!, \qquad n \ne m
\end{cases} $$
Рассмотрим
$$ S(x) \define c_0 + \sum_{n = 1}^\infty c_n(x - x_0)^n, \qquad x \in I $$
Понятно, что $ S(x_0) = c_0 $ \\
Если $ m \ge 1 $, то, по формуле \eref{higher_der:1},
$$ \nder[m]S(x_0) = 0 + c_m \cdot m!, \qquad c_m = \frac{\nder[m]S(x_0)}{m!} $$
$$ \implies S(x) = S(x_0) + \sum_{n = 1}^\infty \frac{\nder S(x_0)}{n!}(x - x_0)^n $$
Этот ряд называется \it{рядом Тейлора} для функции $ S(x) $.

\section{Интегрирование вещественного степенного ряда}

\begin{theorem}
	По-прежнему рассматриваем ряд $ S(x), \qquad p, q \in I $ \nimp[(не обязательно $ p < q $)]
	$$ \implies \dint pq{S(x)} = a_0(q - p) + \sum_{n = 1}^\infty a_n \frac{(q - x_0)^{n + 1} - (p - x_0)^{n + 1}}{n + 1} $$
\end{theorem}

\begin{proof}
	$ S $ равномерно сходится на $ [p \between q] $. \\
	Его можно интегрировать почленно, что и записано в теореме.
\end{proof}

\begin{statement}
	В частности, при $ p = x_0, \quad q = y \in I $,
	\begin{equ}{int_power_series:33}
		\dint{x_0}y{S(x)} = a_0(y - x_0) + \sum_{n = 1}^\infty a_n \frac{(y - x_0)^{n + 1}}{n + 1}
	\end{equ}
\end{statement}

\section{Разложение в степенной ряд функций \tpst{$ \ln(1 + x) $}{ln(1 + x)} и \tpst{$ \arctg x $}{arctg x}}

Рассмотрим ряд
$$ 1 + \sum_{n = 1}^\infty (-1)^nx^n = \frac1{1 + x}, \qquad x \in (-1, 1) $$
Понятно, что при $ r < 1 $ ряд сходится равномерно на $ [-r, r] $. \\
Возьмём $ |y| \le r $ и проинтегрируем по формуле \eref{int_power_series:33}:
$$ \boxed{\ln (1 + y)} = \dfint0y{1 + x} = y + \sum_{n = 1}^\infty (-1)^n \frac{y^{n + 1}}{n + 1} = \boxed{\sum_{n = 1}^\infty (-1)^{n - 1} \frac{y^n}n} $$
Радиус сходимости этого ряда равен 1. При $ y = 1 $ он сходится. По теореме Абеля он сходится равномерно на $ [0, 1] $.

Напишем в этом равенстве $ x^2 $ вместо $ x $:
$$ 1 + \sum_{n = 1}^\infty (-1)^n x^{2n} = \frac1{1 + x^2} $$
Рассмотрим $ |y| < 1 $:
$$ \boxed{\arctg y} = \dfint0y{1 + x^2} = y + \sum_{n = 1}^\infty (-1)^n \frac{y^{2n + 1}}{2n + 1} = \boxed{\sum_{n = 1}^\infty \frac{y^{2n - 1}}{2n - 1}} $$
При $ y = 1 $ этот рад сходится как знакочередующийся. По теореме Абеля он непрерывен на $ [0, 1] $

\section{Формула Тейлора с интегральным остатком}

\begin{theorem}
	$ f \in \Cont[n]{(a, b)}, \qquad x, x_0 \in (a, b), \quad x \ne x_0 $
	$$ \implies f(x) = f(x_0) + \frac{f'(x_0)}{1!}(x - x_0) + \dots + \frac{f^{(n - 1)}(x_0)}{(n - 1)!}(x - x_0)^{n - 1} + \frac1{(n - 1)!} \dint[t]{x_0}x{(x - t)^{n - 1}f^{(n)}(t)} $$
\end{theorem}

\begin{proof}
	Докажем \bt{по индукции}.
	\begin{itemize}
		\item \bt{База.} $ n = 1 $
		$$ f(x) \stackrel?= f(x_0) + \dint[t]{x_0}x{f'(t)} $$
		Это "--- формула Ньютона"--~Лейбница.
		\item \bt{Переход.} $ n \to n + 1 $
		$$ f \in \Cont[n + 1]{(a, b)} $$
		Проинтегрируем по частям по $ t $:
		$$ \bigg( -\frac{(x - t)^n}n \bigg)_t' = (x - t)^{n - 1} $$
		\begin{multline*}
			\dint[t]{x_0}x{ \bigg( -\frac{(x - t)^n}n \bigg)'}f^{(n)}(t) = \bigg( -\frac{(x - t)^n}n f^{(n)}(t) \bigg) \clamp[x]{x_0} - \dint[t]{x_0}x{\bigg( -\frac{(x - t)^n}n \bigg)f^{(n + 1)}(t)} = \\
			= \frac{(x - x_0)^n}nf^{(n)}(x_0) + \frac1n \dint[t]{x_0}x{\frac{(x - t)^n}n f^{(n + 1)}(t)}
		\end{multline*}
		\begin{multline*}
			\underimp{\bt{предп.}} f(x) = f(x_0) + \\
			+ \frac{f'(x_0)}{1!}(x - x_0) + \dots + \frac{f^{(n - 1)}(x_0)}{(n - 1)!}(x - x_0)^{n - 1} + \frac{f^{(n)}(x_0)}{n!}(x - x_0)^n + \frac1{n!} \dint[t]{x_0}x{\frac{(x - t)^n}nf^{(n + 1)}(t)}
		\end{multline*}
	\end{itemize}
\end{proof}

\section({Разложение в степенные ряды e\tcir{}x, cos x, sin x}){Разложение в степенные ряды $ e^x $, $ \cos x $, $ \sin x $}

Рассматриваем $ x_0 = 0 $

\begin{enumerate}
	\item $ e^x $
	$$ (e^x)^{(n)} = e^x $$
	$$ (e^x)^{(n)}\clamp{x = 0} = 1 $$
	$$ e^0 = 1 $$
	\begin{equ}{series_expans:45}
		e^x \overset T= 1 + x + \frac{x^2}{2!} + \dots + \frac{x^n}{n!} + e^c \cdot \frac{x^{n + 1}}{(n + 1)!}, \qquad c < |x|, \quad cx > 0
	\end{equ}
	$$ \bigg| e^c \frac{x^{n + 1}}{(n + 1)!} \bigg| \le e^{|x|} \cdot \frac{|x|^{n + 1}}{(n + 1)!} \infarr{n} 0 $$
	$$ \underimp{\eref{series_expans:45}} \boxed{e^x = 1 + \sum_{n = 1}^\infty \frac{x^n}{n!}, \qquad x \in \R} $$
	При $ x = 1 $ получаем
	$$ e = 2 + \sum_{n = 2}^\infty \frac1{n!} $$
	\item $ \cos x $
	$$ (\cos x)' = -\sin x $$
	$$ (\cos x)'' = -\cos x $$
	$$ (\cos x)''' = \sin x $$
	$$ (\cos x)^{(4)} = \cos x $$
	$$ (\cos x)^{(2n - 1)}\clamp{x = 0} = 0 $$
	$$ (\cos x)^{(2n)}\clamp{x = 0} = (-1)^n $$
	$$ \cos x \overset T= 1 - \frac{x^2}{2!} + \frac{x^4}{4!} + \dots + (-1)^n \frac{x^{2n}}{(2n)!} \pm \sin c \cdot \frac{x^{2n + 1}}{(2n + 1)!} $$
	$$ \underimp{\text{лемма}} \boxed{\cos x = 1 + \sum_{n = 1}^\infty (-1)^n \frac{x^{2n}}{(2n)!}} $$
	\item $ \sin x $
	$$ (\sin x)' = \cos x $$
	$$ (\sin x)'' = -\sin x $$
	$$ (\sin x)''' = -\cos x $$
	$$ (\sin x)^{(4)} = \sin x $$
	$$ (\sin x)^{(2n)}\clamp{x = 0} = 0 $$
	$$ (\sin x)^{2n - 1}\clamp{x = 0} = (-1)^{n - 1} $$
	$$ \sin x \overset T= x - \frac{x^3}{3!} + \frac{x^5}{5!} - \dots + (-1)^{n - 1} \frac{x^{2n - 1}}{(2n - 1)!} \pm \sin c \cdot \frac{x^{2n}}{2n!} $$
	$$ \underimp{\text{лемма}} \boxed{\sin x = \sum_{n = 1}^\infty (-1)^{n - 1} \frac{x^{2n - 1}}{(2n - 1)!}} $$
\end{enumerate}

\section{Разложение в степенной ряд \tpst{$ (1 + x)^r $}{(1 + x)\tcir{}r}}

$ (1 + x)^r, \qquad r \nin \N, \quad r \ne 0 $ \nimp[(чтобы была нетривиальность)]
$$ \bigg( (1 + x)^r \bigg)' = r(1 + x)^{r - 1} $$
$$ \bigg( (1 + x)^r \bigg)'' = r(r - 1)(1 + x)^{r - 2} $$
$$ \bigg( (1 + x)^r \bigg)^{(n)} = r(r - 1)(r - 2)\dots(r - n + 1)(1 + x)^{r - n} $$
$$ \bigg( (1 + x)^r \bigg)^{(n)}\clamp{x = 0} = r(r - 1)\dots(r - n + 1) $$
Применим формулу Тейлора с остатком в форме Коши:
\begin{multline}\lbl{series_expans:413}
	(1 + x)^r = 1 + \frac{rx}{1!}x + \frac{r(r -1)}{2!}x^2 + \dots + \frac{r(r - 1)\dots(r - n + 1)}{n!}x^n + \\
	+ \underbrace{\frac1{n!} \dint[t]0x{(x - t)^nr(r - 1)\dots(r - n)(1 + t)^{r - n - 1}}}_{I_n}
\end{multline}
$$ (x - t)^n(1 + t)^{-n} = \bigg( \frac{x - t}{1 + t} \bigg)^n $$
Всё это верно при $ 0 \le |t| \le x, \quad tx \ge 0 $
\begin{itemize}
	\item $ x > 0 $
	$$ 0 \le \frac{x - t}{1 + t} \le x $$
	\item $ x < 0 $
	$$ \frac{x - t}{1 + t} = \frac{-|x| + |t|}{1 - |t|} \implies \bigg| \frac{x - t}{1 + t} \bigg| = \frac{|x| - |t|}{1 - |t|} \le |x| $$
\end{itemize}

Объединим последние два выражения:
$$ \bigg| \frac{x - t}{1 + t} \bigg| \le |x|, \qquad \text{при } |t| \le |x|, \quad tx \ge 0 $$
\begin{multline*}
	\implies |I_n| \le \frac{|r(r - 1)\dots(r - n)|}{n!} \bigg| \dint[t]0x{\bigg| \frac{x - t}{1 + t} \bigg|^n \cdot (1 + t)^{r - 1}} \bigg| \le \\
	\le \frac{|r(r - 1)\dots(r - n)|}{n!}|x^n| \bigg| \dint[t]0x{(1 + t)^{r - 1}} \bigg|
\end{multline*}
Обозначим
$$ \alpha_n \define \frac{|r(r - 1)\dots(r - n)|}{n!}|x|^n $$
Считаем, что $ n > r + 1 $
$$ \frac{\alpha_{n + 1}}{\alpha_n} = \frac1{n + 1} \cdot |r - n - 1| \cdot |x| $$
$$ |r - n + 1| = n + 1 - r $$
\begin{equ}{series_expans:418}
	\implies \frac{\alpha_{n + 1}}{\alpha_n} = \frac{n + 1 - r}{n + 1} \cdot |x| = \bigg( 1 - \frac{r}{n + 1} \bigg) |x| \infarr n |x|
\end{equ}
Обозначим
$$ q \define \frac{1 + |x|}2, \qquad q < 1, \quad |x| < q $$
В новых обозначениях,
$$ \frac{\alpha_{n + 1}}{\alpha_n} \le q \quad \forall n \ge \nimp[\text{ некторого }] n_0 $$
Значит, при $ n > n_0, ~ \alpha_n > 0 $, $ \alpha_n $ монотонно убывает
\begin{equ}{series_expans:420}
	\implies \exist \limi{n} \alpha_n \fed \alpha \ge 0
\end{equ}
$$ \eref{series_expans:418} \iff \alpha_{n + 1} = \alpha_n \bigg( 1 - \frac{r}{n + 1} \bigg) \cdot |x| $$
$$ \underimp{\eref{series_expans:420}} \alpha = \alpha|x| \implies \alpha = 0 $$
\begin{equ}{series_expans:39}
	\underimp{\eref{series_expans:413}} (1 + x)^r = 1 + \frac{rx}{1!} + \frac{r(r - 1)x^2}{2!} + \cdots + \frac{r(r - 1)\dots(r - n + 1)}{n!}x^n + \dots
\end{equ}

\section{Разложение в степенной ряд \tpst{$ \arcsin x $}{arcsin x}}

\textcolor{blue}{\huge Этого вопроса нет.}

Подставим в \eref{series_expans:39} $ x = -y^2, \quad 0 < |y| < 1, \quad r = \frac12 $:
\begin{multline*}
	(1 - y^2)^{-\frac12} = 1 + \bigg( -\frac12 \bigg) (-y^2) + \frac{\big( -\frac12 \big)\big( -\frac32 \big)}{2!}y^4 + \dots + (-1)^n \frac{\big( -\frac12 \big) \big( -\frac32 \big) \cdots \big( -\frac{2n - 1}2 \big)}{n!}y^{2n} + \dots = \\
	= 1 + \frac{y^2}2 + \frac{1 \cdot 3}{2^2 \cdot 2!}y^4 + \dots + \frac{1 \cdot 3 \cdots (2n - 1)}{2^nn!}y^{2n} + \dots
\end{multline*}
\begin{multline*}
	\implies \arcsin x = \dfint[y]0\infty{\sqrt{1 - y^2}} = \dint[y]0x{\bigg( 1 + \frac{y^2}2 + \frac{1 \cdot 3}{2^2 \cdot 2!}y^4 + \dots + \frac{1 \cdot 3 \cdots (2n - 1)}{2^nn!}y^{2n} + \dots \bigg)} = \\
	= x + \frac1{2 \cdot 3}x^3 + \frac{1 \cdot 3}{2^2 \cdot 2! \cdot 5}x^5 + \dots + \frac{1 \cdot 3 \cdots (2n - 1)}{2^n \cdot n!(2n + 1)}x^{2n + 1} + \dots
\end{multline*}
