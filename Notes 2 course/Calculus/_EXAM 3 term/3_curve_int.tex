\part{Криволинейные интегралы}

\section{Спрямляемые кривые, длина кривой, аддитивность длины кривой}

\begin{definition}
	$ \Gamma : [a, b] \to \R^{n \ge 2}, \qquad \Gamma \in \Cont{[a, b]} $ \\
	$ \Gamma $ будем называть разомкнутой кривой, если оно биективно
\end{definition}

Образ $ \Gamma([a, b]) $ будем называть кривой и обозначать $ \Gamma $
$$ \Par \quad a = t_0 < t_1 < \dots < t_m = b $$
Множество точек $ \seq[m]{\Gamma(t_k)}k $ будем называть разбиением

$$ l \bigg( \seq[m]{\Gamma(t_k)}k \bigg) \define \sum_{k = 0}^{m - 1} \norm{\Gamma(t_{k + 1}) - \Gamma(t_k)}_n $$
Рассмотрим величину
$$ \sup\limits_{\set{\Gamma(t_k)}} l \bigg( \seq[m]{\Gamma(t_k)}k \bigg) $$
Если $ \sup < \infty $, то $ \Gamma $ будем называть спрямляемой, а $ \sup $ "--- длиной $ \Gamma $
$$ \Gamma_{[a, c]}(t) \define \Gamma(t)\clamp{[a, c]}, \qquad \Gamma_{[c, b]} (t) \define \Gamma(t)\clamp{[c, b]} $$

\begin{statement}
	Если $ \Gamma $ спрямляема, то $ \Gamma_{[a, c]} $ и $ \Gamma_{[c, b]} $ тоже спрямляемы, и
	$$ l \bigg( \Gamma([a, b]) \bigg) = l \bigg( \Gamma([a, c]) \bigg) + l \bigg( \Gamma([c, b]) \bigg) $$
\end{statement}

\begin{proof}
	Возьмём $ c \in (t_{k_0}, t_{k_0 + 1}) $
	$$ \norm{\Gamma(t_{k + 1}) - \Gamma(t_k)} \le \norm{\Gamma(t_{k + 1}) - \Gamma(c)} + \norm{\Gamma(c) - \Gamma(t_k)} $$
	$$ l \bigg( \seq[m]{\Gamma(t_k)}k \bigg) \le l \bigg( \seq{\Gamma(t_k) \cup \Gamma(c)}k \bigg) $$
	$$ l \bigg( \seq[m]{\Gamma(t_k)}k \cup \Gamma(c) \bigg) = l \bigg( \seq[k_0]{\Gamma(t_k)}k \cup \Gamma(c) \bigg) + l \bigg( \seqv[m]{\Gamma(t_k)}k{k_0} \cup \Gamma(c) \bigg) $$
	Далее нужно перейти к супремуму в левой и правой частях.
\end{proof}

\section{Непрерывность длины кривой как функции от параметра}

\begin{property}
	$ \Gamma(t) $ спрямляема. Тогда
	$$ \liml{c \to b^-} l \bigg( \Gamma[a, c] \bigg) = l \bigg( \Gamma([a, b] \bigg) $$
\end{property}

\begin{proof}
	Если $ a < c_1 < c_2 < b $, то по последнему свойству,
	$$ l \big( \Gamma[a, c_2] \big) > l \big( \Gamma[a, c_1] \big), \qquad l \big( \Gamma[a, c_2] \big) + l \big( \Gamma[c_2, b] \big) = l \big( \Gamma[a, b] \big) $$
	Поэтому функция $ f : (a, b) \to \R : \quad f = l \big( \Gamma[a, c] \big) $ возрастает и $ f < l \big( \Gamma[a, b] \big) \quad \forall c < b $. \\
	Значит, $ \exist \liml{c \to b^-} l \big( \Gamma[a, c] \big) \le l \big( \Gamma[a, b] \big) $

	Предположим, что $ \liml{c \to b^-} l \big( \Gamma[a, c] \big) < l \big( \Gamma[a, b] \big) $ и положим
	$$ \delta \define l \big( \Gamma[a, b] \big) - \liml{c \to b^-} l \big( \Gamma[a, c] \big) > 0 $$

	Поскольку $ l \big( \Gamma[a, c] \big) \le \liml{c \to b^-} l \big( \Gamma[a, c] \big) $, то $ \forall c : ~ a < c < b \quad l \big( \Gamma[c, b] \big) \ge \delta $.
\end{proof}

\section{Вычисление длины гладкой кривой}

\begin{lemma}
	$ F : [a, b] \to \R^n, \qquad F \in \Cont{[a, b]} $
	$$ F(t) = \column{f_1(t)}{f_n(t)} $$
	Определим символ:
	$$ \dint[t]ab{F(t)} \define \column{\dint[t]ab{f_1(t)}}{\dint[t]ab{f_n(t)}} $$
	Тогда справедливо соотношение:
	$$ \norm{\dint[t]ab{F(t)}} \le \dint[t]ab{\norm{F(t)}} $$
\end{lemma}

\begin{proof}
	Будем считать, что $ \dint[t]ab{F(t)} \ne \On $ (иначе "--- очевидно) \\
	Обозначим $ q \define \norm{\dint[t]ab{F(t)}} > 0 $ \\
	Введём числа
	$$ \alpha_k \define \dint[t]ab{f_k(t)}, \qquad a_k \define \frac{\alpha_k}q $$
	Рассмотрим сумму
	$$ \sum_{k = 1}^n a_k \alpha_k \bdefeq{a_k} \sum \frac{\alpha_k}q\alpha_k = \frac1q \sum \alpha_k^2 \bdefeq{\alpha_k, q} \frac{q^2}q = q $$
	\begin{multline*}
		\implies \bm{q} = \sum_{k = 1}^n a_k\alpha_k = \sum a_k \dint[t]ab{f_k(t)} = \dint[t]ab{\sum a_kf_k(t)} \underset{\text{КБШ}}{\bm\le} \dint[t]ab{ \bigg( \sum a_k^2 \bigg)^\frac12 \bigg( \sum f_k^2(t) \bigg)^\frac12 } = \\
		= \bigg( \sum a_k^2 \bigg)^\frac12 \cdot \dint[t]ab{ \bigg( \sum f_k^2(t) \bigg)^\frac12 } = \bm{\bigg( \sum a_k^2 \bigg)^\frac12} \dint[t]ab{\norm{F(t)}}
	\end{multline*}
	$$ \bm{\sum a_k^2} \bdefeq{a_k} \sum \frac{\alpha_k^2}{q^2} = \frac1{q^2} \sum \alpha_k^2 = \frac{q^2}{q^2} \bm{= 1} $$
	$$ \norm{\dint[t]ab{F(t)}} = q \le 1 \cdot \dint[t]ab{F(t)} $$
\end{proof}

\begin{theorem}
	$ \Gamma \in \Cont[1]{[a, b]} $
	$$ \implies l(\Gamma) = \dint[t]ab{\norm{\mc D\Gamma(t)}} $$
\end{theorem}

\begin{iproof}
	\item $ l \le \int $ \\
	Пусть имеется любое разбиение любой $ \Gamma \quad \seq{\Gamma(t_k)}k $
	$$ \Gamma(t_{k + 1}) - \Gamma(t_k) = \column{\gamma_1(t_{k + 1}) - \gamma_1(t_k)}{\gamma_n(t_{k + 1}) - \gamma_n(t_k)} \undereq{\text{ф. Н."--~Лейбница}} \column{\dint[t]{t_k}{t_{k + 1}}{\gamma_1'(t)}}{\dint[t]{t_k}{t_{k + 1}}{\gamma_n'(t)}} \bdefeq{\int F} \dint[t]{t_k}{t_{k + 1}}{\mc D\Gamma(t)} $$
	$$ \underimp{\text{лемма}} \norm{\Gamma(t_{k + 1}) - \Gamma(t_k)} = \norm{\dint[t]{t_k}{t_{k + 1}}{\mc D\Gamma(t)}} \le \dint[t]{t_k}{t_{k + 1}}{\norm{\mc D\Gamma(t)}} $$
	$$ \implies \sum_{k = 0}^{m - 1} \norm{\Gamma(t_{k + 1}) - \Gamma(t_k)} \le \sum \dint[t]{t_k}{t_{k + 1}}{\norm{\mc D(t)}} = \dint[t]ab{\norm{\mc D\Gamma(t)}} $$
	Перепишем в обозначениях длины:
	$$ l \bigg( \seq{\Gamma(t_k)}k \bigg) \le \dint[t]ab{\norm{\mc D\Gamma(t)}} \implies l(\Gamma) \le \dint[t]ab{\norm{\mc D\Gamma(t)}} $$

	\item $ l \ge \int $
	$$ \Gamma \in \mc C^1 \implies \gamma_k' \in \Cont{[a, b]} $$
	То есть,
	$$ \forall \veps > 0 \quad \exist \delta > 0 : \quad \forall t'', t' \in [a, b] \quad \nimp[\bigg(] |t'' - t'| < \delta \implies |\gamma_1'(t'') - \gamma_k'(t')| < \frac\veps{\sqrt n} \nimp[\bigg)], \qquad k = 1, \dots, n $$
	$$ \implies \sqrt{\bigg( \gamma_1'(t'') - \gamma_1'(t') \bigg)^2 + \dots + \bigg( \gamma_n'(t'') - \gamma_n'(t') \bigg)^2} < \sqrt{\frac{\veps^2}n} \cdot n = \veps $$
	$$ \iff \norm{\mc D\Gamma(t'') - \mc D\Gamma(t')} < \veps $$
	Возьмём разбиение $ \seq[m]{\Gamma(t_k)}k $ такое, что $ t_{k - 1} - t_k < \delta \quad k = 0, \dots, m - 1 $ \\
	Для $ \forall t \in [t_k, t_{k + 1}] $ имеем соотношение
	\begin{equ}{rect_curve:10}
		\norm{\mc D \Gamma(t)} \trile \norm{\mc D \Gamma(t) - \mc D \Gamma(t_k)} + \norm{\mc D \Gamma(t_k)} < \norm{\mc D \Gamma(t_k)} + \veps
	\end{equ}
	Рассмотрим выражение
	\begin{multline*}
		\Gamma(t_{k + 1}) - \Gamma(t_k) - (t_{k + 1} - t_k)\mc D\Gamma(t_k) = \column{\dint[t]{t_k}{t_{k + 1}}{\gamma_1'(t)}}{\dint[t]{t_k}{t_{k + 1}}{\gamma_n'(t)}} - \column{(t_{k + 1} - t_k)\gamma_1'(t_k)}{(t_{k + 1} - t_k)\gamma_n'(t)} = \\
		= \column{\dint[t]{t_k}{t_{k + 1}}{ \bigg( \gamma_1'(t) - \gamma_1'(t_k) \bigg)}}{\dint[t]{t_k}{t_{k + 1}}{ \bigg( \gamma_n'(t) - \gamma_n'(t_k) \bigg)}} \bdefeq{\int F} \dint[t]{t_k}{t_{k + 1}}{\mc D\Gamma(t) - \mc D\Gamma(t_k)}
	\end{multline*}
	Применим лемму:
	$$ \norm{\Gamma(t_{k + 1}) - \Gamma(t_k) - (t_{k + 1} - t_k)\mc D\Gamma(t_k)} \le \dint[t]{t_k}{t_{k + 1}}{\norm{\mc D\Gamma(t) + \mc D\Gamma(t_k)}} \underset{\eref{rect_curve:10}}\le \dint[t]{t_k}{t_{k + 1}}\veps = \veps(t_{k + 1} + t_k) $$
	\begin{equ}{rect_curve:13}
		\implies \norm{\Gamma(t_{k + 1}) - \Gamma(t_k)} \trige \norm{(t_{k + 1} - t_k) \mc D \Gamma(t_k)} - \veps (t_{k + 1} - t_k)
	\end{equ}
	\begin{equ}{rect_curve:14}
		(t_{k + 1} - t_k) \norm{\mc D\Gamma(t_k)} = \sqrt{(t_{k + 1} - t_k)^2 \bigg( \gamma_1'(t_k) \bigg)^2 + \dots + (t_{k + 1} - t_k)^2 \bigg( \gamma_n'(t_k) \bigg)^2}
	\end{equ}
	Если взять $ t \in [t_k, t_{k + 1}] $, то
	$$ \norm{\mc D \Gamma(t_k)} \ge \norm{\mc D\Gamma(t)} - \norm{\mc D\Gamma(t_k) - \mc D \Gamma(t)} > \norm{\mc D \Gamma(t)} - \veps $$
	Проинтегрируем:
	$$ \dint[t]{t_k}{t_{k + 1}}{\norm{\mc D\Gamma(t_k)}} > \dint[t]{t_k}{t_{k + 1}}{\norm{ \mc D\Gamma(t)}} - \veps \dint[t]{t_k}{t_{k + 1}}{} $$
	$$ (t_{k + 1} - t_k) \norm{\mc D\Gamma(t_k)} \ge \dint[t]{t_k}{t_{k + 1}}{\norm{\mc D\Gamma(t)}} - \veps(t_{k + 1} - t_k) $$
	$$ \underimp{\eref{rect_curve:13}, \eref{rect_curve:14}} \norm{\Gamma(t_{k + 1}) - \Gamma(t_k)} > \dint[t]{t_k}{t_{k + 1}}{\norm{\mc D\Gamma(t)}} - \veps(t_{k + 1} - t_k) $$
	$$ \implies \sum_{k = 0}^{m - 1} \norm{\Gamma(t_{k + 1}) - \Gamma(t_k)} > \sum_{k = 0}^{m - 1} \dint[t]{t_k}{t_{k + 1}}{\norm{\mc D\Gamma(t)}} - \veps \sum_{k = 0}^{m - 1}(t_{k + 1} - t_k) = \dint[t]ab{\norm{\mc D\Gamma(t)}} - \veps(b - a) $$
	$$ \iff l \bigg( \seqz[m]{\Gamma(t_k)}k \bigg) > \dint[t]ab{\norm{\mc D\Gamma(t)}} - \veps(b - a) $$
	$$ \implies l(\Gamma) > \dint[t]ab{\norm{\mc D \Gamma(t)}} - \veps(b - a) \quad \implies \quad l(\Gamma) \ge \dint[t]ab{\norm{\mc D \Gamma(t)}} $$
\end{iproof}

\section{Определение криволинейного интеграла первого рода; суммы Римана криволинейного интеграла первого рода; криволинейный интеграл первого рода как предел сумм Римана}

\begin{definition}
	$ \Gamma : [a, b] \to \R^n, \qquad \Gamma \in C^1, \qquad f \in \Cont\Gamma $ \\
	\it{Криволинейным интегралом первого рода} по кривой $ \Gamma $ называется
	$$ \cint[l(M)]\Gamma{f(M)} \define \dint[t]ab{f \big( \Gamma(t) \big)\norm{\mc D\Gamma(t)}} $$
\end{definition}

\begin{definition}
	$ \Gamma_\circ : [a, b] \to \R, \qquad c_0 = a < c_1 < \dots < c_m = b, \qquad f \in \Cont{\Gamma_0} $
	$$ \forall [c_k, c_{k + 1}] \quad \Gamma_\circ([c_k, c_{k + 1}]) \text{ "--- } C^1 \text{-кривая} $$
	Криволинейный интеграл первого рода для ``кусочной'' кривой определяется как
	$$ \cint[l(M)]{\Gamma_0}{f(M)} = \sum_{k = 0}^{m - 1} \cint[l(M)]{\Gamma_0([c_k, c_{k + 1}])}{f(M)} $$
\end{definition}

\begin{definition}
	$ \Gamma : [a, b] \to \R^n $ "--- $ C^1 $-кривая, $ \qquad f \in \Cont\Gamma $ \\
	$ \Teq = \seq[m]{t_k}k, \quad a = t_0 < t_1 < \dots < t_m = b $ "--- разбиение, $ \qquad \Par = \seq[m]{\tau_k}k, \quad \tau_k \in [t_{k - 1}, t_k] $ "--- оснащение
	$$ \Sr_\Gamma (f, \Teq, \Par) \define \sum_{k = 1}^m f \big( \Gamma(\tau_k) \big) l \big( \Gamma([t_{k - 1}, t_k]) \big) $$
\end{definition}

\begin{theorem}
	$ \Gamma \in \mc C^1 $
	$$ \forall \veps > 0 \quad \exist \delta > 0 : \quad \forall \Teq : t_k - t_{k - 1} < \delta \quad \forall \Par \quad \bigg| \Sr_\Gamma(f, \Teq, \Par) - \cint\Gamma{f(M)} \bigg| < \veps $$
	То есть, $ \Sr(f, \Teq, \Par) \to \cint\Gamma{f(M)} $
\end{theorem}

\begin{proof}
	$ \Gamma([a, b]) $ "--- компакт в $ \R^n $ \\
	$ f \in \Cont\Gamma \underimp{\text{т. Кантора}} f $ равномерно непрерывна на $ \Gamma $, \ie
	\begin{equ}{first_order_curve_int:25}
		\forall \veps > 0 \quad \exist \lambda > 0 : \quad \forall M', M'' \in \Gamma : \quad \norm{M'' - M'} < \lambda \implies |f(M'') - f(M')| < \veps
	\end{equ}
	$$ \Gamma(t) \fed \column{\gamma_1(t)}{\gamma_n(t)}, \qquad \Gamma \in \mc C^1 \implies \gamma_k'(t) \in \Cont{[a, b]} $$
	$$ \underimp{\text{\rom1 т. Вейерштрасса}} \exist c_1 : |\gamma_k'(t)| \le c_1 \quad \forall t \in [a, b] \quad \forall k $$
	Рассмотрим любые два значения $ t', t'' \in [a, b] $ и применим теорему Лагранжа:
	$$ |\gamma_k(t'') - \gamma_k(t')| = |\gamma_k'(\vawe t)(t'' - t')| \le c_1|t'' - t'| $$
	$$ \implies \sqrt{\bigg( \gamma_1(t'') - \gamma_1(t') \bigg)^2 + \dots + \bigg( \gamma_n(t'') - \gamma_n(t') \bigg)^2} \le \sqrt{nc^2|t'' - t'|^2} = \sqrt nc_1|t'' - t'| $$
	\begin{equ}{first_order_curve_int:29}
		\iff \norm{\Gamma(t'') - \Gamma(t')} < \sqrt n c_1 |t'' - t'|
	\end{equ}
	Выберем $ \delta $:
	$$ \delta \define \frac\lambda{\sqrt n c_1} $$
	\begin{equ}{first_order_curve_int:211}
		\eref{first_order_curve_int:25}, \eref{first_order_curve_int:29} \bdefimp\delta \text{ при } |t'' - t'| < \delta \quad \big| f \big( \Gamma(t'') \big) - f \big( \Gamma(t') \big) \big| < \veps
	\end{equ}
	Обозначим $ M_k \define \Gamma(\tau_k) $
	\begin{equ}{first_order_curve_int:213}
		\implies |f(M_k) - f(M)| < \veps
	\end{equ}
	\begin{multline*}
		\Sr_\Gamma(f, \Teq, \Par) - \cint\Gamma{f(M)} = \sum_{k = 1}^m f \big( \Gamma(c_k) \big) l \big( \Gamma([t_{k - 1}, t_k]) \big) - \sum \cint{\Gamma([t_{k - 1}, t_k])}{f(M)} = \\
		= \sum \cint{\Gamma([t_{k - 1}, t_k])}{f(M_k)} - \sum \cint{\Gamma([t_{k - 1}, t_k])}{f(M)} = \sum \cint{\Gamma([t_{k - 1}, t_k])}{ \bigg( f(M_k) - f(M) \bigg)}
	\end{multline*}
	\begin{multline*}
		\implies |S_\Gamma(f, \Teq, \Par) - \cint\Gamma{f(M)}| \trile \sum_{k = 1}^m \bigg| \cint{\Gamma([t_{k - 1}, t_k])}{\bigg( f(M_k) - f(M) \bigg)} \le \\
		\le \sum \cint{\Gamma([t_{k - 1}, t_k])}{|f(M_k) - f(M)|} \underset{\eref{first_order_curve_int:213}}\le \sum \cint{\Gamma(t_{k - 1}, t_k)}\veps = \veps l \big( \Gamma([t_{k - 1}, t_k]) \big) = \veps l(\Gamma)
	\end{multline*}
\end{proof}

\section{Ориентация кривой, ориентированные кривые}

\begin{definition}
	$ \Gamma : [a, b] \to \R^{n \ge 2} $ "--- разомкнутая или замкнутая кривая. \\
	$ \Gamma(a) $ называется началом кривой, $ \quad \Gamma(b) $ "--- концом. \\
	Начало и конец задают ориентацию кривой.
	$$ a < c_1 < \dots < c_m < b $$
	Точки $ \Gamma(a), \Gamma(c_1), \dots, \Gamma(c_m), \Gamma(b) $ проходятся в соответствии с выбранной ориентацией.
\end{definition}

Рассмотрим образ кривой:
$$ \Gamma \sub \R^n $$
$$ \Gamma(a) \fed A, \qquad \Gamma(b) \fed B, \qquad \Gamma(c_k) \fed M_k $$
Точки $ A, M_1, \dots, M_m, B $ проходятся в соответствии с выбранной ориентацией.

Можно выбрать \soc обратную ориентацию:
$$ \Gamma_1 : [a, b] \to \R^n $$
$$ \Gamma_1(t) \define \Gamma(a + b - t) $$
$$ \Gamma_1(a) = \Gamma(b), \qquad \Gamma_1(b) = \Gamma(a) $$

При определении длины кривой мы вводили следующие суммы (записанные теперь через образ):
$$ \sum_{k = 0}^m \norm{M_{k + 1} - M_k} $$
Рассмотрим противоположную ориентацию:
$$ M_k' = M_{m + 1 - k} $$
Поменяем индексы:
$$ \sum_{k = 0}^m \norm{M_{k + 1} - M_k} = \sum_{k = 0}^m \norm{M_{m + 1 - k} - M_{m - k}} = \sum_{k = 0}^m \norm{M_{k + 1}' - M_k'} $$
Таким образом мы доказали, что

\begin{statement}
	Длина кривой не зависит от ориентации.
\end{statement}

\begin{restate}
	Длина кривой зависит только от её образа.
\end{restate}

Рассмотрим $ \Gamma \sub \R^n $ "--- образ замкнутой кривой. \\
Пусть даны разбиение $ \Teq = \set{t_k} $ и оснащение $ \Par = \set{\tau_j} $.
$$ \Gamma(a) = A, \qquad \Gamma(b) = B, \qquad \Gamma(t_k) = M_k $$
$$ A = B, \qquad \Gamma(\tau_j) \fed N_j $$
Можно переписать суммы Римана в новых обозначениях:
$$ \Sr(f, \Teq, \Par) = \sum_{k = 1}^m f(N_k) l \bigg( \Gamma(M_{k - 1}, M_k) \bigg) $$
Они (при стремлении диаметра разбиения к нулю) стремились к интегралу первого рода. \\
Аналогично длине кривой, здесь можно поменять индексы определённым образом. \\
Таким образом верно следующее:

\begin{statement}
	Криволинейный интеграл первого рода не зависит от ориентации кривой.
\end{statement}

\section{Определение криволинейного интеграла второго рода; суммы Римана криволинейного интеграла второго рода; криволинейный интеграл второго рода как предел интегральных сумм}

\begin{definition}
	$$ \curvedir\Gamma(t) = \column{\gamma_1(t)}{\gamma_n(t)} \text{ "--- } C^1 \text{-кривая}, \qquad f \in \Cont\Gamma $$
	Криволинейным интегралом второго рода по ориентированной кривой функции $ f $ называется
	$$ \cint[x_j]{\curvedir\Gamma}{f(M)} \define \dint[t]ab{f \big( \Gamma(t) \big)\gamma_j'} $$
\end{definition}

\begin{definition}
	$$ c_0 = a < c_1 < \dots < c_ m < b = c_{m + 1} $$
	$$ \curvedir\Gamma[a, b], \qquad \Gamma([c_{k - 1}, c_k]) \text{ "--- } C^1 \text{-кривая при } k = 1, \dots, m + 1 $$
	Тогда
	$$ \cint[x_j]{\curvedir\Gamma}{f(M)} \define \sum_{k = 1}^{m + 1} \cint[x_j]{\curvedir\Gamma([c_{k - 1}, c_k])}{f(M)} $$
\end{definition}

\begin{definition}
	$ \Gamma $ "--- $ C^1 $-кривая, $ \qquad f \in \Cont\Gamma, \qquad \Teq = \seqz[m]{t_k}k, \qquad \Par = \seq[m]{\tau_k}k, \qquad \tau_k \in [t_{k - 1}, t_k] $ \\
	Суммой Римана для интеграла второго рода будем называть
	$$ \Sr_{\curvedir\Gamma}(f, \Teq, \Par, j) = \sum_{k = 1}^m f \big( \Gamma(\tau_k) \big) \bigg( \gamma_j(t_k) - \gamma_j(t_{k - 1}) \bigg) $$
\end{definition}

\begin{theorem}
	$ \curvedir\Gamma $ "--- $ C^1 $-кривая
	$$ \implies \forall \veps > 0 \quad \exist \delta > 0 : \quad \forall \Teq : t_{k + 1} - t_k < \delta \quad \forall \Par \quad \bigg| \Sr_{\curvedir\Gamma}(f, \Teq, \Par, j) - \cint[x_j]{\curvedir\Gamma}{f(M)} \bigg| < \veps $$
	То есть, $ \Sr \to \int $.
\end{theorem}

\begin{proof}
	$ \gamma_\nu' \in \Cont{[a, b]} $
	$$ c_1 > 0 \quad |\gamma_n'(t)| \le c_1 \quad \forall t \in [a, b], \quad \nu = 1, \dots, n $$
	$$ \dint[t]ab{f \big( \Gamma(t) \big)\gamma_j'(t)} \bydef \sum_{k = 1}^m \dint[t]{t_{k - 1}}{t_k}{f \big( \Gamma(t) \big)} $$
	\begin{multline}\lbl{second_order_curve_int:7}
		\implies \Sr(\dots) - \dint[t]ab{f \big( \Gamma(t) \big)\gamma_j'(t)} \bdefeq{\Sr} \\
		= \sum_{k = 1}^m \bigg( f \big( \Gamma(\tau_k) \big) \big( \gamma_j(t_k) - \gamma_j(t_k) \big) - \dint[t]{t_{k - 1}}{t_k}{f \big( \Gamma(t) \big)\gamma_j'(t)} \bigg) \undereq{\text{ф. Ньютона"--~Лейбница}} \\
		= \sum_{k = 1}^m \bigg( f \big( \Gamma(\tau_k) \big) \dint[t]{t_{k - 1}}{t_k}{\gamma_j'(t)} - \dint[t]{t_{k - 1}}{t_k}{f \big( \Gamma(t) \big)\gamma_j'(t)}) \undereq{
			\begin{subarray}{c}
				\text{в первом слагаемом вносим константу} \\
				\text{разность интегралов как интеграл разности}
			\end{subarray}} \\
		= \sum_{k = 1}^m \dint[t]{t_{k - 1}}{t_k}{ \bigg( f \big( \Gamma(\tau_k) \big) - f \big( \Gamma(t) \big) \bigg) \gamma_j'(t)}
	\end{multline}
	По теореме Кантора $ f $ равномерно непрерывна на $ \Gamma $:
	\begin{equ}{second_order_curve_int:8}
		\exist \lambda > 0 : \quad \forall M', M'' \in \Gamma \quad \nimp[\bigg(] \norm{M'' - M'} < \lambda \implies |f(M'') - f(M')| < \veps \nimp[\bigg)]
	\end{equ}
	В конце прошлой лекции мы выяснили, что
	\begin{equ}{second_order_curve_int:9}
		|t'' - t'| < \delta \implies \norm{\Gamma(t'') - \Gamma(t')} \le c_1\sqrt n \delta
	\end{equ}
	$ c_1 $ играло ту же роль, что сейчас $ \veps $. \\
	Выберем $ \delta $ так, чтобы выполнялось
	\begin{equ}{second_order_curve_int:10}
		c_1\sqrt n \delta = \lambda
	\end{equ}
	Если $ t_k - t_{k - 1} < 0 $, то при $ t \in [t_{k - 1}, t_k], \quad \tau \in [t_{k - 1}, t_k] $ выполнено
	$$ |t - \tau| < \delta, \qquad k = 1, \dots, m $$
	Тогда
	$$ \eref{second_order_curve_int:8}, \eref{second_order_curve_int:9}, \eref{second_order_curve_int:10} \implies \bigg| f \big( \Gamma(\tau_k) \big) - f \big( \Gamma(t) \big) \bigg| < \veps $$
	\begin{multline*}
		\underimp{\eref{second_order_curve_int:7}} \bigg| \Sr_{\curvedir\Gamma}(\dots) - \cint[x_j]{\curvedir\Gamma}{f(M)} \bigg| \trile \sum_{k = 1}^m \bigg| \dint[t]{t_{k - 1}}{t_k}{\bigg( f \big( \Gamma(\tau_k) \big) - f \big( \Gamma(t) \big) \bigg)\gamma_j'} \bigg| \le \\
		\le \sum_{k = 1}^m \dint[t]{t_{k - 1}}{t_k}{ \bigg| f \big( \Gamma(\tau_k) \big) - f \big( \Gamma(t) \big) \bigg| \cdot |\gamma_j'|} < \sum_{k = 1}^m \dint[t]{t_{k - 1}}{t_k}{\veps|\gamma_j'(t)} = \\
		= \veps \dint[t]ab{|\gamma_j'(t)} \underset{|\gamma_j'(t)| \le \norm{\mc D\Gamma(t)}_n}\le \veps \dint[t]ab{\norm{\mc D \Gamma(t)}} = \veps l(\Gamma)
	\end{multline*}
\end{proof}

\section{Зависимость криволинейного интеграла второго рода от ориентации кривой}

\begin{implication}
	$$ \Gamma(t_k) \fed M_k = \column{x_{1k}}{x_{nk}}, \qquad \Gamma(\tau_k) \fed N_k $$
	$$ \implies x_{jk} = \gamma_j(t_k) $$
	$$ \implies \Sr_{\curvedir\Gamma}(f, \Teq, \Par, j) = \sum_{k = 1}^m f(N_k)(x_{jk} - x_{j ~ k - 1}) $$
	$ N_k $ лежит на дуге $ \Gamma(M_{k - 1}, M_k) $ \\
	В этой формуле нет отображения. Есть только образ и ориентация. \\
	Значит, криволинейный интеграл второго рода зависит только от образа и ориентации кривой.
\end{implication}

\begin{property}
	Определим $ t_\nu' \define t_{m - \nu}, \quad \tau_\nu' \define \tau_{m - \nu + 1} $
	$$ \Teq' \define \seqz[m]{t_k}k, \qquad \Par' \define \seq[m]{\tau_k}m, \qquad M_\nu' = M_{m - \nu}, \qquad N_\nu' = N_{m - \nu + 1} $$
	В соответствии с выбранной ориентацией проходились точки $ M_0, \dots, M_m $ \\
	Точки $ M_0', \dots, M_m' $ "--- это те же самые точки, проходимые в обратном порядке. То есть мы имеем дело с противоположной ориентацией $ \curvedir[0]\Gamma $
	$$ x_{j\nu}' = x_{j ~ m - \nu} $$
	\begin{multline*}
		\vawe \Sr = \sum_{k = 1}^m f(N_k') (x_{jk}' - x_{j ~ k - 1}') = \boxed{\Sr_{\curvedir[0]\Gamma}(f, \Teq', \Par', j)} = \sum_{k = 1}^m f(N_{m - k + 1})(x_{j ~ m - k} - x_{j ~ m - k + 1}) = \\
		= -\sum_{k = 1}^m f(N_{m - k + 1})(x_{j ~ m - k + 1} - x_{j ~ m - k}) \undereq{m - k + 1 \fed \nu} -\sum_{\nu = m}^1 f(N_\nu)(x_{j\nu} - x_{j ~ \nu - 1}) \undereq{k \define \nu} \boxed{-\Sr_{\curvedir\Gamma}(f, \Teq, \Par, j)}
	\end{multline*}
\end{property}

\section{Свойства криволинейного интеграла второго рода}

\begin{props}
	\item $ \curvedir\Gamma = \bigcup_{j = 1}^l \curvedir\Gamma_j, \qquad \curvedir\Gamma_j $ "--- $ C^1 $-кривая, $ \qquad f \in \Cont\Gamma $
	$$ \implies \cint[x_j]{\curvedir[0]\Gamma}{f(M)} = -\cint[x_j]{\curvedir\Gamma}{f(M)} $$

	\item $ \Gamma : [a, b] \to \R^n, \qquad \Gamma(t) \in \Cont{[a, b]}, \qquad c \in \R $
	$$ \Gamma(t) = \column{\gamma_1(t)}{\gamma_n(t)} $$
	$$ \implies \cint[x_j]{\curvedir\Gamma[a, b]}c = c \big( \gamma_j(b) - \gamma_j(a) \big) $$
	В частности, если $ \Gamma(a) = \Gamma(b) $, то
	$$ \cint[x_j]{\curvedir\Gamma} = 0 $$

	\item $ \curvedir\Gamma = \bigcup_{\nu = 1}^l \curvedir\Gamma_\nu, \qquad f \in \Cont\Gamma $
	$$ \bigg| \cint[x_j]{\curvedir\Gamma}{f(M)} \bigg| \le \cint\Gamma{|f(M)|} $$
\end{props}

\begin{eproof}
	\item
	\begin{itemize}
		\item Докажем для $ C^1 $-кривой:
		$$ \forall \veps > 0 \quad \exist \delta > 0 : \quad \forall \Teq \quad \forall \Par : t_k - t_{k - 1} < \delta \quad \bigg| \Sr_{\curvedir\Gamma}(f, \Teq, \Par, j) - \cint[x_j]{\curvedir\Gamma}{f(M)} \bigg| < \veps $$
		В силу свойства о зависимости от ориентации,
		$$ \bigg| \Sr_{\curvedir[0]\Gamma}(f, \Teq', \Par', j) - \cint[x_j]{\curvedir[0]\Gamma}{f(M)} \bigg| < \veps $$
		\begin{multline*}
			\bigg| \cint[x_j]{\curvedir\Gamma}{f(M)} + \cint[x_j]{\curvedir[0]\Gamma}{f(M)} \bigg| = \\
			= \bigg| \bigg( \cint[x_j]{\curvedir\Gamma}{f(M)} - \Sr_{\curvedir\Gamma}(f, \Teq, \Par, j) \bigg) + \bigg( \cint[x_j]{\curvedir[0]\Gamma}{f(M)} - \Sr_{\curvedir[0]\Gamma}(f, \Teq', \Par', j) \bigg) \bigg| \trile \\
			\le \bigg| \cint[x_j]{\curvedir\Gamma}{f(M)} - \Sr_{\curvedir\Gamma}(f, \Teq, \Par, j) \bigg| + \bigg| \cint[x_j]{\curvedir[0]\Gamma}{f(M)} - \Sr_{\curvedir[0]\Gamma}(f, \Teq', \Par', j) \bigg| < \veps + \veps = 2\veps
		\end{multline*}
		\item Общий случай:
		$$ \curvedir\Gamma = \bigcup_{\nu = 1}^l \curvedir\Gamma_\nu \quad \iff \quad \curvedir[0]\Gamma = \bigcup_{\nu = 1}^l \curvedir[0]\Gamma_\nu $$
		\begin{multline*}
			\cint[x_j]{\curvedir[0]\Gamma}{f(M)} \bydef \sum_{\nu = 1}^l \cint[x_j]{\curvedir[0]\Gamma_\nu}{f(M)} = \sum_{\nu = 1}^l \bigg( -\cint[x_j]{\curvedir[0]\Gamma_\nu}{f(M)} \bigg) = \\
			= -\sum_{\nu = 1}^l \cint[x_j]{\curvedir\Gamma}{f(M)} \bydef \cint[x_j]{\curvedir\Gamma}{f(M)}
		\end{multline*}
	\end{itemize}

	\item
	\begin{itemize}
		\item $ C^1 $-кривая
		$$ \cint[x_j]{\curvedir\Gamma} \bydef \dint[t]ab{c\gamma_j'(t)} \undereq{\text{ф. Ньютона"--~Лейбница}} c \big( \gamma_j(b) - \gamma_j(a) \big) $$
		\item $ \curvedir\Gamma = \bigcup_{\nu = 1}^l \curvedir\Gamma_\nu, \qquad \Gamma([t_{k - 1}, t_k]) $ "--- $ C^1 $-кривая
		$$ \cint[x_j]{\curvedir\Gamma}c \bydef \sum_{\nu = 1}^l \cint[x_j]{\curvedir\Gamma[t_{\nu - 1}, t_\nu]} = \sum_{\nu = 1}^l c \big( \gamma(t_\nu) - \gamma(t_{\nu - 1}) \big) = c \big( \gamma(t_l) - \gamma(t_0) \big) \bdefeq{t_0, t_l} c \big( \gamma(b) - \gamma(a) \big) $$
	\end{itemize}

	\item
	\begin{itemize}
		\item $ \Gamma \in C^1 $
		\begin{multline*}
			\bigg| \cint[x_j]{\curvedir\Gamma}{f(M)} \bigg| \bydef \bigg| \dint[t]ab{f \big( \Gamma(t) \big)\gamma_j'(t)} \bigg| \le \dint[t]ab{|f \big| ( \Gamma(t) \big) \big| \cdot |\gamma_j'(t)|} \le \\
			\le \dint[t]ab{ \big| f \big( \Gamma(t) \big) \big| \norm{\mc D \Gamma(t)}_n} \bydef \cint\Gamma{f(M)}
		\end{multline*}
		\item $ \curvedir\Gamma = \bigcup_{\nu = 1}^l \curvedir\Gamma_\nu, \qquad \curvedir\Gamma_\nu $ "--- $ C^1 $-кривая
		\begin{multline*}
			\bigg| \cint[x_j]{\curvedir\Gamma}{f(M)} \bigg| \bydef \bigg| \sum_{\nu = 1}^l \cint[x_j]{\curvedir\Gamma_\nu}{f(M)} \bigg| \le \sum_{\nu = 1}^l \bigg| \cint[x_j]{\curvedir\Gamma_\nu}{f(M)} \bigg| \trile \\
			\le \sum_{\nu = 1}^l \cint{\Gamma_\nu}{|f(M)|} = \cint\Gamma{f(M)}
		\end{multline*}
	\end{itemize}
\end{eproof}
