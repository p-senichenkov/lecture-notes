\chapter{ТФКП}

\section{Доказываем теорему о разложении в ряд Лорана}

\begin{proof}
	$$ f \in \mc A \bigg( D_{r, R}(a) \bigg), \qquad r_1 < r_2 < R_2 < R_1, \qquad r < r_1 < R_1 < R $$
	Введём $ S_\rho = \set{ z \mid |z - a| = \rho} $.
	\begin{itemize}
		\item Пусть $ z \in \ol D_{r, R}(a) $. \\
		Возьмём $ \veps < \min\set{ r_2 - r_1, R_1 - R_2} $. \\
		Обозначим $ \sigma_\veps = \set{\zeta \mid |\zeta - z| = \veps} $. \\
		Рассмотрим функцию
		$$ \vphi(\zeta) = \frac{f(\zeta)}{\zeta - z}, \qquad \vphi \in \mc A \bigg( D_{r, R}(a) \setminus \set{z} \bigg) $$
		$$ G_\veps = D_{r_1, R_1}(a) \setminus \set{\zeta \mid |\zeta - z| \le \veps} $$
		\begin{equ}1
			\cint[\zeta]{\curvedir \partial G_\veps}{\vphi(\zeta)} = 0
		\end{equ}
		$$ \eref1 \iff \cint[\zeta]{S_{R_1}}{\vphi(\zeta)} - \cint[\zeta]{S_{r_1}}{\vphi(\zeta)} - \cint[\zeta]{\sigma_\veps}{\vphi(\zeta)} = 0 $$
		\begin{equ}2
			\implies \frac1{2\pi i} = \cint[\zeta]{\sigma_\veps}{\frac{f(\zeta)}{\zeta - z}} = \cint[\zeta]{S_{R_1}}{\frac{f(\zeta)}{\zeta - z}} - \cint[\zeta]{S_{r_1}}{\frac{f(\zeta)}{\zeta - z}}
		\end{equ}
		$$ \set{\zeta \mid |\zeta - z| \le \veps} \sub D_{r, R}(a) $$
		\begin{equ}3
			\frac1{2\pi i}\cint[\zeta]{\sigma_\veps}{\frac{f(\zeta)}{\zeta - z}} = f(z)
		\end{equ}
		\begin{equ}4
			\eref2, \eref3 \implies f(z) = \frac1{2\pi i} \cint[\zeta]{S_{R_1}}{\frac{f(\zeta)}{\zeta - z}} - \frac1{2\pi i}\cint[\zeta]{S_{r_1}}{\frac{f(\zeta)}{\zeta - z}}
		\end{equ}
		$$ \frac1{\zeta - z} = \frac1{(\zeta - a) - (z - a)} = \frac1{\zeta - a} \cdot \frac1{1 - \frac{z - a}{\zeta - a}} $$
		Обозначим $ q_1(z, \zeta) = \frac{z - a}{\zeta - a} $.
		\begin{equ}5
			|q_1(z, \zeta)| \le \frac{R_2}{R_1} = Q_1 < 1
		\end{equ}
		\begin{multline}\lbl6
			\implies \frac1{2\pi i} \cint[\zeta]{S_{R_1}}{\frac{f(\zeta)}{\zeta - z}} = \frac1{2\pi i}\cint[\zeta]{S_{R_1}}{\frac{f(\zeta)}{\zeta - a} \cdot \frac1{1 - \frac{z - a}{\zeta - a}}} = \frac1{2\pi i}\cint[\zeta]{S_{R_1}}{\frac{f(\zeta)}{\zeta - a} \sum_{n = 0}^\infty \bigg( \frac{z - a}{\zeta - a} \bigg)^n} = \\
			= \sum_{n = 0}^\infty \frac1{2\pi i}(z - a)^n \cint[\zeta]{S_{R_1}}{\frac{f(\zeta)}{(\zeta - a}^{n + 1})}
		\end{multline}
		Обозначим $ c_n(R_1) = \frac1{2\pi i}\cint[\zeta]{S_{R_1}}{\frac{f(\zeta)}{(\zeta - a)^{n + 1}}} $.
		$$ \eref6 \implies \frac1{2\pi i} \cint[\zeta]{S_{R_1}}{\frac{f(\zeta)}{\zeta - z}} = c_0(R_1) + \sum_{n = 1}^\infty c_n(R_1)(z - a)^n $$
		\item Пусть $ z $ лежит на окружности.
		\begin{equ}9
			-\frac1{\zeta - z} = -\frac1{(\zeta - a) - (z - a)} = -\frac1{z - a} \cdot \frac1{\frac{\zeta - a}{z - a} - 1} = \frac1{z - a} \cdot \frac1{1 - \frac{\zeta - a}{z - a}}
		\end{equ}
		$$ \comment{\widedots[10cm]} $$
		\begin{equ}{12}
			c_{-n - 1}(r_1) = \frac1{2\pi i} \cint[\zeta]{S_{r_1}}{f(\zeta)(\zeta - a)^n}
		\end{equ}
		$$ \implies -\frac1{2\pi i}\cint[\zeta]{S_{r_1}}{\frac{f(\zeta)}{\zeta - z}} = \sum_{n = 1}^\infty c_{-n}(r_1) (z - a)^{-n} $$
	\end{itemize}

	Воспользуемся леммой:
	\begin{equ}{15}
		\iff -\cint[\zeta]{S_{\rho_1}}{\vphi(\zeta)} + \cint[\zeta]{S_{\rho_2}}{\vphi(\zeta)} = 0 \iff \frac1{2\pi i}\cint[\zeta]{S_{R_1}}{\frac{f(\zeta)}{(\zeta - a)^{n + 1}}} = \frac1{2\pi i} \cint[\zeta]{S_{\rho_0}}{\frac{f(\zeta)}{(\zeta - a)^{n + 1}}} \bydef c_n
	\end{equ}
	\begin{equ}{16}
		\frac1{2\pi i} \cint[\zeta]{S_{r_1}}{f(\zeta)(\zeta - a)^n} = \frac1{2\pi i}\cint[\zeta]{S_{\rho_0}}{f(\zeta)(\zeta - a)^n} \bydef c_{-n - 1}
	\end{equ}
	$$ \eref4, \eref{15}, \eref{16} \implies f(z) = \sum_{n = 1}^\infty c_{-n}(z - a)^{-n} + c_0 + \sum_{n = 1}^\infty c_n(z - a)^n $$
\end{proof}

\begin{lemma}
	$ r < \rho_1 < \rho_2 < R, \qquad m \in \Z $
	\begin{equ}{14}
		\implies \frac1{2\pi i} \cint[\zeta]{S_{\rho_1}}{f(\zeta)(\zeta - a)^m} = \frac1{2\pi i}\cint[\zeta]{S_{\rho_2}}{f(\zeta)(\zeta - a)^m}
	\end{equ}
\end{lemma}

\begin{proof}
	Возьмём $ \vphi(\zeta) = f(\zeta)(\zeta - a)^m \quad \in \mc A \big( D_{r, R}(a) \big) $

	По теореме Коши,
	$$ \cint[\zeta]{\partial D_{\rho_1, \rho_2}(a)}{\vphi(\zeta)} = 0 $$
	$$ \comment{\widedots[5cm]} $$
\end{proof}

\section{Особые точки аналитических функций}

\begin{definition}
	$ f \in \mc A \big( D_{0, R}(a) \big) $

	Говорят, что $ a $ --- \it{особая точка} функции $ f $.
\end{definition}

\subsection{Классификация особых точек}

Доказано, что
\begin{equ}{21}
	f(z) = \sum_{n = 1}^\infty c_{-n}(z - a)^{-n} + c_0 + \sum_{n = 1}^\infty c_n(z - a)^n
\end{equ}

Говорят, что
\begin{enumerate}
	\item $ a $ --- устранимая особая точка, если $ c_{-n} = 0 \quad \forall n \ge 1 $;
	\item функция $ f $ имеет в $ a $ \it{полюс}, если
	$$ \exist n_0 \ge 1 : \quad c_{-n_0} \ne 0, \quad c_n = 0 \quad \forall n > n_0 $$
	\item функция $ f $ имеет \it{\dots}, если
	$$ \exist \seq{n_k} : \quad c_{-n_k} \ne 0 $$
\end{enumerate}

\subsection{Характеристика устранимой особой точки}

\begin{theorem}
	Чтобы точка $ a $ была устранимой особой точкой, \bt{необхдимо и достаточно}, чтобы
	\begin{equ}{26}
		\exist 0 < r < R, \quad \exist M : \quad |f(z)| \le M \quad \forall z \in D_{0, r}(a)
	\end{equ}
\end{theorem}

\begin{eproof}
	\item Необходимость

	Из условия на устранимую особую точку и \eref{21} следует, что
	$$ f(z) \in \dots $$
	(по последней теореме предыдущего семетра)
	\item Достаточность

	Возьмём $ 0 < \veps < r $ и $ \veps < |z - a| < r $.
	$$ f(z) = \frac1{2\pi i} \cint[\zeta]{S_r}{\frac{f(\zeta)}{\zeta - z}} - \frac1{2\pi i}\cint[\zeta]{S_\veps}{\frac{f(\zeta)}{\zeta - z}} $$
	\begin{equ}{27}
		\bigg| -\frac1{2\pi i} \cint[\zeta]{S_\veps}{\frac{f(\zeta)}{\zeta - z}} \bigg| \le \frac1{2\pi} \acint[\zeta]{S_\veps}{\frac{|f(\zeta)|}{|\zeta - z|}}
	\end{equ}
	\begin{equ}{28}
		|\zeta - z| \ge |z - a| - |\zeta - a| = |z - a| - \veps \ge \frac12|z - a|
	\end{equ}
	\begin{equ}{29}
		\eref{26}, \eref{27}, \eref{28} \implies \bigg| %-\frac1{\2pi i} \cint[\zeta]{S_\veps}{\frac{f(\zeta)}{\zeta - z}} \le \frac1{2\pi} \cdot \frac M{\frac12 |z - a|} \cdot 2\pi \veps = \frac{2 M \veps}{|z - a|}
	\end{equ}
	$$ \eref{27}, \eref{29} \implies f(z) = \liml{\veps \to 0} \bigg( \frac1{2\pi i} \cint[\zeta]{S_r}{\frac{f(\veps)}{\zeta - z}} - \frac1{2\pi i} \cint[\zeta]{S_\veps}{\frac{f(\zeta)}{\zeta - z}} \bigg) = \frac1{2\pi i} \cint[\zeta]{S_r}{\frac{f(\zeta)}{\zeta - a}} = f(a) + \sum_{n = 1}^\infty \frac{\nder f(a)}{n!} (z - a)^n $$
\end{eproof}

\subsection{Характеристика полюса}

\begin{theorem}
	Для того, чтобы $ a $ была полюсом \bt{необходимо и достаточно}, чтобы
	\begin{equ}{210}
		|f(z)| \underarr{z \to a} +\infty
	\end{equ}
\end{theorem}

\begin{eproof}
	\item Достаточность
	\begin{equ}{211}
		f(z) = \sum_{n = 1}^{n_0} c_{-n}(z - a)^{-n} + c_0 + \sum_{n = 1}^\infty (z - a)^n
	\end{equ}
	$$ \implies f(z) = (z - a)^{-n_0} \bigg( \sum_{n = 1}^\infty c_{-n}(z - a)^{n_0 - n} + c_0(z - a)^{n_0} + \sum_{n = 1}^\infty c_n(z - a)^{n + n_0} \bigg) = (z - a)^{-n_0} \bigg( c_{-n_0} + c_{-n_0 + 1}(z - a) + \dots \bigg) $$
	\begin{equ}{212}
		\exist \delta_0 > 0 : \quad |c_{-n_0} + c_{-n_0 + 1}(z - a) + \dots| \ge \frac12 |c_{-n_0}| \quad \text{ при } |z - a| < \delta_0
	\end{equ}
	$$ \implies |f(z)| \ge |z - a|^{-n_0} \cdot \frac{|c - n_0|}2 \underarr{z \to a} +\infty $$

	\item Необходимость
	$$ \eref{210} \implies \exist \delta_1 : \quad |f(z)| > 1 \quad \text{ при } |z - a| < \delta_1 $$
	$$ \implies f(z) \ne 0 \quad \text{ при } z \in D_{0, \delta_1}(a) \quad \implies \quad \vphi(z) \define \frac1{f(z)} \in \mc A \big( D_{0, \delta_1}(a) \big) $$
	$$ \vphi(z) \not\equiv 0 $$
	$$ \eref{210} \implies |\vphi(z)| \underarr{z \to 0} 0 $$
	$$ |\vphi(z)| < 1 $$
	$$ \implies \vphi \in \mc A \big( \ttm B_{\delta_1}(a) \big) $$
	$$ \implies \vphi(a) = 0 $$
	$$ \implies \exist n_0 \in \N, \quad \exist h \in \mc A \big( \ttm B_{\delta_1}(a) \big) : \quad \psi(z) = (z - a)^{n_0}h(z), \quad h(z) \ne 0, \quad z \in \ttm B_{\delta_2}(a) $$
	$$ \implies f(z) = \frac1{(z - a)^{n_0}} \cdot \frac1{h(z)} $$
	$$ \implies g(z) = \frac1{h(z)} \in \mc A \big( \ttm B_{\delta_2}(a) \big) $$
	$$ \implies g(z) = c_0 + c_1(z - a) + c_2(a - a)^2 + \dots $$
	$$ \implies f(z) = \frac1{(z - a)^{n_0}} \bigg( c_0 + c_1(z - a) + \dots \bigg) = c_0(z - a)^{-n_0} + c_1(z - a)^{-n_0 + 1} + \dots $$
\end{eproof}
