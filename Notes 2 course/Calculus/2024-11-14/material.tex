\chapter{Функциональные последовательности и ряды}

\section{Равномерно сходящиеся комплексные функциональные ряды}

\begin{definition}
	$ E $ "--- метрическое пространство, $ \qquad \seq{\gamma_n(x)}n, \quad \gamma E \to \Co $ \\
	Комплексным функциональным рядом называется символ
	\begin{equ}{19}
		\sum_{n = 1}^\infty \gamma_n(x)
	\end{equ}
	Частичной суммой называется $ w_n(x) \define \gamma_1(x) + \dots + \gamma_n(x) $ \\
	Говорят, что ряд равномерно сходится на множестве $ E $, если
	$$ \exist w : E \to \Co : \quad w_n(x) \uniarr{x \in E} w(x) $$
\end{definition}

\subsection{Критерий Коши}

\begin{theorem}
	Для того чтобы ряд \eref{19} \bt{равномерно} сходился на $ E $, \bt{необходимо и достаточно}, чтобы
	$$ \forall \veps > 0 \quad \exist N : \quad \forall m > n > N \quad \forall x \in E \quad |\gamma_{n + 1}(x) + \dots + \gamma_m(x)| < \veps $$
\end{theorem}

\begin{definition}
	МОжно применить критерий Коши для комплексной функциональной последовательности
\end{definition}

\subsection{Признак Вейерштрасса}

\begin{theorem}
	\begin{equ}{21}
		\seq{a_n}n, \quad a_n > 0 : \quad |\gamma_n(x)| \le a_n \quad \forall n \quad \forall x \in E
	\end{equ}
	Ряд $ a_n $ сходится
	\begin{equ}{22}
		\sum_{n = 1}^\infty a_n < \infty
	\end{equ}
	$$ \implies \eref{19} \text{ сходится равномерно} $$
\end{theorem}

\begin{proof}
	Возьмём $ \forall \veps > 0 $
	\begin{equ}{23}
		\eref{22} \implies \exist N : \quad \forall m > n > N \quad a_{n + 1} + \dots + a_n < \veps
	\end{equ}
	$$ |\gamma_{n + 1}(x) + \dots + \gamma_n(x)| \le |\gamma_{n + 1}(x)| + \dots + |\gamma_m(x)| \underset{a_n > 0}\le a_{n + 1} + \dots + a_m \underset{\eref{23}}< \veps $$
	По критерию Коши получаем равномерную сходимость
\end{proof}

\section{Комплексные степенные ряды}

$ E = \Co $
$$ \seqz{c_n}n, \qquad c_n \in \Co $$
$ z_0 \in \Co $ \\
Положим $ \gamma_0(z) \define c_0, \quad \gamma_n(z) \define c_n(z - z_0)^n $
\begin{equ}{32}
	c_0 + \sum_{n = 1}^\infty c_n(z - z_0)^n
\end{equ}
Такое выражение будем называть комплексным степенным рядом с центром $ z_0 $ (рядом по степеням $ (z - z_0) $)

\begin{remark}
	$ \seq{c_n}n, \qquad c_n \to c \in \Co $
	$$ \implies \exist M : \quad |c_n| \le M \quad \forall n $$
\end{remark}

\begin{proof}
	Положим $ c_n = a_n + ib_n, \quad c = a + ib $
	$$ a_n \to a, \qquad b_n \to b $$
	Дальше применяем теорему из первого семестра
\end{proof}

\begin{remark}[необходимый признак сходимости комплексных числовых рядов]
	$$ \sum_{n = 1}^\infty \text{ сх. } \implies \gamma_n \infarr{n} 0 $$
\end{remark}

\begin{proof}
	$ c_n = \gamma_1 + ... + \gamma_n $
	$$
	\begin{rcases}
		c_n \to c \\
		c_{n - 1} \to c
	\end{rcases} \implies \underbrace{c_n - c_{n - 1}}_{\gamma_n} \to c - c = 0 $$
\end{proof}

\begin{implication}
	$$ \exist M : \quad |\gamma_n| \le M \quad \forall n $$
\end{implication}

\begin{lemma}[Абеля]
	$$ \exist z_1 \ne z_0 : \quad \eref{32} \text{ сходится при } z_1 $$
	Обозначим $ R \define |z_1 - z_0| $
	\begin{equ}{33}
		\implies \eref{32} \text{ сх. } \quad \forall z : |z - z_0| < R
	\end{equ}
	\begin{equ}{34}
		\implies \forall 0 < r < R \quad \eref{32} \text{ равн. сх. при } |z - z_0| \le r
	\end{equ}
\end{lemma}

\begin{proof}
	Докажем \eref{34}: \\
	Обозначим $ 0 < q \define \frac r R < 1 $ \\
	Сходимость при $ z_1 $, по первому замечанию, означает, что
	\begin{equ}{35}
		c_n(z_1 - z_0)^n \infarr{n} 0
	\end{equ}
	Тогда, по следствию,
	\begin{equ}{36}
		\exist M : \quad |c_n(z_1 - z_0)^n| \le M
	\end{equ}
	\begin{equ}{37}
		\iff |c_n| \cdot |z_1 - z_0| \le M \overset{\operatorname{def}~R}\iff |c_n| \le \frac{M}{R^n}
	\end{equ}
	\begin{equ}{38}
		|c_n(z - z_0)^n| = |c_n| \cdot |z - z_0|^n \underset{\text{усл. } \eref{34}, \eref{37}}\le \frac{M}{R^n} \cdot r^n = Mq^n
	\end{equ}
	$$ \sum_{n = 1}^\infty Mq^n = \frac{Mq}{1 - q} $$
	Можно применить признак Вейерштрасса, тем самым доказывая \eref{34}
	$$ \eref{34} \implies \eref{32} \text{ сх. абс. при } |z - z_0| < R $$
\end{proof}

\subsection{Радиус сходимости и  круг сходимости}

\begin{definition}
	\begin{enumerate}
		\item Пусть \eref{32} сходится только при $ z = z_0 $ \\
		Будем полагать радиус сходимости $ R \define 0 $, круг сходимости $ \B \define \O $
		\item \eref{32} сходится при всех $ z $ \\
		Полагаем $ R \define +\infty, \quad \B \define \Co $
		\item $ \exist z_1 \ne z_0 : \quad \eref{32} \text{ сх. в } z_1, \qquad \exist z_2 : \quad \eref{32} \text{ расх. в } z_2 $
		$$ R \define \sup\set{r \mid r = |z_* - z_0|, \quad \eref{32} \text{ сх. в } z_*}, \qquad \B \define \set{z_0 \mid |z - z_0| < R} $$
	\end{enumerate}
\end{definition}

Положим $ r_1 \define |z_1 - z_0|, \quad r_2 \define |z_2 - z_1| $ \\
По определению $ R $
$$ R \ge r_1 > 0 $$
Возьмём $ z_3 : \quad r_3 \define |z_3 - z_0| > r_2 $ \\
Если бы \eref{32} сходился при $ z_3 $, можно было бы применить к $ z_3 $ лемму Абеля. Тогда бы \eref{32} сходился в $ z_2 $ "--- \contra \\
То есть, в $ z_3 $ ряд расходится \\
Значит, $ R \le r_2, \quad r_1 < r_2 $

\subsection{Свойства круга сходимости}

Рассматриваем только случай, когда $ 0 < R < \infty $

\begin{theorem}
	\begin{equ}{312}
		\eref{32} \text{ сх. } \quad \forall z \in \B
	\end{equ}
	\begin{equ}{313}
		\eref{32} \text{ расх. } \quad \forall z_2 \in \Co \setminus \ol{B}
	\end{equ}
\end{theorem}

\begin{iproof}
	\item Докажем \eref{312}: \\
	Возьмём $ r \define |z - z_0| < R $ \\
	По определению $ R $
	$$ \exist z_* : \quad |z_* - z_0| > R, \quad \eref{32} \text{ сх. в } z_* $$
	По лемме Абеля \eref{32} сх. в $ z $
	\item Докажем \eref{313}: \\
	Возьмём $ \rho \define |\hat z - z_0| > R $ \\
	Если ряд сходится, то $ \rho $ больше супремума, что невозможно
\end{iproof}

\begin{definition}
	Определим
	\begin{equ}{314}
		t \define \ulim{n \to \infty} \sqrt[n]{c_n}
	\end{equ}
\end{definition}

\begin{theorem}
	\hfill
	\begin{enumerate}
		\item $ R = 0 $, если $ t = +\infty $
		\item $ R = +\infty $, если $ t = 0 $
		\item $ R = \frac1t $ иначе
	\end{enumerate}
\end{theorem}

\begin{proof}
	Будем рассматривать только последний случай \\
	Определим $ R_0 \define \frac1t $
	\begin{itemize}
		\item Возьмём $ z_2 : \quad |z_2 - z_0| > R_0 $ \\
		Обозначим $ \veps \define |z_2 - z_0| - R_0 > 0 $ \\
		Определим
		$$ \delta \define \frac{\veps t^2}{1 + \veps t} $$
		По определению верхнего предела
		\begin{equ}{316}
			\exist \seq{n_k}k : \quad \sqrt[n_k]{c_{n_k}} > t - \delta
		\end{equ}
		$$ \iff |c_{n_k}| > (t - \veps)^{n_k} $$
		$$ \implies |c_{n_k}(z_2 - z_0)^{n_k}| = |c_{n_k}| \cdot |z_2 - z_0|^{n_k} > (t - \delta)^{n_k} \cdot (R_0 + \veps)^{n_k} = \bigg( (t - \delta)(R_0 + \veps) \bigg)^{n_k} $$
		$$ (t - \delta)(R_0 + \veps) = \bigg( t - \frac{\veps t^2}{1 + \veps t} \bigg)\bigg( \frac1t + \veps \bigg) = \frac{t + \veps t^2 - \veps t^2}{1 + \veps t} \cdot \frac{1 + \veps t}t = 1 $$
		\begin{equ}{319}
			\implies |c_{n_k}(z_2 - z_0)^{n_k}| \ge 1
		\end{equ}
		По второму замечанию ряд в $ z_2 $ расходится
		\item Возьмём $ z_1 : \quad |z_1 - z_0| < R_0 $ \\
		Пусть
		$$ \veps_0 \define R_0 - |z_1 - z_0|, \quad \delta_0 \define \half \cdot \frac{\veps_0t^2}{1 - \veps_0t} $$
		По свойствам верхнего предела
		$$ \exist N : \quad \forall n > N \quad \sqrt[n]{|c_n|} < t + \delta_0 $$
		$$ \iff |c_n| < (t + \delta_0)^n $$
		$$ \implies \forall n > N \quad |c_n(z_1 - z_0)^n| = |c_n| \cdot |z_1 - z_0|^n < (t + \delta_0)^n \cdot (R_0 - \veps_0)^n = \bigg( (t - \delta_0)(R_0 - \veps_0) \bigg)^n $$
		$$ (t + \delta_0)(R_0 - \veps_0) \bdefeq{\delta} \bigg( t + \half \cdot \frac{\veps_0t^2}{1 - \veps_0t} \bigg) \bigg( \frac1t - \veps_0 \bigg) = \frac{t - \veps_0t^2 + \half\veps_0t^2}{1 - \veps_0t} \cdot \frac{1 - \veps_0t}{t} = 1 - \half\veps_0t $$
		$$ 0 < q \define 1 -\half \veps_0t < 1 $$
		$$ \implies |c_n(z_1 - z_0)^n| < q^n < 1 $$
		Значит, ряд сходится при $ z_1 $
	\end{itemize}
\end{proof}

\begin{theorem}
	$ c_n \ne 0 \quad \forall n, \qquad \exist \limi{n} \frac{|c_n|}{|c_{n + 1}|} $ \\
	Тогда этот предел и равен радиусу сходимости
\end{theorem}
