\section{Теорема Лагранжа для вектор-функций}

\begin{definition}
	Отображение $ F : (a, b) \to \R^{n \ge 2} $ называют вектор-функцией, заданной на $ (a, b) $
\end{definition}

\begin{remark}
	$ \R $ можно трактовать как пространство вектор-столбцов, состоящих из одного элемента
\end{remark}

\begin{statement}
	$ F : (a, b) \to \R^{n \ge 2}, \qquad F(t) = \column{f_1(t)}{f_n(t)}, \qquad t_0 \in (a, b) $
	\begin{remind}
		По теореме из прошлого семестра, $ F $ дифференцируема в $ t_0 $ тогда и только тогда, когда $ f_j(t) $ дифф. в $ t_0 \quad j = 1, ..., n $
	\end{remind}
	\begin{remind}
		$ f_j(t) $ дифф. в $ t_0 $ тогда и только тогда, когда $ \exist f_j'(t_0) $
	\end{remind}
	$$ \mc{D}F(t_0) = \column{f_1'(t_0)}{f_n'(t_0)} $$
	\begin{multline*}
		\underset{\text{(как линейного отображения)}}{\norm{\mc{D}F(t_0)}} = \sup\limits_{
			\begin{subarray}{c}
				|h| \le 1 \\
				h \in \R
			\end{subarray}} \norm{\mc{D}F(t_0)h}_n = \sup\norm{h\mc{D}F(t_0)}_n = \sup |h| \norm{\mc{D}F(t_0)}_n = \\
		= \norm{\mc{D}F(t_0)}_n = \norm{\column{f_1'(t_0)}{f_n'(t_0)}}_n
	\end{multline*}
\end{statement}

\begin{theorem}[Лагранжа]
	$ F : [a, b] \to \R^{n \ge 2}, \qquad F \in \Cont{[a, b]} $
	\begin{remind}
		$ F \in \Cont{[a, b]} \iff f_j \in \Cont{[a, b]} \quad j = 1, ..., n $
	\end{remind}
	$ \forall t \in (a, b) \quad F $ дифф. в $ t $
	\begin{equ}1
		\implies \exist c \in (a, b) : \norm{F(b) - F(a)}_n \le \norm{\mc{D}F(c)}_n(b - a)
	\end{equ}
\end{theorem}

\begin{proof}
	Возьмём
	\begin{equ}2
		\vphi(t) \define F(t)^T \bigg( F(b) - F(a) \bigg) = f_1(t) \bigg( f_1(b) - f_1(a) \bigg) + ... + f_n(t) \bigg( f_n(b) - f_n(a) \bigg)
	\end{equ}
	Будем считать, что $ F(b) \ne F(a) $ (иначе -- очевидно) \\
	По напоминанию из условия теоремы, $ \vphi \in \Cont{[a, b]} $. Значит, $ \forall t \in (a, b) \quad \exist \vphi'(t) $
	\begin{equ}4
		\vphi'(t) = f_1'(t) \bigg( f_1(b) - f_1(a) \bigg) + ... + f_n'(t) \bigg( f_n(b) - f_n(a) \bigg)
	\end{equ}
	К $ \vphi $ можно применить теорему Лагранжа из первого семестра:
	\begin{equ}5
		\exist c \in (a, b) : \vphi(b) - \vphi(a) = \vphi'(c)(b - a)
	\end{equ}
	\begin{multline}\lbl7
		\vphi(b) - \vphi(a) \undereq{\eref2} \bigg\lgroup f_1(b) \bigg(f_1(b) - f_1(a) \bigg) + \widedots[4em] \bigg\rgroup - \bigg\lgroup f_1(a) \bigg( f_1(b) - f_1(a) \bigg) + \widedots[4em] \bigg\rgroup = \\
		= \bigg( f_1(b) - f_1(a) \bigg)^2 + ... + \bigg( f_n(b) - f_n(a) \bigg)^2 = \norm{F(b) - F(a)}_n^2
	\end{multline}
	Применим к \eref4 неравенство КБШ:
	\begin{multline}\lbl8
		|\vphi'(c)| \le \bigg\lgroup \bigg( f_1'(c) \bigg)^2 + ... + \bigg(f_n'(c) \bigg)^2 \bigg\rgroup^\frac12 \cdot \bigg\lgroup \bigg( f_1(b) - f_1(a) \bigg)^2 + ... + \bigg( f_n(b) - f_n(a) \bigg)^2 \bigg\rgroup^\frac12 = \\
		= \norm{\mc{D}F(c)} \cdot \norm{F(b) - F(a)}
	\end{multline}
	$$ \eref7, \eref8 \implies \norm{F(b) - F(a)}^2 \le \norm{\mc{D}F(c)} \cdot \norm{F(b) - F(a)} \implies \eref1 $$
\end{proof}

\section{Теорема об обратимости линейного отображения, близкого к обратимому}

\begin{theorem}
	$ \mc{A} : \R^{n \ge 2} \to \R^n $ -- линейное, т. е. $ \mc{A}(X) = AX, \quad X \in \R^n $ \\
	$ A $ обратимо, т. е. $ \exist D : AD = I $ и $ DA = I $ ($ D $ называется обратной матрицей и обозначается $ D = A^{-1} $) \\
	Вспомним две теоремы из алгебры:
	\begin{remind}
		$ A $ обратима $ \iff \det A \ne 0 $
	\end{remind}
	\begin{remind}
		$ A $ обратима $ \iff AX \ne \On \quad \forall X \ne \On $
	\end{remind}
	$ \norm{A^{-1}} = \frac1\alpha, \quad \alpha > 0, \qquad \norm{B - A} = \beta, \quad 0 < \beta < \alpha $ \\
	Тогда $ B $ обратима и $ \norm{A^{-1} - B^{-1}} \le \dfrac{\beta}{\alpha(\beta - \alpha)} $
\end{theorem}

\begin{iproof}
	\item Докажем, что $ B $ обратима: \\
	Возьмём $ X \in \R^n $
	\begin{multline}\lbl9
		X = IX = (A^{-1}A)X = A^{-1}(AX) \implies \norm{X} = \norm{A^{-1}(AX)} \le \\
		\le \norm{A^{-1}}\norm{AX} \bdefeq\alpha \frac1\alpha \norm{AX} \implies \norm{AX} \ge \alpha\norm{X}
	\end{multline}
	\begin{equ}{10}
		BX = AX + (BX - AX) \implies \norm{BX} \trige \norm{AX} - \norm{BX - AX}
	\end{equ}
	\begin{equ}{11}
		\norm{BX - AX} = \norm{(B - A)X} \le \norm{B - A}\norm{X}_n
	\end{equ}
	\begin{equ}{12}
		\norm{BX}_n \underset{\eref{10}}\ge \underbrace{\alpha \norm{X}}_{\eref9} - \underbrace{\norm{B - A}\norm{X}}_{\eref{11}} = \underbrace{(\alpha - \beta)}_{> 0} \norm{X}
	\end{equ}
	Это означает, что $ B $ обратима (по второй теореме из алгебры)
	\item Докажем соотношение для $ \norm{A^{-1} - B^{-1}} $: \\
	Возьмём $ \forall ~ Y \ne \On $ и $ X \define B^{-1}Y $
	\begin{equ}{13}
		\norm{B(B^{-1}Y)} \bdefeq{Y} \norm{BX} \underset{\eref{12}}\ge (\alpha - \beta) \norm{X} \bdefeq{X} (\alpha - \beta) \norm{B^{-1}Y}
	\end{equ}
	$$ B(B^{-1}Y) \undereq{\text{асс.}} (BB^{-1})Y = IY = Y $$
	\begin{equ}{15}
		\eref{13} \implies \norm{B^{-1}Y} \le \frac1{\alpha - \beta} \norm{Y} \implies \norm{B^{-1}} \le \frac1{\alpha - \beta}
	\end{equ}
	\begin{multline*}
		A(A^{-1} - B^{-1})B \undereq{\text{асс.}} \bigg( A(A^{-1} - B^{-1}) \bigg) B \undereq{\text{дистр.}} ( AA^{-1} - AB^{-1})B = (I - AB^{-1})B \undereq{\text{дистр.}} \\
		= IB - (AB^{-1})B \undereq{\text{асс.}} B - A(B^{-1}B) = B - AI = B - A
	\end{multline*}
	\begin{equ}{17}
		\implies \underbrace{(A^{-1}A)}_{= I}(A^{-1} - B^{-1})\underbrace{(BB^{-1})}_{= I} = A^{-1}(B - A)B^{-1} \implies A^{-1} - B^{-1} = A^{-1}(B - A)B^{-1}
	\end{equ}
	$$ \norm{A^{-1} - B^{-1}} \underset{\eref{17}}\le \norm{A^{-1}} \cdot \norm{B - A} \cdot \norm{B^{-1}} \underset{
		\begin{subarray}{c}
			\eref{15} \\
			\text{определения } \alpha \text{ и } \beta
		\end{subarray}}\le \frac1\alpha \cdot \beta \cdot \frac1{\alpha - \beta} $$
\end{iproof}

\section{Обратимые отображения}

\begin{statement}
	$ E \sub \R^{n \ge 2}, \qquad X_0 $ -- внутр. точка, $ \qquad F : E \to \R^n $ -- биекция, $ \qquad G = F(E) $
	\begin{remark}
		$ F $ -- биекция $ \iff F $ обратима
	\end{remark}
	$ \Phi : G \to E, \qquad \Phi \bigg( F(X) \bigg) = X \quad \forall X, \qquad F \bigg( \Phi(X) \bigg) = Y \quad \forall Y \in G \qquad $ (т. е. $ \Phi = F^{-1} $) \\
	$ F $ дифференцируемо в $ X_0, \qquad Y_0 = F(X_0), \qquad \Phi $ дифференцируемо в $ Y_0 $ \\
	Обозначим $ I(X) = X \quad \forall X \in E $ -- тождественное отображение
	\begin{remark}
		Тождественное отображение дифференцируемо. Его матрица Якоби: $ \mc{D}I(X) = I_n $
	\end{remark}
	Тогда $ \Phi \bigg( F(X) \bigg) = I(X) $ \\
	Применим теорему о дифференцируемости суперпозиции (утверждение про матрицы Якоби из неё):
	$$ \Phi \bigg(F(X) \bigg) = I(X) \implies \mc{D}\Phi(Y_0)\mc{D}F(X_0) = \mc{D}I(X_0) = I_n $$
	По определению, если произведение матриц является единичной матрицей, то они обратны друг другу:
	$$ \mc{D}\Phi(Y_0) = \bigg( \mc{D}F(X_0) \bigg)^{-1} $$
	Получили необходимое условие для обратных отображений
\end{statement}

\begin{theorem}[об обратном отображении]
	$ E \sub \R^{n \ge 2}, \qquad E $ открыто, $ \qquad X_0 \in E, \qquad F : E \to \R^1 $ \\
	$ F \in \Cont[1]E $, т. е. все координатные функции $ \in \mc{C}^{1}, \qquad Y_0 = F(X_0), \qquad \mc{D}F(X_0) $ обратима
	$$ \implies \exist U \text{ -- окрестность } X_0, V \text{ -- окрестность } Y_0 :
	\begin{cases}
		F\clamp{U} \text{ обратимо} \\
		F(U) = V \\
		\Phi = \bigg( F\clamp{U} \bigg)^{-1} \implies \Phi \in \Cont[1]{V}
	\end{cases} $$
\end{theorem}

\begin{proof}
	\hfill
	\begin{enumerate}
		\item Определение множества $ U $ \\
		Обозначим $ A \define \mc{D}F(X_0) $. По условию, она обратима \\
		Положим $ \lambda \define \dfrac1{4\norm{A^{-1}}} $ \\
		Обозначим $ F = \column{f_1}{f_n} $
		$$ \mc{D}F(X) =
		\begin{bmatrix}
			f_{1x_1}'(X) & ... & f_{1x_n}'(X) \\
			. & . & . \\
			f_{1x_1}'(X) & ... & f_{nx_n}'(X)
		\end{bmatrix} $$
		$$ \mc{D}F(X) - \mc{D}F(X_0) =
		\begin{bmatrix}
			f_{1x_1}'(X) - f_{1x_1}'(X_0) & ... & f_{1x_n}'(X) - f_{1x_n}'(X_0) \\
			. & . & . \\
			f_{nx_1}'(X) - f_{nx_1}'(X_0) & ... & f_{nx_n}'(X) - f_{nx_n}'(X_0)
		\end{bmatrix} $$
		По свойству 6 нормы матрицы (лекция от 05.09.2024),
		$$ \norm{\mc{D}F(X) - A} \bdefeq{A} \norm{\mc{D}F(X) - \mc{D}F(X_0)} \le \bigg\lgroup \sum_{
			\begin{subarray}{c}
				i = 1, ..., n \\
				j = 1, ..., n
			\end{subarray}} \bigg( f_{ix_j}'(X) - f_{ix_j}'(X_0) \bigg)^2 \bigg\rgroup^{\frac12} $$
		$$ \underimp{F \in \mc{C}^1} \norm{\mc{D}F(X) - \mc{D}F(X_0)} \underarr{X \to X_0} 0 $$
		\begin{equ}{25}
			\implies \exist r > 0 : \forall X \in B_r(X_0) \quad \norm{\mc{D}F(X) - A} < 2\lambda
		\end{equ}
		Положим $ U \define B_r(X_0) $
		\item Инъективность $ F $ \\
		Далее будем рассматривать $ F $ только на $ U $ (т. е. будем писать $ F \define F\clamp{U} $)
		\begin{remark}[о выпуклости шара]
			$ X_1, X_2 \in U, \qquad 0 < t < 1 $
			$$ \implies tX_1 + (1 - t)X_2 \in U $$
			(см. рис. \ref{fig:1})
		\end{remark}
		\begin{remark}
			Используем наши обозначения (т. е. $ U = B_r(X_0) $)
		\end{remark}
		\begin{proof}
			\begin{multline*}
				\norm{tX_1 + (1 - t)X_2 - X_0} = \norm{t(X_1 - X_0) + (1 - t)(X_2 - X_0)} \trile \\
				\le \norm{t(X_1 - X_0)} + \norm{(1 - t)(X_2 - X_0)} < t \cdot r + (1 - t) \cdot r = r
			\end{multline*}
		\end{proof}
		\begin{implication}
			$ X \in U, \qquad X + H \in U, \qquad 0 < t < 1 $
			$$ \implies X + tH \in U $$
		\end{implication}
		\begin{proof}
			$ X_1 \define X + H, \qquad X_2 \define X $
			$$ tx_1 + (1 - t)X_2 = tX + tH + (1 - t)X = X + tH $$
		\end{proof}
		Возьмём $ X \in U $ и $ H \ne \On $, такие что $ X + H \in U $ \\
		Докажем, что $ F(X + H) - F(X) \stackrel?\ne \On $. Это и будет означать инъективность \\
		Возьмём $ t \in [0, 1] $ и $ P(t) \define F(X + tH) - tAH $ \\
		Это вектор-функция $ P : [0, 1] \to \R^n $
		\begin{equ}{26}
			\mc{D}P(t) \bdefeq{P} \mc{D} \bigg( F(X + tH) \bigg) - \mc{D}(tAH)
		\end{equ}
		Положим $ q(t) \define X + tH $ \\
		Теперь можно переписать \eref{26}:
		\begin{equ}{27}
			\mc{D}P(t) = \mc{D} \bigg( F \big( q(t) \big) \bigg) - \mc{D}(tAH)
		\end{equ}
		\begin{remind}
			Мы уже доказали, что для $ Y \in \R^n $ и отображения $ t \mapsto tY $ выполнено $ \mc{D}(tY) = Y $
		\end{remind}
		$$ \mc{D}(tAH) = AH, \qquad \mc{D}q(t) = H $$
		$$ \mc{D}F \bigg( q(t) \bigg) = \mc{D}F \bigg( q(t) \bigg)\mc{D}q(t) = \mc{D}F(X + tH)H $$
		Подставим это в \eref{27}:
		\begin{equ}{28}
			\mc{D}P(t) = \mc{D}F(X + tH)H - AH
		\end{equ}
		$$ \lambda \bydef \frac1{4\norm{A^{-1}}} \implies \norm{A^{-1}} = \frac1{4\lambda} $$
		Возьмём $ H \ne \On $
		\begin{multline}\lbl{29}
			H = (A^{-1}A)H = A^{-1}AH \implies \norm{H} = \norm{A^{-1}(AH)} \le \norm{A^{-1}}\norm{AH} = \frac1{4\lambda}\norm{AH} \implies \\
			\implies \norm{AH} \ge 4\lambda\norm{H}
		\end{multline}
		\begin{equ}{210}
			P(1) - P(0) \bdefeq{P} F(X + H) - AH - F(X) = F(X + H) - F(X) - AH
		\end{equ}
		Применим к $ P $ теорему Лагранжа для вектор-функции:
		\begin{multline}\lbl{211}
			\exist c \in [0, 1] : \norm{P(1) - P(0)} \le \norm{\mc{D}P(c)} \cdot (1 - 0) = \norm{\mc{D}P(c)} \underset{\eref{28}}= \norm{ \bigg( \mc{D}F(X + cH) - A \bigg)H} \le \\
			\le \norm{\mc{D}F\underbrace{(X + cH)}_{\in U} - A}\norm{H} \underset{\eref{25}}< 2\lambda \norm{H} \underset{\eref{29}}\le \half\norm{AH}
		\end{multline}
		\begin{equ}{213}
			\norm{F(X + H) - F(X) - AH} \undereq{\eref{210}} \norm{P(1) - P(0)} \underset{\eref{211}}< \half\norm{AH}
		\end{equ}
		\begin{multline*}
			\norm{F(X + H) - F(X)} = \norm{AH + \bigg( F(X + H) - F(X) - AH \bigg)} \ge \\
			\ge \norm{AH} - \norm{F(X + H) - F(X) - AH} \underset{\eref{213}}> \norm{AH} - \half\norm{AH} = \half\norm{AH} \underset{\eref{29}}\ge 2\lambda \norm{H} > 0
		\end{multline*}
	\end{enumerate}
\end{proof}

\begin{figure}[!ht]
	\begin{tikzpicture}[>=Stealth]
		\draw[->, thin, gray] (-2.3, 0) -- (2.3, 0) node[right]{$ x $};
		\draw[->, thin, gray] (0, -2.3) -- (0, 2.3) node[above]{$ y $};

		\draw (0, 0) circle [radius=2];

		\draw[->] (0, 0) -- +(45:1.5) node[right]{$ X_1 $};
		\draw[->] (0, 0) -- +(150:1.8) node[below]{$ X_2 $};

		\path[name path=X_1] (0, 0) -- +(45:1.5);
		\path[name path=circ] (0, 0) circle [radius=1];

		\draw[->, blue, thick, name intersections={of=X_1 and circ}] (0, 0) -- (intersection-1) node[anchor=north west]{$ tX_1 $};
		\draw[->, blue, thick, name intersections={of=X_1 and circ}] (intersection-1) -- +(150:0.6) node[above]{$ (1 - t)X_2 $};
	\end{tikzpicture}
	\caption{Выпуклость шара}
	\label{fig:1}
\end{figure}
