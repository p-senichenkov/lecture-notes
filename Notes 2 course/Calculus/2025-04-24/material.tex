\chapter{Мера Лебега}

Все утверждения, приведённые здесь без доказательств, легко проверяются в случае $ \R $ при помощи картинок.

Мера в нашем случае будет обозначать меру Лебега.

\begin{definition}
	Имеется некоторое непустое множество множеств $ \msc R $. Будем называть его \it{кольцом}, если
	\begin{enumerate}
		\item $ A \in \msc R, ~ B \in \msc R \implies A \cup B \in \msc R $;
		\item $ \dots \implies A \setminus B \in \msc R $.
	\end{enumerate}
\end{definition}

В частности, $ A \setminus A = \O \in \msc R $.

Вследствие того, что $ A \cap B = A \setminus (A \setminus B) $, $ A \cap B \in \msc R $.

\begin{definition}
	$ \msc R $ называется $ \sigma $-\it{кольцом}, если
	$$ \set{A_n}_{n = 1}^\infty, \quad A_n \in \msc R \implies \bigcup_{n = 1}^\infty A_n \in \msc R $$
\end{definition}

Можно проверить, что
$$ A_1 \setminus \bigg( \bigcup_{n = 2}^\infty (A_1 \setminus A_n) \bigg) = \cap_{n = 1}^\infty A_n $$
$$ \implies \bigcap_{n = 1}^\infty A_n \in \msc R $$

$ \R^{m \ge 1}, \qquad a, b \in \R, \qquad a \le b $ \\
Будем обозначать $ \braket{a, b} $, где $ \langle $ "--- это ( или [, а $ \rangle $ "--- это ) или ].

Рассмотрим $ m \ge 2, \qquad A \in \R^m, \quad B \in \R^m $. $ \braket{A, B} $ будем также называть \it{промежутком} в $ \R^m $, где $ A = (a_1, \dots, a_m), ~ B = (b_1, \dots, b_m), \quad a_j \le b_j $.

\begin{remark}
	Вообще, это параллелепипед.
\end{remark}

\begin{definition}
	\it{Мерой} промежутка будем называть
	$$ \operatorname m \big( \braket{A, B} \big) = \prod_{j = 1}^m (b_j - a_j) $$
\end{definition}

\begin{definition}
	\it{Элементарным множеством} будем называть конечное объединение промежутков:
	$$ I = \bigcup_{k = 1}^N \braket{A_k, B_k} $$
\end{definition}

\begin{notation}
	$ \msc E $ "--- множество всех элементарных множеств.
\end{notation}

\begin{statement}
	$ I \sub \msc E $. Тогда $ I $ можно представить в виде объединения промежутков, таких что
	$$ \braket{A_k, B_k} \cap \braket{A_l, B_l} = \O \quad \forall k \ne l $$
\end{statement}

\begin{definition}
	\it{Мерой} элементарного множества будем называть
	$$ \operatorname m I = \sum_{k = 1}^N \operatorname m \big( \braket{A_k, B_k} \big) $$
\end{definition}

\begin{statement}
	Определение множества элементарного множества \bt{корректно}, то есть, мера не зависит от способа разбиения.
\end{statement}

\begin{definition}
	Промежуток будем называть \it{открытым}, если все символы $ \langle $ и $ \rangle $ обозначают $ ($ и $ ) $.
\end{definition}

\begin{notation}
	$ (a_k, b_k) $
\end{notation}

\begin{definition}
	Элементарное множество будем называть \it{открытым}, если $ I = \bigcup (a_k, b_k) $.
\end{definition}

Пусть имеется некоторое множество $ E \sub \R^m $. Обозначим через $ U(E) $ множество следующих открытых элементарных множеств:
$$ U(E) = \set{ \set{A_n}_{n = 1}^\infty }, \qquad A_n \text{ "--- открытое элементарное множество}, $$
таких, что $ E \sub \bigcup_{n = 1}^\infty A_n $.

\begin{remark}
	Объединение может оказаться конечным.
\end{remark}

\begin{definition}
	\it{Внешней мерой} множества $ E $ называется
	$$ \operatorname m^* E = \inf\limits_{\set{A_n}_{n = 1}^\infty \in U(E)} \sum_{n = 1}^\infty \operatorname m A_n \quad \le +\infty $$
	Если ряд расходится, приписываем внешней мере значение $ \infty $.
\end{definition}

Понятно, что $ \operatorname m^* $ определена для любого множества. Также очевидно, что $ \operatorname m^* \O = 0 $.

\begin{props}
	\item $ \operatorname m^* E \ge 0 $;
	\item $ E_1 \sub E_2 \implies \operatorname m^* E_1 \le \operatorname m^* E_2 $;
	\item $ I \in \msc E \implies \operatorname m^* I = \operatorname m I $;
	\item
	\begin{equ}1
		E \sub \bigcup_{n = 1}^\infty E_n \implies \operatorname m^* E \le \sum_{n = 1}^\infty \operatorname m^* E_n.
	\end{equ}
\end{props}

\begin{eproof}
	\item Очевидно.
	\item $ U(E_2) \sub U(E_1) $.
	\item Очевидно.
	\item Будем считать, что $ \operatorname m^* E_n < \infty \quad \forall n $.

	Выберем $ \forall \veps > 0 $, $ \set{A_{n_k}}_{k = 1}^\infty, \quad A_{n_k} \in \msc E, \quad \set{A_{n_k}} \sub U(E_n) $ такие, что
	\begin{equ}2
		\sum_{k = 1}^\infty \operatorname m A_{n_k} < \operatorname m^* E_n + \frac\veps{2^n}
	\end{equ}
	Тогда
	$$ \set{A_{n_k}}_{n = 1~k = 1}^{\infty~\infty} \in U(E) $$
	$$ \implies \operatorname m^* E \le \sum_{n = 1}^\infty \left( \sum_{k = 1}^\infty \operatorname mA_{n_k} \right) $$
	(т. к. внешняя мера "--- это инфимум)

	Применим теперь \eref2:
	$$ \sum \sum \operatorname m A_{n_k} \le \sum_{n = 1}^\infty \left( \operatorname m^* E_n + \frac\veps{2^n} \right) = \sum_{n = 1}^\infty \operatorname m^* E_n + \veps $$
\end{eproof}

\begin{remind}
	$ A \sub \R^m, \quad B \sub \R^m $
	$$ A \vartriangle B = (A \setminus B) \cup (B \setminus A) $$
\end{remind}

Определим неотрицательное число
$$ \operatorname d(A, B) = \operatorname m^* (A \vartriangle B) \ge 0 $$

Понятно, что $ A \vartriangle \O = A $, поэтому $ \operatorname d(A, \O) = \operatorname m^* A $.

\begin{props}
	\item $ \operatorname d(A, B) = \operatorname d(B, A) $;
	\item $ \operatorname d(A, B) \le \operatorname d(A, C) + \operatorname d(C, B) $;
	\item $ \operatorname d(A_1 \cup A_2, ~ B_1 \cup B_2) \le \operatorname d(A_1, B_1) + \operatorname d(A_2, B_2) $;
	\item $ \operatorname d(A_1 \cap A_2, ~ B_1 \cap B_2) \le \operatorname d(A_1, B_1) + \operatorname d(A_2, B_2) $;
	\item $ \operatorname d(A_1 \setminus A_2, ~ B_1 \setminus B_2) \le \operatorname d(A_1, B_1) + \operatorname d(A_2, B_2) $.
\end{props}

\begin{proof}
	Все свойства основаны на теоретико-множественных соображениях. Например, 3 основано на включении
	$$ A \vartriangle B \sub (A \vartriangle C) \cup (C \vartriangle B) $$
	Далее нужно воспользоваться свойством \eref1 внешней меры.
\end{proof}

\begin{definition}
	Будем говорить, что множество $ A \sub \R^m $ \it{конечно-измеримо (по Лебегу)}, если
	\begin{equ}5
		\exist \set{A_n}_{n = 1}^\infty, \quad A_n \in \msc E : \quad \operatorname d(A_n, A) \underarr{n \to \infty} 0
	\end{equ}
\end{definition}

\begin{notation}
	$ \mathfrak M_F $ "--- множество всех конечно-измеримых множеств.
\end{notation}

Понятно, что $ \msc E \sub \mathfrak M_F $.

\begin{definition}
	Множество $ B \sub \R^m $ будем называть \it{измеримым (по Лебегу)}, если
	\begin{equ}6
		\exist \set{A_n}_{n = 1}^\infty, \quad A_n \in \fm M_F : \quad B = \bigcup_{n = 1}^\infty A_n
	\end{equ}
\end{definition}

Понятно, что $ \fm M_F \sub \fm M $.

\begin{remark}
	В множестве $ \R^m $ \bt{не все} подмножества измеримы: $ 2^{R^m} \neq \fm M $ (в отличие от внешней меры).
\end{remark}

Для $ B \in \fm M $ будем рассматривать $ \operatorname m^* B $.

\begin{theorem}
	Совокупность всех измеримых множеств является $ \sigma $-кольцом.

	Внешняя мера, определённая на $ \fm M $ обладает свойством \it{счётной аддитивности} ($ \sigma $-\it{аддитивности}):
	\begin{equ}7
		\set{B_n}_{n = 1}^\infty, \quad B_n \in \fm M, \quad B_n \cap B_k = \O \quad \implies \operatorname m^* \bigcup_{n = 1}^\infty B_n = \sum_{n = 1}^\infty \operatorname m^* B_n
	\end{equ}
\end{theorem}

Доказывать теорему не будем. Докажем, что $ \fm M_{\bt F} $ является кольцом, и мера на нём аддитивна.

\begin{proof}[$ \fm M_F $ "--- кольцо]
	Пусть есть $ A \in \fm M_F $ и $ B \in \fm M_F $. Тогда
	$$ \exist A_n \in \msc E : \quad \operatorname d(A_n, A) \to 0 $$
	$$ \exist B_n \in \msc E : \quad \operatorname d(B_n, B) \to 0 $$
	Тогда, по одному из свойств d,
	$$ \operatorname d(A_n \cup B_n, ~ A \cup B) \le \operatorname d(A_n, A) + \operatorname d(B_n, B) \to 0 $$
	$$ \op (A_n \setminus B_n, ~ A \setminus B) \le \op d(A_n, A) + \op d(B_n, B) \to 0 $$
	Отсюда $ A_n \cup B_n \in \msc E, \quad A_n \setminus B_n \in \msc E $.
	$$ \implies A \cup B \in \fm M_F, \quad A \setminus B \in \fm M_F $$
\end{proof}

\begin{statement}
	$ A, B \in \msc E $
	\begin{equ}8
		\implies \op m(A \cup B) + \op m(A \cap B) = \op m A + \op m B
	\end{equ}
\end{statement}

В частности, при $ A \cap B = \O $,
$$ \op m(A \cup B) = \op m A + \op m B $$

\begin{proof}[аддитивность меры]
	Пусть $ A, B \in \fm M_F, \quad A \cap B = \O $. Тогда
	$$ \exist \set{A_n}, \set{B_n} : \quad \op d(A_n, A) \to 0, \quad \op d(B_n, B) \to 0 $$
	Отдельно будет доказано, что
	\begin{statement}\label{st:9}
		Если $ \op d(C_n, C) \to 0 $, то $ \operatorname m^* C_n \to \operatorname m^* C $
	\end{statement}
	В соотношении \eref8 можно поставить внешнюю меру вместо меры:
	$$ \operatorname m^*(A_n \cup B_n) + \op m^*(A_n \cap B_n) = \op m^* A_n + \op m^* B_n $$
	Из \autoref{st:9}, $ \op m^*(A_n \cup B_n) \to \op m^*(A \cup B) $.
	$$ \op m^*(A_n \cap B_n) \to \op m^*(A \cap B) = 0 $$
	$$ \op m^* A_n \to \op m^* A, \qquad \op m^* B_n \to \op m^* B $$
	Это всё влечёт, что
	$$ \op m^*(A \cup B) = \op m^* A + \op m^* B $$
\end{proof}

\begin{statement}
	$ |\op m^* A - \op m^* B| \le \op d(A, B) $
\end{statement}

\begin{proof}
	Пусть $ \op m^* A < \op m^* B $. Тогда
	$$ \op m^* B = d(B, \O) \le \op d(B, A) + \op d(A, \O) = \op d(B, A) + \op m^* A $$
\end{proof}

\vspace{0.5em}

\begin{proof}[\autoref{st:9}]
	$ |\op m^* C_n - \op m^* C| \le \op d(C_n, C) \to 0 $
\end{proof}

Теперь для $ E \in \fm M $ будем полагать $ \op m E = \op m^* E $. Это \it{мера Лебега}.
