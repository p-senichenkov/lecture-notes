\part{Теория меры}

Все утверждения, приведённые здесь без доказательств, легко проверяются в случае $ \R $ при помощи картинок.

Мера в нашем случае всегда будет обозначать меру Лебега.

\section{Кольцо и \tpst{$ \sigma $}{сигма}-кольцо множеств; промежутки в \tpst{$ \R^m $}{R\textasciicircum m} и их мера; элементарные множества и их меры}

\begin{definition}
	Имеется некоторое непустое множество множеств $ \msc R $. Будем называть его \emph{кольцом}, если
	\begin{enumerate}
		\item $ A \in \msc R, ~ B \in \msc R \implies A \cup B \in \msc R $;
		\item $ \dots \implies A \setminus B \in \msc R $.
	\end{enumerate}
\end{definition}

В частности, $ A \setminus A = \O \in \msc R $.

Вследствие того, что $ A \cap B = A \setminus (A \setminus B) $, $ A \cap B \in \msc R $.

\begin{definition}
	$ \msc R $ называется $ \sigma $-\emph{кольцом}, если
	$$ \set{A_n}_{n = 1}^\infty, \quad A_n \in \msc R \implies \bigcup_{n = 1}^\infty A_n \in \msc R $$
\end{definition}

Можно проверить, что
$$ A_1 \setminus \bigg( \bigcup_{n = 2}^\infty (A_1 \setminus A_n) \bigg) = \cap_{n = 1}^\infty A_n $$
$$ \implies \bigcap_{n = 1}^\infty A_n \in \msc R $$

$ \R^{m \ge 1}, \qquad a, b \in \R, \qquad a \le b $ \\
Будем обозначать $ \braket{a, b} $, где $ \langle $ "--- это ( или [, а $ \rangle $ "--- это ) или ].

Рассмотрим $ m \ge 2, \qquad A \in \R^m, \quad B \in \R^m $. $ \braket{A, B} $ будем также называть \emph{промежутком} в $ \R^m $, где $ A = (a_1, \dots, a_m), ~ B = (b_1, \dots, b_m), \quad a_j \le b_j $.

\begin{definition}
	\emph{Мерой} промежутка будем называть
	$$ \operatorname m \big( \braket{A, B} \big) = \prod_{j = 1}^m (b_j - a_j) $$
\end{definition}

\begin{definition}
	\emph{Элементарным множеством} будем называть конечное объединение промежутков:
	$$ I = \bigcup_{k = 1}^N \braket{A_k, B_k} $$
\end{definition}

\begin{notation}
	$ \msc E $ "--- множество всех элементарных множеств.
\end{notation}

\begin{statement}
	$ I \sub \msc E $. Тогда $ I $ можно представить в виде объединения промежутков, таких что
	$$ \braket{A_k, B_k} \cap \braket{A_l, B_l} = \O \quad \forall k \ne l $$
\end{statement}

\begin{definition}
	\emph{Мерой} элементарного множества будем называть
	$$ \operatorname m I = \sum_{k = 1}^N \operatorname m \big( \braket{A_k, B_k} \big) $$
\end{definition}

\begin{statement}
	Определение множества элементарного множества \bt{корректно}, то есть, мера не зависит от способа разбиения.
\end{statement}

\begin{definition}
	Промежуток будем называть \emph{открытым}, если все символы $ \langle $ и $ \rangle $ обозначают $ ($ и $ ) $.
\end{definition}

\begin{definition}
	Элементарное множество будем называть \emph{открытым}, если $ I = \bigcup (a_k, b_k) $.
\end{definition}

Пусть имеется некоторое множество $ E \sub \R^m $. Обозначим через $ U(E) $ множество следующих открытых элементарных множеств:
$$ U(E) = \set{ \set{A_n}_{n = 1}^\infty }, \qquad A_n \text{ "--- открытое элементарное множество}, $$
таких, что $ E \sub \bigcup_{n = 1}^\infty A_n $.

\section{Внешняя мера \tpst{$ \op m^* E $}{m*E} множества \tpst{$ E $}{E}}

\begin{definition}
	\emph{Внешней мерой} множества $ E $ называется
	$$ \operatorname m^* E = \inf\limits_{\set{A_n}_{n = 1}^\infty \in U(E)} \sum_{n = 1}^\infty \operatorname m A_n \quad \le +\infty $$
	Если ряд расходится, приписываем внешней мере значение $ \infty $.
\end{definition}

Понятно, что $ \operatorname m^* $ определена для любого множества. Также очевидно, что $ \operatorname m^* \O = 0 $.

\section{Свойства внешней меры}

\begin{props}
	\item $ \operatorname m^* E \ge 0 $;
	\item $ E_1 \sub E_2 \implies \operatorname m^* E_1 \le \operatorname m^* E_2 $;
	\item $ I \in \msc E \implies \operatorname m^* I = \operatorname m I $;
	\item
	$$ E \sub \bigcup_{n = 1}^\infty E_n \implies \operatorname m^* E \le \sum_{n = 1}^\infty \operatorname m^* E_n. $$
\end{props}

\begin{eproof}
	\item Очевидно.
	\item $ U(E_2) \sub U(E_1) $.
	\item Очевидно.
	\item Будем считать, что $ \operatorname m^* E_n < \infty \quad \forall n $.

	Выберем $ \forall \veps > 0 $, $ \set{A_{n_k}}_{k = 1}^\infty, \quad A_{n_k} \in \msc E, \quad \set{A_{n_k}} \sub U(E_n) $ такие, что
	\begin{equ}{measure_def:2}
		\sum_{k = 1}^\infty \operatorname m A_{n_k} < \operatorname m^* E_n + \frac\veps{2^n}
	\end{equ}
	Тогда
	$$ \set{A_{n_k}}_{n = 1~k = 1}^{\infty~\infty} \in U(E) $$
	$$ \implies \operatorname m^* E \le \sum_{n = 1}^\infty \left( \sum_{k = 1}^\infty \operatorname mA_{n_k} \right) $$
	(\as внешняя мера "--- это инфимум)

	Применим теперь \eref{measure_def:2}:
	$$ \sum \sum \operatorname m A_{n_k} \le \sum_{n = 1}^\infty \left( \operatorname m^* E_n + \frac\veps{2^n} \right) = \sum_{n = 1}^\infty \operatorname m^* E_n + \veps $$
\end{eproof}

\section{Функция \tpst{$ \op d(A, B) $}{d(A, B)} и её свойства}

Определим неотрицательное число
$$ \operatorname d(A, B) = \operatorname m^* (A \vartriangle B) \ge 0 $$

Понятно, что $ A \vartriangle \O = A $, поэтому $ \operatorname d(A, \O) = \operatorname m^* A $.

\begin{props}
	\item $ \operatorname d(A, B) = \operatorname d(B, A) $;
	\item $ \operatorname d(A, B) \le \operatorname d(A, C) + \operatorname d(C, B) $;
	\item $ \operatorname d(A_1 \cup A_2, ~ B_1 \cup B_2) \le \operatorname d(A_1, B_1) + \operatorname d(A_2, B_2) $;
	\item $ \operatorname d(A_1 \cap A_2, ~ B_1 \cap B_2) \le \operatorname d(A_1, B_1) + \operatorname d(A_2, B_2) $;
	\item $ \operatorname d(A_1 \setminus A_2, ~ B_1 \setminus B_2) \le \operatorname d(A_1, B_1) + \operatorname d(A_2, B_2) $.
\end{props}

\begin{proof}
	Все свойства основаны на теоретико-множественных соображениях. Например, 3 основано на включении
	$$ A \vartriangle B \sub (A \vartriangle C) \cup (C \vartriangle B) $$
	Далее нужно воспользоваться свойством \eref1 внешней меры.
\end{proof}

\begin{statement}
	$ |\op m^* A - \op m^* B| \le \op d(A, B) $
\end{statement}

\begin{proof}
	Пусть $ \op m^* A < \op m^* B $. Тогда
	$$ \op m^* B = d(B, \O) \le \op d(B, A) + \op d(A, \O) = \op d(B, A) + \op m^* A $$
\end{proof}

\section{Определение \tpst{$ \mathfrak M $}{M} и \tpst{$ \mathfrak M_F $}{MF}}

\begin{definition}
	Будем говорить, что множество $ A \sub \R^m $ \emph{конечно-измеримо (по Лебегу)}, если
	$$ \exist \set{A_n}_{n = 1}^\infty, \quad A_n \in \msc E : \quad \operatorname d(A_n, A) \underarr{n \to \infty} 0 $$
\end{definition}

\begin{notation}
	$ \mathfrak M_F $ "--- множество всех конечно-измеримых множеств.
\end{notation}

Понятно, что $ \msc E \sub \mathfrak M_F $.

\begin{definition}
	Множество $ B \sub \R^m $ будем называть \emph{измеримым (по Лебегу)}, если
	$$ \exist \set{A_n}_{n = 1}^\infty, \quad A_n \in \fm M_F : \quad B = \bigcup_{n = 1}^\infty A_n $$
\end{definition}

Понятно, что $ \fm M_F \sub \fm M $.

\begin{remark}
	В множестве $ \R^m $ \bt{не все} подмножества измеримы: $ 2^{R^m} \neq \fm M $ (в отличие от внешней меры).
\end{remark}

Для $ B \in \fm M $ будем рассматривать $ \operatorname m^* B $.

\section(M --- сигма-кольцо){$ \mathfrak M $ "--- $ \sigma $-кольцо}

\begin{theorem}
	Совокупность всех измеримых множеств является $ \sigma $-кольцом.

	Внешняя мера, определённая на $ \fm M $ обладает свойством \emph{счётной аддитивности} ($ \sigma $-\emph{аддитивности}):
	$$ \set{B_n}_{n = 1}^\infty, \quad B_n \in \fm M, \quad B_n \cap B_k = \O \quad \implies \operatorname m^* \bigcup_{n = 1}^\infty B_n = \sum_{n = 1}^\infty \operatorname m^* B_n $$
\end{theorem}

Доказывать теорему не будем. Докажем, что $ \fm M_{\bt F} $ является кольцом, и мера на нём аддитивна.

\begin{proof}[$ \fm M_F $ "--- кольцо]
	Пусть есть $ A \in \fm M_F $ и $ B \in \fm M_F $. Тогда
	$$ \exist A_n \in \msc E : \quad \operatorname d(A_n, A) \to 0 $$
	$$ \exist B_n \in \msc E : \quad \operatorname d(B_n, B) \to 0 $$
	Тогда, по одному из свойств d,
	$$ \operatorname d(A_n \cup B_n, ~ A \cup B) \le \operatorname d(A_n, A) + \operatorname d(B_n, B) \to 0 $$
	$$ \op (A_n \setminus B_n, ~ A \setminus B) \le \op d(A_n, A) + \op d(B_n, B) \to 0 $$
	Отсюда $ A_n \cup B_n \in \msc E, \quad A_n \setminus B_n \in \msc E $.
	$$ \implies A \cup B \in \fm M_F, \quad A \setminus B \in \fm M_F $$
\end{proof}

\section{\tpst{$ \op m^* $}{m*} счётно-аддитивна на \tpst{$ \mathfrak M $}{M}}

\begin{theorem}
	Совокупность всех измеримых множеств является $ \sigma $-кольцом.

	Внешняя мера, определённая на $ \fm M $ обладает свойством \emph{счётной аддитивности} ($ \sigma $-\emph{аддитивности}):
	$$ \set{B_n}_{n = 1}^\infty, \quad B_n \in \fm M, \quad B_n \cap B_k = \O \quad \implies \operatorname m^* \bigcup_{n = 1}^\infty B_n = \sum_{n = 1}^\infty \operatorname m^* B_n $$
\end{theorem}

Доказывать теорему не будем. Докажем, что $ \fm M_{\bt F} $ является кольцом, и мера на нём аддитивна.

\begin{statement}
	$ A, B \in \msc E $
	\begin{equ}{caratheodory:8}
		\implies \op m(A \cup B) + \op m(A \cap B) = \op m A + \op m B
	\end{equ}
\end{statement}

В частности, при $ A \cap B = \O $,
$$ \op m(A \cup B) = \op m A + \op m B $$

\begin{proof}[аддитивность меры]
	Пусть $ A, B \in \fm M_F, \quad A \cap B = \O $. Тогда
	$$ \exist \set{A_n}, \set{B_n} : \quad \op d(A_n, A) \to 0, \quad \op d(B_n, B) \to 0 $$
	Отдельно будет доказано, что
	\begin{statement}\label{st:caratheodory:9}
		Если $ \op d(C_n, C) \to 0 $, то $ \operatorname m^* C_n \to \operatorname m^* C $
	\end{statement}
	В соотношении \eref{caratheodory:8} можно поставить внешнюю меру вместо меры:
	$$ \operatorname m^*(A_n \cup B_n) + \op m^*(A_n \cap B_n) = \op m^* A_n + \op m^* B_n $$
	Из \autoref{st:9}, $ \op m^*(A_n \cup B_n) \to \op m^*(A \cup B) $.
	$$ \op m^*(A_n \cap B_n) \to \op m^*(A \cap B) = 0 $$
	$$ \op m^* A_n \to \op m^* A, \qquad \op m^* B_n \to \op m^* B $$
	Это всё влечёт, что
	$$ \op m^*(A \cup B) = \op m^* A + \op m^* B $$
\end{proof}

\vspace{0.5em}

\begin{proof}[\autoref{st:caratheodory:9}]
	$ |\op m^* C_n - \op m^* C| \le \op d(C_n, C) \to 0 $
\end{proof}

Теперь для $ E \in \fm M $ будем полагать $ \op m E = \op m^* E $. Это \emph{мера Лебега}.

\section{Простые функции; аппроксимация простыми функциями}

\begin{definition}
	$ E \sub \R^m, \quad E \ne \O $.
	\emph{Характеристической функцией} множества $ E $ называется функция $ K_E(x) : \R^m \to \R $,
	$$ K_e(x) =
	\begin{cases}
		1, \quad x \in E, \\
		0, \quad x \not\in E
	\end{cases} $$
\end{definition}

\begin{definition}
	\emph{Простой функцией} $ f_0 : E \to \R $ будем называть функцию, множество значений которой конечно.
\end{definition}

Если $ c_1, \dots, c_n $ "--- все различные значения функции $ f_0, \quad E_j = \set{x \in E \mid f_0(x) = c_j} $, то $ E_j \cap E_k = \O, $ \\
$ \bigcup E_j = E $.
Полагая $ \chi_{E_j}(x) = K_{E_j}(x) \big|_E $, имеем соотношение
\begin{equ}{lebeg_int:1}
	f_0(x) = \sum_{j = 1}^n c_j \chi_{E_j}(x)
\end{equ}

\begin{theorem}
	$ f : E \to \R $

	\begin{enumerate}
		\item Тогда существует последовательность простых функций, определённых на $ E $ таких, что
			$$ \forall x \in \quad f_n(x) \underarr{n \to \infty} f(x) $$

		\item Если множество $ E $ измеримо по Лебегу и функция $ f $ измерима, то все функции $ f_n $ можно выбрать измеримыми.

		\item Если $ f(x) \ge 0, \quad x \in E $, то можно выбрать функции $ f_n(x) $, которые при $ \forall x $ монотонно возрастают по $ n $.
	\end{enumerate}
\end{theorem}

\begin{eproof}
	\item Пусть $ f(x) \ge 0 \quad \forall x $. \\
		Положим для $ n = 1, 2, \dots, \quad i = 1, \dots, n \cdot 2^n $
		$$ E_{n~i} = \set{x \in E \mid \frac{i - 1}{2^n} \le f(x) < \frac i{2^n}} $$
		$$ E_{n~0} = \set{x \in E \mid f(x) \ge n} $$
		Далее пусть $ \chi_{E_{n~i}}(x) = K_{E_{n~i}}(x) \big|_E, \qquad i = 0, 1, \dots, n \cdot 2^n $, и пусть
		\begin{equ}{lebeg_int:2}
			f_n(x) = \sum_{i = 1}^{n2^n} \frac{i - 1}{2^n} \chi_{E_{n~i}}(x) + n \chi_{E_{n~0}}(x)
		\end{equ}
		Тогда для $ x \in \bigcup E_{n~i} $ имеем
		\begin{equ}{lebeg_int:3}
			|f_n(x) - f(x)| \le \frac1{2^n}
		\end{equ}
		Для $ \forall x \in E $ возьмём $ N > f(x) $, тогда $ \forall n > N $ выполнено \eref{lebeg_int:3} и $ f_n(x) \to f(x) $.

	\item Если $ f $ измерима, то множества $ E_{n~i} $ измеримы из \eref{lebeg_int:2} следует измеримость $ f_n $.

	\item Монотонность $ f_n $ также следует из \eref{lebeg_int:2}.

	\item Для произвольной функции $ f $ положим $ f = f^+ - f^- $ и \eref{lebeg_int:2} применим к $ f^+ $ и $ f^- $.
\end{eproof}

\begin{remark}
	Пусть $ E_j \sub E, $ не предполагаем условия $ E_j \cap E_k = \O, \quad E = \bigcup E_j $, числа $ c_j $ не обязательно различны,
	\begin{equ}{lebeg_int:4}
		f_1(x) = \sum_{j = 1}^n c_j \chi_{E_j}(x)
	\end{equ}

	Тогда $ f_1 $ "--- простая функция, которую можно записать в виде \eref{lebeg_int:1} с какими-то множествами $ E_l' $ и числами $ c_l' $.
\end{remark}

\section{Примеры измеримых по Лебегу множеств}

\begin{exmpls}
	\item Любое элементарное множество $ A $ измеримо.
	\item $ \R^m $ измеримо.
	\item Открытые множества измеримы.
	\item Замкнутые множества измеримы.
\end{exmpls}

\begin{eproof}
	\item По определению.

	\item $ \R^m = \bigcup_{n = 1}^\infty (a_n, b_n) $, где $ a_n = (-n, \dots, -n), ~ b_n = (n, \dots, n) $.

	\item Пусть $ \Q^m $ "--- множество всех точек с рациональными координатами в $ \R^m, \quad G \sub \R^m $ открыто, $ \quad G \ne \O $. \\
		Для любой точки $ M \in G \cap \Q^m $ обозначим через $ I(M) $ максимальный промежуток $ \bigl( a(M), b(M) \bigr) $ со следующими свойствами:
		\begin{itemize}
			\item $ a(M) = \bigl( a_1(M), \dots, a_m(M) \bigr) $;
			\item $ b(M) = \bigl( b_1(M), \dots, b_m(M) \bigr) $;
			\item если $ M = (M_1, \dots, M_m) $, то $ a_j(M) = M_j - \delta(M) $;
			\item $ b_j = M_j + \delta(M) $;
			\item $ \bigl( a(M), b(M) \bigr) \sub G $.
		\end{itemize}
		Тогда
		$$ G = \bigcup_{M \in G \cap \Q^m} \bigl( a(M), b(M) \bigr) $$

	\item Если $ F \sub \R^m $ замкнуто, то $ F = \R^m \setminus \bigl( \R^m \setminus F \bigr) $, множество $ \R^m \setminus F $ открыто.
\end{eproof}

\section{Измеримые функции; теорема о множествах Лебега}

\begin{definition}
	Пусть $ E \sub \R^m $ измеримо, $ E \ne \O, \quad f : E \to \R \cup \set{-\infty, +\infty}, \quad a \in \R $.

	\emph{Множествами Лебега} будем называть множества
	$$ E_{<a}(f) = \set{M \in E \mid f(M) < a}, \quad E_{\le a}(f) = \set{M \in E \mid f(M) \le a} $$
	$$ E_{>a}(f) = \set{M \in E \mid f(M) > a}, \quad E_{\ge a}(f) = \set{M \in E \mid f(M) \ge a} $$
\end{definition}

\begin{definition}
	Будем говорить, что функция $ f : E \to \hat\R, \quad E \sub \mathfrak M $ \emph{измерима по Лебегу}, если $ \forall a \in \R $ имеем
	$$ E_{<a}(f), E_{\le a}(f), E_{>a}(f), E_{\ge a} \in \mathfrak M $$
\end{definition}

\begin{theorem}
	Для того, чтобы при $ \forall a \in \R $ были измеримы множества Лебега, \textbf{необходимо и достаточно}, чтобы при $ \forall a \in \R $ было измеримо какое-то из них.
\end{theorem}

\begin{proof}
	Имеем следующие соотношения:
	$$ E_{\ge a} = \bigcap_{n = 1}^\infty \set{x \in E \mid f(x) > a - \frac1n}, \qquad E_{<a} = E \setminus E_{\ge a} $$
	$$ E_{\le a} = \bigcap_{x \in E \mid f(x) < a + \frac1n}, \qquad E_{>a} = E \setminus E_{\le a} $$
	Поскольку $ \mathfrak M $ "--- кольцо, $ E \in \mathfrak M $,
	$$ E_{>a} \in \mathfrak M \quad \forall a \implies E_{\ge a} \in \mathfrak M \quad \forall a $$
	$$ E_{\ge a} \in \mathfrak M \quad \forall a \implies E_{<a} \in \mathfrak M \quad \forall a $$
	$$ E_{<a} \in \mathfrak M \quad \forall a \implies E_{\le a} \in \mathfrak M \quad \forall a $$
	$$ E_{\le a} \in \mathfrak M \quad \forall a \implies E_{>a} \in \mathfrak M \quad \forall a $$
\end{proof}

\section{Измеримость \tpst{$ |f| $}{|f|}}

\begin{property}
	$ f $ измерима $ \implies |f| $ измерима.
\end{property}

\begin{proof}
	$ E_{<a}(|f|) = E_{<a}(f) \cap E_{>-a}(f) $.
\end{proof}

\section{Измеримость \tpst{$ \varlimsup\limits_{n \to \infty} f_n(x), ~ \sup\limits_n f_n(x) $}{верхнего предела fn(x), sup fn(x)}}

\begin{property}\label{prop:meas_prop:2}
	Пусть $ f_n $ измерима на $ E $. Тогда
	$$ g_+(x) \coloneq \sup\limits_{n \ge 1} f_n(x), \quad h_+(x) \coloneq \varlimsup\limits_{n \to \infty} f_n(x) $$
	измеримы.
\end{property}

\begin{proof}
	Имеем соотношение
	$$ E_{>a}(g_+) = \bigcup_{n = 1}^\infty E_{>a}(f_n) $$
	Положим $ g_m(s_{n \ge m} f_n(x) $, тогда $ g_m $ измеримы и $ h(x) = \inf\limits_m g_m(x) $.
\end{proof}

\section{Измеримость \tpst{$ \varliminf\limits_{n \to \infty} f_n(x), ~ \inf\limits_n f_n(x) $}{нижнего предела fn(x), inf fn(x)}}

\begin{property}
	Пусть $ f_n $ измерима на $ E $. Тогда
	$$ g_-(x) \coloneq \inf\limits_{n \ge 1} f_n(x), \quad h_-(x) \coloneq \varliminf_{n \to \infty} f_n(x) $$
	измеримы.
\end{property}

\begin{proof}
	Имеем соотношение
	$$ E_{<a}(g_-) = \bigcup_{n = 1}^\infty E_{<a}(f_n) $$
	Положим $ g_m(x) = \inf\limits_{n \ge m} f_n(x) $, тогда $ g_m $ измеримы и $ h(x) = \sup\limits_m g_m(x) $.
\end{proof}

\section{Измеримость \tpst{$ f^+, ~ f^- $}{f+, f-}}

\begin{property}
	Положим $ f^+(x) = \max \set{f(x), 0}, \quad f^-(x) = -\min \set{f(x), 0} $.

	Тогда $ f^+, f^- $ измеримы.
\end{property}

\begin{proof}
	Пусть $ f, g $ измеримы. \\
	Положим $ f_{2n - 1}(x) = f(x), \quad f_{2n}(x) = g(x) $, тогда
	$$ \sup\limits_{n \ge 1} f_n(x) = \max \set{f(x), g(x)}, \quad \int\limits_{n \ge 1} f_n(x) = \min \set{f(x), g(x)} $$
	То есть, $ \max\set{f(x), g(x)} $ и $ \min \set{(x), g(x)} $ измеримы.

	Если $ c \in \R, \quad f $ измерима, то и $ cf $ измерима:
	\begin{itemize}
		\item если $ c > 0 $, то $ E_{>ca}(cf) = E_{>a}(f) $;
		\item если $ c < 0 $, то $ E_{>ca}(cf) = E_{<a}(f) $;
		\item если $ c = 0 $, то $ 0 \cdot f \equiv f $.
	\end{itemize}
\end{proof}

\section{Измеримость \tpst{$ \lim f_n(x) $}{lim fn(x)}}

\begin{property}
	Пусть $ f_n(x) $ измеримы $ \forall n $ и $ \forall x \in E \quad \exists f(x) = \lim\limits_{n \to \infty} f_n(x) \in \ol \R $.

	Тогда $ f $ измерима.
\end{property}

\begin{proof}
	Имеем $ \lim\limits_{n \to \infty} f_n(x) = \varlimsup_{n \to \infty} f_n(x) $, утверждение следует из \ref{en:meas_prop:2}.
\end{proof}

\section{Измеримость \tpst{$ f_n + g_n, ~ f_ng_n $}{fn + gn, fn gn}}

\begin{property}
	Пусть $ F(u, v) : \R^2 \to \R, \quad F \in \mathcal C(\R^2), \quad f(x), g(x) $ измеримы.

	Тогда $ h(x) \coloneq F \bigl( f(x), g(x) \bigr) $ измерима.
\end{property}

\begin{proof}

	Пусть $ G_a = \set{(u, v) \in \R^2 \mid F(u, v) > a} $. \\
	Тогда $ F \in \mathcal C(\R^2) \implies G_a $ открыто в $ \R^2 $.

	Пусть $ G_a \ne \O $. Тогда можно представить $ G_a = \bigcup_{n = 1}^\infty (a_n, b_n) $, где $ a_n = (u_n^-, v_n^-), ~ b_n = (u_n^+, v_n^+) $. \\
	Теперь
	$$ \set{x \in E \mid u_n^- < f(x) < u_n^+} = E_{>u_n^-}(f) \cap E_{<u_n^+}(f), \quad \set{x \in E \mid v_n^- < g(x) < v_n^+} = E_{>v_n}(g) \cap E_{<v_n^+}(g), $$
	поэтому
	\begin{multline*}
		\set{x \in E \mid \bigl( f(x), g(x) \bigr) \in (a_n, b_n)} = \set{x \in E \mid u_n^- < f(x) < u_n^+} \cap \set{x \in E \mid v_n^- < g(x) < v_n^+} = \\
		= E_{>u_n^-}(f) \cap E_{<u_n^+}(f) \cap E_{>v_n^-}(g) \cap E_{<v_n^+}(g)
	\end{multline*}
	$$ E_{>a}(h) = \set{x \in E \mid \bigl( f(x), g(x) \bigr) \in G_a} = \bigcup_{n = 1}^\infty \set{x \in E \mid \bigl( f(x), g(x) \bigr) \in (a_n, b_n)} $$
	Это доказывает свойство.
\end{proof}

В частности, $ F_1(u, v) = u + v \in \mathcal(\R^2) $ и $ F_2(u, v) = uv \in \mathcal(\R^2) $.

\section{Определение \tpst{$ I_E(f) $}{IE(f)} и его свойства}

\begin{definition}
	Пусть $ E, E_j $ измеримы, $ \quad E = \bigcup E_j, \quad c_{j_0} = 0 $, если $ \op m E_{j_0} = +\infty $.
	Положим
	\begin{equ}{lebeg_int:7}
		I_E \Bigl( \sum_{j = 1}^n c_j \chi_{E_j} \Bigr) \coloneq \sum_{j = 1}^n c_j \op m E_j
	\end{equ}
	В этой формуле считаем, что $ 0 \cdot +\infty = 0 $.
\end{definition}

\begin{properties}
	$ f_1 $ "--- простая функция, записанная в виде \eref{lebeg_int:4}.
	\begin{enumerate}
		\item $ a \le f_1(x) \le b, \quad \op m E < +\infty $
			$$ \implies a \op m E \le I_E(f_1) \le b \op m E $$

		\item Если $ f_1(x) \le g_1(x), \quad x \in E $, то $ I_E(f_1) \le I_E(g_1) $.

		\item Если $ c \in \R $, то $ I_E(cf_1) = cI_E(f_1) $.

		\item \label{en:lebeg_int:4} Если $ \op m E = 0 $, то $ I_E(f_1) = 0 $.

		\item \label{en:lebeg_int:6} $ F_1 \cap F_2 = \O, \quad F_1 \cup F_2 = E $
			\begin{equ}{lebeg_int:6}
				I_{F_1}(f_1) + I_{F_2}(f_1) = I_E(f_1)
			\end{equ}
	\end{enumerate}
\end{properties}

\begin{proof}[\ref{en:lebeg_int:6}]
	Пусть $ f_1(x) = \sum c_j \chi_{E_j} $, пусть $ E_j' = E_j \cap F_1, \quad E_j'' = F_j \cap F_2 $. \\
	Тогда $ \op m E_j' + \op m E_j'' = \op m E_j $,
	$$ I_{F_1}(f_1) = \sum c_j \op m E_j', \qquad I_{F_2}(f_1) = \sum c_j \op m E_j'', \qquad I_E(f) = \sum c_j \op m E_j $$
	Отсюда следует свойство \eref{lebeg_int:6}.
\end{proof}

\section{Определение интеграла Лебега для \tpst{$ f(x) \ge 0 $}{f(x) >= 0}}

\begin{definition}
	Пусть $ E \sub \R^m $ "--- измеримо, $ \quad E \ne \O, \quad f : E \to \R \cup \set{+\infty}, \quad f(x) \ge 0 \quad \forall x \in E, \quad f $ измерима.

	Через $ \mathtt B(f) $ обозначим множество всех простых функций $ f_0 : E \to \R $, удовлетворяющих условиям:
	\begin{itemize}
		\item $ f_0(x) \ge 0 $;
		\item $ f_0 $ измерима;
		\item $ f_0(x) \le f(x) \quad \forall x \in E $.
	\end{itemize}

	\emph{Интегралом Лебега} назовём следующую величину
	\begin{equ}{lebeg_int:8}
		\int\limits_E f \di \op m \coloneq \sup \set{I_E(f_0) \mid f_0 \in \mathtt B(f)}
	\end{equ}
\end{definition}

\begin{definition}
	Если $ \int\limits_E f \di \op m < +\infty $, то функцию $ f $ называют \emph{суммируемой} на множестве $ E $.
\end{definition}

\begin{notation}
	$ f \in \msc L(E) $
\end{notation}

\section{Определение интеграла Лебега для функции любого знака}

\begin{definition}
	Если $ f : E \to \R \cup \set{-\infty, +\infty} $ может принимать значения разных знаков, считаем $ f = f^+ - f^- $ и называем $ f $ \emph{суммируемой}, если $ f^+ \in \msc L(E) $ и $ f^- \in \msc L(E) $.
	Тогда полагаем
	\begin{equ}{lebeg_int:9}
		\int\limits_E f \di \op m \coloneq \int\limits_E f^+ \di \op m - \int\limits_E f^- \di \op m
	\end{equ}
\end{definition}

\section{Счётная аддитивность \tpst{функции $ \int\limits_A f \di \op m $}{интеграла Лебега}: характеристическая функция, простая функция \tpst{$ f $}{f}}

\begin{theorem}
	$ f \in \msc L(E), \quad A \in \mathfrak M, \quad A \sub E, \quad \phi(A) = \int\limits_A f \di \op m $.

	Тогда $ \phi $ счётно-аддитивна на $ \mathfrak M $, суженном на $ E $.
\end{theorem}

\begin{proof}
	Требуется установить равенство
	\begin{equ}{lebeg_int:10}
		\phi(A) = \sum_{n = 1}^\infty \phi(A_n), \quad \text{ если } A_n \sub E, \quad A_n \cap A_k = \O
	\end{equ}

	\begin{enumerate}
		\item Пусть $ f(x) = \chi_F(x), \quad F \sub E $, тогда
			$$ \phi(A) = \int\limits \chi_F(x) \di \op m = I_E(\chi_{F \cap A}) = \op m (F \cap A) $$
			$$ \phi(A_n) = I_E(\chi_{F \cap A_n}) = \op m (F \cap A_n) $$
			В силу счётной аддитивности меры Лебега имеем $ \op m(F \cap A) = \sum \op m(F \cap A_n) $, откуда следует \eref{lebeg_int:10}.

		\item \label{en:lebeg_int:2} $ f(x) = \sum_{j = 1}^N c_j \chi_{F_j}(x) $.
			$$ \phi(A) = \int\limits_A \sum_{j = 1}^N c_j \chi_{F_j} \di \op m = I_A \bigl( \sum c_j \chi_{F_j} \bigr) = \sum c_j \op m(F_j \cap A) $$
			$$ \phi(A_n) = I_{A_n} \bigl( \sum c_j \chi_{F_j} \bigr) = \sum c_j \op m (F_j \cap A_n) $$
			Отсюда следует \eref{lebeg_int:10}.
	\end{enumerate}
\end{proof}

\section{Счётная аддитивность \tpst{$ \int\limits_A f \di \op m $}{интеграла Лебега}: \tpst{$ f(x) \ge 0 $}{f(x) >= 0}}

\begin{theorem}
	$ f \in \msc L(E), \quad A \in \mathfrak M, \quad A \sub E, \quad \phi(A) = \int\limits_A f \di \op m $.

	Тогда $ \phi $ счётно-аддитивна на $ \mathfrak M $, суженном на $ E $.
\end{theorem}

\begin{proof}[$ 0 \le f(x) \le +\infty, \quad f $ измерима]
	Пусть $ f_0 \in \mathtt B(f) $. Тогда, по пункту \ref{en:lebeg_int:2},
	$$ I_A(f_0) = \int\limits_A f_0 \di \op m = \sum_{n = 1}^\infty \int\limits_{A_n} f_0 \di \op m \le \sum \phi(A_n) $$
	$$ \implies \phi(A) = \sup \set{I_A(f_0) \mid f_0 \in \mathtt B(f)} \le \sum_{n = 1}^\infty \phi(A_n) $$
	Поскольку $ f \in \msc L(E) $, то $ \phi(A) < +\infty, \quad \phi(A_n) < +\infty $. \\
	Возьмём $ \forall N $ и зафиксируем $ \eps > 0 $. \\
	Выберем $ f_1, \dots, f_N $ "--- простые функции, $ f_j \in \mathtt B(f) $, удовлетворяющие условию
	$$ I_{A_j}(f_j) > \int\limits_{A_j} d \di \op m - \frac\eps N, \quad j = 1, \dots, N $$
	Определим функцию $ f_0 : E \to \R $:
	$$ f_0(x) =
	\begin{cases}
		f_j(x), \quad j \ge 2, \quad x \in A, \\
		f_1(x), \quad x \in E \setminus \bigcup_{n = 2}^N A_n
	\end{cases} $$
	Тогда $ f_0 \in \mathtt B(f), \quad \bigcup_{n = 1}^N A_n \sub A $ и по пункту \ref{en:lebeg_int:2}
	$$ \phi(A) \ge \phi \Bigl( \bigcup_{n = 1}^N A_n \Bigr) \ge I_{\bigcup A_n} (f_0) = \sum_{n = 1}^N I_{A_n} (f_0) = \sum_{n = 1}^N I_{A_n}(f_n) > \sum \Bigl( \phi(A_n) - \frac\eps N \Bigr) = \sum \phi(A_n) - \eps $$
	В силу произвольности $ N $ и $ \eps > 0 $
	$$ \implies \phi(A) \ge \sum_{n = 1}^\infty \phi(A_n) $$
\end{proof}

\section{Счётная аддитивность \tpst{$ \int\limits_A f \di \op m $}{интеграла Лебега}: \tpst{$ f(x) \in \msc L(E) $}{f(x) суммируема}}

Из \eref{lebeg_int:9} следует, что достаточно установить \eref{lebeg_int:10} для $ f(x) \ge 0, \quad x \in E $.

\section{Следствие для \tpst{$ f \sim g $}{f \textasciitilde g}}

Поскольку из свойства \ref{en:lebeg_int:4} следует, что $ \int\limits_E f \di \op m = 0 $, если $ \op m E = 0 $, то из теоремы получаем важное следствие.

\begin{implication}
	Пусть $ f_1, f_2 \in \msc L(E), \quad \op m \set{x \in E \mid f_1(x) \ne f_2(x)} = 0 $. Тогда
	$$ \int\limits_E f_1 \di \op m = \int\limits_E f_2 \di \op m $$
\end{implication}

\begin{proof}
	Пусть $ F = \set{x \in E \mid f_1(x) \ne f_2(x)} $, тогда
	$$ \int\limits_E f_1 \di \op m = \int\limits_{E \setminus F} f_1 \di \op m + \int\limits_F f_1 \di \op m = \int\limits_{E \setminus F} f_1 \di \op m = \int\limits_{E \setminus F} f_2 \di \op m = \int\limits_{E \setminus F} f_2 \di \op m + \int\limits_F f_2 \di \op m = \int\limits_E f_2 \di \op m $$
\end{proof}

\section(модуль интеграла не превосходит интеграла модуля){$ \bigl| \int\limits_E f \di \op m \bigr| \le \int\limits_E |f| \di \op m $}

\begin{theorem}
	Пусть $ f \in \msc L(E) $, тогда $ |f| \in \msc L(E) $ и
	$$ \Bigl| \int\limits_E f \di \op m \Bigr| \le \int\limits_E |f| \di \op m $$
\end{theorem}

\begin{proof}
	Пусть $ E_+ = \set{x \in E \mid f(x) \ge 0}, \quad E_- = \set{x \in E \mid f(x) < 0} $. \\
	Тогда $ \int\limits_E f \di \op m = \int\limits_{E_+} + \int\limits_{E_-} = \int\limits_E f^+ \di \op m - \int\limits_E f^- \di \op m $,
	$$ \int\limits_E |f| \di \op m = \int \limits_{E_+} + \int \limits_{E_-} = \int\limits_{E_+} f^+ \di \op m + \int \limits_{E_-} f^- \di \op m = \int \limits_E f^+ \di \op m + \int \limits_E f^- \di \op m $$
\end{proof}

\section{Дальнейшие свойства интеграла Лебега}

\begin{props}
	\item Пусть $ \exist c < \infty $ такая, что $ |f(x)| \le c, \quad x \in E, \quad f $ измерима на $ E $ и $ \op m E < +\infty $.

	Тогда $ f \in \msc L(E) $.

	\item Если $ f $ измерима, $ \op m E < \infty, \quad a \le f(x) \le b, \quad x \in E $, то
	$$ a \op m E \le \int_E f \di m \le b \op m E $$

	\item Если $ f, g \in \msc L(E) $ и $ f(x) \le g(x), \quad x \in E $, то
	$$ \int_E f \di \op m \le \int g \di \op m $$

	\item $ f \in \op L(E), \quad c \in \R $
	$$ \implies
	\begin{cases}
		cf \in \msc L(E), \\
		\int\limits_E cf \di \op m = c \int\limits_E f \di \op m
	\end{cases} $$

	\item Если $ \op m E = 0, \quad f $ измерима, то
	$$ \lim\limits_E \di \op m = 0 $$

	\item Если $ f \in \msc L(E), \quad F \sub E, \quad F $ измеримо, то $ f \in \msc L(F) $.

	В частности, если $ E = E_0 \cup S, \quad E_0 \cap S = 0, \quad \op m S = 0 $, то
	$$ \int\limits_E f \di \op m = \int\limits_{E_0} f \di \op m + \int\limits_S f \di \op m = \int\limits_{E_0} f \di \op m, $$
	поскольку $ \int\limits_S f \di \op m = 0 $.

	Отсюда следует важное свойство интеграла Лебега:

	\item Пусть $ f \sim h, \qquad f \in \msc L(E) $. Тогда
	$$ \int\limits_E f \di \op m = \int\limits_E g \di \op m $$

	\item Пусть $ f, g \in \msc L(E) $. Тогда $ f + g \in \msc L(E) $ и
	$$ \int\limits_E (f + g) \di \op m = \int\limits_E f \di \op m + \int\limits_E g \di \op m $$
\end{props}

\begin{eproof}
	\item Следует из того, что $ f \in \msc L(E) \iff |f| \in \msc l(E) $ для любой простой функции $ s : ~ 0 \le s(x) \le |f(x) $ справедливо $ s(x) \le c $, поэтому
	$$ \int\limits_E s \di \op m \le \int\limits_E c \di \op m = c \op m E, \qquad \int |f| \di \op m \le c \op m E $$

	\item Аналогично.

	\item Без доказательства.

	\item Докажем для $ f(x) \ge 0, \quad x \in E, \quad c > 0 $. Пусть $ s \in \mc A(F) $, \ie $ s $ "--- простая функция, $ \quad 0 \le s(x) \le f(x) \quad \forall x \in E $. \\
	Тогда $ cs \in \mc A(cf) $,
	$$ \int\limits_{E} cs \di \op m = \sum_{j = 1}^n ca_j \op m F_j = c \sum_{j = 1}^n a_j \op m F_j = c \int\limits_E s \di \op m, $$
	если $ s(x) = \sum a_j \xi_{F_j}(x), \quad F_j \cap F_k = \O $. \\
	Переходя к супремуму, получаем нужное свойство.

	\item Если $ F_j \sub E $, то $ 0 \le \op m F_j \le \op m E = 0 $, для любой простой функции $ s \in \mc A(|f|) $ имеем $ 0 \le s(x) \le |f(1)| = 0, \quad s(x) = 0, \quad \int\limits_E s \di \op m = 0 $, поэтому $ \int\limits_E |f| \di \op m = 0 $
	$$ 0 \le \int\limits_E f^+ \di \op m \le \int\limits_E |f| \di \op m = 0, \qquad 0 \le \int\limits_E f^- \di \op m \le \int |f| \di \op m = 0 $$
	$$ \int\limits_E \di \op m = \int\limits_E f^+ \di \op m - \int\limits_E f^- \di \op m = 0 $$

	\item Для $ \forall s \in \mc A(|f|) $ на множестве $ F $ положим
	$$ s_0(x) =
	\begin{cases}
		s(x), \quad x \in F, \\
		0, \quad x \in E \setminus F
	\end{cases} $$
	$$ \int\limits_E s_0 \di \op m = \int\limits_F s \di \op m \le \int\limits_E |f| \di \op m $$
	$$ \int \limits_F |f| \di \op m \le \int\limits_E |f| \di \op m, \qquad |f| \in \msc L(F) $$

	\item Пусть $ E_0 = \set{x \in E \mid f(x) = g(x)}, \quad S = E \setminus E_0 $. Тогда $ \op m S = 0 $,
	$$ \int\limits_E f \di \op m = \int\limits_{E_0} f \di \op m, \qquad \int\limits_E g \di \op m = \int\limits_{E_0} g \di \op m $$
\end{eproof}

\section{Интеграл Римана и интеграл Лебега}

\begin{theorem}
	Пусть функция $ f $ интегрируема по Риману на промежутке $ (a, b) $.

	Тогда она измерима по Лебегу на множестве $ E = (a, b) $, суммируема, и справедливо равенство
	$$ \int_a^b f(x) \di x = \int\limits_{[a, b]} f \di \op m $$
\end{theorem}

\begin{noproof}
\end{noproof}

\section{Теорема Фубини}

\begin{theorem}
	Имеется некое множество $ E \sub \R^{m + n}, \quad m, n \ge 1, \quad E \sub \mathfrak M_{m + n} $
	$$ \quad M \in \R^{m + n}, \quad M = (X, Y), \quad X \in \R^m, ~ Y \in \R^n $$
	Возьмём $ \forall X \in \R^m $. Определим множества
	$$ E (X, \cdot) = \set{ Y \in \R^n \mid (X, Y) \in E}, \qquad E(\cdot, Y) = \set{X \in \R^m \mid (X, Y) \in E} $$

	Тогда
	\begin{enumerate}
		\item
			\begin{itemize}
			\item Для $ m $-п.~в. $ X \qquad E(X, \cdot) \in \mathfrak M_n $.
			\item Для $ n $-п.~в. $ Y \qquad E(\cdot, Y) \in \mathfrak M_m $
		\end{itemize}
		\item Пусть $ \mu_k $ "--- мера Лебега в $ \R^k $. Тогда
			$$ \mu_{m + n} E = \int\limits_{\R^{m + n}} \mu_n E(X, \cdot) \di \mu_m(X) = \int\limits_{\R^{m + n}} \mu_m E(\cdot, Y) \di \mu_n(X) $$
		\item $ f : E \to \R $
			$$ \forall X \in \R^m \quad f_X : E(X, \cdot) \to \R : \quad f_X(Y) = f(X, Y) $$
			$$ \forall Y \in \R^n \quad f_Y : E(\cdot, Y) \to \R : \quad f_Y(X) = f(X, Y) $$

			Для $ m $-п.~в. $ X \qquad f_X $ измерима по $ Y $ на $ E(X, \cdot) $. \\
			Для $ n $-п.~в. $ Y \qquad f_Y $ измерима по $ X $ на $ E(\cdot, Y) $.
			
		\item $ f \in \msc L(E) $. Тогда
			\begin{itemize}
				\item для $ m $-п.~в. $ X \qquad f_X \in \msc L \bigl( E(X, \cdot) \bigr) $;
				\item для $ n $-п.~в. $ Y \qquad f_Y \in \msc L \bigl( E(\cdot, Y) \bigr) $;
				\item
					$$ \int\limits_E f \di \mu_{m + n} = \int\limits_{\R^m} \Bigl( \int\limits_{E(X, \cdot)} f_X \di \mu_n \Bigr) \di \mu_m(X) = \int\limits_{\R^n} \Bigr(\int\limits_{E(\cdot, Y)} f_Y \di \mu_m \Bigr) \di \mu_n(Y), $$
					или
					$$ \int\limits_E f \di \mu_{m + n} = \int\limits_{\R^m} \Bigl( f(X, Y) \di \mu_n(Y) \Bigr) \di \mu_m(X) = \int\limits_{\R^n} \Bigl( \int\limits_{E(\cdot, Y)} f(X, Y) \di \mu_m(X) \Bigr) \di \mu_n(Y) $$
			\end{itemize}
	\end{enumerate}
\end{theorem}

\section{Параметризованная поверхность в \tpst{$ \R^m $}{R\textasciicircum m}; измеримые множества на параметризованной поверхности}

\begin{definition}
	$ D \sub \R^n $ "--- открыто, связно, $ \quad m > n $.

	$ \mc C^1 $-поверхностью будем называть отображение $ F : D \to \R^n $ такое, что $ F \in \mc C^1(D) $, \ie
	$$ F = \column{f_1}{f_m}, \quad f_k \in \mc C^1(D), $$
	$ F $ "--- биекция, $ \quad \op{rank} \mc DF(X) = n \quad \forall X \in D $.
\end{definition}

\begin{definition}
	$ S = F(D), \quad E \sub S $

	Будем говорить, что $ E ~ S $-\emph{измеримо}, если $ F^{-1}(E) \sub \mathfrak M_n $
\end{definition}

\section{Определение \tpst{$ \mu_S(E) $}{меры} для параметризованной поверхности \tpst{$ S $}{S}}

\begin{definition}
	Определим $ S $-\emph{меру}:
	$$ \mu_S E \coloneq \int\limits_{F^{-1}(E)} \sqrt{\det \Bigl( \bigl( \mc DF(X) \bigr)^T \mc DF(X) \Bigr)} \di \mu_n(X) $$
\end{definition}

\begin{definition}
	$ F : S \to \R $

	Будем говорить, что $ f ~ S $-\emph{измерима}, если $ \phi(X) = f \bigl( F(X) \bigr) $ измерима на $ F^{-1}(E) $.
\end{definition}

\section{Кусочно-гладкие поверхности \tpst{$ S $}{S}; \tpst{мера}{$ \mu_S(E) $}}

\begin{definition}
	\emph{Кусочно-гладкой} поверхностью будем называть $ S = \bigcup_{k = 1}^N S_k $, \\
	где $ S_k $ "--- $ \mc C^1 $-поверхность, при этом $ S_k \cap S_l = \O $ или $ \mu_{S_k}(S_k \cap S_l) = 0 $.
\end{definition}

\begin{definition}
	$ E \sub S $

	Будем говорить, что $ E ~ S $-\emph{измеримо}, если $ E \cap S_k \quad S_k $ измеримо $ \forall k $
	$$ \mu_S E = \sum_{k = 1}^N \mu_{S_k}(E \cap S_k) $$
\end{definition}

\begin{definition}
	$ f : E \to \R $

	Будем говорить, что $ f ~ S $-\emph{измерима}, если $ f\big|_{S_k} ~ S_k $-измерима $ \forall k $.
\end{definition}

\section(Поверхностный интеграл){$ \int\limits_S f \di \mu_S $}

\begin{definition}
	$ f \in \msc L_S(E) $
	$$ \int\limits_E f \di \mu_S \coloneq \int\limits_{F^{-1}(E)} f \bigl( F(X) \bigr) \sqrt{ \det \Bigl( \bigl(  \mc DF(X) \bigr)^T \mc DF(X) \Bigr)} \di \mu_n(X) $$
\end{definition}

\begin{definition}
	$ f \in \msc L_S(E) \iff f\big|_{S_k} \in \msc L_{S_k}(E \cap S_k) $
	$$ \int\limits_E f \di \mu_S = \sum_{k = 1}^N \int\limits_{E \cap S_k} f\big|_{S_k} \di \mu_{S_k} $$
\end{definition}

\section{Параметризованная ориентированная поверхность в \tpst{$ \R^3 $}{R3}}

\begin{definition}
	$ D \sub \R^2 $ открыто, связно, $ \quad F : D \to \R^3 $ "--- $ \mc C^1 $-поверхность в $ \R^3 $ \\
	$ S = F(D), \quad F =
	\begin{bmatrix}
		f_1 \\
		f_2 \\
		f_3
	\end{bmatrix}, \quad X \in D, \quad T_1(X) =
	\begin{bmatrix}
		f_{1~x_1}'(X) \\
		f_{2~x_2}'(X) \\
		f_{3~x_3}'(X)
	\end{bmatrix}, \quad T_2(X) =
	\begin{bmatrix}
		f_{1~x_2}'(X) \\
		f_{2~x_2}'(X) \\
		f_{3~x_2}'(X)
	\end{bmatrix} $ \\
	Рассмотрим ориентацию $ \curvedir S \quad \bigl(T_1(X), T_2(X) \bigr) $.
	$$ f \in \msc L_S(E), \quad E \sub S, \quad i \ne j, \quad i, j \in \set{1, 2, 3} $$
	$$ \int\limits_{\curvedir S \cap E} f(Y) \di y_i \wedge \di y_j \define \int\limits_{F^{-1}(E)} f \bigl(F(X) \bigr)
	\begin{vmatrix}
		f_{i~x_1}'(X) & f_{i~x_2}'(X) \\
		f_{j~x_1}'(X) & f_{j~x_2}'(X)
	\end{vmatrix} \di \mu_2(X) $$
\end{definition}

\section{\tpst{Интеграл \rom2 рода}{$ \int\limits_{\curvedir S} f(M) \di x_i \wedge x_j $} для параметризованной и кусочно-гладкой ориентированной поверхности в \tpst{$ \R^3 $}{R3}}

\begin{definition}
	$ \curvedir S = \bigcup_{k = 1}^N \curvedir S_k $ "--- \emph{ориентированная кусочно-гладкая} поверхность в $ \R^3, \quad E \sub S $.
	
	$$ \int\limits_{\curvedir S \cap E} f(Y) \di y_i \wedge \di y_j \define \sum_{k = 1}^N \int f\big|_{S_k}\di y_i \wedge \di y_j $$
\end{definition}

\section{Формула Гаусса\tpst{"--~}{--}Остроградского}

\begin{theorem}
	$ V \sub \R^3 $ ограничено, связно, $ \quad \partial V = \bigcup_{k = 1}^N \ol S_k, \quad S_k \cap S_l = \O $ \\
	$ \curvedir S_k, \quad y \in S_k \quad \bigl( T_1(Y), T_2(Y) \bigr), \quad T_1(Y) \times T_2(Y) $ направлен вне $ V $.

	$ \phi \in \mc C(\ol V), \quad i \in \set{1, 2, 3}, \quad \phi_{y_i}' \in \mc C(\ol V) $
	$$ \sigma =
	\begin{cases}
		1, \quad (i, j, k) \text{ "--- чётная},
		-1, \quad \text{иначе}
	\end{cases} $$

	Тогда
	$$ \int\limits_{\curvedir \partial V} \phi(Y) \di y_i \wedge \di y_j \wedge \di y_k = \sigma \cdot \int\limits_V \phi_{y_i}'(Y) \di \mu_3(Y) $$

	В частности, при $ \phi(Y) = y_1 $,
	$$ \int\limits_{\curvedir \partial V} x_1 \di x_2 \wedge \di x_3 = \int\limits_V 1 \di \mu_3 = \mu_3 V $$
\end{theorem}

\section{Формула Грина}

\begin{theorem}
	$ D \sub \R^2 $ "--- область, $ \quad \partial D, \quad \curvedir \partial D, \quad f \in \mc C(\ol D), f_{x_1}' \in \mc C(\ol D), \quad M = (x_1, x_2) $. Тогда
	$$ \int\limits_{\curvedir \partial D} f(M) \di x_2 = \int\limits_D f_{x_1}'(M) \di \mu_2(M) $$

	$ g \in \mc C(\ol D), \quad g_{x_2}' \in \mc C(\ol D) $. Тогда
	$$ \int\limits_{\curvedir \partial D} g(M) \di x_1 = - \int\limits_D g_{x_2}'(M) \di \mu_2(M) $$
\end{theorem}

\section{Коэффициенты Фурье; формула для частичной суммы ряда Фурье}

\begin{definition}
	$ f : \R \to \R, \quad f \in \mathfrak M(\R), \quad f(x + 2\pi) = f(x) \quad \forall x, \quad f \in \msc L([0, 2\pi]) $

	Функции $ f $ сопоставляются \emph{коэффициенты Фурье} и \emph{ряд Фурье}
	$$ a_0 = \frac1{2\pi} \int\limits_{[0, 2\pi]} f \di \op m = \frac1{2\pi} \int_0^{2\pi} f(x) \di \op m $$
	$$ a_n = \frac1\pi \int_0^{2\pi} f(x) \cos n x \di \op m $$
	$$ b_n = \frac1\pi \int_0^{2\pi} f(x) \sin nx \di \op m $$
	$$ f \sim a_0 + \sum_{n = 1}^\infty (a_n \cos nx + b_n \sin nx) $$
\end{definition}

Рассмотрим частичную сумму
\begin{multline*}
	S_n(x) = a_0 + \sum_{k = 1}^\infty (a_k \cos kx + b_k \sin kx) = \\
	= \frac1{2\pi} \int_0^{2\pi} f(y) \di \op m + \sum_{k = 1}^n \frac1\pi \int_0^\pi f(y) \bigl( \cos ky \cdot \cos kx + \sin ky \cdot \sin kx \bigr) \di \op m = \\
	\frac1\pi \int_0^{2\pi} f(y) \Bigl( \frac12 + \sum_{k = 1} \bigl( \cos ky \cdot \cos kx + \sin ky \cdot \sin kx \bigr) \Bigr) \di \op m =
	\frac1\pi \int_0^{2\pi} f(y) \Bigl( \frac12 + \sum_{k = 1}^n \cos k(y - x) \Bigr) \di \op m
\end{multline*}

Сумму вычислим отдельно:
$$ D_n(t) = \frac12 + \sum_{k = 1}^n \cos kt, \qquad D_n(2\pi l) = n + \frac12 $$
Будем считать, что $ t \ne \pi n $. Тогда $ \sin \frac t2 \ne 0 $.
$$ \sin \frac t2 D_n(t) = \sin \frac t2 + \sum_{k = 1}^n \sin \frac t2 \cdot \cos kt $$
При этом,
$$ \sin \frac t2 \cdot \cos kt = \frac12 \Bigl( \sin (k + \frac12) t - \sin (k - \frac12) t \Bigr) $$

Тогда
$$ \sin \frac t2 D_n(t) = \sin \frac t2 + \sum_{k = 1}^n \frac12 \Bigl( \sin (k + \frac12) t - \sin (k - \frac12) t \Bigr) $$

$$ D_n(t) = \frac{\sin(n + \frac12)t}{2 \sin \frac t2} $$
Пусть $ y - x = t $. \\
Теперь
$$ S_n(x) = \frac1{2\pi} \int_0^{2\pi} f(y) \frac{\sin (n + \frac12)(y - x)}{\sin \frac{y - x}2} \di \op m $$

\begin{statement}
	$ \phi \in \mathfrak M(\R), \quad \phi(x) = \phi(x + 2\pi) \quad \forall x, \quad \phi(x) \in \msc L([0, 2\pi]) $
	$$ \implies \forall a \in \R \quad \phi \in \msc L([a, a + 2\pi]) $$
	$$ \int\limits_{[0, 2\pi]} \phi \di \op m = \int\limits_{[a, a + 2\pi]} \phi \di \op m $$
\end{statement}

Применим это утверждение:
\begin{multline*}
	S_n(x) = \frac1{2\pi} \int_x^{x + 2\pi} f(y) \frac{\sin (n + \frac12)(y - x)}{\sin \frac{y - x}2} \di \op m = \frac1{2\pi} \int_0^{2\pi} f(x + t) \frac{\sin(n + \frac12)t}{\sin \frac t2} \di t = \\
	= \frac1{2\pi} \int_{-\pi}^\pi f(x + t) \frac{\sin(n + \frac12)t}{\sin \frac t2} \di \op m(t)
\end{multline*}

\section{Лемма Римана\tpst{"--~}{--}Лебега}

\begin{lemma}
	$ E \sub \R, \quad E \in \mathfrak M(\R), \quad \psi $ измерима на $ E \quad \phi \in \msc L(E) $
	$$ \implies \int\limits_E \cos A x \psi(x) \di \op m \underarr{|A| \to \infty} $$
	$$ \int\limits_E \sin A x \psi(x) \di \op m \underarr{|A| \to \infty} $$
\end{lemma}

\section{Признак Дини сходимости ряда Фурье}

\begin{theorem}
	$ f(x) = f(x + 2\pi), \quad f \in \msc L([0, 2\pi]), \quad x \in (-\pi, \pi), \quad \phi(t) = \frac{f(x + t) - f(x)}t \in \msc L(-\eps, \eps), \quad 0 < \eps < \frac \pi 2 $
	$$ \implies S_n(x) \underarr{n \to \infty} f(x) $$
\end{theorem}

\begin{proof}
	\begin{equ}{dini:4}
		\frac1{2\pi} \int_{-\pi}^\pi D_n(t) \di \op m = \frac1 \pi \int_{-\pi}^\pi \Bigl( \frac12 + \sum_{k = 1}^n \cos kt \Bigr) \di \op m = \frac1 \pi \int_{-\pi}^\pi \bigl( \frac12 + \sum_{k = 1}^n \cos kt \bigr) \di \op t = 1
	\end{equ}
\end{proof}

Отсюда
$$ S_n(x) - f(x) = \frac1{2\pi} \int_{-\pi}^\pi \bigl( f(x + t) - f(x) \bigr) \frac{\sin(n + \frac12)t}{\sin \frac t2} \di \op m =
\frac1{2\pi} \int_{-\pi}^{-\eps} \dots + \frac1{2\pi} \int_{-\eps}^\eps \dots + \frac1{2\pi} \int_\eps^\pi \dots $$
Рассмотрим первый интеграл:
$$ \frac1{2\pi} \int_{-\pi}^{-\eps} \frac{f(x + t) - f(x)}{\sin \frac t2} \cdot \sin (n + \frac12)t \di \op m $$
$$ |\sin \frac t2 | \ge \frac 2\pi \cdot \frac{|t|}2 = \frac{|t|}\pi \ge \frac \eps \pi \quad \implies \quad \frac{f(x + t) - f(x)}{2 \sin \frac t2} \in \msc L(-\pi, -\eps), \quad \in \msc L(\eps, \pi) $$

Теперь
$$ \frac1{2\pi} \int_{-\pi}^{-\eps} \frac{f(x + t) - f(x)}{\sin \frac t2} \sin(n + \frac12)t \underarr{n \to \infty} 0 $$
$$ \\frac1{2\pi} \int_\eps^\pi \dots \to 0 $$
\begin{multline*}
	\frac1{2\pi} \int_{-\eps}^\eps \Bigl( f(x + t) - f(x) \Bigr) \frac{\sin(n + \frac12)t}{\sin \frac t2} \di \op m = \\
	= \frac1 \pi \int_{-\eps}^\eps \frac{f(x + t) - f(x)}{\sin \frac t2} \cdot \sin (n + \frac12) t \di \op m + \frac1{2\pi} \int_{-\eps}^\eps \Bigl( f(x + t) - f(x) \Bigr) \Bigl( \frac1{\sin \frac t2} - \frac 2t \Bigr) \sin(n + \frac12)t \di \op m
\end{multline*}

Напомним, что
$$ \frac1{\sin \tau} \cdot \frac1\tau = \frac{\tau - \sin \tau}{\tau \cdot \sin \tau} = \frac{-\frac{\tau^3}\tau + \dots}{\tau \sin \tau} \in \mc C \Bigl( [-\frac\pi 2, \frac\pi2] \Bigr) $$

$$ \implies \eref{dini:4} $$

\TODO{Здесь как-то вперемешку.}

\section{Равенство Парсеваля}

\begin{theorem}
	$ f^2 \in \msc L([0, 2\pi]) $
	$$ \implies \int f^2 \di \op m = 2 \pi a_0^2 + \pi \sum_{n = 1}^\infty (a_n^2 + b_n^2) $$
\end{theorem}

\section{Теорема о единственности ряда Фурье}

\begin{theorem}
	$ f, g $ измеримы на $ \R, \quad f(x) = f(x + 2\pi), ~ g(x) = g(x + 2\pi), \quad f, g \in \msc L(0, 2\pi) $
	$$ a_0(f) = a_0(g), \quad a_n(f) = a_n(g), \quad b_n(f) = b_n(g) \quad \forall n \ge 1 $$
	$$ \implies f \sim g $$
\end{theorem}

\section{Преобразование Фурье; пример}

Рассматриваем функции $ \R \to \Co $.

\begin{definition}
	$ u \in \msc L(\R), \quad v \in \msc L(\R), \quad f = u + iv $
	\begin{remind}
		$$ \int\limits_\R f \di \op m \bydef \int\limits_\R u \di \op m + i \int\limits_\R v \di \op m $$
	\end{remind}

	Будем говорить, что $ f $ \emph{суммируема} на всей оси, если $ u $ и $ v $ суммируемы на всей оси.
\end{definition}

\begin{definition}
	$ f \in \msc L(\R) $

	Её \emph{преобразованием Фурье} называется
	$$ \hat f(t) = \frac1{\sqrt{2\pi}} \int\limits_\R f(x) e^{-xt} \di \op m (x) $$
\end{definition}

\begin{definition}
	$ \phi \in \msc L(\R) $

	Её \emph{обратным преобразованием Фурье} называется
	$$ \vawe \phi(x) = \frac1{\sqrt{2\pi}} \int\limits_\R \phi(t) e^{ixt} \di \op m(t) $$
\end{definition}

$$ \hat{ \bigl( e^{-\frac{x^2}2} \bigr)}(t) = e^{-\frac{t^2}2} $$

\section(Преобразование Фурье от производной){$ \hat{(f')} $; $ \hat f' $}

Следующие формулы верны для широкого класса функций, который получается, если обосновать все шаги.

$$ \hat f(t) = \frac1{\sqrt{2\pi}} \int_{-\infty}^\infty f(x) e^{-itx}\di \op m(x) $$
$$ \hat f'(t) = \frac1{\sqrt{2\pi}} \int_{-\infty}^\infty f(x) \cdot (-ix) e^{-itx} \di \op m(x) = \hat{ \bigl( -ixf(x) \bigr)}(t) $$
По лемме Римана"--~Лебега $ \hat{f}(t) \underarr{|t| \to \infty} 0 $.

Рассмотрим преобразование Фурье от производной.
\begin{multline*}
	\hat{(f')}(t) = \frac1{\sqrt{2\pi}} \int_{-\infty}^\infty f'(x) e^{-itx} \di \op m(x) =
	\frac1{\sqrt{2\pi}} \int_{-\infty}^\infty f'(x) e^{-itx}\di x = \\
	= \frac1{2\pi} \lim\limits_{A \to \infty} \underbrace{\Bigl( f(x) e^{-itA} -f(-A) e^{itA} \Bigr)}_0 - \frac1{\sqrt{2\pi}} \int_{-\infty}^\infty (-it) f(x) e^{-itx} \di x =
	\frac1{\sqrt{2\pi}} it \int_{-\infty}^\infty f(x) e^{-itx} \di x = it \hat f(t)
\end{multline*}

\section{Равенство Планшереля}

\begin{theorem}
	$ f \in \msc L(\R), \quad |f|^2 \in \msc L(\R) $

	$$ \implies |\hat f|^2 \in \msc L(\R) $$
	$$ \int\limits_\R |f|^2 \di \op m = \int\limits_\R |\hat f|^2 \di \op m $$
\end{theorem}

\section(Обратное преобразование от преобразования Фурье){$ \vawe{\hat f} $}

\begin{statement}
	$ f \in \msc L(\R), \quad \hat f \in \msc L(\R), \quad |f|^2, |\hat f|^2 \in \msc L(\R) $

	Для почти всех $ x \in \R $ справедливо
	$$ \vawe{(\hat f)}(x) = f(x) $$
\end{statement}

\begin{note}
	Требование $ f, \hat f \in \msc L $ избыточно, если более обще определить преобразование Фурье.
\end{note}
