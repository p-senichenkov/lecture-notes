\part{Уравнения первого порядка, разрешённые относительно производной}

\begin{quote}
    \flushright
	Ааа! Дифуры!
\end{quote}

\begin{equ}1
	\frac{\di y(x)}{\di x} = f \big( x, y(x) \big) \quad \text{или} \quad y' = f(x, y)
\end{equ}

\section{Продолжимость решения на границу и за границу; теорема о продолжимости решения на границу}

\begin{definition}
	Пусть $ y = \vphi(x) $ -- решение уравнения \eref1 на $ \braket{a, b} $. Если этот промежуток произвольным образом сузить, то на новом промежутке функция $ y = \vphi(x) $ останется решением, которое называют \it{сужением} исходного решения.
\end{definition}

\begin{definition}
	Решение уравнения \eref1, заданное на промежутке $ \langle a, b) $ \it{продолжимо вправо в точку} $ b $ или \it{на границу}, если найдётся такое решение $ y = \vawe{\vphi}(x) $, определённое на промежутке $ \langle a, b] $, что сужение $ \vawe{\vphi}(x) $ на $ \langle a, b) $ совпадает с $ \vphi(x) $.
\end{definition}

\begin{definition}
	Решение уравнения \eref1, заданное на промежутке $ \braket{a, b} $ \it{продолжимо вправо за точку} $ b $ или \it{за границу}, если найдутся такие $ \vawe{b} > b $ и решение $ y = \vawe{\vphi}(x) $, определённое на промежутке $ \braket{a, \vawe{b}} $, что сужение $ \vawe{\vphi}(x) $ на $ \braket{a, b} $ совпадает с $ \vphi(x) $.
\end{definition}

\begin{theorem}[о продолжимости решения на границу]\label{th:cont}
	\hfill \\
    $ \vphi(x) $ -- решение уравнения \eref1 на промежутке $ \langle a, b), \quad b < +\infty $ \\
    Для того чтобы это решение было продолжимо вправо в точку $ b $ необходимо и достаточно, чтобы существовали последовательность $ \seq{x_k}k $ и число $ \eta \in \R^1 $ такие, что
    \begin{equ}{13}
        \forall k \quad
        \begin{cases}
        	x_k \in \langle a, b) \\
            \big( x_k, \vphi(x_k) \big) \underarr{k \to \infty} (b, \eta) \in \vawe{G}
        \end{cases}
    \end{equ}
\end{theorem}

\begin{iproof}
	\item Достаточность \\
    Пусть выполняется условие \eref{13}
    \begin{statement}
        В силу того, что функция $ f(x, y) $ определена и непрерывна на множестве $ \vawe{G} $, найдутся такие $ c > 0 $ и $ M \ge 1 $, что
        $$ \forall (x, y) \in \vawe{G} \cap \ol{B_c}(b, \eta) \quad |f(x, y)| \le M $$
    \end{statement}
    \begin{iproof}
        \item $ (b, \eta) \in G $, т. е. является внутренней \\
        Тогда существует $ \ol{B_c}(b, \eta) \sub G $ -- компакт, и на нём функция ограничена
        \item $ (b, \eta) \sub \vawe{G} $ и ``вблизи'' находятся точки ``плохой'' границы \\
        Приведём рассуждение \textbf{от противного}: \\
        Допустим, $ |f(b, \eta)| = M - 1 $ и существует последовательность $ c_m \underarr{m \to \infty} 0 $ ($ c_m > 0 $) и последовательность точек $ (x_m, y_m) \in \vawe{G} \cap \ol{B_{c_m}}(b, \eta) $ такие, что $ |f(x_m, y_m)| > M $ \\
        Тогда $ (x_m, y_m) \underarr{m \to \infty} (b, \eta) $, а это значит, что функция $ |f(x, y)| $ терпит разрыв в точке $ (b, \eta) $, так как $ |f(x_m, y_m)| - |f(b, \eta)| > 1 $ для любого $ m $
    \end{iproof}
    Докажем, что существует $ \liml{x \to b-} \vphi(x) $ и он равен $ \eta $: \\
    Для этого покажем, что для любого сколь угодно малого $ \veps > 0 $ найдётся число $ \delta \in \langle a, b) $, что
    \begin{equ}{14}
    	\forall x \in [\delta, b) : |\vphi(x) - \eta| < \veps
    \end{equ}
    Зафиксируем произвольный $ 0 < \veps \le c $ \\
    Тогда $ |f(x, y)| \le M $ для любой точки $ (x, y) \in \vawe{G} \cap \ol{B_\veps}(b, \eta) $ и по условию \eref{13} найдётся такой номер $ m $, что выполняются равентсва
    \begin{equ}{15}
        b - x_m > \frac\veps{2M}, \qquad |\vphi(x_m) - \eta| < \frac\veps2
    \end{equ}
    По формуле Ньютона-Лейбница для всякого $ x \in [x_m, b) $ имеем:
    \begin{multline*}
        |\vphi(x) - \vphi(x_m)| = \bigg| \dint[s]{x_m}x{\vphi'(s)}\bigg| = \bigg| \dint[s]{x_m}x{f \big( s, \vphi(s) \big)}\bigg| \le \dint[s]{x_m}x{|f \big( s, \vphi(s) \big)|} \le \\
        \le M(x - x_m) < M(b - x_m) \underset{\eref{15}_1}< \frac\veps2 \qquad (x_m \le x < b)
    \end{multline*}
    Поэтому
    $$ |\vphi(x) - \eta| \le |\vphi(x) - \vphi(x_m)| + |\vphi(x_m) - \eta| \underset{\eref{15}_2}< \frac\veps2 + \frac\veps2 = \veps $$
    Неравенство \eref{14} верно при $ \delta = x_m $, а значит, $ \vphi(x) \underarr{x \to b^{-0}} \eta $ \\
    Доопределим функцию $ y = \vphi(x) $ в точке $ b $, положив $ \vphi(b) = \eta $ \\
    Согласно лемме о записи решения в интегральном виде
	$$ \vphi(x) = \vphi(x_0) + \dint[s]{x_0}x{f \big(s, \vphi(s) \big)} \quad \forall x_0, x \in \langle a, b) $$
    В этом тождестве можно перейти к пределу при $ x \to b^{-0} $, получая равенство $ \eta = \vphi(x_0) + \dint[s]{x_0}x{f \big( s, \vphi(s) \big)} $, так как по условию точка $ (b, \eta) \in \vawe{G} $, а занчит, функция $ f(x, y) $ определена и непрерывна в этой точке \\
    В результате функция
    $$ \vawe\vphi(x) =
    \begin{cases}
    	\vphi(x), \qquad x \in \langle a, b) \\
        \eta \qquad x = b
    \end{cases} $$
    по определению является продолжением решения $ y = \vphi(x) $ на $ \langle a, b] $
    \item Необходимость \\
    Допустим, что на промежутке $ \langle a, b] $ существует решение $ y = \vawe\vphi(x) $ такое, что $ \vawe\vphi(x) \equiv \vphi(x) $ на $ \langle a, b) $ \\
    Поскольку $ \vawe\vphi(x) $ непрерывна, то $ \vawe\vphi(x) = \eta = \liml{x \to b}\vawe\vphi(x) $ \\
    Но тогда $ \eta = \liml{x \to b^-}\vphi(x) $ и требуемая послеовательность точек $ x_k $ существует, причём по поределению решения точка $ (b, \eta) \in \vawe{G} $
\end{iproof}

\section{Продолжимость решения на границу и за границу; леммы о продолжимости решения за границу отрезка и интервала}

\begin{lemma}[о продолжимости решения за границу отрезка]
    Пусть решение $ y = \vphi(x) $ уравнения \eref1 определено на промежутке $ \langle a, b] $ и точка $ \big( b, \vphi(b) \big) \in G $ \\
    Тогда это решение продолжимо вправо за точку $ b $ на полуотрезок Пеано, построенный для точки $ \big( b, \vphi(b) \big) $.
\end{lemma}

\begin{proof}
    По теореме Пеано на отрезке Пеано $ \ol{P_h} \big( b, \vphi(b) \big) $ существует внутреннее решение $ y = \psi(x) $ \caupr{b, \vphi(b)}. \\
    Тогда функция $ y = \vawe\vphi(x) $, где
    $$ \vawe\vphi(x) =
    \begin{cases}
        \vphi(x), \qquad x \in \langle a, b] \\
        \psi(x), \qquad x \in [b, b + h]
    \end{cases} $$
    по определению является решением уравнения \eref1 на $ \langle a, b + h] $ \\
    В самом деле, в точке $ b $ производная функции $ \vawe\vphi(x) $ существует, так как
    $$ \vawe\vphi_-'(b) = \vphi_-'(b) = f \big( b, \vphi(b) \big) = \psi_+'(b) = \vawe\psi_+'(b) $$
    А выполнение других условий из определения решения для $ \vawe\vphi(x) $ очевидно
\end{proof}

\begin{implication}
    Если решение $ y = \vphi(x) $ уравнения 1 определено на промежутке $ \langle a, b] $ и не продолжимо вправо за точку $ b $, то $ \big( b, \vphi(b) \big) \in \hat{G} $
\end{implication}

\begin{proof}
	Предположение противного противоречит лемме
\end{proof}

Из теоремы о продолжимости решения на границу и последней леммы вытекает следующее утверждение:

\begin{lemma}[о продолжимости решения на границу интервала]
    Пусть решение $ y = \vphi(x) $ уравнения \eref1 определено на промежутке $ \langle a, b) $, существует число $ \eta = \liml{x \to b^-}\vphi(x) $ и точка $ (b, \eta) \in G $ \\
    Тогда это решение продолжимо вправо за точку $ b $
\end{lemma}

\section{Теорема о поведении интегральной кривой полного внутреннего решения}

\begin{theorem}[о поведении интегральной кривой полного внутреннего решения]
    Предположим, что внутреннее решение $ y = \vphi(x) $ уравнения \eref1 определено на промежутке $ \langle a, \beta) $ и не продолжимо вправо. \\
    Тогда для любого компакта $ \ol{H} \sub G $ найдётся такое число $ \delta \in \langle a, \beta) $, что для всякого $ x \in (\delta, \beta) $ точка $ \big( x, \vphi(x) \big) \in G \setminus \ol{H} $
\end{theorem}

\begin{restate}
	При стремлении аргумента полного внутреннего решения к границе максимального интервала существования дуга интегральной кривой покидает любой компакт, лежащий в области $ G $, и никогда в него не возвращается
\end{restate}

\begin{proof}
    Переходя в условиях теоремы на язык последовательностей, докажем, что для любого компакта $ \ol{H} \sub G $ и для любой последовательности $ x_k \underarr{k \to \infty} \beta $, $ x_k \in \langle a, \beta) $ существует $ K > 0 $ такое, что $ \big( x_k, \vphi(x_k) \big) \in G \setminus \ol{H} $ при всех $ k > K $ \\
    Рассуждая \textbf{от противного}, допустим, что существуют компакт $ \ol{H}_* \sub G $ и последовательность $ x_k \to \beta $, $ x_k \in \langle a, \beta) $ такие, что $ \big( x_k, \vphi(x_k) \big) \in \ol{H}_* $ для $ k = 1, 2, ... $ \\
    Отсюда сразу же вытекает, что $ \beta < +\infty $, так как в противном случае найдётся такой индекс $ k^* $, что точка $ \big( x_{k^*}, \vphi(x_{k^*}) \big) $ будет лежать вне компакта в силу его ограниченности \\
    НУО считаем, что последовательность $ x_k $ -- сходящаяся (иначе перейдём к сходящейся подпоследовательности) \\
    Пусть $ (\beta, \eta) = \limi{k} \big( x_k, \vphi(x_k) \big) $ \\
    Тогда предельная точка $ (\beta, \eta) $ также принадлежит компакту $ \ol{H}_* $, а значит, выполняются условия теоремы о продолжимости решения (теор. \ref{th:cont}), согласно которой решение $ y = \vphi(x) $ продолжимо на промежуток $ \langle a, \beta] $ -- \contra с условием теоремы
\end{proof}

\section{Ломаные Эйлера. Лемма о ломаных Эйлера в роли \tpst{$ \veps $}{эпсилон}-решения}

Выберем в области $ G $ произвольную точку $ (x_0, y_0) $ и построим в ней отрезок поля направлений столь малой длины, что он целиком лежит в $ G $, начинаясь в какой-то точке $ (x_{-1}, y_{-1}) $ и заканчиваясь в точке $ (x_1, y_1) $ \\
Проведём вправо через точку $ (x_1, y_1) $ и влево через точку $ (x_{-1}, y_{-1}) $ полуотрезки поля, лежащие в $ G $ и заканчивающиеся в точках $ (x_2, y_2) $ и $ (x_{-2}, y_{-2}) $ соответственно, и так далее \\
Этот процесс можно продолжать любое конечное число шагов $ N $, поскольку область $ G $ -- открытое множество \\
График полученной таким образом непрерывной кусочно-линейной функции $ y = \psi(x) $ называется ломаной Эйлера \\
Итак, установлено, что ломаная Эйлера лежит в области $ G $, проходит через точку $ (x_0, y_0) $ и абсциссы её угловых точек равны $ x_j $ ($ j = \ol{-N, N} $)

\begin{definition}
    \it{Рангом дробления} ломаной Эйлера назовём число, равное $ \max\set{x_j - x_{j - 1}} $.
\end{definition}

Формула, реккурентно задающая ломаную Эйлера $ y = \psi(x) $, иммеет вид: $ \psi(x_0) = y_0 $ и далее при $ j = 0, 1, ..., N - 1 $ для любого $ x \in (x_j, x_{j + 1}] $ или при $ j = 0, -1, ..., 1 - N $ для любого $ x \in [x_{j - 1}, x_j) $
\begin{equ}{1.8}
	\psi(x) = \psi(x_j) + f \big( x_j, \psi(x_j) \big)(x - x_j)
\end{equ}
В частности, при $ j = 0 $ отрезок ломаной Эйлера определён для любого $ x \in [x_{-1}, x_1] $ и, делясь на два полуотрезка, проходит через точку $ (x_0, y_0) $ под углом, тангенс которого равен $ f(x_0, y_0) $ \\
Из формулы \eref{1.8} вытекает, что для всякого $ j = \ol{0, N - 1} $ производная $ \psi'(x) = f \big( x_j, \psi(x_j) \big) $ при $ x \in (x_j, x_{j + 1}) $, а в точке $ x_{j + 1} $ она не определна, как и в точках $ x_{j - 1} $ при $ j \le 0 $ \\
Доопределим $ \psi'(x) $ в точках разрыва как левостороннюю производную при $ x > x_0 $ и как правостороннюю производную при $ x < x_0 $, положив
$$ \psi'(x_j) = \psi_{\mp}'(x_j) \liml{x \to x_j^{\mp0}}\frac{\psi(x) - \psi(x_j)}{x - x_j} \qquad (j = \pm 1. ..., \pm N) $$
А при $ j = 0 $ существует полная производная $ \psi'(x_0) = f(x_0, y_0) $ \\
Таким образом, для любого $ x \in (x_j, x_{j + 1}] $ ($ j = 0, 1, ..., N - 1 $) или для любого $ x \in [x_{j - 1}, x_j) $ ($ j = 0, -1, ..., 1 - N $), дифференцируя равенство \eref{1.8} по $ x $, получаем
\begin{equ}{1.9}
    \psi'(x) = f \big( x_j, \psi(x_j) \big), \qquad j \in \set{1 - N, ..., N - 1}
\end{equ}

Покажем, что на некотором промежутке всегда можно построить функцию, график которой проходит через заданную точку области $ G $, такую, что при подстановке этой функции в уравнение \eref1 окажется, что разность между левой и правой частями уравнения по модулю не превосходит любого сколь угодно малого наперёд заданного положительного числа

\begin{definition}
    Для всякого $ \veps > 0 $ непрерывная и кусочно-гладкая на отрезке $ [a, b] $ функция $ y = \psi(x) $ называется $ \mathit\veps $-\it{решением} уравнения \eref1 на $ [a, b] $, если для любого $ x \in [a, b] $ точка $ \big( x, \psi(x) \big) \in G $ и
    \begin{equ}{1.10}
    	\big| \psi'(x) - f \big( x, \psi(x) \big) \big| \le \veps
    \end{equ}
\end{definition}

\begin{lemma}[о ломаных Эйлера в роли $ \veps $-решения]
    Для любой точки $ (x_0, y_0) \in G $ и для любого отрезка Пеано $ \ol{P_h}(x_0, y_0) $ имеем:
    \begin{enumerate}
        \item Для любого $ \delta > 0 $ на $ \ol{P_h} $ можно построить ломаную Эйлера $ y = \psi(x) $ с рангом дробления, не превосходящим $ \delta $, график которой лежит в прямоугольнике $ \ol{R} $ \nimp[из определения отрезка Пеано]
        \item Для любого $ \veps > 0 $ найдётся такое $ \delta > 0 $, что всякая ломаная Эйлера $ y = \psi(x) $ с рангом дробления, не превосходящим $ \delta $, является $ \veps $-решением уравнения \eref1 на $ \ol{P_h}(x_0, y_0) $
    \end{enumerate}
\end{lemma}

\begin{proof}
	\hfill
    \begin{enumerate}
        \item Для произвольной точки $ (x_0, y_0) $ из $ G $ построим прямоугольник $ \ol{R} \sub G $ с центром в $ (x_0, y_0) $ и два лежащих в нём равнобедренных треугольника $ \ol{T^-}, \ol{T^+} $ с общей вершиной в точке $ (y_0, x_0) $ и основаниями, параллельными оси ординат, как это было сделано при построении отрезка Пеано \\
        При этом зафиксируются константы $ a, b, M, h $ \\
        Выберем $ \delta_* < \delta $ так, чтобы число $ \frac{h}{\delta_*} \fed N \in \N $ \\
        Положим $ x_{j + 1} \define x_j + \delta_* $ ($ j = \ol{0, N - 1}) $, тогда $ x_N = x_0 + h $ \\
        Для всякого $ x > x_0 $ будем последовательно строить отрезки ломаной Эйлера $ y = \psi(x) $ с узлами в точках $ x_j $ \\
        Для любого $ j = 0, ..., N $ это сделать возможно, так как модуль тангенса укла наклона каждого отрезка равен $ \big| f \big(x_j, \psi(x_j) \big) \big| $, а тангенсы углов наклона боковых сторон треугольника $ \ol{T^+} $ по построению равны $ \pm M $, где $ M = \max|f(x, y)| $ на компакте $ \ol{R} $ \\
        Поэтому любой отрезок ломаной Эйлера, начиная с первого, не может пересечь боковую стенку $ \ol{T^+} $, а значит, содержится в нём \\
        В результате для всех $ x \in [x_0, x_0 + h] $ точка $ \big( x, \psi(x) \big) \in \ol{T^+} $ и требуемая ломаная Эйлера построена на $ [x_0, x_0 + h] $ \\
        Для левого полуотрезка Пеано всё аналогично
        \item Зафиксируем теперь произвольное положительное число $ \veps $ \\
        Функция $ f(x, y) $ непрерывна на компакте $ \ol{R} $, следовательно, по теореме Кантора $ f $ равномерно непрерывна на нём. По определнию это занчит, что существует такое $ \delta_1 > 0 $, что для любых двух точек $ (x' y') $ и $ (x'', y'') $ из прямоугольника $ \ol{R} $ таких, что $ |x' - x''| \le \delta_1 $ и $ |y' - y''| < \delta_1 $, выполняется неравенство $ |f(x', y') - f(x'', y'')| \le \veps $ \\
        Положим $ \delta \define \min\set{\delta_1, \frac{\delta_1}M} $ и покажем, что для любой ломаной Эйлера $ y = \psi(x) $ с рангом дробления меньшим, чем $ \delta $ на отрезке Пеано $ \ol{P_h}(x_0, y_0) = [x_0 - h, x_0 + h] $, справедливо неравенство \eref{1.10}: \\
        Возьмём любую точку $ x $ из отрезка Пеано, например справа от $ x_0 $ \\
        Найдётся индекс $ j \in \set{0, ..., N - 1} $ такой, что $ x \in (x_j, x_{j + 1}] $, т. е. $ x_j $ -- ближайшая к $ x $ левая угловая точка ломаной Эйлера \\
        Согласно \eref{1.9}
        $$ \psi'(x) - f \big(x, \psi(x) \big) = f \big( x_j, \psi(x_j) \big) - f \big( x, \psi(x) \big) $$
        Оценим близость аргументов функции $ f $: \\
        По выбору $ \delta $ и $ j $ имеем
        $$ |x - x_j| \le \delta \le \delta_1, \qquad |\psi(x) - \psi(x_j)| \undereq{\eref{1.8}} \big| f \big( x_j, \psi(x_j) \big) \big| \cdot |x - x_j| \le M\delta \bdef[\le]\delta \delta_1 $$
        Поэтому из равномерной непрерывности функции $ f $ вытекает, что
        $$ \big| f(x_j, \psi(x_j) \big) - f \big( x, \psi(x) \big) \big| \le \veps $$
        А значит, неравенство \eref{1.10} из определения $ \veps $-решения выполняется на отрезке Пеано
    \end{enumerate}
\end{proof}

\section{Лемма Асколи--Арцела}

\begin{lemma}[Арцела-Асколи; о существовании равномерно сходящейся подпоследовательности]
    Из любой ограниченной и равностепенно непрерывной на $ [a, b] $ последовательности функций $ \seq{h_n}n $ можно выделить равномерно сходящуюся на $ [a, b] $ подпоследовательность
\end{lemma}

\begin{proof}
	Рациональные числа образуют счётное всюду плотное множество на любом промежутке вещественной прямой \\
    Счётность множества рациональных чисел, расположенных на отрезке $ [a, b] $ означает, что их можно перенумеровать: $ r_1, r_2, ... $ \\
    В точке $ r_1 $ числовая последовательность $ \seq{h_n}n $ по предположению сходится, поэтому из неё можно выбрать сходящуюся подпоследовательность, т. е. существует такая последовательность натуральных чисел
    $$ n^{(1)} = \seq{n_i^{(1)}}i, \qquad n_i^{(1)} < n_{i + 1}^{(1)} $$
    что последовательность значений $ \seq{h_{n_i^{(1)}}(r_1)}i $ сходится \\
    В точке $ r_2 $ последовательность $ \seq{h_{n_i}^{(1)}(r_2)}i $ также ограничена, и из ней можно извлечь сходящуюся подпоследовательность, т. е. у последовательности индексов $ n^{(1)} $ имеется такая подпоследовательность индексов $ n^{(2)} = \seq{n_i^{(2)}}i $, что последовательность значений $ \seq{h_{n_i^{(2)}}(r_2)}i $ тоже сходится. При этом она сходится и в точке $ r_1 $ как подпоследовательность сходящейся последовательности \\
    Продолжаем этот процесс \\
    Введём последовательность индексов $ \seq{n_i^{(i)}}i \quad (n_i^{(i)} < n_{i + 1}^{(i)}) $, где $ n_i^{(i)} $ -- $ i $-й член подпоследовательности $ n^{(i)} $ \\
    Функциональная подпоследовательность $ \seq{h_{n_i}^{(i)}(x)}i $ сходится во всех рациональных точках $ [a, b] $, поскольку в любой рациональной точке $ r_k $ последовательность $ \seq{h_{n_i^{(k)}}(x)}i $ сходится по построению, а любая другая с меньшим верхним индексом является её подпоследовательностью \\
    Покажем, что $ \seq{h_{i_*}(x)}i $, где $ i_* = n_i^{(i)} $ является искомой подпоследовательностью: \\
    Зафиксируем произвольное $ \veps > 0 $ \\
    По условию леммы последовательность $ \seq{h_{i_*}(x)}i $ равностепенно непрерывна, следовательно, по выбранному $ \veps $ найдётся такое число $ \delta > 0 $, что
    $$ \forall i \in \N \quad \forall x', x'' \in [a, b] : \quad \nimp[\bigg(] |x' - x''| < \delta \implies |h_{i_*}(x') - h_{i_*}(x'')| \le \frac\veps3 \nimp[\bigg)] $$
    По построению последовательность функций $ \seq{h_{i_*}(x)}i $ сходится поточечно во всех рациональных точках $ r_k $ из $ [a, b] $ \\
    Поэтому по выбранному $ \veps $ для любого $ k \in \N $ найдётся такой номер $ N_{r_k} > 0 $, что $ |h_{i_*}(r_k) - h_{j_*}(r_k)| \le \faktor\veps3 $ для любых $ i_*, j_* > N_{r_k} $ \\
    Последовательность индексов $ N_{r_1}, N_{r_2}, ..., $ -- счётная, поэтому она может стремиться к бесконечности. Перейти к конечной подпоследовательности позволяет использование появившейся из определения равностепенной непрерывности универсальной константы $ \delta $ и плотности множества рациональных чисел: \\
    Разобьём отрезок $ [a, b] $ на непересекающиеся промежутки, длина которых не превосходит $ \delta $. Пусть их окажется $ l $ штук \\
    Множество рациональных чисел всюду плотно, поэтому в каждом промежутке можно выбрать по рациональному числу: $ r_1^*, ..., r_l^* $ \\
    Пусть $ N = \max\set{N_{r_1^*}, ..., N_{r_l^*}} $, где константы $ N_r $ взяты из определения поточечной сходимости последовательности $ \seq{h_{i_*}(x)}i $ \\
    Возьмём произвольное число $ x \in [a, b] $. Предположим, что оно попало в промежуток с номером $ p $. Тогда для любых $ i_*, j_* > N $ получаем:
    $$ |h_{i_*}(x) - h_{j_*}(x)| \trile |h_{i_*}(x) - h_{i_*}(r_p^*)| + |h_{i_*}(r_p^*) - h_{j_*}(r_p^*)| + |h_{j_*}(r_p^*) - h_{j_*}(x)| \le \veps $$
    так как $ |x - r_p^*| \le \delta $ и верна оценка из определения равномерной сходимости \\
    Итак, для любого $ \veps > 0 $ нашлось такое $ N $, что для любых $ i_*, j_* \ge N $ и $ x \in [a, b] $ справедливо неравенство $ |h_{i_*}(x) - h_{j_*}(x)| \le \veps $
\end{proof}

\begin{remark}
	При выполнении условий леммы Арцела-Асколи она позволяет ``объявить о рождении'' функции $ h(x) $, определённой на отрезке $ [a, b] $ и предельной для некоторой подпоследовательности функций $ h_n(x) $ \\
    При этом, по теореме Стокса-Зайделя предельная функция непрерывна на $ [a, b] $
\end{remark}

\section{Ломаные Эйлера. Теорема Пеано о существовании внутреннего решения}

\begin{theorem}{Пеано; о существовании внутреннего решения}
    Пусть правая часть уравнения \eref1 непрерывна в области $ G $ \\
    Тогда для любой точки $ (x_0, y_0) \in G $ и для любого отрезка Пеано $ \ol{P_h}(x_0, y_0) $ существует по крайней мере одно решение задачи Коши уравнения \eref1 с начальными данными $ x_0, y_0) $, определённое на $ \ol{P_h}(x_0, y_0) $
\end{theorem}

\begin{proof}
    Возьмём произвольную точку $ (x_0, y_0) $ из области $ G $ и построим какой-либо отрезок Пеано $ \ol{P_h}(x_0, y_0) $ \\
    Выберем произвольную последовательность положительных чисел $ \veps_n $, стремящуюся к нулю при $ n \to \infty $ \\
    Тогда по лемме об $ \veps $-решении для всякого $ n $ можно построить ломаную Эйлера $ \psi_n(x) $, проходящую через точку $ (x_0, y_0) $, определённую на $ \ol{P_h}(x_0, y_0) $ и являющуюся $ \veps_n $-решением уравнения \eref1 на отрезке $ \ol{P_j}(x_0, y_0) $ \\
    Поэтому для любых $ n \in \N $ и $ x \in \ol{P_h}(x_0, y_0) $ точка $ \bigg( x, \psi_n(x) \bigg) \in \ol{R} $ и выполняется неравенство \eref{1.10} $ |\psi_n'(x) - f \big( x, \psi_n(x) \big) | < \veps_n $ \\
    Покажем, что последовательность ломаных Эйлера $ \seq{\psi_n(x)}n $ на отрезке Пеано удовлетворяет лемме Арцела--Асколи \\
    Последовательность $ \seq{\psi_n(x)}n $ равномерно ограничена, так как график любой функции $ y = \psi_n(x) $ лежит в прямоугольнике $ \ol{R} $, а значит, $ |\psi_n(x)| \le |y_0| + b $ для любого $ x \in [x_0 - h, x_0 + h] $ \\
    Для доказательства равностепенной непрерывности зафиксируем произвольное $ \veps > 0 $ \\
    Положим $ \delta = \faktor\veps{M} $, где $ M = \max\limits_{(x, y) \in \ol{R}}|f(x, y)| $ \\
    Тогда для любых $ n \in \N $ и $ x', x'' \in \ol{P_h}(x_0, y_0) $ таких, что $ |x'' - x'| \le \delta $, получаем:
    \begin{multline*}
        |\psi_n(x'') - \psi_n(x')| = \bigg| \dint[s]{x_0}{x''}{\psi_n'(s)} - \dint[s]{x_0}{x'}{\psi_n'(s)} \bigg| = \bigg| \dint[s]{x'}{x''}{\psi_n'(s)} \underset{\eref{1.9}}\le \\
        \le \bigg| \dint[s]{x'}{x''}{\max\limits_{j = 1 - N, ..., N - 1} \big| f \big( x, \psi_n(x_j) \big) \big|} \bigg| \le M|x'' - x'| \le M\delta = \veps
    \end{multline*}
    Действительно, интегрируя кусочно-постоянную функцию $ \psi'(x) $ по $ s $ от $ x_0 $ до $ x $, для любого $ x \in [x_{-N}, x_N] $ имеем: $ \psi(x) = \psi(x_0) + \dint[s]{x_0}x{\psi'(s)} $, где
    $$ \dint[s]{x_0}x{\psi(s)} = \sum_{k = 0}^{j - 1} \dint[s]{x_k}{x_{k + 1}}{\psi'(s)} + \dint[s]{x_j}x{\psi'(s)}, \qquad x \in (x_j, x_{j + 1}], \quad j \in \set{0, ..., N - 1} $$
    $$ \dint[s]{x_0}x{\psi'(s)} = \sum_{k = j + 1}^{-1} \dint[s]{x_{k + 1}}{x_k}{\psi'(s)} + \dint[s]{x_{j + 1}}x{\psi'(s)}, \qquad x \in [x_j, x_{j + 1}), \quad j \in \set{-N, ..., -1} $$
    В результате последовательность ломаных Эйлера $ \psi_n(x) $ удовлетворяет условиям леммы Арцела-Асколи, и из неё можно выделить равномерно сходящуюся подпоследовательность $ \seq{\psi_{i_*}(x)}{i_*} $ \\
    Пусть $ \psi_{i_*} \xrightrightarrows[i_* \to \infty]{x \in \ol{P_h}} \vphi(x) $ \\
    Тогда, согласно замечанию \nimp[после леммы Арцела-Асколи] функция $ y = \vphi(x) $ непрерывна на отрезке Пеано \\
    Поскольку $ \psi_{i_*}(x) $ по построению является $ \veps_{i_*} $-решением, из неравенства \eref{1.10} вытекает, что
    $$ \forall x \in \ol{P_h}(x_0, y_0) \quad \forall i_* \in \N : \quad \psi_{i_*}'(x) = f \big( x, \psi_{i_*}(x) \big) + \Delta_{i_*}(x), \qquad |\Delta_{i_*}(x)| \le \veps_{i_*} $$
    Интегрируя это равенство по $ s $ от $ x_0 $ до $ x $ получаем:
    \begin{equ}{1.11}
        \psi_{i_*}(x) - \psi_{i_*}(x_0) = \dint[s]{x_0}x{f \big( s, \psi_{i_*}(s) \big)} + \dint[s]{x_0}x{\Delta_{i_*}(s)}
    \end{equ}
    причём $ \psi_{i_*}(x_0) = y_0 $ и $ \bigg| \dint[s]{x_0}x{\Delta_{i_*}(s)} \bigg| \le \veps_{i_*}|x - x_0| \underarr{i_* \to \infty} 0 $, так как $ |x - x_0| \le h $ \\
    Кроме того, $ f \big( s, \psi_{i_*}(s) \big) \xrightrightarrows[i_* \to \infty]{s \in \ol{P_h}} f \big( s, \vphi(s) \big) $, поскольку любая точка $ \big( s, \psi_{i_*}(s) \big) \in \ol{R} $ и $ f(x, y) $ по теореме Кантора равномерно непрерывна на $ \ol{R} $

    Поэтому можно осуществить предельный переход под знаком интеграла:
    $$ \dint[s]{x_0}x{f \big( s, \psi_{i_*}(s) \big)} \underarr{i_* \to \infty} \dint[s]{x_0}x{f \big( s, \vphi(s) \big)} $$
    Переходя в обеих частях равенств \eref{1.11} к пределу при $ i_* \to \infty $, получаем тождество
    $$ \vphi(x) \overset{[x_0 - h, x_0 + h]}\equiv \vphi(x_0) + \dint[s]{x_0}x{f \big(s, \vphi(s) \big)} $$
    Поэтому, согласно лемме о записи решения в интегральном виде, предельная функция $ y = \vphi(x) $ является решением ВЗК($ x_0 $, $ y_0 $) уравнения \eref1 на отрезке Пеано $ [x_0 - h, x_0 + h] $
\end{proof}

\section{Теорема о существовании решения для одного из случаев \tpst{$ U_1^+ $, $ O_1^+ $, $ B_{1<}^+ $, $ B_{1=}^+ $}{U1+, O1+, B1<+, B1=+}}

Для упрощения обозначений и формул, используемых в дальнейшем при решении граничной задачи Коши, НУО будем считать, что задача всегда ставится в начале координат и функция $ f $ там равна нулю, т. е. уравнение \eref1 имеет вид
\begin{equ}{1.12}
	y' = f_0(x, y)
\end{equ}
где функция $ f_0 $ определена и непрерывна на множестве $ \vawe{G} = G \cup \hat{G} $, точка $ O = (0, 0) \in \vawe{G} $, $ f_0(0, 0) = 0 $ и поставлена граничная задача Коши с начальными данными $ 0, 0 $.

НУО будем считать, что выполняются условия:
\begin{equ}{1.13+}
    \begin{cases}
        b_{a, u}^+(a) \le a \quad \text{при } \tau_u = 0 \\
        \forall x \in [0, a] \quad b_{a, u}^+{}'(x) \ge \tau_u \quad \text{при } \tau_u > 0 \\
        -b_{a, l}^+(a) \le a \quad \text{при } \tau_l = 0 \\
        \forall x \in [0, a] \quad -b_{a, l}^+{}'(x) \ge \tau_l \quad \text{при } \tau_l > 0
    \end{cases}
\end{equ}

Во всех точках кривых $ \gamma_{a, u}^+ $ и $ \gamma_{a, l}^+ $ введём ограничения на функцию $ f_0 $ в случаях $ U_1^{+, =} $, $ O_{1, =}^+ $, $ B_{1, =}^{+, =} $, $ B_{1, =}^{+, >} $ и $ B_{1, <}^{+, =} $:
\begin{equ}{1.15+}
	\forall x \in (0, a] \quad
    \begin{cases}
        f_0 \big( x, b_{a, u}^+(x) \big) \le b_{a, u}^+{}'(x), \qquad \text{если } b_{a, u}^+{}'(0) = 0 \\
        f_0 \big( x, b_{a, l}^+(x) \big) \ge b_{a, l}^+{}'(x), \qquad \text{если } b_{a, l}^+{}'(0) = 0,
    \end{cases}
\end{equ}
означающие, что в любой точке $ \gamma_{a, u}^+ $ и $ \gamma_{a, l}^+ $ правый полуотрезок поля направлений уравнения \eref{1.12} направлен внутрь или по границе области $ G $.

\begin{theorem}[о существовании решения граничной задачи Коши]\label{th:ex:bound:1}
    Предположим, что в уравнении \eref{1.12} функция $ f_0 $ определена и непрерывна на множестве $ \vawe{G} $. \\
    Тогда в каждом из случаев $ (N_1^+), (U_1^{+, >}), (O_{1, <}^+), (B_{1, <}^{+, >}) $ и в каждом из случаев $ (U_1^{+, =}), (O_{1, =}^+) $, $ (B_{1, =}^{+, =}) $, $ (B_{1, =}^{+, >}), (B_{1, <}^{+, =}) $ при условиях \eref{1.15+} на любом правом граничном отрезке Пеано существует по крайней мере одно решение граничной задачи Коши с начальными данными $ (0, 0) $
\end{theorem}

\begin{proof}
    Рассмотрим, например, случай $ (B_{1, =}^{+, >}) $ \\
    Согласно \eref{1.13+} \nimp[(первые два неравенства)] правая верхнеграничная функция $ b_{a, u}^+(x) $, параметризующая кривую $ \gamma_{a_u, u}^+{}'(x) \ge \tau_u $ для любого $ x \in (0, a_u] $. А у правой нижнеграничной привой $ \gamma_{a_l, l}^+ $ константа $ a_l = c_O $ в силу \eref{1.13+} \nimp[(последние два нераенства)] \\
    Пусть $ c_* \define \min\set{c_U, c_O} $, тогда множество $ B_{c_*}^+ \setminus \big( \gamma_{a_u, u}^+ \cup \gamma_{a_l, l}^+ \big) \sub G $

    Далее, для $ \tau_u $ найдётся (по непрерывности $ f_0 $) такая $ \delta_{\tau_u} $, что $ |f_0(x, y)| \le \tau_u $ в любой точке $ \delta_{\tau_u} $-окрестности начала координат, принадлежащей $ \vawe{G} $ \\
    Положим $ \vawe{c} \define \min\set{c_*, \delta_{\tau_u}} $, тогда на множестве $ B_{\vawe{c}}^+ $ для функции $ |f_0| $ справедлива та же оценка \\
    Построим теперь лежащий в $ B_{\vawe{c}}^+ $ криволинейный треугольник $ \ol{T_b^+} $, как это было сделано при описании случая $ (B_{1, =}^{+, >}) $. Его высота $ h^+ = \vawe{a} $ \\
    Поскольку отрезок оси абсцисс $ [0, h^+] $ лежит в $ \vawe{G} $ и является отрезком поля направлений в точке $ O \in \hat{G} $, из точки $ O $ вправо можно начать строить ломаную Эйлера с проивольным рангом дробления \\
    Ломаная Эйлера не может покинуть $ \ol{T_b^+} $ через верхнюю боковую сторону, лежащую на прямой $ y = \tau_ux $, так как в любой её точке $ |f_0(x, y)| \le \tau_u $. Аналогично при попадании ломаной Эйлера при $ x = x_* > 0 $ на нижнюю боковую сторону, являющуюся частью правой нижнеграничной кривой $ \gamma_{\vawe{a}, l}^+ $, по условию \eref{1.15+} \nimp[(второе неравенство)] $ f_0 \big( x_*, b_{\vawe{a}, l}^+(x_*) \big) \ge b_{\vawe{a}}^+{}'(x_*) $, а значит, при $ x > x_* $ следующий отрезок ломаной будет либо лежать на $ \gamma_{\vawe{a}, l}^+ $, либо внутри треугольника в силу выпуклости $ \gamma_{\vawe{a}, l}^+ $. Поэтому ломаная Эйлера с произвольным выбранным рангом дробления может быть продолжена на весь правный граничный отрезок Пеано $ [0, h^+] $ \\
    Дальше дословно повторяется доказательство теоремы Пеано. \\
    Аналогичные рассуждения проводятся и в остальных случаях.
\end{proof}

\section{Теорема об отсутствии решения граничной задачи Коши}

\begin{theorem}[об отсутствии решений граничной задачи Коши]
    В каждом из случаев $ (U_2^{+, >}) $, $ (O_{2, <}^+) $, $ (B_{2, <}^{+, >}) $, $ (N_2^+) $ граничная задача Коши с начальными данными $ (0, 0) $ не имеет решений в правой полуплоскости
\end{theorem}

\begin{proof}
    Допустим, что в каждом случае из условия теоремы на некотором отрезке $ [0, a] $ существует решение $ y = \vphi(x) $ задачи Коши уравнения \eref{1.12} с начальными данными $ (0, 0) $, т. е. $ \vphi(0) = 0 $. Тогда $ \vphi'(0) = f_0 \big( 0, \vphi(0) \big) = 0 $. Но график любого решения должен лежать в $ \vawe{G} $, а значит, располагаться не ниже правой верхнеграничной кривой, у которой в точке $ O $ тангенс угла наклона согласно \eref{1.13+} равен $ 2\tau_u > 0 $, или не выше правой нижнеграничной кривой, имеющей в точке $ O $ тангенс угла наклона, равный $ -2\tau_l < 0 $. Поэтому $ \vphi'(0) \ne 0 $ -- \contra
\end{proof}

\section{Лемма о продолжимости решений на отрезок Пеано; лемма о верхнем и нижнем решениях}

\begin{lemma}[о продолжимости решений на отрезок Пеано]
    Пусть $ y = \vphi(x) $ -- это решение внутренней задачи Коши с начальными данными $ x_0, y_0 $, определённое на $ \ol{P_h}(x_0, y_0) $. \\
    Тогда любое другое решение уравнения \eref1 $ y = \psi(x) $ этой же задачи Коши, определённое на промежутке $ \braket{a, b} \subsetneq [x_0 - h, x_0 + h] $, продолжимо на $ \ol{P_h}(x_0, y_0) $
\end{lemma}

\begin{proof}
	Докажем, например, продолжимость решения $ y = \psi(x) $ с $ \psi(x_0) = y_0 $ на правый полуотрезок Пеано: \\
    Если $ \braket{a, b} = \langle a, b) $ (т. е. $ b \le x_0 + h $), то график решения $ y = \psi(x) $ при $ x \in [x_0, b) $ лежит в треугольнике $ \ol{T^+} $, построенном для решения $ y = \vphi(x) $. Поэтому у любой последовательности $ x_k \in [x_0, b) $ и $ x_k \infarr{k} b $ точки $ \big( x+k, \psi(x_k) \big) \in \ol{T^+} \sub \ol{R} $, а значит, найдётся сходящаяся последовательность $ \big( x_{k_l}, \psi(x_{k_l}) \big) $. Её предел -- точка $ (b, \eta) \in \ol{T^+} $ \\
    Следовательно, по теореме о продолжимости решения (теор. \ref{th:cont}) $ y = \psi(x) $ продолжимо на $ [x_0, b] $, хотя могло быть там сразу и задано
    \begin{itemize}
        \item Если теперь $ b = x_0 + h $, то лемма доказана
        \item Пусть $ b < x_0 + h $. Построим равнобедренный треугольник $ \ol{T_1^+} $ с вершиной в точке $ (b, \eta) $, боковыми сторонами, имеющими тангенсы углов наклона $ \pm M $, и основанием, лежащим на основании треугольника $ \ol{T^+} $ с абсциссой $ x_0 + h $. Тогда $ \ol{T_1^+} \sub \ol{T^+} $ и по теореме Пеано на $ [b, x_0 + h] $ существует решение задачи Коши с начальными данными $ (b, \eta) $, продолжающее $ \psi(x) $ до точки $ x_0 + h $ включительно.
    \end{itemize}
\end{proof}

\section{Теорема о локальной единственности решения внутренней задачи Коши}

Пусть $ (x_0, y_0) \in G, \quad \ol{P_h}(x_0, y_0) $ -- некий отрезок Пеано и $ \seq{\chi_k(x)}k $ -- произвольная последовательность решений ЗК($ x_0, y_0 $) уравнения \eref1, определённых на $ [x_0 - h, x_0 + h] $

\begin{statement}\label{st:2}
	Для любых $ k \in \N, \quad x \in [x_0 - h, x_0 + h] $ функции
    $$ \chi_k^l(x) \define \min\set{\chi_1(x), ..., \chi_k(x)}, \qquad \chi_k^u(x) \define \max\set{\chi_1(x), ..., \chi_k(x)} $$
    также являются решениями поставленной задачи на $ \ol{P_h}(x_0, y_0) $
\end{statement}

\begin{proof}
    Действительно, эти функции удовлетворяют всем трём условиям из определения решения, поскольку для любого $ x_* \in [x_0 - h, x_0 + h] $ найдётся такой индекс $ 1 \le \bm{j} \le k $, что, например, $ \chi_k^l(x_*) = \chi_j(x_*) $, и если $ \chi_j(x_*) = \chi_m(x_*) $, то $ \chi_j'(x_*) = \chi_m'(x_*) = f \big( x_*, \chi_k^l(x_*) \big) $
\end{proof}

\begin{lemma}[о нижнем и верхнем решениях]
    Существуют решения ЗК($ x_0, y_0 $) $ y = \chi^l(x) $ и $ y = \chi^u(x) $ уравнения \eref1 такие, что
    \begin{equ}{1.19}
    	\forall k \in \N \quad \forall x \in [x_0 - h, x_0 + h] : \quad
        \begin{cases}
        	\chi^l(x) \le \chi_k^l(x) \\
            \chi^u(x) \ge \chi_k^u(x)
        \end{cases}
    \end{equ}
\end{lemma}

\begin{proof}
    Рассмотрим, например, последовательность решений $ \seq{x_k^l(x)}k $ на отрезке \\
    $ [x_0, x_0 + h] $. Поскольку все их графики лежат в треугольнике $ \ol{T^+} $, полученном при построении отрезка Пеано, эта последовательность равномерно ограничена и равностепенно ограничена (см. док-во теоремы Пеано). Следовательно, по лемме Арцела-Асколи из неё можно выделить равномерно на $ \ol{P_h}{(x_0, y_0)} $ сходящуюся подпоследовательность, предел которой тоже будет решением уравнения \eref1 на отрезке Пеано \\
    Но последовательность $ \chi_k^l(x) $ монотонно убывает, поэтому она сама будет сходиться к нижнему решению $ y = \chi^l(x) $, для которого, очевидно, будет верно неравенство \eref{1.19} \\
    Рассуждения для отрезка аналогичны так же, как и доказательство сходиомости функции $ \chi_k^u(x) $ к верхнему решению $ y = \chi^u(x) $
\end{proof}

\section{Лемма Гронуола}

\begin{lemma}[Гронуолла; об интегральной оценке функции сверху]\label{lm:Gron}
    Пусть функция $ h(x) \in \Cont{\braket{a, b}} $ и существуют такие $ x_0 \in \braket{a, b}, \quad \lambda \ge 0, \quad \mu > 0 $, что
    \begin{equ}{1.20}
        \forall x \in \braket{a, b} \quad 0 \le h(x) \le \lambda + \mu \bigg| \dint[s]{x_0}x{h(s)} \bigg|
    \end{equ}
    Тогда для любого $ x \in \braket{a, b} $ справедливо неравенство
    \begin{equ}{1.21}
        h(x) \le \lambda e^{\mu|x - x_0|}
    \end{equ}
\end{lemma}

\begin{iproof}
	\item Предположим, что $ x \ge x_0 $ \\
    Введём в рассмотрение функцию $ g(x) = \dint[s]{x_0}x{h(s)} $
    $$ \implies \quad g(x_0) = 0, \qquad g(x) \ge 0, \qquad g(x) \in \Cont[1]{[x_0, b \rangle}, \qquad g'(x) = h(x) \ge 0 $$
    Подставим $ g(x) $ в \eref{1.20}:
    $$ g'(x) \le \lambda + \mu g(x) \quad \implies \quad g'(x) - \mu g(x) \le \lambda \quad \implies \quad e^{-\mu(x - x_0)} \bigg( g'(x) - \mu g(x) \bigg) \le \lambda e^{-\mu(x - x_0)} $$
    При этом,
    $$ \bigg( g(x) e^{-\mu(x - x_0)} \bigg)' = g'(x)e^{-\mu(x - x_0)} - \mu e^{-\mu(x - x_0)} g(x) = e^{-\mu(x - x_0)} \bigg( g'(x) - \mu g(x) \bigg) $$
    Отсюда
    $$ \bigg( g(x) e^{-\mu(x - x_0)} \bigg)' \le \lambda $$
    Проинтегрируем по $ s $ от $ x_0 $ до $ x $:
    $$ g(x)e^{-\mu(x - x_0)} - \underbrace{g(x_0)}_0 \le \lambda \dint[s]{x_0}x{e^{-\mu(s - x_0)}} = -\frac\lambda\mu(e^{-\mu(x - x_0)} - 1) $$
    Умножим на $ e^{\mu(x - x_0)} $:
    $$ g(x) \le \frac\lambda\mu (e^{\mu(x - x_0)} - 1) $$
    Подставим в \eref{1.20}:
    $$ h(x) \le \lambda + \mu g(x) \le \lambda e^{\mu(x - x_0)} $$
    Таким образом, неравенство доказано для всех $ x \in [x_0, b \rangle $
    \item Если $ x \le x_0 $, то в \eref{1.20}
    $$ h(x) \le \lambda - \mu \dint[s]{x_0}x{h(s)}, \qquad g(x) \le 0 $$
    Дальнейшее доказательство аналогично
\end{iproof}

\begin{implication}
	Если $ \lambda = 0 $, то есть
    $$ 0 \le h(x) \le \mu \bigg| \dint[s]{x_0}x{h(s)} \bigg| $$
    то $ h(x) \overset{\braket{a, b}}\equiv 0 $
\end{implication}

\section{Условия Липшица; теорема о множестве единственности}

\begin{definition}
	Функция $ f(x, y) $ удовлетворяет условию Липшица по $ y $ глобально на множестве $ D \sub \R^2 $, если
    \begin{equ}{1.22}
        \exist L > 0 : \quad \forall (x, y_1), (x, y_2) \in D \quad |f(x, y_1) - f(x, y_2)| \le L|y_1 - y_2|
    \end{equ}
\end{definition}

\begin{notation}
    $ f \in \operatorname{Lip}_y^{gl}(D) $
\end{notation}

\begin{definition}
    Функция $ f(x, y) $ удовлетворяет условию Липшица по $ y $ локально на множестве $ \vawe{G} $, если для любой точки $ (x_0, y_0) \in \vawe{G} $ найдётся замкнутая $ c $-окрестность $ \ol{B}_c(x_0, y_0) $ такая, что функция $ f $ удовлетворяет условию Липшица по $ y $ глобально на множестве $ U_c = \vawe{G} \cap \vawe{B}_c(x_0, y_0) $
\end{definition}

\begin{notation}
    $ y \in \operatorname{Lip}_y^{loc}(\vawe{G}) $
\end{notation}

\begin{theorem}[о множестве единственности]
    Пусть в уравнении \eref1 функция $ f(x, y) $ опредлена и непрерывна на множестве $ \vawe{G} $ и удовлетворяет условию Липшица по $ y $ локально на множестве $ \vawe{G^\circ} = G^\circ \cup \hat{G^\circ} $, \nimp[где $ G^\circ \sub G $ -- область, а $ \hat{G^\circ} \sub \partial G^\circ \cap \hat{G} $]. \\
    Тогда $ \vawe{G^\circ} $ -- множество единственности для уравнения \eref1.
\end{theorem}

\begin{proof}
    Возьмём любую точку $ (x_0, y_0) $ из множества $ \vawe{G^\circ} $ и покажем, что она является точкой единственности. \\
    Поскольку $ f \in \operatorname{Lip}_y^{loc}(\vawe{G^\circ}) $, найдутся $ \ol{B}_c(x_0, y_0) $ и $ L > 0 $ такие, что $ f \in \operatorname{Lip}_y^{gl}(U_c) $ с константой $ L $, где $ U_c = \vawe{G^\circ} \cap \ol{B}_c(x_0, y_0) $
    \begin{itemize}
        \item Если $ (x_0, y_0) \in G^\circ $, то найдётся $ c > 0 $ такое, что $ U_c = \ol{B}_c(x_0, y_0) $, решение ЗК($ x_0, y_0 $) существует на некотором интервале $ (a, b) \ni x_0 $ и для любого решения этой задачи, уменьшая при необходимости $ (a, b) $, можно добиться, чтобы его график лежал в $ U_c $
        \item Пусть $ (x_0, y_0) \in \hat{G^\circ} $
        \begin{itemize}
            \item Если решение ЗК($ x_0, y_0 $) отсутсвует, то $ (x_0, y_0) $ -- это точка единственности по определению
            \item Пусть решение существует на некотром промежутке $ \braket{a, b} $ таком, что $ x_0 \in \braket{a, b} \sub [x_0 - c, x_0 + c] $
            \begin{statement}
                Тогда, уменьшая $ \braket{a, b} $ при необходиости можно добиться, чтобы график решения лежал в $ U_c $
            \end{statement}
            \begin{proof}
                Действительно, очевидно, что с уменьшением $ \braket{a, b} $ график решения попадает в $ \ol{B}_c(x_0, y_0) $. А ситуация, когда при $ x < x_0 $ и (или) $ x > x_0 $ график, оставаясь в $ \vawe{G} $, не принадлежит $ \vawe{G^\circ} $, преодолевается за счёт выбора константы $ c_1 > c $ такой, что в $ \ol{B}_{c_1}(x_0, y_0) $ юудет выполняться глобальное условие Липшица с константой, скажем, $ L_1 \define L + 1 $. В результате с учётом непрерывности функции $ f(x, y) $ бласть $ \vawe{G^\circ} $ увеличиться, включив в себя дугу интегральной кривой в малой окрестности точки $ (x_0, y_0) $
            \end{proof}
        \end{itemize}
    \end{itemize}
    Рассмотрим любые два решения $ y = \vphi_1(x) $ и $ y = \vphi_2(x) $ ЗК($ x_0, y_0 $), которые определены по крайней мере на некотором общем промежутке $ \braket{\alpha, \beta} $ таком, что $ x_0 \in \braket{\alpha, \beta} \sub [x_0 - c, x_0 + c] $ \\
    Как установлено выше, уменьшая при необходимости $ \braket{\alpha, \beta} $, можно добиться, чтобы для всякого $ x \in \braket{\alpha, \beta} $ точки $ \big( x, \vphi_1(x) \big), \big( x, \vphi_2(x) \big) \in U_c $ \\
    По лемме о записи решения в интегральном виде для любого $ x \in \braket{\alpha, \beta} $ справедливо
    $$ \vphi_j(x) = \vphi_j(x_0) + \dint[s]{x_0}x{f \big( s, \vphi_j(s) \big)}, \qquad j = 1, 2 $$
    Поэтому
    $$ \vphi_2(x) - \vphi_1(x) = \dint[s]{x_0}x{\bigg( f \big( s, \vphi_2(s) \big) -  \big( s, \vphi_1(s) \big) \bigg)} $$
    точки $ \big( s, \vphi_j(s) \big) \in U_c $ и для них выполнено неравенство \eref{1.22}. Тогда
    $$ |\vphi_2(x) - \vphi_1(x)| \le \bigg| \dint[s]{x_0}x{\big| f \big( s, \vphi_2(s) \big) - f \big( s, \vphi_1(s) \big) \big|} \bigg| \le \bigg| \dint[s]{x_0}x{L \big| \vphi_2(s) - \vphi_1(s) \big|} \bigg| $$
    К последнему неравенству можно применить следствие к лемме Гронуолла (лемма \ref{lm:Gron}), где $ h(x) = |\vphi_2(x) - \vphi_1(x)|, \quad \lambda = 0, \quad \mu = L $ \\
    Тогда $ |\vphi_2(x) - \vphi_1(x)| \overset{\braket{\alpha, \beta}}\equiv 0 $, т. е. решения $ y = \vphi_1(x) $ и $ \vphi_2(x) $ ЗК($ x_0, y_0 $) совпадают в каждой точке $ \braket{\alpha, \beta} \ni x_0 $. Поэтому по определению $ (x_0, y_0) $ -- это точка единственности
\end{proof}

\section{Теорема Осгуда}

\begin{theorem}[Осгуда; о единственности в области; сильная]
    Пусть в уравнении \eref1 функция $ f(x, y) $ непрерывна в области $ G $ и
    \begin{equ}{1.24}
    	\forall ~ (x, y_1), (x, y_2) \in G \quad |f(x, y_2) - f(x, y_1)| \le h \big( |y_2 - y_1| \big)
    \end{equ}
    где функция $ h(s) $ определена, непрерывна и положительна для всякого $ s \in (0, +\infty) $ и
    $$ \dint[s]\veps{a}{h^{-1}(s)} \underarr{\veps \to 0} \infty, \qquad a > \veps > 0 $$
    Тогда $ G $ -- это область единственности для уравнения \eref1.
\end{theorem}

\begin{proof}
	Без доказательства
\end{proof}

\section{Область существования общего решения, лемма о поведении в ней решений, формула общего решения}

Опишем множество $ A^* $, в котором можно построить общее решение, поскольку гарантировать его существование во всей области единственности $ G^\circ $ нельзя, какой бы малой она ни была \\
В этом параграфе в роли $ A^* $ будет выступать вводимый ниже компакт $ \ol{A} $

\begin{algorithm}[построения $ \ol{A} $]
    Пусть $ G^\circ $ -- область единственности для уравнения \eref1. \\
    Возьмём любую точку $ (x_0^*, y_0^*) \in G^\circ $ \\
    Поскольку $ G^\circ $ является открытым множеством, существует такое $ \delta > 0 $, что $ \ol{B}_{2\delta}(x_0^*, y_0^*) \sub G^\circ $ \\
    Пусть числа $ y_1, y_2 $ таковы, что
    $$
    \begin{cases}
    	0 < y_0^* - y_1 < \delta \\
        0 < y_2 - y_0^* < \delta
    \end{cases} $$
    и найдётся отрезок $ [a, b] \ni x_0^* $ такой, что графики решений ЗК($ x_0^*, y_1 $) $ y = \vphi_1(x) $ и ЗК($ x_0^*, y_2 $) $ y = \vphi_2(x) $ лежат в $ \ol{B}_c $ при $ x \in [a, b] $. Тогда в $ \ol{B}_\delta $ содержится компакт
    \begin{equ}{1.25}
        \ol{A} = \set{(x, y) | a \le x \le b, \quad \vphi_1(x) \le y \le \vphi_2(x)}
    \end{equ}
\end{algorithm}

При этом $ A $ \nimp[(то же самое, со строгими неравенствами)] -- это область, так как по построению $ \vphi_1(x_0^*) = y_1 < y_2 = \vphi_2(x_0^*) $, а значит, $ \vphi_1(x) < \vphi_2(x) $ для всякого $ x \in [a, b] $, поскольку в области единственности $ G^\circ $ дуги интегральных кривых не могут соприкасаться и разбивать $ A $ на несвязные подмножества

\begin{lemma}[о поведении решений на компакте $ \ol{A} $]\label{lm:comp}
    Для любой точки $ (x_0, y_0) \in \ol{A} $ решение \caupr[\eref1]{x_0, y_0} $ y = \vphi(x) $ продолжимо на отрезок $ [a, b] $
\end{lemma}

\begin{proof}
    Для любой точки $ (x_0^*, y_0^*) \in G^\circ $ построим компакт $ \ol{A} $ вида \eref{1.25}, тогда $ \ol{A} \sub \ol{B}_\delta \sub \ol{B}_{2\delta} \sub G^\circ $ \\
    Возьмём произвольную точку $ (x_0, y_0) \in \ol{A} $. Тогда прямоугольник
    $$ \ol{R} \define \set{(x, y) | {} |x - x_0| \le \delta, \quad | y - y_0| \le \delta} \quad \sub \ol{B}_{2\delta} $$
    Пусть $ M \define \max_{\ol{B}_{2\delta}}|f(x, y)| > 0 $ (при $ M = 0 $ лемма очевидна) \\
    Положим $ h \define \min\set{\delta, \faktor\delta{M}} $. Тогда $ P_h(x_0, y_0) = [x_0 - h, x_0 + h] $ -- отрезок Пеано, построенный для произвольной точки $ (x_0, y_0) \in \ol{A} $ \\
    Следовательно, по теореме Пеано решение \caupr{x_0, y_0} $ y = \vphi(x) $ определено на отрезке Пеано $ [x_0 - h, x_0 + h] $, длина которого неизменна для всех точек $ (x_0, y_0) \in \ol{A} $
    \begin{itemize}
    	\item Рассмотрим функцию $ \vphi(x) $ при $ x > x_0 $:
        \begin{itemize}
        	\item Если $ x_0 + h < b $, то $ \vphi_1(x_0 + h) \le \vphi(x_0 + h) \le \vphi_2(x_0 + h) $, а значит, точчка $ x_0 + h, \big( x_0 + h, \vphi(x_0 + h) \big) $ \\
            Выбрав эту точку в качетстве начальной, решение $ y - \vphi(x) $ можно продолжить вправо на полуотрезок Пеано $ [x_0 + h, x_0 - h] $
            \begin{itemize}
            	\item Если $ x_0 + 2h \ge b $, то лемма доказана
                \item Иначе сделаем очередное продолжение решения вправо на длину $ h $ \\
                В результате за конечное число шагов будет продолжено вправо до точки $ b $ включительно
            \end{itemize}
        \end{itemize}
        \item Аналогично $ y = \vphi(x) $ можно продолжить влево до точки $ a $
    \end{itemize}
\end{proof}

Для любой точки $ (x_0, y_0) \in \ol{A} $ обозначим через $ y = y(x, x_0, y_0) $ решение \caupr[\eref1]{x_0, y_0} \\
Тогда $ y(x_0, x_0, y_0) = y_0 $, и по лемме о поведении решений на компакте (лемма \ref{lm:comp}) решение $ y = (x, x_0, y_0) $ определено для всякого $ x \in [a, b] $ \\
Для произвольной точки $ \zeta \in [a, b] $ рассмотрим функцию
\begin{equ}{1.26}
    \vphi(x C) = y(x, \zeta, C), \qquad (\zeta, C) \in \ol{A}
\end{equ}
на прямоугольнике $ \ol{Q} = \ol{Q}_{\ol{A}} \define \set{(x, C) | a \le x \le b, \quad \vphi_1(\zeta) \le C \le \vphi_2(\zeta)} $, который является частным случаем множества $ Q_{A^*} $ из определения общего решения. \\
В самом деле, $ \vphi_1(\zeta) \le C \le \vphi_2(\zeta) $ по построению $ \ol{A} $. А по лемме решение $ y = y(x, \zeta, C) $ определено для любого $ x \in [a, b] $ и при $ x = \zeta $ по определению решения ЗК $ \vphi(\zeta, C) = y(\zeta, \zeta, C) = C $

\section{Теорема о существовании общего решения}

\begin{theorem}[о существовании общего решения]
    Введённая в формуле \eref{1.26} функция $ y = \vphi(x, C) $ является общим решением уравнения \eref1 на компакте $ \ol{A} $ из \eref{1.25}, построенном в окрестности произвольной точки из области единственности $ G^\circ $
\end{theorem}

\begin{proof}
    Покажем, что функция $ y = \vphi(x, C) $ удовлетворяет определению общего решения уравнения \eref1:
    \begin{enumerate}
        \item Возьмём произвольную точку $ (x_0, y_0) \in \ol{A} $ и рассмотрим уравнение $ y_0 = \vphi(x_0, C) $ или согласно \eref{1.26} уравнение
        \begin{equ}{1.27}
        	y_0 = y(x_0, \zeta, C)
        \end{equ}
        Наличие у него решения $ C = C_0 $ фактически означает, что ``выпущенное'' из точки $ (\zeta, C_0) \in \ol{A} $ решение уравнения \eref1 в момент $ x_0 $ попадает в точку $ (x_0, y_0) \in \ol{A} $ \\
        Покажем, что решение уравнения \eref{1.27} сущетсувует и единственно: \\
        ``Выпустим'' из точки $ (x_0, y_0) $ решение $ y = y(x, x_0, y_0) $, которое по лемме \ref{lm:comp} определено на всём отрезке $ [a, b] $ и, в частности, при $ x = \zeta \in [a, b] $ по определению \eref{1.26} \\
        Пусть $ C_0 = y(\zeta, x_0, y_0) $. Тогда $ (\zeta, C) $ -- это точка единственности, так как принадлежит графику решения $ y = y(x, x_0, y_0) $ \\
        Поэтому решение \caupr{\zeta, C} $ y = u(x, \zeta, C_0) $ с начальными данными $ \zeta, C_0 $ по лемме о поведении решений на компакте $ \ol{A} $ (лемма \ref{lm:comp}) продолжимо на $ [a, b] $ и совпадает с решением $ y = y(x, x_0, y_0) $ \\
        Следовательно, $ y_0 = y(x_0, \zeta, C) $, т. е. график функции $ y = y(x, \zeta, C_0) $ проходит через точку $ (x_0, y_0) $. Другими словами, дуга интегральной кривой, проходящая через точки $ (x_0, y_0) $, $ (\zeta, C_0) $, имеет на отрезке $ [a, b] $ две параметризации $ y = y(x, x_0, y_0) $ и $ y = (x, \zeta, C_0) $ \\
        Итак, установлено, что уравнение \eref{1.27} имеет единственное решение $ C = C_0 = y(\zeta, x_0, y_0) $, т. е. $ y_0 = y \big( x_0, \zeta, y(\zeta, x_0, y_0) \big) $
        \item Функция $ y = \vphi(x, C_0) $ является решением \caupr[\eref1]{x_0, y_0}, поскольку согласно \eref{1.26} и \eref{1.27} $ \vphi(x_0, C_0) = y(x_0, \zeta, C_0) = y_0 $
        \item Осталось доказать, что функция $ y = \vphi(x, C) $ из \eref{1.26} непрерывна на компакте $ \ol{Q} $ по совокупности переменных:
        \begin{itemize}
        	\item Поскольку для всякого $ C \in [\vphi_1(\zeta), \vphi_2(\zeta)] $ функция $ y = \vphi(x, C) $ -- это решение уравнения \eref1, она непрерывна по $ x $ при $ x \in [a, b] $
            \item Покажем, что для всякого $ x \in [a, b] $ функция $ y = \vphi(x, C) $ непрерывна по $ C $ при $ C \in [\vphi_1(\zeta), \vphi_2(\zeta)] $: \\
            Допуская \bt{противное}, предположим, что найдутся $ \vawe\veps > 0 $, $ \vawe{x} \in [a, b] $ и последовательность $ C_k \infarr{k} \vawe{C} $, $ C_k \in [\vphi_1(\zeta), \vphi_2(\zeta)] $ такие, что $ \big| \vphi(\vawe{x}, C_k) - \vphi(\vawe{x}, \vawe{C}) \big| \ge \vawe\veps $ при всех $ k \ge 1 $. Это значит, что при $ x = \vawe{x} $ функция $ \vphi(\vawe{x}, C) $ терпит разрыв в точке $ \vawe{C} \in [\vphi_1(\zeta), \vphi_2(\zeta)] $, поскольку любой компакт, в частности отрезок $ [\vphi_1(\zeta), \vphi_2(\zeta)] $, содержит все свои предельные точки. В этом случае, кстати, $ \vawe{x} \ne \zeta $, так как по определению $ \vphi(\zeta, C_k) = C_k \infarr{k} C = \vphi(\zeta, C) $ \\
            Выпуская из точек $ (\zeta, C_k) \in \ol{A} $ дуги интегральных кривых, получаем последовательность решений $ y = y(x, \zeta, C_k) = \vphi(x, C_k) $. Поскольку из любой сходящейся последовательности можно выдулить монотонную подпоследовательность, НУО считаем, что последовательность $ C_k $ монотонно возрастает, т. е. $ C_k < C_{k + 1} < \vawe{C} $ для любого $ k \ge 1 $ \\
            В области $ G^\circ $ интегральные кривые не имеют общих точек, поэтому последовательность $ \vphi(\vawe{x}, C_k) $ тоже монотонно возрастает и ограничена, так как $ \vphi(\vawe{x}, C_k) \le \vphi(\vawe{x}, \vawe{C}) - \vawe\veps $ по предположению. Но любая ограниченная монотонная последовательность имеет предел \\
            Пусть $ \vawe{y} = \limi{k} \vphi(\vawe{x}, C_k) $, тогда $ \vawe{y} \le \vphi(\vawe{x}, \vawe{C}) - \vawe\veps $ \\
            Выберем произвольную точку $ y^* $ из интервала $ \big( \vawe{y}, \vphi(\vawe{x}, \vawe{C}) \big) $ \\
            Рассмотрим определённое на $ [a, b] $ решение \caupr{\vawe{x}, y^*}, обозначаемое $ y = y(x, \vawe{x}, y^*) $ \\
            Пусть $ C^* = y(\zeta, \vawe{x}, y^*) $. Тогда $ C^* < \vawe{C} $, так как $ y^* < \vphi(\vawe{x}, \vawe{C}) = y(\vawe{y}, \zeta, \vawe{C}) $ \\
            Дугу интегральной кривой решения $ y = y(x, \vawe{x}, y^*) $ на $ [a, b] $, как было установлено, параметризует также решение с начальными данными $ \zeta, C^* $, имеющее согласно формуле \eref{1.26} вид $ y = \vphi(x, C^*) $, причём $ \vphi(\vawe{x}, C^*) = y^* $ \\
            Однако существует индекс $ k^* $ такой, что член $ C^{k*} $ сходящейся к $ \vawe{C} $ последовательности $ C_k $ будет больше, чем $ C^* $ \\
            В результате получилось так, что дуги интегральных кривых решений $ y = \vphi(x, C_{k*}) $ и $ y = \vphi(x, C^*) $ пересекаются в некоторой точке $ x^* $, лежащей между $ \zeta $ и $ \vawe{x} $, поскольку $ \vphi(\zeta, C_{k*}) = C_{k*} > C^* = \vphi(\zeta, C^*) $, а $ \vphi(\vawe{x}, C_{k*}) < \vawe{y} < y^* = y(\vawe{x}, \zeta, C^*) = \vphi(\vawe{x}, C^*) $ -- \contra с тем, что $ G $ -- область единственности
        \end{itemize}
        Итак, доказано, что функция $ y = \vphi(x, C) $ непрерывна по каждой из переменных в прямоугольнике $ \ol{Q} $. Но этого недостаточно для её непрерывности по совокупности переменных \\
        Воспользуемся ещё одним свойством функции $ \vphi $: \\
        Поскольку $ y = \vphi(x, C) $ при любой константе $ C \in [\vphi_1(\zeta), \vphi_2(\zeta)] $ есть решение уравнения \eref1, то $ \pder{\vphi(x, C)}x \equiv f \big( x, \vphi(x, C) \big) $ на $ [a, b] $ \\
        Но $ \big( x, \vphi(x, C) \big) \in \ol{A} $, когда точка $ (x, C) \in \ol{Q} $, а на компакте $ \ol{A} $ выполняется неравенство $ |f(x, y)| \le M $. Следовательно, функция $ \big| \pder{\vphi(x, C)}x \big| $ ограничена на $ [a, b] $ \\
        С учётом теоремы Лагранжа заключаем, что для любой константы $ C \in [\vphi_1(\zeta), \vphi(\zeta)] $ и для любых $ x_1, x_2 \in [a, b], ~ x_1 < x_2 $ найдётся такое $ x_C \in (x_1, x_2) $, что $ \vphi(x_2, C) - \vphi(x_1, C) = \pder{\vphi(x_C, C)}x(x_2 - x_1) $ \\
        Этого достаточно, чтобы непрерывность функции $ y = \vphi(x, C) $ по $ x $ на $ [a, b] $, равномерная относительно $ C \in [\vphi_1(\zeta), \vphi_2(\zeta)] $ в силу признака Вейрештрасса с $ \delta = \faktor\veps{M} $, стала очевидной \\
        Последнее свойство функции $ \vphi $ наряду с её поточечной непрерывностью по $ C $ гранатирует непрерывность $ \vphi(x, C) $ по совокупности переменных в прямоугольнике $ \ol{Q} $ \\
        Действительно, возьмём произвольную точку $ (x_0, C_0) \in \ol{Q} $ и покажем, что функция $ \vphi(x, C) $ непрерывна в этой точке: \\
        Для этого зафиксируем любое число $ \veps > 0 $. Тогда в силу непрерывности функции $ \vphi $ по $ C $ найдётся такое $ \delta_{x_0} > 0 $, что
        $$ \forall C \quad \nimp[\bigg(] |C - C_0| < \delta_{x_0} \implies |\vphi(x_0, C) - \vphi(x_0, C_0)| < \frac{\veps}2 \nimp[\bigg)] $$
        А из равномерной непреывности $ \vphi(x, C) $ по $ x $ относительно $ C $ вытекает, что
        $$ \exist \delta_0 > 0 : \quad \forall C \in [x\vphi_1(\zeta), \vphi_2(\zeta)] \quad \forall x \quad \nimp[\bigg(] |x - x_0| < \delta_0 \implies |\vphi(x, C) - \vphi(x_0, C)| < \frac{\veps}2 \nimp[\bigg)] $$
        Выберем число $ \delta \define \min\set{\delta_{x_0}, \delta_0} $, тогда для любой точки $ (x, C) $ получаем:
        $$ \norm{(x, C) - (x_0, C_0)} \define \max\set{|x - x_0|, |C - C_0|} < \delta $$
        Следовательно,
        $$ |\vphi(x, C) - \vphi(x_0, C_0)| \trile |\vphi(x, C) - \vphi(x_0, C)| + |\vphi(x_0, C) - \vphi(x_0, C_0)| = \veps $$
    \end{enumerate}
\end{proof}

\section{Формула общего решения, теорема о дифференцируемости общего решения}

\begin{definition}
    Общее решение $ y = \vphi(x, C) $, определённое формулой \eref{1.26}, будем называть общим решением в форме Коши или классическим общим решением уравнения первого порядка \eref1.
\end{definition}

\begin{theorem}[о дифференцируемости общего решения]
    Пусть на компакте $ \ol{A} $ из \eref{1.25} при некотором $ \zeta \in [a, b] $ формула \eref{1.26} задаёт общее решение $ y = \vphi(x, C) $, и в уравнении \eref1 $ f(x, y) $ непрерывно дифференцируема по $ y $ в некоторой окрестности $ \ol{A} $
    \begin{equ}{1.28}
        \implies \quad \forall (x, C) \in \ol{Q} : \quad \pder{\vphi(x, C)}x = \exp \bigg( \dint[t]\zeta{x}{\pder{f \big( t, \vphi(t, C) \big) }y} \bigg)
    \end{equ}
\end{theorem}

\begin{proof}
	Зафиксируем произвольным образом константу $ C \in [\vphi_1(\zeta), \vphi_2(\zeta)] $, после чего для всякого $ x \in [a, b] $ положим $ \Delta \vphi = \vphi(x, C + \Delta C) - \vphi(x, C) $, где $ \Delta C $ -- приращение аргумента $ C $ \\
    Поскольку при фиксированной $ C $ функция $ y = \vphi(x, C) $ является решением уравнения \eref1, справедлива цепочка равенств:
    \begin{multline*}
        \frac{\di(\Delta \vphi)}{\di x} = f \bigg( x, \vphi(x, C + \Delta C) \bigg) - f \bigg( x, \vphi(x, C) \bigg) = \dint[\bigg( f \big( x, \vphi(x, ) + \Delta \vphi \cdot s \big) \bigg)]01{} = \\
        = \dint[s]01{\frac{\di f \bigg( x, \vphi(x, C) + \Delta \vphi \cdot s \bigg)}{\di s}} = p(x, \Delta C) \Delta \vphi, \qquad p(x, \Delta C) \define \dint[s]01{\pder{f \bigg( x, \vphi(x, C) + \Delta \vphi \cdot s)}y}
    \end{multline*}
    \begin{itemize}
        \item Пусть $ \Delta C \ne 0 $, тогда, поделив первое и последнее выражение в цепочке на $ \Delta C $, убеждаемся, что функция $ \psi(x, \Delta C) \define \dfrac{\Delta \vphi}{\Delta C} $ является решением \caupr{\zeta, 1} линейного однородного уравнения $ \frac{\di u}{\di x} = p(x, \Delta C)u $, так как
        $$ \psi(\zeta, \Delta C) = \frac{\vphi(\zeta, C + \Delta C) - \vphi(\zeta, C)}{\Delta C} \undereq{\eref{1.26}} \frac{C + \Delta C - C}{\Delta C} = 1 $$
        Следовательно, $ \psi(x, \Delta C) = \exp \bigg( \dint[t]\zeta{x}{p(t, \Delta C)} \bigg) $
        \item Но $ p(x, \Delta C) $ существует и при $ \Delta C = 0 $:
        $$ p(x, 0) = \pder{f \bigg( x, \vphi(x, C) \bigg)}y $$
    \end{itemize}
    Поэтому
    $$ \pder{\vphi(x, C)}C = \limz{\Delta C} \psi(x, \Delta C) = \exp \bigg( \limz{\Delta C} \dint[t]\zeta{x}{p(t, \Delta C)} \bigg) $$
    В результате частная производная общего решения $ y = \vphi(x, C) $ по $ C $ существует, непрерывна и вычисляется по формуле \eref{1.28}
\end{proof}
