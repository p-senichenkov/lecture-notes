\part{Уравнения первого порядка в симметричной форме}

\begin{quote}
	\flushright
	Ааа! Симметричные дифуры!
\end{quote}

\begin{equ}{2.1}
	M(x, y)\di x + N(x, y) \di y = 0
\end{equ}

\section{Определение интеграла, теорема о характеристическом свойстве интеграла}

\begin{definition}
	Непрерывную в области $ B \sub \R^2 $ функцию $ U(x, y) $ будем называть допустимой, если для любой точки $ (x_0, y_0) \in B $ найдётся такая непрерывная функция $ y = \xi(x) $ или $ x = \eta(y) $, определённая на интервале $ (\alpha, \beta) $, содержащем точку $ x_0 $ или $ y_0 $, что:
    \begin{enumerate}
    	\item $ y_0 = \xi(x_0) $ или $ x_0 = \eta(y_0) $
        \item точка $ \big( x, \xi(x) \big) \in B $ для любого $ x \in (\alpha, \beta) $ или \\
        точка $ \big( \eta(y), y \big) \in B $ для любого $ y \in (\alpha, \beta) $
        \item $ y = \xi(x) $ или $ x = \eta(y) $ -- единственное решение уравнения
        \begin{equ}{2.6}
        	U(x, y) = U(x_0, y_0)
        \end{equ}
    \end{enumerate}
\end{definition}

\begin{remark}
	Условие 3 означает, что выполняется по крайней мере одно из тождеств:
    $$
    \begin{vars}
        U \big( x, \xi(x) \big) \overset{(\alpha, \beta)}\equiv U(x_0, y_0) \\
        U \big( \eta(y), y \big) \overset{(\alpha, \beta)}\equiv U(x_0, y_0)
    \end{vars} $$
\end{remark}

\begin{definition}
    Допустимая функция $ U(x, y) $ называется \it{интегралом} уравнения \eref{2.1} в области единственности $ B^0 $, если для любой точки $ (x_0, y_0) \in B^0 $ единственная функция $ y = \xi(x) $ или $ x = \eta(y) $ из определения допустимой функции --- это решение \caupr[\eref{2.1}]{x_0, y_0} на $ (\alpha, \beta) $.
\end{definition}

\begin{theorem}[о характеристическом свойстве интеграла]
    Для того чтобы допустимая функция $ U(x, y) $ была интегралом уравнения в симметричной форме \eref{2.1} в области единственности $ B^\circ $, \bt{необходимо и достаточно}, чтобы $ U(x, y) $ обращалась в постоянную вдоль любого решения \eref{2.1}, т. е. чтобы:
    \begin{itemize}
        \item $ U \big( x, \vphi(x) \big) \overset{\braket{a, b}}\equiv C $ для любого решения $ y = \vphi(x) $, определённого на $ \braket{a, b} $
        \item $ U \big( \psi(y), y \big) \overset{\braket{a, b}}\equiv C $ для любого решения $ x = \vphi(y) $, определённого на $ \braket{a, b} $
    \end{itemize}
\end{theorem}

\begin{iproof}
	\item Необходимость: \\
    Пусть $ U(x, y) $ -- интеграл уравнения \eref{2.1} в области единственности $ B^\circ $, и пусть, например, $ y = \vphi(x) $ -- какое-либо решение уравнения \eref{2.1}, определённое на промежутке $ \braket{a, b} $ \\
    НУО\footnotemark будем считать, что $ \braket{a, b} = (a, b) $ \\
    Возьмём произвольную точку $ x_0 \in (a, b) $ и положим $ y_0 \define \vphi(x_0) $ \\
    Точка $ (x_0, y_0) \in B^\circ $, поэтому по определению допустимой функции уравнение \eref{2.6} $ U(x, y) = U(x_0, y_0) $ однозначно разрешимо или относительно $ x $, или относительно $ y $:
    \begin{itemize}
        \item Пусть \eref{2.6} однозначно разрешимо относительно $ y $, т. е. существует такая единственная функция $ y = \xi(x) $, заданная на некотором $ (\alpha, \beta) \ni x_0 $, что $ U \big( x, \xi(x) \big) \overset{(\alpha, \beta)}\equiv U(x_0, y_0) $ \\
        Эта функция по опреелению интеграла является решением \caupr[\eref{2.1}]{x_0, y_0} \\
        Поскольку $ B^\circ $ -- область единственности, $ \vphi(x) \overset{(\vawe\alpha, \vawe\beta)}\equiv \xi(x) $, где $ (\vawe\alpha, \vawe\beta) = (a, b) \cap (\alpha, \beta) $. Следовательно,
        \begin{equ}{2.7}
            U \big( x, \vphi(x) \big) \overset{(\vawe\alpha, \vawe\beta)}\equiv U(x_0, y_0)
        \end{equ}
        \item Пусть \eref{2.6} однозначно разрешимо относительно $ x $, т. е. на некотором интервале $ (\alpha, \beta) \ni y_0 $ существует единственная функция $ x = \eta(y) $ такая, что $ \eta(y_0) = x_0 $ и $ U \big( \eta(y), y \big) \equiv U(x_0, y_0) $ на $ (\alpha, \beta) $ \\
        Тогда по определению интеграла $ x = \eta(y) $ на $ (\alpha, \beta) $ является решением \caupr[\eref{2.1}]{y_0, x_0}, а значит, единственное решение этой ЗК имеет два представления: $ y = \vphi(x) $ и $ x = \eta(y) $. Поэтому дуга интегральной кривой такого решения в некоторой окрестности точки $ (x_0, y_0) $, не имея вертикальных и горизонтальных касательных, может быть параметризована как функцией $ y = \vphi(x) $, так и функцией $ x = \eta(x) $ \\
        Иными словами, сущетвуют такие интервалы $ (\vawe{a}, \vawe{b}) $ и $ (\vawe\alpha, \vawe\beta) $, что
        $$ x_0 \in (\vawe{a}, \vawe{b}) \sub (a, b), \quad y_0 \in (\vawe\alpha, \vawe\beta) \sub (\alpha, \beta), \qquad y \overset{(\vawe\alpha, \vawe\beta)}\equiv \vphi \big( \eta(y) \big), \quad x \overset{(\vawe{a}, \vawe{b})}\equiv \eta \big( \vphi(x) \big) $$
        Поэтому справедлива доказывающая \eref{2.7} цепочка равенств:
        $$ U \big( x, \vphi(x) \big) \overset{)\vawe{a}, \vawe{b}}\equiv U \bigg( \eta \big( \vphi(x) \big), \vphi(x) \bigg) \overset{(\vawe\alpha, \vawe\beta)}\equiv U \big( \eta(y), y \big) \overset{(\vawe\alpha, \vawe\beta)}\equiv U(x_0, y_0) $$
        \item Осталось показать, что \eref{2.7} выполняется на всём интервале $ (a, b) $: \\
        \bt{Допустим}, что $ \vawe\beta < b $ и найдутся такие $ x_1, x_2 \in [\vawe\beta, b) $, ($ x_1 < x_2 $), что $ U \big( x, \vphi(x) \big) \overset{(\vawe\alpha, x_1]}\equiv U(x_0, y_0), \quad U \big( x, \vphi(x) \big) \neq U(x_0, y_0) $ для любого $ x \in (x_1, x_2) $ \\
        При $ y_1 = \vphi(x_1) $ в последнем тождестве $ U(x_1, y_1) = U(x_0, y_0) $. По определению решения точка $ (x_1, y_1) \in B^\circ $, поэтому для неё верны все рассуждения, касающщиеся точки $ (x_0, y_0) $ \\
        Пусть $ y = \xi_1(x) $ -- единственное на $ (\alpha_1, \beta_1), \quad \bigg( x_! \in (\alpha_1, \beta_1) \sub (x_0, x_2) \bigg) $ решение уравнения $ U(x, y) = U(x_1, y_1) $, т. е. $ U \big( x, \xi_1(x) \big) \equiv U(x_1, y_1) $ на $ (\alpha_1, \beta_1) $, и оно же по определению интеграла является единственным решением \caupr{x_1, y_1}. Тогда $ \xi_1(x) \equiv \vphi(x) $ на $ (\alpha_1, \beta_1) $, и $ U \big( x, \vphi(x) \big) \overset{[x_1, \beta_1)}\equiv U(x_1, y_1) = U(x_0, y_0) $ -- \contra \\
        Ситуация с точками $ x_1, x_2 \in (a, \vawe\alpha] $ рассматривается аналогично
        \item Достаточность: \\
        Пусть допустимая функция $ U(x, y) $ обращается в постоянную на любом решении уравнения \eref{2.1}. Покажем, что в таком случае $ U(x, y) $ -- интеграл этого уравнения в области едиснтвенности $ B^\circ $ \\
        Возьмём произвольную точку $ (x_0, y_0) \in B^\circ $. Тогда существует единственное решение \\ \caupr{x_0, y_0} вида $ y = \vphi(x) $ на $ (a ,b) \ni x_0 $, или $ x = \psi(y) $ на $ (a, b) \ni y_0 $ \\
        Пусть, например, $ x = \psi(y) $ является решением уравнения \eref{2.1}. Тогда по условию теоремы $ U \big( \psi(y), y \big) \equiv U(x_0, y_0) $ на $ (a, b) $ \\
        Если функция $ U(x, y) $, будучи допустимой, однозначно разрешима относительно $ x $, т. е. на некотором $ (\alpha, \beta) \ni y_0 $ существует и единственна функция $ x = \eta(y) $ такая, что $ U \big( \eta(y), y \big) \equiv U(x_0, y_0) $ на $ (\alpha, \beta) $, то $ \psi(y) \equiv \eta(y) $ на $ (a, b) \cap (\alpha, \beta) $. А если уравнение \eref{2.6} однозначно разрешимо относительно $ y $, то можно показать, как и при доказательстве необходимости, что функция $ y = \xi(x) $ -- решение уравнения \eref{2.1}, поскольку является обратной к решению $ x = \psi(y) $ \\
        В результате допустимая функция $ U(x, y) $ -- это интеграл уравнения \eref{2.1} в области единственности $ B^\circ $
    \end{itemize}
    \footnotetext{Действительно, если $ \braket{a, b} = [a, b] $, то по лемме о продолжимости решения, решение может быть продолжено на интервал $ (a_1, b_1) \supset [a, b] $}
\end{iproof}

\section{Определение гладкого интеграла, теорема о характеристическом свойстве гладкого интеграла}

\begin{definition}
	Гладкую функцию $ U(x, y) $ будем называть галдкой допустимой в области $ B $, если $ U_x'^2 + U_y'^2 > 0 $ для любой точки $ (x, y) \in B $
\end{definition}

\begin{definition}
    Интеграл $ U(x, y) $ уравнения \eref{2.1} будем называть гладким, если $ U $ -- гладкая допустимая функция
\end{definition}

\begin{theorem}[о характеристическом свойстве гладкого интеграла]
    Для того чтобы гладкая допустимая функция $ U(x, y) $ была гладким интегралом уравнения \eref{2.1} в области единственности $ B^\circ $, \bt{необходимо и достаточно}, чтобы выполнялось тождество
    \begin{equ}{2.8}
        N(x, y) U_x'(x, y) - M(x, y)U_y'(x,y) \overset{B^\circ}\equiv 0
    \end{equ}
\end{theorem}

\begin{iproof}
    \item Необходимость \\
    Пусть $ U(x, y) $ -- это гладкий интеграл уравнения \eref{2.1}. Возьём любую точку $ (x_0, y_0) \in B^\circ $ \\
    Тогда $ M^2(x_0, y_0) + N^2(x_0, y_0) \ne 0 $. Пусть, например, $ N(x_0, y_0) \ne 0 $ \\
    Тогда $ (x_0, y_0) \in B_N^\circ $, где $ B_N^\circ $ -- некая компонента связности открытого множества $ B^\circ \setminus \ol{N}_0 $, в которой $ N(x, y) \ne 0 $ и уравнение \eref{2.1} равносильно уравнению $ y' = -\frac{M(x, y)}{N(x, y)} $. \\
    Пусть $ y = \vphi(x) $ -- решение \caupr[\eref{2.1}]{x_0, y_0}, определённое на некотором интервале $ (a, b) \ni x_0 $ \\
    Тогда по определению решениия
    $$ \vphi'(x) \equiv -\frac{M \big( x, \vphi(x) \big)}{N \big( x, \vphi(x) \big)} \quad \text{ на } (a, b) $$
    По теореме о характеристическом свойстве интегала имеем:
    $$ U \big( x, \vphi(x) \big) \overset{(a, b)}\equiv U(x_0, y_0) $$
    Продиффиренцируем по $ x $:
    $$ U_x' \big( x, \vphi(x) \big) + U_y' \big( x, \vphi(x) \big) \vphi'(x) \overset{(a, b)}\equiv 0 $$
    Подставляя $ \vphi'(x) $ и домножая на $ N $, получаем:
    $$ N \big( x, \vphi(x) \big)U_x' \big( x, \vphi(x) \big) - M \big( x, \vphi(x) \big)U_y' \big( x, \vphi(x) \big) \overset{(a, b)}\equiv 0 $$
    Положим $ x = x_0 $, тогда $ \vphi(x_0) = y_0 $, и для любой точки $ (x_0, y_0) \in B^\circ $ получаем равенство \eref{2.8}
    \item Достаточность \\
    Пусть в $ B^\circ $ выполняется тождество \eref{2.8} \\
    Возьмём любую точку $ (x_0, y_0) \in B^\circ $, и пусть, например, $ U_y'(x_0, y_0) \ne 0 $ \\
    Тогда $ U_y'(x, y) \ne 0 $ в некоторой окрестности $ V(x_0, y_0) $ и в ней уравнение \eref{2.6} $ U(x, y) = U(x_0, y_0) $ однозначно разрешимо относительно $ y $, т. е. существует и единственна функция $ y = \xi(x) $, определённая на нектором интервале $ (\alpha, \beta) \ni x_0 $ такая, что $ \xi(x_0) = y_0, \quad \xi \in \Cont[1]{(\alpha, \beta)} $ и $ U \big( x, \xi(x) \big) \equiv U(x_0, y_0) $ на $ (\alpha, \beta) $ \\
    Дифференцируя последнее тождество, получаем
    $$ U_x' \big( x, \xi(x) \big) + U_y' \big( x, \xi(x) \big) \xi'(x) \overset{(\alpha, \beta)}\equiv 0, \qquad \big( x, \xi(x) \big) \in V $$
    а значит, $ \xi'(x) \equiv -\dfrac{U_x' \big(x, \xi(x) \big)}{U_y' \big( x, \xi(x) \big)} $ \\
    Покажем, что $ y = \xi(x) $ является решением уравнения \eref{2.1}, т. е. на интервале $ (a, b) $, например, удовлетоворяет тождеству $ 3_1 $ из определения решения. Подставляя $ \xi(x) $ в левую часть этого тождества, получаем:
    $$ M \big(x, \xi(x) \big) + N \big(x, \xi(x) \big) \xi'(x) \equiv \frac{M \big(x, \xi(x) \big)U_y' \big( x, \xi(x) \big) - N \big( x, \xi(x) \big)U_x' \big( x, \xi(x) \big)}{U_y' \big( x, \xi(x) \big)} \overset{\eref{2.8}}\equiv 0 $$
\end{iproof}

\begin{implication}
    Гладкая допустимая функция $ U(x, y) $ есть гладкий интеграл уравнения \eref1 $ y' = f(x, y) $ в области единственности $ G^\circ $ \bt{тогда и только тогда}, когда верно тождество
    $$ U_x'(x, y) + f(x, y)U_y'(x, y) \overset{G^\circ}\equiv 0 $$
\end{implication}

\section{Теоремы о существовании непрерывного интеграла и о связи между интегралами}

\begin{theorem}[о существовании непрерывного интеграла]
    Для любой точки $ (x_0, y_0) $ из области единственности $ B^\circ $ найдётся окрестность $ S \sub B^\circ $, в которой уравнение \eref{2.1} имеет интеграл $ U(x, y) $
\end{theorem}

\begin{proof}
    Пусть $ (x_0, y_0) $ --- это произвольная точка из области единственности $ B^\circ $ и, например, $ N(x_0, y_0) \ne 0 $. Тогда найдётся окрестность $ B_N^\circ $, в которой $ N(x, y) \ne 0 $, а значит, в ней уравнение в симметричной форме \eref{2.1} равносильно уравнению $ y' = - \frac{M(x, y)}{N(x, y)} $. \\
    Согласно теореме о существовании общего решения в области
    $$ A = \set{(x,y) | a < x < b, \quad \vphi_1(x) < y < \vphi_2(x)} \sub B_N^\circ $$
    существует общее решение $ y = \vphi(x, C) $. \\
    По определению общего решения уравнение $ y = \vphi(x, C) $ однозначно разрешимо относительно $ C $ для любой точки $ (x, y) \in A $, т. е. $ C = U(x, y) $, причём $ U \big( x, \vphi(x, C) \big) \overset{(a, b)}\equiv C $ \\
    В результате уравнение $ U(x, y) = C $ однозначно разрешимо относительно $ y $, а значит, функция $ U $ -- допустимая и постоянна вдоль любого решения, график которого лежит в области $ A $ \\
    По теореме о характеристическом свойстве интеграла функция $ U(x, y) $ является интегралом уравнения \eref{2.1} в области $ A $
\end{proof}

\section{Теоремы о существовании гладкого интеграла и о связи между интегралами}

\begin{theorem}[о существовании гладкого интеграла]
    \hfill \\
    В уравнении \eref{2.1} функции $ M(x, y), ~ N(x, y) \in \Cont[1]B $ \\
    Тогда для любой точки $ (x_0, y_0) $ из области $ B $ существует её окрестность $ A \sub B $, в которой уравнение \eref{2.1} имеет гладкий интеграл $ U(x, y) $
\end{theorem}

\begin{proof}
	По слабой теореме о единственности в области множество $ B $ является областью единственности \\
    Возьмём любую точку $ (x_0, y_0) $ из $ B $. И пусть, например, $ N(x_0, y_0) \ne 0 $, $ B_N $ --- окрестность $ (x_0, y_0) $, в которой $ N(x, y) \ne 0 $ и уравнение \eref{2.1} равносильно уравнению $ y' = f_*(x, y) $ с $ f_* \define - \frac{M(x, y)}{N(x, y)} $. При этом по условию теоремы в области $ B_N $ определена и непрерывна частная производная $ \pder{f_*(x, y)}y $ \\
    Пусть $ A \define \set{(x, y) | a < x < b, \quad \vphi_1(x) < y < \vphi_2(x)} $ --- окрестность точки $ (x_0, y_0) $, лежащая в $ B_N $ вместе со своим замыканием. По теореме о существовании общего решения в $ A $ существует общее решение $ y = \vphi(x, C) $ уравнения $ y' = f_*(x, y) $, задаваемое формулой \eref{1.26} $ \vphi(x, C) = y(x, \xi C) $, в которой $ \xi \in (a, b) $ выбирается произвольным образом, $ (\xi, C) \in \ol{A} $, т. е. $ C \in [\vphi_1(\xi), \vphi_2(\xi)] $, а $ y(x, \xi, C) $ --- решение \caupr{\xi, C} \\
    Положим $ \xi = x_0 $. Согласно \eref{1.28}
    $$ \pder{\vphi(x, C)}C = \exp \bigg( \dint[t]{x_0}x{\pder{f_* \big( t, \vphi(t, C) \big)}y} \bigg), \qquad \pder{\vphi(x_0, C)}C = 1 \quad \forall C \in [\vphi_1(x_0), \vphi_2(x_0)] $$
    Следовательно, по теореме о неявной функции уравнение $ \vphi(x, C) - y = 0 $ однозначно разрешимо относительно $ C $. Его решение $ C = U(x, y) $, как установлено в доказательстве теоремы о существовании непрерывного интеграла, является интегралом уравнения \eref{2.1} и непрерывно дифференцируемо по $ y $ в области $ A $. \\
    Остаётся заметить, что функция $ U(x, y) $ является также гладкой по $ x $, (\as обратная к ней $ y = \vphi(x, C) $ гладкая по определнию общего решения). \\
    Поэтому $ U(x, y) $ --- гладкая допустимая функция, а значит, и гладкий интеграл. \\
    Случай, когда $ N(x_0, y_0) = 0, ~ M(x_0, y_0) \ne 0 $ рассматривается аналогично.
\end{proof}

\begin{theorem}[о связи между интегралами]
    $ U(x, y) $ --- интеграл уравнения \eref{2.1} в некоторой области $ A $. \\
    Тогда:
    \begin{enumerate}
    	\item если $ U_1(x, y) $ --- ещё один интеграл в $ A $, то существует функция $ \Phi(x) $ такая, что $ U_1(x, y) \equiv[A] U(x, y) $;
        \item если функци $ \Phi \big( U(x, y) \big) $ допустима, то $ U_1(x, y) \equiv[A] \Phi \big( U(x, y) \big) $ --- это интеграл уравнения \eref{2.1} в области $ A $.
    \end{enumerate}
\end{theorem}

\begin{eproof}
	\item Пусть интеграл $ U(x, y) $ построен в области $ A $ при помощи общего решения $ \vphi(x, C) $. \\
    Тогда $ U \big( x, \vphi(x, C) \big) \equiv[(a, b)] C $. \\
	Поскольку $ U_1(x, y) $ --- тоже интеграл в $ A $, то
	$$ \forall C \in \R \quad U_1 \big( x, \vphi(x, C) \big) \equiv[(a, b)] \Phi \bigg( U \big( x, \vphi(x, C) \big) \bigg) $$
	Но точки $ \big(x, \vphi(x, C) \big) $ заполняют всю область $ A $, поэтому в $ A $ справедливо тождество $ U_1(x, y) \equiv \Phi \big( U(x, y) \big) $.

	\item Пусть $ \Phi $ --- произвольная вещественная функция такая, что функция $ \Phi \big( u(x, y) \big) $ допустима. \\
	Положим $ U_1(x, y) \define \Phi \big( U(x, y) \big) $. Тогда функция $ U_1 $ допустима и обращается в постоянную вдоль любого решения (\as по предположению, $ U $ --- это интеграл). Поэтому $ U_1 $ является интегралом.
\end{eproof}

\section{Теорема об интеграле уравнения с разделяющимися переменными}

\begin{definition}
    Уравнением с разделяющимися переменными в симметрической форме будем называть уравнение \eref{2.1} вида
    \begin{equ}{2.9}
    	g_1(x)h_2(y)\di x + g_2(x)h_1(y)\di y = 0
    \end{equ}
    в котором $ g_1(x), ~ g_2(x) \in \Cont{\braket{a, b}}, \quad h_1(y), ~ h_2(y) \in \Cont{\braket{c, d}} $, причём
    \begin{equ}{2.10_1}
    	(a, b) \setminus (g_1^\circ \cup g_2^\circ) = \bigcup_{k = 1}^{k_*}(a_k, b_k), \qquad (c, d) \setminus (h_1^\circ \cup h_2^\circ) = \bigcup_{l = 1}^{l_*}(c_l, d_l)
    \end{equ}
    \begin{equ}{2.10_2}
        \forall x \in (a, b) \quad g_1^2(x) + g_2^2(x) \ne 0, \qquad \forall y \in (c, d) \quad h_1^2(y) + h_2^2(y) \ne 0
    \end{equ}
    где $ g_i^\circ = \set{x \in \braket{a, b} \mid g_i(x) = 0}, \quad h_i^\circ = \set{y \in \braket{c, d} \mid h_i(y) = 0} $ --- замкнутые множества нулей функций $ g $ и $ h $
\end{definition}

Таким образом,
$$ M(x, y) = g_1(x)h_2(y) \in \Cont{\vawe R}, \qquad N(x, y) = g_2(x)h_2(y) \in \Cont{\vawe R} $$
где прямоугольник $ \vawe R = \set{(x, y) \mid x \in \braket{a, b}, \quad y \in \braket{c, d}} $ \\
Условие \eref{2.10_1} позволяет избежать ``экзотических'' ситуаций, типа канторовых множеств. \\
Условие \eref{2.10_2} означает, что $ \vawe R $ не пересекают ни горизонтальные, ни вертикальные прямые, состоящие из особых точек и ``разрезающие'' его на части. Только любой из четырёх отрезков, ограничивающих $ \vawe R $ может целиком состоять из особых точек. Рассмотрим
$$ H_i \define \set{(x, y) \mid x \in g_i^\circ, \quad y \in h_i^\circ}, \qquad i = 1, 2 $$
Тогда $ H_i $ может состоять из не более чем счётного объединения точек, отрезков и четырёхугольников. Кроме того, $ H_1 \cap H_2 $ может содержать только вершины $ \vawe R $. \\
В результате уравнение \eref{2.9} рассматриваем на множестве $ \vawe B = B \cup \hat B \cup \breve B $, в котором
$$ B = R \setminus ( H_1 \cup H_2), \qquad \breve B = (H_1 \cup H_2) \cap \partial B, \qquad \hat B = \partial B \setminus \breve B, \qquad R = \set{(x, y) \mid x \in (a, b), \quad y \in (c, d)} $$
Для любых $ x_2 \in g_2^\circ $ и $ y_2 \in h_2^\circ $ функции $ N(x_2, y) \equiv M(x, y_2) \equiv 0 $. Поэтому функции $ x(y) = x_2 $ при $ y \in (c, d) $ и $ y(x) = y_2 $ при $ x \in (a, b) $ удовлетворяют уравнению, являясь полными внутренними решениями соответственно на всех интервалах $ (c_l, d_l) \sub (c, d) \setminus g_2^\circ $ и $ (a_k, b_k) \sub (a, b) \setminus g_2^\circ $. \\
Остаётся решить уравнение в каждой из областей
$$ B_{kl} \define \set{(x, y) \mid x \in (a_k, b_k), \quad y \in (c_l, d_l)} \setminus (H_1 \cup H_2), \qquad \bigcup_{k, l \ge 1} B_{kl} \fed B $$
причём для любой точки $ (x, y) \in B_{kl} $ справедливы условия
\begin{equ}{2.11}
    g_2(x) \ne 0, \qquad h_2(y) \ne 0, \qquad g_1^2(x) + h_1^2(y) \ne 0
\end{equ}
Покажем, что любая область $ B_{kl} $ --- это область единственности: \\
Возьмём произвольную точку $ (x_k, y_l) \in B_{kl} $ и рассмотрим случай, когда $ h_1(y_l) \ne 0 $: \\
Существует интеграл $ (\vawe c, \vawe d) \sub c_l, d_l $ такой, что $ h_1(y) \ne 0 $ для всякого $ y \in (\vawe c, \vawe d) $. Поэтому в области
$$ G^\circ \define \set{(x, y) \mid x \in (a_k, b_k), \quad y \in (\vawe c, \vawe d)} $$
уравнение \eref{2.9} равносильно уравнению \eref1 вида
\begin{equ}{2.12}
	y' = g(x)h(y)
\end{equ}
в котором в данном случае $ g = -g_1(x)g_2^{-1}(x), \quad h = h_2(y)h_1^{-1}(y) \ne 0 $, и $ f(x, y) = g(x)h(y) $ непрерывна в прямоугольной области $ G^\circ $

\begin{definition}
    Уравнение \eref{2.12}, в котором $ g \in \Cont{(a_k, b_k)}, \quad h \in \Cont{(\vawe c, \vawe d)} $, называют уравнением с разделяющимися переменными, разрешённым относительно производной
\end{definition}

Покажем, что $ G^\circ $ --- область единственности для уравнения \eref{2.12}. Этого достаточно, чтобы произвольным образом выбранная точка $ (x_k, y_l) $ из $ B_{kl} $ оказаласть точкой единственности для уравнения \eref{2.9}. \\
Пусть $ H(y) \define \uint[y]{h^{-1}(y)} $, и, для определённости, функция $ h(y) > 0 $ при $ y \in (\vawe c, \vawe d) $. Тогда $ H(y) $ --- гладкая, строго возрастающая функция. \\
Сделаем в уравнени \eref{2.12} замену $ u \define H(y) $. Для этого продифференцируем тождество $ u(x) = H \big( y(x) \big) $ по $ x $ в силу уравнения \eref{2.12}, получая
$$ \frac{\di u(x)}{\di x} = \frac{|di H \big( y(x) \big)}{\di y} \cdot \frac{\di y(x)}{\di x} = h^{-1} \big( y(x) \big) \cdot g(x) \cdot h \big( y(x) \big) = g(x) $$
$$ u' = g(x) $$
Это уравнение определно в области
$$ G_u^\circ = \set{(x, y) \mid x \in (a, b), \quad u \in \big( H(\vawe c), H(\vawe d) \big)} $$
Его общее решение:
$$ u(x, C) = \uint{g(x)} + C $$
Область $ G_u^\circ $ является областью единственности для уравнения $ u' = g(x) $, так как интегральные кривые в ней не могут иметь общих точек. Они получены параллельными переносами одной и той же первообразной. А поскольку замена $ u = H(y) $ обратима, $ G^\circ $ оказывается областью единственности для уравнения \eref{2.12}. \\
В результате установлено, что $ B_{kl} $ --- область единственности для уравнения \eref{2.9}, и  в ней \eref{2.9} с учётом \eref{2.11} равносильно уравнению с разделёнными переменными:
\begin{equ}{2.13}
    \frac{g_1(x)}{g_2(x)}\di x + \frac{h_1(y)}{h_2(y)}\di y = 0
\end{equ}
Рассмотрим в любой области $ B_{kl} $ гладкую функцию
\begin{equ}{2.14}
    U(x, y) = \dint[s]{x_0}x{\frac{g_1(s)}{g_2(s)}} + \dint[s]{y_0}y{\frac{h_1(s)}{h_2(s)}}, \qquad x_0, y_0 \in B_{kl}
\end{equ}
Тогда
$$ U_x'(x, y) = \frac{g_1(x)}{g_2(x)}, \qquad U_y'(x, y) = \frac{h_1(y)}{h_2(y)} $$
$$ \underimp{\eref{2.11}} U_x'^2 + U_y'^2 \ne 0 $$
$ U $ --- гладкая допустимая функция и для неё, очевидно, выполняется тождество \eref{2.8}, а значит, по теореме о характеристическом свойстве гладкого интеграла функция $ U(x, y) $ является интегралом уравнения \eref{2.13}. \\
В результате, доказана следующая теорема:

\begin{theorem}[об интеграле уравнения с разделяющимися переменными]
    Любая область $ B_{kl} $ с учётом условий \eref{2.11} является областью единственности уравнения \eref{2.9}, и в ней функция $ U(x, y) $ является гладким интегралом уравнения \eref{2.9}
\end{theorem}

\section{Теорема об интеграле уравнения в полных дифференциалах; теорема об уравнении в полных дифференциалах, локальная}

\begin{definition}
    Уравнение \eref{2.1} называется уравнением в полных дифференциалах (УПД) в области $ B $, если существует функция $ U(x, y) \in \Cont[1]B $ такая, что для всякой точки $ (x, y) \in B $,
    \begin{equ}{2.15}
    	U_x'(x, y) = M(x, y), \qquad U_y'(x, y) = N(x, y)
    \end{equ}
\end{definition}

\begin{theorem}[об интеграле УПД]
    $ U(x, y) $ ~--- это гладкий интеграл УПД в $ B $
\end{theorem}

\begin{proof}
    Пусть существует гладкая функция $ U(x, y) $, для которой в $ B $ выполняются равенства \eref{2.15}. Тогда $ U_x'^2 + U_y'^2 \ne 0 $, а значит, по определению $ U $ --- гладкая допустимая функция. \\
    При этом, в $ B $ очевидым образом выполняется тождество \eref{2.8}, следовательно, по теореме о характеристическом свойстве гладкого интеграла функция $ U(x, y) $ явлется глдаким интегралом в $ B $. \\
    Остаётся показать, что $ B $ --- это область единственности. \\
    Возьмём произвольную точку $ (x_0, y_0) \in B $ и произвольное решение $ y = \vphi(x) $ \caupr[\eref{2.1}]{x_0, y_0} на каком-либо интервале $ (a, b) \ni x_0 $. Тогда $ \vphi(x_0) = y_0 $, и по определению решения
    $$ M \big( x, \vphi(x) \big) + N \big( x, \vphi(x) \big)\vphi'(x) = 0 \quad \forall x \in (a, b) $$
    $$ \implies \di U \big( x, \vphi(x) \big) = U_x' \big( x, \vphi(x) \big)\di x + U_y'(x, \vphi(x) \big)\di\vphi(x) = 0 $$
    $$ \implies U \big( x, \vphi(x) \big) \equiv[(a, b)] U \big( x_0, \vphi(x_0) \big) $$
    В результате любое решение поставленной ЗК\textsubscript{УПД} удовлетворяет уравнению \eref{2.6} в некоторой окрестности точки $ x_0 $. А функция $ U $, будучи допустимой, однозначно разершима, следоваетельно, в $ B $ не существует двух различных решений одной и той же ЗК.
\end{proof}

\begin{theorem}[об УПД; локальная]
    Предположим, что для уравнения \eref{2.1} выолняются условия:
    \begin{enumerate}
        \item прямоугольник $ R = \set{(x, y) \mid x \in (a, b), \quad y \in (c, d)} \sub B $;
        \item в $ B $ существуют и непрерывны частные производные $ M_y', N_y' $;
        \item верно тождество
        \begin{equ}{2.16}
        	M_y'(x, y) - N_x'(x, y) \equiv 0
        \end{equ}
        Тогда \eref{2.1} --- УПД в $ R $, и для любых $ x_0, x \in (a, b), \quad y_0, y \in (c, d) $ его интегралами являются функции
        \begin{equ}{2.17_1}
            U_1(x, y) = \dint[s]{x_0}x{M(s, y_0)} + \dint[s]{y_0}y{N(x, s)}
        \end{equ}
        $$ U_2(x, y) = \dint[s]{x_0}x{M(s, y)} + \dint[s]{y_0}y{N(x_0, s)} $$
    \end{enumerate}
\end{theorem}

\begin{proof}
    Возьмём, например, гладкую функцию $ U_1(x ,y) $ и покажем, что она удовлетворяет равенствам \eref{2.15} для любой точки $ (x, y) \in R $. Этого достаточно, чтобы \eref{2.1} было УПД в $ R $. \\
    Дифференцируя \eref{2.17_1} сначала по $ y $, а затем по $ x $, получаем:
    $$ \pder{U_1(x, y)}y = N(x, y), \qquad \pder{U_1(x, y)}x = M(x, y_0) + \dint[s]{y_0}y{\pder{N(x, s)}x} $$
    Теперь во втором равенстве испольуем тождество \eref{2.16}:
    $$ \pder{U_1(x, y)}x = M(x, y_0) + \dint[s]{y_0}y{\pder{M(x, s)}y} = M(x, y) $$
\end{proof}

\section{Теоремы о существовании и нахождении интегрирующего \tpst{\\}{} множителя, решение линейного уравнения при помощи интегрирующего множителя}

	\begin{definition}
    Функция $ \mu(x, y) $, определённая, непрерывная и не обращающаяся в ноль в области $ B $, называется интегрирующим множителем дифференциального уравнения \eref{2.1}, если уравнение
    \begin{equ}{2.18}
    	\mu(x, y) M(x, y) \di x + \mu(x, y)N(x, y)\di y = 0
    \end{equ}
    является УПД в $ B $.
\end{definition}

\begin{theorem}[о существовании интегрирующего множителя]
    Если в области единственности $ B^\circ \sub B $ уравнение \eref{2.1} имеет гладкий интеграл, тогда в $ B^\circ $ существует интегрирующий множитель.
\end{theorem}

\begin{proof}
    Пусть $ U(x, y) $ --- гладкий интеграл уравнения \eref{2.1} в области $ B^\circ $. \\
    Тогда из тождества \eref{2.8} вытекает, что в $ B^\circ $
    $$ \frac{U_x'(x ,y)}{M(x, y)} = \frac{U_y'(x, y)}{N(x, y)} $$
    причём числитель и значенатель в одной из частей равенства могут одновременно обращаться в ноль. \\
    Поэтому функция
    $$ \mu(x, y) \define \frac{U_x'(x, y)}{M(x, y)} = \frac{U_y'(x, y)}{N(x, y)} $$
    удовлетворяет определнию интегриующего множителя.
\end{proof}

Если \eref{2.18} --- УПД, то сголасно тождеству \eref{2.16} $ (\mu M)_y' - (\mu N)_x' = 0 $. \\
Перегруппируем:
\begin{equ}{2.19}
	\mu_x'N - \mu_y'M - (M_y' - N_x')\mu
\end{equ}

\begin{theorem}[о нахождении интегрирующего множителя]
    \hfill \\
	Пусть нашлась такая функция $ \omega(x, y) \in \Cont[1]B $, что
    \begin{equ}{2.20}
        \frac{M_y'(x, y) - N_x'(x, y)}{\omega_x'(x, y)N(x, y) - \omega_y'(x, y)M(x, y)} = \psi(\omega)
    \end{equ}
    Тогда уравнение \eref{2.1} имеет интегрирующий множитель $ \mu(\omega) = \exp \bigg( \uint[\omega]{\psi(\omega)} \bigg) $
\end{theorem}

\begin{proof}
	Будем искать $ \mu $ как функцию $ \omega $. \\
    В этом случае уравнение \eref{2.19} примет вид
    $$ \frac{\di\mu}{\di\omega}\omega_x'N - \frac{\di\mu}{\di\omega}\omega_y'M = (M_y' - N_x')\mu $$
    или с учётом предположения \eref{2.20}:
    $$ \frac{\di\mu(\omega)}{\di\omega} = \psi(\omega)\mu(\omega) $$
    Функция $ \mu(\omega) = C\exp \bigg( \uint[\omega]{\psi(\omega)} \bigg) $ является общим решением этого линейного однородного уравнения. Можно выбрать $ C = 1 $.
\end{proof}

\begin{definition}
	Уравнение, разрешённое относительно производной, вида
    \begin{equ}{2.21}
        y' + p(x)y = q(x), \qquad p(x), q(x) \in \Cont{(a, b)}
    \end{equ}
    называется линейным диффренциальным уравнением первого порядка.
\end{definition}

Найдём общее решение уравнения \eref{2.21} и решение \caupr{x_0, y_0}, используя интегрирующий множитель, для чего перепишем уравнение \eref{2.21} в симметричной форме:
\begin{equ}{2.22}
	\bigg( p(x)y - q(x) \bigg)\di x + \di y = 0
\end{equ}
Очевидно, что в $ G $ существуют и непрерывны $ M_y', N_x' $. \\
Будем искать $ \mu $ как функцию $ x $, т. е. $ \omega(x, y) = x $. \\
Тогда в формуле \eref{2.20} $ \psi(x) = p(x) $ и по теореме о нахождении интегрирующего множителя для любого $ x_0 \in (a, b) $ имеем:
$$ \mu(x) = e^{P(x)} \ne 0, \qquad P(x) \define \dint[t]{x_0}x{p(t)} $$
Умножая \eref{2.22} на $ \mu $, получаем УПД:
$$ e^{P(x)} \bigg( p(x)y - q(x) \bigg) \di x + e^{P(x)}\di y = 0 $$
При $ y_0 = 0 $ из \eref{2.17_1} находим
$$ U = -\dint[s]{x_0}x{e^{P(s)}q(s)} + \dint[s]0y{e^{P(x)}} $$
Это --- интеграл уравнения \eref{2.21}. \\
Тогда равенство
$$ e^{P(x)}y - \dint[s]{x_0}x{e^{P(s)}q(s)} = C $$
является общим интегралом уравнения \eref{2.22}. Отсюда
$$ y = \vphi(x, C) = e^{-P(x)} \bigg( C + \dint[s]{x_0}x{e^{P(s)}q(s)} \bigg) $$
является классическим общим решением линейного уравнения \eref{2.21}, а формула
$$ y = y(x, x_0, y_0) = \exp \bigg( -\dint[t]{x_0}x{p(t)} \bigg) \bigg\lgroup y_0 + \dint[s]{x_0}x{ \exp \bigg( \dint[t]{x_0}s{p(t)} \bigg)} \bigg\rgroup $$
задаёт решение \caupr{x_0, y_0}, определённое на $ (a, b) $ и называется формулой Коши.
