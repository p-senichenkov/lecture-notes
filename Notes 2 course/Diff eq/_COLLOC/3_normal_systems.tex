\part{Нормальные системы ОДУ}

\begin{quote}
	\flushright
	Ааа! Нормальные системы!
\end{quote}

\begin{equ}{3.1}
    \begin{cases}
        y_1' = f_1(x, y_1, \dots, y_n) \\
        \widedots \\
        y_n' = f_n(x, y_1, \dots, y_n)
    \end{cases}, \qquad f_1, \dots, f_n \in \Cont G, \qquad G \sub \R^{n + 1}
\end{equ}

\section{Лемма о связи между локальным и глобальным условиями Липшица, достаточные условия для выполнения локального условия Липшица}

\begin{definition}
	Функция $ f(x, y) $ удовлетворяет условию Липшица глобально по $ y $ на множестве $ B \sub G $, если найдётся такая константа $ L = L_B > 0 $, что
    \begin{equ}{3.8}
        \forall (x, \vawe y), (x, \hat y) \in B \quad \norm{f(x, \hat y) - f(x, \vawe y)} \le L \norm{\hat y - \vawe y}
    \end{equ}
\end{definition}

\begin{notation}
    $ f \in \operatorname{Lip}_y^{gl}(B) $
\end{notation}

\begin{definition}
    Функция $ f(x, y) $ удовлетворяет условию Липшица локально по $ y $ в области $ G $, если для любой точки $ (x_\circ, y^\circ) \in G $ существуют окрестность $ V(x_\circ, y^\circ) \sub G $ и константа Липшица $ L = L_V > 0 $ такие, что для любых двух точек $ (x, \vawe y), (x, \hat y) \in V(x_\circ, y^\circ) $ выолняется неравенство \eref{3.8}.
\end{definition}

\begin{notation}
    $ f \in \operatorname{Lip}_y^{loc}(G) $
\end{notation}

\begin{lemma}[о связи между локальным и глобальным условиями Липшица]\label{lm:Lip:gl_and_loc}
    Если $ f(x, y) \in \operatorname{Lip}_y^{loc}(G) $, то для любого компакта $ \ol H \sub G $ выполнено $ f(x, y) \in \operatorname{Lip}_y^{gl}(\ol H) $
\end{lemma}

\begin{proof}
    Рассуждая \bt{от противного}, допустим, что существует компакт $ \ol H \in G $, в котором $ f(x, y) \nin \operatorname{Lip}_y^{gl}(\ol H) $. \\
    Это значит, что найдутся такие последовательности точек $ (x_k, \vawe{y}^{(k)}), (x_k, \hat{y}^{(k)}) \in \ol H $ и костант $ L_k \infarr{k} \infty $, что
    \begin{equ}{3.9}
        \forall k \ge 1 \quad \norm{f(x_k, \hat{y}^{(k)}) - f(x_k, \vawe{y}^{(k)})} \ge L_k \norm{\hat{y}^{(k)} - \vawe{y}^{(k)}}
    \end{equ}
    Надо показать, что при каком-то $ k $ это неравенство нарушается. \\
    Разряжая при необходимости два раза подряд последовательность инексов $ k $ и пользуясь принципом выбора Больцано"--~Вейерштрасса, выберем такую подпоследовательность индексов $ k_l \infarr{l} \infty $, что $ (x_k, \vawe{y}^{(k_l)}) \to (x_\circ, \vawe{y}^{(\circ)}), \quad (x_{k_l}, \hat{y}^{(k_l)}) \to (x_\circ, \hat{y}^{(\circ)}) $. При этом обе точки $ (x_\circ, \vawe{y}^{(\circ)}), (x_\circ, \hat{y}^{(\circ)}) \in \ol H $, поскольку замкнутое множество содержит все свои предельные точки. \\
    В результате векторы $ \vawe{y}^{(0)} $ и $ \hat{y}^{(0)} $ либо совпадают, либо нет.
    \begin{itemize}
        \item $ \vawe{y}^{(0)} \ne \hat{y}^{(0)} $ \\
        Тогда можно ввести в рассмотрение функцию
        $$ h(x, \vawe y, \hat y) \define \frac{\norm{f(x, \hat y) - f(x, \vawe y)}}{\norm{\hat y - \vawe y}} $$
        определённую в некоторой окрестности точки $ (x_\circ, \vawe{y}^{(0)}, \hat{y}^{(0)}) $. \\
        Положим $ h(x_\circ, \vawe{y}^{(0)}, \hat{y}^{(0)}) \fed L_\circ $. Тогда существует окрестность $ V(x_0, \vawe{y}^{(0)}, \hat{y}^{(0)}) $, в которой $ h $ непрерывна и $ h(x, \vawe{y}, \hat y) < L_\circ + 1 $.
        $$ \implies \exist K > 0 : \quad \forall k_l > K \quad (x_{k_l}, \vawe{y}^{(k_l)}, \vawe{y}^{(k_l)}) \in V(x_\circ, \vawe{y}^{(0)}, \vawe{y}^{(0)}) $$
        а значит, $ h(x_{k_l}, \vawe{y}^{(k_l)}, \hat{y}^{(k_l)}) < L_\circ + 1 $, или
        $$ \norm{f(x_{k_l}, \hat{y}^{(k_l)}) - f(x_{k_l}, \vawe{y}^{(k_l)})} < (L_\circ + 1)\norm{\hat{y}^{(k_l)} - \vawe{y}^{(k_l)}} $$
        Однако это неравенство при $ l = l^* $ противоречит неравенству \eref{3.9}, поскольку всегда найдётся индекс $ l^* $ такой, что $ L_{k_{l^*}} > L_\circ + 1 $, \as $ L_{k_l} \infarr{l} +\infty $.
        \item $ y^{(0)} \define \vawe{y}^{(0)} = \hat{y}^{(0)} $ \\
        Тогда точка $ (x_\circ, y^{(0)}) \in \ol H \sub G $. В этом случае используем предположение о том, что функция $ f $ удовлетворяет локальному условию Липшица. \\
        По определению для точки $ (x_\circ, y^{(0)}) $ существуют лежащая в $ G $ окрестность $ V(x_\circ, y^{(0)}) $ и константа Липшица $ L > 0 $ такие, что для любых двух точек $ (x, \vawe y), (x, \hat y) \in V(x_\circ, y^{(0)}) $ верно неравенство \eref{3.8}. При этом обе подпоследовательности --- $ (x_{k_l}, \vawe{y}^{(k_l)}) $ и $ (x_{k_l}, \hat{y}^{(k_l)}) $ --- имеют общий предел --- точку $ (x_\circ, y^{(0)}) $. \\
        Поэтому найдётся такое число $ K > 0 $, что для всякого $ k_l > K $ точки $ (x_{k_l}, \vawe{y}^{(k_l)}) $ и $ (x_{k_l}, \hat{y}^{(k_L)}) \in V(x_\circ, y^{(0)}) $, а значит, выполняется неарвенство \eref{3.8}. Но существует такой индекс $ l^* $, что $ L_{k_{l^*}} > L $. Следовтельно, неравенства \eref{3.8} и \eref{3.9} несовместны при $ l = l^* $.
    \end{itemize}
\end{proof}

\begin{lemma}[о достаточном условии локальной липшицевости]
	Если вектор-функция $ f(x, y) $ непрерывна всесте со своими частными производными по $ y_1, \dots, y_n $ в области $ G $, то она удовлетворяет условию Липшица по $ y $ локально в $ G $.
\end{lemma}

\begin{proof}
	Пусть $ V $ --- окрестнгость произвольной точки из области $ G $. Очевидно, что её можно выбрать выпуклой по $ y $ и такой, что $ \ol V \sub G $. Для этого достаточно в качестве $ V $ взять куб с центром в выбранной точке и достаточно маленьким ребром. \\
    Покажем, что $ f(x, y) \in \operatorname{Lip}_y^{gl}(V) $: \\
    По формуле конечных приращений имеем:
    $$ \forall (x, \vawe y), (x, \hat y) \in V \quad f(x, \hat y) - f(x, \vawe y) = \sum_{j = 1}^nh^{(j)}(x, \vawe y, \hat y) \cdot (\hat{y}_j - \vawe{y}_j) $$
    где
    $$ h^{(j)} \define \dint[s]01{\pder{f \big( x, u(s) \big)}{y_j}}, \qquad u(s) \define \vawe y + s(\hat y - \vawe y) \quad \forall s \in [0, 1] $$
    При этом $ \big( x, u(s) \big) \in V $ в силу выпуклостти окрестности по $ y $. \\
    Поскольку чатсные производные $ f $ по $ y $ непрерывны в $ g $ и их конечное число, а компакт $ \ol V \sub G $ по построению, то
    $$ \exist M > 0 : \quad \forall s \in [0, 1] \quad \forall j = \ol{1, n} \quad \norm{\pder{f \big( x, u(s) \big)}{y_j}} \le M $$
    Поэтому
    \begin{multline*}
        \norm{f(x, \hat y) - f(x, \vawe y)} \le \sum_{j = 1}^n \norm{ \dint[s]01{\pder{f \big( x, u(s) \big)}{y_j}} \cdot (\hat{y}_j - \vawe{y}_j)} \le \sum_{j = 1}^n \dint[s]01{\norm{ \pder{f \big( x, u(s) \big)}{y_j}}} \cdot |\hat{y}_j - \vawe{y}_j| \le \\
        \le Mn \cdot \max\limits{j = \ol{1, n}}|\hat{y}_j - \vawe{y}_j| = nM\norm{\hat y - \vawe y}
    \end{multline*}
    и верно неравенство \eref{3.8} с глобальной константой Липшица $ L = nM $, обслуживающей окрестность $ V $ произвольной точки из области $ G $.
\end{proof}

\section{Теорема Пикара}

Введём $ (k + 1) $-е приближение по Пикару:

\begin{equ}{3.10}
    y^{(k + 1)}(x) = y^\circ + \dint[s]{x_0}x{f \big( s, y^{(k)}(s) \big)}.
\end{equ}

\begin{theorem}[Пикара]
    $ f(x, y) \in \Cont G, \quad f(x, y) \in \operatorname{Lip}_y^{loc}(G) $

    Для любой точки $ (x_\circ, y^\circ) \in G $ последовательные приближения Пикара $ y^{(k)}(x) $ ($ k = 0, 1, \dots) $ с начальными данными $ x_\circ, y^\circ $ определены на некотором отрезке $ [\alpha, \beta] $, причём существует такой компакт $ \ol H \sub G $, что для любых $ k \ge 0 $ и $ x \in [\alpha, \beta] $ точка $ \big( x, y^{(k)}(x) \big) \in \ol H $.

    Тогда функции $ y^{(k)}(x) $ равномерно относительно $ [\alpha, \beta] $ стремятся при $ k \to \infty $ к предельной функции $ y(x) $, являющейся решением \caupr[\eref{3.1}]{x_\circ, y^\circ} на отрезке $ [\alpha, \beta] $.
\end{theorem}

\begin{proof}
	Возьмём произвольную точку $ (x_\circ, y^\circ) \in G $

    По условию теоремы для этой точки надётся отрезок $ [\alpha, \beta] \ni x_\circ $ и компакт $ \ol H \sub G $ такие, что можно построить последовательные пикаровские приближения
    $$ y^{(k)}(x) = y^\circ + \dint[s]{x_\circ}x{ f \big( s, y^{(k - 1)}(s) \big)}, \qquad k = 1, 2, \dots, $$
    определённые для всякого $ x \in [\alpha, \beta] $ такие, что их графики, \ie точки $ \big( x, y^{(k)}(x) \big) $, при всех $ x $ и $ k $ принадлежат $ \ol H $.

    Наличие компакта позволяет ввести на нём две глобальные константы:
    \begin{itemize}
        \item Обозначим через $ L > 0 $ константу Липшица, обслуживающую $ \ol H $. Она существует по лемме о связи между условиями Липшица (лемма \ref{lm:Lip:gl_and_loc}), согласно которой $ f(x, y) \in \operatorname{Lip}_y^{gl}(\ol H) $.
        \item Положим $ M \define \max\limits_{\ol H}\norm{f(x, y)} $.
    \end{itemize}

    Нужно установить равномерную сходимость последовательности пикаровских отображений. Сделаем это при помощи функциональных рядов:

    Введём последовательность функций $ \vphi^{(k)}(x) $, определённых на отрезке $ [\alpha, \beta] $:
    $$ \vphi^{(0)}(x) \define y^{(0)}(x), \quad \vphi^{(1)}(x) \define y^{(1)}(x) - y^{(0)}(x), \quad \dots, \quad \vphi^{(k)}(x) \define y^{(k)}(x) - y^{(k - 1)}(x), \quad \dots $$

    Рассмотрим функциональный ряд
    $$ \vphi(x) = \sum_{k = 0}^\infty \vphi^{(k)}(x) $$

    По определению $ \vphi^{(k)} $,
    $$ S_n(x) = \sum_{k = 0}^n \vphi^{(k)}(x) = y^{(n)}(x) $$
    Поэтому сходимость ряда $ \vphi(x) $ равносильна сходимости последовательности пикаровских приближений $ y^{(k)}(x) $.

    Построим для ряда $ \vphi(x) $ мажорантный ряд, оценив сверху по норме методом \bt{индукции} члены $ \vphi^{(k)}(x) $:

    \begin{itemize}
        \item \bt{База.}

        Для всякого $ x \in [\alpha, \beta] $ имеем:
        $$ \norm{\vphi^{(0)}(x)} = \norm{y^{(0)}(x)}, $$
        $$ \norm{\vphi^{(1)}(x)} = \norm{y^{(1)}(x) - y^{(0)}(x)} = \norm{\dint[s]{x_\circ}x{f \big( s, y^{(0)}(s) \big)}} \le \bigg| \dint[s]{x_\circ}x{\norm{f \big( s, y^{(0)}(s) \big)}} \bigg| $$

        Но по условию теоремы любая точка $ \big( s, y^{(0)}(s) \big) $ лежит в $ \ol H $, \as $ [x_\circ \between x] \sub [\alpha, \beta] $. Следовательно,
        $$ \norm{y^{(1)}(x)} \le M|x - x_\circ|. $$
        Далее,
        \begin{multline*}
            \bm{\norm{\vphi^{(2)}(x)}} \le \bigg| \dint[s]{x_\circ}x{L \norm{y^{(1)}(s) - y^{(0)}(s)}} \bigg| = L \bigg| \dint[s]{x_\circ}x{\norm{\vphi^{(1)}(s)}} \bigg| \le \\
            \le L \bigg| \dint[s]{x\circ}x{M|s - x_\circ|} \bigg| \le LM \frac{|x - x_\circ|^2}2 = \bm{\frac ML \cdot \frac{(L|x - x_\circ|)^2}{2!}}
        \end{multline*}
        \item \bt{Предположим}, что для любых $ k \ge 2 $ и $ x \in [\alpha, \beta] $
        \begin{equ}{3.11}
            \norm{\vphi^{(k)}(x)} \le \frac ML \cdot \frac{(L|x - x_\circ|)^2}{2!}.
        \end{equ}
        \item \bt{Переход.} Оценим $ \vphi^{(k + 1)}(x) $:
        \begin{multline*}
            \norm{\vphi^{(k + 1)}(x)} = \norm{\nder[k + 1]y(x) - \nder[k]y(x)} = \norm{\dint[s]{x_\circ}x{f \big( s, \nder[k]y(s) \big)} - \dint[s]{x_\circ}x{f \big( s, \nder[k - 1]y(s) \big)}} \le \\
            \le \bigg| \dint[s]{x_\circ}x{f \big( s, \nder[k]y(s) \big) - f \big( s, \nder[k - 1]y(s) \big) } \bigg|.
        \end{multline*}

        Поскольку аргументы $ f \in \ol H $, используем для оценок глобальное условие Липшица:
        \begin{multline*}
            \norm{\nder[k + 1]\vphi(x)} \underset{\operatorname{Lip}}\le \bigg| \dint[s]{x_\circ}x{L \norm{\nder[k]y(s) - \nder[k - 1]y(s)}} \bigg| = L \bigg| \dint[s]{x_\circ}x{\norm{\nder[k]\vphi(s)}} \bigg| \underset{\bt{предп.}}\le \\
            \le L \bigg| \dint[s]{x_\circ}x{\frac MN \cdot \frac{(L|s - x_\circ|)^k}{k!}} \bigg| \le \frac MN \cdot \frac{(L|x - x_\circ|^{k + 1})}{(k + 1)!}
        \end{multline*}

        Таким образом, \bt{индукцонное предположение} доказано.
    \end{itemize}

    Поскольку $ |x - x_\circ| \le \beta - \alpha $, справедлива равномерная оценка членов ряда $ \vphi(x) $:
    $$ \norm{\nder[k]\vphi(x)} \le \frac MN \cdot \frac{ \big( L(\beta - \alpha) \big)^k}{k!} \quad \forall x \in [\alpha, \beta] $$

    Мажорантный для $ \vphi(x) $ числовой ряд
    $$ \norm{y^\circ} + \frac ML \cdot \sum_{k = 1}^\infty \frac{\big( L(\beta - \alpha) \big)^k}{k!} $$
    сходится при любых конечных $ \alpha, \beta $.

    По признаку Вейерштрасса функциональный ряд $ \sum \nder[k]\vphi(x) $ сходится равномерно на $ [\alpha, \beta] $, а значит, последовательноть $ \nder[k]y \uniarr[k \to \infty]{[\alpha, \beta]} y(x) $.

    Для всякого $ x \in [\alpha, \beta] $ предельная функция $ y(x) $ непрерывна по теореме Стокса"--~Зайделя и точка $ \big( x, y(x) \big) $, являясь предельной, содержится в $ \ol H $. Следовательно, $ \dint[s]{x_\circ}x{f \big( s, y(s) \big)} $ существует.

    Рассмотрим равенство \eref{3.10}, устремив в нём $ k $ к бесконечности. Тогда слева получим $ y(x) $, а справа
    $$ \dint[s]{x_\circ}x{f \big( s, \nder[k]y(s) \big)} \to \dint[s]{x_\circ}x{f \big( s, y(s) \big)}, $$
    \ie возможен переход к пределу под знаком интеграла.

    Таким образом, в правой части \eref{3.10} тоже можно перейти к пределу, получая формулу
    $$ y(x) = y^\circ + \dint[s]{x_\circ}x{f \big( s, y(s) \big)} \quad \forall x \in [\alpha, \beta], $$
    \ie $ y(x) $ удовлетворяет интегральному уравнению, что равносильно тому, что $ y(x) \in $ \nimp[(является решением)] \caupr[\eref{3.1}]{x_\circ, y^\circ} на отрезке $ [\alpha, \beta] $.
\end{proof}

\section{Теорема о существовании и единственности решений нормальной системы}

\begin{theorem}[о существовании и единственности решения]
    Пусть в системе \eref{3.1} $ f(x, y) $ непрерывна и $ f \in \operatorname{Lip}_y^{loc}(G) $.

    Тогда для любой точки $ (x_0, y^0) \in G $ и для любого отрезка Пеано $ P_h(x_0, y^0) $ на этом отрезке существует и единственно решение \caupr{x_0, y^0}.
\end{theorem}

\begin{iproof}
	\item Существование.

    Возьмём любую точку $ (x_0, y^0) \in G $ и найдём для неё отрезок $ [\alpha, \beta] $ и компакт $ \ol H $ из теоремы Пикара.

    Сначала построим отрезок Пеано с центром в т. $ x_0 $. Для этого возьмём такие $ a, b > 0 $, что компакт $ \ol R = \set{(x, y) \mid |x - x_0| < a, ~ \norm{y - y^0} \le b} \sub G $.

    Положим
    $$ M = \max\limits_{(x, y) \in \ol R} \norm{f(x, y)}, \quad h = \min \set{a, \frac b M}, \alpha = x_0 - h, \quad \beta = x_0 + h $$
    Тогда $ [\alpha, \beta] $ --- это искомый отрезок Пеано $ P_h(x_0, y^0) $.

    Выберем $ \ol H = \set{(x, y) | \alpha \le x \le \beta, ~ \norm{y - y^0} \le b} $. Тогда $ \ol H \sub \ol R $.

    Докажем \bt{индукцией} по $ k = 0, 1, \dots $, что
    \begin{equ}{3.13}
        \forall x \in [\alpha, \beta] \quad \norm{\nder[k]y(x) - y^0} \le b
    \end{equ}
    Тогда точка $ \big( x, \nder[k]y(x) \big) $ попадёт в компакт $ \ol H $, что позволит определить пикаровское приближение $ \nder[k + 1]y $ на всём отрезке Пеано $ [\alpha, \beta] $.

    \begin{itemize}
        \item По определению, $ \nder[0](x) \equiv y^0 $, поэтому \bt{база} очевидна.
        \item Допустим, что неравенство \eref{3.13} верно. Тогда для любого $ x \in [\alpha, \beta] $
        $$ \norm{\nder[k + 1]y(x) - y^0} = \norm{\dint[s]{x_0}x{f \big( s, \nder[k]y(s) \big)}} \le \bigg| \dint[s]{x_0}x{ \norm{f \big( s, \nder[k]y(s) \big)}} \bigg| $$
        Но согласно \eref{3.13} точка $ \big( s, \nder[k]y(s) \big) \in \ol H \sub \ol R $, поэтому под знаком интеграла $ \norm f \le M $ и $ \norm{ \nder[k + 1]y(x) - y^0} \le M|x - x_0| \le Mh \le b $.
    \end{itemize}

    \item Единственность \\
    Докажем \bt{от противного}. \\
    Предположим, что существует ещё одно решение $ \vawe y(x) $ с теми же начальными данными, \ie $ \vawe y(x_0) = y^0 $, определённое на некотором интервале $ (\vawe \alpha, \vawe \beta) \ni x_0 $.

    Пусть $ [a, b] $ --- отрезок, на котором определены оба решения. Достаточно показать, что на $ (a, b) $ решения $ y(x) $ и $ \vawe y(x) $ совпадают.

    Используя интегральную формулу для любого $ x \in (a, b) $ запишем разность этих решений:
    $$ y(x) - \vawe y(x) = \dint[s]{x_0}x{ \bigg( f \big( s, y(s) \big) - f \big( s, \vawe y(s) \big) \bigg)} $$
    При этом, существует такой компакт $ \ol H \sub G $, что для всякого $ s \in [a, b] $ точки $ \big( s, y(s) \big), ~ \big( s, \vawe y(s) \big) \in \ol H $.

    По условию теоремы в области $ G $ для функции $ f(x, y) $ выполняется локальное условие Липшица. А значит, по лемме о связи между локальным и глобальным условиями Липшица функция $ f \in \operatorname{Lip}_y^{gl}(\ol H) $ и $ L $ --- глобальная константа Липшица. Поэтому
    $$ \norm{y(x) - \vawe y(x)} \le \bigg| \dint[s]{x_0}x{\norm{ f \big( s, y(s) \big) - f \big( s, \vawe y(s) \big)}} \bigg| \le L \bigg| \dint[s]{x_0}x{\norm{ y(s) - \vawe y(s)}} \bigg| $$

    Применяя следствие из теоремы Гронуолла с $ \mu = L $ заключаем, что $ \norm{ y(x) - \vawe y(x)} \equiv[(a, b)] 0 $. Тогда $ y(x) - \vawe y(x) \equiv[(a, b)] 0 $.
\end{iproof}

\begin{implication}
	$ G $ является областью единственности.
\end{implication}

\section{Линейные системы, теоремы о существовании, единственности и продолжимости решений линейных систем}

\begin{definition}
    Система \eref{3.1} называется \it{линейной}, если она имеет вид
    \begin{equ}{3.14}
        \begin{cases}
            y_1' = p_{11}(x)y_1 + \dots + p_{1n}(x)y_n + q_1(x) \\
            \dots \\
            y_n' = p_{n1}(x)y_1 + \dots + p_{nn}(x)y_n + q_n(x)
        \end{cases}
    \end{equ}
    или в векторной записи
    $$ y' = P(x)y + q(x) $$
    где функции $ p_{ij}(x) $ и $ q_i(x) \in \Cont{(a, b)} $.
\end{definition}

\begin{restate}
	Нормальная система является линейной, если $ f(x, y) = P(x)y + q(x) $, а $ G = (a, b) \times \R^n $.
\end{restate}

\begin{theorem}[о существовании и единственности решений линейных систем]
    Для любой точки $ x_0 \in (a, b) $, для любого вектора $ y^0 \in \R^n $ и для любого отрезка Пеано $ P_h(x_0, y^) $ существует и единственно решение \caupr[\eref{3.14}]{x_0, y^0}, определённое на $ P_h(x_0, y^0) $.
\end{theorem}

\begin{proof}
    Поскольку функция $ f(x, y) \in \Cont G $ и $ f_y'(x, y) = P(x) \in \Cont G $, а значит, $ f \in \operatorname{Lip}_y^{loc}(G) $, к системе \eref{3.14} применима предыдущая теорема.
\end{proof}

\begin{definition}
    Система \eref{3.1} называется \it{почти линейной}, если $ f(x, y) \in \mc C(G) $, где область $ G = (a, b) \times \R^n $, и существуют непрерывные и неотрицательные на $ (a, b) $ функции $ L(x), M(x) $ такие, что $ \| f(x, y) \| \le L(x) + M(x) \| y \| $ для любой точки $ (x, y) \in G $.
\end{definition}

\begin{theorem}[о продолжимости решений почти линейных систем]
    \hfill \\
	Любое решение почти линейной системы продолжимо на интервал $ (a, b) $.
\end{theorem}

\begin{proof}
    Рассмотрим произвольное решение почти линейной системы $ y = \vphi(x) $, заданное на максимальном интервале существования $ (\alpha, \beta) $. Для всякого $ x_0 \in (\alpha, \beta) $ по интегральной формуле,
    $$ \vphi(x) \equiv[(\alpha, \beta)] \vphi(x_0) + \dint[s]{x_0}x{f \big( s, \vphi(s) \big)} $$
    $$ \implies \norm{\vphi(x_0)} + \bigg| \dint[s]{x_0}x{\norm{f \big( s, \vphi(s) \big)}} \bigg| < \norm{\vphi(x_0)} + \bigg| \dint[s]{x_0}x{\bigg( L(s) + M(s) \norm{\vphi(s)} \bigg)} \bigg| $$
    Если $ \beta < b $, то отрезок $ [x_0, \beta] \sub (a, b) $, и в силу непрерывности функций $ L $ и $ M $ имеем:
    $$ L(x) \le L_0, \quad M(x) \le M_0 \qquad \forall x \in [x_0, \beta] $$
    Поэтому
    $$ \norm{\vphi(x)} \le \norm{\vphi(x_0)} + L_0(\beta - x_0) + M_0 \bigg| \dint[s]{x_0}x{\norm{\vphi(s)}} \bigg| $$
    По лемме Гронуолла
    $$ \norm{\vphi(x)} \le \bigg( \norm{\vphi(x_0)} + L_0(\beta - x_0) \bigg)e^{M_0(\beta - x_0)} \quad \forall x \in [x_0, \beta], $$
    что \bt{противоречит} теореме о поведении интегральной кривой полного решения.

    Аналогично рассматривается случай, когда $ \alpha > a $.
\end{proof}

\begin{theorem}[о продолжимости решений линейных систем]
    Любое решение линейной системы \eref{3.14} продолжимо на интервал $ (a, b) $.
\end{theorem}

\begin{proof}
	Покажем, что линейная система является почти линейной. Положим
    $$ p_0(x) \define \max\limits_{i, j = \ol{1, n}} \set{|p_{ij}(x)|}, \qquad q_0 \define \max\limits_{i = \ol{1, n}} \set{|q_i(x)|} $$
    Тогда функции $ p_0(x), q_0(x) \in \Cont{a, b} $.

    Оценим сверху компоненты правой части системы \eref{3.14}:
    \begin{multline*}
        |f_i(x, y)| = |p_{i1}(x)y_1 + \dots + p_{in}(x)y_n + q_i(x)| \le \sum_{j = 1}^n |p_{ij}(x)| \cdot |y_j| + |q_i(x)| \le \\
        \le \sum_{j = 1}^n p_0(x) |y_j| + q_0(x) \le np_0(x) \max\limits_{j = \ol{1, n}}|y_j| + q_0(x)
    \end{multline*}
    По определению нормы $ \norm{f(x, y)} \le np_0(x)\norm y + q_0(x) $, \ie система \eref{3.14} почти линейна.
\end{proof}

\section{Малые возмущения начальных данных по параметру, рассуждение о сдвиге}

Рассматриваем нормальную систему, зависящую от векторного параметра $ \mu = (\mu_1, \dots, \mu_n) $:
\begin{equ}{3.15}
	y' = f(x, y, \mu),
\end{equ}
где вещественная функция $ f(x, y, \mu) $ непрерывна и $ f \in \operatorname{Lip}_y^{loc} $ в некоторой области $ F \sub \R^{1 + n + m} $.

Фактически, система \eref{3.15} представляет собой семейство систем, каждая из которых отвечает своему значению вектора $ \mu $. \\
Пусть функция $ y = y(x, x_0, y^0, \mu), \quad y(x_0, x_0, y^0, \mu) = y^0 $ обозначает решение $ \text{ЗК}_{\eref{3.15}} $, заданное на множестве
$$ D = \set{(x, x_0, y^0, \mu) \mid x \in I(x_0, y^0, \mu), \quad (x_0, y^0, \mu) \in F}, $$
где $ I $~--- максимальный интервал существования решения.

Особое место среди систем занимает \it{порождающая} (\it{невозмущённая}) система
\begin{equ}{3.15^}
	y' = f(x, y, \hat \mu),
\end{equ}
в которой $ \mu $ --- числовой вектор \it{расчётных} значений параметров. \eref{3.15} можно трактовать как \it{возмущённую} систему.

Зафиксируем \it{расчётные} значения начальных данных $ x_0 = \hat x_0, ~ y^0 = \hat y^0 $ так, чтобы точка $ (\hat x_0, \hat y^0, \hat \mu) \in F $.

Рассмотрим решение $ \text{ЗК}_{\eref{3.15^}} $, определённое на $ (\alpha, \beta) $ $ \vphi(x) = y(x, \hat x_0, \hat y^0, \hat \mu), \quad \vphi(\hat x_0) = \hat y^0 $, и выберем произвольный отрезок $ [a, b] : \hat x_0 \in [a, b] \sub (\alpha, \beta) $. \\
Решение $ y = \vphi(x) $ будем также называть \it{расчётным}.

Задача заключается в том, чтобы установить продолжимость решения $ y(x, x_0, y^0, \mu) $ на $ [a, b] $ и наличие его непрерывной зависимости от начальных данных и вектора параметров.

Введём следующие обозначения:
\begin{equ}{3.16}
    \ol U_d^{x, y} \define \set{(x, y, \mu) \mid x \in [a, b], \quad \| y - \phi (x) \| \le d, \quad \| \mu - \hat \mu \| \le d},
\end{equ}
это замкнутая трубчатая окрестность ``радиуса'' $ d > 0 $ графика функции $ y = \vphi(x) $ на отрезке $ [a, b] $.

\begin{statement}
    Существует такое $ \sigma > 0 $, что компакт $ \ol U_\sigma^{x, y} \sub F $.
\end{statement}

\begin{proof}
	Пусть $ \Gamma $ --- график функции $ y = \vphi(x) $ при $ \mu = \hat \mu $ и $ x \in [a, b] $. Тогда, по четвёртой аксиоме отделимости, $ \exist U : \ol \Gamma \sub U \sub \ol U \sub F $. \\
    Отсюда $ \ol \Gamma \cap \partial U = \O $.
\end{proof}

Будем рассматривать ситуацию, когда $ y^0 $ зависит от $ \mu $, а $ x_0 $ --- нет. \\
Итак, будем исследовать решение ЗК
$$ y(x, \mu) = y \big( x, x_0, y^0(\mu), \mu \big), \quad y(x_0, \mu) = y^0(\mu) $$
системы \eref{3.15} при $ \big( x_0, y^0,(\mu), \mu) \in U_\delta^{x_0, y^0(\mu)}(\vphi, \hat \mu) $, где $ \delta < \sigma $. \\
Будем предполагать, что
\begin{equ}{3.17}
	y^0(\mu) = y^0 + \psi(\mu), \quad \psi(\hat \mu) = 0,
\end{equ}
где функция $ \psi $ непрерывна в \it{поликруге} $ U_\sigma = \set{ \mu \mid \| \mu - \hat \mu \| < \sigma} $.

Таким образом, $ \psi(\mu) $ является малым возмущением ветора начальных данных $ y^0 $ решения $ y(x, x_0, y^0, \hat \mu) $ порождающей системы.

\begin{undefthm}{Рассуждение о сдвиге}
    \begin{itemize}
    	\item $ \to $

        Замена $ y = u + \psi(\mu) $ сводит систему \eref{3.15} к системе
        \begin{equ}{3.15*}
            u' = h(x, u, \mu),
        \end{equ}
        с $ h(x, u, \mu) = f(x, u + \psi(\mu), \mu) $, в которой решением являвется функция $ u(x, \mu) = y(x, x_0, y^0(\mu), \mu) - \psi(\mu) $. При этом $ u(x_0, \mu) = y^0(\mu) - \psi(\mu) = y^0 $. Поэтому $ u(x, \mu) = u(x, x_0, u^0, \mu) $, где $ u^0 = y^0 $.

        Поскольку $ \psi(\hat \mu) = 0 $, совпадают как порождающие системы, так и их решения:
        $$ y(x, x_0, y^0, \hat \mu) = u(x, x_0, y^0, \hat \mu) $$

        Обозначим через $ F_* $ область, в которой правая часть системы \eref{3.15*} удовлетворяет локальному условию Липшица по $ u $. Тогда при $ \mu = \hat \mu $ области $ G_{\hat \mu}^* = \set{ (x, y) \mid (x, u, \hat \mu) \in F_*} $ и $ G_{\hat \mu} $

        Зафиксируем любое $ 0 < \sigma_* < \frac\sigma2 $, при котором $ \ol U_{\sigma_*}^{x, y}(\vphi, \hat \mu) \sub F_* $, и с учётом того, что $ \psi(\hat \mu) = 0 $, справедливо неравенство $ \max\limits_{\mu : \norm{\mu - \hat \mu} \le \sigma_*} \norm{\psi(u)} \le \frac\sigma2 $. совпадают.

        Значит, если точка $ (x, u, \mu) \in \ol U_{\sigma_*}^{x, u}(\vphi, \hat \mu) $, то точка $ (x, y, \mu) \in \ol U_\sigma^{x, y}(\vphi, \hat \mu) $, где $ u = u + \psi(\mu) $. В самом деле,
        $$ \norm{ u - \vphi(x) } \le \sigma_* \iff \norm{y - \vphi(\mu) - \vphi(x)} \le \sigma_* \quad \implies \quad \norm{y - \vphi(x) \le \sigma_*} + \norm{\vphi(\mu)} \le \frac\sigma2 + \frac\sigma2 = \sigma $$

        \item $ \circ $

        Допустим, уставновлено, что имеется такое $ 0 < \delta_* < \frac{\sigma_*}2 $, что для любой точки $ (x_0, u^0, \mu) \in U_{\delta_*}^{x_0, y^0}(\vphi, \hat \mu) $ решение $ u(x, x_0, u^0, \mu) $ системы \eref{3.15*} определено при всех $ x \in [a, b] $ и обладает рядом свойств, зависящим от предположений относительно правой части системы, на множестве $ V_{\delta_*} = [a, b] \times U_{\delta_*}^{x_0, u^0}(\vphi, \hat \mu) $, при этом для любого $ x \in [a, b] $ точка $ \big( x, u(x, x_0, u^0, \mu) \mu \big) \in \hat U_{\sigma_*}^{x, u}(\vphi, \hat \mu) $.

        \item $ \leftarrow $

        Покажем, что аналогичными свойствами будет обладать решение $ y \big( x, x_0, y^0(\mu) \mu \big) $ системы \eref{3.15}. Для этого зафиксируем $ 0 < \delta < \frac{\delta_*}2 $, при котором $ U_\delta^{x_0, y^0(\mu)}(\vphi, \mu) \sub F $ и $ \max\limits_{\mu : \norm{\mu - \hat \mu} \le \delta} \norm{\psi(\mu)} \le \frac{\delta_*}2 $.

        Тогда если $ (x_0, y^0(\mu), \mu) \in U_\delta^{x_0, y^0(\mu)}(\vphi, \hat \mu) $, то$ (x_0, u^0, \mu) \in U_{\delta_*}^{x_0, u^0}(\vphi, \hat \mu) $, где $ u^0 = y^0 $. \\
        Действительно,
        $$ \norm{y^0(\mu) - \vphi(x_0)} < \delta \iff \norm{u^0 + \psi(\mu) - \vphi(x_0)} < \delta \implies \norm{u^0 - \vphi(x_0)} < \delta + \norm{\psi(\mu)} \le \frac{\delta_*}2 + \frac{\delta_*}2 = \delta_* $$
    \end{itemize}

    Учитывая, что $ \delta < \delta_* < \sigma_* < \sigma $, заключаем:
    \begin{enumerate}
        \item решение $ y \big( x, x_0, y^0(\mu), \mu) = u(x, x_0, u^0, \mu) + \psi(\mu) $ определено, непрерывно по совокупности аргументов на множестве $ V_\delta^{x_0, y^0(\mu)}(\vphi, \hat \mu) $ и обладает теми же свойствами, что и решение $ u(x, x_0, u^0, \mu) $ на $ V_{\delta_*}^{x_0, u^0} (\vphi, \hat \mu) $;
        \item точка $ \big( x, y(x, x_0, y^0, y^0(\mu), \mu) \mu \big) \in \ol U_\sigma^{x, y}(\vphi, \hat \mu) $ для всякого $ x \in [a, b] $, поскольку если $ (x, u, \mu) \in \ol U_{\sigma_*}^{x, u}(\vphi, \hat \mu) $, то $ (x, y, \mu) \in \ol U_\sigma^{x, y}(\vphi, \hat \mu) $, где $ y = u + \psi(\mu) $.
    \end{enumerate}

\end{undefthm}

\section{Теорема о непрерывной зависимости решений от начальных данных и параметра}

\begin{theorem}[о непрерывной зависимости решения от начальных данных и параметра]
    Пусть в системе \eref{3.15} функция $ f(x, y, \mu) $ определена, непрерывна и $ f \in \operatorname{Lip}_y^{loc}(F) $, а $ y = \phi(x) $ --- решение системы \eref{3.15^} на $ [a, b] $.

    Тогда для любого $ \sigma > 0 $, при котором $ \ol U_\sigma^{x, y} (\phi, \hat \mu) \sub F $, найдётся такое $ 0 < \delta < \sigma $, что для произвольной точки $ \big( x_0, y^0(\mu), \mu \big) \in U_\delta^{x_0, y^0(\mu)}(\phi, \hat \mu) $, где $ y^0(\mu) $ из \eref{3.17}, решение $ y = y \big( x, x_0, y^0(\mu), \mu \big) $ системы \eref{3.15} определено при всех $ x \in [a, b] $, непрерывно по совокупности аргументов на множестве $ V_\delta^{x_0, y^0(\mu)} = [a, b] \times U_\delta^{x_0, y^0(\mu)}(\phi, \hat \mu) $ и точка $ \bigg( x, y \big( x, x_0, y^0(\mu), \mu \big), \mu \bigg) \in \ol U_\sigma^{x, y}(\phi, \hat \mu) $ для любого $ x \in [a, b] $.
\end{theorem}

\begin{proof}
    Пусть число $ \sigma > 0 $ такое, что окрестность $ \ol U_\sigma^{x, y}(\phi, \hat \mu) $ решения $ y = \phi(x) $ порождающей системы \eref{3.15^}, определённого на $ [a, b] $ лежит в области $ F $.

    Решение $ y(x, \mu) = y \big( x, x_0, y^0(\mu), \mu \big) $ будем строить методом последовательных приближений Пикара. Но, перед этим, проведём рассуждение о сдвиге ($ \to $), где выбрано такое $ \sigma_* $, что компакт $ \ol U_{\sigma^*}^{x, y}(\phi, \hat \mu) $ содержится в области $ F_* $ системы $ u' = h(x, u, \mu) $, полученной при помощи замены $ y = u + \psi(\mu) $.

    \begin{itemize}
        \item Пусть сначала $ \delta_* = \frac{\sigma_*}2 $. Построим решение $ u(x, \mu) = u(x, x_0, y^0, \mu) $, у которого $ x \in [a, b] $, ф $ (x_0, u^0, \mu) \in U_{\delta_*}^{x_0, u^0}(\phi, \hat \mu) $, методом последовательных приближений Пикара, при необходимости уменьшая $ \delta_* $, то так, чтобы сохранилась его положительность.

        Нулевое пикаровское приближение выберем не постоянным, а лежащем в окрестности дуги интегральной кривой расчётного решения $ u = \phi(x) $, положив
        $$ u^{(0)}(x) = u^{(0)}(x, x_0, u^0) - u^0 - \phi(x_0) + \phi(x), \qquad x \in [a, b] $$
    Тогда $ u^{(0)}(x_0) = u^0 $, и для любого $ x \in [a, b] $ имеем:
    \begin{enumerate}
        \item $ u^{(0)}(x) = u^0 + \int_{x_0}^x h \big( s, \phi(s), \hat \mu \big) \di s $;
        \item $ \| u^{(0)}(x) - \phi(x) \| = \| u^0 - \phi(x_0) \| $;
        \item $ \big( x, u^{(0)}(x), \mu \big) \in \ol U_{\sigma_*}^{x, y}(\phi, \hat \mu) $.
    \end{enumerate}

    Из вида $ u^{(0)}(x, x_0, u^0) $, указанного в 1 вытекает, что нулевое пикаровское приближение непрерывно по каждому из аргументов $ x_0, u^0, $, а по $ x $ оно непрерывно дифференцируемо, что гарантирует непрерывность функции $ u^{(0)}(x, x_0, u^0, \hat \mu) $ по совокупности аргументов. Кроме того, свойство 3, очевидно, вытекает из свойства 2, так как $ \| u^0 - \phi(x_0) \| < \delta_*, ~ \| \mu - \hat \mu \| < \delta_* $ в $ U_{\delta_*}^{x_0, u^0}(\phi, \hat \mu) $, а $ \delta_* = \frac{\sigma_*}2 $.

    \item Теперь для любого $ k \ge 1 $ введём $ k $-е пикаровское приближение
    \begin{equ}{3.19}
        u^{(k)}(x) = u^{(k)}(x, x_0, u^0, \mu) = u^0 + \int_{x_0}^x h \big( s, u^{(k - 1)}(s), \mu \big) \di s,
    \end{equ}
    определённое в некоторой окрестности точки $ x_0 $, и $ u^{(k)}(x_0) = u^0 $.

    Пусть $ L = L_{\sigma_*} \ge 1 $ --- глобальная константа Липшица по $ u $ для функции $ h(x, u, \mu) $ на компакте $ \ol U_{\sigma_*}^{x, u}(\phi, \hat \mu) $, а
    $$ \tau = \tau_{\sigma_*, L} = \frac{\sigma_*L}{2(e^{L(b - a)} - 1)} $$

    Покажем \bt{по индукции}, что существует такое $ 0 < \delta_*(\tau) < \frac{\sigma_*}2 $, что для всякого $ x \in [a, b] $:
    \begin{enumerate}
        \item функция $ u^{(k)}(x, x_0, u^0, \mu) $ определна и непрерывна на множестве $ V_{\delta_*}^{x_0, u^0} = [a, b] \times U_{\delta_*}^{x_0, u^0}(\phi, \hat \mu) $;
        \item $ \| u^{(k)}(x) - u^{(k - 1)}(x) \| \le \frac\tau L \cdot \frac{(L|x - x_0|)^k}{k!} $;
        \item $ \big( x, u^{(k)}(x), \mu \big) \in \ol U_{\sigma_*}^{x, u}(\phi, \hat \mu) $.
    \end{enumerate}
    Как и раньше, 3 означает, что $ \| u^{(k)}(x) - \phi(x) \| \le \sigma_* $ и $ \| \mu - \hat \mu \| \le \sigma_* $.

    \bt{База:}
    \begin{enumerate}
        \item По свойству 3 нулевого приближения для любых $ \delta_* \le \frac{2\sigma_*}2 $ и $ x \in [a, b] $ функция $ h \big( x, u^{(0)}(x), \mu \big) $ определена и непрерывна. Поэтому первое пикаровское приближение
        $$ u^{(1)}(x, x_0, u^0, \mu) = u^0 + \int_{x_0}^x h \big( s, u^{(0)}(s), \mu \big) \di s $$
        определено для всех $ x \in [a, b] $ и непрерывно на множестве $ V_{\delta_*}^{x_0, u^0} $ как композиция непрерывных функций.

        \item С учётом свойства 1 для всякого $ x \in [a, b] $ имеем
        \begin{multline*}
            \| u^{(1)}(x) - u^{(0)}(x) \| = \bigg\| \int_{x_0}^x h \big( s, u^{(0)}(s), \mu \big) \di s - \int_{x_0}^x h \big( s, \phi(s), \hat \mu \big) \di s \bigg\| \le \\
            \le \bigg| \int_{x_0}^x \| h \big( s, u^{(0)}(s), \mu \big) - h \big( s, \phi(s) \hat \mu \big) \| \di s \bigg|
        \end{multline*}
        Посольку функция $ h(x, u, \mu) $ непрерывна в области $ F_* $, она равномерно непрерывна на компакте $ \ol U_{\sigma_*}^{x, y}(\phi, \hat \mu) $, а аргументы $ h $ под интегралом принадлежат этому компакту. Следовательно, взяв $ \tau $ в качестве $ \veps $,
        $$ \exist 0 < \delta_*(\tau) < \frac{\sigma_*}2 : \quad
        \begin{rcases}
            \| u^{(0)}(s) - \phi(s) \| \le \delta_* \\
            \| \mu - \hat \mu \| \le \delta_*
        \end{rcases} \implies \big\| h \big( s, u^{(0)}, \mu \big) - h \big( s, \phi(s), \hat \mu \big) \big\| \le \tau $$
        В результате $ \| u^{(1)}(x) - u^{(0)}(x) \| \le \tau |x - x_0| $ для любого $ x \in [a, b] $, что совпадает с 2 при $ k = 1 $.
    \end{enumerate}
    \end{itemize}

\end{proof}

\section{Теорема о дифференцируеости решений по начальным данным}

Предположим, что функция $ y \big(x, x_0, y^0(\mu), \mu \big) $ непрерывно дифференцируема по каждому из четырёх аргументов, а $ y^0(\mu) $ --- по $ \mu $.

Будем обозначать частную производную по $ x $ как $ y_x'\big( x, x_0, y^0(\mu), \mu \big) $. Частную производную по $ \mu $ будем обозначать $ \pder{}\mu y \big( x, x_0, y^0(\mu), \mu) $, поскольку $ \mu $ входит не в один аргумент. В результате
$$ \pder{}\mu y = y_{y_0}' \cdot \big( y^0(\mu) \big)_\mu' + y_\mu' $$

Также будем предполагать, что функция $ f $ непрерывно дифференцируема по $ y $ и $ \mu $.
\begin{notation}
    $ f(x, y, \mu) \in \mc C_{x, y, \mu}^{0, 1, 1}(F) $.
\end{notation}

Введём обозначения для начальных данных, параметра и их расчётных значений:
$$ \kappa = \big( x_0, y^0(\mu), \mu \big), \qquad \hat \kappa = (\hat x_0, \hat y^0, \hat \mu) $$

\begin{theorem}[о дифференцируемости решения по начальным данным и параметру]
    Пусть в системе \eref{3.15} функция $ f(x, y, \mu) \in \mc C_{x, y, \mu}^{0, 1, 1}(F) $, а $ \phi(x) = y(x, \hat \kappa) $ --- решение системы \eref{3.15^} на $ [a, b] $.

    Тогда для любого $ \sigma > 0 $ такого, что $ \ol U_\sigma^{x, y}(\phi, \hat \mu) \sub F $, найдётся такое $ 0 < \delta < \sigma $, что для любой точки $ \kappa \in U_\delta^{x_0, y^0(\mu)}(\phi, \hat \mu) $ решение $ y = y(x, \kappa) $ системы \eref{3.15} с $ y^0(\mu) $ из \eref{3.17}, где $ \psi(\mu) \in \mc C^1 \big( K_\sigma(\hat \mu) \big) $, имеет непрерывные частные производные по каждому из аргументов в любой точке множества $ V_\delta^{x_0, y^0(\mu)} = [a, b] \times U_\delta^{x_0, y^0(\mu)}(\phi, \hat \mu) $, причём:
    \begin{enumerate}
        \item функция $ \xi^{(j)} \big( x, x_0, \pder{y^0(\mu)}P\mu_j, \mu) = \pder{y(x, \kappa)}{\mu_j}, \quad j = \ol{1, m} $ --- это решение ЗК ЛНС
        \begin{equ}{3.20}
            u' = f_y' \bigg( x, y \big(x, x_0, y^0(\mu), \mu \big), \mu \bigg) u + f_{\mu_j}' \bigg( x, y \big( x, x_0, y^0(\mu), \mu \big), \mu \bigg);
        \end{equ}

        \item функция $ \eta^{(i)}(x, x_0, e^{(i)}, \mu) = y_{u_i^0}'(x, \kappa) $, где $ e^{(i)} = (0, \dots, 1, \dots, 0) $ --- это решение ЗК ЛНС
        \begin{equ}{3.21}
        	u' = f_y' \bigg( x, y \big( x, x_0, y^0(\mu), \mu \big), \mu \bigg) u;
        \end{equ}

        \item функция $ \eta^{(0)} \bigg( x, x_0, -f \big( x_0, y^0(\mu), \mu \big), \mu \bigg) =
        y_{x_0}'(x, \kappa) $ также является решением $ \text{ЗК}_{\eref{3.21}} $.
    \end{enumerate}
\end{theorem}

\begin{definition}
    Линейные системы \eref{3.20} и \eref{3.21} называются \it{системами в вариациях} соответственно по параметру и по начальным данным относительно решения $ u \big( x, x_0, y^0(\mu), \mu \big) $.
\end{definition}

\section{Теорема о дифференцируемости решений по вектору параметров}

\section{Теорема о многократной дифференцируемости решения по начальным данным и параметру}

\begin{theorem}[о многократной дифференцируемости решения по начальным данным и параметру]
    Пусть в системе \eref{3.15} $ f(x, y, \mu) \in \mc C_{x, y, \mu}^{0, k, k} (F) $, $ y(x, \kappa) $ --- решение из теоремы о дифференцируемости решения по начальным данным и параметру с $ y^0(\mu) $ из \eref{3.17}, причём $ \psi(\mu) \in \mc C^k \big( K_\sigma(\hat \mu) \big) $.

    Тогда $ y (x, \kappa) \in \mc C_{x, x_0, y^0, \mu}^{1, k, k, k} \big( V_\delta^{x_0, y^0(\mu)} \big) $, где $ V_\delta^{x_0, y^0(\mu)} = [a, b] \times U_\delta^{x_0, y^0(\mu)}(\psi, \hat \mu) $.
\end{theorem}

\section{Теорема Ляпунова--Пуанкаре о разложении решения в ряд по степеням начальных данных и параметра}

\begin{definition}
    Функцию $ f(x, y, \mu) $ будем называть равномерно аналитической по $ y, \mu $ относительно $ x $ в замкнутой трубчатой окрестности $ \ol U_\sigma^{x, y}(\phi, 0) $, если она представима в виде ряда
    \begin{equ}{3.23}
        f(x, y, \mu) = \sum_{p, q} f^{(p, q)}(x) \big( y - \phi(x) \big)^p \mu^q
    \end{equ}
    с вещественными непреывными на отрезке $ [a, b] $ коэффициентами $ f^{(p, q)}, \quad f^{(0, 0)} = \phi'(x) $, который абсолютно сходится при $ \| y - \phi(x) \| \le \sigma, \quad \| \mu \| \le \sigma $ для вяского $ x \in [a, b] $.
\end{definition}

\begin{theorem}[Ляпунова--Пуанкаре]
    Предволожим, что в системе \eref{3.15} $ f(x, y, \mu) \in \mc C(F), \quad f(x, y, \mu) \in \operatorname{Lip}_y^{loc}(F), \quad (\hat x_0, \hat y^0, 0) \in F, \quad $ решение $ \phi(x) = u(\hat x, \hat x_0, \hat y^0, 0), \quad \phi(\hat x_0) = \hat y^0 $ системы \eref{3.15^} с $ \hat \mu = 0 $ определено на $ [a, b] $, и существует $ \sigma > 0 $ такое, что функция $ f(x, y, \mu) $ является равномерно аналитической по $ y, \mu $ относительно $ x $ на компакте $ \ol U_\sigma^{x, y}(\phi, 0) $, \ie справедливо разложение \eref{3.23}.

    Тогда найдётся такое $ \delta > 0 $, что решение системы \eref{3.15} $ y \big( x, \hat x_0, y^0(\mu), \mu) $ с $ y^0(\mu) = y^0 + \psi(\mu) $ и $ \psi $ из \eref{3.24} будет функцией равномерно аналитической по $ y^0, \mu $ относительно $ x $ на $ V_\delta^{y^0(\mu)} = [a, b] \times U_\delta^{y^0(\mu)}(\hat y^0, 0) $, \ie будет представимо в виде ряда
    \begin{equ}{3.25}
        y \big( x, \hat x_0, \hat y^0(\mu), \mu \big) = \phi(x) + \sum_{p, q : ~ |p| + |q| \ge 1} y^{(p, q)}(x) (y^0 - \hat y^0)^p \mu^q + \sum_{q : ~ |q| \ge 1} \psi^{(q)} \mu^q
    \end{equ}
    с вещественными, непрерывно дифференцируемыми на $ [a, b] $ коэффициентами, который для любого $ x \in [a, b] $ абсолютно сходится при $ \| y^0 - \hat y^0 \| < \delta $, $ \| \mu \| < \delta $.

    При этом $ y_1^{(e_1, 0)}(\hat x_0), \dots, y_n^{(e_n, 0)}(\hat x_0) = 1 $, а остальные $ y_i^{(p, q)}(\hat x_0) $ равны нулю. Ряд \eref{3.25} можно почленно дифференцировать по $ x $, и полученный после дифференцирования ряд равномерно относительно $ x $ сходится на том же множестве $ V_\delta^{y^0(\mu)} $.
\end{theorem}

\section{Теорема о разложении решения в ряд по степеням малого параметра}

Будем предполагать, что
\begin{equ}{3.24}
    \psi(\mu) = \sum_{q : |q| \ge 1}\psi^{(q)}\mu^q,
\end{equ}
причём радиус сходимости ряда $ \psi $ больше, чем $ \sigma $.

\begin{theorem}[о разложении решения в ряд по степеням малого параметра]
    Предположим, что для системы \eref{3.15} выполняются условия теоремы Ляпунова--Пуанкаре.

    Тогда найдётся такое $ \delta > 0 $, что решение $ y \big( x, \hat x_0, \hat y_0(\mu), \mu \big) $, где $ \hat y^0(\mu) = \hat y_0 + \psi(\mu) $ с $ \psi $ из \eref{3.24}, является равномерно аналитической по $ \mu $ относительно $ x $ функцией на множестве
    $$ V_\delta = \set{(x, \mu) \mid x \in [a, b], ~ \| \mu \| < \delta}, $$
    \ie раскладывается на нём в абсолютно сходящийся степенной ряд
    \begin{equ}{3.30}
        y \big( x, \hat x_0, \hat y^0(\mu), \mu \big) = \phi(x) + \sum_{q : |q| \ge 1} y^{(q)}(x) \mu^q, \qquad y^{(q)}(\hat x_0) = \psi^{(q)}
    \end{equ}

    При этом вещественные коэффициенты $ y^{(k)}(x) \in \mc C^1([a, b]) $.
\end{theorem}

\section{Теорема Коши об аналитичности решения по независимой переменной}

\begin{definition}
    Функцию $ f(x, y) $ будем называть вещественно-аналитической в области $ G \sub \R^{1 + n} $, если для любой точки $ (x_0, y^0) \in G $ существует $ d(x_0, y^0) > 0 $ такое, что $ f $ раскладывается в степенной ряд
    $$ f(x, y) = \sum_{k = 0}^\infty \sum_{p : |p| = 0} f^{(k, p)} \cdot (x - x_0)^k(y - y_0)^p $$
    с вещественными коэффициентами $ f^{(k, p)} $, абсолютно сходящийся при $ |x - x_0| < d, ~ \| y - y^0 \| < d $.
\end{definition}

\begin{theorem}[Коши]
    Пусть в системе \eref{3.1} $ f(x, y) $~--- вещественно-аналитическая функция в области $ G $.

    Тогда для любой точки $ (x_0, y^0) \in G $ существует такое $ \rho(x_0, y^0) > 0 $, что решение $ y = y(x, x_0, y^0) $ системы \eref{3.1} раскладывается в абсолютно сходящийся степенной ряд
    \begin{equ}{3.31}
        y(x, x_0, y^0) = \sum_{k = 0}^\infty y^{(k)} \cdot (x - x_0)^m, \qquad y^{(0)} = y^0
    \end{equ}
    с вещественными коэффициентами $ y^{(k)} $ и радиусом сходимости $ \rho $.
\end{theorem}

\section{Теорема об аналитичности решения ЛНС по независимой переменной}

\begin{theorem}[об аналитичности решения ЗК ЛНС]
    Пусть в системе $ y' = P(x)y + q(x) $ матрица $ P(x) $ и вектор $ q(x) $ --- вещественно-аналитические функции на интервале $ (a, b) $ и пусть отрезок $ [x_0 - d, x_0 + d] \sub (a, b), \quad d > 0 $.

    Тогда для любого $ y^0 \in \R^n $ решение $ y = y(x, x_0, y^0) $ системы раскладывается в абсолютно сходящийся степенной ряд \eref{3.31} с радиусом сходимости $ \rho \ge d $.
\end{theorem}
