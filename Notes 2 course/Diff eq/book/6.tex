\part{Автономные системы}

\begin{definition}
	Нормальная система называется \emph{автономной}, если её правая часть не зависит от независимой переменной $ x $, \ie система имеет вид $ y' = f(y) $.
\end{definition}

Рассмотрим автономную систему
\begin{equ}{6.1}
	\dot x = X(x)
\end{equ}

\section{Инвариантность решений относительно сдвигов по времени; теорема о единственности для траекторий}

\begin{lemma}[об инвариантности решений относительно сдвигов по времени]
	Пусть $ x = \phi(t, t_0, p) $ "--- решение системы \eref{6.1} на $ I_max = (\alpha, \beta) $.

	Тогда для любого $ \tau \in \R $ функция $ \psi(t) = \phi(t + \tau, t_0, p) $ является решением системы \eref{6.1} при всех $ t \in (\alpha - \tau, \beta - \tau) $.
\end{lemma}

\begin{theorem}
	Если две траектории автономной системы \eref{6.1} имеют одну общую точку, то они совпадают.
\end{theorem}

\section{Групповое свойство решений; система для траекторий}

\begin{statement}
	Пусть $ x = \phi(t, t_0, p) $ "--- решение системы \eref{6.1} на максимальном интервале существования $ (\alpha, \beta) $.

	Тогда
	$$ \forall \tau \in \R \quad \phi(t + \tau, t_0 + \tau, p) \equiv[(\alpha, \beta)] \phi(t, t_0, p) $$
\end{statement}

\begin{notation}
	$ \phi(t, p) = \phi(t, 0, p) $
\end{notation}

\begin{theorem}[о системе для траекторий]
	Для любой обыкновенной точки $ p \in D $ автономной системы \eref{6.1} с $ X_1(p) \ne 0 $ найдётся окрестность $ U_p $, в которой траектории системы \eref{6.1} совпадают с интегральными кривыми нормальной системы порядка $ n - 1 $
	\begin{equ}{6.4}
		\frac{\di x_2}{\di x_1} = \frac{X_2(x_1, \dots, x_n)}{X_1(x_1, \dots, x_n)}, \quad \dots, \quad \frac{\di x_n}{\di x_1} = \frac{X_n(x_1, \dots, x_n)}{X_1(x_1, \dots, x_n)}
	\end{equ}
\end{theorem}

\begin{definition}
	Система \eref{6.4} называется \emph{системой для траекторий} автономной системы \eref{6.1}.
\end{definition}

\section{Особая точка, цикл; теорема о типах траекторий автономных систем}

\begin{definition}
	Если движение $ x = \phi(t, p) $ таково, что $ \phi(t, p) = p $ при $ t \in \R $, то точка $ p $, являющаяся траекторией этого движения, называется \emph{точкой покоя}, или \emph{особой точкой}, или \emph{положением равновесия} системы.
\end{definition}

\begin{statement}
	Точка $ p \in D $ является точкой покоя \textbf{тогда и только тогда}, когда в системе \eref{6.1} $ X(p) = 0 $.
\end{statement}

\begin{definition}
	Если существует число $ \omega > 0 $ такое, что для любого $ t \in \R $ движение $ \phi(t, p) = \phi(t + \omega, p) $, то замкнутая кривая $ l $, являющаяся траекторией этого движения, называется \emph{циклом}.
\end{definition}

\begin{theorem}
	Траектории автономных систем бывают трёх типов:
	\begin{enumerate}
		\item точка покоя $ p $;
		\item цикл $ l $ (простая замкнутая кривая);
		\item незамкнутая траектория $ L $ без самопересечений.
	\end{enumerate}
\end{theorem}

\section{Предельные множества, теорема о свойствах предельных\texorpdfstring{\\}{} множеств траекторий}

\begin{definition}
	Пусть в системе \eref{6.1} движение $ x = \phi(t - t_0, p) $ определено при всех $ t \le t_0 $.

	Точка $ q \in D $ называется $ \alpha $-\emph{предельной точкой} траектории, порождённой этим движением, если найдётся последовательность моментов времени $ t_k $ такая, что $ t_k \to -\infty $ и $ \phi(t_k -t_0, p) \underarr{k \to \infty} q $.
\end{definition}

\begin{definition}
	Пусть в системе \eref{6.1} движение $ x = \phi(t - t_0, p) $ определено при всех $ t \ge t_0 $.

	Точка $ q \in D $ называется $ \omega $-\emph{предельной точкой} траектории, порождённой этим движением, если найдётся последовательность моментов времени $ t_k $ такая, что $ t_k \to +\infty $ и $ \phi(t_k -t_0, p) \underarr{k \to \infty} q $.
\end{definition}

\begin{definition}
	$ A $-\emph{предельным множеством} траектории движения $ x = \phi(t - t_0, p) $ называется множество $ \alpha $-предельных точек этой траектории, а множество $ \omega $-предельных точек "--- $ \Omega $-предельным множеством.
\end{definition}

\begin{theorem}
	$ \Omega $-предельное множество любой траектории системы \eref{6.1} замкнуто и инвариантно.
\end{theorem}

Аналогичная теорема справедлива для $ A $-предельных множеств.

\section{Устойчивость траекторий по Лагранжу; теорема о свойствах предельных множеств траекторий, устойчивых по Лагранжу}

\begin{definition}
	Траектория движения $ x = \phi(t - t_0, p) $ системы \eref{6.1}, определённого на $ I_{\max} = (\alpha, \beta) $, называется \emph{положительно устойчивой по Лагранжу}, если в фазовом пространстве найдётся такой компакт $ \ol H^+ \sub D $, что $ \phi(t - t_0, p) \in \ol H^+ $ для любого $ t \in [t_0, \beta) $, и называется \emph{отрицательно устойчивой по Лагранжу}, если найдётся компакт $ \ol H \sub D $ такой, что $ \phi(t - t_0, p) \in \ol H^- $ для любого $ t \in (\alpha, t_0] $.
\end{definition}

\begin{theorem}
	$ \Omega $-предельное множество любой положительно устойчивой по Лагранжу траектории системы \eref{6.1} ну пусто, компактно и связно.
\end{theorem}

Аналогичная теорема справедлива для $ A $-предельных множеств.

\section{Классификация Пункаре; узлы, седло}

Рассмотрим автономную систему
\begin{equ}{6.9}
	\dot y = Ay,
\end{equ}
где $ y =
\begin{pmatrix}
	y_1 \\
	y_2
\end{pmatrix} $, $ A $ "--- двумерная вещественная постоянная матрица, $ |A| \ne 0 $.

Известно, что существует постоянная неособая матрица $ S $ такая, что линейная замена $ y = Sz $ сводит \eref{6.9} к системе
\begin{equ}{6.11}
	\dot z = Jz,
\end{equ}
где $ z =
\begin{pmatrix}
	z_1 \\
	z_2
\end{pmatrix} $, а $ J = S^{-1}AS $ "--- жорданова форма матрицы $ A $.
Такую замену называют \emph{центроаффинной}.

Пусть собственные числа $ \lambda_1, \lambda_2 $ вещественны и различны.
Тогда система \eref{6.11} имеет вид
$$
\begin{cases}
	\dot z_1 = \lambda_1z_1 \\
	\dot z_2 = \lambda_2z_2
\end{cases} $$
Интегрируя каждое из уравнений, находим общее решение:
\begin{equ}{6.12}
	z_1(t) = c_1e^{\lambda_1t}, \quad z_2(t) = c_2e^{\lambda_2t}
\end{equ}
Пусть $ c_1^2 + c_2^2 \ne 0 $.
Возможны два случая:
\begin{enumerate}
	\item $ \lambda_1 \lambda_2 > 0, \quad \lambda_1 \ne \lambda_2 $

		Пусть, например $ \lambda_1, \lambda_2 < 0 $.
		Тогда в \eref{6.12} с ростом времени $ z_k(t) \to 0 $, а с убыванием "--- $ |z_k(t)| \to \infty $.

		\begin{enumerate}
			\item Если $ c_1 = 0 $, то решения $ z_1(t) \equiv 0, ~ z_2(t) = \pm c^{-2}e^{\lambda_2t} $ задают две траектории, являющиеся положительной и отрицательной полуосями ординат.
			\item При $ c_2 = 0 $ имеем две траектории, являющиеся положительной и отрицательной полуосями абсцисс.
				Их параметризуют движения $ z_1(t) = \pm c^{-2}e^{\lambda_1t}, ~ z_2(t) \equiv 0 $.
			\item При $ c_1, c_2 \ne 0 $, избавляясь от $ t $, получаем $ (c_1^{-1}z_1)^{\lambda_2} = (c_2^{-1}z_2)^{\lambda_1} $.
				\begin{enumerate}
					\item Если $ |\lambda_2| > |\lambda_1| $, уравнение для траекторий можно записать в виде $ z_2 = c^{-1}|z_1|^{\frac{\lambda_2}{\lambda_1}} $.
						Тогда для каждого $ c_0 > 0 $, фиксируя $ |c| = c_0 $, из уравнения для траекторий получаем четыре траектории: две параболы, движение по которым происходит в направлении начала координат.
					\item Если $ |\lambda_2| < |\lambda_1| $, то те же четыре траектории являются ``лежачими'' параболами.
				\end{enumerate}
		\end{enumerate}
		Полученный фазовый портрет и особая точка называются \emph{узел}.
	\item $ \lambda_1\lambda_2 < 0 $

		Пусть, например, $ \lambda_2 < 0 < \lambda_1 $.
		Тогда в \eref{6.12} с ростом времени $ |z_1(t)| \to \infty, ~ z_2(t) \to 0 $.

		\begin{enumerate}
			\item при $ c_1 = 0 $ траекториями являются полуоси ординат;
			\item при $ c_2 = 0 $ траектории "--- полуоси абсцисс;
			\item при $ c_1, c_2 \ne 0 $ траектории "--- гиперболы.
		\end{enumerate}
		Полученный фазовый портрет и особая точка называются \emph{седло}.
\end{enumerate}

Если $ \lambda_1 = \lambda_2 = \mu $, то
$$ J =
\begin{pmatrix}
	\mu & \sigma \\
	0 & \mu
\end{pmatrix}, \quad \sigma \in \set{0, 1} $$
\begin{enumerate}
	\item при $ \sigma = 1 $ имеем \emph{вырожденный узел};
	\item при $ \sigma = 0 $ имеем \emph{дикритический узел}.
\end{enumerate}

\section{Классификация Пуанкаре; фокус, центр}

Пусть $ \lambda_1 = \alpha + \ii \beta $.
Поскольку исходная система вещественна, то $ \lambda_2 = \ol\lambda_1 $.
Система \eref{6.11} имеет вид
\begin{equ}{6.13}
	\dot z = Jz, \quad J =
	\begin{pmatrix}
		\lambda_1 & 0 \\
		0 & \ol \lambda_1
	\end{pmatrix}, \quad \text{ или }
	\begin{cases}
		\dot z_1 = (\alpha + \ii \beta)z_1 \\
		\dot z_2 = (\alpha - \ii \beta)z_2
	\end{cases}
\end{equ}

\begin{enumerate}
	\item $ \alpha \ne 0 $

		Пусть, например, $ \alpha < 0, ~ \beta > 0 $.
		Тогда с ростом времени полярный угол $ \phi(t) $ любого решения монотонно возрастает, а радиус-вектор $ r(t) $ стремится к нулю.

		Полученный фазовый портрет и особая точка называются \emph{фокус}.
	\item $ \alpha = 0, ~ \beta > 0 $

		С ростом времени полярный угол $ \phi(t) $ любого движения монотонно возрастает, а радиус-вектор $ r(t) \equiv r_0 $.
		Таким образом, траектории представляют собой семейство концентрических окружностей.

		Полученный фазовый портрет и особая точка называются \emph{центр}.
\end{enumerate}
