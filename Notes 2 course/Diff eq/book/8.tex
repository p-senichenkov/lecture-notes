\part{Теория нормальных форм Пуанкаре}

\section{Формальная и аналитическая эквивалентность систем, нормализация линейной части системы}

Рассмотрим автономную систему порядка $ n $, правая часть которой является формальным или абсолютно сходящимся векторным степенным рядом в окрестности некоторой точки покоя.
НУО считаем, что точка покоя смещена в начало координат и в правой части выделены линейные члены:
\begin{equ}{8.1}
	\dot x = Ax + X(x), \qquad x =
	\begin{pmatrix}
		x_1 \\
		\vdots \\
		x_n
	\end{pmatrix}, \quad \dot x =
	\begin{pmatrix}
		\frac{\di x_1}{\di t} \\
		\vdots \\
		\frac{\di x_n}{\di t}
	\end{pmatrix}, \quad X =
	\begin{pmatrix}
		X_1 \\
		\vdots \\
		X_n
	\end{pmatrix} = \sum_{p : |p| = 2}^\infty \nder[p]X x^p
\end{equ}

Упрощение системы осуществляется при помощи обратимой замены, сохраняющей равновесие в нуле:
\begin{equ}{8.2}
	x = Sy + f(y), \qquad y =
	\begin{pmatrix}
		y_1 \\
		\vdots \\
		y_n
	\end{pmatrix}, \quad |S| \ne 0, \quad f_i = \sum_{p : |p| = 2}^\infty \nder[p]f_i y^p
\end{equ}

Пусть эта замена преобразует систему \eref{8.1} в систему
\begin{equ}{8.3}
	\dot y = By + Y(y)
\end{equ}

\begin{statement}
	Пусть система \eref{8.1} сходится.
	\begin{enumerate}
		\item если замена \eref{8.2} сходится, то система \eref{8.3} сходится;
		\item если система \eref{8.3} расходится, то замена \eref{8.2} расходится.
	\end{enumerate}
\end{statement}

\begin{definition}
	Системы \eref{8.1} и \eref{8.3} называются \emph{формально эквивалентными}, если существует формальная замена переменных \eref{8.2}, которая преобразует \eref{8.1} в \eref{8.3}.
	Если при этом ряды $ X $ и $ f $ сходятся, то системы называются \emph{аналитически эквивалентными}.
\end{definition}

Удобнее всего разбить замену \eref{8.2} в композицию двух замен: линейной
\begin{equ}{8.6}
	x = Sz, \qquad J = S^{-1}AS
\end{equ}
и почти тождественной
\begin{equ}{8.7}
	z = y + h(y), \qquad h = \sum_{p : |p| = 2}^\infty \nder[p]h y^p
\end{equ}
Тогда в замене \eref{8.2} $ f(y) = Sh(y) $.

Делаем линейную замену и дифференцируем \eref{8.6} по $ t $ в силу системы \eref{8.1}:
$$ \dot z = Jz + Z(z), \qquad Z(z) = S^{-1}X(Sz) $$

Введём числа $ \sigma_1 = 0, ~ \sigma_j = 0 $, если $ \lambda_{j - 1} \ne \lambda_j $:
\begin{equ}{8.9}
	\dot z_i = \lambda_i z_i + \sigma_i z_{i - 1} + Z_i(z), \qquad Z_i = \sum_{p : |p| = 2}^\infty \nder[p]Z_i z^p
\end{equ}

\section{Почти тождественная замена, вывод связующей системы}

Запишем замену \eref{8.7} в координатах:
\begin{equ}{8.10}
	z_i = y_i + h_i(y), \qquad h \in \Phi_2
\end{equ}
Предположим, что она преобразует систему \eref{8.9} в систему
\begin{equ}{8.11}
	\dot y_i = \lambda_iy_i + \sigma_iy_{i - 1} + Y_i(y), \qquad Y_i = \sum_{p : |p| = 2}^\infty \nder[p]Y_i y^p
\end{equ}
Продифференцируем замену по $ t $ в силу системы \eref{8.11}:
$$ \sum_{j = 1}^n \pder[y_j]{h_i}\lambda_jy_j - \lambda_ih_i + Y_i \equiv Z_i(y + h) + \sigma_i h_{i - 1} - \sum_{j = 1}^n \pder[y_j]{h_i}(\sigma_jy_{j - 1} + Y_j) $$

Приравняем коэффициенты, стоящие при всех ЛНЗ мономах $ y^p $:
$$ \pder[y_j]{h_i} = \sum_{p : |p| = 2}^\infty p_j \nder[p]h_i y^{p - e_j} = \sum_{p : |p| = 1}^\infty (p_j + 1)\nder[p + e_j]h_i y^p $$
$$ \pder[y_j]{h_i}y{j - 1} = \sum_{p : |p| = 1}^\infty (p_j + 1)\nder[p + e_j]h_i y^{p + e_{j - 1}} = \sum_{p : |p| = 2}^\infty (p_j + 1)\nder[p - e_{j - 1} + e_j]h_iy^p $$

Просуммируем полученные произведения по всем возможным векторам $ q $:
\begin{multline}\lbl{8.12}
	\sum_{j = 1}^n p_j\lambda_j\nder[p]h_i - \lambda_i\nder[p]h_i + \nder[p]Y_i = \set{Z_i(u + h)}^{(p)} + \sigma_i\nder[p]h_{i - 1} - \sum_{j = 1}^n \sigma_j(p_j + 1)\nder[p - e_{j - 1} + e_j]h_i - \\
	- \sum_{j = 1}^n \sum_{|q| = 2}^{|p| - 1}(p_j - q_j + 1)\nder[p - q + e_j]h_i \nder[q]Y_j
\end{multline}
Запись $ \set{Z-i(y + h)}^{(p)} $ означает, что после переразложения ряда $ Z_i(u + h) $ по степеням $ y $ из него выделены члены, стоящие при $ y^p $.

Полученные равенства будем называть \emph{связующей системой}.

\section{Рекуррентность связующей системы}

Векторы коэффициентов будем сравнивать лексикографически.

\begin{definition}
	Будем говорить, что коэффициент $ \nder[q]\alpha_j $ \emph{предшествует} коэффициенту $ \nder[p]\alpha_i $, если пара $ (q, j) $ предшествует паре $ (p, i) $.
\end{definition}

Покажем, что в правой части системы \eref{8.12} стоят коэффициенты $ \nder[q]h_j $ и $ \nder[q]Y_j $, предшествующие коэффициентам $ \nder[p]h_i $ и $ \nder[p]Y_i $ из левой части, то есть связующая система является рекуррентной.

\begin{itemize}
	\item В первом и четвёртом слагаемых правой части равенств содержатся $ \nder[q]h_j $ и $ \nder[q]Y_j $ с $ |q| < |p| $, поскольку разложения рядов $ Z, Y, h $ начинаются не ниже, чем со второго порядка.
	\item В третьем слагаемом правой части $ |p - e_{j - 1} - + e_j| = |p| $, но $ (j - 1) $-я компонента верхнего индекса у коэффициента ряда $ h_i $ на единицу меньше компоненты $ p_j $ вектора $ p $.
	\item Во втором слагаемом правой части верхний индекс у коэффициента ряда $ h $ равен $ p $, как и в левой част, но нижний индекс на единицу меньше.
\end{itemize}

\section{Резонансные и нерезонансные объекты, теорема о формальной эквивалентности систем}

Введём в рассмотрение величины
$$ \delta_{pi}(\lambda) = \sum_{j = 1}^n p_j \lambda_j - \lambda_i = (p, \lambda) - \lambda_i = (p - e_i, \lambda) $$

Связующую систему \eref{8.12} можно теперь записать в виде
\begin{equ}{8.13}
	\delta_{pi}\nder[p]h_i + \nder[p]Y_i = \nder[p]{\vawe Y}_i,
\end{equ}
где через $ \nder[p]{\vawe Y} $ обозначена уже известная правая часть системы \eref{8.12}.

\begin{definition}
	Пара индексов $ (p, i) $ "--- \emph{резонансная}, если она удовлетворяет уравнению
	$$ \delta_{pi} = 0 $$
	Иначе "--- \emph{нерезонансная}.
\end{definition}

\begin{definition}
	$ f(z) \in \Phi_1 $

	Коэффициенты $ \nder[p]f_i $ и члены $ \nder[p]f_iz^p $, у которых пара $ (p, i) $ резонансна, а также векторный ряд $ f^0 = f_\lambda^0(z) $ с компонентами $ f_i^0 = \sum_{p : \delta_{pi} = 0} \nder[p]f_i z^p $ будем называть \emph{резонансными}.

	Остальные коэффициенты и члены ряда $ f $, а также ряд $ f_\lambda^*(z) $ с компонентами $ f_i^* = \sum_{p : \delta_{pi} \ne 0} \nder[p]f_iz^p $ "--- \emph{нерезонансные}.
\end{definition}

\begin{theorem}[о формальной эквивалентности автономных систем]
	Пусть формальная замена \eref{8.10} сводит систему \eref{8.9} к системе \eref{8.11}.

	Тогда коэффициенты рядов $ Z, Y, h $ удовлетворяют алгебраической связующей системе \eref{8.13}.

	Если заранее зафиксировать произвольным образом ряд $ h^0 $ в замене \eref{8.10} и нерезонансный ряд $ Y^* $ в системе \eref{8.11}, то нерезонансная часть $ h^* $ ряда $ h $ и резонансная часть $ Y^0 $ ряда $ Y $ определяются однозначно из уравнений
	$$ \nder[p]h_i = \delta_{pi}^{-1} \bigl( \nder[p]{\vawe Y_i} - \nder[p]Y_i \bigr), \qquad \nder[p]Y_i = \nder[p]{\vawe Y_i} \qquad \text{(пара $ (p, i) $ "--- нерезонансная)} $$
\end{theorem}

\section{Нормальная форма системы, существование, лемма о единственности младших членов}

\begin{definition}
	Система \eref{8.11} находится в \emph{нормальной форме}, если все её нерезонансные коэффициенты равны нулю, \ie ряд $ Y = Y^0 $.
\end{definition}

\begin{definition}
	Замена \eref{8.10}, преобразующая систему \eref{8.9} в НФ \eref{8.11}, называется \emph{нормализующей заменой}.

	Нормализующая замена, в которой резонансный ряд $ h^0 \equiv 0 $, называется \emph{стандартной}.
\end{definition}

\begin{theorem}[о существовании нормализующей замены]
	Для любой системы \eref{8.9} существует нормализующая замена \eref{8.10}.
	Существует и единственна нормализующая замена с любыми заданными резонансными членами, в том числе, стандартная.
\end{theorem}

\begin{lemma}[о единственности младших членов НФ]
	В НФ \eref{8.11}, полученной из системы \eref{8.9}, первый (в смысле введённой упорядоченности) член $ \nder[\breve p]Y_{\breve i}\breve y^{\breve p} $ определяется однозначно.

	Если матрица $ J $ диагональна, \ie $ \sigma_2 = \dots = \sigma_n = 0 $, то все члены порядка $ |\breve p| $ определяются однозначно.
\end{lemma}

\begin{implication}
	Никакая нелинейная НФ не может оказаться формально эквивалентной линейной НФ.
\end{implication}

\section{НФ с периодическими возмущениями: определение и вывод связующей системы}

Рассмотрим систему, в которой возмущение является периодической функцией времени:
\begin{equ}{8.18}
	\dot x = Ax + X(t, x), \qquad X = \sum_{p : |p| = 2}^\infty \nder[p]X(t)x^p
\end{equ}

Сделаем линейную замену $ x = Sz $:
\begin{equ}{8.19}
	\dot z = Jz + Z(t, z), \quad \text{ или } \dot z_i = \lambda_i z_i + \sigma_i z_{i - 1} + Z_i(t, z)
\end{equ}
Сделаем формальную почти тождественную замену
\begin{equ}{8.20}
	z_i = y_i + h_i(t, y)
\end{equ}
Получим систему
\begin{equ}{8.21}
	\dot y_i = \lambda_i y_i + \sigma_i y_{i - 1} + Y_i(t, y)
\end{equ}
Продифференцируем по $ t $ в силу этой системы:
$$ \dot h_i + \sum_{j = 1}^n \pder{h_i}{y_j}\lambda_jy_j - \lambda_ih_i + Y_i \equiv Z_i(y + h) + \sigma_i h_{i - 1} - \sum_{j = 1}^n \pder{h_i}{y_j}(\sigma_jy_{j - 1} + Y_j) $$
Приравнивая коэффициенты, получаем связующую систему:
\begin{equ}{8.22}
	\nder[p]{\dot h_i}(t) + \delta_{pi}\nder[p]h_i(t) = \nder[p]{\vawe Y_i}(t) - \nder[p]Y_i(t)
\end{equ}

\section{НФ с периодическими возмущениями: определение и условие автономности}

\begin{definition}
	Пару $ (p, i) $ будем называть $ m_\omega $-\emph{резонансной}, если
	\begin{equ}{8.23}
		\delta_{pi} + \frac{2\pi\ii m}\omega = 0
	\end{equ}

	В противном случае пара $ (p, i) $ "--- \emph{нерезонансная}.
\end{definition}

\begin{itemize}
	\item Если пара $ (p, i) $ "--- нерезонансная, то ЛНУ \eref{8.22} имеет единственное $ \omega $-периодическое решение
		\begin{equ}{8.24}
			\nder[p]h_i(t) = \bigl( q - e^{\delta_{pi}\omega} \bigr)^{-1} \int_{t - \omega}^t e^{\delta_{pi}(t - s)}\nder[p]{\vawe Y_i}(s)\di s
		\end{equ}
	\item Если пара $ (p, i) $ "--- резонансная, то получаем $ \omega $-периодическое решение
		\begin{equ}{8.26}
			\nder[p]h_i(t, t_0, y_0) = e^{\frac{2\pi\ii m(t - t_0)}\omega}y_0 + e^{\frac{2\pi\ii mt}\omega}\int_{t_0}^t\nder[p]{\breve Y_i}(s) \di s
		\end{equ}
\end{itemize}

\begin{definition}
	Коэффициенты $ \nder[p]h_i(t) $ и члены $ \nder[p]h_i(t)y^p $ у которых пара $ (p, i) $ $ m_\omega $-резонансна при некотором $ m \in \Z $, будем называть $ m_\omega $-\emph{резонансными}.

	Остальные коэффициенты и члены ряда буем называть \emph{нерезонансными}.
\end{definition}

\begin{definition}
	Систему \eref{8.21} будем называть \emph{нормальной формой}, если все её нерезонансные коэффициенты равны нулю, то есть
	$$ y_i(t, y) = \sum_{\nder[m]p}\nder[p^{(m)}]Y_i(t) y^{\nder[m]p}, $$
	где $ \nder[m]p $ удовлетворяет равенству \eref{8.23}, а $ \nder[p^{(m)}]Y_i(t) = \nder[m]a_ie^{\frac{2\pi\ii mt}\omega} $.
\end{definition}

\begin{definition}
	Замену \eref{8.20}, преобразующую систему \eref{8.19} в систему \eref{8.21} будем называть \emph{нормализующей}.

	Нормализующая замена называется \emph{стандартной}, если для всех её $ m_\omega $-резонансных коэффициентов $ \nder[p]h_i(t, t_0, y_0) $ выбраны нулевые начальные данные $ t_0, y_0 $.
\end{definition}

\begin{theorem}
	Композиция линейной замены \eref{8.6} и формальной почти тождественной нормализующей замены \eref{8.20} с $ \omega $-периодическими нерезонансными коэффициентами из \eref{8.24} и $ m_\omega $-резонансными коэффициентами из \eref{8.26} приводит систему \eref{8.18} к нормальной форме \eref{8.21}.

	При этом существует и единственна стандартная нормализующая замена \eref{8.20}.
\end{theorem}

\begin{implication}
	Пусть с. ч. $ \lambda_1, \dots, \lambda_n $ матрицы $ A $ системы \eref{8.18} таковы, что $ m_\omega $-резонансные пары с $ m \ne 0 $ отсутствуют, \ie выполняется условие
	$$ \forall m \in \Z \setminus \set{0} \quad \nder[m]p_1\lambda_1 + \dots + \nder[m]p_n\lambda_n - \lambda_i + \frac{2\pi\ii m}\omega \ne 0 $$

	Тогда НФ системы \eref{8.18} с периодической возмущённой частью автономна и совпадает с НФ \eref{8.11} системы \eref{8.1}.
\end{implication}

\section{Структура нормальной формы в критическом случае двух чисто мнимых собственных чисел}

Рассмотрим вещественную двумерную систему
\begin{equ}{8.30}
	\dot x = Ax + X(x), \quad \text{ или }
	\begin{cases}
		\dot x_1 = a_{11}x_1 + a_{12}x_2 + X_1(x_1, x_2), \\
		\dot x_2 = a_{21}x_1 + a_{22}x_2 + X_2(x_1, x_2),
	\end{cases}
\end{equ}
в которой матрица $ A $ имеет с. ч. $ \lambda_{1, 2} = \pm\ii\beta $.

Существует линейная неособая замена $ x = Sz $, которая преобразует систему \eref{8.30} в систему
\begin{equ}{8.34}
	\dot z = Jz + Z(z), \quad \text{ или }
	\begin{cases}
		\dot z_1 = \ii\beta z_1 + Z_1(z_1, z_2), \\
		\dot z_2 = -\ii\beta z_2 + Z_2(z_1, z_2)
	\end{cases}
\end{equ}
При этом можно выбрать матрицу $ S = (s, \ol s) $, тогда будет выполнено $ z_2 = \ol z_1 $.

Рассмотрим матрицу $ D =
\begin{pmatrix}
	0 & 1 \\
	1 & 0
\end{pmatrix} $.
Можно доказать, что $ \ol{Z(z)} = DZ(z) $.

Таким образом, в системе второе уравнение является сопряжённым к первому и его можно не выписывать.

Сделаем вещественную замену вида $ x = SCu $:
$$ \dot u = J_Ru + U(u), \quad \text{ или }
\begin{cases}
	\dot u_1 = -\beta u_2 + U_1(u_1, u_2), \\
	\dot u_2 = \beta u_1 + U_2(u_1, u_2)
\end{cases} $$

Снова рассмотрим систему \eref{8.34}.
Она является НФ, если в ней
$$ Z_1 = \sum_{r = 1}^\infty \nder[r + 1, r]Z_1z_1^{r + 1}z_2^r, \qquad Z_2 = \sum_{r = 1}^\infty \nder[r, r + 1]Z_2z_1^rz_2^{r + 1} $$

Предположим, что для этой НФ выполняются условия вещественности:
$$ z_2 = \ol z_1, \quad Z_2(z_1, z_2) = \ol{Z_1(z_1, z_2)} $$
При этих условиях второе уравнение системы становится сопряжённым к первому.
Поэтому
$$ \nder[r, r + 1]Z_2 = \ol{\nder[r + 1, r]Z_1} $$

В итоге в критическом случае НФ при наличии условий вещественности имеет вид
\begin{equ}{8.37}
	\dot z_1 = \ii\beta z_1 + Z_1^0(z_1, z_2)
\end{equ}
Такую НФ будем называть \emph{вещественной НФ}.

\section{Нормализующая замена, вывод связующей системы, лемма о преобразовании НФ в НФ}

Приведём систему \eref{8.34} с условиями вещественности к НФ \eref{8.37}.

Покажем, что нормализующая почти тождественная замена вида
\begin{equ}{8.38}
	y_1 = z_1 + h_1(z_1, z_2), \quad y_2 = z_2 + \ol h_1(z_2, z_1)
\end{equ}
сводит \eref{8.34} к НФ \eref{8.37}.

Дифференцируя по $ t $ первое уравнение замены в силу системы \eref{8.37}, получаем
$$ \ii\beta h_1 + Y_1(z_1 + h_1, z_2 + \ol h_1) = Z_1^0 + \pder{h_1}{z_1}(\ii\beta z_1 + Z_1^0) + \pder{h_1}{y_2}(-\ii\beta z_2 + \ol Z_1^0) $$
Приравняем коэффициенты при мономах $ z_1^{p_1}z_2^{p_2} $:
$$ \ii\beta(p_1 - p_2 - 1)\nder[p_1, p_2]h_1 + \set{Z_1^0(z_1, z_2)}^{(p_1, p_2)} = \set{Y_1(z + h)}^{(p_1, p_2)} - \nder[p_1, p_2]\Phi_1 $$
\begin{multline*}
	\nder[p_1, p_2]\Phi_1 = \sum_{r = 1}^{\frac{|p|}2 - 1}\nder[p_1 - r, p_2 - r]h_1 \Bigl( (p_1 - r)\nder[r + 1, r]Z_1 + (p_2 - r)\ol{\nder[r + 1, r]Z_1} \Bigr) = \\
	= \sum_{r = 1}^{\frac{|p|}2 - 1} \nder[p_1 - r, p_2 - r]h_1 \bigl( (p_1 + p_2 - 2r)\Re \nder[r + 1, r]Z_1 + \ii(p_1 - p_2)\Im \nder[r + 1, r]Z_1 \bigr),
\end{multline*}
и второе уравнение системы является комплексно сопряжённым к первому.

\begin{lemma}[о преобразовании НФ в НФ]
	Нормализующая замена \eref{8.38}, которая преобразует вещественную НФ в вещественную НФ, может иметь только резонансные члены.
\end{lemma}

\section{Алгебраический и трансцендентный случаи, их инвариантность; лемма о вторичной нормализации}

Множество всех вещественных НФ $ \dot y_1 = \ii\beta y_1 + Y_1^0(y_1, y_2) $ разобьём на два непересекающихся подмножества.

\begin{definition}
	Для НФ \eref{8.37} имеет место \emph{алгебраический случай}, если
	$$ \exists p_0 \ge 1 : \quad \Re \nder[r + 1, r]Y_1 = 0, \quad \Re \nder[p_0 + 1, p_0]Y_1 = a \ne 0 $$

	Если такого числа $ p_0 $ не существует, то имеет место \emph{трансцендентный случай}.
\end{definition}

В трансцендентном случае будем считать $ p_0 = +\infty $.
Обозначим через $ q_0 \in \N $ первое число такое, что $ \Im \nder[q_0 + 1, q_0]Y_1 = b $.

Запишем систему в виде
\begin{equ}{8.40}
	\begin{cases}
		\dot y_1 = y_1 \bigl( \ii\beta + Y_1(y_1, y_2) \bigr), \\
		\dot y_2 = y_2 \bigl( -\ii\beta + \ol Y_1(y_1y_2) \bigr)
	\end{cases}
\end{equ}

Согласно лемме о преобразовании НФ в НФ любая замена вида
\begin{equ}{8.41}
	\begin{cases}
		y_1 = \theta \bigl( 1 + f_1(\theta_1\theta_2) \bigr), \\
		y_2 = \theta_2 \bigl( 1 + \ol f_1(\theta_1\theta_2) \bigr)
	\end{cases}
\end{equ}
преобразует эту НФ в НФ
\begin{equ}{8.42}
	\begin{cases}
		\dot \theta_1 = \theta_1 \bigl( \ii\beta + \Theta_1(\theta_1\theta_2) \bigr), \\
		\dot \theta_2 = \theta_2 \bigl( -\ii\beta + \ol \Theta_1(\theta_1\theta_2) \bigr)
	\end{cases}
\end{equ}

\begin{lemma}[о вторичной нормализации]
	Пусть в НФ $ q-0 \le p_0 + 1 $.

	\begin{enumerate}
		\item в НФ, полученной из \eref{8.40} произвольной заменой \eref{8.41}, $ \Im \nder[q_0]\Theta_1 = b, ~ \Re \nder[r]\Theta_1 = 0 $;
		\item существует и единственна замена \eref{8.41} с $ \Im \nder[1]f_1, \dots, \Im \nder[p_0 - q_0]f_1 = 0 $ такая, что $ \Im \nder[r]\Theta_1 = 0 $ при $ r = \ol{q_0 + 1, ~ p_0 - 1} $.
	\end{enumerate}
\end{lemma}

\begin{theorem}[об инвариантности констант в алгебраическом и трансцендентном случаях]
	Пусть нормализующие замены \eref{8.38} и
	$$ z_1 = \theta_1 + \xi_1(\theta_1, \theta_2), \quad z_2 = \theta_2 + \ol\xi_1(\theta_2, \theta_1) $$
	преобразуют систему \eref{8.34} с произвольным образом зафиксированными резонансными рядами $ h_1^0 $ и $ \xi_1^0 $ соответственно в НФ \eref{8.40} и \eref{8.42}.

	Тогда эти НФ относятся либо к алгебраическому случаю с одинаковыми константами $ p_0 $ и $ a $, либо к трансцендентному случаю с одинаковыми константами $ q_0 $ и $ b $.
\end{theorem}

\section{Теорема о вторичной нормализации}

\begin{theorem}
	Существует замена
	$$ y_1 = \theta_1 \bigl( 1 + f_1(\theta_1\theta_2) \bigr), \quad y_2 = \theta_2 \bigl( 1 + f_1(\theta_1\theta_2) \bigr), \qquad f_1 = \sum_{s = 1}^\infty \nder[s]f_1(\theta_1\theta_2)^s, $$
	преобразующая НФ \eref{8.40} в НФ \eref{8.42}
	$$ \dot \theta_1 = \theta_1 \bigl( \ii\beta + \Theta_1(\theta_1\theta_2) \bigr), \quad \dot \theta_2 = \theta_2 \bigl( -\ii\beta + \ol\Theta_1(\theta_1\theta_2) \bigr), \qquad \Theta_1 = \sum_{r = r_0}^\infty \nder[r]\Theta_1(\theta_1\theta_2)^r, $$
	у которой
	\begin{itemize}
		\item в трансцендентном случае
			$$ \Theta_1 = \ii b(\theta_1\theta_2)^{q_0} $$
		\item в алгебраическом случае
			$$ \Theta_1 = a(\theta_1\theta_2)^{p_0} + \breve a(\theta_1\theta_2)^{2p_0} + \ii\sum_{k = q_0}^{p_0} \nder[k]b_1 (\theta_1\theta_2)^k $$
	\end{itemize}
\end{theorem}

\section{Теорема о сходимости нормализующей замены в трансцендентном случае}

\begin{theorem}
	В трансцендентном случае вещественная аналитическая в нуле система \eref{8.30} аналитически эквивалентна своей нормальной форме.
\end{theorem}
