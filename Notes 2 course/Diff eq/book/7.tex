\part{Теория устойчивости движения по Ляпунову}

Рассмотрим систему
\begin{equ}{7.1}
	\dot x = f(t, x),
\end{equ}
в которой $ x = (x_1, \dots, x_n) $, функция $ f = (f_1, \dots, f_n) $ определена, непрерывна и удовлетворяет локальному условию Липшица по $ x $ в области $ G = (c, +\infty) \times D $, а $ D $ "--- это область фазового пространства $ \R^n $.

\begin{definition}
	Выбранное движение $ x = \phi(t) $ системы \eref{7.1}, определённое на промежутке $ [t_0, +\infty) $, называется \emph{невозмущённым}.
	Остальные движения $ x = x(t, x^0) $ "--- \emph{возмущённые}.
	При этом $ \|x^0 - \phi(t_0)\| $ называется \emph{возмущением}.
\end{definition}

\begin{definition}
	Невозмущённое движение $ x = \phi(t) $, определённое на промежутке $ [t_0, +\infty) $, называется \emph{устойчивым по Ляпунову}, если для всякого $ \eps > 0 $ существует такое $ \delta \in (0, \eps] $, что для любого $ x^0 : \|x^0 - \phi(t_0)\| < \delta $ решение $ x = x(t, x^0) $ определено на промежутке $ [t_0, +\infty) $ и для всякого $ t \in [t_0, +\infty) $ выполняется неравенство $ \|x(t, x^0) - \phi(t)\| < \eps $.
\end{definition}

\begin{definition}
	Невозмущённое движение $ x = \phi(t) $ называется \emph{асимптотически устойчивым}, если оно устойчиво по Ляпунову и существует $ \delta_0 > 0 $ такое, что для любого $ x^0 $, при котором возмущение $ \|x^0 - \phi(t_0)\| < \delta_0 $, для решения $ x = x(t, x^0) $ выполняется условие $ \|x(t, x^0) - \phi(t)\| \underarr{t \to +\infty} 0 $.
\end{definition}

\begin{definition}
	\emph{Областью притяжения} асимптотически устойчивого движения $ x = \phi(t) $ называется множество точек $ x^0 = (x_1^0, \dots, x_n^0) $ фазового пространства таких, что если $ x^0 $ "--- точка из этого множества, то $ x(t, x^0) \underarr{t \to \infty} \phi(t) $.
\end{definition}

\begin{definition}
	Асимптотически устойчивое движение называется \emph{устойчивым в целом}, если его область притяжения совпадает со всем фазовым пространством.
\end{definition}

\section{Теорема об устойчивости линейных систем; устойчивость ЛС с постоянными и периодическими коэффициентами}

Рассмотрим систему
\begin{equ}{7.4}
	\dot y = P(t)y,
\end{equ}
где матрица $ P(t) $ определена и непрерывна на $ (c, +\infty) $.

\begin{theorem}
	ЛОС \eref{7.4} устойчива по Ляпунову \textbf{тогда и только тогда}, когда у неё имеется фундаментальная матрица, ограниченная на промежутке $ [t_0, +\infty) $.
\end{theorem}

Если $ P(t) = A $ "--- постоянная, то ФМ $ \Phi(t) = e^{At} $.
Если $ P(t) = Q(t) $ "--- $ \omega $-периодическая, то ФМ $ \Phi(t) = D(t)e^{Rt} $.

Разобьём множество векторов с. ч. $ \lambda = (\lambda_1, \dots, \lambda_n) $ матриц $ A $ и $ R $ на три непересекающихся множества:
\begin{enumerate}
	\item Если $ \lambda \in M_1 = \set{\lambda \mid \Re \lambda_k < 0} $, то найдётся число $ \lambda_0 $ такое, что $ \Re \lambda_k < \lambda_0 < 0 $.
		Поэтому по теореме об оценке нормы ФМ для любой ФМ $ \Phi(t) $ и для любого $ t_0 \in \R $ найдётся $ K > 0 $ такое, что $ \|\Phi(t)\| \le Ke^{\lambda_0t} $ для всякого $ t \in [t_0, +\infty) $.

		Из теоремы об устойчивости линейных систем выткает устойчивость систем \eref{7.4}, а также асимптотическая устойчивость таких систем.
	\item Если $ \lambda \in M_2 = \set{\lambda \mid \exists k_* : \quad \Re \lambda_{k_*} > 0} $, то из структуры ФМ вытекает, что одним из её столбцов является решение $ y(t) = s(t)e^{\lambda_*t} $, причём $ \Re \lambda_{k_*} > 0 $.
		Поэтому всегда найдётся такая последовательность $ t_k \to +\infty $, что $ \|y(t_k)\| \to +\infty $, а значит, фундаментальная матрица не ограничена и линейная система неустойчива.
	\item Если $ \lambda \in M_3 = \set{\lambda \mid \Re \lambda_k \le 0, \quad \exists k_0 : \quad \Re \lambda_{k_0} = 0} $, то система имеет решение $ y(t) = s(t)e^{\lambda_{k_0}t} = s(t) \bigl( \cos(|\lambda_{k_0}t|) + \ii \sin(|\lambda_{k_0}t) \bigr) $.
		Поэтому если векторный полином $ s(t) $ имеет ненулевую степень, то $ y(t) $ не ограничено и линейная система неустойчива.
		А если в ФМ нулевую степень имеют все полиномы $ s(t) $, то ФМ ограничена и ЛОС устойчива, но не асимптотически устойчива.
\end{enumerate}

\section{Теорема Ляпунова об устойчивости по первому приближению}

\begin{theorem}
	$ \dot y = Ay + Y(t, y), \qquad Y \in \mathcal C(G_0), \quad Y \in \op{Lip}_y^{loc}(G_0), \quad G_0 = (c, +\infty) \times D_0 $
	$$ D_0 = \set{y \mid | \|y\| < c_0}, \quad \exists t_0 > c : \quad \frac{\|Y(t, y)\|}{\|y\|} \uniarr[t \in [t_0, \infty)]{\|y\| \to 0} 0 $$
	Решение $ y(t) = 0, \quad t \in [t_0, +\infty) $ асимптотически устойчиво, если $ \Re \lambda_1, \dots, \Re \lambda_n < 0 $, и неустойчиво, если $ \exists k_* : \quad \Re \lambda_{k_*} > 0 $.
\end{theorem}

\section{Лемма о поведении положительно определённой функции Ляпунова}

Рассмотрим систему размерности $ n $, имеющую решение $ y(t) = 0 $ при $ t \in [t_0, +\infty) $:
\begin{equ}{7.8}
	\dot x = f(t, x), \quad f \in \mathcal C(G), \quad f \in \op{Lip}_y^{loc}(G), \quad G = G_\tau^a = \set{(t, x) \mid t \in (\tau, +\infty), ~ \|x\| < a}, \quad f(t, 0) \equiv[t > \tau] 0
\end{equ}

\begin{definition}
	\emph{Функцией Ляпунова} называется любая непрерывная функция $ V(t, x) : G \to \R $, если $ V(t, 0) \equiv[t > \tau] 0 $.
\end{definition}

\begin{definition}
	Функция Ляпунова называется \emph{знакопостоянной}, если существуют такие $ \tau, a $, что $ V $ не меняет знак в области $ G_\tau^a $.

	Она \emph{положительна}, если $ V(t, x) \ge 0 $, и \emph{отрицательна}, если $ V(t, x) \le 0 $.
\end{definition}

\begin{definition}
	Функция Ляпунова $ V(t, x) $ называется \emph{стационарной}, если она не зависит от $ t $, \ie $ V(t, x) \equiv W(x) $ и $ W(0) = 0 $.
\end{definition}

\begin{definition}
	Стационарная функция Ляпунова $ W(x) $ называется \emph{положительно определённой}, если существует такое число $ a > 0 $, что $ W(x) > 0 $ для всякого $ x \ne 0, ~ \|x\| < a $.
\end{definition}

\begin{definition}
	Функция Ляпунова $ V(t, x) $ называется \emph{положительно определённой}, если существует положительно определённая стационарная функция Ляпунова $ W_1(x) $ такая, что $ V(t, x) \ge W_1(x) > 0 $ при $ x \ne 0 $ в некоторой области $ G_\tau^a $, и называется \emph{отрицательно определённой}, если функция $ -V(t, x) $ положительно определена.
\end{definition}

\begin{definition}
	Функция Ляпунова $ V(t, x) $ допускает \emph{бесконечно малый высший предел}, если найдётся такая положительно определённая функция Ляпунова $ W_2(x) $, что $ |V(t, x)| \le W_2(x) $ в некоторой области $ G_\tau^a $.
\end{definition}

\begin{definition}
	Пусть $ W(x) $ "--- стационарная положительно определённая функция Ляпунова.

	Замкнутое множество $ W(x) = C $ называется \emph{поверхностью уровня}, а при $ n = 2 $ "--- \emph{линией уровня}.
\end{definition}

\begin{lemma}[о поведении положительно определённой функции Ляпунова]
	Пусть функция Ляпунова $ V(t, x) $ положительно определена в области $ G_\tau^a $, функция $ x(t) $ непрерывна и $ \|x(t)\| \le a_1 < a_1 $ при любом $ t \in [t_0, +\infty) $.
	\begin{enumerate}
		\item если $ V \bigl( t, x(t) \bigr) \underarr{t \to +\infty} 0 $, то $ \|x(t)\| \to 0 $;
		\item если $ V(t, x) $ допускает бесконечно малый высший предел и $ \|x(t)\| \underarr{t \to +\infty} 0 $, то $ V \bigl( t, x(t) \bigr) \to 0 $.
	\end{enumerate}
\end{lemma}

\section{Теорема Ляпунова об устойчивости, исследование уравнения \texorpdfstring{$ \ddot x + g(x) = 0 $}{x'' + g(x) = 0}}

Будем предполагать, что функция Ляпунова $ V \in \mathcal C^1(G_\tau^a) $, и на множестве таких функций введём линейный дифференциальный оператор $ \mathcal D $:
$$ \mathcal DV = \pder Vt + \pder Vf, \quad \text{ или } \mathcal Dv(t, x) = \pder{V(t, x)}t + \sum_{i = 1}^n \pder{V(t, x)}{x_i}f_i(t, x) $$

\begin{theorem}
	Пусть в области $ G_\tau^a $ существует положительно определённая функция Ляпунова $ V(t, x) $, у которой $ \mathcal DV(t, x) $ отрицательна.

	Тогда в системе \eref{7.8} невозмущённое движение устойчиво по Ляпунову.
\end{theorem}

\begin{implication}
	Если система \eref{7.8} имеет в области $ G $ положительно определённый интеграл $ U(t, x) $ и $ U(t, 0) \equiv 0 $, то невозмущённое движение $ x(t) \equiv 0 $ устойчиво по Ляпунову.
\end{implication}

\begin{eg}
	Рассмотрим уравнение
	\begin{equ}{7.10}
		\ddot x + g(x) = 0,
	\end{equ}
	где функция $ g(x) $ определна, непрерывна и удовлетворяет условию Липшица при $ |x| < a, ~ g(0) = 0, ~ xg(x) > 0 $.

	Уравнение имеет решение $ x(t) \equiv 0 $ и при $ g(x) = \sin x $ описывает колебания математического маятника.

	Сделаем замену $ x = x_1, ~ \dot x = x_2 $, получая автономную систему
	\begin{equ}{7.11}
		\dot x_1 = x_2, \quad \dot x_2 = -g(x_1)
	\end{equ}
	Решение $ x = 0 $ уравнения \eref{7.10} устойчиво по Ляпунову, асимптотически устойчиво или неустойчиво одновременно с тривиальным решением системы \eref{7.11}.

	Рассмотрим функцию Ляпунова
	$$ W(x_1, x_2) = \int_0^{x_1} g(s) \di s + \frac12 x_2^2 $$
	Эта функция стационарна и положительно определена, поскольку $ x_1g(x_1) > 0 $ при $ x \ne 0 $.
	$$ \mathcal Dw = g(x_1) \dot x_1 + x_2 \dot x_2 \equiv 0 $$
	Следовательно, $ W(x_1, x_2) $ "--- интеграл, который определяет полную энергию системы \eref{7.11}, тождество $ \mathcal DW \equiv 0 $ означает закон сохранения энергии.

	Положение равновесия $ x(t) = 0 $ устойчиво, но не асимптотически устойчиво.
\end{eg}

\section{Теорема Ляпунова об асимптотической устойчивости, исследование уравнения \texorpdfstring{$ \ddot x + h(x) \dot x + g(x) = 0 $}{x'' + h(x) x' + g(x) = 0}}

\begin{theorem}
	Пусть в области $ G_\tau^a $ существует положительно определённая функция Ляпунова $ V(t, x) $, допускающая бесконечно малый выскший предел, а её производная в силу системы $ \mathcal DV(t, x) $ отрицательно определена.

	Тогда невозмущённое движение $ x(t) \equiv 0 $ системы \eref{7.8} асимптотически устойчиво.
\end{theorem}

\begin{eg}
	Рассмотрим уравнение
	\begin{equ}{7.12}
		\ddot x + h(x) \dot x + g(x) = 0,
	\end{equ}
	где функции $ g(x), h(x) $ определены, непрерывны и удовлетворяют локальному условию Липшица при $ |x| < a, ~ g(0) = 0, ~ xg(x) > 0 $ при $ x \ne 0 $, $ h(x) > 0 $ при $ 0 < |x| < a $.
	Сделаем замену $ x = x_1, ~ \dot x = x_2 $, получая автономную систему
	\begin{equ}{7.13}
		\dot x_1 = x_2, \quad \dot x_2 = -g(x) - h(x_1)x_2
	\end{equ}
	Возьмём стационарную положительно определённую функцию Ляпунова из предыдущего примера:
	$$ W(x_1, x_2) = \int_0^{x_1}g(s) \di s + \frac12 x_2^2 $$
	По определению она допускает БМВП.

	Продифференцируем $ W $ в силу системы \eref{7.13}:
	$$ \mathcal DW(x_1, x_2) = g(x_1)x_2 + x_2 \bigl( -g(x_1) - h(x_1)x_2) \bigr) = -h(x_1)x_2^2 \le 0 $$
	Поскольку $ \mathcal DW(x_1, 0) \equiv 0 $, производная $ \mathcal DW $ не является отрицательно определённой, и применить теорему нельзя.
\end{eg}

\section{Теорема Ляпунова об асимптотической устойчивости для автономных систем, исследование уравнения \texorpdfstring{$ \ddot x + h(x) \dot x + g(x) = 0 $}{x'' + h(x)x' + g(x) = 0}}

Если система автономна, то она имеет вид
\begin{equ}{7.14}
	\dot x = f(x), \qquad f \in \mathcal C(D), \quad f \in \op{Lip}_x^{loc}(D), \quad f(0) = 0, \quad D = D_{-\infty}^a = \set{x \mid \|x\| < a}
\end{equ}

\begin{theorem}
	Пусть система \eref{7.14} в области $ D $ обладает положительно определённой функцией Ляпунова $ W(x) $, у которой $ \mathcal DW(x) \le 0 $, а множество $ M^a = \set{ x \mid 0 < \|x\| < a, ~ \mathcal DW(x) = 0} $ не содержит целых траекторий.

	Тогда невозмущённое движение $ x(t) \equiv 0 $ этой системы асимптотически устойчиво.
\end{theorem}

\begin{eg}[продолжение]
	Итак, $ \mathcal DW = -h(x_1)x_2^2 \equiv[x_2 = 0, ~ |x_1| \le a_1] 0 $.

	Во втором уравнении системы \eref{7.13} имеем $ 0 = -g(x_1) \ne 0 $ при $ x_1 \ne 0 $.
	Следовательно, $ x_2 \equiv 0 $ не является решением \eref{7.13}, и множество $ M $ не содержит целых траекторий, отличных от точки покоя $ x_1 = x_2 = 0 $.

	Теперь по теореме тривиальное решение системы \eref{7.13} асимптотически устойчиво.
\end{eg}

\section{Теорема Ляпунова об устойчивости в целом для автономных систем, исследование уравнения \texorpdfstring{$ \ddot x + h(x) \dot x + g(x) = 0 $}{x'' + h(x)x' + g(x) = 0}}

\begin{definition}
	Стационарная функция Ляпунова $ W(x) $ называется \emph{бесконечно большой}, если
	$$ W(x) \underarr{\|x\| \to +\infty} \infty $$
\end{definition}

\begin{theorem}
	Пусть у автономной системы \eref{7.14} в области $ D^\infty = \R^n $ существует положительно определённая ББ функция Ляпунова $ W(x) $ с $ \mathcal DW(x) \le 0 $, а множество $ M = \set{x \mid x \ne 0, ~ \mathcal DW(x) = 0 } $ не содержит целых траекторий.

	Тогда невозмущённое движение $ x(t) \equiv 0 $ системы устойчиво в целом.
\end{theorem}

\begin{eg}[продолжение]
	Предположим дополнительно, что в уравнении \eref{7.12}
	$$ \int_0^{x_1}g(s) \di s \underarr{|x_1| \to \infty} \infty $$
	Это условие выполняется, например, для любой функции $ g(x) \ge \frac1x $ при $ x > 0 $.

	Тогда стационарная положительно определённая функция Ляпунова
	$$ W(x_1, x_2) = \int_0^{x_1} g(s)\di s + \frac12 x_2^2 $$
	будет ББ.

	Следовательно, по теореме невозмущённое движение $ x = 0 $ уравнения \eref{7.12} устойчиво в целом.
\end{eg}
