% Hyphenation doesn't want to work fine here, have to imitate\dots
\part[Уравнения первого порядка, разрешённые относительно производной]{Уравнения первого порядка, разрешённые относи-\\тельно производной}

Рассматриваем уравнение
\begin{equ}{1.1}
	\frac{\di y(x)}{\di x} = f \bigl( x, y(x) \bigr), \quad \text{ или } y' = f(x, y)
\end{equ}

\section{Продолжимость решения на границу и за границу; теорема о продолжимости решения за границу}

\begin{definition}
	Пусть $ y = \phi(x) $ "--- решение уравнения \eref{1.1} на $ \braket{a, b} $.
	Если этот промежуток произвольным образом сузить, то на новом промежутке функция $ y = \phi(x) $ останется решением, которое называют \emph{сужением} исходного решения
\end{definition}

\begin{definition}
	Решение уравнения \eref{1.1}, заданное на промежутке $ \langle a, b) $ \emph{продолжимо вправо в точку} $ b $ или \emph{на границу}, если найдётся такое решение $ y = \vawe{\phi}(x) $, определённое на промежутке $ \langle a, b] $, что сужение $ \vawe{\phi}(x) $ на $ \langle a, b) $ совпадает с $ \phi(x) $
\end{definition}

\begin{definition}
	Решение уравнения \eref{1.1}, заданное на промежутке $ \braket{a, b} $ \emph{продолжимо вправо за точку} $ b $ или \emph{за границу}, если найдутся такие $ \vawe{b} > b $ и решение $ y = \vawe{\phi}(x) $, определённое на промежутке $ \braket{a, \vawe{b}} $, что сужение $ \vawe{\phi}(x) $ на $ \braket{a, b} $ совпадает с $ \phi(x) $
\end{definition}

\begin{theorem}
	$ \phi(x) $ "--- решение уравнения \eref{1.1} на промежутке $ \langle a, b), \quad b < +\infty $ \\
	Для того чтобы это решение было продолжимо вправо в точку $ b $ необходимо и достаточно, чтобы существовали последовательность $ \seq{x_k}k $ и число $ \eta \in \R^1 $ такие, что
	$$ \forall k \quad
	\begin{cases}
		x_k \in \langle a, b) \\
		\bigg( x_k, \phi(x_k) \bigg) \underarr{k \to \infty} (b, \eta) \in \vawe{G}
	\end{cases} $$
\end{theorem}

\section{Продолжимость решения на границу и за границу; леммы о продолжимости решения за границу отрезка и интервала}

\begin{lemma}[о продолжимости решения за границу отрезка]
	Пусть решение $ y = \phi(x) $ уравнения \eref{1.1} определено на промежутке $ \langle a, b] $ и точка $ \big( b, \phi(b) \big) \in G $ \\
	Тогда это решение продолжимо вправо за точку $ b $ на полуотрезок Пеано, построенный для точки $ \big( b, \phi(b) \big) $
\end{lemma}

\begin{lemma}[о продолжимости решения за границу интервала]
	Пусть решение $ y = \phi(x) $ уравнения \eref{1.1} определено на промежутке $ \langle a, b) $, существует число $ \eta = \liml{x \to b^-}\phi(x) $ и точка $ (b, \eta) \in G $ \\
	Тогда это решение продолжимо вправо за точку $ b $
\end{lemma}

\section{Теорема о поведении интегральной кривой полного внутреннего решения}

\begin{definition}
	Решение называется \emph{полным}, \emph{максимально продолженным}, или \emph{непродолжимым} в случае, если его нельзя продолжить ни влево, ни вправо.
\end{definition}

\begin{definition}
	Внутреннее (граничное) решение называется \emph{полным}, если его нельзя продолжить ни влево, ни вправо так, чтобы оно осталось внутренним (граничным)
\end{definition}

\begin{definition}
	Промежуток, на котором определено полное решение, будем называть \emph{максимальным интервалом существования} и обозначим $ I_{\max} $, а если для полного решения была поставлена задача Коши с начальными данными $ x_0, y_0 $, то $ I(x_0, y_0) $
\end{definition}

\begin{definition}
	График полного решения будем называть \emph{интегральной кривой} уравнения \eref{1.1} \\
	\emph{Дуга интегральной кривой} "--- это график решения, заданного на любом промежутке $ \braket{a, b} \subsetneq I_{\max} $
\end{definition}

\begin{theorem}[о поведении интегральной кривой полного внутреннего решения]
	Предположим, что внутреннее решение $ y = \phi(x) $ уравнения \eref{1.1} определено на промежутке $ \langle a, \beta) $ и не продолжимо вправо. \\
	Тогда для любого компакта $ \ol{H} \sub G $ найдётся такое число $ \delta \in \langle a, \beta) $, что для всякого $ x \in (\delta, \beta) $ точка $ \big( x, \phi(x) \big) \in G \setminus \ol{H} $
\end{theorem}

\begin{restate}
	При стремлении аргумента полного внутреннего решения к границе максимального интервала существования дуга интегральной кривой покидает любой компакт, лежащий в области $ G $, и никогда в него не возвращается
\end{restate}

\section{Ломаные Эйлера; лемма о ломаных Эйлера в роли \texorpdfstring{$ \eps $}{эпсилон}-решения}

\begin{definition}
	Точка $ (x_0, y_0) \in \vawe{G} $ называется точкой \emph{неединственности}, если существуют такие решения $ y = \phi_1(x) $ и $ y = \phi_2(x) $ задачи Коши уравнения \eref{1.1} с начальными данными $ x_0, y_0 $, определённые на промежутке $ \braket{a, b} $, и такая последовательность $ x_k \infarr{k} x_0 $, $ x_k \in \braket{a, b} $, что $ \phi_1(x_k) \ne \phi_2(x_k) \quad (k = 1, 2, \dots) $ \\
	В противном случае точка $ (x_0, y_0) $ называется точкой единственности
\end{definition}

\begin{definition}
	Точку $ (x_0, y_0) \in \vawe{G} $ будем называть \emph{точкой единственности} в следующих случаях:
	\begin{enumerate}
		\item задача Коши уравнения \eref{1.1} с начальными данными $ x_0, y_0 $ не имеет решений
		\item для любых двух решений $ y = \phi_1(x) $ и $ y = \phi_2(x) $ этой задачи Коши, определённых на некотором промежутке $ \braket{a, b} $, найдётся интервал $ (\alpha, \beta) \ni x_0 $ такой, что
		      $$ \forall x \in (\alpha, \beta) \cap \braket{a, b} \quad \phi_1(x) = \phi_2(x) $$
	\end{enumerate}
\end{definition}

\begin{definition}
	\emph{Общим решением} уравнения \eref{1.1} на некотором связном множестве $ A^* $, лежащем в области единственности $ G^\circ $, называется функция $ y = \phi(x, C) $, определённая и непрерывная по совокупности аргументов на множестве $ Q_{A^*} = \set{(x, C) | x \in \braket{a(C), b(C)}, \quad C \in \braket{C_1, C_2}} $, если выполняются следующие два условия:
	\begin{enumerate}
		\item для любой точки $ (x_0, y_0) \in A^* $ уравнение $ y_0 = \phi(x_0, C) $ имеет единственное решение $ C = C_0 $
		\item функция $ y = \phi(x, C_0) $ "--- это решение задачи Коши уравнения \eref{1.1} с начальными данными $ x_0, y_0 $, определённое на промежутке $ \braket{a(C_0), b(C_0)} $
	\end{enumerate}
\end{definition}

\begin{figure}[!ht]
	\centering
	\includegraphics[width=0.445\textwidth]{euler-polylines}
\end{figure}

Выберем в области $ G $ произвольную точку $ (x_0, y_0) $ и построим в ней отрезок поля направлений столь малой длины, что он целиком лежит в $ G $, начинаясь в какой-то точке $ (x_{-1}, y_{-1}) $ и заканчиваясь в точке $ (x_1, y_1) $ \\
Проведём вправо через точку $ (x_1, y_1) $ и влево через точку $ (x_{-1}, y_{-1}) $ полуотрезки поля, лежащие в $ G $ и заканчивающиеся в точках $ (x_2, y_2) $ и $ (x_{-2}, y_{-2}) $ соответственно, и так далее \\
График полученной таким образом непрерывной кусочно"=линейной функции $ y = \psi(x) $ называется \emph{ломаной Эйлера} \\
Итак, установлено, что ломаная Эйлера лежит в области $ G $, проходит через точку $ (x_0, y_0) $ и абсциссы её угловых точек равны $ x_j $ ($ j = \ol{-N, N} $)

\begin{definition}
	\emph{Рангом дробления} ломаной Эйлера назовём число, равное
	$$ \max\limits_{j = \ol{1 - N, N}}\set{x_j - x_{j - 1}} $$
\end{definition}

Формула, рекуррентно задающая ломаную Эйлера $ y = \psi(x) $, имеет вид: $ \psi(x_0) = y_0 $ и далее при $ j = 0, 1, \dots, N - 1 $ для любого $ x \in (x_j, x_{j + 1}] $ или при $ j = 0, -1, \dots, 1 - N $ для любого $ x \in [x_{j - 1}, x_j) $
\begin{equ}{1.8}
	\psi(x) = \psi(x_j) + f \big( x_j, \psi(x_j) \big)(x - x_j)
\end{equ}
В частности, при $ j = 0 $ отрезок ломаной Эйлера определён для любого $ x \in [x_{-1}, x_1] $ и, делясь на два полуотрезка, проходит через точку $ (x_0, y_0) $ под углом, тангенс которого равен $ f(x_0, y_0) $ \\
Понятно, что для всякого $ j = \ol{0, N - 1} $ производная $ \psi'(x) = f \big( x_j, \psi(x_j) \big) $ при $ x \in (x_j, x_{j + 1}) $.

Доопределим $ \psi'(x) $ в точках разрыва как левостороннюю производную при $ x > x_0 $ и как правостороннюю производную при $ x < x_0 $, положив
$$ \psi'(x_j) = \psi_{\mp}'(x_j) \liml{x \to x_j^{\mp0}}\frac{\psi(x) - \psi(x_j)}{x - x_j} \qquad (j = \pm 1. \dots, \pm N) $$
А при $ j = 0 $ существует полная производная $ \psi'(x_0) = f(x_0, y_0) $ \\
Таким образом, для любого $ x \in (x_j, x_{j + 1}] $ ($ j = 0, 1, \dots, N - 1 $) или для любого $ x \in [x_{j - 1}, x_j) $ ($ j = 0, -1, \dots, 1 - N $), дифференцируя равенство \eref{1.8} по $ x $, получаем
\begin{equ}{1.9}
	\psi'(x) = f \big( x_j, \psi(x_j) \big), \qquad j \in \set{1 - N, \dots, N - 1}
\end{equ}

\begin{definition}
	Для всякого $ \eps > 0 $ непрерывная и кусочно-гладкая на отрезке $ [a, b] $ функция $ y = \psi(x) $ называется $ \eps $-решением уравнения \eref{1.1} на $ [a, b] $, если для любого $ x \in [a, b] $ точка $ \big( x, \psi(x) \big) \in G $ и
	\begin{equ}{1.10}
		\big| \psi'(x) - f \big( x, \psi(x) \big) \big| \le \eps
	\end{equ}
\end{definition}

\begin{lemma}[о ломаных Эйлера в роли $ \eps $-решения]
	Для любой точки $ (x_0, y_0) \in G $ и для любого отрезка Пеано $ \ol{P_h}(x_0, y_0) $ имеем:
	\begin{enumerate}
		\item Для любого $ \delta > 0 $ на $ \ol{P_h} $ можно построить ломаную Эйлера $ y = \psi(x) $ с рангом дробления, не превосходящим $ \delta $, график которой лежит в прямоугольнике $ \ol{R} $.
		\item Для любого $ \eps > 0 $ найдётся такое $ \delta > 0 $, что всякая ломаная Эйлера $ y = \psi(x) $ с рангом дробления, не превосходящим $ \delta $, является $ \eps $-решением уравнения \eref{1.1} на $ \ol{P_h}(x_0, y_0) $.
	\end{enumerate}
\end{lemma}

\section{Лемма Асколи\texorpdfstring{"--~}{--}Арцела}

\begin{lemma}
	Из любой ограниченной и равностепенно непрерывной на $ [a, b] $ последовательности функций можно выделить равномерно сходящуюся на $ [a, b] $ подпоследовательность.
\end{lemma}

\section{Ломаные Эйлера; теорема Пеано о существовании внутреннего решения}

\begin{theorem}
	Пусть правая часть уравнения \eref{1.1} непрерывна в области $ G $.

	Тогда для любой точки $ (x_0, y_0) \in G $ и для любого отрезка Пеано $ \ol{P_h}(x_0, y_0) $ существует по крайней мере одно решение \caupr[\eref{1.1}]{x_0, y_0}, определённое на $ \ol{P_h}(x_0, y_0) $
\end{theorem}

\section{Теорема о существовании решения для одного из случаев \texorpdfstring{$ U_1^+ $, $ O_1^+$, $ B_{1<}^+ $, $ B_{1=}^+ $}{U1+, O1+, B1<+, B1=+}}

Для упрощения обозначений и формул, используемых в дальнейшем при решении граничной задачи Коши, НУО будем считать, что задача всегда ставится в начале координат и функция $ f $ там равна нулю, \ie уравнение \eref{1.1} имеет вид
\begin{equ}{1.12}
	y' = f_0(x, y),
\end{equ}
где функция $ f_0 $ определена и непрерывна на множестве $ \vawe{G} = G \cup \hat{G} $, точка $ O = (0, 0) \in \vawe{G} $, $ f_0(0, 0) = 0 $ и поставлена граничная задача Коши с начальными данными $ 0, 0 $.

\begin{definition}
	Функцию $ y = b_{a, u}^+(x) $ заданную на отрезке $ [0, a] $ будем называть \emph{верхнеграничной}, если для неё выполняются следующие пять условий:
	\begin{enumerate}
		\item $ b_{a, u}^+(x) \in \Cont[1]{[0, a]} $;
		\item $ b_{a, u}^+(0) = 0 $;
		\item $ b_{a, u}^+{}'(0) \ge 0 $;
		\item $ b_{a, u}^+ $ вогнута на $ [0, a] $, если $ b_{a, u}^+{}'(0) = 0 $;
		\item \emph{правая верхнеграничная кривая} $ \gamma_{a, u}^+ = \set{x \in [0, a], \quad y = b_{a, u}^+(x)} \sub \hat{G} $.
	\end{enumerate}
\end{definition}
Аналогично вводится правая нижнеграничная функция $ y = b_{a, l}^+(x) $, и правая нижнеграничная кривая $ \gamma_{a, l}^+ $.

Введём две ключевые константы:
$$ \tau_u = \frac{b_{a, u}^+{}'(0)}2, \qquad \tau_u = 1 \text{, если } b_{a, u}^+{}'(0) = +\infty $$
$$ \tau_l = -\frac{b_{a, l}^+{}'(0)}2, \qquad \tau_l = -1 \text{, если } b_{a, l}^+{}'(0) = -\infty $$
НУО будем считать, что выполняются условия:
$$ \begin{cases}
	b_{a, u}^+(a) \le a \quad \text{при } \tau_u = 0 \\
	\forall x \in [0, a] \quad b_{a, u}^+{}'(x) \ge \tau_u \quad \text{при } \tau_u > 0
\end{cases} \quad
\begin{cases}
	-b_{a, l}^+(a) \le a \quad \text{при } \tau_l = 0 \\
	\forall x \in [0, a] \quad -b_{a, l}^+{}'(x) \ge \tau_l \quad \text{при } \tau_l > 0
\end{cases} $$

В результате $ \gamma_{a, u}^+ $ "--- гладкая кривая из $ \hat{G} $, параметризованная неубывающей функцией $ b_{a, u}^+(x) $.
Она расположена в первой четверти и содержит точку $ O $.
$ \gamma_{a, l}^+ $ "--- гладкая кривая из $ \hat{G} $, параметризованная невозрастающей функцией $ b_{a, l}^+(x) $.
Она расположена в четвёртой четверти и содержит точку $ O $.

Для всякого $ c > 0 $ рассмотрим правую $ c $-\textit{окрестность} точки $ O $:
$$ N_c^+ \define \set{(x, y) | x \in (0, c], \quad |y| \le c} $$
В прямоугольнике $ N_c^+ $ длина верхней стороны выбирается так, чтобы каждая ``выходящая'' из точки $ O $ правая граничная кривая, при наличии хотя бы одной, имела пересечение с одной из его сторон. При этом ``поведение'' граничных кривых после первого попадания на границу $ N_c^+ $ интереса не представляет. \\
Любое последующее уменьшение $ c $ ситуацию не меняет, разве что отсекаются части граничных кривых, попавших в прямоугольник снаружи. \\
В частности, неравенства $ b_{a, u}^+(a) \le a $ или $ -b_{a, l}^+(a) \le a $ при всех $ c \le a $ гарантируют пересечение правых граничных кривых $ \gamma_a^+ $ именно с боковой стороной прямоугольника $ N_c^+ $. \\
Всегда в дальнейшем, ``обрезая'' при необходимости кривые $ \gamma_a^+ $, будем считать, что правый конец $ \big( a, b_a^+(a) \big) $ любой из них "--- это первая точка выхода граничной кривой на границу прямоугольника $ N_c^+ $ \\
Выделим для уравнения \eref{1.12} четыре варианта расположения граничных кривых в малой окрестности точки $ O $ при $ x > 0 $:
\begin{definition}
	Будем говорить, что
	\begin{enumerate}
		\item реализуется случай $ (W^+) $, если
		      $$ \exists c_W > 0 : \quad W_{c_W}^+ \cap \hat{G} = \O, \qquad W_{c_W}^+ = N_{c_W}^+ $$
		\item реализуется случай $ (U^+) $, если
		      $$ \exists c_U > 0 : \quad U_{c_U}^+ \cap \hat{G} = \gamma_{a, u}^+ \setminus \set{O} $$
		      $$ U_{c_U}^+ \define \set{(x, y) | \bigg( x \in (0, a], \quad -c_u \le y \le b_{a, u}^+(x) \bigg) \cup \bigg( x \in (a, c_U], \quad y \le c_U \bigg)} $$
		\item реализуется случай $ (O^+) $, если
		      $$ \exists c_O > 0 : \quad O_{c_O}^+ \cap \hat{G} = \gamma_{a, l}^+ \setminus \set{O} $$
		      $$ O_{c_O}^+ \define \set{(x, y) | \bigg( x \in (0, a], \quad b_{a, l}^+(x) \le y \le c_O \bigg) \cup \bigg( x \in (a, c_O], \quad |y| \le c_O \bigg)} $$
		\item реализуется случай $ (B^+) $, если
		      $$ \exists c_B > 0 : B_{c_B}^+ \cap \hat{G} = \big( \gamma_{a, u}^+ \cup \gamma_{a, l}^+ \big) \setminus \set{O}, \qquad B_{c_B}^+ \define U_{c_B}^+ \cap O_{c_B}^+ $$
	\end{enumerate}
\end{definition}

В случае $ (X^+) $ или $ (X_{c*}^+) $ имеет место одна из двух возможностей:
\begin{enumerate}
	\item $ X_{c*}^+ \cap G \ne \O $, что равносильно тому, что $ X_{c*}^+ $ без входящих в него граничных кривых лежит в $ G $
	\item $ X_{c*}^+ \cap G = \O $
\end{enumerate}

В результате случай $ (X^+) $ в зависимости от расположения множества $ X_{c*} $ распадается на два подслучая, которые будем обозначать $ (X_1^+) $ и $ (X_2^+) $ \\
А дополнительный индекс $ >, =, < $, при его наличии в обозначении любого из шести возникших случаев (кроме $ (W_1^+) $ и $ (W_2^+) $), будет уточнять знак производной соответсвующих правых граничных функций в нуле \\
В итоге, получаются случаи:
\begin{enumerate}
	\item[$ \bm{U_1^+} $:] $ (U_{c_U}^+ \setminus \gamma_{a, u}^+) \sub G $, два подслучая:
	      \begin{enumerate}
		      \item[$ \bm{U_1^{+, >}} $:] $ b_{a, u}^+{}'(0) > 0 $
		      \item[$ \bm{U_1^{+, =}} $:] $ b_{a, u}^+{}'(0) = 0 $
	      \end{enumerate}
	\item[$ \bm{U_2^+} $:] $ U_{c_U}^+ \cap G = \O $, подслучаи те же
	\item[$ \bm{O_1^+} $:] $ (O_{c_O}^+ \setminus \gamma_{a, l}^+) \sub G $, два подслучая:
	      \begin{enumerate}
		      \item[$ \bm{O_{1, <}^+} $:] $ b_{a, l}^+{}'(0) < 0 $
		      \item[$ \bm{O_{1, =}^+} $:] $ b_{a, l}^+{}'(0) = 0 $
	      \end{enumerate}
	\item[$ \bm{O_2^+} $:] $ O_{c_O}^+ \cap G = \O $, подслучаи те же
	\item[$ \bm{B_1^+} $:] $ \bigg(B_{c_B}^+ \setminus (\gamma_{a, u}^+ \cup \gamma_{a, l}^+) \bigg) \sub G $, четыре подслучая:
	      \begin{enumerate}
		      \item[$ B_{1, <}^{+, >} $:] $ b_{a, u}^+{}'(0) > 0, \quad b_{a, l}^+{}'(0) < 0 $
		      \item[$ B_{1, =}^{+, =} $:] $ b_{a, u}^+{}'(0) = 0, \quad b_{a, l}^+{}'(0) = 0 $
		      \item[$ B_{1, =}^{+, >} $:] $ b_{a, u}^+{}'(0) > 0, \quad b_{a, l}^+{}'(0) = 0 $
		      \item[$ B_{1, <}^{+, =} $:] $ b_{a, u}^+{}'(0) = 0, \quad b_{a, l}^+{}'(0) < 0 $
	      \end{enumerate}
	\item[$ \bm{B_2^+} $:] $ B_{c_B}^+ \cap G = \O $, подслучаи те же
\end{enumerate}

\begin{figure}[!ht]
	\includegraphics[width=0.99\textwidth]{boundary_curves_1} \\
	\includegraphics[width=0.99\textwidth]{boundary_curves_2}
\end{figure}

Введём ограничения на функцию $ f_0 $ в случаях $ U_1^{+, =} $, $ O_{1, =}^+ $, $ B_{1, =}^{+, =} $, $ B_{1, =}^{+, >} $ и $ B_{1, <}^{+, =} $:
\begin{equ}{1.15+}
	\forall x \in (0, a] \quad
	\begin{cases}
		f_0 \bigl( x, b_{a, u}^+(x) \bigr) \le b_{a, u}^+{}'(x), \quad \text{ если } b_{a, u}^+{}'(0) = 0, \\
		f_0 \bigl( x_0, b_{a, l}^+(x) \bigr) \ge b_{a, l}^+{}'(x), \quad \text{ если } b_{a, l}^+{}'(0) = 0
	\end{cases}
\end{equ}

\begin{theorem}[о существовании решения граничной задачи Коши]
	Предположим, что в уравнении \eref{1.12} функция $ f_0 $ определена и непрерывна на множестве $ \vawe{G} $.

	Тогда в каждом из случаев $ (N_1^+), (U_1^{+, >}), (O_{1, <}^+), (B_{1, <}^{+, >}) $ и в каждом из случаев $ (U_1^{+, =}), (O_{1, =}^+) $, $ (B_{1, =}^{+, =}) $, $ (B_{1, =}^{+, >}), (B_{1, <}^{+, =}) $ при условиях \eref{1.15+} на любом правом граничном отрезке Пеано существует по крайней мере одно решение граничной задачи Коши с начальными данными $ (0, 0) $
\end{theorem}

\begin{theorem}[об отсутствии решений граничной задачи Коши]
	В каждом из случаев $ (U_2^{+, >}) $, $ (O_{2, <}^+) $, $ (B_{2, <}^{+, >}) $, $ (N_2^+) $ граничная задача Коши с начальными данными $ (0, 0) $ не имеет решений в правой полуплоскости
\end{theorem}

\section{Лемма о продолжимости решений на отрезок Пеано; лемма о верхнем и нижнем решениях}

\begin{lemma}[о продолжимости решений на отрезок Пеано]
	Пусть $ y = \phi(x) $ "--- это решение внутренней задачи Коши с начальными данными $ x_0, y_0 $, определённое на $ \ol{P_h}(x_0, y_0) $. \\
	Тогда любое другое решение уравнения \eref{1.1} $ y = \psi(x) $ этой же задачи Коши, определённое на промежутке $ \braket{a, b} \subsetneq [x_0 - h, x_0 + h] $, продолжимо на $ \ol{P_h}(x_0, y_0) $
\end{lemma}

Пусть $ (x_0, y_0) \in G, \quad \ol{P_h}(x_0, y_0) $ "--- некий отрезок Пеано и $ \seq{\chi_k(x)}k $ "--- произвольная последовательность решений ЗК($ x_0, y_0 $) уравнения \eref{1.1}, определённых на $ [x_0 - h, x_0 + h] $

\begin{statement}
	Для любых $ k \in \N, \quad x \in [x_0 - h, x_0 + h] $ функции
	$$ \chi_k^l(x) \define \min\set{\chi_1(x), \dots, \chi_k(x)}, \qquad \chi_k^u(x) \define \max\set{\chi_1(x), \dots, \chi_k(x)} $$
	также являются решениями поставленной задачи на $ \ol{P_h}(x_0, y_0) $
\end{statement}

\begin{lemma}[о нижнем и верхнем решениях]
	Существуют решения ЗК($ x_0, y_0 $) $ y = \chi^l(x) $ и $ y = \chi^u(x) $ уравнения \eref{1.1} такие, что
	$$ \forall k \in \N \quad \forall x \in [x_0 - h, x_0 + h] \quad
		\begin{cases}
			\chi^l(x) \le \chi_k^l(x) \\
			\chi^u(x) \ge \chi_k^u(x)
		\end{cases} $$
\end{lemma}

\section{Теорема о локальной единственности решения внутренней задачи Коши}

\begin{theorem}
	Пусть $ (x_0, y_0) \in G $ "--- это точка единственности.

	Тогда решение ЗК($ x_0, y_0 $) уравнения \eref{1.1} является локально единственным
\end{theorem}

\section{Лемма Гронуолла}

\begin{lemma}
	Пусть функция $ h(x) \in \Cont{\braket{a, b}} $ и существуют такие $ x_0 \in \braket{a, b}, \quad \lambda \ge 0, \quad \mu > 0 $, что
	$$ \forall x \in \braket{a, b} \quad 0 \le h(x) \le \lambda + \mu \bigg| \dint[s]{x_0}x{h(s)} \bigg| $$

	Тогда для любого $ x \in \braket{a, b} $ справедливо неравенство
	$$ h(x) \le \lambda e^{\mu|x - x_0|} $$
\end{lemma}

\section{Условия Липшица; теорема о множестве единственности}

\begin{definition}
	Функция $ f(x, y) $ удовлетворяет \emph{условию Липшица} по $ y $ \emph{глобально} на множестве $ D \sub \R^2 $, если
	\begin{equ}{1.22}
		\exists L > 0 : \quad \forall (x, y_1), (x, y_2) \in D \quad |f(x, y_1) - f(x, y_2)| \le L|y_1 - y_2|
	\end{equ}
\end{definition}

\begin{notation}
	$ f \in \operatorname{Lip}_y^{gl}(D) $
\end{notation}

\begin{definition}
	Функция $ f(x, y) $ удовлетворяет \emph{условию Липшица} по $ y $ \emph{локально} на множестве $ \vawe{G} $, если для любой точки $ (x_0, y_0) \in \vawe{G} $ найдётся замкнутая $ c $-окрестность $ \ol{B}_c(x_0, y_0) $ такая, что функция $ f $ удовлетворяет условию Липшица по $ y $ глобально на множестве $ U_c = \vawe{G} \cap \vawe{B}_c(x_0, y_0) $
\end{definition}

\begin{notation}
	$ y \in \operatorname{Lip}_y^{loc}(\vawe{G}) $
\end{notation}

\begin{theorem}[о множестве единственности]
	Пусть в уравнении \eref{1.1} функция $ f(x, y) $ определена и непрерывна на множестве $ \vawe{G} $ и удовлетворяет условию Липшица по $ y $ локально на множестве $ \vawe{G^\circ} = G^\circ \cup \hat{G^\circ} $.

	Тогда $ \vawe{G^\circ} $ "--- множество единственности для уравнения \eref{1.1}
\end{theorem}

\begin{theorem}[о множестве единственности; слабая]
	Предположим, что в уравнении \eref{1.1} функция $ f(x, y) $ определена и непрерывна на множестве $ \vawe{G} $, функция $ \pder{f(x, y)}y $ определена и непрерывна в области $ G^\circ \sub G $.

	Тогда множество $ \vawe{G^\circ} = G^\circ \cup \hat{G^\circ} $, является множеством единственности для уравнения \eref{1.1}, если при этом для любой точки $ (x_0, y_0) \in \hat{G^\circ} $ найдётся $ \ol{B}_c(x_0, y_0) $ такая, что множество $ \vawe{G^\circ} \cap \ol{B}_c(x_0, y_0) $ выпукло по $ y $
\end{theorem}

\section{Теорема Осгуда}

\begin{theorem}
	Пусть в уравнении \eref{1.1} функция $ f(x, y) $ непрерывна в области $ G $ и
	$$ \forall ~ (x, y_1), (x, y_2) \in G \quad |f(x, y_2) - f(x, y_1)| \le h \big( |y_2 - y_1| \big) $$
	где функция $ h(s) $ определена, непрерывна и положительна для всякого $ s \in (0, +\infty) $ и
	$$ \dint[s]\eps{a}{h^{-1}(s)} \underarr{\eps \to 0} \infty, \qquad a > \eps > 0 $$
	Тогда $ G $ "--- это область единственности для уравнения \eref{1.1}
\end{theorem}

\section{Область существования общего решения, лемма о поведении в ней решений, формула общего решения}

Опишем множество $ A^* $, в котором можно построить общее решение, поскольку гарантировать его существование во всей области единственности $ G^\circ $ нельзя, какой бы малой она ни была \\
В этом параграфе в роли $ A^* $ будет выступать вводимый ниже компакт $ \ol{A} $

\begin{algorithm}[построения $ \ol{A} $]
	Пусть $ G^\circ $ "--- область единственности для уравнения \eref{1.1} \\
	Возьмём любую точку $ (x_0^*, y_0^*) \in G^\circ $ \\
	Поскольку $ G^\circ $ является открытым множеством, существует такое $ \delta > 0 $, что $ \ol{B}_{2\delta}(x_0^*, y_0^*) \sub G^\circ $ \\
	Пусть числа $ y_1, y_2 $ таковы, что
	$$
		\begin{cases}
			0 < y_0^* - y_1 < \delta \\
			0 < y_2 - y_0^* < \delta
		\end{cases} $$
	и найдётся отрезок $ [a, b] \ni x_0^* $ такой, что графики решений ЗК($ x_0^*, y_1 $) $ y = \phi_1(x) $ и ЗК($ x_0^*, y_2 $) $ y = \phi_2(x) $ лежат в $ \ol{B}_c $ при $ x \in [a, b] $. Тогда в $ \ol{B}_\delta $ содержится компакт
	$$ \ol{A} = \set{(x, y) | a \le x \le b, \quad \phi_1(x) \le y \le \phi_2(x)} $$
\end{algorithm}

\begin{lemma}
	Для любой точки $ (x_0, y_0) \in \ol{A} $ решение \caupr[\eref{1.1}]{x_0, y_0} $ y = \phi(x) $ продолжимо на отрезок $ [a, b] $
\end{lemma}

Для любой точки $ (x_0, y_0) \in \ol{A} $ обозначим через $ y = y(x, x_0, y_0) $ решение \caupr[\eref{1.1}]{x_0, y_0} \\
Тогда $ y(x_0, x_0, y_0) = y_0 $, и по лемме о поведении решений на компакте решение $ y = (x, x_0, y_0) $ определено для всякого $ x \in [a, b] $ \\
Для произвольной точки $ \zeta \in [a, b] $ рассмотрим функцию
$$ \phi(x, C) = y(x, \zeta, C), \qquad (\zeta, C) \in \ol{A} $$
на прямоугольнике $ \ol{Q} = \ol{Q}_{\ol{A}} \define \set{(x, C) | a \le x \le b, \quad \phi_1(\zeta) \le C \le \phi_2(\zeta)} $, который является частным случаем множества $ Q_{A^*} $ из определения общего решения.

В самом деле, $ \phi_1(\zeta) \le C \le \phi_2(\zeta) $ по построению $ \ol{A} $. А по лемме решение $ y = y(x, \zeta, C) $ определено для любого $ x \in [a, b] $ и при $ x = \zeta $ по определению решения ЗК $ \phi(\zeta, C) = y(\zeta, \zeta, C) = C $

\begin{theorem}
	Введённая функция $ y = \phi(x, C) $ является общим решением уравнения \eref{1.1} на компакте $ \ol{A} $, построенном в окрестности произвольной точки из области единственности $ G^\circ $
\end{theorem}

\section{Формула общего решения, теорема о дифференцируемости общего решения}

\begin{definition}
	Общее решение $ y = \phi(x, C) $ будем называть \emph{общим решением в форме Коши} или \emph{классическим общим решением} уравнения первого порядка \eref{1.1}
\end{definition}

\begin{theorem}
	Пусть на компакте $ \ol{A} $ при некотором $ \zeta \in [a, b] $ задано общее решение $ y = \phi(x, C) $, и в уравнении \eref{1.1} $ f(x, y) $ непрерывно дифференцируема по $ y $ в некоторой окрестности $ \ol{A} $
	$$ \implies \quad \forall (x, C) \in \ol{Q} : \quad \pder{\phi(x, C)}x = \exp \bigg( \dint[t]\zeta{x}{\pder{f \big( t, \phi(t, C) \big) }y} \bigg) $$
\end{theorem}
