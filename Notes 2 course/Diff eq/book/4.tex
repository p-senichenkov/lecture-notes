\part{Линейные уравнения высокого порядка}

Рассмотрим \emph{линейное дифференциальное уравнение} порядка $ n $ ($n \ge 2 $), разрешённое относительно старшей производной
\begin{equ}{4.1}
	\nder y + p_1(x) \nder[n - 1]y + \dots + p_{n - 1}(x)y' + p_n(x)y = q(x),
\end{equ}
в котором функции $ p_j(x), q(x) \quad (j = \ol{1, n}) $ определены и непрерывны на некотором интервале $ (a, b) $ и принимают на нём вещественные значения.

\section{ЛОУ порядка \texorpdfstring{$ n $}{n}, существование ФСР, общее решение, овеществление ФСР}

\begin{definition}
	Линейное уравнение \eref{4.1} называется \emph{однородным}, если функция $ q(x) \equiv[(a, b)] 0 $.

	В противном случае \eref{4.1} "--- \emph{неоднородное}, при этом функция $ q(x) $ называется \emph{неоднородностью}.
\end{definition}

Далее будем рассматривать линейное однородное уравнение
\begin{equ}{4.3}
	\nder[n]y + p_1(x)\nder[n - 1]y + \dots + p_{n - 1}(x)y' + p_n(x)y = 0
\end{equ}

\begin{definition}
	\emph{Фундаментальной системой решений} ЛОУ называются любые $ n $ ЛНЗ на $ (a, b) $ решений этого уравнения.
\end{definition}

\begin{theorem}
	Фундаментальная система решений ЛОУ существует.
\end{theorem}

\begin{theorem}[об общем решении ЛОУ]
	Если $ \psi_1(x), \dots, \psi_n(x) $ "--- фундаментальная система решений на $ (a, b) $.

	Тогда
	$$ \psi(x, C_1, \dots, C-n) = C_1\psi_1(x) + \dots + C_n\psi_n(x) $$
	является общим решением ЛОУ в области $ G = (a, b) \times \R^n $.
\end{theorem}

Пусть $ y(x) = u(x) + \ii v(x) $ "--- решение уравнения \eref{4.3}, тогда $ u(x), v(x) $ "--- также решения ЛОУ.

\begin{lemma}[об овеществлении ФСР]
	Предположим, что набор функций
	$$ \Theta_1 = \set{\phi_1(x), \ol \phi_1(x), \phi_l(x), \ol \phi_l(x), \phi_{2l + 1}(x), \dots, \phi_n(x)}, \quad 1 \le l \le \frac n2, $$
	где $ \phi_j = u_j + \ii v_j $, а $ \phi_{2l + 1}, \dots, \phi_n $ вещественны, образует ФСР ЛОУ \eref{4.3}.

	Тогда набор $ \Theta_2 = \set{u_1(x), v_1(x), \dots, u_l(x), v_l(x), \phi_{2l + 1}(x), \dots, \phi_n(x)} $ является вещественной ФСР.
\end{lemma}

\section{Построение ЛОУ по фундаментальной системе решений, его единственность}

\begin{theorem}
	Пусть $ \phi_1(x), \dots, \phi_n(x) $ "--- набор из $ n $ раз гладких на $ (a, b) $ функций и построенный по ним определитель Вронского $ W(x) \ne 0 $ для любого $ x \in (a, b) $.

	Тогда существует и единственно ЛОУ, для которого $ \phi_1(x), \phi_n(x) $ "--- фундаментальная система решений на $ (a, b) $.
\end{theorem}

\section{Формула Лиувилля для ЛОУ}

\begin{undefthm}{Дифференцирование определителя.}
	Если матрица $ \Phi(x) = \set{\phi{ij}(x)}_{i, j = 1}^n $, то
	$$ |\Phi(x)|' =
	\begin{bNiceMatrix}
		\phi_{11}' & \Cdots & \phi_{1n}' \\
		\phi_{21} & \Cdots & \phi_{2n} \\
		\Vdots & \Ddots & \Vdots \\
		\phi_{n1} & \Cdots & \phi_{nn}
	\end{bNiceMatrix} +
	\begin{bNiceMatrix}
		\phi_{11} & \Cdots & \phi_{1n} \\
		\phi_{21}' & \Cdots & \phi_{2n}' \\
		\Vdots & \Ddots & \Vdots \\
		\phi_{n1} & \Cdots & \phi_{nn}
	\end{bNiceMatrix} + \dots +
	\begin{bNiceMatrix}
		\phi_{11} & \Cdots & \phi_{1n} \\
		\phi_{21} & \Cdots & \phi_{2n} \\
		\Vdots & \Ddots & \Vdots \\
		\phi_{n1}' & \Cdots & \phi_{nn}'
	\end{bNiceMatrix} $$
\end{undefthm}

Следовательно,
$$ W'(x) =
\begin{bNiceMatrix}
	\phi_1 & \Cdots & \phi_n \\
	\phi_1' & \Cdots & \phi_n' \\
	\Vdots & \Ddots & \Vdots \\
	\nder[n - 1]\phi_1 & \Cdots & \nder[n - 1]\phi_n
\end{bNiceMatrix}' =
\begin{bNiceMatrix}
	\phi_1' & \Cdots & \phi_n' \\
	\phi_1' & \Cdots & \phi_n' \\
	\Vdots & \Ddots & \Vdots \\
	\nder[n - 1]\phi_1 & \Cdots & \nder[n - 1]\phi_n
\end{bNiceMatrix} + \dots +
\begin{bNiceMatrix}
	\phi_1 & \Cdots & \phi_n \\
	\phi_1' & \Cdots & \phi_n' \\
	\Vdots & \Ddots & \Vdots \\
	\nder \phi_1 & \Cdots & \nder \phi_n
\end{bNiceMatrix} = -r_1(x), $$
в силу того, что все определители, кроме последнего, имеют по две одинаковые строки.

В итоге получили ЛОУ первого порядка $ W'(x) = -p_1(x) W(x) $.
Разделяя переменные и интегрируя по $ s $, имеем
$$ -\int_{x_0}^x p_1(s) \di s = \int_{x_0}^x \frac{\di W(s)}{W(s)} = \ln \frac{W(x)}{W(x_0)}, $$
откуда получаем \emph{формулу Лиувилля}:
$$ W(x) = W(x_0) e^{-\int_{x_0}^x p_1(s) \di s} $$
Таким образом, определитель Вронского зависит только от коэффициента $ p_1(x) $.

\section{Структура общего решения ЛНУ, метод вариации произвольной постоянной}

Рассматриваем ЛНУ \eref{4.1} порядка $ n $:
$$ \nder y + p_1(x) \nder[n - 1]y + \dots + p_{n - 1}(x) y' + p_n(x) y = q(x), \quad \text{ или } Ly = q $$

Сделаем ``сдвигающую'' замену $ y = z + \psi(x) $.
$$ L(z + \psi) = q \iff Lz + L \psi = q \iff Lz = 0 $$
Пусть $ z = \phi(x, C_1, \dots, C_n) $ "--- общее решение ЛОУ $ Lz = 0 $.
Это значит, что $ y = \phi(x, C_1, \dots, C_n) $ "--- общее решение ЛОУ \eref{4.3} $ Ly = 0 $.
Подставляя его в замену, устанавливаем, что функция
$$ y = \phi(x, C_1, \dots, C_n) + \psi(x) $$
является общим решением ЛНУ \eref{4.1}, \ie общее решение ЛНУ есть сумма общего решения ЛОУ и частного решения ЛНУ.

\begin{theorem}[о нахождении частного решения ЛНУ]
	Предположим, что набор $ \phi_1(x), \dots, \phi_n(x) $ "--- это фундаментальная система решений ЛОУ на интервале $ (a, b) $.

	Тогда частное решение $ y = \psi(x) $ ЛНУ может быть найдено в виде квадратур от $ \phi_1(x), \dots, \phi_n(x) $, коэффициентов $ p_1(x), \dots, p_n(x) $ и неоднородности $ q(x) $.
\end{theorem}

\section{ФСР для ЛОУ с постоянными коэффициентами, примеры}

ЛОУ  постоянными коэффициентами $ p_1(x), \dots, p_n(x) $ имеет вид
\begin{equ}{4.3c}
	L^cy = \nder y + a_1 \nder[n - 1]y + \dots + a_{n - 1} y' + a_ny = 0, \quad a_1, \dots, a_n \in \R,
\end{equ}
и любое его решение продолжимо на всю вещественную ось.

Подставим в уравнение функцию $ y = e^{\lambda x} $ с произвольным показателем $ \lambda $.
Чтобы $ e^{\lambda x} $ оказалось решением, должно выполняться тождество
\begin{equ}{4.11}
	L^ce^{\lambda x} = (\lambda^n + a_1\lambda^{n - 1} + \dots + a_{n - 1}\lambda + a_n) e^{\lambda x} \equiv 0
\end{equ}

\begin{definition}
	Функция $ g(\lambda) = \lambda^n + a_1\lambda^{n - 1} + \dots + a_{n - 1}\lambda + a_n $ называется \emph{характеристическим многочленом} линейного однородного уравнения \eref{4.3c}.
	Его нули называются \emph{характеристическими числами}.
\end{definition}

\begin{theorem}[о ФСР ЛОУ с постоянными коэффициентами]
	Пусть характеристическое уравнение ЛОУ \eref{4.3c} имеет корни $ \lambda_1, \dots, \lambda_m $ кратностей $ k_1, \dots, k_m $.

	Тогда ФСР уравнения \eref{4.3c} имеет вид
	$$ e^{\lambda_1x}, xe^{\lambda_1x}, \dots, x^{k_1 - 1}e^{\lambda_1x}, \dots, e^{\lambda_mx}, \dots, x^{k_m - 1}e^{\lambda_mx} $$
\end{theorem}

\begin{eg}
	Рассмотрим ЛОУ второго порядка
	$$ y'' + \beta^2y = 0 $$
	$$ \chi = \lambda^2 + \beta^2, \quad \lambda_{1,2} = \pm\ii\beta $$
	ФСР образуют $ e^{\ii\beta x}, e^{-\ii\beta x} $, вещественную ФСР "--- $ \cos \beta x, ~ \sin \beta x $.

	Общее вещественное решение имеет вид
	$$ y = C_1 \cos \beta x + C_2 \sin \beta x $$
\end{eg}

\section{Метод неопределённых коэффициентов для ЛНУ с постоянными коэффициентами}

Метод неопределённых коэффициентов можно применять только в том случае, когда неоднородность $ q $ имеет специальный вид: $ q(x) = p(x)e^{\lambda x} $, где $ \lambda \in \Co $, а $ p(x) $ "--- многочлен, возможно, с комплексными коэффициентами.
Итак, рассмотрим ЛНУ с постоянными коэффициентами
\begin{equ}{4.15}
	L^cy = p(x) e^{\lambda x}, \quad p(x) = \sum_{t = 0}^l p_tx^t
\end{equ}

\begin{definition}
	Говорят, что в ЛНУ \eref{4.15} имеет место \emph{резонанс кратности} $ k_j $, если $ \lambda = \lambda_j $ ($ j = \ol{1, m}) $, где $ \lambda_j $ "--- это нуль характеристического многочлена оператора $ L^c $ кратности $ k_j $.

	\emph{Резонанс отсутствует}, если $ \lambda \ne \lambda_j $ для любого $ j = \ol{1, m} $.
\end{definition}

\begin{theorem}[о построении решения ЛНУ методом неопределённых коэффициентов]
	Пусть показатель $ \lambda $ из правой части \eref{4.15} совпадает с корнем характеристического уравнения кратности $ k $ ($ k = \ol{0, n} $).

	Тогда существует и единственно частное решение ЛНУ \eref{4.15}
	$$ \psi(x) = x^kr(x)e^{\lambda x}, \quad r(x) = \sum_{s = 0}^l r_sx^s, \quad (r_l \ne 0) $$
\end{theorem}
