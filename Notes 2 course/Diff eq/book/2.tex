\part{Уравнения первого порядка в симметричной форме}

Уравнение первого порядка в симметричной форме имеет вид
\begin{equ}{2.1}
	M(x, y) \di x + N(x, y) \di y = 0,
\end{equ}
где функции $ M $ и $ N $ определены и непрерывны на связном множестве $ \vawe B = B \cup \hat B \cup \mathcal B $.

\section{Определение интеграла, теорема о характеристическом свойстве интеграла}

\begin{definition}
	Непрерывную в области $ B \sub \R^2 $ функцию $ U(x, y) $ будем называть допустимой, если для любой точки $ (x_0, y_0) \in B $ найдётся такая непрерывная функция $ y = \xi(x) $ или $ x = \eta(y) $, определённая на интервале $ (\alpha, \beta) $, содержащем точку $ x_0 $ или $ y_0 $, что:
    \begin{enumerate}
    	\item $ y_0 = \xi(x_0) $ или $ x_0 = \eta(y_0) $
        \item точка $ \big( x, \xi(x) \big) \in B $ для любого $ x \in (\alpha, \beta) $ или \\
        точка $ \big( \eta(y), y \big) \in B $ для любого $ y \in (\alpha, \beta) $
        \item $ y = \xi(x) $ или $ x = \eta(y) $ "--- единственное решение уравнения $ U(x, y) = U(x_0, y_0) $.
    \end{enumerate}
\end{definition}

\begin{definition}
    Допустимая функция $ U(x, y) $ называется \emph{интегралом} уравнения \eref{2.1} в области единственности $ B^\circ $, если для любой точки $ (x_0, y_0) \in B^\circ $ единственная функция $ y = \xi(x) $ или $ x = \eta(y) $ из определения допустимой функции "--- это решение \caupr[\eref{2.1}]{x_0, y_0} на $ (\alpha, \beta) $.
\end{definition}

\begin{theorem}[о характеристическом свойстве интеграла]
    Для того чтобы допустимая функция $ U(x, y) $ была интегралом уравнения в симметричной форме \eref{2.1} в области единственности $ B^\circ $, \textbf{необходимо и достаточно}, чтобы $ U(x, y) $ обращалась в постоянную вдоль любого решения \eref{2.1}, \ie чтобы:
    \begin{itemize}
        \item $ U \big( x, \phi(x) \big) \overset{\braket{a, b}}\equiv C $ для любого решения $ y = \phi(x) $, определённого на $ \braket{a, b} $
        \item $ U \big( \psi(y), y \big) \overset{\braket{a, b}}\equiv C $ для любого решения $ x = \phi(y) $, определённого на $ \braket{a, b} $
    \end{itemize}
\end{theorem}

\section{Определение гладкого интеграла, теорема о характеристическом свойстве гладкого интеграла}

\begin{definition}
	Гладкую функцию $ U(x, y) $ будем называть гладкой допустимой в области $ B $, если $ U_x'^2 + U_y'^2 > 0 $ для любой точки $ (x, y) \in B $
\end{definition}

\begin{definition}
    Интеграл $ U(x, y) $ уравнения \eref{2.1} будем называть гладким, если $ U $ "--- гладкая допустимая функция
\end{definition}

\begin{theorem}[о характеристическом свойстве гладкого интеграла]
    Для того чтобы гладкая допустимая функция $ U(x, y) $ была гладким интегралом уравнения \eref{2.1} в области единственности $ B^\circ $, \textbf{необходимо и достаточно}, чтобы выполнялось тождество
	$$ N(x, y) U_x'(x, y) - M(x, y)U_y'(x,y) \overset{B^\circ}\equiv 0 $$
\end{theorem}

\begin{implication}
    Гладкая допустимая функция $ U(x, y) $ есть гладкий интеграл уравнения \eref{1.1} $ y' = f(x, y) $ в области единственности $ G^\circ $ \textbf{тогда и только тогда}, когда верно тождество
    $$ U_x'(x, y) + f(x, y)U_y'(x, y) \overset{G^\circ}\equiv 0 $$
\end{implication}

\section{Теоремы о существовании гладкого интеграла и о связи между интегралами}

\begin{definition}
    $ U(x, y) $ "--- интеграл уравнения \eref{2.1} в области единственности $ B^\circ $ \\
    Тогда равенство $ U(x, y) = C $ называется \emph{общим интегралом} уравнения \eref{2.1}
\end{definition}

\begin{theorem}[о существовании гладкого интеграла]
    В уравнении \eref{2.1} функции $ M(x, y), ~ N(x, y) \in \Cont[1]B $ \\
    Тогда для любой точки $ (x_0, y_0) $ из области $ B $ существует её окрестность $ A \sub B $, в которой уравнение \eref{2.1} имеет гладкий интеграл $ U(x, y) $
\end{theorem}

\begin{theorem}[о связи между интегралами]
    $ U(x, y) $ "--- интеграл уравнения \eref{2.1} в некоторой области $ A $. \\
    Тогда:
    \begin{enumerate}
    	\item если $ U_1(x, y) $ "--- ещё один интеграл в $ A $, то $ \exists \Phi(x) : \quad U_1(x, y) \equiv[A] \Phi \bigl( U(x, y) \bigr) $;
        \item если функция $ \Phi \big( U(x, y) \big) $ допустима, то $ U_1(x, y) \equiv[A] \Phi \big( U(x, y) \big) $ "--- это интеграл уравнения \eref{2.1} в области $ A $.
    \end{enumerate}
\end{theorem}

\section{Теорема об интеграле уравнения с разделяющимися переменными}

\begin{definition}
    Уравнением с разделяющимися переменными в симметрической форме будем называть уравнение \eref{2.1} вида
    \begin{equ}{2.9}
    	g_1(x)h_2(y)\di x + g_2(x)h_1(y)\di y = 0
    \end{equ}
    в котором $ g_1(x), ~ g_2(x) \in \Cont{\braket{a, b}}, \quad h_1(y), ~ h_2(y) \in \Cont{\braket{c, d}} $, причём
	$$ (a, b) \setminus (g_1^\circ \cup g_2^\circ) = \bigcup_{k = 1}^{k_*}(a_k, b_k), \qquad (c, d) \setminus (h_1^\circ \cup h_2^\circ) = \bigcup_{l = 1}^{l_*}(c_l, d_l) $$
	$$ \forall x \in (a, b) \quad g_1^2(x) + g_2^2(x) \ne 0, \qquad \forall y \in (c, d) \quad h_1^2(y) + h_2^2(y) \ne 0, $$
    где $ g_i^\circ = \set{x \in \braket{a, b} \mid g_i(x) = 0}, \quad h_i^\circ = \set{y \in \braket{c, d} \mid h_i(y) = 0} $ "--- замкнутые множества нулей функций $ g $ и $ h $
\end{definition}

\begin{theorem}[об интеграле уравнения с разделяющимися переменными]
    Любая область $ B_{kl} $, для которой выполнено
	$$ \forall (x, y) \in B_{kl} \quad g_2(x) \ne 0, \quad h_2(y) \ne 0, \quad g_1^2(x) + h_1^2(x) \ne 0, $$
	является областью единственности уравнения \eref{2.9}, и в ней функция $ U(x, y) $ является гладким интегралом уравнения \eref{2.9}.
\end{theorem}

\section{Теорема об интеграле уравнения в полных дифференциалах; теорема об уравнении в полных дифференциалах, локальная}

\begin{definition}
    Уравнение \eref{2.1} называется \emph{уравнением в полных дифференциалах} в области $ B $, если существует функция $ U(x, y) \in \Cont[1]B $ такая, что для всякой точки $ (x, y) \in B $,
	$$ U_x'(x, y) = M(x, y), \qquad U_y'(x, y) = N(x, y) $$
\end{definition}

\begin{theorem}[об интеграле УПД]
    $ U(x, y) $ "--- это гладкий интеграл УПД в $ B $
\end{theorem}

\begin{theorem}[об УПД; локальная]
    Предположим, что для уравнения \eref{2.1} выполняются условия:
    \begin{enumerate}
        \item прямоугольник $ R = \set{(x, y) \mid x \in (a, b), \quad y \in (c, d)} \sub B $;
        \item в $ B $ существуют и непрерывны частные производные $ M_y', N_y' $;
        \item верно тождество $ M_y'(x, y) - N_x'(x, y) \equiv 0 $;
    \end{enumerate}

	Тогда \eref{2.1} "--- УПД в $ R $, и для любых $ x_0, x \in (a, b), \quad y_0, y \in (c, d) $ его интегралами являются функции
	$$ U_1(x, y) = \dint[s]{x_0}x{M(s, y_0)} + \dint[s]{y_0}y{N(x, s)} $$
	$$ U_2(x, y) = \dint[s]{x_0}x{M(s, y)} + \dint[s]{y_0}y{N(x_0, s)} $$
\end{theorem}

\section[Теоремы о существовании и нахождении интегрирующего множителя, решение линейного уравнения при помощи интегрирующего множителя]
{Теоремы о существовании и нахождении интегрирующего\\ множителя, решение линейного уравнения при помощи интегрирующего множителя}

\begin{definition}
    Функция $ \mu(x, y) $, определённая, непрерывная и не обращающаяся в ноль в области $ B $, называется \emph{интегрирующим множителем} дифференциального уравнения \eref{2.1}, если уравнение
	$$ \mu(x, y) M(x, y) \di x + \mu(x, y)N(x, y)\di y = 0 $$
    является УПД в $ B $.
\end{definition}

\begin{theorem}[о существовании интегрирующего множителя]
    Если в области единственности $ B^\circ \sub B $ уравнение \eref{2.1} имеет гладкий интеграл, тогда в $ B^\circ $ существует интегрирующий множитель.
\end{theorem}

\begin{theorem}[о нахождении интегрирующего множителя]
	Пусть нашлась такая $ \omega(x, y) \in \Cont[1]B $, что
	$$ \frac{M_y'(x, y) - N_x'(x, y)}{\omega_x'(x, y)N(x, y) - \omega_y'(x, y)M(x, y)} = \psi(\omega) $$

	Тогда уравнение \eref{2.1} имеет интегрирующий множитель $ \mu(\omega) = e^{\int \psi(\omega) \di \omega} $ .
\end{theorem}
