\part{Нормальные системы ОДУ}

Нормальная система дифференциальных уравнений порядка $ n $ имеет вид
\begin{equ}{3.1}
	y_1' = f_1(x, y_1, \dots, y_n), \\
	\dots \\
	y_n' = f_n(x, y_1, \dots, y_n)
\end{equ}

\section{Лемма о связи между локальным и глобальным условиями Липшица, достаточные условия для выполнения локального условия Липшица}

\begin{definition}
	Функция $ f(x, y) $ удовлетворяет \emph{условию Липшица глобально} по $ y $ на множестве $ B \sub G $, если найдётся такая константа $ L = L_B > 0 $, что
    \begin{equ}{3.8}
        \forall (x, \vawe y), (x, \hat y) \in B \quad \norm{f(x, \hat y) - f(x, \vawe y)} \le L \norm{\hat y - \vawe y}
    \end{equ}
\end{definition}

\begin{notation}
    $ f \in \operatorname{Lip}_y^{gl}(B) $
\end{notation}

\begin{definition}
    Функция $ f(x, y) $ удовлетворяет \emph{условию Липшица локально} по $ y $ в области $ G $, если для любой точки $ (x_\circ, y^\circ) \in G $ существуют окрестность $ V(x_\circ, y^\circ) \sub G $ и константа Липшица $ L = L_V > 0 $ такие, что для любых двух точек $ (x, \vawe y), (x, \hat y) \in V(x_\circ, y^\circ) $ выполняется неравенство \eref{3.8}.
\end{definition}

\begin{notation}
    $ f \in \operatorname{Lip}_y^{loc}(G) $
\end{notation}

\begin{lemma}[о связи между локальным и глобальным условиями Липшица]
    Если $ f(x, y) \in \operatorname{Lip}_y^{loc}(G) $, то для любого компакта $ \ol H \sub G $ выполнено $ f(x, y) \in \operatorname{Lip}_y^{gl}(\ol H) $
\end{lemma}

\begin{lemma}[о достаточном условии локальной липшицевости]
	Если вектор-функция $ f(x, y) $ непрерывна вместе со своими частными производными по $ y_1, \dots, y_n $ в области $ G $, то она удовлетворяет условию Липшица по $ y $ локально в $ G $.
\end{lemma}

\section{Теорема Пикара}

Зафиксируем произвольную точку $ (x_\circ, y^\circ) \in G $.

\begin{figure}[!ht]
    \centering
	\includegraphics{pikar_approximations}
\end{figure}

В качестве нулевого приближения возьмём функцию $ y^{(0)}(x) \equiv y^\circ $. Очевидно, что она определена для любого $ x \in \R $, но возможно не при всех значениях аргумента точка $ \big( x, y^{(0)}(x) \big) $ окажется в области $ G $. Однако существует интервал $ (\alpha_1, \beta_1) $ такой, что $ x_\circ \in (\alpha_1, \beta_1) $ и для всякого $ x \in (\alpha_1, \beta_1) $ точка $ \big( x, y^{(0)}(x) \big) \in G $, а значит, функция $ f \big( x, y^{(0)}(x) \big) $ определена и непрерывна на $ (\alpha_1, \beta_1) $.

Теперь в качестве первого пикаровского приближения можно выбрать функцию
$$ y^{(1)}(x) \define y^\circ + \dint[s]{x_0}x{f \big( s, y^{(0)}(s) \big)}, $$
и оно определено и непрерывно как композиция непрерывных функций на $ (\alpha_1, \beta_1) $.

Но, опять-таки, возможно не при всех $ x $ точка $ \big( x, y^{(1)}(x) \big) $ попадёт в область $ G $. В этом случае $ (\alpha_1, \beta_1) $ придётся уменьшить.

Существует интервал $ (\alpha_2, \beta_2) \sub (\alpha_1, \beta_1) $ такой, что $ x_0 \in (\alpha_2, \beta_2) $ и для всякого $ x \in (\alpha_2, \beta_2) $ точка $ \big( x, y^{(1)}(x) \big) \in G $, а значит, функция $ f \big( x, y^{(1)}(x) \big) $ определена и непрерывна на $ (\alpha_2, \beta_2) $. И так далее.

$$ \dots $$

Предположим, что пикаровское приближение $ y^{(k)}(x) $ определено и непрерывно на некотором интервале $ (\alpha_k, \beta_k) \ni x_\circ $, и $ y^{(k)}(x_\circ) = y^\circ $. Тогда существует такой интервал $ (\alpha_{k + 1}, \beta_{k + 1}) \sub (\alpha_k, \beta_k) $, что $ x_\circ \in (\alpha_{k + 1}, \beta_{k + 1}) $ и для всякого $ x \in (\alpha_{k + 1}, \beta_{k + 1}) $ точка $ \big( x, y^{(k)}(x) \big) \in G $.

Введём $ (k + 1) $-е приближение по Пикару:

\begin{equ}{3.10}
    y^{(k + 1)}(x) = y^\circ + \dint[s]{x_0}x{f \big( s, y^{(k)}(s) \big)}.
\end{equ}

Оно определено и непрерывно на интервале $ (\alpha_{k + 1}, \beta_{k + 1}) $.

Таким образом каждое пикаровское приближение определено в некоторой окрестности точки $ x_\circ $ и $ y^{(k)}(x_\circ) = y^\circ $ при любом $ k \ge 0 $.

\begin{theorem}[Пикара]
    $ f(x, y) \in \Cont G, \quad f(x, y) \in \operatorname{Lip}_y^{loc}(G) $

    Для любой точки $ (x_\circ, y^\circ) \in G $ последовательные приближения Пикара $ y^{(k)}(x) $ ($ k = 0, 1, \dots) $ с начальными данными $ x_\circ, y^\circ $ определены на некотором отрезке $ [\alpha, \beta] $, причём существует такой компакт $ \ol H \sub G $, что для любых $ k \ge 0 $ и $ x \in [\alpha, \beta] $ точка $ \big( x, y^{(k)}(x) \big) \in \ol H $.

    Тогда функции $ y^{(k)}(x) $ равномерно относительно $ [\alpha, \beta] $ стремятся при $ k \to \infty $ к предельной функции $ y(x) $, являющейся решением \caupr[\eref{3.1}]{x_\circ, y^\circ} на отрезке $ [\alpha, \beta] $.
\end{theorem}

\section{Теорема о существовании и единственности решений нормальной системы}

\begin{theorem}
    Пусть в системе \eref{3.1} $ f(x, y) $ непрерывна и $ f \in \operatorname{Lip}_y^{loc}(G) $.

    Тогда для любой точки $ (x_0, y^0) \in G $ и для любого отрезка Пеано $ P_h(x_0, y^0) $ на этом отрезке существует и единственно решение \caupr{x_0, y^0}.
\end{theorem}

\begin{implication}
	$ G $ является областью единственности.
\end{implication}

\section{Линейные системы, теоремы о существовании, единственности и продолжимости решений линейных систем}

\begin{definition}
    Система \eref{3.1} называется \emph{линейной}, если она имеет вид
    \begin{equ}{3.14}
        \begin{cases}
            y_1' = p_{11}(x)y_1 + \dots + p_{1n}(x)y_n + q_1(x) \\
            \dots \\
            y_n' = p_{n1}(x)y_1 + \dots + p_{nn}(x)y_n + q_n(x)
        \end{cases}
    \end{equ}
    или в векторной записи
    $$ y' = P(x)y + q(x) $$
    где функции $ p_{ij}(x) $ и $ q_i(x) \in \Cont{(a, b)} $.
\end{definition}

\begin{restate}
	Нормальная система является \emph{линейной}, если $ f(x, y) = P(x)y + q(x) $, а $ G = (a, b) \times \R^n $.
\end{restate}

\begin{definition}
    Линейная система \eref{3.14} называется \emph{однородной (ЛОС)}, если в ней $ q(x) \equiv[(a, b)] 0 $. \\
    В противном случае система называется \emph{неоднородной (ЛНС)}.

    Функция $ q(x) $ "--- это \emph{неоднородность} системы \eref{3.14}.
\end{definition}

\begin{definition}
    Линейная система \eref{3.14} называется \emph{вещественной}, если коэффициенты $ p_{ij}(x), ~ q_i(x) $ принимают только вещественные значения.
\end{definition}

\begin{theorem}[о существовании и единственности решений линейных систем]
    Для любой точки $ x_0 \in (a, b) $, для любого вектора $ y^0 \in \R^n $ и для любого отрезка Пеано $ P_h(x_0, y^) $ существует и единственно решение \caupr[\eref{3.14}]{x_0, y^0}, определённое на $ P_h(x_0, y^0) $.
\end{theorem}

\begin{definition}
    Система \eref{3.1} называется \emph{почти линейной}, если $ f(x, y) \in \Cont G $, где $ G = (a, b) \times \R^n $, и существуют непрерывные и неотрицательные на $ (a, b) $ функции $ L(x), ~ M(x) $ такие, что $ \norm{f(x, y)} \le L(x) + M(x)\norm y $ для любой точки $ (x, y) \in G $.
\end{definition}

\begin{theorem}[о продолжимости решений почти линейных систем]
	Любое решение почти линейной системы продолжимо на интервал $ (a, b) $.
\end{theorem}

\begin{theorem}[о продолжимости решений линейных систем]
    Любое решение линейной системы \eref{3.14} продолжимо на интервал $ (a, b) $.
\end{theorem}

\section{Малые возмещуния начальных данных по параметру, рассуждение о сдвиге}

Рассмотрим нормальную систему \eref{3.1}, зависящую от параметра $ \mu = (\mu_1, \dots, \mu_m) $:
\begin{equ}{3.15}
	y' = f(x, y, \mu),
\end{equ}
где вещественная функция $ f(x, y, \mu) $ непрерывна и удовлетворяет условию Липшица по $ y $ локально в некоторой области $ F \sub \R^{1 + n + m} $.

По теореме о существовании и единственности решения для любого $ \vawe \mu $ множество
$$ G_{\vawe \mu} = \set{(x, y) \mid (x, y, \vawe \mu) \in F}, $$
если оно не пусто, является областью единственности для нормальной системы вида \eref{3.1} $ y' = f(x, y, \vawe \mu) $.

Фактически система \eref{3.15} представляет собой семейство систем, каждая из которых отвечает своему значению вектора $ \mu $. Понятно, что не может идти и речи о нахождении общего решения системы \eref{3.15}, поскольку даже приближённое интегрирование осуществимо только для дискретных значений параметра.

Пусть функция $ y = y(x, x_0, y^0, \mu), \quad y(x_0, x_0, y^0, \mu) = y^0 $ обозначает решение $ \text{ЗК}_{\eref{3.15}} $, заданное на множестве
$$ D = \set{(x, x_0, y^0, \mu) \mid x \in I(x_0, y^0, \mu), \quad (x_0, y^0, \mu) \in F}, $$
где $ I $ "--- максимальный интервал существования решения.

Множество $ D $ является областью.

Особое место среди систем \eref{3.15} занимает \soc \emph{порождающая} (\emph{невозмущённая}) система
\begin{equ}{hat 3.15}
    y' = f(x, y, \hat \mu),
\end{equ}
в которой $ \hat \mu $ "--- числовой вектор \emph{расчётных} значений параметров, например, средних или наиболее вероятных. \eref{3.15} можно трактовать как \emph{возмущённую} систему.

Зафиксируем расчётные значения начальных данных $ x_0 = \hat x_0, ~ y^0 = \hat y^0 $ так, чтобы $ (\hat x_0, \hat y^0, \hat \mu) \in F $.

Рассмотрим решение ЗК $ \phi(x) = y(x, \hat x_0, \hat y^0, \hat \mu), \quad \phi(\hat x_0) = \hat y^0 $ системы \eref{hat 3.15} на максимальном интервале существования $ (\alpha, \beta) $, и выберем произвольный отрезок $ [a, b] : \hat x_0 \in [a, b] \sub (\alpha, \beta) $.

Решение $ y = \phi(x) $ при $ x \in [a, b] $ будем также называть расчётным. Оно описывает расчётное (модельное) движение материальной точки в пространственно-временном континууме. Это решение предполагается известным.

Понятно, что реальное движение материальной точки, описываемое решением $ y(x, x_0, y^0, \mu) $ возмущённой системы \eref{3.15} с начальными данными $ x_0 \in [a, b], ~ y^0 $, по норме близким к $ \phi(x_0) $, и вектором параметров $ \mu $, близким к $ \hat \mu $, будет определено в некоторой окрестности точки $ x_0 $ и будет отличаться от расчётного движения. Вопрос в том, можно ли это решение продолжить на весь отрезок $ [a, b] $, и насколько велико окажется отличие.

Введём следующие обозначения:
\begin{equ}{3.16}
    \ol{U}_d^{x, y} (\phi, \hat \mu) \define \set{ (x, y, \mu) \mid x \in [a, b], \quad \| y - \phi(x) \| \le d, \quad \norm{ \mu - \hat \mu } \le d }
\end{equ}

Это "--- замкнутая трубчатая окрестность ``радиуса'' $ d > 0 $ графика функции $ y = \phi(x) $ на отрезке $ [a, b] $. Верхние индексы $ U $ служат для указания, какие переменные, кроме $ \mu $, в ней используются.

Существует такое $ \sigma > 0 $, что компакт $ \ol U_\sigma^{x, y}(\phi, \hat \mu) \sub F $.

Все последующие рассуждения будут проводиться на множестве $ \ol U_\sigma^{x, y}(\phi, \hat \mu) $.

Будем рассматривать случай, когда $ y^0 $ зависит от $ \mu $, а $ x_0 $ не зависит.

Итак, будем исследовать решение ЗК
$$ y(x, \mu) = y \big( x, x_0, y^0(\mu), \mu \big), \qquad y(x_0, \mu) = y^0(\mu) $$
системы \eref{3.15} при $ \big( x_0, y^0(\mu), \mu \big) \in U_\delta^{x_0, y^0(\mu)}(\phi, \hat \mu) $, где константа $ \delta < \sigma $, а
$$ U_\delta^{x_0, y^0(\mu)}(\phi, \hat \mu) = \set{ \big( x_0, y^0(\mu), \mu \big) \mid x_0 \in [a, b], \quad \| y^0(\mu) - \phi(x_0) \| < d, \quad \| \mu - \hat \mu \| < d} $$

Будем предполагать, что
\begin{equ}{3.17}
	y^0(\mu) = y^0 + \psi(\mu), \qquad \psi(\hat \mu) = 0,
\end{equ}
где функция $ \psi $ непрерывна в поликруге $ U_\sigma (\hat \mu) = \set{\mu : \| \mu - \hat \mu \| < \sigma} $.

\comment{Дальше что-то совсем непонятное.}

\section{Теорема о непрерывной зависимости решений от начальных данных и параметра}

\begin{theorem}
    Пусть в системе \eref{3.15} функция $ f(x, y, \mu) $ определена, непрерывна и удовлетворяет условию Липшица по $ y $ локально в области $ F $, а $ y = \phi(x) $ "--- решение системы \eref{hat 3.15} на $ [a, b] $.

    Тогда для любого $ \sigma > 0 $, при котором $ \ol U_\sigma^{x, y} \sub F $, найдётся такое $ 0 < \delta < \sigma $, что для произвольной точки $ \big( x_0, y^0(\mu), \mu \big) \in U_\delta^{x_0, y^0(\mu)}(\phi, \hat \mu) $, где $ y^0(\mu) $ из \eref{3.17}, решение $ y = y \big( x, x_0, y^0(\mu), \mu \big) $ системы \eref{3.15} определено при всех $ x \in [a, b] $, непрерывно по совокупности аргументов на множестве $ V_\delta^{x_0, y^0(\mu)} = [a, b] \times U_\delta^{x_0, y^0(\mu)}(\phi, \hat \mu) $ и точка $ \bigg( x, y \big( x, x_0, y^0(\mu), \mu \big), \mu \bigg) \in \ol U_\delta^{x, y}(\phi, \hat \mu) $ для любого $ x \in [a, b] $.
\end{theorem}

\section{Теорема о дифференцируемости решений по начальным данным и вектору параметров}

Рассмотрим \emph{системы в вариациях} по параметру и по начальным данным:
\begin{equ}{3.20j}
	u' = f_y' \bigl( x, y(x, \kappa), \mu \bigr) u + f_{\mu_j}' \bigl( x, y(x, \kappa), \mu \bigr)
\end{equ}
\begin{equ}{3.21}
	v' = f_y' \bigl( x, y(x, \kappa), \mu \bigr) v
\end{equ}

\begin{theorem}
	Пусть в системе \eref{3.15} функция $ f(x, y, \mu) \in \mathcal C_{x, y, \mu}^{0, 1, 1}(F) $, а $ \phi(x) = y(x, \hat \kappa) $ "--- решение системы \eref{hat 3.15} на $ [a, b] $.

	$$ \implies \forall \sigma > 0 : \ol U_\delta^{x, y}(\phi, \hat \mu) \sub F \quad \exists 0 < \delta < \sigma : \quad \forall \kappa \in U_\delta^{x_0, y^0(\mu)}(\phi, \hat \mu) $$
	решение $ y = y(x, \kappa) $ системы \eref{3.15} с $ y^0(\mu) $ из \eref{3.17}, где $ \psi(\mu) \in \mathcal C^1(\hat \mu - \sigma, \hat \mu + \sigma) $ имеет непрерывные частные производные по каждому из параметров в любой точке множества $ V_\delta^{x_0, y^0(\mu)} = [a, b] \times U_\delta^{x_0, y^0(\mu)}(\phi, \hat \mu) $.
	\begin{enumerate}
		\item $ \nder[j]u(x, \mu) = \pder{y(x, \kappa)}{\mu_j} $ является решением \caupr[\eref{3.20j}]{x_0, y_{\mu_j}^0{}'};
		\item $ \nder[i]v(x, \mu) = y_{y_i^0}'(x, \kappa) $ является решением \caupr[\eref{3.21}]{x_0, e_i};
		\item $ \nder[0]v(x, \kappa) = y_{x_0}'(x, \kappa) $ является решением \caupr[\eref{3.21}]{x_0, -f(x_0, y^0, \mu)}.
	\end{enumerate}
\end{theorem}

\section{Теорема о многократной дифференцируемости решения по начальным данным и параметру}

\begin{theorem}
	Пусть в системе \eref{3.15} $ f(x, y, \mu) \in \mathcal C_{x, y, \mu}^{0, k, k}(F) $, $ y(x, \kappa) $ "--- решение из предыдущей теоремы.

	Тогда $ y(x, \kappa) \in \mathcal C_{x, x_0, y^0, \mu}^{1, k, k, k} \bigl( V_\delta^{x_0, y^0(\mu)} \bigr) $.
\end{theorem}

\section{Теорема Ляпунова\texorpdfstring{"--~}{--}Пуанкаре о разложении решения в ряд по степеням начальных данных и параметра}

\begin{definition}
	Вещественную функцию $ f(x, y, \mu) $ будем называть равномерно аналитической по $ y, \mu $ относительно $ x $ в замкнутой трубчатой окрестности $ \ol U_\delta^{x, y}(\phi, 0) $, если она представима в виде ряда
	$$ f(x, y, \mu) = \sum_{p, q : |p|, |q| \ge 0} \nder[p, q]f(x) \bigl( y - \phi(x) \bigr)^p \mu^q $$
	с вещественными непрерывными на $ [a, b] $ коэффициентами $ \nder[p, q]f $, который для всякого $ x \in [a, b] $ абсолютно сходится при $ \|y - \phi(x)\| \le \sigma, ~ \|\mu\| \le \sigma $.
\end{definition}

\begin{theorem}
	Пусть в системе \eref{3.15} $ f(x, y, \mu) \in \mathcal C(F), ~ f \in \op{Lip}_y^{loc}(F) $, точка $ (\hat x_0, \hat y^0, 0) \in F $, решение $ \phi(x) = y(x, \hat x_0, \hat y^0, 0) $ системы \eref{hat 3.15} с $ \hat u = 0 $ определено на $ [a, b] $, и существует такое $ \sigma > 0 $, что $ f(x, y, \mu) $ "--- функция, равномерно аналитическая по $ y, \mu $ относительно $ x $ на компакте $ \ol U_\sigma^{x, y}(\phi, 0) $.

	Тогда найдётся такое $ \delta > 0 $, что решение системы \eref{3.15} $ y \bigl( x, \hat x_0, y^0(\mu), \mu \bigr) $ с $ y^0(\mu) = y^0 + \psi(\mu) $ будет функцией равномерно аналитической по $ y^0, \mu $ относительно $ x $ на $ V_\delta^{y^0(\mu)} = [a, b] \times U_\delta^{y^0(\mu)}(\hat y^0, 0) $.
\end{theorem}

\section{Теорема о разложении решения в ряд по степеням малого параметра}

\begin{theorem}
	Пусть для системы \eref{3.15} выполняются условия теоремы Ляпунова"--~Пуанкаре.

	Тогда найдётся такое $ \delta > 0 $, что решение $ y \bigl( x, \hat x_0, \hat y^0(\mu), \mu \bigr) $, где $ \hat y^0(\mu) = \hat y^0 + \psi(\mu) $, является равномерно относительно $ x $ аналитической по $ \mu $ функцией на множестве $ V_\delta = \set{(x, \mu) \mid x \in [a ,b], ~ \|\mu\| < \delta} $.
\end{theorem}

\section{Теорема Коши об аналитичности решения нормальной системы по независимой переменной}

Вещественная аналитичность функции $ f $ означает, что
$$ \forall (x_0, y^0) \in G \quad \exists r = r_{x_0, y^0} > 0 : \quad f(x, y) = \sum_{m = 0}^\infty \sum_{p : |p| = 0}^\infty \nder[m, p]f(x - x_0)^m(y - y^0)^p $$
Зафиксируем константу $ d \in (0, r_{x_0, y^0}) $ и введём
$$ \rho = d \Bigl(1 - \frac1{e^{(n + 1)M}}\Bigr), $$
где $ M = \max\set{|f_1|_d, \dots, |f_n|_d} $, а $ |f_i|_d = \sum_{m = 0}^\infty \sum_{p : |p| = 0}^\infty \bigl| \nder[m, p]f_i \bigr| d^{m + |p|} $.

\begin{theorem}
	Пусть в системе \eref{3.1} $ f(x, y) $ вещественно-аналитична в области $ G $.

	Тогда для любой точки $ (x_0, y^0) \in G $ решение ЗК $ y = y(x, x_0, y^0) $ раскладывается в сходящийся при $ |x - x_0| < \rho $ степенной ряд
	$$ y(x, x_0, y^0) = y^0 + \sum_{k = 1}^\infty \nder[k]y(x - x_0)^k $$
\end{theorem}

\section{Теорема об аналитичности решения ЛНС по независимой переменной}

\begin{theorem}
	Пусть в системе \eref{3.14} элементы $ p_{ij}(x) $ матрицы $ P $ и компоненты $ q_i(x) $ вектора $ q $ вещественно"=аналитичны на $ (a, b) $.

	Тогда для любого $ y^0 \in \R^n $ решение ЗК $ y = y(x, x_0, y^0) $ раскладывается в сходящийся при $ |x - x_0| < r $ степенной ряд из предыдущей теоремы.
\end{theorem}
