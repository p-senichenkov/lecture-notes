Учебная практика не на семестр или на год -- мы просто раз в семестр отчитываемся о прогрессе работы

\begin{itemize}
	\item Научно-исследовательская или программно-инженерная работа:
	\begin{itemize}
		\item Решение более-менее сложной практической либо научной задачи
		\item Отчёт (текст)
		\item Код (очень желательно)
	\end{itemize}
	\item По формату близка к научной статье и выступлению на конференции
	\item Тема должна быть интересна той кафедре, на которой планируем защищаться
\end{itemize}

Каждый год делают приложение для заказа еды в Яме \\
Никто (кроме кафедры СП) не против доказательства теоремы в качестве практики \\
Если планируете погрузиться в большую тему, то лучше в первом (третьем) семе прочитать 80 источников и написать обзорную статью \\
Текст пишем в процессе работы и периодически показываем научнику \\
Хорошие работы желательно опубликовать на весенних конференциях \\
Обязательно сдать отчёт и отзыв научника \\
На кафедре СП просят ещё много чего \\
Отзыв научника должен быть подписан \\
Если научник вам не отвечает, значит он хочет отправить вас на пересдачу \\
Им проще отчислить студента без практики, чем покрывать его перед гос. комиссией

\begin{definition}
	Консультант -- ставит задачу, читает и ревьювит код, помогает с техническими проблемами (e. g. Яков)
\end{definition}

\begin{definition}
	Научный руководитель -- преподаватель (обязательно), отвечает за адекватность задачи, следит за метологическими вопросами, следит за ходом работы, помогает с текстом и подготовкой к защите. Пишет отзыв, так что поддерживаем хорошие отношения (e. g. Чернышёв)
\end{definition}

\begin{definition}
	Руководитель практики -- общая организация процесса, следит, чтоы у каждого была тема и научник, поможет найти научника, если тот потерялся (e. g. Литвинов)
\end{definition}

\begin{definition}
	Комиссия -- преподаватели кафедры, представитель индустрии (обычно сидит кто-нибудь типа Кирилла):
	\begin{itemize}
		\item Все защиты всегда с комиссией
		\item Защищаетесь в комиссии той кафедры, с которой научник
	\end{itemize}
\end{definition}

На комиссию ходят те преподы, которые хотят самоутвердиться. Над вами \textbf{будут} смеяться (``О боже, это что, дефис вместо тире в презентации?'') \\
Где брать темы:
\begin{itemize}
	\item Есть сайт кафедры СП, куда всем кафедрам рекомендуют выкладывать темы. Однако, пока что, все это игнорируют (в том числе, СП). Бывает на бумажке на двери кафедры
	\item У преподавателя по программированию
	\item На стажке
	\item Если были в ЛШ -- продолжить начатое
\end{itemize}
Определиться с выбором темы надо \textbf{до конца сентября} \\
Если планируете где-то работать (или стажироваться) -- узнавайте, можно ли у них что-то защитить (иначе бегите оттуда. Не вытянете) \\
На матмехе никто не занимается юнити и питоном -- это не для матмеха \\
Если пишете игру на юнити -- пишите, но не рассчитывайте на помощь научника

\begin{itemize}
	\item Сентябрь -- определиться с темой
	\item Сентябрь -- начало декабря -- работа над практикой:
	\begin{itemize}
		\item Быстрый мини-обзор
		\item Введение, постановка задачи, научиться убеждать окружающих в актуальности темы
		\item Обзор
		\item Проектирование
		\item Реализация (если предполагалась)
		\item Апробация/эксперименты (если предполагались)
		\item Написание текста
	\end{itemize}
	\item Конец декабря -- защиты
\end{itemize}
Пишете игру на юнити -- докажите, что в неё будет кто-то играть \\
Пишете декомпилятор -- извольте показать на большом количестве примеров, что оно что-то декомпилирует \\
Апробация -- проверка, что программа делает то, что от неё хотят \textbf{пользователи} (это не тесты) \\
Кафедра СП ожидает, что хотя бы каждый второй семестр будет код. Остальные разрешают чисто теоретические работы
Минимум раз в неделю отчитываться научнику о ходе работы -- используем \href{https://se.math.spbu.ru/practice}{сайт кафедры СП} \\
Часы работы в присутствии преподавателя -- это для тех научников, кто предпочитет общаться лично. Остальным это \textbf{не надо} \\
На сайте кафедры СП есть раздел ``Студентам'' -- там все работы, которые успешно защитились на кафедре \\
\href{https://github.com/spbu-se/matmex-diploma-template}{Шаблон текста} \\
\href{https://github.com/yrii-litvinov/courses/blob/master/additional/repo-checklist/repo-checklist.pdf}{Чеклист по оформлению репозитория} -- \textbf{обязательно} всё должно быть выполнено, иначе засмеют на защите \\
\href{https://goo.gl/UeDRff}{Чеклист по презентации} \\
В тимс Литвинов будет писать организационные моменты -- что-то пропустил -- пересдача \\
Кинут табличку, надо будет записаться \\
Ограничений на команду нет, но защита \textbf{по-отдельности} (можно попроситься по порядку). Отчёты должны быть полностью разные -- иначе может показаться, что кто-то пытается вкатиться в IT на плечах своего товарища \\
Структура отчёта:
\begin{itemize}
	\item Титульный лист
	\item Оглавление
	\item Введение в предметную область, постановка задачи
	\item Обзор литературы существующих решений
	\item Описание предлагаемого решения (архитектура, реализация)
	\item Апробация/эксперименты
	\item Заключение
	\item Список литературы
\end{itemize}
30-40 страниц не пишем (можно только если 20 из них -- картинки) \\
На ютуб ссылаться не стыдно, на втором курсе так точно \\
Введение:
\begin{itemize}
	\item Известная информация, ``Background''
	\begin{eg}
		Испокон веков люди программируют, и программировать тяжело
	\end{eg}
	\item Неизвестная информация, ``Gap'':
	\begin{itemize}
		\item Актуальность темы
		\item Практическая значимость
		\item Кому конкретно это надо. Обязательно надо понимать, \textbf{зачем} ему это надо
	\end{itemize}
	\item Кратко про наш подход к решению и почему он приведёт к цели (``Гипотеза'' и ``Подход'')
\end{itemize}

Не пишите, что ``мы хотим написать выживач на юнити, выйти в топы стима и заработать миллионы денег'' -- не поверят \\
Лучше сказать, что это для портфолио -- все поразятся вашей сознательности

Постановка задачи:
\begin{itemize}
	\item Цель работы:
	\begin{itemize}
		\item Одним предложением -- что конкретно надо сделать
	\end{itemize}
	\item Задачи:
	\begin{itemize}
		\item Отчуждаемые -- их можно от нас отнять (изучить юнити -- не задача. Надо говорить ``сделать обзор'')
		\item Специфичные -- если бы я хотел проектировать микроволновку, какие бы у меня были задачи? Если они совпали с вашими - пересдача
		\item Решение которых приведёт к цели (смотрим нашу формулировку цели)
	\end{itemize}
\end{itemize}

``Удобное'', ``гибкое'', ``быстрое'' не пишем -- это нельзя измерить

Обзор:
\begin{itemize}
	\item Обзор существующих решений:
	\begin{itemize}
		\item Цель обзора, критерии отбора материалов
		\item Критерии сравнения
		\item Таблица с результатами
		\item Выводы
	\end{itemize}
	\item Обзор используемых чужих результатов:
	\begin{itemize}
		\item Всё, написанное и придуманное не нами -- в обзор
	\end{itemize}
	\item Должен соотноситься с темой работы
\end{itemize}

В обзоре подробно описываем, что было сделано в (e. g. Desbordante) до нас \\
Если потребуется какая-то нестандартная библиотека -- пишем ``Нужна библиотека под такую задачу, есть такие три, выбрали эту, потому что...'' \\
Выбор языка тоже хорошо бы аргументировать

Описание решения:
\begin{itemize}
	\item Желательно, чтобы разделы соответствовали списку задач
	\item Аргументированное обоснование прятных решений и отказа от альтернатив
	\item Выбор инструментария
	\item Описание архитектуры, алгоритмов, ... -- умеете UML диаграммы -- рисуйте, не умеете -- рисуйте хоть что-то
\end{itemize}

``Я взяла эту библиотеку, потому что она первая попалась под руку'' -- сразу минус. Любое сложное решение (думали больше пяти минут) аргументируем \\
Если работаете в (e. g. Desbordante), рисуем целиком архитектуру проекта и выделяем цветом, что мы делали

Описание решения (продолжение):
\begin{itemize}
	\item Рисунки и диаграммы:
	\begin{itemize}
		\item Лучше использовать UML -- он стандартный
		\item Подписи (рисунок 1: то-то то-то)
		\begin{itemize}
			\item Чужие рисунки -- со ссылкой на источник -- иначе плагиат
		\end{itemize}
	\end{itemize}
\end{itemize}

Пишем текст так, как будто будем его печатать (чтоб рисунки было видно, ссылки -- настоящими ссылками)

Апробация:
\begin{itemize}
	\item Доказать, почему всё, что мы делали, вообще осмысленно
	\item Апробация -- внешняя ``оценка'' работы
	\begin{itemize}
		\item Отзывы пользователей, лучше количесвенные (оценка числом)
		\item Внедрение, релиз, оценки
	\end{itemize}
	\item Эксперименты - численное доказательство, что ваш результат лучше аналогов:
	\begin{itemize}
		\item Замеры производительности, точности, ...
		\item Отдельная большая наука, делаем аккуратно
	\end{itemize}
	\item Если до апробации не дошли -- опишите продуманный план апробации и экспериментов
\end{itemize}

Выложили в гугл плей, приложили скриншот с оценкой ``4.9'' -- апробация \\
Выступление на конференции (если вас не освистали) -- апробация \\
Хороший эксперимент содержит ``угрозу валидности'' -- почему результат может быть некорректным (например, ``опрашивали друзей, которые не хотят нас расстроить'') \\
На матмехе принято использовать \href{https://www.usability.gov/how-to-and-tools/methods/system-usability-scale.html}{system usability scale} \\
Результатом эксперимента являются параметры случайной величины -- мат. ожидание и дисперсия \\
Среднее по больнице (без дисперсии) мерить смысла нет

Заключение:
\begin{itemize}
	\item Перечисление результатов, выносимых на защиту
	\item Должно быть согласовано с постановкой задачи (вплоть до полного её повторения)
	\begin{eg}
		Выбрать библиотеку -- выбрана библиотека \textbf{такая-то}
	\end{eg}
	\item Должно быть согласовано с текстом (никаких результатов из ниоткуда)
	\item Если практика предполагает предположениие, реалистичные планы на дальнейшую работу
	\begin{eg}
		В весеннем семестре оптимизировать такие-то компоненты системы
	\end{eg}
	\item Ссылка на репозиторий и свой ник (чтобы можно было фильтровать по коммитам)
\end{itemize}

Если в постановке задачи есть пункты, которых нет в заключении -- сразу пересдача

Литература:
\begin{itemize}
	\item Ссылок примерно столько же, сколько страниц в работе
	\item Обязательно на каждый пункт ссылаться из текста
	\item Лучше ссылаться на научные статьи
	\begin{itemize}
		\item Ещё лучше -- на книги, но по предметной области
		\item Смотрим на индекс Хирша и число цитирования
	\end{itemize}
	\item Реально прочитанные работы
	\item ГОСТ Р 7.0.5-2008
	\begin{itemize}
		\item Так, как это делает BibTeX
		\item Сам ГОСТ имеет порядка 25 страниц примеров вообще для всего, лучше его прочитать
		\item Ссылки генерятся командой \verb|\cite|
	\end{itemize}
	\item В литературу только литературу:
	\begin{itemize}
		\item Ссылки подстраничными сносками (\verb|\footnote|)
		\item В литературу можно только с датой обращения (иначе сайт может поменяться)
	\end{itemize}
\end{itemize}

Никогда не публикуйтесь в журналах за деньги (где вы платите за публикацию) \\
За плагиат -- отчисление без права восстановления (даже если пробелы заменить пробелы на какие-то пробельные символы) \\
Копипаст даже одного предложения без указания источника -- незачёт (определения не копипастим. Они нам не нужны) \\
Нейросети не стоит подключать для написания текста (для улучшения готового текста -- можно) \\
Стоит порепетировать выступление \\
Из презентации должно быть предельно понятно, что и зачем мы делаем и при чём тут наша кафедра \\
Как антиплагиат используется blackboard -- он репортит, что у вас страница с оглавлением называется ``Оглавление'', так же как у других \\
Лучше научник над вами посмеётся, чем комиссия \\
Озаботьтесь получением отзывов заранее -- на сессии у научника тоже сессия \\
Обязательно CI, юнит-тесты, README, лицензия \\
Дипломы пишем самостоятельно -- товарищ может подставить \\
Выступление на конференции за защиту \textbf{не} засчитывается (там нет комиссии) \\
Тему и научника можно менять когда угодно, только надо предупредить Литвинова \\
Если вам сказали в Янексе ``мы вам практику зачли, передайте научнику'' -- это так не работает
