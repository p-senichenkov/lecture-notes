\section{Исчисления предикатов}

\begin{definition}
	Имя предмета называется \emph{предметной константой}.
\end{definition}

\begin{definition}
	\emph{Предметной переменной} называется переменная, которая в качестве своих значений может принимать предметные константы.
\end{definition}

\begin{definition}
	\hfill
	\begin{enumerate}
		\item Предметная константа является \emph{термом}.
		\item Предметная переменная является \emph{термом}.
		\item Если $ t_1, \dots, t_n $ "--- термы, $ f $ "--- $ n $-местный функциональный символ, то выражение $ f(t_1, \dots, t_n) $ является \emph{термом.}
		\item Никакие другие выражения не являются \emph{термами}.
	\end{enumerate}
\end{definition}

\begin{definition}
	\hfill
	\begin{enumerate}
		\item Если $ t_1, \dots, t_n $ "--- термы, $ P $ "--- $ n $-местный предикатный символ, то $ P(t_1, \dots, t_n) $ является \emph{атомарной формулой}.
		\item Никакие другие выражения не являются \emph{атомарными формулами}.
	\end{enumerate}
\end{definition}

\begin{definition}
	\hfill
	\begin{enumerate}
		\item Атомарная формула является \emph{предикатной формулой}.
		\item Если $ A $ "--- атомарная формула, то $ \neg A $ является \emph{предикатной формулой}.
		\item Если $ A, B $ "--- предикатные формулы, $ * $ "--- бинарная логическая связка, то $ (A * B) $ является \emph{предикатной формулой}.
		\item Если $ A $ "--- предикатная формула, $ x $ "--- предметная переменная, то $ \forall x A $ и $ \exists x A $ являются \emph{предикатными формулами}.
		\item Никакие другие выражения не являются \emph{предикатными формулами}.
	\end{enumerate}
\end{definition}


\begin{definition}
	\emph{Кванторным комплексом} называется выражения вида $ \forall x $ или $ \exists x $, где $ x $ "--- имя предметной переменной.
\end{definition}

\begin{definition}
	\emph{Областью действия} квантора называется формула, стоящая непосредственно за кванторным комплексом, содержащем вхождение этого квантора.
\end{definition}

\begin{definition}
	Вхождение предметной переменной в формулу называется \emph{связанным}, если оно находится в кванторном комплексе или в области действия квантора по этой переменной.

	Вхождения переменных, не являющихся связанными, называется \emph{свободным}.
\end{definition}

\begin{definition}
	Формула без свободных переменных называется \emph{замкнутой}.
\end{definition}

\begin{definition}
	Для того, чтобы задать интерпретацию формулы достаточно
	\begin{itemize}
		\item задать область интерпретации $ D $ "--- множество констант;
		\item каждому $ n $-местному функциональному символу $ f $ поставить в соответствие конкретную функцию из $ D^n $ в $ D $;
		\item каждому $ n $-местному предикату $ P $ поставить в соответствие конкретное отношение над $ D^n $.
	\end{itemize}
\end{definition}

\begin{definition}
	Формула называется \emph{истинной} (\emph{ложной}) в заданной интерпретации, если она истинна (ложна) на всех наборах значений из области интерпретации, подставляемых вместо свободных вхождений предметных переменных этой формулы.
\end{definition}

\begin{definition}
	Формула называется \emph{выполнимой} в заданной интерпретации, если она истинна хоть на одном наборе значений из области интерпретации, подставляемых вместо свободных вхождений предметных переменных этой формулы.
\end{definition}

\begin{definition}
	Формула называется \emph{общезначимой} (\emph{противоречием}), если она истинна (ложна) в любой интерпретации.
\end{definition}

\begin{definition}
	Формула называется \emph{выполнимой}, если она выполнима хоть в одной интерпретации.
\end{definition}

\begin{definition}
	Формула $ B $ \emph{логически следует} из формул $ A_1, \dots, A_n $, если в любой интерпретации на любом наборе значений свободных переменных, для которых все формулы $ A_1, \dots, A_n $ истинны, формула $ B $ тоже истинна.
\end{definition}

\begin{notation}
	$ A_1, \dots, A_n \implies B $
\end{notation}

\begin{definition}
	Формулы $ A $ и $ B $ называются \emph{равносильными}, если в любой интерпретации на любом наборе значений свободных переменных их значения совпадают.
\end{definition}

\begin{notation}
	$ A \iff B $
\end{notation}

Все равносильности, справедливые для пропозициональных формул, справедливы и для предикатных формул. Кроме того, имеют место равносильности для работы с кванторами. Пусть $ x $ не входит свободно в $ B $, а $ y $ не входит в $ C $.

\begin{longtable}{r c l}
	$ \neg \forall x A $ & $ \iff $ & $ \exists x \neg A $ \\
	$ \neg \exists x A $ & $ \iff $ & $ \forall x \neg A $ \\
	$ \forall x B $ & $ \iff $ & $ B $ \\
	$ \exists x B $ & $ \iff $ & $ B $ \\
	$ \forall x C $ & $ \iff $ & $ \forall y [C]_y^x $ \\
	$ \exists x C $ & $ \iff $ & $ \exists y [C]_y^x $ \\
	$ \forall x A \amp B $ & $ \iff $ & $ \forall x (A \amp B) $ \\
	$ \exists x A \amp B $ & $ \iff $ & $ \exists x (A \amp B) $ \\
	$ \forall x A \vee B $ & $ \iff $ & $ \forall x (A \vee B) $ \\
	$ \exists x A \vee B $ & $ \iff $ & $ \exists x (A \vee B) $ \\
	$ B \to \forall x A $ & $ \iff $ & $ \forall x(B \to A) $ \\
	$ B \to \exists x A $ & $ \iff $ & $ \exists x (B \to A) $ \\
	$ \forall x A \to B $ & $ \iff $ & $ \exists x (A \to B) $ \\
	$ \exists x A \to B $ & $ \iff $ & $ \forall x (A \to B) $
\end{longtable}

\begin{definition}
	Терм $ t $ называется \emph{свободным для подстановки} в формулу $ A $ вместо свободных вхождений предметной переменной $ x $, если он не содержит предметных переменных, в области действия кванторов по которым имеются свободные вхождения предметной переменной $ x $.
\end{definition}

\begin{undefthm}{Следствия из определения.}
	\hfill
	\begin{enumerate}
		\item Любой постоянный терм свободен для подстановки в любую формулу вместо свободных вхождений любой переменной.
		\item Переменная $ x $ свободна для подстановки в любую формулу вместо своих свободных вхождений.
		\item Если предметная переменная $ x $ не имеет свободных вхождений формулу $ A $, то любой терм свободен для подстановки в формулу $ A $ вместо свободных вхождений переменной $ x $.
		\item Если терм не содержит предметных переменных, входящих в формулу $ A $, то он свободен для подстановки в формулу $ A $ вместо свободных вхождений любой переменной.
	\end{enumerate}
\end{undefthm}

\begin{definition}
	Формула находится в \emph{предварённой нормальной форме}, если она представляет собой последовательность кванторных комплексов, все переменные которых различны, и формулы, не содержащей кванторов.
\end{definition}

\begin{theorem}
	По всякой предикатной формуле $ P $ можно построить такую формулу $ Q $ в предварённой нормальной форме такую, что $ P \iff Q $.
\end{theorem}

\subsection*{Секвенциальное исчисление предикатов}

Формальное описание секвенциального исчисления предикатов:
\begin{enumerate}
	\item Алфавит: множество символов для записи имён предметных переменных и констант, функциональных и предикатных символов, объединённое с множеством логических связок, кванторов, запятой, символа секвенции и скобок.
	\item Формулы исчисления: секвенции, содержащие предикатные формулы в выбранном алфавите.
	\item Аксиомы исчисления: единственная схема аксиом:
		$$ \Gamma_1 ~ A ~ \Gamma_2 \vdash \Delta_1 ~ A ~ \Delta_2 $$
	\item Правила вывода для логических связок такие же, как в секвенциальном исчислении высказываний, но добавлены ещё четыре правила для кванторов и правила сокращения повторений.
		$$ (\vdash \exists) ~ \frac{\Gamma \vdash \Delta_1 ~ [A]_T^x \Delta_2}{\Gamma \vdash \Delta_1 ~ \exists x A ~ \Delta_2} \qquad (\forall \vdash) ~ \frac{\Gamma_1 ~ [A]_T^x ~ \Gamma_2 \vdash \Delta}{\Gamma_1 ~ \forall x A ~ \Gamma_2 \vdash \Delta}, $$
		где терм $ T $ свободен для подстановки в формулу $ A $ вместо свободных вхождений предметной переменной $ x $.
		$$ (\exists \vdash) ~ \frac{\Gamma_1 ~ [A]_y^x ~ \Gamma_2 \vdash \Delta}{\Gamma_1 ~ \exists x A ~ \Gamma_2 \vdash \Delta} \qquad (\vdash \forall) ~ \frac{\Gamma \vdash \Delta_1 ~ [A]_y^x ~ \Delta_2}{\Gamma \vdash \Delta_1 ~ \forall x A ~ \Delta_2}, $$
		где переменная $ y $ не входит свободно в заключение правила и свободна для подстановки в формулу $ A $ вместо свободных вхождений предметной переменной $ x $.
		$$ \frac{\Gamma_1 ~ P ~ \Gamma_2 ~ P ~ \Gamma_3 \vdash \Delta}{\Gamma_1 ~ P ~ \Gamma_2 ~ \Gamma_3 ~ \vdash \Delta} \qquad \frac{\Gamma \vdash \Delta_1 ~ P ~ \Delta_2 ~ P ~ \Delta_3}{\Gamma \vdash \Delta_1 ~ P ~ \Delta_2 ~ \Delta_3} $$
\end{enumerate}

\begin{definition}
	Секвенция называется \emph{чистой}, если ни одна переменная не имеет одновременно свободных и связанных вхождений.
\end{definition}

\begin{theorem}
	Чистая секвенция выводима в секвенциальном исчислении тогда и только тогда, когда её формульный образ общезначим.
\end{theorem}

\begin{theorem}
	Секвенциальное исчисление полно и непротиворечиво.
\end{theorem}

\subsection*{Метод резолюций для исчисления предикатов}

\begin{definition}
	\emph{Общим унификатором} формул $ Q $ и $ R $ называется такая подстановка $ \lambda = \big|_{t_1 \dots t_n}^{x_1 \dots x_n} $ термов $ t_1 \dots t_n $ вместо свободных вхождений предметных переменных $ x_1 \dots x_n $, что результаты этой подстановки в формулы $ Q $ и $ R $ графически совпадают.
\end{definition}

Формальное описание метода резолюций для исчисления предикатов:
\begin{enumerate}
	\item Алфавит: множество символов для записи имён предметных переменных и констант, имён предикатных переменных и функциональных символов, объединённое с множеством из двух логических связок $ \vee $ и $ \neg $, скобок, а также запятой.
	\item Формулы: предложения.
	\item Аксиомы: отсутствуют.
	\item Правила вывода:
		\begin{itemize}
			\item правило резолюции:
				$$ \frac{D_1 \vee R_1 \vee D_2 \qquad D_3 \vee \neg R_2 \vee D_4}{[D_1 \vee D_2 \vee D_3 \vee D_4]\lambda}, $$
				где $ \lambda $ "--- общий унификатор формул $ R_1 $ и $ R_2 $.
			\item правило сокращения повторений:
				$$ \frac{D_1 \vee R \vee D_2 \vee R \vee D_3}{D_1 \vee R \vee D_2 \vee D_3} $$
		\end{itemize}
\end{enumerate}

\begin{definition}
	Множество предложений называется \emph{неудовлетворимым}, если из него выводимо пустое предложение $ nill $.
\end{definition}

\begin{theorem}
	Для того, чтобы множество предложений было неудовлетворимо необходимо и достаточно, чтобы их конъюнкция являлась противоречием.
\end{theorem}

\begin{definition}
	\emph{Сколемовской константой (функцией)} называется константа (функция), существование которой утверждается в формуле, но её значение (определение) может быть неизвестно.
\end{definition}
