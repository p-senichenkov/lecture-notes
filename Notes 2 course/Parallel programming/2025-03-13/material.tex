\section{Введение?}

модель "--- метод "--- программа

\begin{figure}[!ht]
	\begin{tikzpicture}[start chain=going right, node distance=3em,
		every join/.style={->}]

		\node[draw, on chain, align=center, join] {объект исследования\\(исходная задача)};
		\node[draw, on chain, align=center, join] {математическая модель\\(математическая постановка задачи)};
		\node[draw, on chain, align=center, join] {метод решения};
		\node[draw, on chain, continue chain=going below, align=center, join] {реализация алгоритма\\(построение и отладка программы)};
		\node[draw, on chain, align=center, join] {реализация алгоритма\\(построение и отладка программы)};
		\node[draw, on chain, continue chain=going left, align=center, join, join=with chain-1 by {<-}] {проведение вычислений и анализ результатов};
	\end{tikzpicture}
\end{figure}

Точность конечного результата не может превосходить точности исходных данных и точности вычислений.

\begin{definition}
	Бесконечно сходящиеся итерационные процессы будем называть \it{вычислительными алгоритмами}. Основанные на них решения задач будем называть \it{численными методами}.
\end{definition}

Символьные (аналитические) и численные вычисления.

Символьные вычисления реализованы в виде подпрограмм и не имеют аппаратной реализации.

При численных вычислениях получаемый результат всегда зависит от свойств конкретного компьютера. Все свойства, как правило, пользователю неизвестны. Поэтому, разработать алгоритм решения поставленной задачи не удастся.

\begin{definition}
	Будем называть \it{псевдоалгоритмом} последовательность действий считая, что сами действия не обязательно имеют чёткое конструктивное определение.
\end{definition}

Если все эти действия как-то уточнить, то можно построить \it{уточнение псевдоалгоритма}. Последнее уточнение будет иметь реализацию на конкретной платформе "--- это будет алгоритм реализации вычислений.

\section{Архиектура фон-Нейнмана и устройство памяти}

Другая архитектура "--- гарвардская.

Принципы фон-Нейнмана:
\begin{enumerate}
	\item \bt{Принцип однородности памяти и её адресуемости.}

	Основная (оперативная) память состоит из однородных ячеек. Размер ячейки называется \it{машинным словом}. Каждая ячейка имеет уникальный адрес. Это называется RAM-память "--- память с произвольным доступом к ячейке по её адресу.

	Эта память используется для записи как команд, так и данных. Они хранятся в двоичном коде и представляются одинаково. Тип данных (из языков высокого уровня) \bt{не является} неотъемлемой частью ячейки.

	\item \bt{Принцип последовательного программного управления.}

	Управление машиной осуществляется по заранее подготовленной программе.

	\begin{definition}
		\it{Программа} "--- это последовательность команд, расположенных в памяти линейно в естественном порядке.
	\end{definition}

	Команды, которые могут исполняться программой, составляют её \it{машинный язык}. Программа выполняется покомандно в порядке записи команд.

	Естественный порядок моет быть нарушен командами перехода.

	\item \bt{Принцип запоминания программ.}

	При исполнении программу следует поместить в основную память машины и инициировать исполнение первой команды. Затем команды выбираются из основной памяти машины согласно принципу программного управления.

	\item \bt{Принцип параллельной организации вычилений на аппаратном уровне.}

	Операции над машинными словами производятся аппаратурой над всеми разрядами одновременно.

	Остальные принципы вызваны только техническими ограничениями:

	\item \bt{Принцип двоичной системы счисления.}

	\item \bt{Принцип иерархичности запоминающих устройств.}

	Компромисс между ёмкостью, стоимостью и быстродействием памяти.

	\begin{enumerate}
		\item Процессор состоит из арифметико-логического устройства, устройства управления и регистров. Доступ к регистрам осуществляется не по адресу, а по имени.

		\item Основная память: доступ по адресу.

		\item Внещние устройства. У каждого есть свой контроллер.
	\end{enumerate}
	Всё это находится на системной магистрали.
\end{enumerate}

Иерархия памяти:

\begin{enumerate}
	\item[\bt{L0.}] регистры;
	\item[\bt{L1.}] кеш L1;
	\item[\bt{L2.}] кеш L2;
	\item[\bt{L3.}] кеш L3;
	\item[\bt{L4.}] основная память;
	\item[\bt{L5.}] локальные диски;
	\item[\bt{L6.}] удалённые запоминающие устройства.
\end{enumerate}

Уровни L0--L3 "--- статические (SRAM), L4 "--- динамический (DRAM).

Основные характеристики:
\begin{itemize}
	\item \it{Латентность} "--- время доступа.
	\item \it{Длительность цикла} "--- минимальное время между двумя последовательными обращениями к памяти.
\end{itemize}

\begin{definition}
	\it{Локальность} "--- тенденция программ использовать данные и команды, находящиеся в ограниченных областях памяти.

	\it{Временная локальность} "--- обращение к элементу имеет тенденцию повторяться.

	\it{Пространственная локальность} "--- после обращения к элементу, скорее всего, произойдёт обращение к элементам поблизости.
\end{definition}
