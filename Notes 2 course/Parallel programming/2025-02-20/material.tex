\section{Введение}

\begin{note}
	На приветствие ``здравствуйте'' мы встаём.
\end{note}

\begin{definition}
	Сигнал "--- любой процесс, несущий ``информацию''.
\end{definition}

\begin{definition}
	Сообщение "--- сведение или факт, возможно, предназанченный для передачи.
\end{definition}

\begin{definition}
	Данные "--- представление этого сообщения в формализованном виде.
\end{definition}

\begin{definition}
	Информация "--- это смысл, придаваемый данным при их представлении.
\end{definition}

У сообщения должен быть получатель (хотя бы потенциальный). Иначе говорить об информации бессмысленно.

Не всякую информацию можно измерить.

Информация может быть неправильно воспринята.

\begin{definition}
	Обработка данных
\end{definition}

\begin{definition}
	Интуитивно, \it{алгоритм} "--- способ решения любой массовой задачи из некоторого класса задач.
\end{definition}

\begin{definition}
	Массовая задача "--- совокупность задач единой структуры (которые как-то параметризуются).
\end{definition}

\begin{definition}
	Способ "--- конечное описание определённых правил, задающих порядок и содержание действий, после выполнения конечного числа которых будет получен требуемый результат, если таковой существует.

	Если результата не существует, то такой процесс обработки никогда не закончится.
\end{definition}

Считается, что правила элементарны и могут быть выполнены механически исполнителем алгоритма.

``Механически'' означает, что для исполнения не требуются ни понимание, ни искусство, ни изобретательность.

\begin{definition}
	Алгоритм "--- это конечная, однозначно определённая совокупность правил по преобразованию за конечное число шагов исходных данных в конечный результат любой массовой задачи из рассматриваемого класса задач.
\end{definition}

Если для массовой задачи существует алгоритм решения, говорят, что задача \it{алгоритмически разрешима}.
Иначе "--- \it{неразрешима}.

Формальная запись алгоритма "--- это программа.

В теории можно формализовать понятие вычислимости функции. На практике оказывается, что машина имеет ограниченные ресурсы. Поэтому нас интересуют \it{эффективная вычислимость} и \it{эффективные алгоритмы}.
Для этого вводится понятие \it{классов сложности}.

Языки программирования:
\begin{itemize}
	\item синтаксис;
	\item семантика;
	\item прагматика.
\end{itemize}

\begin{definition}[согласно ISO/IEC 12207]
	Программное обеспечение "--- логически связанная совокупность программ и данных, снабжённых документацией.
\end{definition}

После 2002 г. программы нужно распараллеливать.

\begin{definition}
	Численные методы "--- методы вычислений, сводящиеся к арифметическим и логическим операциям над числами, которые компьютер выполняет с заданной точностью. Найденные решения получаются в виде числовых значений и получаются приближёнными. Численное решение получается для конкретных значений исходных данных.
\end{definition}
