\part{Дифференциальная геометрия поверхностей}

\section{Касательная плоскость к поверхности}

\begin{undefthm}{Задание кривой на поверхности.}
	Любая точка кривой лежит на поверхности \\
	Тогда мы можем применить к ней $ r^{-1} $ и ``спустить'' её на $ D $ \\
	Получим координаты $ u(t), v(t) $ \\
	Они называются внутренними координатами кривой на поверхности
\end{undefthm}

\begin{definition}
	Пусть $ u(t), v(t) $ -- внутренние координаты кривой на поверхности \\
	$ \vv r \big( u(t), v(t) \big) $ -- кривая на поверхности \\
	$ \dfrac{\di \vv r}{\di t}\clamp{t = t_0} $ -- касательный вектор
\end{definition}

\begin{definition}
	Касательная плоскость к поверхности -- множество касательных векторов в данной точке
\end{definition}

\begin{statement}
	Это плоскость с базисом $ \pder ru $ и $ \pder rv $
\end{statement}

\begin{proof}
	Распишем и всё получится: \\
	Касательный вектор "--- $ \dfrac{\di \vv r}{\di t}\clamp{t = t_0} $
	$$ \frac{\di \vv r}{\di t}\clamp{t = t_0} = \pder ru \cdot \frac{\di u}{\di t} + \pder rv \cdot \frac{\di v}{\di t} $$
	Это ЛК $ \pder ru $ и $ \pder rv $ \\
	Верно ли, что $ \alpha r_u + \beta r_v $ является касательным вектором? \\
	Верно, для кривой
	$$
	\begin{cases}
		u = \alpha(t + t_0) \\
		v = \beta(t + t_1)
	\end{cases} \qquad
	\begin{cases}
		u_t = \alpha \\
		v_t = \beta
	\end{cases} $$
\end{proof}

\section{Длина кривой на поверхности. Первая квадратичная форма}

Пусть кривая задана регулярной параметризацией:
$$ r(u, v) = \big( x(u, v), ~ y(u, v), ~ z(u, v) \big), \qquad u = u(t), \quad v = v(t) $$
$$ r_t' = (x_u'u' + x_v'v', ~ y_u'u' + y_v'v', ~ z_u'u' + z_v'v') = r_u'u' + r_v'v' $$
$$ |r_t'| = \sqrt{(r_u'u' + r_v'v') \cdot (f_u'u' + f_v'v')} = \sqrt{(r_u')^2(u')^2 + 2r_u'r_v'u'v' + (r_u')^2(v')^2} $$

\begin{definition}
	Обозначим:
	$$
	\begin{cases}
		E \define |r_u|^2 = (r_u, r_u) \\
		F \define (r_u, r_v) \\
		G \define |r_v|^2 = (r_v, r_v)
	\end{cases} $$
	Эти коэффициенты зависят от поверхности, а не от кривой.
\end{definition}

Найдём длину кривой:

\begin{multline*}
	l = \dint[t]ab{\bigg| \frac{\di r}{\di t} \bigg|} = \dint[t]ab{|r_uu_t + r_vv_t|} = \dint[t]ab{\sqrt{(r_uu_t + r_vv_t, ~ r_uu_t + r_vv_t)}} = \\
	= \dint[t]ab{\sqrt{(r_u, r_u)u_t^2 + 2(r_u, r_v)u_tv_t + (r_v, r_v)v_t^2}} = \dint[t]ab{\sqrt{Eu_t^2 + 2Fu_tv_t + Gv_t^2}}
\end{multline*}

\begin{definition}
	\rom1 форма:
	$$ \rom1(u', v') = E(u')^2 + 2Fu'v' + G(v')^2 $$
\end{definition}

\section{Изометрия поверхностей. Внутренняя геометрия}

\begin{definition}
	Даны поверхности:
	$$ r_1 : D_1 \to \R^3, \qquad r_2 : D_2 \to \R^3 $$
	$ \Phi : D_1 \to D_2 $ называется изометрией, если длина кривой на $ r_1(D_1) $ равна длине кривой на $ r_2 \big( \Phi(D_1) \big) $.
\end{definition}

\begin{theorem}
	Поверхности изометричны \bt{тогда и только тогда}, когда в некоторой параметризации у них совпадают коэффициенты $ E, F, G $.
\end{theorem}

\begin{iproof}
	\item $ \impliedby $:

	Формула длины кривой:
	$$ l = \dint[t]{a}b{\sqrt{Eu'^2 + 2Fu'v' + Gv'^2}} $$
	Они должны совпасть.
	$$ \Phi \define r_2 \circ r_1^{-1} $$
	\item $ \implies $: \\
	Возьмём кривую
	$$
	\begin{cases}
		u(t) = t \\
		v(t) = t_0
	\end{cases} $$
	Можно НУО считать, что поверхности параметризованы одинаково (иначе перепараметризуем).
	$$ l_1 = \dint[t]{a}b{\sqrt{E_1}} = \sqrt{E_1}(b - a) $$
	$$ l_2 = \dint[t]{a}b{\sqrt{E_2}} = \sqrt{E_2}(b - a) $$
	По предположению, они равны \\
	Тем самым $ E_1 = E_2 $ \\
	Аналогично, взяв
	$$
	\begin{cases}
		u(t) = t_0 \\
		v(t) = t
	\end{cases} $$
	получаем $ G_1 = G_2 $ \\
	Взяв
	$$
	\begin{cases}
		u(t) = t + t_0 \\
		v(t) = t + t_1
	\end{cases} $$
	получаем
	$$ \dint[t]{a}b{\sqrt{E_1 + 2F_1 + G_1}} = \dint[t]{a}b{\sqrt{E_2 + 2F_2 + G_2}} \quad \implies F_1 = F_2 $$
\end{iproof}

\begin{remark}
	То, что поверхности изометричны, \bt{не} означает, что первые квадратичные формы совпадут. Это лишь значит, что \bt{сущетсвует} параметризация, в которой они совпали
\end{remark}

\begin{definition}
	Говорят, что характеритика поверхности относится к внутренней геометрии поверхности, если он не меняется при изометрии.
\end{definition}

\section{Контрпример Шварца}

\begin{eg}[сапог Шварца]
	Есть цилиндр \\
	Его можно развернуть в прямоугольник и посчитать площадь:
	$$ S = 2\pi R H $$
	Разобъём на треугольники:
	\begin{enumerate}
		\item Порежем на слои
		\item В каждый слой впишем правильный $ n $-уголькик
		\item Построим антипризму \\
		Антипризма:
		\begin{enumerate}
			\item Берём в верхнем и нижнем основании правильный $ n $-угольник
			\item Поворачиваем одно из оснований так, чтобы напротив вершины оказалась середина стороны
			\item Соединим вершины
		\end{enumerate}
		\item Счиатем, что высоту мы разбили на $ k $ частей, каждая часть представляет собой $ n $-угольную антипризму
		\begin{quest}
			Сколько в одном слое треугольников?
		\end{quest}
		\begin{answer}
			$ 2n $
		\end{answer}
		Всего $ 2kn $ треугольников
		\item Посчитаем площадь каждого треугольника:
		$$ S_\triangle = \half lh $$
		Найдём $ l $: \\
		Есть круг радиуса $ R $, мы в него вписали правильный $ n $-угольник, хотим посчитать его сторону.  Угол в центре будет равен $ \dfrac{2\pi}n $. Применим теорему косинусов:
		$$ l^2 = 2R^2 - 2R^2 \cos \frac{2\pi}n $$
		$$ l = R \sqrt{2 \bigg( 1 - \cos \frac{2\pi}n \bigg) } = 2R\sin \frac\pi{n} $$
		Найдём $ h $:
		\begin{remark}
			Треугольник немножко наклоняется. Направление его высоты не совпадает с направлением образующей цилиндра. $ h \ne \dfrac{H}k $
		\end{remark}
		$$ a = R - R \cos \frac\pi{n} $$
		$$ h = \sqrt{\frac{H^2}{k^2} + R^2 \bigg(1 - \cos \frac\pi{n} \bigg)^2} = \sqrt{\frac{H^2}{k^2} + R^2 \cdot 4 \sin^4 \frac\pi{2n}} $$
		$$ S \stackrel?= \limi{\pi n} 2kn \cdot \half \cdot 2R\underbrace{\sin \frac\pi{n}}_{\sim \faktor\pi{n}} \cdot \sqrt{\frac{H^2}{k^2} + 4R^2 \sin^4 \frac\pi{2n}} = 2\pi RH \lim \sqrt{1 + \frac{k^2}{H^2} \cdot 4R^2 \sin^4 \frac\pi{2n}} $$
		Это равно $ 2\pi RH $ только тогда, когда
		$$ \frac{k^2}{H^2} \cdot 4R^2 \sin^4 \frac\pi{2n} \stackrel?\to 0 $$
		$$ \frac{4\pi^4R^2}{4H^2} \cdot \frac{k^2}{n^4} \stackrel?\to 0 $$
		То есть, $ k \stackrel?= o(n^2) $ \\
		Значит, разбивая на много низких, крупно нарезанных антипризм, получаем неправильную площадь
	\end{enumerate}
\end{eg}

\section{Площадь поверхности. Модуль вектора \texorpdfstring{$ r_u \times r_v $}{ru x rv}}

\begin{definition}
	Разбиваем поверхность на многоугольники и проецируем их на свои касательные плоскости. Складываем площади и переходим к пределу.
\end{definition}

\begin{theorem}
	$$ S = \iint\limits_D |r_u \times r_v| \di u \di v $$
\end{theorem}

\begin{noproof}
	На самом деле, это определение двойного интеграла.
\end{noproof}

\begin{theorem}
	$$ S = \iint\limits_{D} \sqrt{EG - F^2}\di u \di v $$
\end{theorem}

\begin{proof}
	Следует из леммы:
\end{proof}

\begin{lemma}\label{lemma:r_u_x_r_v}
	$ |r_u \times r_v| = \sqrt{EG - F^2} $
\end{lemma}

\begin{proof}
	$$ (r_u \times r_v, ~ r_u \times r_v) = |r_u \times r_v|^2 \iseq EG - F^2 $$
	$$ r_u = (x_u, y_u, z_u), \qquad r_v = (x_v, y_v, z_v) $$
	$$ r_u \times r_v = (y_uz_v - z_uy_v, ~ z_ux_v - x_uz_v, ~ x_uy_v - y_ux_v) $$
	$$ (r_u \times r_v)^2 = (y_uz_v - z_uy_v)^2 + (z_ux_v - x_uz_v)^2 + (x_uy_v - y_ux_v)^2 $$
	\begin{multline*}
		EG - F^2 = (x_u^2 + y_u^2 + z_u^2)(x_v^2 + y_v^2 + z_v^2) - (x_ux_v + y_uy_v + z_uz_v)^2 = \\
		= \underbrace{\cancel{x_u^2x_v^2} + \underline{x_u^2y_v^2} + ... + z_u^2z_v^2}_{\text{9 слагаемых}} - \cancel{x_u^2x_v^2} - \cancel{y_u^2y_v^2} - \cancel{z_u^2z_v^2} - \underline{2x_1x_vy_uy_v} - 2x_ux_vz_uz_v - 2y_uy_vz_uz_v
	\end{multline*}
	Подчёркнутые (вместе с ещё одним) запаковываются в квадрат разности. Остальные -- аналогично.
\end{proof}

\section{Кривизна кривой на поверхности. Вторая квадратичная форма}

Кривая задана внутренним уравнением в натуральной параметризации:
$$ \vv{r}(s) = \vv{r} \bigg( u(s), v(s) \bigg) $$
$$ \vv{k} = \vv{r}''(s) \quad (k = |\vv{k}|) $$
$$ r_s' = r_u'u_s' + r_v'v_s' $$
$$ r''(s) = r_{uu}u'^2 + 2r_{uv}u'v' + r_{vv}v'^2 + \underline{r_uu'' + r_vv''} $$
Хотим избавиться от подчёркнутых слагаемых. Домножим равенство на вектор нормали ($ n \perp r_u, ~ n \perp r_v $)
$$ k\cos \theta = \vv{r}''(s) \cdot \vv{n} = \underbrace{(r_{uu}, n)}_L \cdot u'^2 + 2\underbrace{(r_{uv}, n)}_M \cdot u'v' + \underbrace{(r_{vv}, n)}_N \cdot v'^2 $$
где $ \theta $ -- угол между вектором кривизны кривой и вектором нормали к поверхности \\
Получили проекцию кривизны на вектор нормали \\
$ L, M, N $ зависят только от поверхности (не от кривой) \\
Получаем следующую формулу:

\begin{theorem}
	Если кривая в натуральной параметризации, то
	$$ k \cos \theta = Lu'^2 + 2Mu'v' + Nv'^2 $$
\end{theorem}

\begin{definition}
	$ L, M, N $ -- коэффициенты второй квадратичной формы
\end{definition}

\begin{definition}
	$ k\cos \theta $ называется нормальной кривизной поверхности в направлении $ (u', v') $
\end{definition}

\section{Теорема Мёнье. Кривизна кривой на поверхности в произвольной параметризации}

\begin{theorem}\label{th:k_cos}
	$$ k\cos\theta = \frac{\rom2(u', v')}{\rom1(u', v')} = \frac{Lu'^2 + 2Mu'v' + Nv'^2}{Eu'^2 + 2Fu'v' + Gv'^2} $$
\end{theorem}

\begin{proof}
	Будем действовать в натуральной параметризациии \\
	Пусть $ s = \vphi(t) $
	$$ \vphi'(t) = |r_t'| = |r_uu_t' + r_vv_t'| $$
	$$ u_t' = u_s's_t' = u_s' \cdot |r_t'| $$
	$$ u_s' = \frac{u_t'}{|r_t'|}, \qquad v_s' = \frac{v_t'}{|r_t'|} $$
	$$ k\cos\theta = L \frac{u_t'^2}{|r_t'|^2} + 2M \frac{u_t'v_t'}{|r_t'|^2} + N \frac{v_t'}{|r_t'|^2} = \frac{\rom2(u_t', v_t')}{|r_t'|^2} $$
	Мы знаем, что $ |r_t'|^2 = \rom1(u_t', v_t') $, так как
	$$ \uint[t]{|r_t'|} \text{ -- длина кривой } = \uint[t]{\sqrt{Eu_t'^2 + 2Fu_t'v_t' + Gv_t'^2}} $$
\end{proof}

\begin{definition}
	Сечение поверхности плоскостью, содержащей нормаль поверхности в заданной точке, образует кривую, которая называется \it{нормальным сеченением}.
\end{definition}

\begin{definition}
	Пусть $ l $ "--- прямая, лежащая в касательной плоскости.

	Сечение плоскостью, содержащей нормаль и прямую $ l $ образует некую кривую на поверхности.

	Её кривизна $ k_n(l) $ "--- \it{нормальная кривизна} поверхности по направлению $ l $.
\end{definition}

\begin{definition}
	Мимнимальная и максимальная нормальные кривизны на поверхности называются \it{главными кривизнами}.
\end{definition}

\begin{notation}
	$ k_1, k_2 $
\end{notation}

\begin{definition}
	Направления, по которым эти кривизны достигаются, называются \it{главными направлениями}.
\end{definition}

\begin{definition}
	\it{Гауссова кривизна} "--- $ K = k_1k_2 $.
\end{definition}

\begin{definition}
	\it{Средняя кривизна}:
	$$ \frac{k_1 + k_2}2 $$
\end{definition}

\begin{theorem}[Мёнье]
	Если $ \theta $ "--- угол между нормалью к кривой и нормалью к поверхности, то
	$$ k_n(l) = k\cos \theta $$
\end{theorem}

\begin{proof}
	Пусть $ \vv n $ "--- нормаль к кривой, $ \vv m $ "--- нормаль к поверхности.

	По \rom1 формуле Френе $ \dot v = k \vv n $.
	$$ \rom2(v, v) = \dot v \cdot \vv m = \ddot r \cdot \vv m = k \vv n \vv m = k \cos \theta $$
	$$ \rom1(v, v) = 1 $$
\end{proof}

\section{Соприкасающийся параболоид. Типы точек}

В этом параграфе рассмариваем локальное поведение поверхности в окрестности какой-то точки.

Рассмотрим векторы $ r, r_u, r_v, r_{uu}, r_{uv}, r_{vv} $. \\
Введём точку $ M = (0, 0, 0) $. \\
Касательная плокость "--- $ XOY $.
$$ \vv{n} \parallel OZ $$
По теореме о неявной функции поверхность в окрестности $ M $ задаётся $ z = r(x, y) $. \\
Разложим по Тейлору:
$$ z = r(0, 0) + r_x(0, 0)x + r_y(0, 0)y + \frac{r_{xx}(0, 0)}2x^2 + r_{xy}(0, 0)xy + \frac{r_{yy}(0, 0)}2y^2 + o(x^2 + y^2) $$
$$ r(0, 0) = 0, \qquad r_x(0, 0) = r_y(0, 0) = 0 $$

\begin{definition}
	$$ z = \half[A]x^2 + Bxy + \half[C]y^2 $$
	$$ A = r_{xx}(0, 0), \qquad B = r_{xy}(0, 0), \qquad C = r_{yy}(0, 0) $$
	Это называется соприкасающийся параболоид
\end{definition}

Можем сделать поворот $ XOY $ и тогда:
$$ z = \vawe{A}\vawe{x}^2 + \vawe{C}\vawe{y}^2 $$
$$ 2\vawe A = r_{xx}(0, 0), \qquad 2\vawe C = r_{yy}(0, 0) $$

\begin{definition}[классификация]
	\hfill
	\begin{itemize}
		\item $ \vawe{A} \cdot \vawe{C} > 0 $ \\
		Эллиптический параболоид \\
		Эллиптическая точка
		\item $ \vawe{A} \cdot \vawe{C} < 0 $ \\
		Гиперболический параболоид \\
		Гиперболическая точка
		\item $ \vawe{A} \cdot \vawe{C} = 0 $
		\begin{itemize}
			\item Параболический цилиндр \\
			Параболическая точка
			\item Плоскость \\
			Точка уплощения
		\end{itemize}
		\item $ \vawe{A} = \vawe{C} $ \\
		Параболоид вращения \\
		Точка округления
	\end{itemize}
\end{definition}

\section{Совпадение характеристик у поверхности и соприкасающегося параболода}

\begin{theorem}
	У поверхности и соприкасающегося параболоида одинаковые $ E, F, G, L, M, N $ в точке
\end{theorem}

\begin{proof}
	$ z = f(x, y) $ в окрестности точки
	$$
	\begin{cases}
		x = u \\
		y = v \\
		z = f(u, v)
	\end{cases} $$
	$$ r(u, v) = \bigg( u, v, f(u, v) \bigg) $$
	$$ r_u = (1, 0, f_u), \qquad r_v = (0, 1, f_v), \qquad r_{uu} = (0, 0, f_{uu}), \qquad \dots $$
	Это верно как для соприкасающегося параболоида, так и для поверхности
	$$ \vv{n} = (0, 0, 1) \text{ для обеих поверхностей} $$
\end{proof}

\section{Теорема Эйлера}

\begin{theorem}[Эйлера]
	Если $ \theta $ "--- угол между $ l $ и главным направлением $ e_1 $, то
	$$ k_n(l) = k_1 \cos^2 \theta + k_2 \sin^2 \theta $$
\end{theorem}

\begin{figure}[!ht]
	\begin{tikzpicture}[>=Stealth]
		\draw[->] (0, 0) -- (2.2, 0) node[below] {$ x $};
		\draw[->] (0, 0) -- (0, 2.2) node[left] {$ y $};

		\draw[->, blue, thick] (0, 0) -- (1, 0) node[below] {$ e_1 $};
		\draw[->, blue, thick] (0, 0) -- (0, 1) node[left] {$ e_2 $};

		\draw (-1, -0.5) -- (2, 1) node[right] {$ l $};
		\path (2, 1) coordinate (C) -- (0, 0) coordinate (B) -- (1, 0) coordinate (A) pic [draw=green!50!black, angle radius=10mm, "$ \varphi $", angle eccentricity=1.15] {angle};
	\end{tikzpicture}
\end{figure}

\begin{proof}
	Выберем такую систему координат, что $ k_1, k_2 $ "--- кривизны в направлениях $ OX $ и $ OY $ соостветственно. Тогда $ l $ задаётся как
	$$
	\begin{cases}
		x(t) = t\cos \theta \\
		y(t) = t \sin \theta
	\end{cases} \qquad
	\begin{cases}
		x' = \cos \theta \\
		y' = \sin \theta
	\end{cases} $$
	\begin{note}
		$ \theta $ фиксированный.
	\end{note}
	$$ k_n(l) \undereq{\text{т. Мёнье}} k \cos \theta \undereq{\text{т. \ref{th:k_cos}}} \frac{\rom2(x', y')}{\rom1(x', y')} = \frac{\rom2(\cos \theta, \sin \theta)}{\rom1(\cos \theta, \sin \theta)} = \frac{L \cos^2 \theta + 2M \cos\theta\sin\theta + N\sin^2 \theta}{E \cos^2 \theta + 2F\cos\theta\sin\theta + G\sin^2 \theta} $$
	Заменим поверхность на соприкасающийся параболоид:
	$$ k_n(l) = \frac{2A \cos^2 \theta + 2C \sin^2 \theta}{\cos^2 \theta + \sin^2 \theta} = 2A \cos^2 \theta + 2C \sin^2 \theta $$
	При разных значениях $ \theta $:
	\begin{itemize}
		\item $ \theta = 0 \quad \implies \quad 2A = k_1 $
		\item $ \theta = \dfrac\pi2 \quad \implies \quad 2C = k_2 $
	\end{itemize}

	Докажем, что $ k_1, k_2 $ "--- главные кривизны: \\
	Для этого исследуем $ k_n(l) $ на минимум и максимум:
	$$ k_n(l) = k_1\cos^2\theta + k_2(1 - \cos^2 \theta) = (k_1 - k_2)\cos^2 \theta + k_2 $$
	\begin{itemize}
		\item Если $ k_1 \ge k_2 $:
		\begin{itemize}
			\item минимум при $ \cos^2 \theta = 0, \quad k_n = k_2 $;
			\item максимум при $ \cos^2 \theta = 1, \quad k_n = k_1 $.
		\end{itemize}
		\item Если $ k_1 < k_2 $, то максимум и минимум меняются местами.
	\end{itemize}
\end{proof}

\section{Следствия из теоремы Эйлера}

\begin{implication}
	Главные направления перпендикулярны
\end{implication}

\begin{implication}
	Направления $ \theta $ и $ \pi - \theta $ имеют одинаковые нормальные кривизны.
\end{implication}

\section{Вычисление главных кривизн. Формула для гауссовой кривизны}

Пусть $ k \define k_n(l) $ для некоторого $ l $ \\
Введём замену $ x = \dfrac\xi\mu $ (см. следующий вопрос):
$$ k = \frac{Lx^2 + 2Mx + N}{Ex^2 + 2Fx + G} $$
$$ Lx^2 + 2Mx + N = k(Ex^2 + 2Fx + G) $$
При фиксированном $ k $ это "--- квадратное уравнение:
$$ (L - kE)x^2 + 2(M - kF)x + (N - kG) = 0 $$
\begin{itemize}
	\item Если $ k $ "--- не главная кривизна, то у уравнения 2 решения.
	\item Если $ k $ "--- главная кривизна, то у уравнения 1 решение.
	\item Если $ k $ не кривизна, то у уравнения нет решений.
\end{itemize}
Мы ищем главные кривизны, так что $ \faktor D4 = 0 $, то есть
$$ (M - kF)^2 - (L - kE)(N - kG) = 0 $$
$$ k^2(F^2 - EG) - k(2MF - LG - EN) + (M^2 - LN) = 0 $$
$ k_1, k_2 $ "--- корни соответствующего уравнения
$$ K = k_1k_2 = \frac{M^2 - LN}{F^2 - EG} = \frac{LN - M^2}{EG - F^2} = \frac{\det \rom 2}{\det \rom 1} $$
$$ H = \frac{2MF - LG - EN}{2(F^2 - EG)} $$


\section{Вычисление главных направлений}

Выберем направление $ (\xi, \mu) $ в области $ \mc D $:

\begin{figure}[!ht]
	\begin{tikzpicture}[>=Stealth]
		\draw plot[smooth cycle] coordinates {(-2, -2) (2, -1) (3, 2) (-1, 2)} node[anchor=north west] {$ \mc D $};

		\draw[name path=first] (-1, 0.5) -- (1, 0);
		\draw[->, name path=second] (0, -0.25) -- (1, 1) node[anchor=north west] {$ (\xi, \mu) $};

		\path[name intersections={of=first and second, by=x}] (1, 0) coordinate (A) -- (x) coordinate (B) -- (1, 1) coordinate (C) pic[draw, "$ \theta $", angle eccentricity=1.5, angle radius=3mm] {angle};
	\end{tikzpicture}
\end{figure}

Введём замену $ x \define \dfrac\xi\mu $
$$ k_n(\xi, \mu) \undereq{\text{т. Мёнье}} \frac{\rom2(\xi, \mu)}{\rom1(\xi, \mu)} = \frac{L\xi^2 + 2M\xi\mu + N\mu^2}{E\xi^2 + 2F\xi\mu + G\mu^2} = \frac{Lx^2 + 2Mx + N}{Ex^2 + 2Fx + G} $$

Хотим вычислить главные направления, то есть понять, при каких $ x $ параметры $ \xi, \mu $ задают главные направления, \as $ x = \tg \theta $. Найдём максимум и минимум:
$$ \big( k_n(x) \big)' = \frac{(2Lx + 2M)(Ex^2 + 2Fx + G) - (2Ex + 2F)(Lx^2 + 2Mx + N)}{(Ex^2 + 2Fx + G)^2} = 0 $$
$$ Ex^2 + 2Fx + G > 0, \qquad \as ~ \frac D4 = F^2 - EG < 0, \qquad \as ~ \sqrt{EG - F^2} = |r_u' \times r_v'| $$
Значит, нас интересует
$$ (Lx + M)(Ex^2 + 2Fx + G) - (Ex + F)(Lx^2 + 2Mx + N) = 0 $$
$$ 2LFx^2 + LGx + ME x^2 + 2MFx + MG - 2MEx^2 - NEx - FLx^2 - 2FMx - FN = 0 $$
$$ (FL - ME)x^2 + (LG - NE)x + (MG - FN) = 0 $$
Сделаем обратную замену и домножим на $ \mu^2 $:
$$ (FL - ME)\xi^2 + (LG - NE)\xi\mu + (MG - FN)\mu^2 = 0 $$
$$
\begin{vmatrix}
	\xi^2 & -\xi\mu & \mu^2 \\
	G & F & E \\
	N & M & L
\end{vmatrix} = 0 $$

\section{Лемма о смешанных произведениях}

\begin{lemma}[о смешанных произведениях]
	$$ (\vv u, \vv v, \vv w) \cdot (\vv l, \vv m, \vv n) =
	\begin{vmatrix}
		u \cdot l & u \cdot m & u \cdot n \\
		v \cdot l & v \cdot m & v \cdot n \\
		w \cdot l & w \cdot m & w \cdot n
	\end{vmatrix} $$
	(в правой части стоят скалярные произведения)
\end{lemma}

\begin{proof}
	\begin{multline*}
		(\vv u, \vv v, \vv w) \cdot (\vv l, \vv m, \vv n) =
		\begin{vmatrix}
			u_1 & u_2 & u_3 \\
			v_1 & v_2 & v_3 \\
			w_1 & w_2 & w_3
		\end{vmatrix} \cdot
		\begin{vmatrix}
			l_1 & m_1 & n_1 \\
			l_2 & m_2 & n_2 \\
			l_3 & m_3 & n_3
		\end{vmatrix} = \\
		= \left|
		\begin{pmatrix}
			u_1 & u_2 & u_3 \\
			v_1 & v_2 & v_3 \\
			w_1 & w_2 & w_3
		\end{pmatrix} \cdot
		\begin{pmatrix}
			l_1 & m_1 & n_1 \\
			l_2 & m_2 & n_2 \\
			l_3 & m_3 & n_3
		\end{pmatrix} \right| =
		\begin{vmatrix}
			u \cdot l & u \cdot m & u \cdot n \\
			v \cdot l & v \cdot m & v \cdot n \\
			w \cdot l & w \cdot m & w \cdot n
		\end{vmatrix}
	\end{multline*}
	Второй определитель транспонирован, \as он при этом не меняется.
\end{proof}

\section{Блистательная теорема Гаусса}

\begin{theorem}[Egregium]
	Гауссова кривизна зависит только от $ E, F, G $ и их производных.
\end{theorem}

\begin{proof}
	Достаточно доказать, что $ LN - M^2 $ зависит только от \rom1, \as
	$$ K = \frac{LN - M^2}{EG - F^2} $$
	$$ \vv n = \frac{\vv{r_u} \times \vv{r_v}}{|\vv{r_u} \times \vv{r_v}|} $$
	По лемме \ref{lemma:r_u_x_r_v}, $ |r_u \times r_v| = EG - F^2 $
	$$ L \bydef \vv{r_{uu}} \cdot \vv n = \frac{(r_{uu}, r_u, r_v)}{EG - F^2} $$
	Аналогично,
	$$ M = \frac{(r_{uv}, r_u, r_v)}{EG - F^2}, \qquad N = \frac{(r_{vv}, r_u, r_v)}{EG - F^2} $$
	\begin{multline}\lbl1
		LN - M^2 = \frac1{(EG - F^2)^2} \cdot \bigg( (r_{uu}, r_u, r_v) \cdot (r_{vv}, r_u, r_v) - (r_{uv}, r_u, r_v)^2 \bigg) \undereq{\text{лемма}} \\
		=
		\begin{vmatrix}
			r_{uu} r_{vv} & r_{uu}r_u & r_{uu}r_v \\
			r_ur_{vv} & r_ur_u & r_ur_v \\
			r_vr_{vv} & r_vr_u & r_vr_v
		\end{vmatrix} -
		\begin{vmatrix}
			r_{uv}r_{uv} & r_{uv}r_u & r_{uv}r_v \\
			r_ur_{uv} & r_ur_u & r_ur_v \\
			r_vr_{uv} & r_vr_u & r_vr_v
		\end{vmatrix} =
		\begin{vmatrix}
			r_{uu}r_{vv} & r_{uu}r_u & r_{uu}r_v \\
			. & E & F \\
			. & F & G
		\end{vmatrix} -
		\begin{vmatrix}
			. & . & . \\
			. & E & F \\
			. & F & G
		\end{vmatrix}
	\end{multline}
	$$ E_u = (r_u \cdot r_u)_u = 2r_ur_{uu} $$
	$$ G_v = (r_v \cdot r_v)_v = 2r_vr_{vv} $$
	$$ E_v = 2r_ur_{uv} $$
	$$ G_u = 2r_vr_{vu} $$
	$$ F_u = (r_u \cdot r_v)_u = r_{uu}r_v + r_u + r_{uv} $$
	$$ r_vr_{uu} = F_u - r_ur_{uv} = F_u - \half E_v $$
	$$ F_v = r_{uv}r_v + r_ur_{vv} $$
	$$ r_ur_{vv} = F_v - \half G_u $$
	$$ \eref1 =
	\begin{vmatrix}
		r_{uu}r_{vv} & \half E_u & F_u - \half E_v \\
		F_v - \half G_u & E & F \\
		\half G_v & F & G
	\end{vmatrix} -
	\begin{vmatrix}
		r_{uv}^2 & \half E_v & \half G_u \\
		\half E_v & E & F \\
		\half G_u & F & G
	\end{vmatrix} = $$
	$ r_{uu}r_{vv} $ и $ r_{uv}^2 $ \bt{не} вычисляются по-отдельности. Распишем определители по первой строке:
	$$ = r_{uu} \cdot r_{uv} \cdot
	\begin{vmatrix}
		E & F \\
		F & G
	\end{vmatrix} - \dots - r_{uv}^2 \cdot
	\begin{vmatrix}
		E & F \\
		F & G
	\end{vmatrix} + \dots = \underline{(r_{uu}r_{vv} - r_{uv}^2)}
	\begin{vmatrix}
		E & F \\
		F & G
	\end{vmatrix} - \dots $$
	Пропущенные члены зависят только от \rom2. Осталось доказать, что скобка зависит только от \rom2:
	$$ F_{uv} = r_{uuv}r_v + \underline{r_{uu}r_{vv}} + \underline{r_{uv}r_{uv}} + r_ur_{uvv} $$
	$$ G_{uu} = 2r_{uv}r_{uv} + 2r_vr_{uuv} $$
	$$ E_{vv} = 2r_{uv}r_{uv} + 2r_ur_{uvv} $$
	$$ F_{uv} - \half G_{uu} - \half E_{vv} = r_{uu}r_{vv} - r_{uv}r_{uv} $$
\end{proof}

\section{Деривационные формулы}

Разложим вторые производные по базису из первых и $ \vv n $:
$$ \boxed{r_{uu} = \Gamma_{11}^1 \vv r_u + \Gamma_{11}^2 \vv r_v + L \vv n} $$
$$ \boxed{r_{uv} = \Gamma_{12}^1 \vv r_u + \Gamma_{12}^2 \vv r_v + M \vv n} $$
$$ \boxed{r_{vv} = \Gamma_{22}^1 \vv r_u + \Gamma_{22}^2 \vv r_v + N \vv n} $$

Коэффициенты при $ \vv n $ находятся скалярыным умножением на $ \vv n $, например,
$$ \underbrace{r_{uu} \cdot \vv n}_L = \underbrace{\Gamma_{11}^1 \vv r_u \cdot \vv n}_0 + \underbrace{\Gamma_{11}^2 \vv r_v \cdot \vv n}_0 + A \underbrace{\vv n \cdot \vv n}_1 \quad \implies A = L $$

$ \Gamma_{ij}^k $ "--- функции $ u $ и $ v $ . Они называются \it{символами Кристоффеля}.

\section{Коэффициенты в разложении \texorpdfstring{$ n_u, n_v $}{nu, nv}}

Разложим производные $ \vv n $ по базису из производных $ \vv r $ и $ \vv n $: \\
Производная единичного вектора перпендикулярна самому вектору, так что $ n_u, n_v $ не зависят от $ \vv n $:
$$ \boxed{n_u = ar_u + br_v}, \qquad \boxed{n_v = cr_u + dr_v} $$

Домножим первое уравнение на $ r_u $ и $ r_v $:
$$
\begin{cases}
	n_u \cdot r_u = ar_u^2 + br_ur_v = aE + bF \\
	n_u \cdot r_v = ar_ur_v + br_v^2 = aF + bG
\end{cases} $$

Решим методом Крамера:

$$ a = \frac{
	\begin{vmatrix}
		n_ur_u & F \\
		n_ur_v & G
	\end{vmatrix}}{
	\begin{vmatrix}
		E & F \\
		F & G
	\end{vmatrix}}, \qquad b = \frac{
	\begin{vmatrix}
		E & n_ur_u \\
		F & n_ur_v
	\end{vmatrix}}{
	\begin{vmatrix}
		E & F \\
		F & G
	\end{vmatrix}} $$

Аналогично,
$$
\begin{cases}
	n_v \cdot r_u = cE + dF \\
	n_v \cdot r_v = cF + dG
\end{cases} $$
$$ c = \frac{
	\begin{vmatrix}
		n_vr_u & F \\
		n_vr_v & G
	\end{vmatrix}}{
	\begin{vmatrix}
		E & F \\
		F & G
	\end{vmatrix}}, \qquad d = \frac{
	\begin{vmatrix}
		E & n_vr_u \\
		F & n_vr_v
	\end{vmatrix}}{
	\begin{vmatrix}
		E & F \\
		F & G
	\end{vmatrix}} $$

\section{Модуль \texorpdfstring{$ n_u \times n_v $}{nu x nv}}

\begin{theorem}
	$$ |n_u \times n_v | = \frac{LN - M^2}{\sqrt{EG - F^2}} $$
\end{theorem}

\begin{proof}
	$$ n_u = Ar_u + Br_v $$
	$$ n_v = Cr_u + Dr_v $$
	$$ n_u \times n_v = (Ar_u + Br_v) \times (Cr_u + Dr_v) = r_u \times r_v(AD - BC) $$
	$$ |n_u \times n_v| = \underbrace{|r_u \times r_v|}_{\sqrt{EG - F^2}} \cdot |AD - BC| $$
	$$ 0 \undereq{\text{касательный вектор на нормальный}} (\vv{r}_u \cdot \vv n)_u = r_{uu} \cdot n + r_u \cdot n_u \bdefeq L L + r_u \cdot n_u $$
	Аналогично,
	$$ 0 = (r_v \cdot n)_u \bdefeq M M + r_v \cdot n_u $$
	$$ \implies
	\begin{cases}
		AE + DF = n_u \cdot r_u = -L \\
		AF + BG = n_u \cdot r_v = -M
	\end{cases} $$
	$$ A = \frac{FM - GL}{EG - F^2}, \qquad B = \frac{FL - EM}{EG - F^2} $$
	Аналогично найдём $ C $ и $ D $:
	$$
	\begin{cases}
		CE + DF = n_v \cdot r_u = -M \\
		CF + DG = n_v \cdot r_v = -N
	\end{cases} $$
	$$ C = \frac{FN - GM}{EG - F^2}, \qquad D = \frac{FM - EN}{EG - F^2} $$
	\vspace{-1.95em}
	\begin{multline*}
		AD - BC = \frac1{(EG - F^2)^2} \bigg( (FM - GL)(FM - FN) - (FL - EM)(FN - GM) \bigg) = \\
		= \frac1{(EG - F^2)^2} \bigg( F^2M^2 - \cancel{FMEN} - \cancel{GFLM} + GLEN - F^2LN + \cancel{FGLM} + \cancel{EFMN} - EGM^2 \bigg) = \\
		= \frac{(EG - F^2)(LN - M^2)}{(EG - F^2)^2}
	\end{multline*}
\end{proof}

\section({Уравнения Петерсона--Майнарди--Кодацци}){Уравнения Петерсона"--~Майнарди"--~Кодацци}

Вспомним, как мы вводили символы Кристоффеля:
$$ r_{uu} = \Gamma_{11}^1 r_u + \Gamma_{11}^2 r_v + Ln $$
$$ r_{uv} = \Gamma_{12}^1r_u + \Gamma_{12}^2 + Mn $$
$$
\begin{array}l
	r_{uuv} = \Gamma_{11v}^1vr_u + \Gamma_{11}^1r_{uv} + \Gamma_{11v}^2r_v + \Gamma_{11}^2r_{vv} + L_vn + Ln_v \\
	\shortparallel \\
	r_{uvu} = \Gamma_{12u}^1r_u + \Gamma_{12}^1r_{uu} + \Gamma_{12u}^2r_v + \Gamma_{12}^2r_{uv} + M_un + Mn_u
\end{array} $$
Домножим последние два выражения скалярно на $ n $ и приравняем:
$$ \Gamma_{11}^1 \cdot M + \Gamma_{11}^2 \cdot N + L_v = \Gamma_{12}^1 \cdot L + \Gamma_{12}^2M + M_u $$
Обычно это записывается как
$$ \boxed{L_v - M_u = \Gamma_{12}^1L + \Gamma_{12}^2M - \Gamma_{11}^1M - \Gamma_{11}^2N} $$
Аналогично,
$$ r_{uv} = \Gamma_{12}^1r_u + \Gamma_{12}^2r_v + M_n, \qquad r_{vv} = \Gamma_{22}^1r_u + \Gamma_{22}^2r_v + M_n $$
Дифференцируем первое по $ v $, второе "--- по $ u $, домножаем оба на $ n $, приравниваем:
$$ \Gamma_{12}^1M + \Gamma_{12}^2N + M_v = \Gamma_{22}^1L + \Gamma_{22}^2M + N_u $$
$$ \boxed{M_v - N_u = \Gamma_{22}^1L + \Gamma_{22}^2M - \Gamma_{12}^1M - \Gamma_{12}^2N} $$

\section{Символы Кристоффеля относятся к внутренней геометрии}

\begin{theorem}
	$ \Gamma_{ij}^k $ относятся к внутренней геометрии.
\end{theorem}

\begin{proof}
	$$ r_{uu} = \Gamma_{11}^1r_u + \Gamma_{11}^2 r_v + Ln \qquad \bigg| \cdot r_u \qquad \bigg| \cdot r_v $$
	$$
	\begin{cases}
		r_{uu} \cdot r_v = \Gamma_{11}^1E + \Gamma_{11}^2F \\
		r_{uu} \cdot r_v = \Gamma_{11}^1 \cdot F + \Gamma_{11}^2 \cdot G
	\end{cases} $$
	$$ E_u = (r_u \cdot r_u)_u = 2r_{uu} \cdot r_u $$
	$$ F_u = (r_u \cdot r_v)_u = r_{uu} r_v + r_ur_{uv} $$
	$$ E_v = (r_u \cdot r_u)_v = 2r_{uv}r_u $$
	$$ r_{uu}r_v = F_u - \frac12 E_v $$
	$$
	\begin{cases}
		\Gamma_{11}^1E + \Gamma_{11}^2F = \frac12 E_u \\
		\Gamma_{11}^1F + \Gamma_{11}^2G = F_u - \frac12 E_v
	\end{cases} $$
	$$ \boxed{\Gamma_{11}^1 = \frac{
			\begin{vmatrix}
				\frac12E_u & F \\
				F_u - \frac12 E_v & G
			\end{vmatrix}}{EG - F^2}} $$
	Остальные "--- аналогично.
\end{proof}

\section{Геодезическая кривизна отностися к внутренней геометрии}

Есть поверхность. На ней есть кривая. Есть вектор кривизны "--- $ \vv k = k \cdot \vv n_1 $ ($ n_1 $ "--- вектор нормали к кривой).

\begin{definition}
	$ k_g $ "--- проекция $ \vv k $ на касательную плоскость. \\
	Можно рассматривать вектор или скаляр.
\end{definition}

\begin{statement}
	$ k^2 = k_n^2 + k_g^2 $
\end{statement}

\begin{proof}
	$ k_n = \text{Пр}_{n_2} k $, где $ n_2 $ "--- вектор нормали к поверхности. \\
	Тогда утверждение получается по теореме Пифагора.
\end{proof}

\begin{theorem}
	$ k_g $ относится к внутренней геометрии.
\end{theorem}

\begin{proof}
	$$ u = u(s), \qquad v = v(s) $$
	$ S $ "--- постоянный параметр
	\begin{multline*}
		\vv k = \frac{\di[2] r \big( u(s); v(s) \big)}{\di s^2} = \frac{\di}{\di s}(r_u \cdot u_s + r_v \cdot v_s) = \vv{r}_{uu} u_s^2 + 2\vv{r}_{uv}u_sv_v + \vv{r}_{vv}(v_s)^2 = \\
		= (\Gamma_{11}^1 r_u + \Gamma_{11}^2r_v + Ln)u_s^2 + 2(\Gamma_{12}^1r_u + \Gamma_{12}^2r_v + Mn)u_sv_s + (\Gamma_{22}^1 r_u + \Gamma_{22}^2r_v + Nn)v_s^2
	\end{multline*}
	$$ \vv{k}_g = \underbrace{(\Gamma_{11}^1u_s^2 + 2\Gamma_{12}^1u_sv_s + \Gamma_{22}^1v_s^2)}_{\text{зависит от \rom1}} \cdot r_u + \underbrace{(\Gamma_{11}^2u_s^2 + 2\Gamma_{12}^2u_sv_s + \Gamma_{22}^2v_s^2)}_{\text{зависит от \rom2}}r_v $$
\end{proof}

\section{Вычисление геодезической кривизны}

\begin{theorem}
	$$ k_g = \frac{(r_{tt}'', ~ r_t', ~ n)}{|r_t'|^3} $$
\end{theorem}

\begin{proof}
	Пусть $ r_{tt}'' = r_1'' + r_2'' $, где $ r_1'' \perp n, \quad r_2'' \parallel n $
	$$ r' \perp n $$
	$$ r_{tt}'' \times r_t' = \underbrace{r_1'' \times r'}_{\parallel n} + \underbrace{r_2'' \times r'}_{\perp n} $$
	$$ k = \frac{|r'' \times r'|}{|r'|^3} $$
	$$ k_g \iseq \frac{\text{Пр}_{\vv n}(r'' \times r')}{|r'|^3} = \frac{(r'' \times r') \cdot n}{|n| \cdot |r'|^3} = \frac{(r'', r', n)}{|r'|^3} $$
\end{proof}

\begin{statement}
	$$ k_g = \frac{\text{Пр}_{\vv n}(r'' \times r')}{|r'|^3} $$
\end{statement}
\begin{proof}
	Кривизна "--- проекция $ r'' $ на вектор нормали к кривой, а значит,
	$$ r'' \times r' \parallel \vv b \quad \implies (r'' \times r') \times r' \perp \vv b, ~ \perp \vv v \quad \implies \perp \vv n $$
	$$ \implies \vv k = \frac{(r'' \times r') \times r'}{|r'|^4} $$
	$$ |(r'' \times r') \times r'| = |r'' \times r'| \cdot |r'| \cdot \underbrace{\sin \alpha}_{1, \text{ \as } \alpha = \faktor\pi2} $$
	$$ \implies k = \frac{|r'' \times r'}{|r'|^3} = \frac{|(r'' \times r') \times r'}{|r'|^4} $$
	$$ \text{Пр}_{\text{кас. пл.}} \vv k = \frac{\bigg( (r_1'' + r_2'') \times r' \bigg) \times r'}{|r'|^4} = \underbrace{\frac{(r'' \times r') \times r'}{|r'|^4}}_{\perp n} + \underbrace{\frac{\overbrace{(r_2'' \times r')}^{\perp n} \times \overbrace{r'}^{\perp n}}{|r'|^4}}_{\parallel n} = \frac{(r'' \times r') \times r'}{|r'|^4} $$
	$$ k_g = \bigg| \text{Пр}_{\text{кас. пл.}} \vv k \bigg| = \frac{|r_1'' \times r'| \cdot |r'|}{|r'^4|} = \frac{|r_1'' \times r'}{|r'|^3} = \frac{(r'', r', r)}{|r'|^3} $$
\end{proof}

\section{Равносильные определения геодезической}

\begin{theorem}
	Задана кривая на поверхности. Следующие определения геодезических линий равносильны:
	\begin{enumerate}
		\item $ k_g = 0 $;
		\item \label{en:geo:2} вектор главной нормали к кривой параллелен нормали к поверхности;
		\item \label{en:geo:3} соприкасающаяся плоскость кривой содержит нормаль к поверхности;
		\item \label{en:geo:4} спрямляющая плоскость кривой является касательной плоскостью к поверхности;
		\item\label{en:geo:5} $ k $ "--- $ \min $ для всех кривых в данном направлении;
		\item\label{en:geo:6} локально кратчайшие линии.
	\end{enumerate}
\end{theorem}

\begin{eproof}
	\item[\ref{en:geo:2}.]
	$$ k^2 = k_n^2 + k_g^2 $$
	$$ k_n = \text{Пр}_m k, \qquad k = \text{Пр}_n k $$
	При $ n \parallel m $ выполнено $ k_n = k \quad \iff \quad k_g = 0 $

	\item[\ref{en:geo:3}.] Соприкасающаяся плоскость содержит нормаль к кривой.

	\item[\ref{en:geo:4}.] Нормаль к кривой является нормалью к спрямляющей плоскости. Нормаль к поверхность является нормалью к касательной плоскости.

	\item[\ref{en:geo:5}.] $ k_n(l) $ зависит только от направления $ l $.

	\item[\ref{en:geo:6}.] Пока без доказательства.
\end{eproof}

\section{Существование геодезических в данном направлении}

\begin{statement}[из дифуров]
	$ y'' = f(x, y, y'), \qquad f $ непр. по каждому аргументу \\
	$ \implies $ решение существует и единственно (локально).
\end{statement}

\begin{theorem}
	В любой точке в любом направлении можно провести ровно одну геодезичсекую (локально).
\end{theorem}

\begin{proof}
	Условие геодезичсекой "--- $ k_g = 0 $, \ie $ (r_{tt}'', r_t', n) = 0 $ "--- это дифур второго порядка. Надо доказать, что у него существует единственное решение. \\
	Нам надо разрешить дифур относительно $ r'' $.
	$$
	\begin{cases}
		u = t \\
		v = \vphi(t)
	\end{cases} $$
	Нужно доказать, что существует такая $ \vphi $.
	$$ r_t' (u, v) = r_u \cdot u' + r_v \cdot v' = r_u + r_v \vphi' $$
	$$ r_{tt}'' = r_{uu}u'^2 + 2r_{uv}u'v' + r_{vv}v'^2 + r_uu'' + r_vv'' = r_{uu} + 2r_{uv} \vphi' + r_{vv} \vphi'^2 + r_v \vphi'' $$
	\begin{multline*}
		0 = (r'', r', n) = (r_{uu} + 2r_{uv}\vphi' + r_{vv}\vphi'^2 + r_v\vphi'', r_u + r_v \vphi', n) = \\
		= \underbrace{(r_{uu}'' + 2r_{uv}\vphi' + r_{vv}\vphi'^2, r_u + r_v\vphi', n)}_{\rom1} + \underbrace{(r_v \vphi'', r_u + r_v\vphi', n)}_{\rom2}
	\end{multline*}
	$$ \rom2 = \vphi'' \cdot (r_v, r_u + \cancel{r_v\vphi'}, n) = -\vphi'' \cdot (r_u, r_v, n) $$
	(\as $ (r_v, r_u + r_v\vphi', n) = (r_v, r_u, n) + \underbrace{(r_v, r_v\vphi', n)}_{0, \text{ \as } r_v \parallel r_v\vphi'} $)
	$$ \boxed{\vphi'' = \frac{\rom1}{(r_u, r_v, n)}} $$
	$ (r_u, r_v, n) \ne 0 $ (\as $ r_u, r_v, n $ "--- базис, \as повехность регулярная). \\
	Значит, у такого дифура есть ровно одно решение с начальными данными
	$$
	\begin{cases}
		\vphi(t_0) = \vphi_0 \\
		\vphi'(t_0) = \vphi_1
	\end{cases} $$
\end{proof}

\section{Полугеодезическая параметризация}

Полугеодезическая параметризация: $ E = 1, ~ F = 0, ~ G > 0 $

\begin{theorem}
	Полугеодезическая параметризация всегда существует (локально).
\end{theorem}

\begin{proof}
	$$ (r_u \cdot r_v)_u = \underbrace{r_{uu} \cdot r_v}_0 + r_u \cdot r_{uv} = \cancel{f'' \cdot g'} + f' \cdot r_{uv} = \underbrace{f' \cdot (f')_v}_0 = 0 $$
	$ r_{uu} = f'' \parallel $ вектору главной нормали для $ f $ (\as $ f $ в натуральной параметризации) $ \parallel $ нормали к поверхности (\as $ f $ "--- геодезическая) \\
	$ r_v $ "--- касательный вектор. \\
	На первой лекции доказывали полезную лемму:
	$$ |f'| = 1 \implies \pder{f'}v \perp f' $$
	$$ F = r_u \cdot r_v = \const $$
	Но при $ u = 0 \quad F = 0 $
	$$ \implies F = 0 \text{ всюду} $$
\end{proof}

\section{Геодезические как локально кратчайшие}

\begin{proof}
	\hfill \\
	Рассмотрим полугеодезическую параметризацию. \\
	Возьмём точки $ A, B $ на геодезической. \\
	Пусть $ \big( u(t), v(t) \big) $ "--- внутренние координаты некоторой кривой, соединяющей $ A $ и $ B $. Её длина:
	\begin{multline*}
		l = \dint[t]{t_0}{t_1}{\sqrt{\underset1Eu'^2 + 2\underset0Fu'v' + Gv'^2}} = \dint[t]{t_0}{t_1}{u'^2 + \underset{> 0}Gv'^2} \ge \dint[t]{t_0}{t_1}{u'^2} = \\
		= \dint[t]{t_0}{t_1}{u'} = u(t_1) - u(t_0) = \text{ длина геодезической}
	\end{multline*}
	Мы доказали, что геодезическая "--- кратчайшая. В другую сторону "--- без доказательства.
\end{proof}
