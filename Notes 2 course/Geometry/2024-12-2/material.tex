\section{Деривационные формулы}

\begin{theorem}
	$$ |n_u \times n_v | = \frac{LN - M^2}{\sqrt{EG - F^2}} $$
\end{theorem}

\begin{proof}
	$$ n_u = Ar_u + Br_v $$
	$$ n_v = Cr_u + Dr_v $$
	$$ n_u \times n_v = (Ar_u + Br_v) \times (Cr_u + Dr_v) = r_u \times r_v(AD - BC) $$
	$$ |n_u \times n_v| = |\underbrace{r_u \times r_v}| \cdot |AD - BC| $$
	$$ 0 \undereq{\text{касательный вектор на нормальный}} (\vec{r}_u \cdot \vec n)_u = r_{uu} \cdot n + r_u \cdot n_u \bdefeq L L + r_u \cdot n_u $$
	Аналогично,
	$$ 0 = (r_v \cdot n)_u \bdefeq M M + r_v \cdot n_u $$
	$$ \implies
	\begin{cases}
		AE + DF = n_u \cdot r_u = -L \\
		AF + BG = n_u \cdot r_v = -M
	\end{cases} $$
	$$ A = \frac{FM - GL}{EG - F^2}, \qquad B = \frac{FL - EM}{EG - F^2} $$
	Аналогично найдём $ C $ и $ D $:
	$$
	\begin{cases}
		CE + DF = n_v \cdot r_u = -M \\
		CF + DG = n_v \cdot r_v = -N
	\end{cases} $$
	$$ C = \frac{FN - GM}{EG - F^2}, \qquad D = \frac{FM - EN}{EG - F^2} $$
	\begin{multline*}
		AD - BC = \frac1{(EG - F^2)^2} \bigg( (FM - GL)(FM - FN) - (FL - EM)(FN - GM) \bigg) = \\
		= \frac1{(EG - F^2)^2} \bigg( F^2M^2 - \cancel{FMEN} - \cancel{GFLM} + GLEN - F^2LN + \cancel{FGLM} + \cancel{EFMN} - EGM^2 \bigg) = \\
		= \frac{EG - F^2)(LN - M^2)}{(EG - F^2)^2}
	\end{multline*}
\end{proof}

\section(Уравнения Петерсона--Майнарди--Кодаци){Уравнения Петерсона"--~Майнарди"--~Кодаци}

Вспомним, как мы вводили симолы Кристофеля:
$$ r_{uu} = \Gamma_{11}^1 r_u + \Gamma_{11}^2 r_v + Ln $$
$$ r_{uv} = \Gamma_{12}^1r_u + \Gamma_{12}^2 + Mn $$
$$ r_{uuv} = \Gamma_{11v}^1vr_u + \Gamma_{11}^1r_{uv} + \Gamma_{11v}^2r_v + \Gamma_{11}^2r_{vv} + L_vn + Ln_v $$
$$ = r_{uvu} = \Gamma_{12u}^1r_u + \Gamma_{12}^1r_{uu} + \Gamma_{12u}^2r_v + \Gamma_{12}^2r_{uv} + M_un + Mn_u $$
Домножим последние два выражения скалярно на $ n $ и приравняем:
$$ \Gamma_{11}^1 \cdot M + \Gamma_{11}^2 \cdot N + L_v = \Gamma_{12}^1 \cdot L + \Gamma_{12}^2M + M_u $$
Обычно это записывается как
$$ \boxed{L_v - M_u = \Gamma_{12}^1L + \Gamma_{12}^2M - \Gamma_{11}^1M - \Gamma_{11}^2N} $$
Аналогично,
$$ r_{uv} = \Gamma_{12}^1r_u + \Gamma_{12}^2r_v + M_n, \qquad r_{vv} = \Gamma_{22}^1r_u + \Gamma_{22}^2r_v + M_n $$
Дифференцируем первое по $ v $, второе "--- по $ u $, домножаем оба на $ n $, приравниваем:
$$ \Gamma_{12}^1M + \Gamma_{12}^2N + M_v = \Gamma_{22}^1L + \Gamma_{22}^2M + N_u $$
$$ \boxed{M_v - N_u = \Gamma_{22}^1L + \Gamma_{22}^2M - \Gamma_{12}^1M - \Gamma_{12}^2N} $$

\begin{theorem}
	$ \Gamma_{ij}^k $ относятся к внутренней геометрии.
\end{theorem}

\begin{proof}
	$$ r_{uu} = \Gamma_{11}^1r_u + \Gamma_{11}^2 r_v + Ln \qquad \bigg| \cdot r_u \qquad \bigg| \cdot r_v $$
	$$
	\begin{cases}
		r_{uu} \cdot r_v = \Gamma_{11}^1E + \Gamma_{11}^2F \\
		r_{uu} \cdot r_v = \Gamma_{11}^1 \cdot F + \Gamma_{11}^2 \cdot G
	\end{cases} $$
	$$ E_u = (r_u \cdot r_u)_u = 2r_{uu} \cdot r_u $$
	$$ F_u = (r_u \cdot r_v)_u = r_{uu} r_v + r_ur_{uv} $$
	$$ E_v = (r_u \cdot r_u)_v = 2r_{uv}r_u $$
	$$ r_{uu}r_v = F_u - \frac12 E_v $$
	$$
	\begin{cases}
		\Gamma_{11}^1E + \Gamma_{11}^2F = \frac12 E_u \\
		\Gamma_{11}^1F + \Gamma_{11}^2G = F_u - \frac12 E_v
	\end{cases} $$
	$$ \boxed{\Gamma_{11}^1 = \frac{
			\begin{vmatrix}
				\frac12E_u & F \\
				F_u - \frac12 E_v & G
			\end{vmatrix}}{EG - F^2}} $$
	Остальные "--- аналогично.
\end{proof}

\chapter{Геодезическая кривизна}

Есть поверхность. Есть кривая на поверхности. Есть вектор кривизны (вектор нормали умножить на кривизну). Есть картинка.

\begin{definition}
	$ k_g $ "--- проекция $ \vec k $ на касательную плоскость. \\
	Можно рассматривать как скаляр или как вектор.
\end{definition}

\begin{statement}
	$ k^2 = k_n^2 + k_g^2 $
\end{statement}

\begin{proof}
	См. картинку.
\end{proof}

\begin{theorem}
	$ k_g $ относится к внутренней геометрии.
\end{theorem}

\begin{proof}
	$$ u = u(s), \qquad v = v(s) $$
	$ S $ "--- постоянный параметр
	\begin{multline*}
		\vec k = \frac{\di[2] r \big( u(s); v(s) \big)}{\di s^2} = \frac{\di}{\di s}(r_u \cdot u_s + r_v \cdot v_s) = \vec{r}_{uu} u_s^2 + 2\vec{r}_{uv}u_sv_v + \vec{r}_{vv}(v_s)^2 = \\
		= (\Gamma_{11}^1 r_u + \Gamma_{11}^2r_v + Ln)u_s^2 + 2(\Gamma_{12}^1r_u + \Gamma_{12}^2r_v + Mn)u_sv_s + (\Gamma_{22}^1 r_u + \Gamma_{22}^2r_v + Nn)v_s^2
	\end{multline*}
	$$ \vec{k}_g = \underbrace{(\Gamma_{11}^1u_s^2 + 2\Gamma_{12}^1u_sv_s + \Gamma_{22}^1v_s^2)}_{\text{зависит от \rom1}} \cdot r_u + \underbrace{(\Gamma_{11}^2u_s^2 + 2\Gamma_{12}^2u_sv_s + \Gamma_{22}^2v_s^2)}_{\text{зависит от \rom2}}r_v $$
\end{proof}

\section{Вычисление \texorpdfstring{$ k_g $}{kg}}

$$ r(u, u), \qquad u(t), \quad v(t) $$
$ \vec n $ "--- нормаль к поверхности.
$$ (r_t', r_t'', n) (r_uu' + r_vv'; r_{uu}u') \widedots[3cm] $$
