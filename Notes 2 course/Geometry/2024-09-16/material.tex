\section{Продолжаем репер Френе}

\begin{notation}
	$ \dot{r} $ -- производная по натуральному параметру
\end{notation}

\begin{undefthm}{На чём мы остановились}
	$ \vect{r}(s) $ -- натуральная параметризация
	$$ v(t) \define r'(s) \quad (\define \dot{r}(s)), \qquad |v| = 1 $$
	$ \dot{v} \perp v $ (по лемме)
	$$ \vect{n} \define \frac{\dot{v}}{|\dot{v}|} \qquad \vect{n}(s) \text{ -- вектор главной нормали} $$
	$$ \vect{b} \vect{v} \times \vect{n} $$
	Тем самым, $ (v, n, b) $ -- правая тройка (называется репер Френе)
\end{undefthm}

\begin{remark}
	Репер Френе существует только если $ \dot{v} \ne \vect{0} $
\end{remark}

\begin{definition}
	Кривая называется бирегулярной, если $ \dot{v} \ne \vect{0} $ ни при каких $ s $ (т. е. в любой точке существет репер Френе)
\end{definition}

\begin{exmpls}
	\item Прямая \textbf{не} бирегулярна
	\item Если у кривой есть точка перегиба, то она \textbf{не} бирегулярна в этой точке (чуть позже будет объяснение)
\end{exmpls}

\begin{statement}
	Если $ \forall $ регулярной параметризации $ r''(t) \not\parallel r'(t) $, то кривая бирегулярна \\
	Обратное \textbf{верно}
\end{statement}

\begin{proof}
	Позже возникнет естественным образом
\end{proof}

\begin{itemize}
	\item $ \braket{v, n} $ называется соприкасающейся плоскостью
	\item $ \braket{b, n} $ называется нормальной плоскостью
	\item $ \braket{v, b} $ называется спрямляющей плоскостью
\end{itemize}

\begin{remark}
	Мы пока не умеем искать уравнения этих плоскостей (т. к. в начале нам нужна регулярная параметризация, к которой редко удаётся перейти)
\end{remark}

\subsection{Формула Френе}

$$ \boxed{\dot{v} = k \cdot \vect{n}} \text{ -- первая формула Френе} $$
$ k(s) \ge 0 $ (значит, кривая бирегулярна) \\
$ k(s) $ -- скалярная функция

\begin{definition}
	$ k(s) $ называется кривизной кривой
\end{definition}

\begin{remark}
	Первую формулу не надо доказывать. Она получается из определения
\end{remark}

\begin{eg}
	$ x^2 + y^2 = R^2 $
	$$
	\begin{cases}
		x = R \cos t \\
		y = R \sin t \\
		z = 0
	\end{cases} \quad \text{-- \textbf{не} натуральная параметризация} $$
	$$ s = \dint[\tau]0t{|r'(\tau)} = \dint[\tau]0t{\sqrt{R^2 \sin^2 \tau + R^2 \cos^2 \tau}} = R \dint[\tau]0t{} = Rt $$
	$$ t = \frac{s}R $$
	Натуральная параметризация:
	$$
	\begin{cases}
		x = R \cos \faktor{s}R \\
		y = R \sin \faktor{s}R \\
		z = 0
	\end{cases} $$
	Интуитивно, это параметризация, такая, что мы проходим окружность за $ 2\pi R $. Для этого мы ``замедлили время'' в $ R $ раз
	$$ \vect{v} = (\dot{x}, \dot{y}, \dot{z}) = \bigg( -\sin \frac{s}R, \cos \frac{s}R, 0 \bigg) $$
	$$ \dot{v} = \bigg( -\frac1R \cos \frac{s}R, -\frac1R \sin \frac{s}R, 0 \bigg) $$
	$$ |\dot{v}| = \frac1R = k $$
	$$ \vect{n} = \bigg( -\cos \frac{s}R, -\sin \frac{s}R, 0 \bigg) $$
	$$ \vect{b} = (0, 0, 1) $$
\end{eg}

$$ \boxed{\dot{b} = - \text{\ae} n} \text{ -- вторая формула Френе} $$

\begin{definition}
	$ \text{\ae} $ называется кручением кривой
\end{definition}

\begin{theorem}
	$ \vect{\dot{b}} \parallel \vect{n} $
\end{theorem}

\begin{proof}
	$ \dot{b} \perp b $ по лемме \\
	Докажем, что $ \dot{b} \perp v $:
	$$ \dot{b} = (v \times n)^. = \underbrace{\dot{v} \times n}_{= 0 } + \underbrace{v \times \dot{n}}_{\perp v} \perp v $$
\end{proof}

\begin{statement}
	Кривая плоская $ \iff \text{\ae} = 0 $
\end{statement}

\begin{proof}
	Упражнение
\end{proof}

$$ \dot{n} = (b \times v)^. = \dot{b} \times v + b \times \dot{v} = -\text{\ae} \underbrace{n \times v}_{= -b} + \underbrace{b \times kn}_{= -kv} = \text{\ae}b - kv $$
$$ \boxed{\vect{\dot{n}} = \text{\ae}\vect{b} - k\vect{v}} \text{ -- тертья формула Френе} $$

Все формулы Френе: \\
\begin{tabular}{c | c | c | c}
	& $ v $ & $ n $ & $ b $ \\
	\hline
	$ \dot{v} $ & 0 & $ k $ & 0 \\
	\hline
	$ \dot{n} $ & $ -k $ & 0 & \ae \\
	\hline
	$ \dot{b} $ & 0 & -\ae & 0
\end{tabular} \\
Таблица антисимметрична

\section{Соприкасающаяся плоскость}

\begin{theorem}
	$ r(t) $ -- произвольная регулярная параметризация бирегулярной кривой
	$$ \implies r''(t) \in \braket{v, n} $$
\end{theorem}

\begin{proof}
	Пусть $ s $ -- натуральный параметр
	$$ \frac{\di r}{\di t} = \frac{\di r}{\di s} \cdot \frac{\di s}{\di t} = \dot{r} \cdot s' $$
	$$ r'' = \frac{\di^2 r}{\di t^2} = \frac{\di \dot{r}}{\di t} s' + \dot{r} \cdot s'' = \overset{..}r \cdot (s')^2 + \dot{r} s'' = k \vect{n} \cdot (s')^2 + vs'' $$
	$$ \boxed{r'' = k(s')^2 \cdot \vect{n} + s''\vect{v}} $$
	То есть, $ r'' $ раскладывается по векторам $ v $ и $ n $ \\
	Причём, если параметризация бирегулярна, то $ k(s')^2 \ne 0 $
\end{proof}

\begin{implication}
	Соприкасающаяся плоскость $ = \braket{r'(t), r''(t)} $
\end{implication}

\begin{problem}
	Вычислить $ v, n, b $ и плоскости для произвольной параметризации $ r(t) $
\end{problem}

$$ \vect{v} = \frac{r'(t)}{|r'(t)|} $$
$$ \vect{b} = \frac{r'' \times r'}{|r' \times r''|} $$
$$ \vect{n} = \vect{b} \times \vect{n} = \frac{(r' \times r'') \times r'}{|r'| \cdot |r' \times r''|} $$
Пусть
$$ r(t) = \big( x(t), y(t), z(t) \big) $$
$$ \vect{v} \parallel (x', y', z') = r' \text{ -- вектор нормали для нормальной плоскости} $$
Тогда нормальная плоскость пишется так:
$$ \boxed{x'\clamp{t_0}(x - x_0) + y'\clamp{t_0}(y - y_0) + z'\clamp{t_0}(z - z_0) = 0} $$
Найдём соприкасающуюся плоскость:
$$
\begin{vmatrix}
	x'\clamp{t_0} & y'\clamp{t_0} & z'\clamp{t_0} \\
	x''\clamp{t_0} & y''\clamp{t_0} & z''\clamp{t_0} \\
	x - x_0 & y - y_0 & z - z_0
\end{vmatrix} = 0 $$
(это -- смешанное произведение трёх векторов) \\
Спрямляющая плоскость -- упражнение

$$ A = r(t_0), \qquad B = r(t_0) $$
$$ \delta \define \dist(B, \text{сопряж.}) $$

\begin{theorem}
	$$ \liml{t \to t_0} \frac\delta{AB^2} = 0 $$
\end{theorem}

\begin{proof}
	Введём подходящие координаты, такие, что:
	\begin{itemize}
		\item Соприкас. = $ OXY $
		\item Кас. прямая = $ OX $
		\item $ A $ -- начало координат
		\item $ t_0 = 0 $
		\item $ y''(t) \ne 0 $
		\item $ x'(0) \ne 0 $
	\end{itemize}
	Тогда $ r(t) = \bigg( x(t), y(t), z(t) \bigg) $
	$$ x(0) = y(0) = z(0) = 0, \quad \text{т. к. } A \text{ -- начало координат} $$
	$$ y'(0) = z'(0) = 0, \quad \text{т. к. касательная прямая -- } OX $$
	$$ z''(0) = 0, \quad \text{т. к. соприкас. -- } OXY $$
	$$ \delta = z(t) $$
	Нужно сосчитать предел
	$$ \limz{t} \frac{z(t)}{x^2 + y^2 + z^2} $$
	Разложим по Тейлору:
	$$ z(t) = o(t^2) $$
	$$ y(t) = \half[y''(t)]t^2 + o(t^2) $$
	$$ x(t) = x'(0)t + \half[x''(0)]t^2 + o(t^2) $$
	$$ \lim = \limz{t} \frac{o(t^2)}{x'^2(0) t^2 + o(t^2)} = 0 $$
\end{proof}

\begin{implication}
	Соприкасающаяся плоскость -- единственная, обладающая таким свойством
\end{implication}
