\chapter{Дифферинциальная геометрия поверхностей}

\begin{eg}[сфера]
	\hfill
	\begin{undefthm}{Почему сфера параметризуется именно так?}
		Как параметризуется любая поверхность вращения: \\
		Есть кривая на плоскости $ XOZ $ (рис. \ref{tikz:1.a}):
		$$
		\begin{cases}
			x = f(t) \\
			z = g(t)
		\end{cases} $$
		Начали вращать -- получили ``кувшин'' (рис. \ref{tikz:1.b}). Как его запараметризовать?
		$$
		\begin{cases}
			x = f(t) \cdot \cos \alpha \\
			y = f(t) \cdot \sin \alpha
			z = g(t)
		\end{cases} $$
		Сфера -- продукт вращения полуокружности:
		$$
		\begin{cases}
			x = R \cos \psi \\
			z = R \sin \psi
		\end{cases} $$
		Сфера:
		$$
		\begin{cases}
			x = R \cos \vphi \cos \psi \\
			y = R \sin \vphi \cos \psi \\
			z = R \sin \psi
		\end{cases} $$
		где $ \psi $ -- ``широта'', т. е. угол подъёма над экватором, $ \psi $ -- ``долгота'' (то, что было $ \alpha $)
	\end{undefthm}
	$$ r_\vphi =
	\begin{pmatrix}
		-R \cos \vphi \cos \psi \\
		R \cos \vphi \cos \psi \\
		0
	\end{pmatrix}, \qquad r_\psi =
	\begin{pmatrix}
		-R \cos \vphi \sin \psi \\
		-R \sin \vphi \sin \psi \\
		R \cos \psi
	\end{pmatrix} $$
	$$ E = R^2 \cos^2 \psi $$
	$$ F = r_\vphi \cdot r_\psi = 0 $$
	$$ G = r_\psi^2 = R^2 \cos^2 \vphi \sin^2 \psi + R^2 \sin^2 \vphi \sin^2 \psi + R^2 \cos^2 \psi = R^2 $$
\end{eg}

\begin{figure}[!ht]
	\begin{subcaptionblock}{0.49\textwidth}
		\begin{tikzpicture}[>=Stealth]
			\draw[->] (-2.5, 0) -- (2.5, 0) node[right]{$ x $};
			\draw[->] (0, -2.5) -- (0, 2.5) node[above]{$ z $};

			\draw[blue] plot[smooth] coordinates {(1, -2) (0.5, 0) (2, 1) (1, 2)};
		\end{tikzpicture}
		\subcaption{Кривая}
		\label{tikz:1.a}
	\end{subcaptionblock}
	\begin{subcaptionblock}{0.49\textwidth}
		\begin{tikzpicture}[>=Stealth]
			\draw[->] (-2.5, 0) -- (2.5, 0) node[right]{$ x $};
			\draw[->] (0, -2.5) -- (0, 2.5) node[above]{$ z $};

			\draw[blue] plot[smooth] coordinates {(1, -2) (0.5, 0) (2, 1) (1, 2)};
			\draw[blue] plot[smooth] coordinates {(-1, -2) (-0.5, 0) (-2, 1) (-1, 2)};
			\draw[blue] (0, 2) ellipse[x radius = 1, y radius = 0.2];
			\draw[blue] (0, -2) ellipse[x radius = 1, y radius = 0.2];
		\end{tikzpicture}
		\subcaption{Кувшин}
		\label{tikz:1.b}
	\end{subcaptionblock}
\end{figure}

\section{Угол между кривыми на поверхности}

Есть параметризация поверхности $ r(u, v) $ и внутренние параметризации кривых: $ (u, v) = \big( u_1(t), v_1(t) \big) $ и $ (u, v) = \big( u_2(t), v_2(t) \big) $ \\
Пусть касательный вектор к первой кривой будет $ \vec{V}_1 $, ко второй -- $ \vec{V}_2 $
$$ \vec{V}_1 = \frac{\di \vec{r} \big( u_1(t), v_1(t) \big)}{\di t} = r_u' \cdot u_1' + r_v \cdot v_1' $$
Аналогично,
$$ \vec{V}_2 = r_u u_2' + r_v v_2' $$
$$ \cos \alpha = \frac{\vec{V}_1 \cdot \vec{V}_2}{|V_1| \cdot |V_2|} = \frac{Eu_1'u2' + F(u_1'v_2' + u_2'v_1') + Gv_1'v_2'}{\sqrt{Eu_1'^2 + 2Fu_1'v_1' + Fv_1'^2} \cdot \sqrt{Eu_2'^2 + Fu_2'v_2' + Gv_2'^2}} $$

\section{Внутренняя геометрия поверхности}

Есть лист бумаги, на нём живут бумажные клещи. Они не видят ничего, кроме своего листа

\begin{definition}
	Изометрия поверхностей:
	$$ r_1 : D_1 \to \R^3, \qquad r_2 : D_2 \to \R^3 $$
	$ \Phi : D_1 \to D_2 $ назыается изометрией, если длина кривой на $ r_1(D_1) $ равна длине образа \\
	\textit{Тут нужна картинка}
\end{definition}

С учётом некоторой перепараметризации рассмотрим в некотором роде одинаковые параметризации

\begin{theorem}
	Поверхности изометричны \bt{тогда и только тогда}, когда в некоторой параметризации у них совпадают коэффициенты $ E, F, G $
\end{theorem}

\begin{iproof}
	\item $ \impliedby $: \\
	Формула длины кривой:
	$$ l = \dint[t]{a}b{\sqrt{Eu'^2 + 2Fu'v' + Gv'^2}} $$
	Они должны совпасть
	$$ \ol{\Phi} \define r_2 \circ r_1^{-1} $$
	\item $ \implies $: \\
	Возьмём кривую
	$$
	\begin{cases}
		u(t) = t \\
		v(t) = t_0
	\end{cases} $$
	Можно НУО считать, что параметризованы одинаково (иначе перепараметризуем)
	$$ l_1 = \dint[t]{a}b{\sqrt{E_1}} = \sqrt{E_1}(b - a) $$
	$$ l_2 = \dint[t]{a}b{\sqrt{E_2}} = \sqrt{E_2}(b - a) $$
	По предположению, они равны \\
	Тем самым $ E_1 = E_2 $ \\
	Аналогично, взяв
	$$
	\begin{cases}
		u(t) = t_0 \\
		v(t) = t
	\end{cases} $$
	получаем $ G_1 = G_2 $ \\
	Взяв
	$$
	\begin{cases}
		u(t) = t + t_0 \\
		v(t) = t + t_1
	\end{cases} $$
	получаем
	$$ \dint[t]{a}b{\sqrt{E_1 + 2F_1 + G_1}} = \dint[t]{a}b{\sqrt{E_2 + 2F_2 + G_2}} \quad \implies F_1 = F_2 $$
\end{iproof}

\begin{remark}
	То, что поверхности изометричны, \bt{не} означает, что первые квадратичные формы совпадут. Это лишь значит, что \bt{сущетсвует} параметризация, в которой они совпали
\end{remark}

\subsection{Рецепт использования}

\begin{definition}
	$ k $ -- характеристика поверхности \\
	Говорят, что $ k $ относится к внутренней геометрии поверхности, если $ k $ не меняется при изометрии
\end{definition}

\begin{restate}
	$ k $ зависит только от $ E, F, G $
\end{restate}

\begin{remark}
	$ E, F, G $ -- функции. Так что зависимость от их производны тоже пойдойдёт
\end{remark}

\begin{problem}[поставленная Гауссу]
	Можно ли нарисовать карту, не искажающую расстояния?
\end{problem}

\begin{restate}
	Изометрична ли сфера плоскости?
\end{restate}

\begin{answer}
	Нет
\end{answer}

В дальнейшем это будет доказано

\section{Площадь поверхности}

\subsection{Интуиция}

Наверное, площадь не должна меняться при изометрии

\begin{undefthm}{Как хосется определить площадь поверхности:}
	Натыкать много точек, соединить их треугольничками, взять предел суммы их площадей
\end{undefthm}

Оказывается, так сделать нельзя:

\begin{eg}[сапог Шварца]
	Есть цилиндр \\
	Его можно развернуть в прямоугольник и посчитать площадь:
	$$ S = 2\pi R H $$
	Разобъём на треугольники:
	\begin{enumerate}
		\item Порежем на слои
		\item В каждый слой впишем правильный $ n $-уголькик
		\item Построим антипризму \\
		Антипризма:
		\begin{enumerate}
			\item Берём в верхнем и нижнем основании правильный $ n $-угольник
			\item Поворачиваем одно из оснований так, чтобы напротив вершины оказалась середина стороны
			\item Соединим вершины
		\end{enumerate}
		\item Счиатем, что высоту мы разбили на $ k $ частей, каждая часть представляет собой $ n $-угольную антипризму
		\begin{quest}
			Сколько в одном слое треугольников?
		\end{quest}
		\begin{answer}
			$ 2n $
		\end{answer}
		Всего $ 2kn $ треугольников
		\item Посчитаем площадь каждого треугольника:
		$$ S_\triangle = \half lh $$
		Найдём $ l $: \\
		Есть круг радиуса $ R $, мы в него вписали правильный $ n $-угольник, хотим посчитать его сторону.  Угол в центре будет равен $ \dfrac{2\pi}n $. Применим теорему косинусов:
		$$ l^2 = 2R^2 - 2R^2 \cos \frac{2\pi}n $$
		$$ l = R \sqrt{2 \bigg( 1 - \cos \frac{2\pi}n \bigg) } = 2R\sin \frac\pi{n} $$
		Найдём $ h $:
		\begin{remark}
			Треугольник немножко искривляется. Направление его высоты не совпадает с напрвлением образующей цилиндра. $ h \bm{\ne} \dfrac{H}k $
		\end{remark}
		$$ a = R - R \cos \frac\pi{n} $$
		$$ h = \sqrt{\frac{H^2}{k^2} + R^2 \bigg(1 - \cos \frac\pi{n} \bigg)^2} = \sqrt{\frac{H^2}{k^2} + R^2 \cdot 4 \sin^4 \frac\pi{2n}} $$
		$$ S \stackrel?= \limi{\pi n} 2kn \cdot \half \cdot 2R\underbrace{\sin \frac\pi{n}}_{\sim \faktor\pi{n}} \cdot \sqrt{\frac{H^2}{k^2} + 4R^2 \sin^4 \frac\pi{2n}} = 2\pi RH \lim \sqrt{1 + \frac{k^2}{H^2} \cdot 4R^2 \sin^4 \frac\pi{2n}} $$
		Это равно $ 2\pi RH $ тогда и только тогда, когда
		$$ \frac{k^2}{H^2} \cdot 4R^2 \sin^4 \frac\pi{2n} \stackrel?\to 0 $$
		$$ \frac{\pi^4R^2}{4H^2} \cdot \frac{k^2}{n^4} \stackrel?\to 0 $$
		То есть, $ k \stackrel?= o(n^2) $ \\
		То есть, разбивая на много низких, крупно нарезанных антипризм, получаем неправильную площадь
	\end{enumerate}
\end{eg}

Исправить $ k $ и $ n $ мы не можем. Так что проблема в том, что треугольники не совпадают с касательными плоскостями

\subsection{Правильное определение}

Рассмотрим разбиение координатными линиями. Каждый полученный криволинейный четырёхугольник проецируем на касательные плоскости \\
Берём предел суммы площадей этих проекций

\textit{Надо это нормально записать}

\begin{theorem}
	$ S = \iint\sqrt{r_u \times r_v}\di u \di v $
\end{theorem}

\begin{proof}
	Без доказательства (на самом деле, это и есть определение двойного интеграла)
\end{proof}

\begin{theorem}
	$ S = \iint\sqrt{EG - F^2}\di u \di v $
\end{theorem}

Следует из леммы:

\begin{lemma}
	$ |r_u \times r_v | = \sqrt{EG - F^2} $
\end{lemma}
