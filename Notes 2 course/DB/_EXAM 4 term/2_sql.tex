\part{Язык SQL}

\begin{note}
	В этой части предполагается, что читатель в состоянии самостоятельно сходу написать несколько десятков запросов на каждую конструкцию.
	В презентации было слайдов 150 только на примеры.
\end{note}

\section{Язык запросов SQL: операции реляционной алгебры}

Проекция:
\begin{minted}{SQL}
	SELECT name FROM person
\end{minted}

Селекция:
\begin{minted}{SQL}
	SELECT name, price FROM items WHERE price >= 100 AND price <= 150
\end{minted}

Декартово произведение:
\begin{minted}{SQL}
	SELECT * FROM person, dept
\end{minted}

Соединение:
\begin{itemize}
	\item Внутреннее:
		\begin{minted}{SQL}
			SELECT ... FROM t_1, t_2 WHERE t_1.f_1 = t_2.f_2
			SELECT ... FROM t_1 [INNER] JOIN t_2 ON t_1.f_1 = t_2.f_2
		\end{minted}
	\item Внешнее:
		\begin{itemize}
			\item \texttt{JOIN}: строки, в которых есть хотя бы одно совпадение в обеих таблицах;
			\item \texttt{LEFT JOIN}: строки из левой таблицы, даже если нет совпадения в правой;
			\item \texttt{RIGHT JOIN}: строки из правой таблицы, даже если нет соединения в левой;
			\item \texttt{FULL JOIN}: строки из обеих таблиц.
		\end{itemize}
	\item Естественное:
		\begin{minted}{SQL}
			SELECT group, st FROM S JOIN R ON S.lang = R.lang
			SELECT * FROM S JOIN R USING (lang)
			SELECT * FROM S NATURAL JOIN
		\end{minted}
\end{itemize}

Теоретико-множественные операции:
\begin{itemize}
	\item \texttt{UNION};
	\item \texttt{INTERSECT};
	\item \texttt{EXCEPT}.
\end{itemize}
Отношения должны быть совместимы, \ie иметь одинаковое количество полей с совместимыми типами данных.
По-умолчанию все операции \texttt{DISTINCT}.

\section{Язык запросов SQL: вложенные подзапросы}

\begin{definition}
	\emph{Вложенный SQL-запрос} "--- это отдельный запрос, который используется внутри SQL инструкции.

	Инструкцию, в которой используется вложенный запрос, называют \emph{внешним SQL-запросом}.
\end{definition}

Вложенные запросы:
\begin{itemize}
	\item \texttt{SELECT}:
		\begin{minted}{SQL}
			SELECT *, (SELECT name FROM car_owner WHERE inn = owner) FROM car
		\end{minted}
	\item \texttt{FROM}:
		\begin{minted}{SQL}
			SELECT region FROM (SELECT * FROM car WHERE ...)
		\end{minted}
	\item \texttt{WHERE}:
		\begin{minted}{SQL}
			SELECT mark FROM T WHERE mark = (SELECT MAX(mark) FROM T)
		\end{minted}
\end{itemize}

\section{Язык запросов SQL: агрегирование и упорядочение}

Агрегатные функции:
\begin{itemize}
	\item \texttt{COUNT} "--- количество записей в выходном наборе;
	\item \texttt{MIN, MAX} "--- наименьшее/наибольшее из множества значений;
	\item \texttt{AVG} "--- среднее значение;
	\item \texttt{SUM} "--- сумма множества значений.
\end{itemize}

Упорядочение:
\begin{minted}{SQL}
	SELECT name FROM person ORDER BY name [DESC | ASC] [NULLS {FIRST | LAST}]
\end{minted}

\section{Типы данных SQL}

Типы данных могут различаться в зависимости от реализации SQL.
Простейшими типами данных в PostgreSQL можно считать:
\begin{itemize}
	\item целочисленные значения;
	\item вещественные значения;
	\item строки фиксированной длины;
	\item строки переменной длины;
	\item дата и время.
\end{itemize}

Числовые типы: \\
\begin{tabular}{l | c | l}
	\textbf{Имя} & \textbf{Размер, байт} & \textbf{Описание} \\
	\hline
	\texttt{SMALLINT} & 2 & целое в небольшом диапазоне \\
	\texttt{INTEGER} & 4 & целое \\
	\texttt{BIGINT} & 8 & целое в большом диапазоне \\
	\texttt{DECIMAL}, \texttt{NUMERIC} & переменный & вещественное число с указанной точностью \\
	\texttt{REAL} & 4 & вещественное число с переменной точностью \\
	\texttt{DOUBLE PRECISION} & 8 & вещественное число с переменной точностью \\
	\texttt{SMALLSERIAL} & 2 & небольшое целое с автоувеличением \\
	\texttt{SERIAL} & 4 & целое с автоувеличением \\
	\texttt{BIGSERIAL} & 8 & большое целое с автоувеличением
\end{tabular}

\vspace{1em}

Строковые типы: \\
\begin{tabular}{l | l}
	\textbf{Имя} & \textbf{Описание} \\
	\hline
	\texttt{CHARACTER VARYING(n)} & строка ограниченной переменной длины \\
	\texttt{CHARACTER(n)} & строка фиксированной длины, дополненная пробелами \\
	\texttt{TEXT} & строка неограниченной переменной длины
\end{tabular}

\vspace{1em}

Типы даты и времени: \\
\begin{tabular}{l | c | l}
	\textbf{Имя} & \textbf{Размер, байт} & \textbf{Описание} \\
	\hline
	\texttt{TIMESTAMP [(p)] [WITHOUT TIME ZONE]} & 8 & дата и время (без часового пояса) \\
	\texttt{TIMESTAMP [(p)] WITH TIME ZONE} & 8 & дата и время (с часовым поясом) \\
	\texttt{DATE} & 4 & дата \\
	\texttt{TIME [(p)] [WITHOUT TIME ZONE]} & 8 & время суток \\
	\texttt{TIME [(p)] WITH TIME ZONE} & 12 & время суток (с часовым поясом) \\
	\texttt{INTERVAL [поля] [(p)]} & 16 & временной интервал
\end{tabular}

\texttt{p} "--- количество знаков после запятой в поле секунд (от 0 до 6).

Задание пользовательских типов:
\begin{minted}{SQL}
	CREATE TYPE mytype AS VARCHAR(11) NOT NULL
	CREATE TYPE mood AS ENUM ('sad', 'ok', 'happy')
\end{minted}

Явное преобразование типов:
\begin{minted}{SQL}
	SELECT CAST (10.6 AS INT)
	SELECT CONVERT (INT, 10.6)
\end{minted}

\section{Временные таблицы и табличные переменные}

\TODO{Временные таблицы и табличные переменные.}

\section{Представления (обычные, с фиксацией схемы, индексируемые)}

Представления используются:
\begin{itemize}
	\item для упрощения и настройки восприятия информации в БД каждым пользователем;
	\item в качестве механизма безопасности;
	\item для предоставления интерфейса обратной совместимости.
\end{itemize}

Создание представлений:
\begin{minted}{SQL}
	CREATE [OR REPLACE] VIEW citizens AS SELECT client_id, last_name, city FROM client WHERE client = 'Moscow'
\end{minted}

Изменение данных в представлении возможно, если при его создании:
\begin{itemize}
	\item не используется ключевое слово \texttt{DISTINCT};
	\item данные извлекаются только из одной таблицы;
	\item в списке полей нет арифметических выражений;
	\item не используются подзапросы;
	\item не используется группирование;
	\item не используется \texttt{UNION};
\end{itemize}

Создание материализованных представлений:
\begin{minted}{SQL}
	CREATE MATERIALIZED VIEW view_name AS select_stmt [WITH [NO] DATA]
\end{minted}
\texttt{WITH NO DATA} "--- представление создаётся, но не заполняется данными.

Обновление данных:
\begin{minted}{SQL}
	REFRESH MATERIALIZED VIEW view_name
	REFRESH MATERIALIZED VIEW CONCURRENTLY view_name
\end{minted}
Во втором случае необходим уникальный индекс, построенный до обновления.

Создание временных представлений (создаются на действующую сессию):
\begin{minted}{SQL}
	CREATE TEMP VIEW view_name AS select_stmt
\end{minted}

\section{Обобщённые табличные выражения}

\begin{definition}
	\emph{Обобщённое табличное представление} представляет собой временно именованный результирующий набор (CTE), за которым следуют одиночные инструкции \texttt{SELECT, INSERT, UPDATE} или \texttt{DELETE}.
\end{definition}

Структура CTE:
\begin{minted}{SQL}
	WITH expr_name (col_name[, ...]) AS (CTE_query_definition)
\end{minted}

Рекурсивные обобщённые табличные выражения:
\begin{minted}{SQL}
	WITH RECURSIVE t(n) AS (SELECT 1 UNION ALL SELECT n + 1 FROM t WHERE n < 15) SELECT n FROM t
	WITH RECURSIVE t(n) AS (SELECT 1 UNION ALL SELECT n + 1 FROM t) SELECT n FROM t LIMIT 100
\end{minted}

\section{Хранимые процедуры в базе данных}

\begin{definition}
	\emph{Хранимая процедура} "--- это набор операторов процедурного языка и языка SQL, который компилируется системой в единый ``план исполнения''.
\end{definition}

Хранимые процедуры хранятся как объекты в базе.
Они создаются и удаляются операторами \texttt{CREATE} и \texttt{DROP}.

\begin{minted}{SQL}
	CREATE [OR REPLACE] PROCEDURE my_sql_proc (a INT) LANGUAGE SQL AS
		$$ INSERT INTO tbl (field_1) VALUES (a) $$
	CALL my_sql_proc (100)
\end{minted}

\section{Курсоры в базах данных}

\begin{definition}
	\emph{Курсор} "--- это область в памяти базы данных, которая предназначена для хранения запроса SQL.
\end{definition}

В памяти сохраняется и строка данных запроса, называемая \emph{текущим значением}, или \emph{текущей строкой} курсора.
Указанная область в памяти поименована и доступна для прикладных программ.

Действия с курсорами:
\begin{itemize}
	\item \texttt{DECLARE} "--- создание курсора;
	\item \texttt{OPEN} "--- открытие курсора, \ie наполнение его данными;
	\item \texttt{FETCH} "--- выборка из курсора и изменение строк данных с его помощью;
	\item \texttt{CLOSE} "--- закрытие курсора;
	\item \texttt{DEALLOCATE} "--- освобождение курсора.
\end{itemize}

\begin{minted}{SQL}
	DECLARE c_name [[NO] SCROLL] CURSOR [(param)] FOR SELECT ...
	DECLARE c_name REFCURSOR
\end{minted}

В схеме со \emph{статическим курсором} информация читается из БД один раз и хранится в виде моментального снимка.
На время открытия курсора сервер устанавливает блокировку на все строки, включённые его полный результирующий набор.
Статический курсор не изменяется после создания и всегда отображает тот набор данных, который существовал на момент его открытия.

\begin{minted}{SQL}
	DECLARE c_name INSENSITIVE [SCROLL] CURSOR FOR select_stmt
\end{minted}

\emph{Динамические курсоры} отражают все изменения строк в результирующем наборе при прокрутке курсора.
Значения типа данных, порядок и членство строк в результирующем наборе могут меняться для каждой выборки.
Обновления видны сразу, если они сделаны посредством курсора.

\begin{minted}{SQL}
	DECLARE c_name [SCROLL] CURSOR FOR select_stmt
		[FOR {READ_ONLY | UPDATE [OF col[, ...]]}]
\end{minted}

Считать текущую строку в переменные:
\begin{minted}{SQL}
	DECLARE id INT, name VARCHAR;
	FETCH FROM c_1 INTO id, name;
\end{minted}

Прокручиваемый курсор позволяет использовать больше возможностей \texttt{FETCH}:
\begin{minted}{SQL}
	FETCH {NEXT | PRIOR | FIRST | LAST | ABSOLUTE n | RELATIVE n} FROM ...
\end{minted}

\begin{remark}
	Курсор почти всегда использует дополнительные ресурсы сервера, что ведёт к резкому падению производительности.
\end{remark}

\section{Скалярные функции, определяемые пользователем}

\begin{minted}{SQL}
	CREATE [OR REPLACE] FUNCTION name (params) RETURNS res_type {IMMUTABLE | STABLE | VOLATILE} AS
		$$ ... $$ LANGUAGE plpgsql
\end{minted}

Функции:
\begin{itemize}
	\item \texttt{VOLATILE} "--- может изменять данные в базе; может возвращать различные результаты с одинаковыми аргументами;
	\item \texttt{STABLE} "--- не может изменять данные в базе; возвращает одинаковый результат для всех вызовов с одинаковыми аргументами в одном операторе;
	\item \texttt{IMMUTABLE} "--- не может изменять данные в базе; всегда возвращает одинаковые результаты для одних и тех же аргументов.
\end{itemize}

\begin{minted}{SQL}
	CREATE FUNCTION fun_1 (a INT) RETURNS INT AS
		$$ BEGIN
			RETURN a + 1;
		END; $$ LANGUAGE plpgsql;
\end{minted}

\section{Табличные функции, определяемые пользователем}

\begin{minted}{SQL}
	CREATE FUNCTION get_table (a INTEGER)
		RETURNS TABLE (
			new_id INT,
			new_name TEXT
		)
		AS $$ BEGIN
			RETURN QUERY SELECT id, name FROM table_1 WHERE id > a;
		END; $$ LANGUAGE plpgsql;
\end{minted}

\section{Активные базы данных: триггеры}

\begin{definition}
	\emph{Триггер} "--- это откомпилированная SQL-процедура.
	Исполнение обусловлено наступлением определённых событий внутри базы.
\end{definition}

Назначение триггеров:
\begin{itemize}
	\item проверка корректности введённых данных и выполнение сложных ограничений целостности;
	\item накопление аудиторской информации;
	\item автоматическое оповещение других модулей об изменениях информации;
	\item реализация бизнес-правил;
	\item организация каскадных воздействий на таблицы БД;
	\item поддержка репликации.
\end{itemize}

\begin{minted}{SQL}
	CREATE TRIGGER tr BEFORE UPDATE ON table_1 FOR EACH ROW EXECUTE PROCEDURE pt()
\end{minted}

При запуске триггера есть доступ к двум записям \texttt{NEW} и \texttt{OLD}.

Пример триггерной функции:
\begin{minted}{SQL}
	CREATE FUNCTION fun() RETURNS TRIGGER AS
		$$ BEGIN
			IF NEW.id IS NULL THEN
				RAISE EXCEPTION 'id cannot be null';
			END IF;
			RETURN NEW;
		END; $$ LANGUAGE plpgsql;
\end{minted}
