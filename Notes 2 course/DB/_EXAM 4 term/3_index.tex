\part{Индексы}

\section{Кластеризованные таблицы и индексы}

Файлы конфигурации и файлы данных, используемые кластером базы данных, традиционно хранятся вместе в каталоге данных кластера, который обычно называют PGDATA.

Для каждой БД в кластере существует подкаталог внутри PGDATA/base, названный по OID базы в pg\textunderscore database.
Этот подкаталог по-умолчанию является местом хранения файлов базы данных.

Каждая таблица и индекс хранятся в отдельном файле.
Для каждой таблицы и индекса есть карта свободного пространства.
Когда объём таблицы или индекса превышает заданный размер, она делится на сегменты (ноды).

Основной единицей хранения данных является страница.
Место на диске для размещения файла данных в БД логически разделяется на страницы с непрерывной нумерацией.
Дисковые операции ввода-вывода осуществляются на уровне страницы.
СУБД считывает или записывает целые страницы данных.

\section{Первичные и вторичные индексы. Плотные и неплотные индексы}

\section{Индексы: B-дерево}

\section{Индексы: отфильтрованные, покрывающие, составной ключ}

\section{Индексы: полнотекстовые}

\section{Индексы: хеширование}

\section{Индексы: битовые шкалы}

\section{Индексы: R-дерево}
