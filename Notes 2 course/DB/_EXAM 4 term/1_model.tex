\part{Реляционная модель данных}

\section{Основные функции СУБД}

\begin{definition}
	\emph{База данных} "--- это упорядоченный набор структурированной информации или данных которые обычно хранятся в электронном виде в компьютерной системе.
\end{definition}

База данных обычно управляется \emph{системой управления базами данных (СУБД)}.

Функции СУБД:
\begin{itemize}
	\item средства постоянного хранения данных;
	\item поддержка безопасности данных и защита от несанкционированного доступа;
	\item обеспечение согласованности данных;
	\item поддержка высокоуровневых эффективных языков запросов.
\end{itemize}

\section{Модель данных сущность\tpst{"--~}{--}связь: сущности, атрибуты и множества сущностей}

Основные элементы диаграммы ``сущность"--~связь'':
\begin{itemize}
	\item сущность (прямоугольник);
	\item атрибут (овал);
	\item связь (ромб).
\end{itemize}

\begin{definition}
	\emph{Сущность} "--- это абстрактный объект определённого вида.
	Любой предмет или понятие, информацию о котором мы будем хранить.
\end{definition}

У каждой должно быть уникальное имя.
К одному и тому же имени всегда должна применяться одна и та же интерпретация.

\begin{definition}
	Набор экземпляров сущностей образует \emph{множество}.
\end{definition}

Отдельные характеристики объекта называются \emph{атрибутами}.

Требования к атрибутам:
\begin{itemize}
	\item каждый атрибут имеет \emph{имя} и \emph{тип данных};
	\item сущность может обладать любым количеством атрибутов;
	\item значение атрибута атомарно;
	\item сущность и её атрибуты на диаграмме соединяются ненаправленными дугами;
	\item значения атрибутов выбираются из соответствующего множества значений.
\end{itemize}

\begin{definition}
	\emph{Атрибут} "--- именованная характеристика сущности.
	Его наименование должно быть уникальным для конкретного типа сущности, но может быть одинаковым для различного типа сущностей.
	Атрибуты используются определения того, какая информация должна быть собрана о сущности.
\end{definition}

\begin{definition}
	\emph{Ключ} "--- это один или несколько атрибутов объекта, по которому объект можно однозначно идентифицировать.
\end{definition}

\section{Модель данных сущность\tpst{"--~}{--}связь: связи. Использование сущностей и связей при проектировании БД}

\begin{definition}
	\emph{Связь} "--- это отношение между объектами.
\end{definition}

В любой связи выделяются два конца, на каждом из которых указывается:
\begin{itemize}
	\item имя конца связи;
	\item степень конца связи (сколько экземпляров данной сущности связывается).
\end{itemize}

\begin{props}
	\item Связи могут иметь собственные атрибуты.
	\item Подобные связи объединяются в множества.
	\item Связи не могут существовать без связываемых сущностей.
	\item Ключ связи включает ключи связываемых сущностей и, возможно, выделенные атрибуты связи.
\end{props}

Характеристики связи:
\begin{itemize}
	\item Размерность:
		\begin{itemize}
			\item бинарные;
			\item тернарные;
			\item $ n $-арные;
			\item рекурсивные.
		\end{itemize}
	\item Мощность:
		\begin{itemize}
			\item 1:1;
			\item 1:$ N $;
			\item $ M $:$ N $.
		\end{itemize}
	\item Модальность:
		\begin{itemize}
			\item обязательные (``должен'') "--- экземпляр одной сущности \textbf{обязан} быть связан не менее чем с одним экземпляром другой сущности;
			\item необязательные (условные, ``может'') "--- экземпляр одной сущности \textbf{может быть} связан с одним или несколькими экземплярами другой сущности, а может быть и не связан ни с одним экземпляром.
		\end{itemize}
\end{itemize}

Шаги при создании ERD:
\begin{enumerate}
	\item определить сущности;
	\item определить атрибуты сущностей;
	\item определить первичные ключи;
	\item определить отношения между сущностями;
	\item определить мощность связей;
	\item нарисовать ERD;
	\item проверить ERD.
\end{enumerate}

\section{Реляционная модель данных: отношения, таблицы, домены, атрибуты. Описание таблиц в языке SQL}

\begin{definition}
	\emph{Реляционная модель данных} "--- логическая модель данных, описывающая:
	\begin{itemize}
		\item структуры данных в виде наборов отношений;
		\item теоретико-множественные операции над данными: объединение, пересечение, разность и декартово произведение;
		\item специальные реляционные операции: селекция, проекция, соединение и деление;
		\item специальные правила, обеспечивающие целостность данных.
	\end{itemize}
\end{definition}

Структуры данных:
\begin{itemize}
	\item \emph{домены} "--- множества, элементы которых рассматриваются как скалярные значения;
	\item \emph{отношения} "--- предикаты, заданные на прямом произведении доменов;
	\item \emph{атрибуты} "--- заголовок отношения; количество атрибутов "--- \emph{размерность} отношения.
\end{itemize}

Атрибуты: $ A_1, A_2, \dots, A_n $. \\
Домены: $ D_1, D_2, \dots, D_n $. \\
\emph{Кортежи}: $ t = \braket{a_1, a_2, \dots, a_n}, \quad a_i \in D_i $;
\emph{Отношения}: $ R \sub D_1 \times D_2 \times \dots \times D_n $.

\begin{remark}
	Кортежи отличаются друг от друга только значением своих атрибутов.
	В реляционной БД не может быть двух одинаковых кортежей в одной таблице.
\end{remark}

\begin{definition}
	\emph{Таблицы} "--- это единицы хранения данных в базе.
	Таблицы хранят все данные, к которым может обращаться пользователь.
	Реляционную БД можно рассматривать как коллекцию простых таблиц, связанных между собой.
\end{definition}

Таблицы в БД создаются с помощью оператора \texttt{CREATE}:
\begin{minted}[autogobble]{SQL}
	CREATE TABLE Table_Name(
		{{имя столбца} {тип данных} [значение по-умолчанию] [правила целостности]}+)
\end{minted}

\section{Ограничения целостности: уникальность атрибута, нулевые\tpst{\\}{} значения, значения по"=умолчанию}

\begin{definition}
	\emph{Правила целостности} "--- это правила (ограничения), которым должны соответствовать данные.
	Их используют для того, чтобы обеспечить:
	\begin{itemize}
		\item связи между сущностями (\emph{ссылочная целостность});
		\item ограничения по значениям атрибутов в сущностях (\emph{сущностная целостность}).
	\end{itemize}
\end{definition}

Уникальность значения:
\begin{minted}[autogobble]{SQL}
	CREATE TABLE people (id INT UNIQUE NULL)
\end{minted}
Несколько полей (или их комбинаций) могут быть уникальны.
Уникальное значение может быть \texttt{NULL}.

\section{Ограничения целостности: первичный ключ}

\begin{minted}[autogobble]{SQL}
	CREATE TABLE people (id INT PRIMARY KEY)
\end{minted}
\texttt{PRIMARY KEY} \textbf{всегда} \texttt{NOT NULL}.
\texttt{PRIMARY KEY} только один в таблице.

Первичный ключ может состоять из нескольких атрибутов:
\begin{minted}[autogobble]{SQL}
	CREATE TABLE people (name CHAR(20), address VARCHAR(35), PRIMARY KEY(name, address))
\end{minted}

\section{Ограничения целостности: внешний ключ}

\begin{minted}[autogobble]{SQL}
	CREATE TABLE phone (id_p INT REFERENCES people (id))
\end{minted}
или
\begin{minted}[autogobble]{SQL}
	CREATE TABLE phone (id_p INT, FOREIGN KEY (id_p) REFERENCES people (id))
\end{minted}

Требования к внешнему ключу:
\begin{itemize}
	\item поле, на которое ссылается внешний ключ, должно быть \texttt{PRIMARY KEY} или \texttt{UNIQUE};
	\item оба поля должны быть строго одного типа.
\end{itemize}

\section{Ограничения целостности: ограничение на значения атрибута}

\begin{minted}[autogobble]{SQL}
	CREATE TABLE people (id INT PRMARY KEY, gender CHAR CHECK (gender IN ('F', 'M')))
	CREATE TABLE people (id INT PRIMARY KEY, birthday DATE, beg_date DATE,
		CHECK (birthday < beg_date))
	CREATE TABLE people (id INT PRIMARY KEY, gender CHAR,
		CONSTRAINT chk_Person CHECK (gender IN ('F', 'M')))
\end{minted}

\section{Ограничения целостности: вычисляемые атрибуты}

\begin{minted}[autogobble]{SQL}
	CREATE TABLE people (id INT PRIMARY KEY, salary INT, tax AS salary * 0.13)
\end{minted}

Расширение PostgreSQL:
\begin{minted}[autogobble]{SQL}
	CREATE TABLE people (id INT PRIMARY KEY, salary INT,
		tax INT GENERATED ALWAYS AS (salary * 0.13) STORED)
\end{minted}

\section{Реляционная модель данных: алгебраические операции}

% \begin{definition}
% 	\emph{Схема отношения} "--- конечное множество упорядоченных пар вида $ \braket{A, T} $, где $ A $ "--- \emph{имя атрибута}, а $ T $ "--- имя некоторого базового типа или ранее определённого домена.
% \end{definition}
%
% \begin{definition}
% 	Количество атрибутов называется \emph{арностью (размерностью)} отношения.
% \end{definition}

Все операции \emph{реляционной алгебры} производятся над отношением, и результатом операции является отношение.

Реляционная алгебра является \emph{замкнутой}: в качестве аргументов в реляционные операторы можно подставлять другие реляционные операторы, подходящие по типу.
В реляционных выражениях можно использовать вложенные выражения сколь угодно вложенной структуры.

Каждое отношение обязано иметь уникальное имя в пределах БД.
Если отношения подставляются в качестве аргументов в другие реляционные выражения, то они могут быть неименованными.

Операции реляционной алгебры делятся на два типа:
\begin{itemize}
	\item теоретико-множественные операции;
	\item специальные реляционные операции.
\end{itemize}

Теоретико-множественные операции:
\begin{itemize}
	\item объединение отношений;
	\item пересечение отношений;
	\item разность отношений;
	\item декартово произведение отношений.
\end{itemize}

Два отношения \emph{совместимы по взятию декартова произведения} в том и только в том случае, если пересечение имён атрибутов, взятых из схем отношений, пусто.
Любые два отношения могут стать совместимыми по взятию декартова произведения, если применить к одному из них операцию переименования.

Свойства операций:
\begin{itemize}
	\item ассоциативность (кроме разности);
	\item коммутативность (кроме разности).
\end{itemize}

\section{Реляционная модель данных: реляционные операции}
\label{sec:model:operations}

Специальные реляционные операции:
\begin{itemize}
	\item \emph{ограничение отношения (селекция)} "--- горизонтальная вырезка;
	\item \emph{проекция отношения} "--- вертикальная вырезка;
	\item \emph{соединение отношений} (по условию, эквисоединение, естественное соединение);
	\item \emph{деление отношений}.
\end{itemize}

\subsection*{Селекция}

\begin{minted}[autogobble]{SQL}
	R WHERE f
\end{minted}

$$ \sigma_f(R) $$

Условие ограничения имеет вид:
\begin{itemize}
	\item ($ a $ \texttt{операция\textunderscore сравнения} $ b $), где $ a $ и $ b $ "--- имена атрибутов ограничиваемого отношения; атрибуты $ a $ и $ b $ определены на одном домене, для значений которого поддерживается операция сравнения;
	\item ($ a $ \texttt{операция\textunderscore сравнения} $ \const $), где $ a $ "--- имя атрибута ограничиваемого отношения; атрибут $ a $ должен быть определён на домене или базовом типе, для значений которого поддерживается операция сравнения.
\end{itemize}

Условие может состоять из нескольких простых логических выражений, связанных булевскими операторами \texttt{AND, NOT, OR}.

Результатом селекции является отношение, заголовок которого совпадает с заголовком отношения-операнда, а в тело входят те кортежи отношения-операнда, для которых условие ограничения выполнено.

\subsection*{Проекция}

\begin{definition}
	\emph{Проекцией} отношения $ R $ по атрибутам $ X, Y, \dots, Z $ где каждый из атрибутов принадлежит отношению, называется отношение с заголовком $ (X, Y, \dots, Z) $ и телом, содержащим множество кортежей вида $ (x, y, \dots, z) $ таких, для которых найдутся кортежи со значением атрибута $ X $ равным $ x $, \dots, значением атрибута $ Z $ равным $ z $.
\end{definition}

$$ \pi_{(X, Y, \dots, Z)}(R) = \set{x, y, \dots, z : \quad \exists a_1, a_2, \dots, a_n \in R \amp x = a_{i1}, ~ y = a_{i2}, \dots, z = a_{im}} $$

\subsection*{Соединение по условию (\tpst{\theta}{тета}-соединение)}

Три операнда: соединяемые отношения и условие.
Операнды должны быть совместимы по взятию декартова произведения.

\begin{minted}{SQL}
	A JOIN B WHERE f = (A x B) WHERE f
\end{minted}

$$ R \vartriangleright\vartriangleleft_f S = \sigma_f (R \times S) $$

\subsection*{Эквисоединение}

\begin{definition}
	Операция соединения называется операцией \emph{эквисоединения} (\texttt{EQUI JOIN}), если условие соединения имеет вид $ (a = b) $.
\end{definition}

Если соединение происходит по атрибутам с одинаковыми именами, то в результирующем отношении появляется два атрибута с одинаковыми значениями.
В таком случае нужно брать проекцию по всем атрибутам, кроме одного из дублирующих.

\subsection*{Естественное соединение}

Операция \emph{естественного соединения} применяется к паре отношений $ R(A, X) $ и $ S(X, B) $, обладающих общим атрибутом $ X $.

$$ T(A, X, B), \quad A \vartriangleright\vartriangleleft B $$
В синтаксисе естественного соединения не указывается, по каким атрибутам производится соединение.
Естественное соединение производится \textbf{по всем} одинаковым атрибутам.

$$ R \vartriangleright\vartriangleleft S = \pi_{\text{атрибуты } R, ~ S \setminus S.A} \sigma_{R.A = S.A}(R \times S) $$

\subsection*{Деление}

\begin{definition}
	Результатом \emph{деления} $ A $ на $ B $ является ``унарное'' отношение $ C(a) $, тело которого состоит из кортежей $ v $ таких, что в теле отношения $ A $ содержатся кортежи $ \braket{v, w} $ для любого $ w $ из $ B $.
\end{definition}

$$ (A \texttt{ DIVIDE BY } B) = C : \quad C \times B \sub A $$

$$ R \texttt{ DIVIDE BY } S = \pi_{1, 2, \dots, r - s}(R) - \pi_{1, 2, \dots, r - s} \bigl( \pi_{1, 2, \dots, r - s}(R \times S) \setminus R \bigr) $$

\begin{props}
\item Коммутативность
	\begin{itemize}
		\item для декартова произведения:
			$$ R_1 \times R_2 = R_2 \times R_1 $$
		\item для соединений:
			$$ R_1 \vartriangleright\vartriangleleft_f R_2 = R_2 \vartriangleright\vartriangleleft_f R_1 $$
	\end{itemize}
\item Ассоциативность:
	\begin{itemize}
		\item для декартова произведения:
			$$ (R_1 \times R_2) \times R_3 = R_1 \times (R_2 \times R_3) $$
		\item для соединений:
			$$ (R_1 \vartriangleright\vartriangleleft_{f_1} R_2) \vartriangleright\vartriangleleft_{f_2} R_3 = R_1 \vartriangleright\vartriangleleft_{f_1} (R_2 \vartriangleright\vartriangleleft_{f_2} R_3) $$
	\end{itemize}
\item Комбинация (каскад):
	\begin{itemize}
		\item для селекций:
			$$ \sigma_{f_1} \bigl( \sigma_{f_2}(R) \bigr) = \sigma_{f_1 \vee f_2}(R) $$
		\item для проекций:
			$$ \pi_{A_1, A_2, \dots, A_m} \bigl( \pi_{B_1, B_2, \dots, B_n}(R) \bigr) = \pi_{A_1, A_2, \dots, A_m}(R), \quad \text{ где } \set{A_m} \sub \set{B_n} $$
	\end{itemize}
\item Перестановка
	\begin{itemize}
		\item селекции и проекции:
			$$ \sigma_f \pi_{A_1, A_2, \dots, A_m}(R) = \pi_{A_1, A_2, \dots, A_m} \sigma_f (R) $$
		\item селекции и объединения:
			$$ \sigma_f(R_1 \cup R_2) = \sigma_f(R_1) \cup \sigma_f(R_2) $$
		\item селекции и декартова произведения:
			$$ \sigma_f(R_1 \times R_2) = \bigl( \sigma_{f_1}(R_1) \bigr) \times \bigl( \sigma_{f_2}(R_2) \bigr) $$
		\item селекции и разности:
			$$ \sigma_f(R_1 \setminus R_2) = \sigma_f(R_1) \setminus \sigma_f(R_2) $$
		\item селекции и пересечения:
			$$ \sigma_f(R_1 \cap R_2) = \sigma_f(R_1) \cap \sigma_f(R_2) $$
		\item проекции и декартова произведения:
			$$ \pi_{A_1, A_2, \dots, A_m}(R_1 \times R_2) = \bigl( \pi_{B_1, B_2, \dots, B_n}(R_1) \bigr) \times \bigl( \pi_{C_1, C_2, \dots, C_r}(R_2) \bigr) $$
	\end{itemize}
\end{props}

\section{Отображение модели сущность\tpst{"--~}{--}связь в реляционную. Представление объектов}

Каждая сущность превращается в таблицу.
Имя сущности становится именем таблицы.

Каждый атрибут становится столбцом.
Столбцы для необязательных атрибутов могут содержать не"-определённые значения, столбцы для обязательных "--- не могут.

Компоненты уникального идентификатора сущности превращаются в ключ таблицы.

\section{Отображение модели сущность\tpst{"--~}{--}связь в реляционную. Представление связей}

Связи также хранятся в отношении.
Схема данного отношения составляется из ключевых атрибутов, участвующих в связи.

\section{Функциональные зависимости и аномалии вставки, обновления, удаления}

\begin{definition}[функциональная зависимость]
	$ R \set{A_1, A_2, \dots, A_n}, \quad X, Y \sub \set{A_1, A_2, \dots, A_n} $

	$ X \to Y $, если любому значению $ X $ соответствует ровно одно значение $ Y $.

	$ X $ называется \emph{детерминантом}, $ Y $ "--- \emph{зависимой частью}.
\end{definition}

$$ X \to Y \iff \left| \pi_Y \bigl( \sigma_{X = x}(R) \bigr) \right| \le 1 $$

Пусть $ \set{A_1, A_2, \dots, A_n} \to \set{B_1, B_, \dots, B_m} $.
Функциональные зависимости:
\begin{itemize}
	\item тривиальные:
		$$ \set{B_1, B_2, \dots, B_m} \sub \set{A_1, A_2, \dots, A_n} $$
	\item нетривиальные:
		$$ \set{B_1, B_2, \dots, B_m} \not\sub \set{A_1, A_2, \dots, A_n}, \quad \set{A_1, A_2, \dots, A_n} \cap \set{B_1, B_2, \dots, B_m} \ne \O $$
	\item полностью нетривиальные:
		$$ \set{A_1, A_2, \dots, A_n} \cap \set{B_1, B_2, \dots, B_m} = \O $$
\end{itemize}

Аксиомы Армстронга:
\begin{itemize}
	\item рефлексивность:
		$$ B \sub A \implies A \to B $$
	\item пополнение:
		$$ A \to B \implies AC \to BC $$
	\item транзитивность:
		$$ A \to B, ~ B \to C \implies A \to C $$
\end{itemize}

Из аксиом Армстронга можно получить правила вывода:
\begin{itemize}
	\item объединение:
		$$ X \to Y, ~ X \to Z \implies X \to YZ $$
	\item псевдотранзитивность:
		$$ X \to Y, ~ WY \to Z \implies WX \to Z $$
	\item декомпозиция:
		$$ X \to Y, ~ Z \subseteq Y \implies X \to Z $$
\end{itemize}

\section{Нормализация: декомпозиция отношений. Нормальные формы}

\begin{definition}
	\emph{Ключ} "--- минимальный набор атрибутов, который функционально определяет все остальные.
\end{definition}

\begin{definition}
	$ Y $ \emph{полностью функционально зависит} от $ X $, если $ Y $ функционально зависит от всех атрибутов, входящих в состав $ X $, а не от какой-то его части.
\end{definition}

\begin{definition}
	Функциональная зависимость $ A \to C $ называется \emph{транзитивной}, если существует такой атрибут $ B $, что $ A \to B $ и $ B \to C $ и отсутствует функциональная зависимость $ C \to A $.
\end{definition}

\begin{definition}
	\emph{Декомпозиция} "--- это разбиение на множества, может быть пересекающиеся, такие, что их объединение "--- это исходное отношение.
\end{definition}

Восстановить исходное отношение можно только естественным соединением.

Нормальные формы:
\begin{itemize}
	\item Первая: \\
		Значение каждого атрибута в таблице должно быть атомарно.
	\item Вторая: \\
		Таблица находится в 2НФ, если она находится в 1НФ, и каждый атрибут полностью зависит от любого его ключа, но не от подмножества ключа.
	\item Третья: \\
		Отношение находится в 2НФ, и любой атрибут, не являющийся первичным, нетранзитивно зависит от любого возможного ключа.
	\item Бойса"--~Кодда: \\
		Если $ X \to A, ~ A \not\in X $, то $ X \supseteq $ ключ $ R $.
\end{itemize}

\section{Нормализация: многозначные зависимости}

\begin{definition}
	Пусть $ A $ и $ B $ "--- два атрибута отношения $ R $.

	Между этими атрибутами существует \emph{многозначная} зависимость, если значению $ a $ атрибута $ A $ соответствует множество значений $ \set{b_1, b_2, \dots, b_k} $ атрибута $ B $.
\end{definition}

\begin{notation}
	$ A \twoheadrightarrow B $
\end{notation}

\begin{lemma}[Фейджина]
	В отношении $ R \set{A, B, C} $ выполняется MVD $ A \twoheadrightarrow B $ в том и только в том случае, когда выполняется MVD $ A \twoheadrightarrow C $.
\end{lemma}

Чтобы подчеркнуть этот факт, иногда пишут $ A \twoheadrightarrow B|C $.

\begin{theorem}[Фейджина]
	$ R \set{A, B, C} $

	Переменная-отношение $ R $ будет равна соединению её проекций $ [A, B] $ и $ [A, C] $ тогда и только тогда, когда для переменной-отношения $ R $ выполняется многозначная зависимость $ A \twoheadrightarrow B | C $.
\end{theorem}

Четвёртая нормальная форма: \\
Отношение $ R $ находится в 4НФ, если оно находится в НФБК, и детерминант любой нетривиальной MVD является ключом $ R $.
