\part{Оптимизация запросов}

\section{Выполнение запросов: реализация операций реляционной алгебры}

Операндами операций реляционной алгебры являются отношения.
Для выполнения операций необходимо просмотреть все кортежи исходных отношений.
Следствием этого является большая размерность операций РА.
Уменьшения размерности можно достичь, изменяя последовательность выполняемых операций.

Оптимизация выполнения запросов реляционной алгебры основана на понятии \emph{эквивалентности} реляционных выражений (см. вопрос \ref{sec:model:operations}).

Объединение выполняется путём сортировки данных для удаления одинаковых кортежей.

Виды соединения таблиц:
\begin{itemize}
	\item Соединение с помощью вложенных циклов:
		\begin{itemize}
			\item сложность "--- $ O(N \log N) $;
			\item используется, если хотя бы одна таблица достаточно маленькая, а б\'ольшая таблица имеет индекс по ключу соединения.
		\end{itemize}
	\item Соединение слиянием:
		\begin{itemize}
			\item сложность "--- $ O(N + M) $;
			\item используется, если обе таблицы отсортированы по столбцу слияния, а слияние происходит по равенству;
			\item хорошо подходит для больших таблиц.
		\end{itemize}
	\item Хеш-соединение:
		\begin{itemize}
			\item сложность "--- $ O(N \cdot h_c + M \cdot h_m + J) $;
			\item использует таблицу хеширования и динамическую хеш-функцию для строк;
			\item используется только в крайнем случае.
		\end{itemize}
\end{itemize}

\section{Задача оптимизации. Компоненты и функции оптимизатора запросов}

\begin{problem}
	По декларативной формулировке запроса требуется построить программу (\emph{план выполнения запроса}), которая выполнялась бы максимально эффективно и выдавала бы результаты, соответствующие указанным в запросе свойствам.
\end{problem}

Таким образом, задача заключается в том, чтобы построить все возможные программы, дающие требуемый результат и выбрать из них такую, выполнение которой было бы наиболее эффективным.

Работа оптимизатора состоит из пяти стадий:
\begin{enumerate}
	\item лексический и синтаксический анализ;
	\item логическая оптимизация;
	\item выбор альтернативных процедурных планов выполнения на основе информации, которой располагает оптимизатор;
	\item по внутреннему представлению наиболее оптимального плана выполнения формируется процедурное представление;
	\item выполнение запроса в соответствии с планом.
\end{enumerate}

Два основных вида оптимизаторов:
\begin{itemize}
	\item rule-based:
		\begin{itemize}
			\item выбирает методы доступа на основе предположения о статичности СУБД;
			\item учитывает иерархическое старшинство операций;
			\item если для операции есть несколько путей выполнения, то выбирается путь с наименьшим (заранее заданным) рангом.
		\end{itemize}
	\item cost-based;
\end{itemize}

\section{Оптимизация запросов по стоимости. План выполнения запроса}

\begin{definition}
	\emph{Стоимость (затраты)} "--- это оценка ожидаемого времени выполнения запроса с использованием конкретного плана выполнения.
\end{definition}

Для каждого из выбранных планов оценивается предполагаемая стоимость выполнения запроса по этому плану.
При оценках используется либо доступная оптимизатору статистическая информация о распределении данных, либо информация о механизмах реализации путей доступа.

Оптимизация выполнения запроса осуществляется в следующем порядке:
\begin{itemize}
	\item вычисление выражений и условий, содержащих константы;
	\item преобразование сложной команды в эквивалентную ей с использованием соединения;
	\item если команда выполняется над представлением, то оптимизатор обычно объединяет запрос на создание представления и запрос к этому представлению;
	\item выбор метода оптимизации;
	\item выбор путей доступа к таблицам, к которым обращается запрос;
	\item выбор порядка соединения;
	\item выбор операции соединения для каждой команды соединения.
\end{itemize}
