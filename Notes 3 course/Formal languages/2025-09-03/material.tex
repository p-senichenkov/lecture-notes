\chapter{Введение}

Лекции будут по учебнику: Мартыненко. Языки и трансляции (2 издание).

Грамматики:
\begin{enumerate}
	\item контекстно-свободные;
	\item контекстно-зависимые;
	\item регулярные.
\end{enumerate}

\chapter{Языки и их представление}

\section{Основные понятия}

\begin{definition}
	\emph{Алфавит} "--- конечное множество символов.
\end{definition}

\begin{note}
	\emph{Символ} никак не определяется.
\end{note}

\begin{definition}
	\emph{Предложение (строка, слово)} "--- цепочка конечной длины, составленная из символов некоторого алфавита.
\end{definition}

\begin{notation}
	$ \eps $ "--- пустое слово.

	Если $ V $ "--- некоторый алфавит, то $ V^* $ "--- множество всех слов в алфавите $ V $ (включая $ \eps $). \\
	$ V^+ = V^* \setminus \set\eps $
\end{notation}

\begin{statement}
	Множество цепочек над любым алфавитом счётно.
\end{statement}

\begin{definition}
	\emph{Язык} "--- множество предложений над некоторым алфавитом.
\end{definition}

\begin{statement}
	\textbf{Не для каждого} языка существует его конечное представление.
\end{statement}

Языки представляются с помощью:
\begin{itemize}
	\item \emph{распознающего алгоритма} (определяет, принадлежит ли данное предложение языку);
	\item \emph{распознающей процедуры} (то же самое, но может не завершаться на предложениях не из языка);
	\item \emph{порождающей процедуры} (порождает предложения языка в некотором порядке).
\end{itemize}

\subsection{Построение порождающей процедуры по распознающей процедуре}

Пусть $ P $ "--- распознающая процедура для языка $ L $.

Можно пронумеровать пары натуральных чисел:
$$ k = \frac{(i + j - 1)(i + j - 2)}2 + j $$

По пронумерованным парам строится порождающая процедура.

\begin{definition}
	Язык называется \emph{рекурсивно перечислимым}, если существует процедура, которая порождает или распознаёт этот язык.
\end{definition}

\begin{definition}
	Язык называется \emph{рекурсивным}, если существует алгоритм его распознавания.
\end{definition}

\begin{theorem}
	Пусть $ L \subseteq V^* $ "--- некоторый язык, а $ \ol L = V^* \setminus L $ "--- его дополнение.

	Если языки $ L $ и $ \ol L $ оба рекурсивно перечислимы, то язык $ L $ рекурсивен.
\end{theorem}

\begin{proof}
	Пусть язык $ L $ распознаётся процедурой $ P $, а $ \ol L $ "--- $ \ol P $.
	Достаточно показать, как построить алгоритм распознавания $ L $.
	Построим его:
	\begin{enumerate}
		\item $ i \define 1 $
		\item Применим $ i $ шагов прооцедуры $ P $ к цепочке $ x $.
			Если она завершилась, переходим к шагу 3, иначе "--- к шагу 4.
		\item Применим $ i $ шагов процедуры $ \ol P $ к цепочке $ x $.
		\item При некотором $ i $ одна из цепочек завершится.
\end{proof}
