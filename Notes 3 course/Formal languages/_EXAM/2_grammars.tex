\part{Грамматики}

\section*{Формальное определение грамматики.
Типы грамматик.
Пустое предложение}

\begin{definition}
	\emph{Грамматикой} называется четвёрка $ G = (V_N, V_T, P, S) $, где
	\begin{itemize}
		\item $ V_N $, $ V_T $ "---
			\emph{алфавиты нетерминалов} и \emph{терминалов} соответственно, причём
			$ V_N \cap V_T = \O $;
		\item $ P $ "--- конечное \emph{множество правил}, каждое из которых имеет вид
			$ \alpha \to \beta $, где
			\begin{itemize}
				\item $ \alpha \in V^*V_NV^* $;
				\item $ \beta \in V^* $;
			\end{itemize}
		\item $ V = V_N \cup V_T $ "--- \emph{объединённый алфавит} грамматики;
		\item $ S $ "--- \emph{начальный нетерминал}.
	\end{itemize}
\end{definition}

\begin{definition}
	$ \alpha \to \beta \in P $ "--- правило, $ \quad \gamma, \delta $ "--- цепочки из множества
	$ V^* $

	Будем говорить, что из цепочки $ \gamma \alpha \delta $ \emph{непосредственно выводится
	цепочка} $ \gamma \beta \delta $ в грамматике $ G $ при помощи данного правила.
\end{definition}

\begin{notation}
	$ \gamma \alpha \delta \underimp G \gamma \beta \delta $
\end{notation}

\begin{definition}
	$ \alpha_1, \alpha_2, \dots, \alpha_m $ "--- цепочки из множества $ V^* $,
	$$ \alpha_1 \underimp G \alpha_2, \quad \alpha_2 \underimp G \alpha_3, \quad \dots, \quad
	\alpha_{m - 1} \underimp G \alpha_m $$

	Тогда говорим, что из $ \alpha_1 $ \emph{выводится} $ \alpha_m $ в грамматике $ G $.
\end{definition}

\begin{notation}
	$ \alpha_1 \xRightarrow[G]* \alpha_m $
\end{notation}

\begin{definition}
	\emph{Язык, порождаемый} грамматикой $ G $ определим как
	$$ L(G) = \Set{w | w \in V_T^*, \quad S \xRightarrow[G]* w} $$
\end{definition}

\begin{definition}
	Любая цепочка $ \alpha \in V^* $ такая, что $ S \xRightarrow[G]* \alpha $ называется
	\emph{сентенциальной формой}.
\end{definition}

\begin{definition}
	Грамматики $ G_1 $ и $ G_2 $ называются \emph{эквивалентными}, если $ L(G_1) = L(G_2) $.
\end{definition}

\subsection*{Типы грамматик}

\begin{definition}
	Грамматику, на которую не наложено никаких ограничений, назовём \emph{грамматикой типа 0}.
\end{definition}

\begin{definition}
	Грамматика $ G = (V_N, V_T, P, S) $ является \emph{грамматикой типа 1} или \emph{контекстно-
	зависимой грамматикой}, если для каждого её правила $ \alpha \to \beta \in P $ выполняется
	$ |\beta| \ge |\alpha| $.
\end{definition}

\begin{definition}
	Грамматика $ G = (V_N, V_T, P, S) $ является \emph{грамматикой типа 2} или \emph{контекстно-
	свободной} грамматикой, если каждое её правило имеет вид $ A \to \beta \in P $, где $ A \in
	V_N $, $ \beta \in V^+ $.
\end{definition}

\begin{definition}
	Грамматика $ G = (V_N, V_T, P, S) $ является \emph{грамматикой типа 3} или \emph{регулярной
	грамматикой}, если каждое её правило имеет вид $ A \to aB $ или $ A \to a $, где $ a \in V_T, ~
	A, B \in V_N $.
\end{definition}

\subsection*{Пустое предложение}

Расширим данные ранее определения типов грамматик, допустив порождение пустого предложения правилом
$ S \to \eps $, где $ S $ "--- начальный символ, при условии, что $ S $ не появляется в правой части
никакого правила.

\begin{lemma}
	$ G = (V_N, V_T, P, S) $ "--- грамматика типа $ \ge 1 $

	Тогда существует другая грамматика $ G_1 $ такого же типа, которая порождает тот же самый язык,
	и в которой ни одно правило не содержит начальный символ в своей правой части.
\end{lemma}

\begin{theorem}
	$ L $ "--- язык типа $ \ge 1 $

	Языки $ L \cup \Set{\eps} $ и $ L \setminus \Set{\eps} $ имеют тот же тип.
\end{theorem}

\section{Рекурсивность контекстно-зависимых грамматик}

\begin{definition}
	Грамматика $ G = (V_N, V_T, P, S) $ \emph{рекурсивна}, если существует алгоритм, который
	определяет, порождается ли любая данная цепочка $ x \in V_T^* $ грамматикой $ G $.
\end{definition}

\begin{theorem}
	Если грамматика контекстно-зависима, то она рекурсивна.
\end{theorem}

\section{Деревья вывода в контекстно-свободных грамматиках.
Теорема о деревьях вывода}

\begin{definition}
	$ G = (V_N, V_T, P, S) $ "--- cfg

	Дерево является \emph{деревом вывода} в грамматике $ G $, если оно удовлетворяет четырём
	условиям:
	\begin{enumerate}
		\item каждый узел имеет метку "--- символ из алфавита $ V $;
		\item метка корня "--- $ S $;
		\item если узел имеет по крайней мере одного потомка, то его метка должна быть нетерминалом;
		\item если узлы $ n1, n_2, \dots, n_k $ "--- прямые потомки узла $ n $, перечисленные слева-
			направо, с метками $ A_1, A_2, \dots, A_k $ соответственно, а метка узла $ n $ есть
			$ A $, то $ A \to A_1 A_2 \dots A_k \in P $.
	\end{enumerate}
\end{definition}

\begin{theorem}
	$ G = (V_N, V_T, P, S) $ "--- cfg, $ \quad \alpha \in V^*, \quad \alpha \ne \eps $

	Вывод $ S \xRightarrow[G]* \alpha $ существует \textbf{тогда и только тогда}, когда существует
	дерево вывода в грамматике $ G $ с результатом $ \alpha $.
\end{theorem}
