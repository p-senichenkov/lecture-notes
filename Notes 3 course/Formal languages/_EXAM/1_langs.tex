\part{Языки и их представление}

\section{Алфавиты и языки.
Представление языков с помощью распознающих и порождающих процедур.
Теорема о рекурсивности языка}

\begin{definition}
	\emph{Алфавит} или \emph{словарь} есть конечное множество символов.
\end{definition}

\begin{definition}
	\emph{Предложение} (\emph{строка}, \emph{слово}) есть любая цепочка конечной длины,
	составленная из символов некоторого алфавита.
\end{definition}

\begin{definition}
	Предложение, не содержащее ни одного символа, называется \emph{пустым предложением}.
\end{definition}

\begin{notation}
	$ \eps $
\end{notation}

\begin{statement}
	Множество цепочек над алфавитом счётно бесконечно.
\end{statement}

\begin{definition}
	\emph{Язык} есть любое множество предложений над некоторым алфавитом.
\end{definition}

\begin{definition}
	\emph{Распознающий алгоритм} определяет, есть ли данное предложение в данном языке.

	\emph{Распознающая процедура}:
	\begin{itemize}
		\item для предложений в языке прекращает работу с ответом `да';
		\item для предложений не из языка завершается с ответом `нет', или не завершается вовсе.
	\end{itemize}
\end{definition}

\begin{definition}
	\emph{Порождающая процедура} систематически порождает предложения языка последовательно
	в некотором порядке.
\end{definition}

\begin{definition}
	Язык называется \emph{рекурсивно перечислимым}, если существует процедура, которая порождает
	или распознаёт этот язык.
\end{definition}

\begin{definition}
	Язык \emph{рекурсивен}, если существует алгоритм его распознавания.
\end{definition}

\begin{theorem}
	$ L \subseteq V^* $ "--- язык, $ \quad \ol L = V^* \setminus L $

	Если языки $ L $ и $ \ol L $ оба рекурсивно перечислимы, то язык $ L $ рекурсивен.
\end{theorem}
