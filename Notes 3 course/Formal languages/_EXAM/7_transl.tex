\part{Трансляции, их представление и реализация}

\setcounter{section}{0}

\section{Некоторые способы задания трансляций: перечисление, гомоморфизм, схемы
синтаксически"=управляемых трансляций (SDTS), конечные и магазинные преобразователи}

\begin{definition}
	\emph{Трансляцией} из языка $ L_1 \sub \Sigma^* $ называется отношение $ \tau \sub L_1 \times
	L_2 $.

	Здесь $ \Sigma $ "--- входной алфавит, $ L_1 $ "--- входной язык, $ \Delta $ "--- выходной
	алфавит, $ L_2 $ "--- выходной язык.
\end{definition}

По отношению к языкам программирования, трансляция всегда является функцией.

\subsection*{Гомоморфизм}

Гомоморфизм $ h $ определяет трансляцию
$$ \tau(h) = \Set{ \bigl( x, h(x) \bigr) | x \in \Sigma^*} $$

Устройство, которое по заданной цепочке $ x \in \Sigma^* $ находит соответствующую цепочку $ y =
h(x) $, должно посимвольно просмотреть входную цепочку $ x $ и заменить каждый её символ $ a $ на
$ h(a) $.

\subsection*{Схемы синтаксически-управляемой трансляции}

\begin{definition}
	\emph{Схемой синтаксически-управляемой трансляции} называется формальная система $ T = (N,
	\Sigma, \Delta, R, S) $, где
	\begin{itemize}
		\item $ N $ "--- алфавит \emph{нетерминалов};
		\item $ \Sigma $\ "--- конечный \emph{входной алфавит};
		\item $ \Delta $ "--- конечный \emph{выходной алфавит}, причём $ N \cap \Sigma = \O $ и
			$ N \cap \Delta = \O $;
		\item $ R $ "--- конечное множество \emph{правил} вида $ A \to \alpha, \beta $, где
			\begin{itemize}
				\item $ A \in N $;
				\item $ \alpha \in (N \cup \Sigma)^* $;
				\item $ \beta \in (N \cup \Delta)^* $;
				\item каждое вхождение нетерминала в цепочку $ \alpha $ связано с некоторым
					вхождением одноимённого нетерминала в $ \beta $, и эта связь является
					неотъемлемой частью правила;
			\end{itemize}
		\item $ S \in N $ "--- \emph{начальный нетерминал}.
	\end{itemize}

	Цепочка $ \alpha $ называется \emph{синтаксической}, а $ \beta $ "--- \emph{семантической}.
\end{definition}

\begin{definition}
	Введём понятие \emph{трансляционной формы}:
	\begin{itemize}
		\item $ (S, S) $ "--- \emph{начальная трансляционная форма}, причём эти два вхождения
			начального нетерминала связаны друг с другом по определению.
		\item Если $ (\alpha A \beta, \alpha' A \beta') $ "--- трансляционная форма, в которой два
			явно выделенных вхождения нетерминала $ A $ связаны, и если $ A \to \gamma,
			\gamma' $ "--- правило из $ R $, то $ (\alpha \gamma \beta, \alpha' \gamma'
			\beta') $ "--- трансляционная форма.
			Связь между нетерминалами в $ \gamma $ и $ \gamma' $ такая же, как в правиле.
			Нетерминалы в цепочках $ \alpha $ и $ \beta $ связываются с нетерминалами в цепочках
			$ \alpha' $ и $ \beta' $ в новой трансляционной форме так же, как в предыдущей.
		\item Никакие другие пары цепочек не являются трансляционными формами.
	\end{itemize}
\end{definition}

\begin{notation}
	Отношение \emph{непосредственной выводимости}: $ (\alpha A \beta, \alpha' A \beta') \underimp T
	(\alpha \gamma \beta, \alpha' \gamma' \beta') $
\end{notation}

\begin{definition}
	Трансляция, заданная при помощи схемы синтаксически-управляемой трансляции $ T $ есть множество
	$$ \tau(T) = \Set{(x, y) | (S, S) \xRightarrow[T]* (x, y), \quad x \in \Sigma^*, ~ y \in
	\Delta^*} $$
	и называется \emph{синтаксически-управляемой трансляцией}.
\end{definition}

\begin{definition}
	Грамматика $ G_i = (N, \Sigma, P_i, S) $, где
	$$ P_i = \Set{A \to \alpha | \exists A \to \alpha, \beta \in R}, $$
	называется \emph{входной грамматикой} схемы.

	Грамматика $ G_0 = (N, \Delta, P_0, S) $, где
	$$ P_0 = \Set{A \to \beta | \exists A \to \alpha, \beta \in R} $$
	называется \emph{выходной грамматикой} схемы.
\end{definition}

\section{Простые SDTS.
Эквивалентность классов трансляций, задаваемых простыми SDTS и недетерминированными магазинными
преобразователями}

\begin{definition}
	Схема синтаксически-управляемой трансляции называется \emph{простой}, если в каждом её правиле
	$ A \to \alpha, \beta $ связанные нетерминалы в цепочках $ \alpha $ и $ \beta $ встречаются в
	одинаковом порядке.
\end{definition}

\begin{definition}
	\emph{Недетерминированный магазинный преобразователь} "--- это формальная система $ P = (Q,
	\Sigma, \Gamma, \Delta, \delta, q_0, Z_0, F) $, где
	\begin{itemize}
		\item $ Q $ "--- конечное \emph{множество состояний};
		\item $ \Sigma $ "--- конечный \emph{входной алфавит};
		\item $ \Gamma $ "--- конечный \emph{алфавит магазинных символов};
		\item $ \Delta $ "--- конечный \emph{выходной алфавит};
		\item $ q_0 \in Q $ "--- \emph{начальное состояние};
		\item $ Z_0 \in \Gamma $ "--- \emph{начальный символ магазина};
		\item $ F \sub Q $ "--- \emph{множество конечных состояний};
		\item $ \delta $ "--- отображение $ Q \times (\Sigma \cup \Set{\eps}) \times \Gamma \to
			2^{Q \times \Gamma^* \times \Delta^*} $.
	\end{itemize}
\end{definition}

\begin{definition}
	\emph{Конфигурацией} магазинного преобразователя $ P $ назовём четвёрку $ (q, x, \alpha, y) $,
	где
	\begin{itemize}
		\item $ Q $ "--- \emph{текущее состояние};
		\item $ x \in \Sigma^* $ "--- \emph{непросмотренная часть} входной цепочки;
		\item $ \alpha \in \Gamma^* $ "--- \emph{содержимое магазина};
		\item $ y \in \Delta^* $ "--- вся \emph{выходная цепочка}.
	\end{itemize}
\end{definition}

\begin{definition}
	Говорят, что $ y \in \Delta^* $ "--- \emph{выход} для $ x \in \Sigma^* $ \emph{при конечном
	состоянии}, если
	$ (q_0, x, Z_0, \eps) \vdash^* (q, \eps, \alpha, y) $ для некоторых $ q \in F $ и $ \alpha \in
	\Gamma^* $.

	\emph{Трансляция}, определяемая магазинным преобразователем $ P $ \emph{при конечном состоянии},
	есть
	$$ \tau(P) = \Set{(x, y) | (q_0, x, Z_0, \eps) \vdash^* (q, \eps, \alpha, y), \quad q \in F,
	~ \alpha \in \Gamma^*} $$
\end{definition}

\begin{definition}
	Говорят, что $ y \in \Delta^* $ есть \emph{выход} для $ x \in \Sigma^* $ \emph{при пустом
	магазине}, если $ (q_0, x, Z_0, \eps) \vdash^* (q, \eps, \eps, y) $ для некоторого $ q \in Q $.

	\emph{Трансляция}, определяемая магазинным преобразователем $ P $ \emph{при пустом магазине},
	есть
	$$ \tau_e(P) = \Set{(x, y) | (q_0, x, Z_0, \eps) \vdash^* (q, \eps, \eps, y), \quad q \in Q} $$
\end{definition}

\begin{definition}
	Магазинный преобразователь $ P = (Q, \Sigma, \Gamma, \Delta, \delta, q_0, Z_0, F) $ называется
	\emph{детерминированным}, если
	\begin{enumerate}
		\item $ \# \delta(q, a, Z) \le 1 \quad \forall q \in Q, ~ a \in \Sigma \cup \Set{\eps}, ~
			Z \in \Gamma $;
		\item если $ \delta(q, \eps, Z) \ne \O $, то $ \delta(q, a, Z) = \O \quad \forall a \in
			\Sigma $.
	\end{enumerate}
\end{definition}

\begin{lemma}
	$ T $ "--- простая схема синтаксически-управляемой трансляции.

	Существует недетерминированный магазинный преобразователь $ P $ такой, что $ \tau_e(P) =
	\tau(T) $.
\end{lemma}

\begin{lemma}
	$ P $ "--- недетерминированный магазинный преобразователь.

	Существует простая схема синтаксически-управляемой трансляции $ T $ такая, что $ \tau(T) =
	\tau_e(P) $.
\end{lemma}

Есть ещё теорема, которая две леммы зачем-то объединяет в ``тогда и только тогда''.
Я позволю себе здесь её не приводить.

\section[{Эквивалентность классов трансляций, задаваемых магазинными преобразователями
\texorpdfstring{\\}{}
при конечном состоянии и при пустом магазине.
Теорема 1.2}]
{Эквивалентность классов трансляций, задаваемых магазинными преобразователями при конечном
состоянии и при пустом магазине.
Теорема 1.2}

\begin{lemma}
	$ P $ "--- недетерминированный магазинный преобразователь и $ \tau = \tau(P) $.

	Существует недетерминированный магазинный преобразователь $ P' $ такой, что $ \tau_e(P') =
	\tau $.
\end{lemma}

\begin{lemma}
	$ P $ "--- недетерминированный магазинный преобразователь и $ \tau = \tau_e(P) $.

	Существует недетерминированный магазинный преобразователь $ P' $ такой, что $ \tau(P') = \tau $.
\end{lemma}

Теорема 1.2 "--- это такая же бессмысленная теорема, как и в предыдущем параграфе.

\section{Детерминированная генерация выходной цепочки простой SDT по левостороннему анализу входной
цепочки.
Теорема 1.3}

\begin{definition}
	Схема синтаксически-управляемой трансляции называется \emph{семантически однозначной}, если в
	ней не существует двух правил вида $ A \to \alpha, \beta $ и $ A \to \alpha, \gamma $, в которых
	$ \beta \ne \gamma $.
\end{definition}

\begin{definition}
	$ G = (V_N, V_T, P, S) $ "--- cfg, правила которой пронумерованы от 1 до $ p $, и $ S
	\xRightarrow[\mathrm{lm}]\pi x $ "--- левосторонний вывод $ x \in V_T^* $ в грамматике $ G $.

	Последовательность номеров правил $ \pi $, применённых в этом выводе называется
	\emph{левосторонним анализом} цепочки $ x $.
\end{definition}

\begin{theorem}
	$ T = (N, \Sigma, \Delta, R, S) $ "--- простая семантически однозначная схема синтаксически-
	управляемой трансляции, правила которой пронумерованы от 1 до $ p $.

	Существует детерминированный магазинный преобразователь $ P $ такой, что
	$$ \tau_e(P) = \Set{(\pi, y) | (S, S) \xRightarrow[\mathrm{lm}]\pi (x, y) \text{ для некоторой
	цепочки } x \in \Sigma^*} $$
\end{theorem}
