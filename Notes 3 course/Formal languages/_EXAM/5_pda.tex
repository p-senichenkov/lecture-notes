\part{Магазинные автоматы}

\section{МП-автомат.
Неформальное описание.
Формальное определение.
Понятие конфигурации}

\begin{undefthm}{Неформальное описание.}
	Магазинный автомат подобен конечному автомату, но, в отличие от последнего, имеет рабочую
	память "--- \emph{магазин}, в который записываются символы из ещё одного алфавита "---
	\emph{алфавита магазинных символов}.
	Каждое движение магазинного автомата определяется в зависимости от текущего состояния
	управления, входного символа, или независимо от него, и от верхнего символа магазина.
\end{undefthm}

\begin{definition}
	\emph{Недетерминированный магазинный автомат} есть формальная система \\
	$ M = (Q, \Sigma, \Gamma, \delta, q_0, Z_0, F) $, где
	\begin{itemize}
		\item $ Q $ "--- конечное \emph{множество состояний};
		\item $ \Sigma $ "--- конечный \emph{входной алфавит};
		\item $ \Gamma $ "--- конечный \emph{магазинный алфавит};
		\item $ \delta $ "--- отображение $ Q \times \bigl( \Sigma \cup \Set{\eps} \bigr)
			\times \Gamma \to 2^{Q \times \Gamma^*} $, представляющее конечное \emph{управление}
			автомата;
		\item $ q_0 \in Q $ "--- \emph{конечное состояние};
		\item $ Z_0 \in \Gamma $ "--- \emph{начальный символ магазина};
		\item $ F \sub Q $ "--- множество \emph{конечных состояний}.
	\end{itemize}
\end{definition}

\begin{definition}
	Под \emph{конфигурацией} будем подразумевать тройку $ (q, x, \alpha) $, где
	\begin{itemize}
		\item $ q \in Q $ "--- \emph{текущее состояние управления};
		\item $ x \in \Sigma^* $ "--- \emph{непросмотренная часть входной цепочки};
		\item $ \alpha \in \Gamma^* $ "--- \emph{магазинная цепочка}, причём крайний левый её символ
			считается находящимся на вершине магазина.
	\end{itemize}
\end{definition}

\section{Теорема об эквивалентности языков, принимаемых недетерминированными магазинными автоматами
при конечном состоянии и при пустом магазине}

\begin{definition}
	Язык, \emph{распознаваемый} магазинным автоматом
	$ M = (Q, \Sigma, \Gamma, \delta, q_0, Z_0, F) $ \emph{при конечном состоянии}, определим как
	множество
	$$ T(M) = \Set{w \in \Sigma^* | (q_0, w, Z_0) \vdash_M^* (q, \eps, \alpha), \quad q \in F} $$
\end{definition}

\begin{definition}
	Язык, \emph{распознаваемый} магазинным автоматом
	$ M = (Q, \Sigma, \Gamma, \delta, q_0, Z_0, F) $ \emph{при пустом магазине}, определим как
	множество
	$$ N(M) = \Set{w \in \Sigma^* | (q_0, w, Z_0) \vdash_M^* (q, \eps, \eps), \quad q \in Q} $$
\end{definition}

\begin{theorem}
	Язык $ L = N(M_1) $ для некоторого недетерминированного магазинного автомата $ M_1 $
	\textbf{тогда и только тогда}, когда $ L = T(M_2) $ для некоторого недетерминированного
	магазинного автомата $ M_2 $.
\end{theorem}

\section{Эквивалентность недетерминированных магазинных автоматов и контекстно-свободных грамматик}

\begin{definition}
	Магазинный автомат $ M = (Q, \Sigma, \Gamma, \delta, q_0, Z_0, F) $ является
	\emph{детерминированным}, если
	\begin{enumerate}
		\item для любых $ q \in Q, ~ Z \in \Gamma $ и
			$ \alpha \in \bigl( \Sigma \cup \Set{\eps} \bigr) $ значение
			$ \#\delta(q, a, Z) \le 1 $;
		\item для любых $ q \in Q $ и $ Z \in \Gamma $ всякий раз, как множество
			$ \delta(q, \eps, Z) \ne \O $, множество $ \delta(q, a, Z) = \O $ для всех
			$ a \in \Sigma $.
	\end{enumerate}
\end{definition}

\begin{remark}
	Условие 1 означает, что если движение определено, то оно единственно.
	Условие 2 предотвращает выбор между $ \eps $-движением и движением, использующим входной символ.
\end{remark}

% Не знаю, к чему здесь это определение
% \begin{definition}
% 	$ S \xRightarrow[\mathrm{lm}]* x\alpha, \quad x \in V_T^*, $
% 	$$
% 	\begin{vars}
% 		\alpha = A\beta, \quad A \in V_N, \quad \beta \in V^*, \\
% 		\alpha = \eps
% 	\end{vars} $$
%
% 	Цепочка $ x $ называется \emph{закрытой}, а $ \alpha $ "--- \emph{открытой} частью
% 	сентенциальной формы $ x\alpha $.
% \end{definition}

\begin{theorem}
	Если $ L $ "--- КС-язык, то существует недетерминированный магазинный автомат $ M $ такой, что
	$ L = N(M) $.
\end{theorem}

\begin{theorem}
	Если $ M $ "--- недетерминированный магазинный автомат, и $ L = N(M) $, то $ L $ "---
	контекстно-свободный язык.
\end{theorem}
