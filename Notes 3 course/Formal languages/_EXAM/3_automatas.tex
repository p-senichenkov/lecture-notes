\part{Конечные автоматы и регулярные грамматики}

\section{Конечные автоматы.
Теорема об отношениях эквивалентности и конечных автоматах}

\begin{definition}
	\emph{Конечным автоматом} называется формальная система $ M = (Q, \Sigma, \delta, q_0, F) $, где
	\begin{itemize}
		\item $ Q $ "--- конечное непустое \emph{множество состояний};
		\item $ \Sigma $ "--- конечный \emph{входной алфавит};
		\item $ \delta $ "--- отображение $ Q \times \Sigma \to Q $;
		\item $ q_0 \in Q $ "--- \emph{начальное состояние};
		\item $ F \sub Q $ "--- множество конечных состояний.
	\end{itemize}
\end{definition}

\begin{remark}
	Область определения отображения $ \delta $ можно расширить до $ Q \times \Sigma^* $
	следующим образом:
	$$ \delta'(q, \eps) = q, \quad \delta'(q, xa) = \delta \bigl( \delta'(q, x), a \bigr) $$
\end{remark}

\begin{definition}
	Цепочка $ x \in \Sigma^* $ \emph{принимается} конечным автоматом $ M $,
	если $ \delta(q_0, x) = p $ для некоторого $ p \in F $.

	Множество всех цепочек $ x \in \Sigma^* $, принимаемых конечным автоматом $ M $,
	называется языком, \emph{распознаваемым конечным автоматом} $ M $, и обозначается $ T(M) $.

	Любое множество цепочек, принимаемых конечным автоматом, называется \emph{регулярным}.
\end{definition}

\begin{definition}
	$ M $ "--- dfa

	Определим \emph{отношение эквивалентности} $ R $ на множестве $ \Sigma^* $:
	$$ (x, y) \in R \iff \delta(q_0, x) = \delta(q_0, y) $$
\end{definition}

\begin{definition}
	Отношение эквивалентности называется \emph{право-инвариантным}, если
	$$ xRy \implies xzRyz \quad \forall z \in \Sigma $$
\end{definition}

\begin{theorem}\label{th:fa-equiv-rel}
	Следующие утверждения эквивалентны:
	\begin{enumerate}
		\item Язык $ L \sub \Sigma^* $ распознаётся некоторым fa.
		\item Язык $ L $ есть объединение некоторых классов эквивалентности право-инвариантного
			отношения эквивалентности конечного индекса.
		\item\label{it:fa-equiv-rel-3} Определим отношение $ R $:
			$$ xRy \iff \forall z \in \Sigma^* \quad \Bigl( xz \in L \iff yz \in L \Bigr) $$

			Отношение $ R $ имеет конечный индекс.
	\end{enumerate}
\end{theorem}

\section{Теорема о единственности конечного автомата с минимальным числом состояний}

\begin{theorem}
	Конечный автомат с минимальным числом состояний, распознающий язык $ L $, единственен с
	точностью до изоморфизма (переименования состояний), и есть fa, задаваемый отношением $ R $ из
	п.~\ref{it:fa-equiv-rel-3} \autoref{th:fa-equiv-rel}.
\end{theorem}

\section{Недетерминированные конечные автоматы.
Теорема об эквивалентности недетерминированных и детерминированных конечных автоматов}

\begin{definition}
	\emph{Недетерминированным конечным автоматом} называется формальная система $ M = (Q, \Sigma,
	\delta, q_0, F) $, где
	\begin{itemize}
		\item $ Q $ "--- конечное непустое \emph{множество состояний};
		\item $ \Sigma $ "--- \emph{входной алфавит};
		\item $ \delta $ "--- отображение $ Q \times \Sigma \to 2^Q $;
		\item $ q_0 \in Q $ "--- \emph{начальное состояние};
		\item $ F \sub Q $ "--- \emph{множество конечных состояний}.
	\end{itemize}
\end{definition}

Область определения $ \delta $ может быть расширена на $ Q \times \Sigma^* $ следующим образом:
$$ \delta(q, \eps) = \Set{q}, \quad \delta(q, xa) = \bigcup_{p \in \delta(q, x)}\delta(p, a) $$

Область определения $ \delta $ может быть расширена до $ 2^Q \times \Sigma^* $ следующим образом:
$$ \delta \bigl( \Set{p_1, p_2, \dots, p_k}, x \bigr) = \bigcup_{i = 1}^k \delta(p_i, x) $$

\begin{definition}
	Цепочка $ x \in \Sigma^* $ \emph{принимается} недетерминированным конечным автоматом $ M $, если
	существует состояние $ p \in F $ такое, что $ p \in \delta(q_0, x) $.

	Множество всех цепочек, принимаемых ndfa $ M $, обозначается $ T(M) $.
\end{definition}

\begin{theorem}
	$ L $ "--- язык, распознаваемый НКА.

	Тогда существует ДКА, который распознаёт $ L $.
\end{theorem}

\section{Конечные автоматы и языки типа 3.
Теоремы об эквивалентности конечных автоматов и грамматик типа 3}

\begin{definition}
	$ M = (Q, \Sigma, \delta, q_0, F) $ "--- КА

	\emph{Конфигурацией} конечного автомата $ M $ назовём состояние управления $ q \in Q $ в паре с
	непрочитанной частью входной цепочки.
\end{definition}

\begin{theorem}
	$ G $ "--- грамматика типа 3.

	Тогда существует КА $ M $ такой, что $ T(M) = L(G) $.
\end{theorem}

\begin{theorem}
	$ M $ "--- КА.

	Существует грамматика $ G $ типа 3 такая, что $ L(G) = T(M) $.
\end{theorem}

\section{Теорема о том, что класс регулярных языков образует булеву алгебру}

\begin{definition}
	\emph{Булева алгебра множеств} есть совокупность множеств, замкнутая относительно операций
	объединения, дополнения и пересечения.
\end{definition}

\begin{definition}
	$ L \sub \Sigma_1^*, \quad \Sigma_1 \sub \Sigma_2 $

	Под \emph{дополнением} языка $ L $ подразумевается множество $ L = \Sigma_2^* \setminus L $.
\end{definition}

\begin{lemma}
	Класс языков типа 3 замкнут относительно объединения.
\end{lemma}

\begin{lemma}
	Класс множеств, распознаваемых конечными автоматами замкнут относительно дополнения.
\end{lemma}

\begin{theorem}
	Класс множеств, принимаемых конечными автоматами, образует булеву алгебру.
\end{theorem}

\begin{theorem}
	Все конечные множества распознаются конечными автоматами.
\end{theorem}

\section{Замкнутость регулярных языков относительно произведения и замыкания}

\begin{definition}
	\emph{Произведением} или \emph{конкатенацией} языков $ L_1 $ и $ L_2 $ называется множество
	$$ L_1L_2 = \Set{z | z = xy, \quad x \in L_1, ~ y \in L_2} $$
\end{definition}

\begin{theorem}
	Класс множеств, распознаваемых конечными автоматами, замкнут относительно произведения.
\end{theorem}

\begin{definition}
	\emph{Замыкание языка} $ L $ есть множество
	$$ L^* = \bigcup_{k = 0}^\infty L^k $$
\end{definition}

\begin{theorem}
	Класс множеств, принимаемых конечными автоматами, замкнут относительно замыкания.
\end{theorem}

\section{Теорема Клини и следствие из неё}

\begin{theorem}
	Класс множеств, принимаемых конечными автоматами, является наименьшим классом, содержащим все
	конечные множества, замкнутым относительно объединения, произведения и замыкания.
\end{theorem}

\begin{implication}
	Любое выражение, построенное из конечных подмножеств множества $ \Sigma^* $, где $ \Sigma $ "---
	конечный алфавит, и конечного числа операций объединения, произведения и замыкания со скобками
	обозначает множество, принимаемое некоторым конечным автоматом.
\end{implication}
