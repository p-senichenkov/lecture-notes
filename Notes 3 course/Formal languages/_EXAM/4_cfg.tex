\part{Контекстно-свободные грамматики}

\section{Теорема об алгоритмической разрешимости пустоты языка, порождаемого КС"=грамматикой}

\begin{theorem}
	Существует алгоритм для определения, является ли язык, порождаемый данной КС-грамматикой,
	пустым.
\end{theorem}

\section{Теоремы об исключении непродуктивных и недостижимых нетерминалов из КС"=грамматик}

\begin{theorem}
	Для любой КС-грамматики $ G $, порождающей непустой язык, можно найти эквивалентную
	КС-грамматику $ G_1 = (V_N^1, V_T, P^1, S) $, в которой для любого нетерминала $ A $ существует
	терминальная цепочка $ x $ такая, что $ A \xRightarrow[G]* x $.
\end{theorem}

\begin{definition}
	Нетерминалы из $ V_N^1 $ принято называть \emph{продуктивными}.
\end{definition}

\begin{definition}
	Нетерминалы, которые не участвуют ни в каком выводе сентенциальной цепочки, называются
	\emph{недостижимыми}.
\end{definition}

\begin{theorem}\label{th:reduced-cfg}
	Для любой КС-грамматики, порождающей непустой язык $ L $, можно найти КС"=грамматику
	$ G = (V_N, V_T, P, S) $, порождающую язык $ L $ такую, что для каждого её нетерминала $ A $
	существует вывод вида
	$$ S \xRightarrow[G]* x_1Ax_3 \xRightarrow[G]* x_1x_2x_3, \quad x_1, x_2, x_3 \in V_T^* $$
\end{theorem}

\begin{definition}
	Контекстно-свободные грамматики, удовлетворяющие условию \autoref{th:reduced-cfg}, принято
	называть \emph{приведёнными}.
\end{definition}

\section{Лемма о левостороннем выводе.
Теорема об исключении цепных правил из КС"=грамматики}

\begin{definition}
	Вывод в КС-грамматике назовём \emph{левосторонним}, если на каждом его шаге производится замена
	крайнего левого вхождения нетерминала.
\end{definition}

\begin{notation}
	$ \xRightarrow[\mathrm{lm}]{}, \xRightarrow[\mathrm{lm}]* $
\end{notation}

\begin{lemma}
	$ G = (V_N, V_T, P, S) $ "--- контекстно-свободная грамматика.

	Если $ S \xRightarrow[G]* x $, где $ x \in V_T^+ $, то существует и левосторонний вывод
	$ S \xRightarrow[\mathrm{lm}]* x $ в той же грамматике.
\end{lemma}

\begin{theorem}
	Любой контекстно-свободный язык может быть порождён контекстно-свободной грамматикой, не
	содержащей цепных правил, \ie правил вида $ A \to B $, где $ A $ и $ B $ "--- нетерминалы.
\end{theorem}

\section{Теорема о нормальной форме Хомского}

\begin{theorem}
	Любой КС-язык может быть порождён грамматикой, в которой все правила имеют вид $ A \to BC $ или
	$ A \to a $ ($ A, B, C $ "--- нетерминалы, $ a $ "--- терминал).
\end{theorem}

\section{Леммы о подстановке и устранении левой рекурсии}

\begin{lemma}[о подстановке]
	$ G = (V_N, V_T, P, S) $ "--- cfg, $ \quad A \to \alpha_1B\alpha_2, \quad A, B \in V_N, \quad
	\alpha_1, \alpha_2 \in V^*, $ \\
	$ \Set{B \to \beta_i | \beta_i \in V^+, \quad
	i = 1, \dots, m} $ "--- множество всех $ B $-\emph{порождений}, \ie правил с нетерминалом
	$ B $ в левой части.

	Грамматика $ G_1 = (V_N, V_T, P_1, S) $ получается из грамматики
	$ G $ отбрасыванием правила $ A \to \alpha_1B\alpha_2 $ и добавлением правил вида $ A \to
	\alpha_1\beta_i\alpha_2 $.

	Тогда $ L(G) = L(G_1) $.
\end{lemma}

\begin{lemma}[об устранении левой рекурсии]
	$ G = (V_N, V_T, P, S) $ "--- КС-грамматика, \\
	$ \Set{A \to A\alpha_i | A \in V_N, ~
	\alpha_i \in V^*, \quad i = 1, \dots, m} $ "--- множество всех \emph{леворекурсивных}
	$ A $-\emph{порождений}, \\
	$ \Set{A \to \beta_j | j = 1, \dots, n} $ "--- множество всех
	прочих $ A $-порождений.

	$ G_1 = \bigl(V_N \cup \Set{Z}, V_T, P_1, S \bigr) $ "--- КС-грамматика, образованная
	добавлением нового нетерминала $ Z $ и заменой всех $ A $-порождений правилами:
	\begin{itemize}
		\item $ A \to \beta_j, \quad A \to \beta_jZ, \quad j = 1, \dots, n $;
		\item $ Z \to \alpha_i, \quad Z \to \alpha_iZ, \quad i = 1, \dots, m $.
	\end{itemize}

	Тогда $ L(G_1) = L(G) $.
\end{lemma}

\section{Теорема о нормальной форме Грейбах}

\begin{definition}
	Говорят, что КС-грамматика $ G = (V_N, V_T, P, S) $ представлена в \emph{нормальной форме
	Грейбах}, если каждое её правило имеет вид $ A \to a\alpha $, где $ a \in V_T $,
	$ \alpha \in V_N^* $.
\end{definition}

\begin{theorem}
	Каждый КС-язык может быть порождён КС-грамматикой в нормальной форме Грейбах.
\end{theorem}

\section{Теорема ``uvwxy''}

\begin{theorem}
	$ L $ "--- cfl

	Существуют постоянные $ p $ и $ q $, зависящие только от языка $ L $, такие, что если существует
	$ z \in L $ при $ |z| > p $, то цепочка $ z $ представима в виде $ z = uvwxy $, где
	$ |vwx| \le q $, причём $ v $ и $ x $ одновременно не пусты, так что для любого целого
	$ i \ge 0 $ цепочка $ uv^iwx^iy \in L $.
\end{theorem}

\section[{Теоремы об алгоритмической разрешимости конечности КС-языков и исключении не-терминалов,
порождающих конечные языки, из КС-грамматик}]
{Теоремы об алгоритмической разрешимости конечности КС-языков и исключении нетерминалов,
порождающих конечные языки, из КС-грамматик}

\begin{theorem}
	Существует алгоритм для определения, порождает ли данная КС-грамматика $ G $ конечный или
	бесконечный язык.
\end{theorem}

\begin{theorem}
	Для всякой КС-грамматики $ G_1 $ можно найти эквивалентную ей КС-грамматику $ G_2 $ такую, что
	если $ A $ "--- нетерминал грамматики $ G_2 $, не являющийся начальным нетерминалом, то из $ A $
	выводимо бесконечно много терминальных цепочек.
\end{theorem}

\section{Свойство самовставленности.
Теорема о регулярности языков, порождаемых несамовставленными КС-грамматиками}

\begin{definition}
	Говорят, что грамматика $ G $ является \emph{самовставленной}, если существует нетерминал $ A $
	такой, что $ A \xRightarrow[G]+ \alpha_1A\alpha_2 $, где $ \alpha_1, \alpha_2 \in V^+ $.

	Говорят также, что нетерминал $ A $ является \emph{самовставленным}.
\end{definition}

\begin{theorem}
	$ G $ "--- несамовставленная грамматика.

	Тогда $ L(G) $ "--- регулярное множество.
\end{theorem}

\section{Теорема об \texorpdfstring{$ \eps $}{e}-правилах в контекстно-свободных грамматиках}

\begin{theorem}
	$ L $ "--- язык, порождаемый грамматикой $ G = (V_N, V_T, P, S) $, где каждое правило в $ P $
	имеет вид $ A \to \alpha $, где $ A $ "--- нетерминал, $ \alpha \in V^* $.

	Тогда $ L $ может быть порождён грамматикой, в которой каждое правило имеет вид
	$ A \to \alpha $, где $ A $ "--- нетерминал, $ \alpha \in V^+ $, либо $ S \to \eps $, и кроме
	того, начальный терминал грамматики $ S $ не появляется в правой части никакого правила.
\end{theorem}

\section{Специальные типы контекстно-свободных языков и грамматик}

\begin{definition}
	Говорят, что КС-грамматика $ G = (V_N, V_T, P, S) $ \emph{линейна}, если каждое её правило
	имеет вид $ A \to uBv $ или $ A \to u $, где $ A, B \in V_N $, $ u, v \in V_T^* $.

	Если $ v = \eps $, то грамматика называется \emph{праволинейной}, если $ u = \eps $, то она
	\emph{леволинейна}.
\end{definition}

\begin{definition}
	Говорят, что грамматика $ G = (V_N, V_T, P, S) $ \emph{последовательна}, если нетерминалы
	$ A_1, \dots, A_k \in V_N $ можно упорядочить так, что если $ A_i \to \alpha \in P $, то
	$ \alpha $ не содержит ни одного нетерминала $ A_j $ с индексом $ j < i $.
\end{definition}

\begin{definition}
	Если КС-язык $ L $ над алфавитом $ V_T $ есть подмножество языка $ w_1^*w_2^* \dots w_k^* $ для
	некоторого $ k $, где $ w_i \in V_T^+ $, $ i = 1, \dots, k $, то говорят, что $ L $ "---
	\emph{ограниченный} язык.
\end{definition}

\begin{definition}
	Говорят, что контекстно-свободная грамматика $ G $ \emph{неоднозначна}, если в языке $ L(G) $
	существует цепочка, выводимая двумя или более различными левосторонними выводами, то есть
	существуют различные деревья вывода с одинаковыми результатами.
\end{definition}

\begin{definition}
	Если все грамматики, порождающие некоторый контекстно-свободный язык, неоднозначны, то говорят,
	что этот язык \emph{существенно неоднозначен}.
\end{definition}
