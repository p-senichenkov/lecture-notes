\chapter{Линейные функционалы}

\section{Продолжение линейных функционалов}

\subsection{Небольшое отступление: трансфинитная индукция}

\begin{definition}
	$ P $ полуупорядочено, $ A \sub P $

	Будем говорить, что $ A $ "--- \emph{цепь (линейно-упорядоченное подмножество)}, если
	$$ \forall a, b \in A \quad
	\begin{cases}
		a \le b \\
		b \le a
	\end{cases} $$
\end{definition}

\begin{definition}
	$ P $ полуупорядочено

	$ x $ "--- \emph{максимальный элемент} $ P $, если
	$$ \forall y \in P : x \le y \quad x = y $$
\end{definition}

\subsubsection{Лемма Цорна}

\begin{lemma}[Цорн]
	$ P $ полуупорядочено

	Если для любой цепи имеется верхняя грань, то во всём множестве $ P $ существует максимальный
	элемент.
\end{lemma}

\begin{remark}
	Лемма Цорна \textbf{эквивалентна} аксиоме выбора.
\end{remark}

\begin{undefthm}{Аксиома выбора.}
	$ \set{B_\alpha \ne \emptyset}_{\alpha \in A}, \quad B_\alpha $ "--- множества

	$$ \exists \set{b_\alpha}_{\alpha \in A} : \quad b_\alpha \in B_\alpha $$
\end{undefthm}

\subsection{Возвращаемся в анализ}

\begin{definition}
	$ X $ "--- линейное пространство над $ \mathbb K $

	$ p : X \to \mathbb K $ называется \emph{выпуклым функционалом}, если
	\begin{enumerate}
		\item $ p(x + y) \le p(x) + p(y) $;
		\item $ p(tx) = tp(x) \quad \text{ при } t \ge 0 $.
	\end{enumerate}
\end{definition}

\subsection{Продолжение линейного функционала в вещественном линейном пространстве}

\begin{theorem}[Хан"--~Банах]
	$ X $ "--- линейное над $ \R, \quad p : X \to \R $ "--- выпуклый функционал, $ \quad L $ "--- 
	подпространство $ X, \quad f \in \mathscr Lin(L, \R), \quad f(x) \overset L \le p(x) $
	(говорят, что $ f $ \emph{подчинён} $ p $)

	$$ \exists g \in \mathscr Lin(X, \R) : \quad
	\begin{cases}
		g \bigr|_L = f \\
		g(x) \overset X \le p(x)
	\end{cases} $$
\end{theorem}

\begin{eproof}
\item Возьмём $ z \in X \setminus L $
	$$ L_1 \coloneq \mathscr L\set{L, z} = \Set{tx + z | t \in \R} $$

	Построим $ f_1 \in \mathscr Lin(L_1, \R) : \quad
	f_1 \bigr|_L = f, \quad f_1(y) \overset{L_1}\le p(y) $.
	$$ f_1(c) \coloneq c $$
	Докажем, что можно выбрать такой $ c $.
	$$ f_1(x + tz) = f(x) + tc $$
	$$ f_1(y) \overset{L_1}\le p(y) \iff f(x) + tc \overset{L} f(x) + tc \iff
	\begin{cases}
		f(x) + tc \le p(x + tz), \quad t > 0, \\
		f(x) - tc \le p(x - tz), \quad t > 0
	\end{cases} $$
	Разделим на $ t $:
	$$
	\begin{cases}
		f \bigl( \frac x t \bigr) + c \le p \bigl( \frac x t + z \bigr), \\
		f \bigl( \frac x t \bigr) - c \le p \bigl( \frac x t - z \bigr)
	\end{cases} $$
	Обозначим $ u = \frac x t, ~ v = \frac x t $ (никак друг с другом не связанные).
	$$
	\begin{cases}
		f(u) + c \le p(u + z) \quad \forall u \in L, \\
		f(v) - c \le p(v - z) \quad \forall v \in L
	\end{cases} $$
	$$ f(v) - p(v - z) \le c \le p(u + z) - f(u) $$
	Рассмотрим $ A = \Set{p(u + z) - f(u)}_{u \in L} \sub \R, \quad
	B = \Set{f(v) - p(v - z)}_{v \in L} \sub \R $.
	Проверим, что $ \forall a \in B \quad \forall b \in B \quad b \le a $.
	$$ b \le a \iff f(v) - p(v - z) \le p(u + z) - f(u) \iff
	\underbrace{f(u) + f(v)}_{f(u + v) \in L} \le p(u + z) + p(v - z) $$
	$ f $ подчинён $ p $, значит,
	$$ f(u + v) \le p(u + v) \underset{\text{выпуклость}}\le p(u + z) + p(v - z) \implies
	\exists c \in \R : \quad f_1(z) = c $$
\item $ \mathcal P \coloneq \Set{(M, h)}, \quad L \sub M \sub X, \quad
	M $ "--- подпр-во, $ \quad h \in \mathscr Lin(M, \R), \quad h \bigr|_L = f, \quad
	h(x) \overset M \le p(x) $

	Определим порядок:
	$$ (M, h) \le (M_1, h_1) \iff
	\begin{cases}
		M \sub M_1, \\
		h_1 \bigr|_M = h
	\end{cases} $$

	Пусть $ A $ "--- цепь в $ \mathcal P $, \ie
	$$ A = \Set{(M_\alpha, h_\alpha)}_{\alpha \in I} : \quad \forall \alpha, \beta \in I \quad
	\left[
	\begin{array}{l}
		(M_\alpha, h_\alpha) \le (M_\beta, h_\beta) \\
		(M_\beta, h_\beta) \le (M_\alpha, h_\alpha)
	\end{array} \right. $$

	Построим верхнюю грань для $ A $.
	$$ M_0 \coloneq \bigcup_{\alpha \in I} M_\alpha $$

	Проверим, что $ M_0 $ "--- подпространство.
	Пусть $ x, y \in M_0 $.
	$$ \exists \alpha, \beta : x \in M_\alpha, ~ y \in M_\beta \implies \left[
	\begin{array}{l}
		M_\alpha \sub M_\beta, \\
		M_\beta \sub M_\alpha
	\end{array} \right. $$
	Пусть выполняется первое.
	$$ \implies x, y \in M_\beta \text{ "--- подпространство } \implies ax + by \in M_\beta \subset
	M_0 \implies M_0 \text{ "--- подпространство} $$

	Определим $ h_0 : M_0 \to \R $.

	Возьмём $ x \in M_0 $.
	$$ \exists \alpha : x \in M_\alpha $$
	Положим $ h_0(x) = h_\alpha(x) $.

	Проверим корректность.
	Пусть $ x \in M_\beta $.
	$$ \left[
	\begin{array}{l}
		(M_\alpha, H_\alpha) \le (M_\beta, h_\beta), \\
		(M_\beta, h_\beta) \le (M_\alpha, h_\alpha)
	\end{array} \right. $$
	Пусть выполнено первое.
	$$ x \in M_\alpha, ~ x \in M_\beta \implies h_\beta(x) = h_\alpha(x) = h(x) $$

	Можно проверить, что $ h_0 \in \mathscr Lin(M_0, \R) $.
	$$ \forall \alpha \in I \quad (M_\alpha, h_\alpha) \le (M_0, h_0) \implies
	(M_0, h_0) \text{ "---  верхняя грань для } A $$
	По лемме Цорна, в $ \mathcal P $ существует максимальный элемент $ (M, h) $.
	$$ L \sub M, \quad h \bigr|_L = f, \quad h(x) \overset M \le p(x) $$

	Докажем, что $ M = X $.
	\textbf{Пусть} $ \exists z \in X \setminus M $.
	$$ M_1 \coloneq \mathscr Lin\set{M, z} $$
	По первой части доказательства
	$$ \exists h_1 \in \mathscr Lin(M_1, \R) : \quad h_1 \bigr|_M = h, \quad h_1(x) \overset{M_1}\le
	p(x) $$
	$$ \implies (M_1, h_1) \in \mathcal P $$
	$$ (M, h) \le (M_1, h_1), \quad M \subsetneq M_1 $$
	Это \textbf{противоречит} максимальности $ (M, h) $.
\end{eproof}

\begin{statement}\label{st:cont:1}
	$ X $ линейно над $ \R, \quad p $ "--- выпуклый функционал, $ \quad
	f \in \mathscr Lin(X, \R), \quad f(x) \overset X \le p(x) $

	$$ f(x) \ge -p(-x) $$
\end{statement}

\begin{proof}
	$ f(x) \le p(x) \implies f(-x) \le p(-x) \implies -f(x) \le p(-x) $
\end{proof}

\begin{statement}
	$ p $ "--- полунорма, $ \quad f(x) \overset X \le p(x) $

	$$ |f(x)| \le p(x) $$
\end{statement}

\begin{proof}
	$ p(-x) \undereq{p \text{ "--- полунорма}} p(x) \implies f(x) \ge -p(x) $
\end{proof}

\subsection{Обобщённый предел ограниченной последовательности}

\begin{theorem}
	$ l_\R^\infty \coloneq \Set{x = \Set{x_n \in \R}_{n = 1}^\infty |
	\|x\| = \sup\limits_{n \in \N}|x_n| < +\infty} $

	$$ \exists L \in (l^\infty)^* : \quad \|l\|_{(l^\infty)^*} = 1, \quad
	\forall x = \Set{x_n}_{n = 1}^\infty \in l^\infty \quad
	\varliminf\limits_{n \to \infty} x_n \le L(x) \le \varlimsup\limits_{n \to \infty} x_n $$

	В частности, если $ \exists \lim x_n = x_0 $, то $ L(x) = x_0 $.
\end{theorem}

\begin{proof}
	Для $ x \in l^\infty $ положим $ p(x) = \varlimsup\limits_{n \to \infty} x_n $.

	Докажем, что $ p $ "--- выпуклый функционал:
	\begin{itemize}
		\item $ t > 0 \implies p(tx) = tp(x) $ "--- очевидно;
		\item $ p(x + y) \le p(x) + p(y) $
			$$ \varlimsup x_n + \varlimsup y_n \overset?\le \varlimsup(x_n + y_n) $$
			Вспомним определение верхнего предела:
			$$ a_n \coloneq \sup\limits_{m \ge n} \Set{x_m} \implies a_n \downarrow \implies
			\exists \lim a_n = a, \quad \varlimsup x_n \coloneq a $$
			$$ b_n = \sup\limits_{m \ge n}, \quad b = \varlimsup y_n $$
			$$ c_n = \sup\Set{x_m + y_m}, \quad c = \varlimsup \Set{x_n + y_n} $$
			Зафиксируем $ n $.
			$$ \forall m \ge n \quad x_m + y_m \le a_n + b_n \implies c_n \le a_n + b_n $$
			Устремим $ n \to \infty $:
			$$ c \le a + b $$
	\end{itemize}

	$$ C \sub l^\infty, \quad
	C \coloneq \Set{x = \Set{x_n}_{n = 1}^\infty | \exists \lim x_n = x_0} $$

	Определим $ f : C \to \R : \quad f(x) = \lim x_n = x_0 $.
	$$ f(x) = \lim x_n \le \varlimsup x_n \implies f(x) \overset C \le p(x) $$
	Применим теорему Хана"--~Банаха:
	$$ \exists g \in \mathscr Lin(l^\infty, \R), \quad g(x) \overset C = f(x), \quad
	g(x) \overset{l^\infty}\le \varlimsup x_n $$
	По \autoref{st:cont:1}
	$$ g(x) \ge -p(-x) = -\varlimsup(-x_n) \undereq{\text{задача из Демидовича}} \varliminf x_n $$

	$ g = L $ "--- в теореме, $ \quad g \in \mathscr Lin(l^\infty, \R) $.
	$$
	\begin{rcases}
		g(x) \le \varlimsup x_n \le \sup|x_n| \\
		g(x) \ge \varliminf x_n \ge -\sup|x_n|
	\end{rcases} \implies |g(x)| \le \|x\|_\infty \implies \|g\| \le 1 \implies
	g \in (l^\infty)^* $$

	Возьмём $ x = \Set{1, 1, \dotsc} $.
	$$ \|x\| = 1, \quad |g(x)| = 1 \implies \|g\| \ge \frac{|g(x)|}{\|x\|} = 1 \implies \|g\| = 1 $$
\end{proof}

\subsection{Продолжение линейного функционала в комплексном линейном пространстве}

\begin{theorem}[Боненблюст"--~Собчик]
	$ X $ линейно над $ \Co, \quad p $ "--- полунорма, $ \quad L $ "--- подпространство $ X, $ \\
	$ f \in \mathscr Lin(L, \Co), \quad |f(x)| \overset L \le p(x) $

	$$ \exists g \in \mathscr Lin(X, \Co) : \quad
	\begin{cases}
		g \bigr|_L = f, \\
		|g(x)| \overset X \le p(x)
	\end{cases} $$
\end{theorem}
