\chapter{Линейные функционалы}

\section{Спектр оператора}

\begin{definition}
	$ X $ "--- банахово, $ \quad T \in \mathscr B(X) $

	Определим \emph{спектральный радиус}:
	$$ r(T) = \max\limits_{\lambda \in \sigma(T)}|\lambda| $$
\end{definition}

\begin{statement}
	$$ r(T) = \lim\limits_{n \to \infty} \sqrt[n]{\|T^n\|} $$
\end{statement}

\begin{noproof}
\end{noproof}

По теореме о свойствах резольвенты,
$$ r(T) \le \|T\| $$

\section{Компактные операторы}

\begin{definition}
	$ X, Y $ "--- банаховы, $ \quad T \in \mathscr Lin(X, Y) $

	$ T $ называется \emph{компактным}, если $ T \bigl( \mathtt B_1(0) \bigr) $ относительно
	компактен.
\end{definition}

\begin{notation}
	$ \mathrm{Com}(X, Y) $ "--- множество компактных операторов.
\end{notation}

\begin{props}
	\item $ \mathrm{Com}(X, Y) \sub \mathscr B(X, Y) $
	\item $ T \in \mathrm Com(X, Y), \quad A \sub X $ ограничено
		$$ T(A) \text{ относительно компактно} $$
	\item $ T \in \mathrm{Com}(X, Y) $ \textbf{тогда и только тогда}, когда
		$$ \Set{x_n \in X}_{n = 1}^\infty \text{ "--- ограниченная } \implies \exists
		\Set{n_k}_{k = 1}^\infty : \quad \exists \lim\limits_{k \to \infty}T_{n_k} \in Y $$
\end{props}

\begin{eproof}
\item $ T \in \mathrm{Com}(X, Y) $

	Обозначим $ B = \mathtt B_1^X(0) $.
	$$ \implies \ol{T(B)} \text{ "--- компакт } \implies T(B) \text{ ограничено } \implies T \in
	\mathscr B(X, Y) $$
\end{eproof}

\begin{definition}
	$ X, Y $ "--- банаховы, $ T \in \mathscr B(X, Y) $

	Если $ \dim T(X) \le +\infty $, то $ T $ называется \emph{оператором конечного ранга}.
\end{definition}

\begin{eg}
	$ X, Y $ "--- банаховы, $ \quad f_1, \dots, f_n \in X^*, \quad y_1, \dots, y_n \in Y $

	Для $ x \in X $ определим
	$$ Tx = \sum_{j = 1}^n f_j(x) y_j $$

	$$ T(x) \sub \mathscr L \Set{y_j}_{j = 1}^n \implies \dim T(X) \le n \implies
	T \text{ "--- конечного ранга} $$

	Все операторы конечного ранга имеют такой вид.
\end{eg}

\begin{statement}
	$ X, Y $ "--- банаховы, $ \quad T $ "--- конечного ранга

	$$ T \in \mathrm{Com}(X, Y) $$
\end{statement}

\begin{proof}
	$ B \coloneq \mathtt B_1^X(0) $
	\begin{multline*}
		T(B) \sub T(X) \implies T(B) \text{ "--- ограниченное множество в конечномерном пр-ве }
		\implies \\
		\implies T(B) \text{ относительно компактно}
	\end{multline*}
\end{proof}

\subsection{Замкнутое подпространство образа компактного оператора конечномерно}

\begin{theorem}
	$ X, Y $ "--- банаховы, $ \quad T \in \mathrm{Com}(X, Y), \quad L \sub T(X) $ "--- замкнутое
	подпространство

	$$ \dim L < +\infty $$
\end{theorem}

\begin{iproof}
\item $ T(X) $ "--- замкнутое подпространство $ Y $
	\begin{multline*}
		T \in \mathscr B(X, T(X)), \quad T(X) \text{ "--- банахово } \underimp{\text{т. Банаха об
		откр. отобр.}} T \text{ открыто } \implies \\
		\exists r > 0 : \quad \mathtt B_r^{T(x)}(0) \sub \underbrace{T(B)}_{\text{отн. комп.}}
		\implies \mathtt B_r^{T(x)}(0) \text{ относительно компактно } \underimp{\text{т. Рисса}}
		\dim T(X) < +\infty
	\end{multline*}
\item $ L \sub T(X) $ "--- замкнутое подпространство

	Обозначим $ X_1 = T^{-1}(L) $ (прообраз)
	$$ T \in \mathscr B(X, Y) \underimp{L \text{ замкнуто}} T^{-1}(L) \text{ замкнуто в } X \implies
	X_1 \text{ замкнуто } \implies X_1 \text{ "--- банахово} $$
	По первой части доказательства
	$$ T(x_1) = L \implies \dim L < +\infty $$
\end{iproof}

\subsection{Арифметические операции и предельный переход в пространстве компактных операторов}

\begin{theorem}
	\hfill
	\begin{enumerate}
		\item $ X, Y $ "--- банаховы пространства
			$$ \mathrm{Com}(X, Y) \text{ "--- замкнутое подпространство } \mathscr B(X, Y) $$
		\item $ X, Y, Z $ "--- банаховы, $ \quad X \xrightarrow T Y \xrightarrow S Z $
			\begin{enumerate}
				\item $ T \in \mathrm{Com}(X, Y), \quad S \in \mathscr B(Y, Z) $
					$$ ST \in \mathrm{Com}(X, Z) $$
				\item $ T \in \mathscr B(X, Y), \quad S \in \mathrm{Com}(Y, Z) $
					$$ ST \in \mathrm{Com}(X, Z) $$
			\end{enumerate}
	\end{enumerate}
\end{theorem}

\begin{proof}
	Обозначим $ B = \mathtt B_1^X(0) $.
	\begin{enumerate}
		\item
			\begin{itemize}
				\item $ \alpha \in \Co, \quad T \in \mathrm{Com}(X, Y) \implies
					\alpha T \in \mathrm{Com}(X, Y) $ "--- очевидно
				\item Возьмём $ T, S \in \mathrm{Com}(X, Y) $.

					$ T(B) $ относительно компактно $ \implies T(B) $ вполне ограничено.
					Возьмём $ \eps > 0 $ и рассмотрим $ \eps $-сети:
					$$ \exists E \sub Y \text{ "--- конечная $ \eps $-сеть для } T(B) $$
					$$ \exists F \sub Y \text{ "--- конечная $ \eps $-сеть для } S(B) $$
					\begin{multline*}
						E + F = \Set{e + f | e \in E, ~ f \in F} \text{ "--- конечное, $ 2\eps $-сеть для }
						T(B) + S(B) \implies \\
						\implies T(B) + S(B) \text{ относительно компактно}
					\end{multline*}
					$$ (T + S)(B) \sub T(B) + S(B) \implies (T + S)(B) \text{ относительно компактно}
					\implies T + S \in \mathrm{Com}(X, Y) $$
				\item $ \Set{T_n \in \mathrm{Com}(X, Y)}_{n = 1}^\infty, \quad
					\lim\limits_{n \to \infty}\|T_n - T\| = 0 $

					Возьмём $ \eps > 0 $.
					$$ \exists n \in \N : \quad \|T - T_n\| < \eps $$
					При этом, $ T_n(B) $ относительно компактно.
					$$ \implies \exists E \text{ "--- конечная $ \eps $-сеть для } T_n(B) $$

					Проверим, что $ E $ "--- $ \eps $-сеть для $ T(B) $.
					Возьмём $ x \in B $.
					$$ \exists e \in E : \quad \|T_nx - e\| < \eps $$
					$$ \|Tx - e\| \trile \|Tx - T_nx\| + \|T_nx - e\| \le
					\underbrace{\|T - T_n\|}_{< \eps} \cdot \underbrace{\|x\|}_{< 1} + \eps <
					2\eps $$
					Значит, $ T(B) $ вполне ограничено и относительно компактно.
			\end{itemize}
		\item $ X, Y, Z $
			\begin{enumerate}
				\item $ T \in \mathrm{Com}(X, Y) $
					$$ T(B) \text{ относительно компактно }, \quad S \in \mathscr B(Y, Z) $$
					$$ \underimp{\text{непр. } S} S \bigl( T(B) \bigr)
					\text{ относительно компактно} \implies ST \in \mathrm{Com}(X, Z) $$
				\item $ T \in \mathscr B(X, Y) $
					$$ \implies T(B) \text{ ограничено}, \quad S \in \mathrm{Com}(X, Y) $$
					$$ \implies S \bigl( T(B) \bigr) \text{ относительно компактно} $$
			\end{enumerate}
	\end{enumerate}
\end{proof}

\begin{implication}
	$ X $ "--- банахово

	Тогда $ \mathrm{Com}(X) $ "--- \emph{двусторонний замкнутый идеал алгебры } $ \mathscr B(X) $.
\end{implication}

\subsection{Компактность сопряжённого оператора}

\begin{theorem}
	$ H $ "--- гильбертово
	$$ T \in \mathrm{Com}(H) \iff T^* \in \mathrm{Com}(H) $$
	($ T^* $ "--- эрмитово сопряжённый)
\end{theorem}

\begin{iproof}
\item $ T \in \mathrm{Com}(H) $

	Возьмём $ x \in H $.
	\begin{equ}{comp_dual:1}
		\|T^*x\|^2 = (T^*x, T^*x) = (TT^*x, x) \underset{\text{К"--~Б}}\le \|TT^*x\| \cdot \|x\|
	\end{equ}

	Возьмём $ \Set{x_n \in H}_{n = 1}^\infty $ такую, что
	$$ \exists M > 0 : \quad \|x_n\| \le \quad \forall n \in \N $$

	Проверим, что $ \exists \Set{n_k} : \quad \exists \lim T^*x_{n_k} $.
	\begin{multline*}
		\begin{rcases}
			T \in \mathrm{Com}(H) \\
			T^* \in \mathscr B(H)
		\end{rcases} \implies TT^* \in \mathrm{Com}(H) \implies \exists \Set{n_k} : \quad
		\exists \lim TT^*x_{n_k} \implies \\
		\implies \Set{TT^*x_{n_k}} \text{ фундаментальна}
	\end{multline*}
	\begin{multline*}
		\|T^*x_{n_k} - T^*x_{n_j}\|^2 = \bigl\| T^*(x_{n_k} - x_{n_j}) \bigr\|^2
		\underset{\eref{comp_dual:1}}\le
		\underbrace{\|(TT^*)(x_{n_k}) - TT^*x_{n_j}\|}_{\underarr{k, j \to \infty} 0} \cdot
		\underbrace{\|x_{n_k} - x_{n_j}\|}_{\le 2M} \implies \\
		\implies \Set{T^*x_{n_k}} \text{ фундаментальна } \implies \exists \lim T^*x_{n_k}
		\implies T^* \in \mathrm{Com}(H)
	\end{multline*}
\item $ T^* \in \mathrm{Com}(H) \implies T = T^{**} \in \mathrm{Com}(H) $
\end{iproof}

\begin{remark}
	$ X, Y $ "--- банаховы

	$$ T \in \mathrm{Com}(X, Y) \iff T^* \in \mathrm{Com}(Y^*, X^*) $$
\end{remark}

\begin{noproof}
\end{noproof}

\section{Спектр компактного оператора}

\begin{remark}[воспоминания из алгебры]
	$ X $ "--- линейное пространство, $ \quad T \in \mathscr Lin(X), \quad
	\Set{\lambda_j}_{j = 1}^n $ "--- с. ч., \\
	$ Tx_j = \lambda_j x_j, \quad \lambda_j \ne \lambda_k, \quad x_j $ "--- с. в. ($ x_j \ne 0 $)

	$$ \implies \Set{x_j} \text{ ЛНЗ} $$
\end{remark}

\begin{theorem}
	$ X $ "--- банахово, $ \quad T \in \mathrm{Com}(X), \quad \lambda \in \sigma_p(T) $ "--- с. ч.,
	$ \quad X_\lambda = \operatorname{Ker}(\lambda I - T) $ "--- собств. подпр-во, \\
	$ \delta > 0 $

	$$ \sum_{
		\begin{subarray}{c}
			\lambda \in \sigma_p(T) \\
			|\lambda| \ge \delta
		\end{subarray}
	} \dim(X_\lambda) < +\infty $$

	То есть, число линейно-независимых собственных векторов $ T $, соответствующих собственным
	числам $ \lambda $, таких, что $ |\lambda| \ge \delta $, конечно.
\end{theorem}

\begin{proof}
	\textbf{Пусть} $ \set{x_n}_{n = 1}^\infty $ "--- ЛНЗ с. в.:
	$$ Tx_n = \lambda_n x_n, \quad |\lambda_n| \ge \delta $$

	Рассмотрим последовательность подпространств:
	$$ L_n = \mathscr L\Set{x_j}_{j = 1}^n, \quad L_n \subsetneq L_{n + 1} $$
	$$ \underimp{\text{лемма Рисса}} \exists \Set{y_n}_{n = 1}^\infty : \quad \|y_n\| = 1, \quad
	\rho(y_{n + 1}, L_n) = \inf\limits_{x \in L_n} \|y_{n + 1} - x\| \ge \frac12 $$
	(\as $ \dim L_n = n $, то $ \exists y_{n + 1} : \rho(y_{n + 1}, L_n) = 1 $)

	Проверим, что $ \|Ty_n - Ty_m\| \ge \frac\delta2 $.
	Тогда не будет существовать фундаментальной подпоследовательности $ \Set{Ty_n} $, а значит,
	и последовательности $ \Set{n_k} $ такой, что $ \exists \lim T y_{n_k} $ "--- \contra с $ T \in
	\mathrm{Com}(X) $.
\end{proof}
