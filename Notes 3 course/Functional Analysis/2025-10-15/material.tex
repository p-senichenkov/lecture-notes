\chapter{Линейные пространства}

\section{Конечномерные пространства}

\subsection{Изоморфность конечномерных пространств одинаковой размерности}

\begin{theorem}
	$ (X, \|\cdot\|), ~ (Y, \|\cdot\|) $ "--- линейные над $ \R $ (или $ \Co $), $ \quad
	\dim X = \dim Y = n \in \N $.

	Тогда $ X $ линейно изоморфно $ Y $.
\end{theorem}

\begin{proof}
	Рассмотрим $ Z = l_n^2 = (\R^n, \|\cdot\|_2) $.
	Докажем, что $ l_n^2 $ и $ X $ линейно изоморфны.
	Этого достаточно в силу транзитивности.

	Пусть $ \set{f_j}_{j = 1}^n $ "--- базис в $ X, \quad
	e_j = (0, \dots, 0, \underset j 1, 0, \dots, 0) $ "--- базис в $ l_n^2 $.

	Определим $ A : l_n^2 \to X : \quad Ae_j = f_j $.
	$$ A \Bigl( \sum_{j = 1}^n c_je_j \Bigr) \define \sum_{j = 1}^n c_jf_j \implies
	A \in \mathscr Lin(l_n^2, X) $$
	Понятно, что $ A $ "--- биекция.
	\begin{itemize}
		\item Проверим непрерывность
			$$ z \in l_n^2, ~ z = \sum_{j = 1}^n c_je_j \quad \|A(z)\| =
			\Bigl\| \sum_{j = 1}^n c_jf_j\|_X \le \sum_{j = 1}^n |c_j| \cdot \|f_j\| \underset{\text{КБ}}\le
			\Bigl( \sum_{j = 1}^n |c_j|^2 \Bigr)^{\frac12} \Bigl( \sum_{j = 1}^n \|f_j\|^2 \Bigr)^{\frac12} =
			M\|z\|_{l_n^2} $$
			$$ \implies A \in \mathscr B(l_n^2, X), \quad \|A\| \le M $$
		\item Найдём $ c > 0 : \quad \|Az\| \ge c\|z\| \quad \forall z \in l_n^2 $
			$$ g(z) \define \|Az\|, \quad z \in l_n^2 $$
			$ g(z) $ непрерывна на $ l_n^2 $.
			$$ S \define \set{z \in l_n^2 | \|z\| = 1} $$
			$ S $ "--- компакт в $ l_n^2 $.
			$$ \exists \min\limits_{z \in S} g(z) = g(z_0) = r > 0 \implies
			\forall z \in S \quad \|Az\| \ge r $$
			Возьмём $ u \in l_n^2 \ne 0 $.
			$$ \frac u{\|u\|} \in S \implies
			\Bigl\| A \Bigl( \frac u{\|u\|} \Bigr)\Bigr\| \ge r \implies
			\|Au\| \ge r\|u\| \underimp{\text{критерий лин. изоморфности}}
			l_n^2 \text{ линейно изоморфно } X $$
	\end{itemize}
\end{proof}

\begin{implication}
	$ (X, \|\cdot\|), \quad \dim X < +\infty $

	\begin{enumerate}
		\item $ X $ "--- банахово;
		\item $ K \sub X $ "--- компакт $ \iff K $ ограничено и замкнуто;
		\item $ K \sub X $ относительно компактно $ \iff K $ ограничено.
	\end{enumerate}
\end{implication}

\begin{eproof}
\item $ l_n^2 $ "--- банахово $ \implies X $ "--- банахово.
\item $ K \sub X, \quad A : X \to l_n^2 $ "--- линейный изоморфизм $ \implies
	A, A^{-1} $ ограничены.

	$ K $ "--- компакт $ \implies \Bigl( A(K) $ "--- компакт $ \iff
	K $ ограничено и замкнуто $ \Bigr) $.
	$$ \implies K = A^{-1} \bigl( A(K) \bigr), \quad
	K \text{ ограничено и замкнуто} $$
\end{eproof}

\begin{theorem}
	$ (X, \|\cdot\|), \quad \dim X < +\infty, \quad (Y, \|\cdot\|) $

	$$ \mathscr Lin(X, Y) = \mathscr B(X, Y) $$
\end{theorem}

\begin{iproof}
\item Пусть $ T \in \mathscr Lin(l_n^2, X), \quad z \in l_n^2 $.
	$$ z = \sum_{j = 1}^n c_je_j, \quad e_j = (0, \dots, 0, \underset j 1, 0 \dots, 0) $$
	\begin{multline*}
		\|Tz\| = \Bigl\| \sum_{j = 1}^n c_j Te_j \Bigr\| \le
		\sum_{j = 1}^n |c_j| \|Te_j\| \underset{\text{К"--~Б}}\le
		\Bigl( \sum_{j = 1}^n |c_j|^2 \Bigr)^{\frac12} \Bigl( \sum_{j = 1}^n \|Te_j\|^2 \Bigr)^{\frac12} \implies \\
		\implies \|Tz\| \le M \|z\| \implies
		T \in \mathscr B(l_n^2, X)
	\end{multline*}
\item $ U \in \mathscr Lin(X, Y) $

	Пусть $ l_n^2 \overarr A X \overarr U Y, \quad A $ "--- линейный изоморфизм.
	Положим $ T = UA $.
	$$ \implies T \in \mathscr Lin(l_n^2, Y) = \mathrm B(l_n^2, Y) \implies
	T \in \mathscr B(l_n^2, Y) $$
	$$ U = TA^{-1} \implies U \in \mathscr B(X, Y) $$
\end{iproof}

\section{Конечномерные подпространства}

\begin{definition}
	$ (X, \rho) $ "--- метрическое пространство, $ \quad Y \sub X, \quad a \in X $

	$$ \rho(a, Y) = \inf\limits_{y \in Y} \rho(a, y) $$
	Если $ \exists y_0 \in Y : \quad \rho(a, Y) = \rho(a, y_0) $, то $ y_0 $ называется
	\emph{элементом наилучшего приближения}.
\end{definition}

\begin{remark}
	Если $ Y $ "--- компакт, то элемент наилучшего приближения существует
	(\as $ \rho(a, y) $ непрерывна).
\end{remark}

\begin{theorem}
	$ (X, \|\cdot\|) $
	\begin{enumerate}
		\item $ L \in X, \quad L $ "--- конечномерное подпространство в алгебраическом смысле
			$ \implies L $ замкнуто;
		\item $ a \in X \implies $ в $ L $ существует элемент наилучшего приближения.
	\end{enumerate}
\end{theorem}

\begin{eproof}
\item $ \dim L < +\infty \implies L $ "--- банахово $ \implies L $ замкнуто.
\item $ a \in X, \quad f = \rho(a, L) = \inf\|a - y\| $
	$$ \exists \set{y_n}_{n = 1}^\infty : \quad d \le \|a - y_n\| \le d + \frac1n $$
	Проверим, что $ \set{y_n} $ ограничена:
	$$ \|y_n\| \trile \|a\| + \|y_n - a\| \le \|a\| + d + 1 $$
	$$ \dim L < +\infty \implies
	\set{y_n}_{n = 1}^\infty \text{ относительно компактна } \implies
	\exists \set{y_{n_j}} : \quad \exists \lim\limits_{j \to \infty} y_{n_j} = y_0 \in L $$
	$$ d \le \|a - y_{n_j}\| \le d + \frac1{n_j} \implies \|a - y_0\| = d $$
\end{eproof}

\begin{remark}
	Элемент наилучшего приближения не обязательно единственен.
\end{remark}

\begin{exmpls}
	% Set clashes with \| and |
\item $ l_2^\infty = \{(x, y), \|(x, y)\| = \max\{|x|, |y|\}\} $
	$ \mathtt B_1(0, 0) $ выглядит в этом пространстве как квадрат.

	$ L = \set{y = kx}, \quad k \ne 0 $.
	Для $ L $ существует единственный элемент наилучшего приближения.
\item $ L = \set{y = 0} $.
	Для $ L $ элемент наилучшего приближения не единственен.
\item $ l_2^1 = \{ (x, y), \|(x, y)\| = |x| + |y|\} $
	$ B_1(0, 0) $ выглядит как квадрат, повёрнутый на $ \frac\pi4 $.

	$ L = \set{y = kx}, \quad k \ne \pm 1 $.
	Для $ L $ существует единственный элемент наилучшего приближения.
\item $ L = \set{y = x} $.
	Для $ L $ элементов наилучшего приближения бесконечно много.
\end{exmpls}

\begin{implication}
	$ \mathcal C[a, b], \quad \|f\|_\infty = \max|f(x)|, \quad
	\mathscr P_n = \set{\sum_{j = 0}^n a_jx^j, \quad a_j \in \R} $

	$$ \exists p \in \mathscr P_n : \quad \inf\limits_{q \in \mathscr P_n}\|f - q\|_\infty = \|f - p\| $$
	$ p $ называется \emph{многочленом наилучшего приближения}
\end{implication}

\begin{remark}
	Для $ \mathscr P_n $ существует единственный элемент наилучшего приближения.
\end{remark}

\section{Добавление в параграф о конечномерных пространствах}

\begin{implication}
	$ \dim X = n \in \N, \quad \|\cdot\|_1, ~ \|\cdot\|_2 $ "--- нормы на $ X $.

	Тогда $ \|\cdot\|_1 $ и $ \|\cdot\|_2 $ эквивалентны.
\end{implication}

\section{Почти ортогональные элементы}

\subsection{Лемма Рисса о почти ортогональном элементе}

\begin{lemma}
	$ (X, \|\cdot\|), \quad L \subsetneq X $ "--- подпространство, $ L = \ol L, \quad 0 < \eps < 1 $

	$$ \exists x_0 : \quad \|x_0\| = 1, \quad \rho(x_0, L) > 1 - \eps $$
\end{lemma}

\begin{proof}
	Возьмём $ z \in X \setminus L $.
	$$ \rho(z, L) = d > 0 \quad \text{ (\as $ L = \ol L $)} $$
	$$ \rho(z, L) = \inf\limits_{y \in L} \|z - y\| $$
	$$ \exists y \in L : \quad d \le \|z - y\| < \frac d{1 - \eps} $$
	Выберем $ x_0 = \frac{z - y}{\|z - y\|} $.
	Возьмём $ u \in L $.
	$$ \|x_0 - u\| = \Bigl\| \frac{z - y}{\|z - y\|} - u \Bigr\| =
	\frac{\bigl\|z - y - u \cdot \|z - y\| \bigr\|}{\|z - y\|} \ge
	\frac d {\frac d {1 - \eps}} = 1 - \eps $$
\end{proof}

\begin{remark}
	Если $ \exists y_0 : \quad \rho(z, y_0) = d $, то $ x_0 = \frac{z - y_0}{\|z - y_0\|} \implies
	\|x_0 - u\| \ge 1 $
\end{remark}
