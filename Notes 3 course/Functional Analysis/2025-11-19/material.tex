\chapter{Линейные функционалы}

\section{Обратные операторы}

\begin{quote}
	\raggedleft
	Когда существует непрерывный обратный оператор?
\end{quote}

\subsection{Обратимость оператора, близкого к тождественному}

\begin{statement}\label{inv:st:1}
	$ A, B \in \mathscr B(X) $

	$$ \|AB\| \le \|A\| \cdot \|B\| $$
\end{statement}

\begin{proof}
	$$ \|A(Bx)\| \le \|A\| \cdot \|Bx\| \le \|A\| \cdot \|B\| \cdot \|x\| $$
\end{proof}

\begin{theorem}
	$ X $ "--- банахово, $ \quad I $ "--- тождественный, $ \quad A \in \mathscr B(X), \quad
	\|A\| < 1 $

	$$ \exists (I - A)^{-1}, \quad \|(I - A)^{-1}\| \le \frac1{1 - \|A\|}, \quad
	(I - A)^{-1} = \sum_{k = 0}^\infty A^k $$
\end{theorem}

\begin{proof}
	Проверим, что сумма норм операторов сходится.
	По \autoref{inv:st:1}
	$$ \|A^k\| \le \|A\|^k \implies \sum_{k = 0}^\infty \|A^k\| \le \sum_{k = 0}^\infty \|A\|^k =
	\frac1{1 - \|A\|} $$
	$ X $ "--- банахово $ \implies \mathscr B(X) $ "--- банахово.
	Воспользуемся критерием полноты:
	$$ \implies \exists S = \sum_{k = 0}^\infty A^k \in \mathscr B(X) $$
	$$ S_n \coloneq \sum_{k = 0}^n A^k $$
	$$ \|S_n\| \le \sum_{k = 0}^n \|A\|^k < \frac1{1 - \|A\|}, \quad
	\lim\limits_{n \to \infty} \|S - S_n\| = 0 $$
	$$ \implies \|S\| \le \frac1{1 - \|A\|} $$

	$$ (I - A)S_n = (I - A) \sum_{k = 0}^n A^k = \sum_{k = 0}^n A^k - \sum_{k = 0}^n A^{k + 1} =
	I - A^{n + 1} $$
	$$ \lim\limits_{n \to \infty} \|A^{n + 1}\| = 0 \implies \lim (I - A^{n + 1}) = I $$
	$$
	\begin{rcases}
		\begin{rcases}
			\lim(I - A)S_n = I \\
			\lim(I - A)S_n = (I - A)S
		\end{rcases} \implies (I - A)S = I \\
		S_n(I - A) = I - A^{n + 1} \implies S(I - A) = I
	\end{rcases} \implies S = (I - A)^{-1} $$
\end{proof}

\begin{remark}
	Если $ A \cdot B = I $, то \textbf{не обязательно} $ A = B^{-1} $.
\end{remark}

\begin{eg}
	$ l^2, \quad x \in l^2, \quad x = (x_1, x_2, \dotsc) $
	Рассмотрим \emph{операторы сдвига}:
	$$ Sx = (0, x_1, x_2, \dotsc), \quad Tx = (x_2, x_3, \dotsc) $$
	$$ (ST)x = (0, x_2, x_3, \dotsc), \quad (TS)x = x $$
\end{eg}

\begin{definition}
	$ (X, \|\cdot\|), ~ (Y, \|\cdot\|) $

	Определим \emph{множество обратимых операторов}:
	$$ \mathrm{In}(X, Y) = \Set{A \in \mathscr B(X, Y) | \exists A^{-1} \in \mathscr B(Y, X)} $$
\end{definition}

\subsection{Множество обратимых операторов открыто}

\begin{theorem}
	$ X $ "--- банахово, $ \quad (Y, \|\cdot\|) $
	\begin{enumerate}
		\item $ U \in \mathrm{In}(X, Y), \quad V \in \mathscr B(X, Y), $
			$$ \|U - V\| < \frac1{\|U^{-1}\|} $$

			$$ \implies V \in \mathrm{In}(X, Y), $$
			\begin{equ}1
				\|V^{-1}\| \le \frac{\|U^{-1}\|}{1 - \|U - V\| \cdot \|U^{-1}\|}
			\end{equ}
			\begin{equ}2
				\|U^{-1} - V^{-1}\| \le \frac{\|U^{-1}\|^2 \|U - V\|}{1 - \|U - V\| \cdot \|U^{-1}\|}
			\end{equ}
		\item $ \phi : \mathrm{In}(X, Y) \to \mathrm{In}(Y, X) : \quad \phi(A) = A^{-1} $

			Тогда $ \phi $ непрерывно.
	\end{enumerate}
\end{theorem}

\begin{remark}
	$ \mathtt B_{\frac1{\|U^{-1}\|}}(U) \sub \mathrm{In}(X, Y) \implies \mathrm{In}(X, Y) $ открыто.
\end{remark}

\begin{eproof}
\item Рассмотрим оператор $ W = U^{-1}(U - V) $.
	$$ \|W\| \le \|U^{-1}\| \cdot \|U - V\| < 1, \quad W \in \mathscr B(X) $$
	Можно применить предыдущую теорему:
	$$ \exists (I - W)^{-1}, \quad \|(I - W)^{-1}\| \le \frac1{1 - \|W\|} \le
	\frac1{1 - \|U^{-1}\| \|U - V\|} $$
	$$ W = U^{-1}(U - V) = I - U^{-1}V \implies I - W = U^{-1}V $$
	$$ V = U(U^{-1}V) = U(I - W) $$
	Каждый их них обратим, значит,
	$$ \exists V^{-1} = (I - W)^{-1}U^{-1} $$
	$$ \|V^{-1}\| \le \|U^{-1}\| \cdot \|(I - W)^{-1}\| \le
	\frac{\|U^{-1}\|}{1 - \|U^{-1}\| \cdot \|U - V\|} $$
	Получили неравенство \eref{1}.
	Получим из него неравенство \eref{2}:
	$$ U^{-1} - V^{-1} = U^{-1}(V - U)V^{-1} \implies
	\|U^{-1} - V^{-1}\| \le \|U^{-1}\| \|V - U\| \|V^{-1}\| \underset{\eref{1}}\le
	\frac{\|U^{-1}\|^2 \|U - V\|}{1 - \|U^{-1}\| \cdot \|U - V\|} $$
\item $ \phi(U) = U^{-1} $
	$$ \phi(U) - \phi(V) = U^{-1} - V^{-1} \underimp{\eref{2}}
	\lim\limits_{\|U - V\| \to 0}\|\phi(U) - \phi(V)\| = 0 $$
\end{eproof}

\section{Открытые отображения}

\begin{definition}
	$ (X, \tau_1), ~ (Y, \tau_2) $ "--- топологические пространства

	Отображение $ U : X \to Y $ \emph{открыто}, если оно переводит открытые множества в открытые:
	$$ G \sub X \text{ открыто } \implies U(G) \text{ открыто} $$
\end{definition}

\begin{remark}
	$ U $ непрерывно, если \textbf{прообраз} открытого множества открыт.
\end{remark}

\begin{statement}
	$ (X< \tau_1), ~ (Y, \tau_2) $ "--- топологические пространства, $ \quad
	U : X \to Y $ "--- биекция

	$$ U \text{ открыто } \iff U^{-1} \text{ непрерывно} $$
\end{statement}

\begin{proof}
	Очевидно.
\end{proof}

\subsection{Критерий открытости линейного оператора}

\begin{statement}
	$ (X, \|\cdot\|), ~ (Y, \|\cdot\|), \quad U \in \mathscr Lin(X, Y) $

	$$ U \text{ открыт } \iff \exists r > 0 : \quad
	\mathtt B_r(0) \sub U \bigl( \mathtt B_1(0) \bigr) $$
\end{statement}

\begin{iproof}
\item $ \implies $
	$$
	\begin{rcases}
		U(0) = 0 \\
		U \text{ открыто } \implies U \bigl( \mathtt B_1(0) \bigr) \text{ открыто }
	\end{rcases} \implies 0 \in U \bigl( \mathtt B_1(0) \bigr) \implies
	\exists r > 0 : \quad \mathtt B_r(0) \sub U \bigl( \mathtt B_1(0) \bigr) $$
\item $ \impliedby $

	Возьмём $ G \sub X $ "--- открытое.
	Пусть $ y_0 \in U(G) $.
	$$ \exists x_0 \in G : \quad Ux_0 = y_0 $$
	$$ \exists R > 0 : \quad \mathtt B_R(x_0) \sub G $$
	$$ \mathtt B_r(0) \sub U \bigl( \mathtt B_1(0) \bigr) \underimp{\text{линейность}}
	\mathtt B_{rR}(0) \sub U \bigl( \mathtt B_R(0) \bigr) $$
	Сдвинем на $ U(x_0) $:
	$$ \underbrace{Ux_0 + \mathtt B_{rR}(0)}_{\mathtt B_{rR}(Ux_0) \sub
	U(\mathtt B_R(x_0)) \sub U(G)} \sub
	Ux_0 + U \bigl( \mathtt B_R(0) \bigr) = U \bigl( x_0 + \mathtt B_R(0) \bigr) =
	U \bigl( \mathtt B_R(x_0) \bigr) $$
	$ \implies Ux_0 $ "--- внутренняя точка $ U(G) $.
\end{iproof}

\subsection{Необходимое условие открытости линейного отображения}

\begin{implication}
	$ (X, \|\cdot\|), ~ (Y, \|\cdot\|), \quad U \in \mathscr Lin(X, Y) $

	$$ U \text{ открыто } \implies U(X) = Y \text{, \ie $ U $ "--- сюръекция} $$
\end{implication}

\begin{proof}
	По критерию $ \exists r > 0 : \quad \mathtt B_r(0) \sub U \bigl( \mathtt B_1(0) \bigr) $.
	Возьмём $ y \in Y $.
	$$ \exists n \in \mathbb N : \quad \|y\| < nr \implies y \in \mathtt B_{nr}(0) \sub
	U \bigl( \mathtt B_n(0) \bigr) \implies y \in U(X) $$
\end{proof}

\subsection{Теорема об открытом отображении}

\begin{lemma}[редукция]
	$ X $ "--- банахово, $ \quad (Y, \|\cdot\|), \quad U \in \mathscr B(X, Y), \quad
	\exists r > 0 : \quad \mathtt B_r(0) \sub \ol{U \bigl( \mathtt B_1(0) \bigr)} $

	$$ \mathtt B_{\frac r2}(0) \sub U \bigl( \mathtt B_1(0) \bigr) $$
\end{lemma}

\begin{proof}
	Пусть $ y \in \mathtt B_{\frac r2}(0) $, \ie $ \|y\| < \frac r2 $.
	Докажем, что $ \exists x : \quad \|x\| < 1, \quad Ux = y $.
	$$ \mathtt B_r(0) \sub \ol{U \bigl( \mathtt B_1(0) \bigr)} \implies
	\forall k \in \mathbb N \quad \mathtt B_{\frac r {2^k}}(0) \sub
	\ol{U \bigl( \mathtt B_{\frac1{2^k}}(0) \bigr)} $$
	\begin{itemize}
		\item При $ k = 1 $
			$$ \|y\| < \frac r 2 \implies y \in \mathscr B_{\frac r 2}(0) \sub
			\ol{U \bigl( \mathtt B_{\frac12}(0) \bigr)} $$
			$$ \exists x_1 : \quad \|x_1\| < \frac12, \quad \|y - Ux_1\| < \frac r 4 $$
		\item $ k = 2 $
			$$ y - Ux_1 \in \mathtt B_{\frac r 4}(0) \sub
			\ol{U \bigl( \mathtt B_{\frac14}(0) \bigr)} \implies
			\exists x_2 : \quad \|x_2\| < \frac14, \quad \|y - Ux_1 - Ux_2\| < \frac r {2^3} $$
			$$ \dots $$
		\item Построим $ \Set{x_n}_{n = 1}^\infty $ такую, что
			$$ \|x_n\| < \frac1{2^n}, \quad \|y - Ux_1 - \dots - Ux_n\| < \frac r {2^{n + 1}} $$
	\end{itemize}
	$$ \sum_{n = 1}^\infty \|x_n\| < 1 \underimp{\text{критерий полноты}}
	\exists x = \sum_{n = 1}^\infty x_n \in X, \quad \|x\| < 1 $$
	$$ S_n \coloneq \sum_{k = 1}^n x_k $$
	$$
	\begin{rcases}
		\lim S_n = x \implies \lim US_n = Ux \\
		\lim \|y - US_n\| = 0 \implies \lim US_n = y
	\end{rcases} \implies Ux = y $$
\end{proof}

\begin{theorem}[Банах]
	$ X, Y $ "--- банаховы, $ \quad U \in \mathscr B(X, Y) $

	Если $ U(X) = Y $, то $ U $ открыто.
\end{theorem}

\begin{proof}
	Обозначим $ B = \mathtt B_1(0) $.
	$$ X = \bigcup_{n = 1}^\infty (nB) \implies U(X) = Y = \bigcup_{n = 1}^\infty U(nB) $$
	По теореме Бэра о категориях
	$$ \exists n_0 \in \mathbb N : \quad \operatorname{Int} \ol{U(n_0B)} \ne \emptyset $$
	При этом, $ U(n_0B) = n_0U(B) $.
	$$ \implies \operatorname{Int} \ol{U(B)} \ne \emptyset \implies
	\exists a \in Y ~ \exists y > 0 : \quad \mathtt B_r(a) \sub \ol{U(B)} $$

	Чтобы воспользоваться леммой, нужно сдвинуть точку $ a $ в 0.

	Возьмём $ z \in X : \quad \|z\| < r $.
	$$
	\begin{rcases}
		a + z \in \mathscr B_r(a) \sub \ol{U(B)} \\
		a \in \ol{U(B)} \implies -a \in \ol{U(B)}
	\end{rcases} \implies z = (a + z) + (-a) \in
	\ol{U(B)} + \ol{U(B)} \sub \ol{U(2B)} $$
	Воспользуемся подобием:
	$$ \mathtt B_{\frac r2} \sub \ol{U(B)} \underimp{\text{лемма}}
	\mathtt B_{\frac r4}(0) \sub U(B) $$
\end{proof}
