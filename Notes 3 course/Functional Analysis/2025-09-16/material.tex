\chapter{Пространства}

\section{Пространства Лебега}

\subsection{Похожие пространства последовательностей}

\begin{eg}
	$$ l^p = \set{x = \set{x_j}_{j = 1}^\infty, \quad x \in \R, \Co | \sum_{j = 1}^\infty |x_j|^p < +\infty} $$
	$$ \|x\|_p = \Bigl( \sum_{j = 1}^\infty |x_j|^p \Bigr)^{\frac1p} $$

	Докажем, что $ l^p $ "--- банахово:
	$$ T = \N, \quad \mu(j) = 1 \quad \forall j \in \N, \quad L^p(\N, \mu) = l^p, \quad f \in L^p(\N, \mu), \quad f(j) = x_j $$
\end{eg}

\begin{remark}
	\hfill
	\begin{enumerate}
		\item $ \set{\nder[m]x}_{m = 1}^\infty, \quad \nder[m]x \in l^p, \quad 1 \le p \le +\infty, \quad x \in l^p, \quad \lim\limits_{m \to \infty}\|\nder[m]x - x\|_p = 0 $
			$$ \implies \lim\limits_{m \to \infty}\nder[m]x_j = x_j \quad \forall j \in \N $$
		\item В другую сторону \textbf{неверно}.
	\end{enumerate}
\end{remark}

\begin{eg}[в другую сторону]
	Рассмотрим последовательность базисных элементов:
	$$ e_m = (0, \dots, 0, \underset m 1, 0, \dots, 0) $$
	$$ e_m = \set{\delta_j^m}_{j = 1}^\infty, \quad \delta_j^m =
	\begin{cases}
		1, \quad j = m \\
		0, \quad j \ne m
	\end{cases} $$
	$ \delta_j^m $ называются \emph{символами Кронекера}.

	Для любого фиксированного $ j $ $ \lim\limits_{m \to \infty} \delta_j^m = 0 $.
	Поккординатный предел равен $ \On $.

	$$ \|e_m - \On\|_p = 1 \quad \forall m \implies \|e_m - \On\|_p \not\to 0 $$
\end{eg}

\section{Не полные нормированные пространства}

\subsection{Пространство финитных последовательностей}

\begin{definition}
	$ F $ "--- множество \emph{финитных} последовательностей:
	$$ F = \set{x = (x_1, x_2, \dots, x_{N(x)}, 0, \dots), \quad x_j \in \R, \Co} $$

	$$ (F, \|\cdot\|_p), \quad 1 \le p \le +\infty $$
\end{definition}

\begin{statement}
	$ (F, \|\cdot\|_p) $ не полны.
\end{statement}

\begin{proof}
	При фиксированном $ p $ можно считать, что $ (F, \|\cdot\|_p) \sub l^p $ (подпространство в алгебраическом смысле).

	Проверим, что оно не замкнуто.
	Возьмём
	$$ x = \set{\frac1{2^k}}_{k = 1}^\infty, \quad x \in l^p \quad \forall 1 \le p \le +\infty $$
	$$ \sum_{k = 1}^\infty \frac1{2^{kp}} < +\infty $$
	Положим
	$$ \nder[m]x = \Bigl( \frac12, \frac1{2^2}, \dots, \frac1{2^m}, 0, \dots \Bigr) \in F $$
	\begin{itemize}
		\item $ 1 \le p < +\infty $
			$$ \|x - \nder[m]x\|_p = \Bigl( \sum_{k = m + 1}^\infty \frac1{2^{pk}} \Bigr)^{\frac1p} \underarr{m \to \infty} 0 $$
		\item $ p = +\infty $
			$$ \|x - \nder[m]x\|_\infty = \frac1{2^{m + 1}} \underarr{m \to \infty} 0 $$
	\end{itemize}
	Значит, $ F $ не замкнуто.
\end{proof}

\subsection{Пространство непрерывных функций на отрезке с интегральной нормой}

\begin{eg}
	$$ \Bigl( \mathcal C[a, b], \|f\|_p = \bigl( \int_a^b |f(x)|^p \di x \bigr)^{\frac1p} \Bigr), \quad 1 \le p < +\infty $$
\end{eg}

\begin{statement}
	$ (\mathcal C[a, b], \|\cdot\|_p) $  "--- не полное нормированное пространство.
\end{statement}

\begin{proof}
	Возьмём $ f \in \mathcal C[a, b] $.
	$$ \Bigl( \int_a^b |f(x)|^p \di x \Bigr)^{\frac1p} = 0 \implies f(x) \equiv[{[a, b]}] 0 \implies \|\cdot\|_p \text{  "--- норма} $$
	Рассмотрим $ \mathrm L^p[a, b], ~ \lambda $ "--- классическая мера Лебега.

	Докажем, что $ (\mathcal C[a, b], \|\cdot\|_p) $ "--- подпространство $ (\mathrm L^p, \lambda) $ в алгебраическом смысле.
	$$ f \in \mathcal C[a, b] \to \set{g \in \mathrm L^p [a, b], \quad g(x) = f(x) ~ \ale \text{по } \lambda} $$

	Проверим, что $ C[a, b] $ не замкнуто.
	$$ f_n(x) =
	\begin{cases}
		0, \quad -1 \le x \le 0 \\
		nx, \quad 0 \le x \le \frac1n \\
		1, \quad \frac1n \le x \le 1 \\
	\end{cases}, \quad g(x) =
	\begin{cases}
		0, \quad -1 \le x \le 0 \\
		1, \quad 0 < x \le 1
	\end{cases} $$
	$$ \Bigl( \int_{-1}^1 |f_n(x) - g(x)|^p \di x \Bigr)^{\frac1p} \underarr{n \to \infty} 0 $$

	Докажем, что $ \not\exists f \in \mathcal C[-1, 1] $ такой, что $ f(x) = g(x) $ \ale на $ [-1, 1] $.

	\textbf{От противного.}
	Пусть такая $ f $ существует.
	$$ \int_{-1}^1 |f(x) - g(x)|^p \di \lambda = 0 \implies
	\begin{cases}
		\int_{-1}^0 |f(x)|^p \di \lambda = 0 \implies f(x) \equiv[{[-1, 0]}] 0 \\
		\int_0^1 |f(x) - 1|^p \di \lambda = 0 \implies f(x) \equiv[{[0, 1]}] 1
	\end{cases} \text{  "--- \contra} $$
\end{proof}

\subsection{Пространство многочленов}

\begin{eg}
	$$ \mathscr P = \set{p(x) = \sum_{k = 0}^n a_kx^k, \quad a_k \in \R, \quad n \ge 0}, \quad \forall [a, b] $$
	$$ \|p(x)\|_\infty = \max\limits_{x \in [a, b]}|p(x)|, \quad \mathscr P \sub \mathcal C[a, b] $$
\end{eg}

\begin{statement}
	$ (\mathscr P, \|\cdot\|_\infty) $ не полно.
\end{statement}

\begin{proof}
	Ясно, что $ \mathscr P $ "--- подпространство в алгебраическом смысле.
	Докажем, что оно не замкнуто.

	$$ e^x \notin \mathscr P, \quad \as \nder{(e^x)} = e^x \not\equiv 0 $$
	$$ P_n(x) = \sum_{k = 0}^n \frac{x^k}{k!}, \quad \lim\limits_{n \to \infty}\max\limits_{x \in [a, b]}|e^x - P_n(x)| = 0 $$
	$$ \lim\limits_{n \to \infty} \|e^x - P_n\|_\infty = 0 \implies \mathscr P \text{ не замкнуто} $$
\end{proof}

\section{Пополнение метрического пространства}

\begin{theorem}[простейшие свойства метрики]
	$ (X, \rho) $ "--- метрическое пространство.

	\begin{enumerate}
		\item $ x, y, z, u \in X \implies |\rho(x, y) - \rho(z, u)| \le \rho(x, u) + \rho(y, z) $;
		\item $ \rho : X \times X \to \R $ непрерывна (как функция двух переменных);
		\item $ A \sub X, \quad \rho(x, A) \define \inf\limits_{a \in A} \rho(x, a) $

			При фиксированном $ A $ функция $ \rho(x, A) $ непрерывна по $ x $;
		\item $ A $ замкнуто, $ \quad x_0 \notin A \implies \rho(x_0, A) > 0 $.
	\end{enumerate}
\end{theorem}

\begin{eproof}
\item $ \rho(x, y) \trile \rho(y, z) + \rho(x, z) \trile \rho(y, z) + \rho(z, u) + \rho(x, u) $.
\item Пусть есть две последовательности такие, что $ \lim x_n = x, ~ \lim y_n = y $.
	Требуется проверить, что $ \lim \rho(x_n, y_n) = \rho(x, y) $.

	$$ |\rho(x, y) - \rho(x_n, y_n)| \underset1\le \underbrace{\rho(x, x_n)}_{\to 0} + \underbrace{\rho(y, y_n)}_{\to 0} $$
\item $ A \sub X, \quad x, z \in X, \quad y \in A, \quad y $ фиксирован.
	$$ \rho(x, A) \le \rho(x, y) \trile \rho(x, z) + \rho(z, y) $$
	В силу произвольности $ y $ можно взять точную нижнюю грань:
	$$ \rho(x, A) \le \rho(x, z) + \inf\limits_{y \in A}\rho(z, y) = \rho(x, z) + \rho(z, A) \implies \rho(x, A) - \rho(z, A) \le \rho(x, z) $$
	Аналогично, $ \rho(z, A) - \rho(x, A) \le \rho(x, z) $
	$$ \implies |\rho(x, A) - \rho(z, A)| \le \rho(x, z) $$
	Зафиксируем $ x $ и устремим к нему $ z $:
	$$ \lim\limits_{z \to x} \rho(z, A) = \rho(x, A) $$
\item $ A = \ol A \implies X \setminus A $ открыто.
	$$ x_0 \notin A \implies X \setminus A \implies \exists r > 0 : \quad \mathtt B_r(x_0) \in X \setminus A \implies \forall y \in A \quad \rho(x_0, y) \ge r \implies \rho(x_0, A) \ge r $$
\end{eproof}

\begin{definition}
	$ (X, \rho), ~ (Y, d) $ "--- метрические пространства, $ \quad T : X \to Y $.

	\begin{enumerate}
		\item $ T $ называется \emph{изометрическим вложением}, если оно сохраняет расстояния:
			$$ d(Tx, Tz) = \rho(x, z) \quad \forall x, z \in X $$
		\item $ T $ называется \emph{изометрией}, если $ T(X) = Y $ и $ d(Tx, Tz) = \rho(x, z) $.
			Говорят, что $ (X, \rho) $ и $ (Y, d) $ \emph{изометричны}.
	\end{enumerate}
\end{definition}

\begin{props}
\item $ T : X \to Y $ "--- изометрическое вложение.

	Тогда $ T $ инъективно и непрерывно.
\item $ T $ "--- изометрия.

	Тогда существует $ T^{-1} : Y \to X $ "--- изометрия.
\item Изометрия "--- отношение эквивалентности на множестве метрических пространств.
\item $ T $ "--- изометрическое вложение, $ \quad Z = T(X) $.

	Тогда $ T $ "--- изометрия $ X $ и $ Z $.
\end{props}

\begin{eproof}
\item Пусть $ x, z \in X, \quad Tx = Tz $.
	$$ 0 = d(Tx, Tz) = \rho(x, z) \implies x = z $$
	$$ \lim\limits_{n \to \infty} x_n = x \implies \lim \rho(x_n, x) = 0 \implies \lim d(Tx_n, Tx) = 0 $$
\item Очевидно.
\item Следует из предыдущих.
\item Очевидно.
\end{eproof}

\begin{definition}
	$ (X, \rho) $ "--- метрическое пространство, $ \quad (Z, d) $ "--- полное, $ \quad \exists T : X \to Z : \quad T $ "--- изометрическое вложение и $ \ol{T(X)} = Z $.

	Будем говорить, что $ (Z, d) $ "--- \emph{пополнение} $ (X, \rho) $.
\end{definition}

\begin{remark}
	$ (U, d) $ "--- полное, $ \quad T : X \to U $ "--- изометрическое вложение.

	Определим $ Z = \ol{T(X)} $ (в $ U $).
	Замкнутое пространство полно, поэтому $ Z $ будет пополнением.
\end{remark}

\begin{theorem}[о пополнении метрического пространства]
	$ (X, \rho) $ "--- метрическое пространство.

	Тогда $ \exists (Z, d) $ "--- пополнение.
\end{theorem}

\begin{note}
	Есть естественное доказательство, но оно муторное.
	Мы же докажем коротко, но неестественно.
\end{note}
