\chapter{Ряды Фурье}

\section{Разложение в ряд Фурье элемента из замыкания линейной оболочки ОНС}

\begin{remark}
	$ (X, \|\cdot\|), \quad \set{e_n}_{n = 1}^\infty $ "--- базис.

	Тогда $ \set{e_n}_{n = 1}^\infty $ "--- полная система (\ie $ \ol{\mathscr L\set{e_n}} = X $).
\end{remark}

\begin{remark}
	$ (X, \|\cdot\|), \quad \set{e_n}_{n = 1}^\infty $ "--- базис.

	Тогда $ X $ сепарабельно.
\end{remark}

\begin{implication}
	$ H $ "--- гильбертово, $ \quad \set{e_n}_{n = 1}^\infty $ "--- полная ОС.

	Тогда $ \set{e_n}_{n = 1}^\infty $ "--- базис.
\end{implication}

\begin{proof}
	Уже доказано, что
	$$ x \in H, \quad x \in \ol{\mathscr L\set{e_n}} \quad
	\implies x = \sum_{n = 1}^\infty (x, e_n)e_n $$

	Проверим единственность разложения
	$$ x = \sum_{j = 1}^\infty \alpha_j e_j $$
	Рассмотрим частичную сумму:
	$$ \sigma_n = \sum_{j = 1}^n \alpha_j e_j, \quad \lim \sigma_n = x $$
	$$ (\sigma_n, e_k) = \alpha_k \text{ при } n \ge k \implies
	\lim\limits_{n \to \infty}(\sigma_n, e_k) = \alpha_k $$

	При этом,
	$$ \lim(\sigma_n, e_k) \undereq{\text{непр. ск. произв.}} (x, e_k) \implies
	\alpha_k = (x, e_k) $$
\end{proof}

\begin{theorem}
	$ H $ "--- сепарабельное гильбертово пространство.

	Тогда в $ H $ существует ОНБ.
\end{theorem}

\begin{proof}
	Существует всюду плотное $ \set{x_n}_{n = 1}^\infty $, \ie $ \ol{\set{x_n}} = H $.
	Значит, $ \set{x_n}_{n = 1}^\infty $ "--- полная, \ie $ \ol{\mathscr L\set{x_n}} = H $.

	Положим $ x_1 = \dots = x_{n_1 - 1} = 0, \quad x_{n_1} \ne 0, \quad
	z_1 \coloneq x_{n_1}, \quad L_1 = \mathscr L\set{z_1} $.
	$$ L_1 = \mathscr L \set{x_1, \dots, x_{n_1}} $$
	$$ x_{n_1 + 1} \in L_1, \dots, x_{n_2 - 1} \in L_1, \quad x_{n_2} \notin L_1, \quad
	z_2 \coloneq x_{n_2}, \quad L_2 = \mathscr L\set{z_1, z_2} $$
	И так далее по индукции.
	$$ z_1, z_2, \dots, z_m, \quad z_m = x_{n_m} $$
	\begin{itemize}
		\item Если $ x_k \in L_m \quad \forall k > m $, то $ \dim H = m $.
		\item Если $ x_{n_m + 1}, \dots, x_{n_{m + 1} - 1} \in L_m, \quad
			\exists x_{n_{m + 1}} \notin L_m $, то $ z_{m + 1} \coloneq x_{n_{m + 1}} $.
	\end{itemize}

	$ \set{z_m}_{m = 1}^\infty $ ЛНЗ
	$$ \mathscr L\set{z_1, \dots, z_n} = \mathscr L \set{x_1, \dots, x_{n_m}} \implies
	\mathscr L\set{z_m} = \mathscr L\set{x_n} $$
	Значит, $ \set{z_m} $ "--- полная ЛНЗ система.
	Применим к ней процесс ортогонализации Грама"--~Шмидта, получим набор $ \set{e_j} $.
	$$ \mathscr L\set{e_j} = \mathscr L\set{z_j} \implies
	\set{e_j} \text{ "--- полная ОНС} $$
\end{proof}

\begin{undefthm}{Проблема Банаха (1936).}
	$ X $ "--- сепарабельное банахово пространство.
	Верно ли, что в нём существует базис?
\end{undefthm}

П. Энфло в 1973 году построил контрпример.

\begin{theorem}
	Все бесконечномерные сепарабельные гильбертовы пространства линейно изометрически (с
	сохранением нормы) изоморфны друг другу.
\end{theorem}

\begin{proof}
	$ H $ "--- сепарабельное гильбертово пространство.
	Докажем, что оно линейно изометрически изоморфно $ l^2 $.

	Известно, что в $ H $ существует ОНБ $ \set{f_n}_{n = 1}^\infty $.
	$$ x \in H \quad x = \sum_{n = 1}^\infty (x, f_n)f_n $$
	Определим оператор $ T : H \to l^2 $:
	$$ Tx = \set{(x, e_n)}_{n = 1}^\infty $$
	В $ l^2 $ мы попадаем в силу равенства Парсеваля:
	$$ \|x\|^2 = \sum_{n = 1}^\infty |(x, e_n)|^2 \implies
	\begin{cases}
		Tx \in l^2, \\
		T \text{ "--- изометрия, \ie~} \|Tx\|_2 = \|x\|_H.
	\end{cases} $$

	Из свойств скалярного произведения очевидно, что $ T \in \mathscr Lin(H, l^2) $.
	$$ \|Tx\| = \|x\| \implies \|T\| = 1, \quad T \in \mathscr B(H, l^2) $$

	Проверили, что $ T(H) \sub l^2 $.
	Докажем равенство.

	Возьмём $ y = \set{y_n}_{n = 1}^\infty \in l^2 $.
	$$ \sum_{n = 1}^\infty |y_n|^2 = \|y\|_2^2 < +\infty $$
	$$ g_n \coloneq \sum_{j = 1}^n y_j f_j \in H $$
	Проверим, что $ \set{g_n} $ фундаментальна.

	Применим критерий Коши к $ \sum |y_n|^2 < +\infty $:
	$$ \forall \eps > 0 \quad \exists N : \quad \forall m > n \ge N \quad
	\|y_{n + 1}\|^2 + \dots + \|y_m\|^2 < \eps $$
	$$ \|g_m - g_n\|^2 = \Bigl\| \sum_{j = n + 1}^m y_jf_j \Bigr\|^2 =
	\sum_{j = n + 1}^m |y_j|^2 < \eps $$
	Значит, $ \set{g_n} $ фундаментальна.

	$$ \underimp{H \text{ "--- гильбертово }} \exists g = \lim g_n \implies
	g = \sum_{j = 1}^\infty y_jf_j \underimp{\text{разложение единственно}}
	e_j = (g, f_j) \implies T(g) = y $$
\end{proof}

\section{Классические ряды Фурье}

\begin{theorem}[Вейерштрасс]
	$ f \in \vawe{\mathcal C}[-\pi, \pi] $, \ie $ f(-\pi) = f(\pi), ~ f \in \mathcal C[-\pi, \pi] $.

	$$ \forall \eps > 0 \quad
	\exists T_n(x) = \alpha_0 + \sum_{k = 0}^n \bigl( \alpha_k \cos(kx) + \beta_k \sin (kx) \bigr)
	: \quad \|f - T_n\|_\infty < \eps $$
\end{theorem}

\begin{noproof}
\end{noproof}

\begin{exmpls}
\item $ \mathrm L_\R^2[-\pi, \pi], \quad dx $ "--- мера Лебега
	$$ (f, g) = \int_{-\pi}^\pi f(x)g(x) \di x $$

	Докажем, что $ \set{1, \cos nx, \sin nx}_{n = 1}^\infty $ "--- полная ОС.

	$$ \int_{-\pi}^\pi \cos(nx) \sin (mx) \di x = 0, \quad
	\int_{-\pi}^\pi \cos(nx) \cos (mx) \di x =
	\begin{cases}
		0, \quad n \ne m, \\
		\pi, \quad n = m.
	\end{cases} $$
	$$ \int_{-\pi}^\pi 1 \di x = 2\pi $$

	$ f \in \mathrm L^2[-\pi, \pi], \quad \mathcal C[-\pi, \pi] $ плотно в $ \mathrm L^2[-\pi, \pi] $
	$$ \exists g \in \mathcal C[-\pi, \pi] : \quad \|f - g\|_2 < \eps $$

	Утверждается, что $ g $ можно сделать $ 2\pi $-периодической, не сильно испортив её норму
	в $ l^2 $.
	$$ \exists \delta > 0 : \quad
	\Bigl( \in_{\pi - \delta}^\pi |g(x)|^2 \di x \Bigr)^{\frac12} < \eps $$
	$$ h(x) \coloneq
	\begin{cases}
		g(x), \quad x \in [-\pi, \pi - \delta], \\
		h(\pi) = g(-\pi), \\
		\text{по непрерывности иначе}
	\end{cases} $$
	$$ \exists T_n : \quad \|h - T_n\|_\infty < \eps \implies
	\Bigl( \int_{-\pi}^\pi \underbrace{|h(x) - T_n(x)|^2}_{< \eps} \di x \Bigr)^{\frac12} \le
	\eps \sqrt{2\pi} $$
	$$ \|f - h\|_2 \le \|f - g\|_2 + \|g - h\|^2 $$
	$$ \|f - T_n\|_2 \le \underbrace{\|f - g\|_2}_{< \eps} + \|g - T_n\|_2 <
	\eps + \|g - h\|_2 + \|h - T_n\| + \|h - T_n\|_2 < 3\eps + \eps \sqrt{2\pi} $$

	Базис:
	$$ \set{\frac1{\sqrt{2\pi}}, \frac{\cos(nx)}{\sqrt\pi}, \frac{\sin(nx)}{\sqrt\pi}}_{n \in \N} $$
	Коэффициенты Фурье:
	$$ a_k = \frac1\pi \int_{-\pi}^\pi f(x) \cos (kx) \di x, \quad
	b_k = \frac1\pi \int_{-\pi}^\pi f(x) \sin (kx) \di x, \quad k \in \N $$
	$$ a_0 = \frac1{2\pi} \int_{-\pi}^\pi f(x) \di x $$
	$$ S_n(x) = a_0 + \sum_{k = 1}^n \bigl( a_k \cos(kx) + b_k \sin(kx) \bigr) $$

	Ряд Фурье сходится в функции в пространстве $ \mathrm L^2 $:
	$$ \forall f \in \mathrm L^2 [-\pi, \pi] \quad \|f - S_n\|_2 \underarr{n \to \infty} 0 $$
	То есть,
	$$ \lim \Bigl( \int_{-\pi}^\pi |f(x) - S_n(x)|^2 \di x \Bigr)^{\frac12} = 0 $$
\item $ \mathrm L_\Co^2 [-\pi, \pi], \quad f \in \mathrm L_\Co^2 [-\pi, \pi] $
	$$ f = u + \ii v, \quad u, v \in \mathrm L_\R^2 [-\pi, \pi] $$

	Аналогично, $ \set{1, \cos(nx), \sin(nx)}_{n \in \N} $ "--- полная ОНС в $ \mathrm L_\Co^2 $.
\item $ \mathrm L_\Co^2 [-\pi, \pi] $

	Докажем, что $ \set{e^{\ii nx}}_{n \in \Z} $ "--- полная ОС в $ \mathrm L_\Co^2 $.

	$$ (f, g) = \int_{-\pi}^\pi f(x) \ol{g(x)} \di x, \quad
	\in_{-\pi}^\pi e^{\ii nx}e^{-\ii mx} \di x =
	\begin{cases}
		0, \quad n \ne m, \\
		2\pi, \quad n = m
	\end{cases} $$

	Получаем коэффициенты:
	$$ c_n = \frac1{2\pi} \in_{-\pi}^\pi f(x) e^{-\ii nx}\di x, \quad n \in \Z $$

	\begin{itemize}
		\item $ c_0 = a_0 $;
		\item $ n \in \N $
			$$ c_n = \frac12(a_n - \ii b_n); $$
		\item $ n \in \N $
			$$ c_{-n} = \frac12(a_n + \ii b_n). $$
	\end{itemize}

	$$ S_n \coloneq \sum_{k = -n}^n c_ke^{\ii kx} =
	a_0 + \sum \bigl( a_k \cos(kx) + b_k \sin(kx) \bigr) $$

	ОНБ:
	$$ \set{\frac{e^{\ii nx}}{\sqrt{2\pi}}}_{n \in \Z} $$
\item $ \mathrm L_\R^2[0, \pi] $

	Докажем, что $ \set{1, \cos(nx)}_{n \in \Z} $ "--- ОС в $ \mathrm L^2 $.

	$$ f \in \mathrm L_\R^2[0, \pi], \quad f(-x) \coloneq f(x) \implies
	f \in \mathrm L^2[-\pi, \pi] $$
	$$ \|f - S_n\|_2 \underarr{n \to \infty} 0 $$
	$$ b_k = \frac1\pi \int_{-\pi}^\pi f(x) \sin(kx) \di x = 0 $$
	$$ \int_{-\pi}^\pi |f(x) - \sum a_k \cos(kx)|^2 \di x \to 0 \implies
	\int_0^\pi |f(x) - \sum a_k\cos(kx)|^2 \di x \to 0 $$
\end{exmpls}
