\chapter{Линейные функционалы}

\section{Эрмитово-сопряжённый оператор}

\subsection{Теорема о ядре и образе оператора и его сопряжённого}

\begin{theorem}
	$ H $ "--- гильбертово, $ \quad T \in \mathscr B(H), \quad Y $ "--- инвариантное
	подпространство для $ T $

	$$ Y^\perp \text{ "--- инвариантное подпространство для } T^* $$
\end{theorem}

\begin{proof}
	$ z \in Y^\perp, \quad y \in Y $
	$$ (y, T^*z) = (Ty, z) \undereq{
		\begin{subarray}{c}
			Ty \in Y \\
			z \in Y^\perp
		\end{subarray}
	} 0 \implies T^*z \in Y^\perp \implies T^*(Y^\perp) \sub Y^\perp $$
\end{proof}

\begin{definition}
	$ H $ "--- гильбертово, $ \quad T \in \mathscr B(H) $

	$ T $ называется \emph{самосопряжённым}, если $ T = T^* $, то есть
	$ (Tx, y) = (x, Ty) \quad \forall x \in H $.
\end{definition}

\begin{implication}
	$ H $ "--- гильбертово, $ \quad T \in \mathscr B(H), \quad T $ "--- самосопряжённый

	Если $ Y \sub H $ инвариантно для $ T $, то $ Y^\perp $ тоже инвариантно.
\end{implication}

\begin{eg}
	$ M = \ol M \sub H, \quad M $ "--- подпространство
	$$ P \text{ "--- ортопроектор на } M \implies P = P^* $$
\end{eg}

\section{Спектр оператора}

\begin{definition}
	$ X $ "--- банахово, $ T \in \mathscr B(X), \quad I $ "--- тождественный оператор, $ \quad
	\lambda \in \Co $

	Будем говорить, что $ \lambda $ "--- \emph{регулярная точка}, если
	$ V(\lambda) \coloneq \lambda I - T $ "--- биекция.

	По теореме Банаха об обратном операторе,
	$$ V^{-1}(\lambda) \in \mathscr B(X) $$
	$$ R(\lambda, T) = R(\lambda) = V^{-1}(\lambda) $$
	$ R $ называется \emph{резольвентой}.
\end{definition}

\begin{note}[касательно терминологии]
	Рассмотрим уравнение
	$$ \lambda x - Tx = h, $$
	где $ h $ "--- дано, $ x $ "--- неизвестные.

	Если $ \forall h \in X \exists ~ x \in X : \quad \lambda x - Tx = h $, то уравнение разрешимо.
	То есть, $ x = R(\lambda)h $.
\end{note}

\begin{definition}
	Множество всех регулярных точек называется \emph{рещольвентным множеством}:
	$$ \rho(T) = \Set{\lambda \in \Co | \lambda \text{ "--- регулярная}} $$
\end{definition}

\begin{definition}
	Множество всех остальных точек назовём \emph{спектром} $ T $:
	$$ \sigma(T) = \Co \setminus \rho(T) $$
\end{definition}

\begin{definition}[части спектра]
	\hfill
	\begin{enumerate}
		\item $ \sigma_p(T) $ "--- \emph{точечный спектр}
			$$ \sigma_p(T) = \Set{\lambda \in \Co | \lambda T - I \text{ не инъекция}} $$
			Для линейного оператора это означает, что
			$ X_\lambda = \operatorname{Ker}(\lambda I - T) \ne \Set{0} $.
			$$ u \ne 0 \in X \implies Tu = \lambda u $$
			$ u $ "--- собственный вектор $ T $, соответствующий с. ч. $ \lambda $.
			$ X_\lambda $ "--- собственное подпространство.
		\item $ \sigma_c(T) $ "--- \emph{непрерывный спектр}
			$$ \sigma_c(T) = \Set{\lambda \in \Co | \operatorname{Ker}(\lambda I - T) = \Set{0},
			\quad \ol{V(\lambda)(X)} = X} $$
			(\ie образ $ V(\lambda) $ всюду плотен в $ X $).
		\item $ \sigma_r(T) $ "--- \emph{остаточный спектр}
			$$ \sigma_r(T) = \Set{\lambda \in \Co | \operatorname{Ker}(\lambda I - T) = \Set{0},
			\quad \ol{V(\lambda)(X)} \subsetneq X} = \sigma \setminus (\sigma_p \cup \sigma_c) $$
	\end{enumerate}
\end{definition}

\begin{remark}
	В конечномерном случае $ \sigma = \sigma_p $.
\end{remark}

\subsection{Свойства резольвенты}

\begin{properties}
	$ X $ "--- банахово, $ \quad T \in \mathscr B(H) $
	\begin{enumerate}
		\item $ \lambda, \mu \in \rho(T) $
			$$ \implies R(\lambda) R(\mu) = R(\mu) R(\lambda) $$
		\item \emph{Тождество Гильберта}: $ \lambda, \mu \in \rho(T) $
			$$ \implies R(\lambda) - R(\mu) = (\mu - \lambda) R(\lambda) R(\mu) $$
		\item $ \rho(T) $ открыто в $ \Co $

			Кроме того, если $ \mu \in \rho(T) $, то
			$$ |\lambda - \mu| < \frac1{\|R(\mu)\|} \implies \lambda \in \rho(T) $$
		\item $ \lambda \in \Co, \quad |\lambda| > \|T\| $
			$$ \implies \lambda \in \rho(T) $$
		\item $ R(\lambda) $ "--- непрерывная функция, \ie если $ \mu \in \rho(T) $, то
			$$ \lim\limits_{\lambda \to \mu}R(\lambda) = R(\mu), \quad
			\lim\limits_{\lambda \to \infty} R(\lambda) = 0 $$
		\item $ F \in \bigl( \mathscr B(X) \bigr)^*, \quad \lambda \in \rho(T), \quad
			g(\lambda) \coloneq F \bigl( R(\lambda) \bigr) $
			$$ \implies g(\lambda) \text{ аналитична в } \rho(T), \quad
			\lim\limits_{\lambda \to \infty} g(\lambda) = 0 $$
	\end{enumerate}
\end{properties}

\begin{eproof}
\item $ \lambda, \mu \in \rho(T) $
	$$ V(\lambda) V(\mu) = (\lambda I - T)(\mu I - T) = V(\mu) V(\lambda) $$
	$$ \exists \bigl( V(\lambda) \bigr)^{-1}, ~ \bigl( V(\mu) \bigr)^{-1}, \quad
	R(\lambda) = \bigl( V(\lambda) \bigr)^{-1}, \quad R(\mu) = \bigl( V(\mu) \bigr)^{-1} $$
	$$ \implies \bigl( V(\mu) \bigr)^{-1} \bigl( V(\lambda) \bigr)^{-1} =
	\bigl( V(\lambda) \bigr)^{-1} \bigl( V(\mu) \bigr)^{-1} $$
\item $ \lambda, \mu \in \rho(T) $
	$$ V(\lambda) - V(\mu) = (\lambda I - T) - (\mu I - T) = (\lambda - \mu) I $$

	Если $ A, B \in \mathscr B(X), \quad \exists A^{-1}, B^{-1} $, то
	$$ A^{-1} - B^{-1} = A^{-1}(B - A)B^{-1} $$

	Возьмём $ A = V(\lambda), ~ B = V(\mu) $.
	$$ R(\lambda) - R(\mu) = R(\lambda) \bigl( (\mu - \lambda) I \bigr) R(\mu) =
	(\mu - \lambda)R(\lambda)R(\mu) $$
\item Известно, что $ \operatorname{In}(X) $ (множество обратимых операторов) открыто:
	$$ \exists A^{-1} \quad \|A - B\| < \frac1{\|A^{-1}\|} \implies \exists B^{-1} $$
	$$ \mu \in \rho(T), \quad A = V(\mu) \implies \exists R(\mu) = \bigl( V(\mu) \bigr)^{-1} $$
	$$ V(\lambda) - V(\mu) = (\lambda - \mu)I \implies
	\|V(\lambda) - V(\mu)\| = |\lambda - \mu| $$
	$$ |\lambda - \mu| < \frac1{\|R(\mu)\|} \implies \|V(\lambda) - V(\mu) \| < \frac1{\|R(\mu)\|}
	\implies \exists \bigl( V(\lambda) \bigr)^{-1} \implies \lambda \in \rho(T) $$
\item $ \lambda \in \Co, \quad |\lambda| > \|T\| $

	Рассмотрим оператор
	\begin{multline*}
		\Bigl\| \frac1\lambda T \Bigr\| < 1 \underimp{\text{т. об обр. опер., близкого к тожд.}}
		\exists \Bigl( I - \frac1\lambda T \Bigr)^{-1} \implies \\
		\implies V(\lambda) = \lambda I - T = \lambda \Bigl( I - \frac1\lambda T \Bigr) \implies
		\exists R(\lambda) = \bigl( V(\lambda) \bigr)^{-1} =
		\frac1\lambda \Bigl( I - \frac1\lambda T \Bigr)^{-1}
	\end{multline*}
\item $ \mu \in \rho(T) $
	$$ \lim\limits_{\lambda \to \mu} \bigl( V(\mu) - V(\lambda) \bigr) =
	\lim\limits_{\lambda \to \mu} (\mu - \lambda) I = 0 $$
	По теореме об открытости $ \mathrm{In}(A) $,
	$$ \phi : A \to A^{-1}, \quad A \in \mathrm{In}(X) \implies \phi \text{ непрерывно} $$
	$$
	\begin{rcases}
		\phi \bigl( V(\lambda) \bigr) = V(\lambda) \\
		\lim\limits_{\lambda \to \mu} V(\lambda) = V(\mu)
	\end{rcases} \implies \lim\limits_{\lambda \to \mu} R(\lambda) = R(\mu) $$

	Пусть $ |\lambda| > \|T\| $.
	$$ \lim\limits_{\lambda \to \infty} \Bigl( I - \frac1\lambda T \Bigr) = I $$
	$$ R(\lambda) = \frac1\lambda \Bigl( I - \frac1\lambda T \Bigr)^{-1} $$
	По непрерывности,
	$$ \lim \Bigl( I - \frac1\lambda T \Bigr)^{-1} = I \implies
	\lim R(\lambda) = 0 $$
\item $ \mu \in \rho(T), \quad \lambda $ из некоторой окрестности $ \mu $
	$$ \frac{R(\lambda) - R(\mu)}{\lambda - \mu} \undereq{\text{т-во Гильберта}}
	\frac{(\mu - \lambda)R(\lambda)R(\mu)}{\lambda - \mu)} = -R(\lambda)R(\mu) \implies
	\exists \lim\limits_{\lambda \to \mu} \frac{R(\lambda) - R(\mu)}{\lambda - \mu} =
	- \bigl( R(\mu) \bigr)^2 $$

	Возьмём $ F \in \bigl( \mathscr B(X) \bigr)^*, \quad F : \mathscr B(X) \to \Co $.
	Рассмотрим функцию $ g(\lambda) = F \bigl( R(\lambda) \bigr) $ при $ \lambda \in \rho(T) $.
	$$ \lim\limits_{\lambda \to \mu} \frac{g(\lambda) - g(\mu)}{\lambda - \mu} =
	\lim\limits_{\lambda \to \mu} \frac{ F \bigl( R(\lambda) \bigr) - F \bigl( R(\mu) \bigr)}
	{\lambda - \mu} \undereq{\text{лин. } F}
	\lim F \Bigl( \frac{R(\lambda) - R(\mu)}{\lambda - \mu} \Bigr) \undereq{\text{непр. } F}
	-F \Bigl( \bigl( R(\mu) \bigr)^2 \Bigr) $$
	То есть, $ \exists g'(\mu) \quad \forall \mu \in \rho(T) $.

	$$ \lim\limits_{\mu \to \infty} R(\mu) = 0 \underimp{\text{непр. } F}
	\lim\limits_{\mu \to \infty} F \Bigl( \bigl( R(\mu) \bigr)^2 \Bigr) = 0 $$
\end{eproof}

\subsection{Компактность и непустота спектра}

\begin{implication}
	$ X $ "--- банахово, $ \quad T \in \mathscr B(X) $

	$$ \sigma(T) \text{ "--- компакт}, \quad \sigma(T) \ne \emptyset $$
\end{implication}

\begin{eproof}
\item Компактность
	$$ \rho(T) \text{ открыто } \implies \sigma(T) \text{ замкнуто } $$
	$$ |\lambda| > \|T\| \implies \lambda \in \rho(T) $$
	Значит,
	$$ \lambda \in \sigma(T) \implies |\lambda| \le \|T\| $$
	То есть,
	$$ \sigma(T) \sub \Set{\lambda \in \Co | {} |\lambda| \le \|T\|} \implies
	\sigma(T) \text{ ограничено } \implies \sigma(T) \text{ "--- компакт} $$
\item Непустота

	\textbf{Пусть} $ \sigma(T) $ пусто.
	Тогда $ \rho(T) = \Co $, то есть
	$$ \forall F \in \bigl( \mathscr B(X) \bigr)^* \quad g(\lambda) = F \bigl( R(\lambda) \bigr)
	\text{ "--- аналитическая в } \Co \text{ (целая)} $$

	$$ V(0) = -T \implies \exists T^{-1} \in \mathscr B(X) \underimp{\text{сл. из т. Хана"--~Банаха}}
	\exists F \in \bigl( \mathscr B(X) \bigr) : \quad g(0) \ne 0 $$
	Выберем $ g(\lambda) = F \bigl( R(\lambda) \bigr) $.

	Таким образом, $ g(z) $ "--- целая, $ g $ ограничена.
	Значит, по теореме Лиувилля, $ g \equiv \const $ "--- \contra с $ \lim\limits_{g \to 0} = 0 $.
\end{eproof}

\begin{eg}
	$ Ix = x $
	$$ \sigma(I) = \Set{1} = \sigma_p(I) $$

	$$ (\lambda I - I) = (\lambda - 1)I \implies R(\lambda) = \frac1{\lambda - 1}I \quad
	\forall \lambda \ne 1 $$
\end{eg}

\subsection{Спектр и резольвента сопряжённого оператора}

\begin{theorem}
	\hfill
	\begin{enumerate}
		\item $ X $ "--- банахово, $ \quad T \in \mathscr B(X) $

			$$ \implies \sigma(T^*) = \sigma(T) $$

			Если $ \lambda \in \rho(T) $, то
			$$ \bigl( R(\lambda, T) \bigr)^* = R(\lambda, T^*) $$
		\item $ H $ "--- гильбертово, $ \quad T \in \mathscr B(H), \quad
			T $ "--- эрмитово-сопряжённый

			$$ \implies \sigma(T^*) = \Set{\lambda | \ol \lambda \in \sigma(T)} $$

			Если $ \lambda \in \rho(T) $, то
			$$ R(\lambda, T^*) = \bigl( R(\ol \lambda, T) \bigr)^* $$
	\end{enumerate}
\end{theorem}

\begin{eproof}
\item $ X $ "--- банахово, $ \quad \lambda \in \rho(T) $
	$$ V(\lambda) = \lambda I - T \implies \bigl( V(\lambda) \bigr)^* = \lambda I - T^* $$
	$$ \Bigl( \bigl( V(\lambda) \bigr)^{-1} \Bigr)^* =
	\Bigl( \bigl( V(\lambda) \bigr)^* \Bigr)^{-1} $$
\item $ X $ "--- гильбертово
	$$ \bigl( V(\lambda) \bigr)^* = \ol \lambda I - T^* $$
\end{eproof}
