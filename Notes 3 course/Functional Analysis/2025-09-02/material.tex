\chapter{Предисловие}

\section{Основные понятия}

\subsection{Объекты}


$ \R $ "--- алгебра, $ \quad \R $ "--- топологическое пространство.
Для математического анализа важны оба свойства:
\begin{itemize}
	\item пределы;
	\item непрерывность;
	\item $ f' $;
	\item $ \int f $.
\end{itemize}

$ X $ "--- линейное топологическое пространство.
Оно обладает свойствами линейного пространства:
\begin{itemize}
	\item $ (x, y) \to x + y $;
	\item $ (\alpha, x) \to \alpha x $;
\end{itemize}
и свойствами топологического пространства: \\
$ \set{U_\alpha}_{\alpha \in A} $.

Операции линейного пространства \textbf{непрерывны}.

Будем рассматривать нормированные пространства $ (X, \|\cdot\|) $.
В таком случае непрерывность гарантирована.

\subsection{Отображения}

$$ A : X \to Y $$
$ A $ "--- линейный непрерывный оператор:
$$ A(\alpha x + \beta z) = \alpha A x + \beta A z $$

Если $ \dim X < +\infty, \dim Y < +\infty $, то $ X $ "--- линейная алгебра.

Если $ A = A^* $, то существует ОНБ из с. в. $ A $.

\begin{theorem}[Гильберта"--~Шмидта]
	$ X $ "--- Н-сепарабельное гильбертово пространство, $ A = A^* $, $ A $ компактный.

	Тогда существует ОНБ из с. в.
\end{theorem}

\begin{definition}
	Если $ \dim Y = 1 $, \ie $ Y = \Co $ (или $ \R $), то $ A : X \to Y $ называется \emph{линейным функционалом}.
\end{definition}

\begin{undefthm}{Вопрос анализа.}
	Пусть $ f $ "--- функция, у неё имеются некоторые свойства.
	Какие свойства будут у её производной?
\end{undefthm}

\begin{undefthm}{Вопрос функционального анализа.}
	$ D : X \to Y, ~ D(f) = f' $ \\
	Какими свойствами обладает $ D $ (непрерывность, компактность, \dots)?
\end{undefthm}

\subsection{Почему полезно изучать функциональный анализ}

\begin{enumerate}
	\item Более общий взгляд на задачу.
		$$ f \in \mathcal C[a, b], ~ \mathscr P_n = \set{p(x) = \sum_{j = 0}^n a_jx^j, ~ a_j \in \R}, \quad E_n(f) = \inf_{p \in \mathscr P_n} \Bigl( \max_{\gamma \in [a, b]} |f(x) - p(x)| \Bigr) $$

		Существует ли $ p \in \mathscr P_n $ такое, что $ E_n(f) = |f(x) - p(x)|, \quad \dim \mathscr P_n = n + 1 < +\infty $? \\
		Единственный ли?
	\item Язык функционального анализа применим во всей математике.
	\item Математическая физика использует функциональный анализ.
	\item Функциональный анализ "--- это интересно и важно.
\end{enumerate}

\chapter{Метрические пространства}

\section{Определения и свойства}

\begin{definition}
	$ (X, \rho) $, $ X $ "--- множество, $ \rho : X \times X \to \R $, выполняются свойства:
	\begin{enumerate}
		\item $ \rho(x, y) \ge 0, \quad \rho(x, y) = 0 \iff x = y $;
		\item $ \rho(x, y) = \rho(y, x) $;
		\item $ \rho(x, y) \le \rho(x, z) + \rho(z, y) $.
	\end{enumerate}

	$ X $ называется \emph{метрическим пространством}.
\end{definition}

\begin{notation}
	$ \mathtt B_r(x) = \set{t \in X \mid \rho(x, y) < r} $
\end{notation}

$ \set{B_r(x)}_{r > 0} $ "--- база топологии в точке $ x $.
Если не фиксировать $ x $, получим базу топологии $ X $.

\begin{definition}
	$$ G \sub X, ~ G \text{ \emph{открыто} } \iff \forall x \in G \quad \exists \mathtt B_r(x) \sub G $$
\end{definition}

\begin{definition}
	$ F $ \emph{замкнуто}, если $ X \setminus F $ открыто.
\end{definition}

Для метрических пространств справедливо следующее определение \emph{замкнутого пространства}:
\begin{definition}
	$$ \set{x_n}_{n = 1}^\infty, \quad \lim\limits_{n \to \infty} x_n = x_0 \iff \lim\limits_{n \to \infty} \rho(x_n, x_0) = 0 $$
\end{definition}

\begin{definition}
	$ A \sub X $

	$ A $ \emph{ограничено}, если $ \exists \mathtt B_r(x_0) $ такой, что $ A \sub \mathtt B_r(x_0) $.
\end{definition}

\begin{definition}
	$ \set{x_n}_{n = 1}^\infty $ "--- \emph{фундаментальная}, если
	$$ \forall \eps > 0 \quad \exists N \in \N : \quad n, m > N \implies \rho(x_n, x_m) < \eps $$
\end{definition}

\begin{remark}
	Если $ \exists \lim\limits_{n \to \infty} x_n = a $, то $ \set{x_n}_{n = 1}^\infty $ "--- фундаментальная.
\end{remark}

\begin{proof}
	Возьмём $ \eps > 0 $.
	$$
	\begin{rcases}
		\exists N : \quad n > N \implies \rho(x_n, a) < \eps \\
		\text{ возьмём } m > N \implies \rho(x_m, a) < \eps
	\end{rcases} \implies \rho(x_n, x_m) \trile 2\eps $$
\end{proof}

\begin{definition}
	$ (X, \rho) $ "--- \emph{полное}, если любая фундаментальная последовательность имеет предел.
\end{definition}

\begin{remark}[о пользе полноты]
	$ (X, \rho) $ "--- метрическое пространство, $ F : X \to \R $ "--- непрерывное.
	Требуется определить $ x_0 : F(x_0) = 0 $.

	Построим последовательность $ \set{x_n}_{n = 1}^\infty $ такую, что
	$$ \lim\limits_{n \to \infty} F(x_n) = 0, \quad \rho(x_n, x_m) \underarr{
		\begin{subarray}{c}
			n \to \infty \\
			m \to \infty
		\end{subarray}
	} \quad \implies \exists x_0 : F(x_0) = 0, \quad \lim x_n = 0 $$
	Это верно только в полном пространстве.
\end{remark}

\begin{exmpls}
\item $ \R^n, \Co^n $ "--- полные;
\item $ \R^n \setminus \set 0, \Q $ "--- не полные.
\end{exmpls}

\subsection{Теорема о свойствах фундаментальных последовательностей}

\begin{theorem}
	$ (X, \rho), \quad \set{x_n}_{n = 1}^\infty $ "--- фундаментальная.

	\begin{enumerate}
		\item $ \set{x_n}_{n = 1}^\infty $ ограничена;
		\item если существует $ \set{x_{n_k}}_{k = 1}^\infty $ такая, что $ \lim x_{n_k} = a $, то $ \exists \lim x_n = a $;
		\item $ \set{\eps_k}_{k = 1}^\infty, ~ \eps_k > 0 \quad \implies \quad \exists \set{x_{n_k}}_{k = 1}^\infty : \quad \forall j > k \quad \rho(x_{n_j}, x_{n_k}) < \eps_k $.
	\end{enumerate}
\end{theorem}

\begin{eproof}
\item Возьмём $ \eps = 1 $.
	$$ \exists N \in \N : \quad \forall n, m \ge N \quad \rho(x_n, x_m) < 1 \quad \implies \rho(x_N, x_m) < 1 \text{ при } m > N $$
	Пусть $ R = \max\set{\rho(x_1, x_N), \dots, \rho(x_{N - 1}, x_N)} + 1 $.
	Тогда $ \forall n \in \N \quad x_n \in \mathtt B_R(x_N) $.
\item Возьмём $ \eps > 0 $.
	Воспользуемся фундаментальностью:
	$$ \exists N : \quad \forall n, m > N \quad \rho(x_n, x_m) < \eps $$
	Зафиксируем $ n_k $ такой, что:
	\begin{itemize}
		\item $ n_k > N $;
		\item $ \rho(x_{n_k}, a) < \eps $.
	\end{itemize}

	Пусть $ n > N $.
	Тогда $ \rho(x_n, a) \trile \rho(x_n, x_{n_k}) + \rho(x_{n_k}, a) < 2\eps \implies \lim \rho(x_n, a) = 0 $.
\item Докажем \textbf{по индукции}:
	\begin{itemize}
		\item \textbf{База.} $ \eps_1 $
			$$ \exists N_1 : \quad \forall n, m \ge N_1 \quad \rho(x_n, x_m) < \eps_1 \implies \rho(x_{N_1}, x_m) < \eps_1 \text{ при } m > N_1 $$
		\item \textbf{Переход.} Допустим, уже построены $ n_1, \dots, n_{k - 1} $, $ n_j \uparrow $ такие, что
			$$ \forall m > n_j \quad \rho(x_m, x_{n_j}) < \eps_j, \quad j = 1, 2, \dots, k - 1 $$
			\begin{multline*}
				\exists n_k > n_{k - 1} : \quad \forall n, m \ge n_k \quad \rho(x_n, x_m) < \eps_k \implies \rho(x_{n_k}, x_m) < \eps_k \text{ при } m > n_k \\
				\implies \exists \text{ требуемые } \set{x_{n_k}}
			\end{multline*}
	\end{itemize}
\end{eproof}
