\chapter{Пространства}

\section{Метрические пространства}

\begin{implication}
	$ (X, \rho), \quad \set{x_n}_{n = 1}^\infty $ "--- фундаментальная
	$$ \implies \exists \set{x_{n_k}}_{k = 1}^\infty : \quad \sum_{k = 1}^\infty \rho(x_{n_{k + 1}}, x_{n_k}) < +\infty $$
\end{implication}

\begin{proof}
	Следует из свойства 3 фундаментальных последовательностей
\end{proof}

\begin{theorem}[о замкнутом подпротранстве]
	$ (X, \rho), \quad Y \sub X $ ($ Y $ называется \emph{подпространством})
	\begin{enumerate}
		\item $ (X, \rho) $ "--- полное, $ Y $ замкнуто.

			Тогда $ (Y, \rho) $ полно.
		\item $ (Y, \rho) $ "--- полное.

			Тогда $ Y $ замкнуто.
	\end{enumerate}
\end{theorem}

\begin{eproof}
	\item Пусть $ \set{y_n}_{n = 1}^\infty $ фундаментальна в $ Y $.
		Тогда она фундаментальна и в $ X $, а $ X $ полно.
		Значит, $ \exists \lim\limits_{n \to \infty} y_n = a \in X $.
		Так как $ Y $ замкнуто, $ a \in Y \implies (Y, \rho) $ "--- полное.
	\item Пусть $ \set{y_n}_{n = 1}^\infty, \quad y_n \in Y, \lim\limits_{n \to \infty} y_n = a $.
		Проверим, что $ a \in Y $.

		$ \set{y_n}_{n = 1}^\infty $ фундаментальна $ \implies \exists \lim y_n \in Y $ (\as $ Y $ "--- полное).
\end{eproof}

\section{Банаховы пространства}

\subsection{Основные понятия}

\begin{definition}
	$ X $ "--- линейное пространство над полем $ K $ ($ K = \R $ или $ \Co $).

	$ p : X \to \R $ называется \emph{полунормой}, если
	\begin{enumerate}
		\item полуаддитивность: $ p(x + y) \le p(x) + p(y) \quad \forall x, y \in X $;
		\item однородность: $ p(\lambda x) = |\lambda| p(x) \quad \forall \lambda \in K, ~ x \in X $.
	\end{enumerate}
\end{definition}

\begin{properties}
	$ (X, p), \quad p $ "--- полунорма, $ \quad \On $ "--- ноль пространства $ X $

	\begin{enumerate}
		\item $ p(\On) = 0 $;
		\item $ p(x) \ge 0 \quad \forall x \in X $.
	\end{enumerate}
\end{properties}

\begin{eproof}
\item $ p(\On) = p(0 \cdot \On) \undereq{\text{однородность}} 0 \cdot p(\On) = 0 $.
\item Пусть $ x \in X $
	$$ 0 = p(\On) = p \bigl( x + (-x) \bigr) \le p(x) + p(-x) = 2 p(x) \quad \implies p(x) \ge 0 $$
\end{eproof}

\begin{definition}
	$ X $ "--- линейное пространство над $ \R $ или $ \Co $.

	$ p : X \to \R $ называется \emph{нормой}, если
	\begin{enumerate}
		\item $ p $ "--- полунорма;
		\item $ p(x) = 0 \iff x = \On $.
	\end{enumerate}
\end{definition}

\begin{notation}
	$ \|x\| $ "--- норма, $ \quad (X, \|\cdot\|) $ "--- нормированное пространство.
\end{notation}

Определим метрику, \emph{порождённую нормой}: $ \rho(x, y) \define \|x - y\| $.

\begin{definition}
	$ (X, \|\cdot\|) $ называется \emph{банаховым}, если оно полное.
\end{definition}

\subsection{Критерий полноты}

\begin{definition}
	\hfill
	\begin{enumerate}
		\item $ X $ "--- линейное пространство над $ K $.

			$ L \sub X $ называется \emph{подпространством (в алгебраическом смысле)}, если оно является линейным пространством, \ie
			$$
			\begin{rcases}
				x, y \in L \\
				\alpha, \beta \in K
			\end{rcases} \implies \alpha x + \beta y \in L $$
		\item $ (X, \|\cdot\|) $

			$ L \sub X $ называется \emph{(замкнутым) подпространством}, если
			\begin{enumerate}
				\item $ L $ "--- подпространство в алгебраическом смысле;
				\item $ L $ замкнуто.
			\end{enumerate}
	\end{enumerate}
\end{definition}

\begin{definition}
	$ (X, \|\cdot\|), \quad \set{x_k}_{k = 1}^\infty $
	$$ S_n \define \sum_{k = 1}^\infty x_k $$

	Говорят, что ряд $ \sum_{k = 1}^\infty x_k $ \emph{сходится}, если $ \exists \lim\limits_{n \to \infty} S_n = S \in X $.
	Тогда $ S = \sum_{k = 1}^\infty x_k $.

	Говорят, что ряд $ \sum x_k $ \emph{сходится абсолютно}, если $ \sum \|x_k\| $ сходится.
\end{definition}

\begin{theorem}[критерий полноты нормированного пространства]
	$ (X, \|\cdot\|) $ "--- полное \textbf{тогда и только тогда}, когда из абсолютной сходимости ряда следует его сходимость.
\end{theorem}

\begin{iproof}
\item $ \implies $ ($ X $ "--- полное)

	Возьмём $ \set{x_k}_{k = 1}^\infty $ такую, что $ \sum_{k = 1}^\infty \|x_k\| $ сходится.
	Применим к этому ряду критерий Коши:
	$$ \forall \eps > 0 \quad \exists N \in \N : \quad \forall n > N, ~ p \in \N : \quad \|x_{n + 1}\| + \dots + \|x_{n + p}\| < \eps $$
	$$ S_n = \sum_{k = 1}^n x_k $$
	Требуется доказать, что $ S_n $ образуют фундаментальную последовательность.
	Для этого оценим норму разности:
	$$ \|S_{n + p} - S_n\| = \bigl\| \sum_{k = 1}^p x_{n + k} \bigr\| \trile \sum_{k = 1}^p \|x_{n + k}\| < \eps $$
\item $ \impliedby $ (абсолютно сходящийся ряд сходится)

	Пусть $ \set{x_n}_{n = 1}^\infty $ фундаментальна.
	Нужно доказать, что у неё есть предел.

	Воспользуемся следствием из начала лекции:
	$$ \exists \set{x_{n_k}}_{k = 1}^\infty : \quad \sum_{k = 1}^\infty \|x_{n_{k + 1}} - x_{n_k}\| < +\infty $$
	$$ \implies \|x_{n_1}\| + \sum_{k = 1}^\infty \|x_{n_{k + 1}} - x_{n_k}\| < +\infty $$

	Рассмотрим ряд без нормы:
	$$ \exists S = x_{n_1} + \sum_{k = 1}^\infty (x_{n_{k + 1}} - x_{n_k}) $$
	$$ S_m = x_{n_1} + (x_{n_2} - x_{n_1}) + \dots + (x_{n_m} - x_{n_{m - 1}}) = x_{n_m} $$
	При этом,
	$$ \exists \lim\limits_{m \to \infty} S_m = S \quad \implies \exists \lim x_{n_m} = S \underimp{\text{св-во фунд. посл. 2}} \lim x_n = S $$
\end{iproof}

\section{Пространство ограниченных функций}

\begin{definition}
	$ X $ "--- множество.

	$ m(X) $ "--- \emph{пространство ограниченных функций}:
	$$ m(X) = \set{f : X \to \R \text{ (или $ \Co $)} \mid \sup\limits_{x \in X} |f(x)| < +\infty} $$

	Норма на таком пространстве называется \emph{равномерной, чебышёвской или} $ \sup $-\emph{нормой}:
	$$ \|f\|_\infty = \sup\limits_{x \in X}|f(x)| $$
\end{definition}

\begin{theorem}
	$ m(X) $ "--- банахово пространство.
\end{theorem}

\begin{eproof}
\item Проверим, что $ \|f\|_\infty $ удовлетворяет аксиомам нормы:
	$$ \|f\|_\infty = 0 \iff \sup\limits_{x \in X} |f(x)| = 0 \iff f(x) \equiv 0 \iff f = \On $$
	$$ \lambda \in K, \quad \|\lambda f\|_\infty = \sup\limits_{x \in X}|\lambda f(x)| = |\lambda| \sup\limits_{x \in X} |f(x)| = |\lambda| \cdot \|f\|_\infty $$
	Пусть $ f, g \in m(X), ~ x $ фиксирован.
	Тогда $ f(x), g(x) $ "--- числа.
	$$ \|f\|_\infty + \|g\|_\infty \ge \implies |f(x)| + |g(x)| \trige |f(x) + g(x)| \quad \forall x \in X $$
	В силу произвольности $ x $,
	$$ \implies \|f\|_\infty + \|g\|_\infty \sup\limits_{x \in X} |f(x) + g(x)| = \|f + g\|_\infty $$
\item Проверим полноту.

	Возьмём фундаментальную последовательность $ \set{f_n}_{n = 1}^\infty $ (в смысле нормы $ \|\cdot\|_\infty $).
	\begin{equ}1
		\forall \eps > 0 \quad \exists N \in \N : \quad \forall n, m > N \quad \|f_n - f_m\|_\infty < \eps
	\end{equ}
	Зафиксируем $ x \in X $.
	$$ \eref1 \implies |f_n(x) - f_m(x)| < \eps $$
	Из полноты $ K $ следует, что $ \exists \lim\limits_{n \to \infty} f_n(x) $.

	Обозначим $ f(x) \define \lim\limits_{n \to \infty} f_n(x) $ (\emph{поточечный}).

	$$ |f_n(x) - f_m(x)| < \eps \text{ при фиксированном } x, \quad n, m > N $$
	Перейдём к пределу по $ n $:
	$$ |f(x) - f_m(x)| \le \eps \quad \forall x \in X, \quad m >  N $$
	Воспользуемся произвольностью $ x $:
	$$ \|f - f_m\|_\infty = \sup\limits_{x \in X} |f(x) - f_m(x)| \le \eps \implies (f - f_m) \in m(X) $$
	$$ f = (f - f_m) + f_m $$
	В силу линейности $ m(X) $ это означает, что $ f \in m(X) $, $ \|f - f_m\|_\infty < \eps $ при $ m > N $
	$$ \implies \lim\limits_{m \to \infty} = f \text{ в пространстве } m(X) $$
\end{eproof}

\begin{note}
	Почему молодёжь ностальгирует по советской власти?
\end{note}

\begin{remark}
	Сходимость по норме в $ m(X) $ совпадает с равномерной сходимостью.
	$$ \lim\limits_{n \to \infty} \|f - f_n\|_\infty = 0 \iff \lim\limits_{n \to \infty} \bigl( \sup\limits_{x \in X}|f(x) - f_n(x)| \bigr) = 0 \iff f_n \uniarr{X} f $$
\end{remark}

\section{Пространства с sup-нормой}

\begin{definition}
	Зафиксируем $ n \in \N $.
	Рассмотрим $ l_n^\infty = (\R^n, \|\cdot\|_\infty) $ или $ (\Co^n, \|\cdot\|_\infty) $, где
	$$ \R^n = \set{x = (x_1, \dots, x_n), ~ x_j \in \R}, \quad \|x\|_\infty = \max\limits_{1 \le j \le n} |x_j| $$

	При этом, $ l_n^\infty = m(X) $, где $ X = \set{1, 2, \dots, n}, ~ f(j) = x_j $.
	Значит, $ l_n^\infty $ "--- банахово пространство.
\end{definition}

\begin{definition}
	$ l^\infty $ "--- \emph{пространство ограниченных последовательностей}, \ie
	$$ l^\infty = \set{x = \set{x_j}_{j = 1}^\infty, ~ x_j \in \R \text{ или } \Co, ~ \sup\limits{j \ge 1}|x_j| < +\infty} $$
	$$ l^\infty = m(X), ~ X = \N \implies l^\infty \text{ "--- банахово} $$
\end{definition}

\begin{definition}
	$$ C = \set{x = \set{x_j}_{j = 1}^\infty \mid \exists \lim\limits_{j \to \infty} x_j = x_0} $$
	$$ C_0 = \set{x = \set{x_j}_{j = 1}^\infty \mid \exists \lim x_j = 0} $$
\end{definition}

\begin{statement}
	Пространства $ C $ замкнуты.
\end{statement}
