\chapter{Линейные пространства}

\section{Гильбертовы и предгильбертовы пространства}

\subsection{Критерий принадлежности множеству ортогональных проекторов}

\begin{theorem}
	$ H $ "--- гильбертово
	\begin{enumerate}
		\item $ M \sub H $ "--- замкнутое подпространство, $ \quad
			P \coloneq P_M $ "--- ортогональный проектор на $ M $

			Тогда
			\begin{enumerate}
				\item $ P \in \mathscr B(H) $;
				\item $ P^2 = P $;
				\item \emph{самосопряжённость}: $ (Px, y) = (x, Py) $.
			\end{enumerate}
		\item $ P $ удовлетворяет свойствам (a), (b), (c), $ \quad
			M \coloneq P(H) $

			Тогда $ P = P_M $, \ie $ P $ "--- ортогональный проектор на $ P(H) $.
	\end{enumerate}
\end{theorem}

\begin{eproof}
\item $ M \sub H, \quad x \in H $
	$$ \exists ! y, z ; \quad x = y + z, \quad y \in M, ~ z \in M^\perp $$
	$$ Px = y $$
	\begin{enumerate}
		\item Проверим линейность

			Возьмём $ \alpha \in \Co $.
			$$ \alpha x = \alpha y + \alpha z, \quad \alpha y \in M, ~ \alpha z \in M^\perp $$
			Значит, разложение единственно.
			$$ \implies P(\alpha x) = \alpha y = \alpha P(x) $$

			Возьмём $ w \in H $
			$$ \exists ! u \in M, ~ v \in M^\perp : \quad w = u + v $$
			$$ \implies x + w = (u + y) + (v + z), \quad (u + y) \in M, ~ (v + z) \in M^\perp $$
			Разложение единственно.
			$$ \implies P(x + w) = u + y = Px + Pw $$

			Возьмём $ x \in H, \quad x = y + z, \quad y \in M, ~ z \in M^\perp $
			$$ \|x\|^2 = (y + z, y + z) = \|y\|^2 + \|z\|^2 \underimp{y = Px}
			\|Px\|^2 \le \|x\|^2 \implies \|Px\| \le \|x\| \quad \forall x \in H $$
			$$ P \in \mathscr B(H), \quad \|P\| \le 1 $$
			$$ M \ne \set{0} \implies \exists x \ne 0 \in M : \quad \|P\| =
			\sup\limits_{u \ne 0 \in H} \frac{\|Pu\|}{\|u\|} \ge \frac{\|Px\|}{\|x\|} = 1 \implies
			\|P\| = 1 $$
		\item $ P^2 = P $

			$$ x \in M \implies Px = x, \quad x \in H \implies Px \in M \implies P(Px) = Px \implies
			P^2 = P $$
		\item Самосопряжённость

			$$ x, y \in H, \quad P = P_M, \quad H = M \oplus M^\perp $$
			Обозначим $ Q = P_{M^\perp} $.
			$$ x = Px + Qx $$
			$$ (Px, y) = (Px, Py + Qy) \undereq{Px \perp Qy} (Px, Py) $$
			$$ (x, Py) = (Px + Qx, Py) = (Px, Py) $$
	\end{enumerate}
\item $ M \coloneq P(H) $
	\begin{enumerate}
		\item Проверим, что если $ x \in M $, то $ Px = x $.
			$$ x \in M \implies \exists y \in H : \quad Py = x \implies
			Px = P(Py) = P^2y = Py = x $$
		\item Проверим, что $ M $ замкнуто.

			Возьмём $ \set{x_n \in M}_{n = 1}^\infty, \quad \lim x_n = x $.
			В силу непрерывности $ P $
			$$ \lim Px_n = Px $$
			При этом,
			$$ Px_n = x_n \implies \lim x_n = Px \implies Px = x \implies x \in P(H) $$

			$ P_M $ "--- ортогональный проектор на $ M \implies
			$ если $ x \in M $, то $ Px = P_Mx = x $

			Возьмём $ y \in M^\perp, \quad P_My = 0 $.
			$$ \|Py\|^2 = (Py, Py) = \bigl(y, P(Py)\bigr) = (y, Py) \undereq{
				\begin{subarray}{c}
					y \in M^\perp \\
					Py \in M
				\end{subarray}
			} 0 \implies Py = 0 $$
			Возьмём $ x \in H $.
			$$ x = y + z, \quad y \in M, ~ y = P_Mx \implies Px = P_Mx $$
	\end{enumerate}
\end{eproof}

\begin{implication}[ортогональный проектор на конечномерное пространство]
	$ H $ "--- гильбертово, $ \quad M \sub H $ "--- подпространство, $ \quad \dim M = n, \quad
	\set{e_j}_{j = 1}^n $ "--- ОНБ

	$$ P_Mx = \sum_{j = 1}^n (x, e_j)e_j \quad \forall x \in H $$
\end{implication}

\begin{proof}
	$$ S_n \coloneq \sum_{j = 1}^n (x, e_j)e_j \implies S_n \in M $$
	$$ x = S_n + (x - S_n) $$
	Возьмём $ w = (x - S_n) $.
	Проверим, что $ w \in M^\perp $.
	$$ (S_n, e_j) = (x, e_j) \implies (w, e_j) = (x, e_j) - (x, e_j) = 0 \implies
	w \perp e_j \implies w \in M^\perp $$
\end{proof}

\begin{implication}[критерий полноты семейства элементов гильбертова пространства]
	\hfill \\
	$ H $ "--- гильбертово, $ \quad \set{x_\alpha \in H}_{\alpha \in A} $ "--- семейство

	$ \set{x_\alpha}_{\alpha \in A} $ "--- полное \textbf{тогда и только тогда}, когда
	$$ x \in H, \quad x \perp x_\alpha \quad \forall \alpha \in A \implies x = 0 $$
\end{implication}

\begin{proof}
	Пусть $ L = \ol{\mathscr L\set{x_\alpha}_{\alpha \in A}} $ "--- замыкание линейной оболочки.

	$ \set{x_\alpha}_{\alpha \in A} $ "--- полное $ \iff L = H \iff L^\perp = \set{0} \iff
	\Bigl( x \in H \perp x_\alpha \quad \forall \alpha \in A \implies x = 0 \Bigr) $
\end{proof}

\begin{statement}
	$ l^2, \quad L = \set{x = \set{x_n \in l^2}_{n = 1}^\infty | \sum_{n = 1}^\infty x_n = 0} $

	$$ \ol L = l^2 $$
\end{statement}

\begin{proof}
	Упражнение
\end{proof}

\begin{statement}
	$ z \in \Co, \quad |z| < 1, \quad x_z = \set{1, z, z^2, \dotsc} = \set{z^n}_{n = 0}^\infty \in l^2 $
	$$ \set{z_m}_{m = 1}^\infty, \quad |z_m| = 1, \quad \lim z_m = 0 $$

	Тогда $ \set{x_{z_m}}_{m = 1}^\infty $ "--- полное в $ l^2 $.
\end{statement}

\begin{proof}
	Упражнение
\end{proof}

\begin{statement}
	$ \lim z_m = a, \quad |a| < 1 $

	$$ \set{x_{z_m}}_{m = 1}^\infty \text{ полна} $$
\end{statement}

\begin{proof}
	Упражнение
\end{proof}

\section{Ряды Фурье}

\begin{definition}
	$ H $ "--- гильбертово, $ \quad
	\set{e_n}_{n = 1}^\infty $ "--- ОНС (\emph{ортонормированное семейство}), \ie
	$$ (e_n, e_m) =
	\begin{cases}
		1, \quad n = m \\
		0, \quad n \ne m
	\end{cases} $$
	$$ x \in H, \quad L = \set{ce_n}_{c \in \Co}, \quad \dim L = 1 \implies Px = (x, e_n)e_n $$

	Числа $ (x, e_n) $ будем называть \emph{абстрактными коэффициентами Фурье} по системе
	$ \set{e_n}_{n = 1}^\infty $.
\end{definition}

\begin{remark}
	Сходимость "--- это отдельная тема.
	Пока будем говорить, что мы каждому $ x $ \textbf{сопоставили} \emph{ряд Фурье}:
	$$ x \sim \sum_{n = 1}^\infty (x, e_n)e_n $$
\end{remark}

\begin{definition}
	$ \set{e_n} $ "--- ОС (\emph{ортогональная система}), $ \quad \|e_n\| > 0 $
	$$ L = \set{ce_n}_{c \in \Co}, \quad \vawe{e_n} = \frac{e_n}{\|e_n\|}, \quad
	P_Lx = (x, \vawe{e_n}) = (x, \vawe{e_n})\vawe{e_n} = \frac{(x, e_n)}{\|e_n\|^2}e_n $$

	$$ x \sim \sum_{k = 1}^\infty c_ne_n, \quad c_n = \frac{(x, e_n)}{\|e_n\|^2}
	\text{ "--- \emph{коэффициенты Фурье по ОС }} \set{e_n}_{n = 1}^\infty $$
\end{definition}

\subsection{Неравенство Бесселя}

\begin{implication}[из предыдущих следствий]
	$ \set{e_n}_{n = 1}^\infty $ "--- ОНС в $ H, \quad x \in H $

	$$ \sum_{n = 1}^\infty |(x, e_n)|^2 \le \|x\|^2 $$
\end{implication}

\begin{proof}
	Возьмём $ \alpha_j \in \Co $.
	Запишем теорему Пифагора:
	$$ \Bigl\| \sum_{j = 1}^n \alpha_j e_j \Bigr\|^2 =
	\Bigl( \sum_{j = 1}^n \alpha_j e_j, \sum_{k = 1}^n \alpha_k e_k \Bigr) \undereq{e_j \perp e_k}
	\sum_{j = 1}^k |\alpha_j|^2 $$

	Пусть $ L_n = \mathscr L \set{e_j}_{j = 1}^\infty $
	$$ P_{L_n}x = \sum_{j = 1}^n (x, e_j)e_j \implies
	\|P_{L_n}x\|^2 = \sum_{j = 1}^n |(x, e_j)|^2, \quad \|P_{L_n}\| \le 1 $$
	$$ \implies \sum_{j = 1}^n |(x, e_j)|^2 \le \|x\|^2 \quad \forall n \implies
	\sum_{j = 1}^\infty |(x, e_j)|^2 \le \|x\|^2 $$
\end{proof}

\subsection{Разложение в ряд Фурье элемента из замыкания линейной оболочки ОНС}

\begin{theorem}
	$ H $ "--- гильбертово, $ \quad \set{e_n}_{n = 1}^\infty $ "--- \textbf{ОНС}

	Следующие условия равносильны:
	\begin{enumerate}
		\item $ x \in \mathscr \ol{L \set{e_n}_{n = 1}^\infty} $;
		\item $ x = \sum_{n = 1}^\infty (x, e_n)e_n $;
		\item \emph{равенство Парсеваля}: $ \|x\|^2 = \sum_{n = 1}^\infty |(x, e_n)|^2 $.
	\end{enumerate}
\end{theorem}

\begin{iproof}
\item $ 1 \implies 2 $

	Возьмём $ \eps > 0 $.
	$$ \exists y \in \mathscr L\set{e_n} : \quad \|x - y\| < \eps $$
	$$ y = \sum_{j = 1}^m \alpha_je_j $$

	$$ L_n = \mathscr L\set{e_j}_{j = 1}^n \implies y \in L_m $$
	$$ P_{L_m}x = \sum_{j = 1}^m (x, e_j) e_j \eqqcolon S_m $$
	$$ \|x - S_m\| = \min\limits_{h \in L_m}\|x - h\| \le \|x - y\| < \eps $$

	$$ L_{m + 1} \supset L_m \implies \rho(x, L_{m + 1}) \le \rho(x, L_m) = \|x - S_m\| < \eps $$
	$$ \|x - S_{m + 1}\| \le \|x - S_m\| \implies \|x - S_n\| < \eps \quad \forall n, m \implies
	\lim\limits_{m \to \infty} S_m = x $$
\item $ 2 \implies 1 $ "--- очевидно
	$$ x = \lim S_m \implies x \in \ol{\mathscr L\set{e_j}} $$
\item $ 2 \implies 3 $
	$$ x = \lim S_n, \quad S_n = \sum_{j = 1}^n (x, e_j) e_j $$
	$$ (x, x) = \lim (S_n, S_n) $$
	$$ \|S_n\|^2 = (S_n, S_n) = \sum_{j = 1}^n |(x, e_j)|^2 \eqqcolon \sigma_n $$
	$$ \|x\|^2 = \lim\limits_{n \to \infty} \sigma_n $$
\item $ 3 \implies 2 $
	$$ x = S_n + (x - S_n), \quad w \coloneq x - S_n, \quad w \perp S_n $$
	$$ \implies \|x\|^2 = \|S_n\|^2 + \|x - S_n\|^2 $$
	$$ \|S_n\|^2 = \sigma_n, \quad \lim\|S_n\|^2 = \|x\|^2 $$
	$$ \implies \lim\|x - S_n\|^2 = 0 \implies \lim S_n = x $$
\end{iproof}

\begin{implication}
	$ H $ "--- гильбертово, $ \quad \set{e_n}_{n = 1}^\infty $ "--- полная ОНС

	$$ \forall x \in H \quad x = \sum_{n = 1}^\infty (x, e_n)e_n, \quad
	\|x\|^2 = \sum_{n = 1}^\infty |(x, e_n)|^2 $$
\end{implication}

\begin{definition}
	$ X $ "--- нормированное

	$ \set{e_n}_{n = 1}^\infty $ "--- \emph{базис} в $ X $, если
	$$ \forall x \in X \quad \exists ! \set{\alpha_j \in \Co} : \quad
	x = \sum_{j = 1}^\infty \alpha_je_j $$
\end{definition}

\begin{exmpls}
\item $ 1 \le p \le +\infty, \quad e_n = (0, \dots, 0, \underset n 1, 0, \dots, 0) $
	$$ x \in l^p, \quad x = \set{x_n}_{n = 1}^\infty, \quad \|x\|_p^p = \sum_{n = 1}^\infty |x_n|^p $$
	$$ x = \sum_{n = 1}^\infty x_ne_n, \quad \|x - S_n\|_p = \Bigl( \sum_{j = n + 1}^\infty |x_j|^p \Bigr)^{\frac1p} \to 0 $$
\end{exmpls}
