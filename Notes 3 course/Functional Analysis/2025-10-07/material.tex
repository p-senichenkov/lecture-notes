\chapter{Линейные пространства}

\begin{definition}
	$ X $ "--- линейное пространство над полем $ K, \quad B \sub X $.

	$ B $ называется \emph{выпуклым}, если
	$$ \forall x, y \in B \quad \forall t \in [0, 1] \quad tx + (1 - t)y \in B $$
\end{definition}

\begin{theorem}
	$ X, Y $ "--- линейные пространства, $ A \in \mathscr Lin(X, Y) $.

	\begin{enumerate}
		\item $ L \sub X $ "--- подпространство $ \implies A(L) $ "--- подпространство $ Y $;
		\item $ M \sub Y $ "--- подпространство $ \implies \underset{\text{(прообраз)}}{A^{-1}(Y)} $ "--- подпространство $ X $;
		\item $ B \sub X $ "--- выпуклое $ \implies A(B) $ "--- выпуклое в $ Y $;
		\item $ C \sub Y $ "--- выпуклое $ \implies \underset{\text{(прообраз)}}{A^{-1}(C)} $ "--- выпуклое в $ X $;
		\item $ A $ "--- биекция $ \implies \exists A^{-1} \in \mathscr Lin(Y, X) $.
	\end{enumerate}
\end{theorem}

\begin{note}
	Не надо думать, что $ A $ "--- биекция (кроме последнего).
\end{note}

\begin{noproof}
	Было в алгебре.
\end{noproof}

\begin{definition}
	$ A \in \mathscr Lin(X, Y), \quad X, Y $ линейны над $ K $.

	Определим \emph{ядро оператора}:
	$$ \operatorname{Ker} A = \set{x \in X | Ax = 0} $$
	и \emph{образ оператора}:
	$$ \operatorname{Im} A = A(X) = \set{y \in Y | \exists x \in X : \quad y = Ax} $$
\end{definition}

\begin{implication}
	$ A \in \mathscr Lin(X, Y) \implies \operatorname{Ker} A $ "--- подпространство $ X, \quad \operatorname{Im} A $ "--- подпространство $ Y $.
\end{implication}

\begin{definition}
	$ X, Y, Z $ линейны, $ \quad X \overarr A Y \overarr B Z $.

	$ C = B \cdot A $ называется \emph{произведением операторов}, если $ C(x) = B \bigl( A(x) \bigr) $.
\end{definition}

\begin{statement}
	Если $ A, B $ линейны, то $ A \cdot B $ линеен.
\end{statement}

\section{Линейные операторы в нормированных пространствах}

\begin{definition}
	$ (X, \|\cdot\|), ~ (Y, \|\cdot\|), \quad A \in \mathscr Lin(X, Y) $.

	$ A $ \emph{ограничен}, если
	$$ \forall \text{ ограниченного } B \sub X \quad A(B) \text{ ограничено} $$
\end{definition}

\subsection{Эквивалентность непрерывности и ограниченности линейного оператора}

\begin{theorem}
	$ (X, \|\cdot\|), ~ (Y, \|\cdot\|), \quad A \in \mathscr Lin(X, Y) $.

	Следующие условия равносильны (FCE):
	\begin{enumerate}
		\item $ A $ непрерывен в $ \On $;
		\item $ A $ непрерывен на $ X $;
		\item $ \exists c > 0 : \quad \|Ax\|_Y \le c\|x\|_X \quad \forall x \in X $;
		\item $ A $ ограничен;
		\item $ \exists r > 0 : \quad A \bigl( \mathtt B_r(\On) \bigr) $ ограничено.
	\end{enumerate}
\end{theorem}

\begin{iproof}
\item 1 $ \implies $ 2

	В силу линейности $ A(\On) = \On $.

	$ A $ непрерывен в $ \On \implies \exists \delta > 0 : \quad \forall \|x\| < \delta \quad \|Ax\| < \eps $.

	Зафиксируем $ x_0 \in X $, обозначим $ y_0 = Ax_0 $.
	$$ \|x - x_0\| < \delta \underimp{\text{непр. в } \On} \|A(x - x_0) \| < \eps \iff
	\|Ax - Ax_0\| < \eps \iff \text{ непр. в } x_0 $$
\item 2 $ \implies $ 1 "--- очевидно.
\item 1 $ \implies $ 3

	$$ \forall \eps > 0 \exists \delta > 0 : \quad \forall \|x\| \le \delta \quad \|Ax\| < \eps $$
	Возьмём $ z \in X, ~ z \ne \On $.
	$$ \Bigl\| \frac z{\|z\|} \Bigr\| = 1 $$
	\begin{multline*}
		x = \delta \cdot \frac z{\|z\|} \implies \|x\| = \delta \implies \|Ax\| < \eps \implies
		\Bigl\| A \Bigl(\delta \cdot \frac z{\|z\|} \Bigr) \Bigr\| < \eps \iff \\
		\iff \frac{\delta}{\|z\|} \cdot \| Az \| < \eps \iff
		\|Az\| < \frac\eps\delta \|z\|
	\end{multline*}
\item 3 $ \implies $ 4

	Возьмём $ B \sub X $ "--- ограниченное, \ie
	$$ \exists R > 0 : \quad \forall x \in B \quad \|x\| \le R \implies \|Ax\| \underset{3)}\le c\|x\| \le cR \implies
	A(B) \text{ ограничено} $$
\item 4 $ \implies $ 5

	$$ \mathtt B_r(\On) \text{ ограничено } \underimp{\text{очевидно}} A \bigl( \mathtt B_r(0) \bigr) $$
\item 5 $ \implies 1 $

	$$ \exists R > 0 : \quad A \bigl( \mathtt B_r(0) \bigr) \sub \mathtt B_R(\On) $$
	\ie $ \|x\| < r \implies \|Ax\| < R $.

	Пусть $ \|Ax\| < \eps $.
	Найдём $ \delta : \|x\| < \delta $.
	$$ \delta = \eps \cdot \frac r R $$
	Пусть $ \|x\| < \delta $
	$$ \implies \|x\| < \eps \cdot rR \implies \Bigl\| x \cdot \frac R\eps \Bigr\| < r \implies \Bigl\| A \Bigl( x \cdot \frac R\eps \Bigr) \Bigr\| < R \implies \|Ax\| \cdot \frac R\eps < R \iff
	\|Ax\| < \eps $$
\end{iproof}

\begin{definition}
	$ \mathscr B(X, Y) $ "--- множество всех ограниченных линейных операторов из $ X $ в $ Y $ \\
	($ \iff $ множество всех непрерывных).
\end{definition}

\begin{remark}
	$ \mathscr B(X, Y) $ "--- подпространство $ \mathscr Lin(X, Y) $.
\end{remark}

\begin{definition}
	$ A \sub \mathscr B(X, Y) $.

	Введём \emph{норму} на $ \mathscr B $:
	$$ \|A\| = \inf\set{c > 0 | \|Ax\| \le c\|x\| \quad \forall x \in X} $$
	(по свойству 3 она конечна).
\end{definition}

\begin{statement}
	$ (X, \|\cdot\|), ~ (Y, \|\cdot\|) $

	\begin{enumerate}
		\item $ A \in \mathscr B(X, Y) \implies \|Ax\| \le \|A\| \cdot \|x\| \quad \forall x \in X $;
		\item $ \|\cdot\|_{\mathscr B(X, Y)} $ удовлетворяет аксиомам нормы.
	\end{enumerate}
\end{statement}

\begin{eproof}
\item Зафиксируем $ x \in X $, возьмём $ c > \|A\| $.
	$$ \implies \|Ax\| \le c\|X\| \implies \|Ax\| \le \int\set{c > 0 | c > \|A\|}\|x\| \implies \|Ax\| \le \|A\| \cdot \|x\| $$
\item $ A \in \mathscr B(X, Y) $.

	\begin{enumerate}
		\item $ \lambda \in K, \quad x \in X $.
			$$ \bigl\| (\lambda A)x \bigr\| = \bigl\|\lambda \cdot (Ax) \bigr\| =
			|\lambda| \cdot \|Ax\| \le |\lambda| \cdot \|A\| \cdot \|x\| \quad \forall x $$
			$$ \implies \|\lambda A\| \le |\lambda| \cdot \|A\| $$
			Пусть $ \lambda \ne 0 $.
			$$ \Bigl\| \frac1\lambda (\lambda A) \Bigr\| \le \Bigl| \frac1\lambda \Bigr| \cdot \|\lambda A\| $$
			$$ \implies \|A\| \cdot |\lambda| \le \|\lambda A\| $$
		\item Неравенство треугольника

			Пусть $ x \in X $
			$$ \|(A + B)x\| =
			\|Ax + Bx\| \le \|Ax\| + \|Bx\| \le \|A\| \cdot \|x\| + \|B\| \cdot \|x\| =
			(\|A\| + \|B\|) \|x\| \quad \forall x $$
			$$ \implies \|A + B\| \le \|A\| + \|B\| $$
		\item $ \|A\| = 0 $

			$$ \implies \forall x \in X \quad \|Ax\| \le \|A\| \cdot \|x\| = 0 $$
	\end{enumerate}
\end{eproof}

\subsection{Вычисление нормы непрерывного оператора}

\begin{theorem}
	$ A \in \mathscr B(X, Y) $

	$$ \|A\| = \sup\limits_{\|x\| \le 1}\|Ax\| =
	\sup\limits_{\|x\| < 1}\|Ax\| =
	\sup\limits_{\|x\| = 1}\|Ax\| =
	\sup\limits_{x \ne 0} \frac{\|Ax\|}{\|x\|} $$
\end{theorem}

\begin{proof}
	Обозначим $ a = \sup\limits_{\|x\| \le 1}, \quad b = \sup\limits_{\|x\| < 1}, \quad c = \sup\limits_{\|x\| = 1}, \quad
	d = \sup\limits_{x \ne 0} $. \\
	Очевидно, что $ a \le b, \quad d \ge c $.

	Докажем, что
	$$ \|A\| \ge a \ge b \ge \|A\|, \quad \|A\| \ge d \ge c \ge \|A\| $$

	\begin{enumerate}
		\item Пусть $ x : \|x\| \le 1 $.

			$$ \|Ax\| \le \|A\| \cdot \|x\| \le \|A\| \implies a \le \|A\| $$
		\item Пусть $ z \in X \ne 0, \quad \eps > 0 $.
			Рассмотрим
			$$ x = \frac z{\|z\| \cdot (1 + \eps)} \quad \implies \|x\| < 1 \implies \|Ax\| \le b $$
			$$ \implies \Bigl\| A \Bigl( \frac z{\|z\| (1 + \eps)} \Bigr) \Bigr\| \le b \iff
			\|Az\| \le (1 + \eps)b \cdot \|z\| \quad \forall z \implies
			\|A\| \le (1 + \eps)b \quad \forall \eps > 0 \implies
			\|A\| \le b $$
		\item Пусть $ x \in X \ne 0 $
			$$ \|Ax\| \le \|A\| \cdot \|x\| \implies \frac{\|Ax\|}{\|x\|} \le \|A\| \implies
			\sup\limits_{x \ne 0} \frac{\|Ax\|}{\|x\|} \le \|A\| \implies
			d \le \|A\| $$
		\item Пусть $ z \in X \ne 0 $
			$$ \Bigl\| \frac z{\|z\|} \Bigr\| = 1 \implies \Bigl\| A \Bigl( \frac z{\|z\|} \Bigr) \le c \implies
			\|Az\| \le c\|z\| \implies
			\|A\| \le c $$
	\end{enumerate}
\end{proof}

\begin{exmpls}
\item $ X = \mathcal C[a, b], \quad h(x) \in \mathcal C[a, b] $
	$ M_h(f) = h(x) \cdot f(x) $ "--- мультипликатор.

	Проверим, что
	$$ M_h \in \mathscr B(\mathcal C[a, b]), \quad
	\|M_h\|_{\mathscr B(\mathcal C[a, b])} = \|h\|_\infty $$

	$$ x \in [a, b] \implies \bigl( M_h(f)(x) \bigr) = h(x)f(x) \implies
	\|h \cdot f\|_\infty \le \max\limits_{[a, b]} |h(x)| \cdot \max\limits_{[a, b]}|f(x)| =
	\|h\|_\infty \cdot \|f\|_\infty $$

	$$ \forall f \in \mathcal C[a, b] \quad \|M_h(f)\|_\infty \le \|h\|_\infty \|f\|_\infty \implies
	\forall M_h \in \mathscr B(\mathcal C[a, b]) \quad \|M_h\|_{\mathscr B(\mathcal C[a, b])} \le \|h\|_\infty $$

	$$ f(x) = \chi_{[a, b]}(x) \in \mathcal C[a, b], \quad \|\chi_{[a, b]}\|_\infty = 1 $$
	$$ \|M_h\| = \sup\limits_{\|f\| = 1} \|M_h(f)\|_\infty \ge \|M_h(\chi_{[a, b]})\|_\infty =
	\|h\|_\infty \implies
	\|M_h\|_{\mathscr B(\mathcal C[a, b])} \|h\|_\infty $$
\item $ Y = \mathcal C[0, 1], \quad X = \set{f | f' \in \mathcal [0, 1]} $

	$ X $ "--- подпространство $ Y $, \ie $ \|f\|_X = \max\limits_{[0, 1]}|f(x)| $.
	$$ f \in X, \quad \mathscr D(f) = f', \quad \mathscr D \in \mathscr Lin(X, Y) $$
	Проверим, что $ \mathscr D $ не является непрерывным.
	$$ x^n \in \mathcal C[0, 1], \quad \|x^n\|_\infty = 1 \implies
	\sup\limits_{\|f\| = 1} \|\mathscr D(f)\|_\infty \ge \sup\limits_{n \in \N} \|\mathscr D(x^n)\| $$
	$$ \mathscr D(x^n) = nx^{n - 1} \implies \|\mathscr D(x^n)\|_\infty = n $$
	$$ \implies \sup \|\mathscr D(f)\| \ge \sup \set{\mathscr D(x^n)} = +\infty $$
\end{exmpls}
