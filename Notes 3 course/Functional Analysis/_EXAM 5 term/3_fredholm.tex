\section{Условие разрешимости уравнения Фредгольма.
Замкнутость образа оператора Фредгольма}

Будем изучать гильбертовы пространства, так как там проще доказательства.
Всё это верно и для банаховых пространств.

\begin{definition}
	$ H $ "--- гильбертово, $ \quad T \in \mathrm{Com}(H), \quad S = I - T $

	$ S $ будем называть \emph{оператором Фредгольма}
\end{definition}

\begin{definition}
	Рассмотрим серию уравнений:
	\begin{enumerate}
		\item \emph{уравнение Фредгольма}: $ Sx = a $;
		\item \emph{однородное уравнение Фредгольма}: $ Sx = 0 $;
		\item \emph{сопряжённое уравнение Фредгольма}: $ S^*y = h $;
		\item \emph{однородное сопряжённое уравнение Фредгольма}: $ S^*y = 0 $.
	\end{enumerate}
\end{definition}

$$ \lambda \in \Co \ne 0, \quad V(\lambda) = \lambda I - T = \lambda \Bigl( I - \frac1\lambda T \Bigr) $$
$$ \frac1\lambda \in \mathrm{Com}(H) \implies I - \frac1\lambda T \text{ "--- оператор Фредгольма} $$
Значит, свойства $ S $ можно будет распространить на $ V(\lambda) $.

\begin{theorem}
	$ H $ "--- гильбертово, $ \quad T \in \mathrm{Com}(H), \quad S = I - T $

	\begin{enumerate}
		\item $ S(H) $ замкнуто;
		\item $ S^*(H) $ замкнуто;
		\item $ H = S(H) \oplus \operatorname{Ker} S^* $;
		\item $ H = S^*(H) \oplus \operatorname{Ker}(S) $.
	\end{enumerate}
\end{theorem}

\begin{remark}
	3. и 4. эквивалентны тому, что

	Уравнение $ Sx = a $ разрешимо \textbf{для тех и только тех} $ a $, которые
	ортогональны решениям однородного сопряжённого уравнения Фредгольма $ S^*y = 0 $.
\end{remark}

\begin{lemma}[об ограниченных проообразах]
	В условиях теоремы $ \Set{y_n \in S(H)}_{n = 1}^\infty $ ограничена

	$$ \exists \Set{x_n}_{n = 1}^\infty : \quad \Set{x_n} \text{ ограничена}, \quad
	Sx_n = y_n $$
\end{lemma}

\begin{proof}
	$ M = \operatorname{Ker} S, \quad L = M^\perp, \quad H = M \oplus L $
	$$ S(M) = 0 \implies S(H) = S(L) $$
	$$ y \in S(H) \implies \exists! x \in L : \quad Sx = y $$
	$$ \forall y_n \quad \exists x_n \in L : \quad Sx_n = y_n $$

	Проверим, что $ \Set{x_n} $ ограничена.
	\textbf{Пусть} это не так, \ie
	$$ \exists \Set{n_k}_{k = 1}^\infty : \quad \lim\|x_{n_k}\| = +\infty $$
	НУО будем считать, что сама последовательность $ \Set{x_n} $ обладает таким свойством:
	$$ \lim\|x_n\| = +\infty $$

	Возьмём последовательность
	$$ \left\{\frac{x_n}{\|x_n\|}\right\}_{n = 1}^\infty $$
	$$ T \in \mathrm{Com}(H) \implies \exists \Set{n_k} : \quad \exists
	\lim T \Bigl( \frac{x_{n_k}}{\|x_{n_k}\|} \Bigr) \eqqcolon z_0 $$
	\begin{multline*}
		y_{n_k} = S(x_{n_k}) = x_{n_k} - T(x_{n_k}) \implies
		\underbrace{\frac{y_{n_k}}{\|x_{n_k}\|}}_{\underarr{k \to \infty} 0} =
		\frac{x_{n_k}}{\|x_{n_k}\|} -
		\underbrace{T \Bigl( \frac{x_n}{\|x_{n_k}\|} \Bigr)}_{\underarr{k \to \infty} z_0}
		\implies \\
		\implies \lim\limits_{k \to \infty} \Bigl( \frac{x_{n_k}}{\|x_{n_k}\|} \Bigr) =
		z_0 \underimp{T \in \mathscr B(H)} \lim\limits_{k \to \infty} T
		\Bigl( \frac{x_{n_k}}{\|x_{n_k}\|} \Bigr) = T(z_0) \implies \\
		\implies z_0 = T(z_0) \implies S(z_0) = z_0 - Tz_0 = 0 \implies z_0 \in M =
		\operatorname{Ker} S
	\end{multline*}
	$$ \frac{x_{n_k}}{\|x_{n_k}\|} \in L \text{ "--- замкнуто}, \quad
	\lim \frac{x_{n_k}}{\|x_{n_k}\|} = z_0 \implies z_0 \in L, \quad \|z_0\| = 1 $$
	При этом, $ z_0 \in L \cap M \implies z_0 = 0 $ "--- \contra.
\end{proof}

\begin{replacementproof}[теоремы]
	\hfill
	\begin{enumerate}
		\item 	Докажем, что $ S(H) $ замкнуто.
			Возьмём $ \Set{y_n \in S(H)}_{n = 1}^\infty : \quad \exists \lim\limits_{n \to \infty} y_0 $.
			$$ \exists \lim y_n \implies \Set{y_n} \text{ ограничена } \underimp{\text{лемма}}
			\exists \Set{x_n \in H} : \quad \Set{x_n} \text{ ограничена}, \quad Sx_n = y_n $$
			$$ T \in \mathrm{Com}(H) \implies \exists \Set{n_k} : \quad \lim T_{x_{n_k}} = z_0 $$
			$$ \underbrace{y_{n_k}}_{\to y_0} = S(x_{n_k}) = x_{n_k} - \underbrace{Tx_{n_k}}_{\to z_0} $$
			$$ \implies \exists \lim\limits_{k \to \infty} x_{n_k} = x_0
			\underimp{T \in \mathscr B(H)}
			\lim T(x_{n_k}) = Tx_0 $$
			$$ \implies z_0 = Tx_0 \implies y_0 = x_0 - Tx_0 = S(x_0) \implies y_0 \in S(H) $$
		\item $ S^* = I - T^*, \quad T^* \in \mathrm{Com}(H) \implies S^*(H) $ замкнуто.
		\item
			$$ \forall A \in \mathscr B(H) \quad H = \ol{A(H)} \oplus \operatorname{Ker} A^* $$
			$$ S(H) = \ol{S(H)} \implies H = S(H) \oplus \operatorname{Ker} S^* $$
		\item Аналогично.
	\end{enumerate}
\end{replacementproof}

\TODO{Замкнутость образа оператора Фредгольма?}

\section[Альтернатива Фредгольма: инъективность оператора Фредгольма эквивалентна его
\texorpdfstring{\\}{}
сюръективности]
{Альтернатива Фредгольма: инъективность оператора Фредгольма эквивалентна его
сюръективности}

\begin{theorem}\label{thm:alt_fred}
	$$ \operatorname{Ker} S = \Set{0} \iff S(H) = H $$
\end{theorem}

\begin{restate}
	Оператор $ S $ "--- инъекция \textbf{тогда и только тогда}, когда $ S $ "--- сюръекция.
	Из любого из этих выражений следует, что $ S $ "--- биекция.
\end{restate}

\begin{restate}
	\hfill
	\begin{itemize}
		\item \textbf{Либо} уравнение Фредгольма разрешимо для любой правой части и решение
			единственно.
		\item \textbf{Либо} существует ненулевое решение уравнения Фредгольма.
	\end{itemize}
\end{restate}

\begin{lemma}[о стабилизации]
	$ k \ge 0, \quad H_{k + 1} \coloneq S(H_k) $

	$$ \exists n \ge 0 : \quad H_n = H_{n + 1} $$
\end{lemma}

\begin{proof}
	\textbf{Допустим}, что $ H_{k + 1} \subseteq H_k $.
	$$ \implies H_k = H_{k + 1} \oplus
	\underbrace{H_{k + 1}^\perp}_{\text{ортогональное дополнение в } H_k} $$
	$$ \implies \exists x_k \in H_k : \quad \|x_k\| = 1, \quad x_k \perp H_{k + 1} $$

	Пусть $ n > m $.
	Хотим оценить разность $ \|Tx_n - Tx_m\| $.
	$$ Sx_n - Sx_m = x_n - Tx_n - x_m + Tx_m $$
	$$ Sx_n = x_n - Tx_n $$
	$$ \|Tx_n - Tx_m\| = \|x_n - Sx_n - x_m + Sx_m\| \ge 1 $$
	$$ Sx_n \in H_{n + 1}, \quad Sx_m \in H_{m + 1} $$
	$$ x_n, Sx_n, Sx_m \in H_{m + 1} \implies x_m \perp H_{m + 1} $$
	Значит, не существует $ \Set{Tx_{n_k}} : \quad \exists \lim T_{x_{n_k}}, \quad
	T \in \mathrm{Com}(H) $ "--- \contra.
\end{proof}

\hspace{.5em}

\begin{iproof}[теоремы]
\item $ \implies $

	$ \operatorname{Ker} S = \Set{0} $, \ie $ S $ "--- инъекция.

	Докажем, что $ S(H) = H $.
	\textbf{Пусть} $ S(H) \subsetneq H $, \ie $ \exists x \in H : \quad x \notin S(H) $.

	Подействуем на $ x $ оператором $ S $:
	$$ Sx \in S(H), \quad S(Sx) \notin S(H) $$
	(\as прообраз у элемента ровно один, и это "--- $ Sx $)
	$$ \implies Sx \in S(H) \setminus S^2(H) $$
	И так далее для любого $ n $:
	$$ S^nx \in S^n(H) \setminus S^{(n + 1)}(H) $$
	Это \textbf{противоречит} лемме.
\item $ \impliedby $
	\begin{multline*}
		\begin{rcases}
			S(H) = H \\
			H = S(H) \oplus \operatorname{Ker} S^*
		\end{rcases} \implies \operatorname{Ker} S^* = \Set{0} \underimp{S^* = I - T^*} \\
		\implies S^* \text{ "--- оператор Фредгольма} \underimp{\text{только что доказано}} S^*(H) = H
	\end{multline*}
	$$ H = S^*(H) \oplus \operatorname{Ker} S \implies \operatorname{Ker} S = \Set{0} $$
\end{iproof}

\section{Теорема о числе линейно-независимых решений однородного уравнения Фредгольма.
Следствие о спектре компактного оператора}

\begin{theorem}
	$ S = I - T, \quad T \in \mathrm{Com}(H) $

	$$ \dim \operatorname{Ker} S = \dim \operatorname{Ker} S^* < +\infty $$
\end{theorem}

\begin{restate}
	Однородные уравнения Фредгольма $ Sx = 0 $ и $ S^*y = 0 $ имеют одно и то же, и при том
	конечное, число ЛНЗ решений.
\end{restate}

\begin{proof}
	Пусть $ \dim \operatorname{Ker} S = n < +\infty $ (уже доказано для банахова пространства).
	$$ \dim \operatorname{Ker} S^* \eqqcolon m < +\infty $$

	\textbf{Пусть} $ n < m $.

	Выберем в этих пространствах ОНБ:
	$$ \Set{e_i}_{i = 1}^n \text{ "--- ОНБ } \operatorname{Ker} S, \quad \Set{f_j}_{j = 1}^m
	\text{ "--- ОНБ } \operatorname{Ker} S^* $$

	Определим $ V : H \to \operatorname{Ker} S^* $:
	$$ x \in H \quad Vx = \sum_{k = 1}^n (x, e_k) f_K $$
	$$ V(H) \sub \operatorname{Ker}S^* \implies \dim \bigl( V(H) \bigr) < +\infty \implies
	V \text{ "--- конечного ранга } \implies V \in \mathrm{Com}(H) $$
	$$ U \coloneq S + V = I - \underbrace{(T - V)}_{\in \mathrm{Com}(H)} \implies
	U \text{ "--- оператор Фредгольма} $$

	Для $ U $ верна \autoref{thm:alt_fred}.
	Проверим, что $ \operatorname{Ker} U = \Set{0} $.
	Пусть $ x \in \operatorname{Ker} U $.
	$$ Ux = 0 \implies Sx + Vx = 0 $$
	$$
	\begin{rcases}
		Sx \in S(H) \\
		Vx \in \operatorname{Ker} S^*
	\end{rcases} \underimp{S(H) \perp \operatorname{Ker} S^*} Sx \perp Vx \implies \\
	\begin{cases}
		Sx = 0 \\
		Vx = 0
	\end{cases} $$
	$$ Vx = \sum_{k = 1}^n (x, e_k)  f_k = 0 \implies (x, e_K) = 0 \quad \forall k \implies
	x \perp \operatorname{Ker} S \underimp{x \in \operatorname{Ker} S} x = 0 $$

	По \ref{thm:alt_fred} $ U(H) = H $.
	Проверим, что это невозможно для $ n + 1 $, \ie что
	$$ f_{n + 1} \notin U(H) $$

	Возьмём $ x \in H $.
	$$ Ux = Sx + Vx $$
	$$ f_{n + 1} \in \operatorname{Ker} S^* \implies f_{n + 1} \perp S(H) \implies
	f_{n + 1} \perp Sx $$
	$$ Vx = \sum (x, e_k) f_k \underimp{f_{n + 1} \perp f_k} f_{n + 1} \perp Vx \implies
	f_{n + 1} \in \bigl( U(H) \bigr)^\perp = \Set{0} \text{ "--- } \contra $$

	Тем самым, мы доказали, что
	$$ \dim \operatorname{Ker} S \ge \dim \operatorname{Ker} S^* \ge \dim \operatorname{Ker} S^{**}
	= \dim \operatorname{Ker} S $$
\end{proof}

\begin{implication}
	$ T \in \mathrm{Com}(T) $
	\begin{enumerate}
		\item $ \lambda \notin 0, \quad \lambda \in \sigma(T) $

			$$ \lambda \in \sigma_p(T) $$
		\item $ \lambda \in \sigma_p(T), \quad H_\lambda = \operatorname{Ker}(\lambda I - T), \quad
			Y_{\ol \lambda} = \operatorname{Ker}(\ol \lambda I - T^*) $

			$$ \dim H_\lambda = \dim Y_{\ol \lambda} < +\infty $$
		\item
			$$ \sigma_p(T) \setminus \Set{0} = \Set{\lambda_n}_{n = 1}^N $$

			При этом, если $ N = +\infty $, то
			$$ \lim\limits_{n \to \infty} \lambda_n = 0 $$
	\end{enumerate}
\end{implication}

\section{Простейшие свойства самосопряжённого оператора}

\begin{properties}
	$ H $ "--- гильбертово, $ \quad T \in \mathscr B(H), \quad T = T^* $ (\ie $ (Tx, y) = (x, Ty) $)

	\begin{enumerate}
		\item $ (Tx, x) \in \R \quad \forall x $;
		\item $ \lambda \in \sigma_p(T) \implies \lambda \in \R $;
		\item $ \lambda, \mu \in \sigma_p(T), \quad Tu = \lambda u, \quad Tv = \mu v $
			$$ u \perp v $$
		\item $ \|T\| = \sup\limits_{\|x\| = 1} |(Tx, x)| $;
	\end{enumerate}
\end{properties}

\begin{eproof}
\item
	$$
	\begin{rcases}
		(Tx, x) = \ol{(x, Tx)} \\
		(Tx, x) = (x, Tx)
	\end{rcases} \implies (x, Tx) \in \R $$
\item
	$$ \lambda \in \sigma_p(T) \implies \exists u \ne 0 $$
	$$ Tu = \lambda u \implies (Tu, u) = (\lambda u, u) \implies
	\lambda = \frac{(Tu, u)}{\|u\|^2} \in \R $$
\item $ Tu = \lambda u, \quad Tv = \mu v, \quad \lambda \ne \mu, \quad u, v \ne 0 $
	$$
	\begin{rcases}
		(Tu, v) = = (\lambda u, v) = \lambda (u, v) \\
		(Tu, v) = (u, Tv) = (u, \mu v) \undereq{\mu \in \R} \mu (u, v)
	\end{rcases} \underimp{\text{вычтем}} \underbrace{(\lambda - \mu)}_{\ne 0}(u, v) = 0 \implies
	u \perp v $$
\item $ Q = \sup\limits_{\|x\| = 1}|(Tx, x)| $
	\begin{itemize}
		\item Пусть $ x \in H, \quad \|x\| = 1 $.
			$$ |(Tx, x)| \underset{\text{К"--~Б}}\le \|Tx\| \cdot \|x\| \le
			\|T\| \cdot \underbrace{\|x\| \cdot \|x\|}_{= 1} = \|T\| \quad \forall x : ~ \|x\| = 1 $$
			$$ \implies Q = \sup\limits_{\|x\| = 1} |(Tx, x)| \le \|T\| $$
		\item Пусть $ u\in H, \quad u \ne 0 $.
			\begin{equ}{self_dual_elem_props}
				\Bigl\| \frac{u}{\|u\|} \Bigr\| = 1 \implies \Bigl| \Bigl\lgroup T
				\Bigl( \frac{u}{\|u\|} \Bigr), \frac{u}{\|u\|} \Bigr\rgroup \Bigr| \le Q \implies
				|(Tu, u)| \le Q\|u\|^2 \quad \forall u \in H
			\end{equ}

			Возьмём $ x, y \in H $.
			Применим \eref{self_dual_elem_props} к $ (x + y) $ и к $ (x - y) $:
			$$ \bigl( T(x + y), x + y \bigr) = (Tx, x) + (Tx, y) + (Ty, x) + (Ty, y) \le Q\|x + y\|^2 $$
			$$ -\bigl( T(x - y), x - y \bigr) = - \bigl( (Tx, x) - (Tx, y) - (Ty, x) + (Ty, y) \bigr) \le
			Q\|x - y\|^2 $$
			Сложим эти два неравенства:
			$$ 2 \bigl( (Tx, y) + (Ty, x) \bigr) \le Q \bigl( \|x + y\|^2 + \|x - y\|^2 \bigr) $$
			К правой части применим тождество параллелограмма, а левую преобразуем:
			\begin{multline*}
				(Tx, y) + (Ty, x) = (Tx, y) + (y, Tx) = (Tx, y) + \ol{(Tx, y)} = 2 \Re (Tx, y) \implies \\
				\implies 4 \Re (Tx, y) \le Q \cdot 2(\|x\|^2 + \|y\|^2)
			\end{multline*}

			Выберем $ x : \quad \|x\| = 1 $ такой, что $ Tx \ne 0 $.
			$$ y \coloneq \frac{Tx}{\|Tx\|} \implies \|y\| = 1 $$
			Подставим $ x, y $ в неравенство:
			$$ 4 \Re \Bigl( Tx, \frac{Tx}{\|Tx\|} \Bigr) \le 4Q \implies
			\|Tx\| \le Q \quad \forall x : ~ \|x\| = 1 \implies \|T\| \le Q $$
	\end{itemize}
\end{eproof}

\section{Существование собственного числа, модуль которого равен норме компактного
самосопряжённого оператора}

\begin{definition}
	$ T = T^* $
	$$ M \coloneq \sup\limits_{\|x\| = 1} (Tx, x), \quad m \coloneq \inf\limits_{\|x\| = 1}(Tx, x) $$

	$ m $ и $ M $ называются \emph{границами} оператора $ T $.
\end{definition}

\begin{remark}
	$$ \|T\| = \max\{|M|, |m|\} $$
\end{remark}

\begin{theorem}
	$ H $ "--- гильбертово, $ \quad T = T*, \quad T \in \mathrm{Com}(H) $

	$$ \exists \lambda \in \sigma_p(T) : \quad |\lambda| = \|T\| $$
\end{theorem}

\begin{proof}
	$ \|T\| = \sup\limits_{\|x\| = 1}|(Tx, x)| $
	\begin{itemize}
		\item $ \|T\| = |M| $
			$$ M = \sup\limits_{\|x\| = 1}(Tx, x) \implies \exists \Set{x_n}_{n = 1}^\infty : \quad
			\|x_n\| = 1, \quad \lim(Tx_n, x_n) = M $$
			$$
			\begin{rcases}
				m \le M \\
				\max\{|M|, |m|\} = |M|
			\end{rcases} \implies M > 0 $$
			$$ T \in \mathrm{Com}(H) \implies \exists \Set{n_k}_{k = 1}^\infty : \quad
			\exists \lim\limits_{k \to \infty} Tx_{n_k} = y $$

			Переименуем: $ \lim\limits_{n \to \infty} Tx_n = y $.
			Проверим, что $ y $ "--- с. в. $ T $ и $ Ty = My $.
			\begin{multline*}
				0 \le \|Tx_n - M \cdot x_n \|^2 = (Tx_n - Mx_n, Tx_n - Mx_n) = \\
				= \|Tx_n\|^2 - \underbrace{M(Tx_n, x_n)}_{\to M} - \underbrace{M(x_n, Tx_n)}_{\to M} +
				M^2 \underbrace{\|x_n\|^2}_{= 1} \underarr{n \to \infty} \|y\|^2 - M^2 \implies
				\|y\| \ge M
			\end{multline*}
			$$ \|y\| = \lim \|Tx_n\| \le \|T\| \cdot \|x_n\| = \|T\| = M \implies \|y\| \le M $$
			$$ \implies \lim M \cdot x_n = \lim Tx_n = y $$

			Воспользуемся тем, что $ T $ непрерывен:
			$$ M \cdot \lim x_n = y \implies M \underbrace{\lim Tx_n}_{= y} = Ty \implies
			Ty = My, \implies M \in \sigma_p(T), \quad M = \|T\| $$

		\item $ \|T\| = |m| $
			\TODO{дописать этот случай}
	\end{itemize}
\end{proof}

\section{Теорема Гильберта\texorpdfstring{"--~}{--}Шмидта о представлении компактного
самосопряжённого оператора в виде суммы ортогональных проекторов}

\TODO{Теорема Гильберта"--~Шмидта о представлении компактного самосопряжённого оператора}

\section{Теорема Гильберта\texorpdfstring{"--~}{--}Шмидта о существовании ОНБ из собственных
векторов компактного самосопряжённого оператора}

\TODO{Теорема Гильберта"--~Шмидта о существовании ОНБ}

\section{Компактность интегрального оператора с ядром из
\texorpdfstring{$ \mathrm L^2 $}{пространства Лебега}}

\TODO{Компактность интегрального оператора}
