\part{Линейные функционалы}

\section{Теорема Рисса о представлении линейного функционала для вещественного
пространства.}

\begin{theorem}[Рисс]
	$ H $ "--- гильбертово пространство

	\begin{enumerate}
		\item $ y \in H $ "--- фиксирован, $ \quad f_y : H \to \Co : \quad f_y(x) = (x, y) $
			$$ f_y \in H^*, \quad \|f_y\|_{H^*} = \|y\|_H $$
		\item $ f \in H^* $
			$$ \exists ! y \in H : \quad f = f_y $$
	\end{enumerate}
\end{theorem}

\begin{eproof}
\item $ y \in H $
	\begin{itemize}
		\item Проверим, что $ f_y \in \mathscr Lin(H, \Co) $

			$ \alpha \in \Co, \quad x, y \in H $
			$$ f_y(\alpha x + z) = (\alpha x + z, y) =
			\alpha (x, y) + (z, y) = \alpha f(x) + f(z) $$
		\item Оценим $ \|f_y\| $
			$$ \|f_y\| = |(x, y)| \underset{\text{К"--~Б}}\le \|y\| \cdot \|x\| \implies
			\|f_y\| \le \|y\| \implies f \in H^* $$

			\begin{itemize}
				\item Если $ y = 0 $, то $ f_y(x) = 0 \implies f_y = \On[], \quad \|f_y\| = 0 $
				\item $ y \ne 0 $
					$$ \|f_y\| = \sup\limits_{x \ne 0} \frac{|f_y(x)|}{\|x\|} \ge
					\frac{|f_y(y)|}{\|y\|} = \frac{(y, y)}{\|y\|} = \|y\| $$
			\end{itemize}
	\end{itemize}
\item $ f \in H^* $
	\begin{itemize}
		\item Если $ f = \On[] $, \ie $ f(x) \equiv 0 $, то $ f = f_0 $
		\item $ f \ne \On[] $

			Рассмотрим $ N = \operatorname{Ker} f \ne H $
			$$ H = N \oplus N^\perp, \quad N^\perp \ne \set{0} $$
			Возьмём $ y_0 \in N^\perp $.
			Докажем, что $ \dim N^\perp = 1 $.

			$$ f(y_0) \ne 0 $$
			$$ v \coloneq \frac{y_0}{f(y_0)} \implies f(v) = 1 $$

			Возьмём $ z \in N^\perp $.
			Докажем, что $ z $ пропорционально $ v $.
			$$ u \coloneq z - f(z)v \implies u \in N^\perp $$
			$$ f(u) = f(z) - f(z)\underbrace{f(v)}_1 = 0 \implies u \in N $$
			Значит, $ u = 0 \implies z = f(z)v $.

			Будем искать $ y $ в виде $ y = \alpha v $, где $ \alpha \in \Co $.
			$$ 1 = f_y(v) = f_y(v) = (v, \alpha v) = \alpha \|v\|^2 \implies
			\alpha = \frac1{\|v\|^2} \implies y = \frac{v}{\|v\|^2} $$
		\item Проверим единственность

			Пусть $ f = f_y, ~ f = f_z $.
			$$ \forall x \in H \quad f_y(x) = f_z(x) \implies (x, y) = (x, z) \implies
			(x, y - z) = 0 \implies y - z \in H^\perp = \set{0} $$
	\end{itemize}
\end{eproof}

\begin{remark}
	$ C : H \to H^* : \quad C(y) = f_y $

	Знаем, что $ \|f_y\|_{H^*} = \|y\|_H $.

	$$ C(y + z) = f_{y + z} $$
	$$ \forall x \in H \quad f_{y + z}(x) = (x, y + z) = (x, y) + (x, z) = f_y + f_z $$
	$$ C(y + z) = C(y) + C(z) $$

	Возьмём $ \alpha \in \Co $.
	$$ C(\alpha y) = f_{\alpha y} $$
	$$ f_{\alpha y}(h) = (x, \alpha y) = \ol\alpha (x, y) $$
	$ C $ "--- сопряжённо-линейный изометрический изоморфизм между $ H $ и $ H^* $.
	Говорят, что $ H^* = H, $ при этом имеют в виду, что $ C(H) = H^* $.
\end{remark}

\section{Геометрический смысл линейного функционала.
Теорема о норме линейного функционала}

\begin{theorem}
	$ X $ "--- линейное пространство над $ \mathbb K $
	\begin{enumerate}
		\item $ f \in \mathscr Lin(X, \mathbb K), \quad
			f \ne \On[], \quad L = \operatorname{Ker} f $
			$$ \operatorname{codim} L = \dim \bigl( \faktor X L \bigr) = 1 $$
			Это называется \emph{коразмерность} $ L $.
		\item $ L \sub X, \quad \operatorname{codim} L = 1, \quad x_0 \in X \setminus L $
			$$ \exists ! f \in \mathscr Lin(X, \mathbb K) : \quad L = \operatorname{Ker} f, \quad
			f(x_0) = 1 $$
	\end{enumerate}
\end{theorem}

\begin{eproof}
\item $ f \ne \On[] \implies \exists y_0 \notin L $
	$$ v \coloneq \frac{y_0}{f(y_0)} \implies f(v) = 1 $$

	Возьмём $ x \in X $.
	\begin{multline*}
		u \coloneq x - f(x)v \implies f(u) = f(x) - f(x)f(v) = 0 \implies u \in L \implies
		\ol u = \ol 0 \implies \ol 0 = \ol x - f(x) \ol v \implies \\
		\implies \ol x = f(x) \ol v \implies
		\faktor X L = \set{\alpha \ol v | \alpha \in \mathbb K} \implies
		\dim \bigl( \faktor X L \bigr) = 1
	\end{multline*}
\item $ L : \quad \dim \bigl( \faktor X L \bigr) = 1 $

	Возьмём $ x \in X $.
	$$ \exists \alpha \in \mathbb K : \ol x = \alpha \cdot \ol x_0 $$
	Определим $ f(x) = \alpha $.
	$$ \ol x_0 = 1 \cdot \ol x_0 \implies f(x_0) = 1 $$

	Проверим, что $ f \in \mathscr Lin(X, \mathbb K) $.
	Возьмём $ \alpha \in \mathbb K, \quad x, y \in X $.
	$$
	\begin{rcases}
		\ol x = \beta \ol v \\
		\ol y = \gamma \ol v
	\end{rcases} \implies \alpha \ol x + \ol y = (\alpha \beta + \gamma) \ol v \implies
	f(\alpha x + y) = \alpha \beta + \gamma = \alpha f(x) + f(y) $$
	$$ f(x) = 0 \iff \ol x = 0 \cdot \ol v = \ol 0 \iff x \in L \implies \operatorname{Ker} f = L $$

	Проверим единственность.
	Пусть $ \exists g \in \mathscr Lin(X, \mathbb K), \quad \operatorname{Ker} g = L, \quad
	g(x_0) = 1 $.
	$$ \forall x \in X \quad x = \alpha \cdot \ol x_0 \implies
	x = \alpha x_0 + y, \quad y \in L \implies
	\begin{cases}
		f(x) = \alpha \\
		g(x) = \alpha
	\end{cases} $$
\end{eproof}

\begin{remark}
	В условиях второго пункта
	$$ f^{-1}(1) = x_0 + L = \set{x_0 + y | y \in L} $$
\end{remark}

\begin{proof}
	Обозначим $ M = f^{-1}(1) $.
	\begin{itemize}
		\item $ z \in x_0 + L $, \ie $ z = x_0 + y, \quad y \in L $
			$$ f(z) = \underbrace{f(x_0)}_1 + \underbrace{f(y)}_0 = 1 \implies z \in M \implies
			x_0 + L \sub M $$
		\item $ z \in M $
			$$ f(z) = 1 \implies f(z - x_0) = 0 \implies z - x_0 \in L \implies z - x_0 = y \implies
			z = x_0 + y \in x_0 + L \implies M \sub x_0 + L $$
	\end{itemize}
\end{proof}

\begin{theorem}[норма линейного функционала]
	$ (X, \|\cdot\|), \quad f \in X^*, \quad f \ne \On[], \quad f(x_0) = 1, \quad
	L = \operatorname{Ker} f $
	$$ \| f \| = \frac1{\rho(x_0, L)} $$
\end{theorem}

\begin{proof}
	Обозначим $ d = \rho(x_0, L) = \inf \|x_0 - y\| $.
	\begin{itemize}
		\item $ 1 = f(x_0) = f(x_0 - y) \le \|f\| \cdot \|x_0 - y\| \quad \forall y \in L \implies
			1 \le \|f\| \inf \|x_0 - y\| = \|f\|d \implies \frac1d \le \|f\| $
		\item $ x \in X \setminus L, \quad f(x) \ne 0 $
			\begin{multline*}
				f \Bigl( \frac{x}{f(x)} \Bigr) = 1 \implies
				f \Bigl( x_0 - \frac{x}{f(x)} \Bigr) = 0 \implies
				x_0 - \frac{x}{f(x)} \in L \implies
				\Bigl\| x_0 - \Bigl( x_0 - \frac{x}{f(x)} \Bigr) \Bigr\| \ge d \iff \\
				\iff \frac{\|x\|}{\|f(x)\|} \ge d \implies |f(x)| \le \frac1d \|x\| \implies
				\|f\| \le \frac1d
			\end{multline*}
	\end{itemize}
\end{proof}

\begin{remark}
	В обозначениях теоремы $ M = f^{-1}(1) $
	$$ \rho(x_0, L) = \rho(0, M) $$
\end{remark}

\begin{proof}
	$$ \rho(x_0, L) = \inf \|x_0 - y\| $$
	$$ \rho(0, M) = \rho(0, x_0 + L) = \inf \|x_0 + y\| = \inf \|x_0 - y\| = \rho(x_0, L) $$
\end{proof}

\section{Формулировка теоремы Хана\texorpdfstring{"--~}{--}Банаха о продолжении
линейного функционала для вещественного пространства.
Продолжение линейного функционала с подпространства на линейную оболочку подпространства и
вектора}

\begin{definition}
	$ X $ "--- линейное пространство над $ \mathbb K $

	$ p : X \to \mathbb K $ называется \emph{выпуклым функционалом}, если
	\begin{enumerate}
		\item $ p(x + y) \le p(x) + p(y) $;
		\item $ p(tx) = tp(x) \quad \text{ при } t \ge 0 $.
	\end{enumerate}
\end{definition}

\begin{theorem}[Хан"--~Банах]
	$ X $ "--- линейное над $ \R, \quad p : X \to \R $ "--- выпуклый функционал, $ \quad L $ "--- 
	подпространство $ X, \quad f \in \mathscr Lin(L, \R), \quad f(x) \overset L \le p(x) $
	(говорят, что $ f $ \emph{подчинён} $ p $)

	$$ \exists g \in \mathscr Lin(X, \R) : \quad
	\begin{cases}
		g \bigr|_L = f \\
		g(x) \overset X \le p(x)
	\end{cases} $$
\end{theorem}

\begin{eproof}
\item Возьмём $ z \in X \setminus L $
	$$ L_1 \coloneq \mathscr L\set{L, z} = \Set{tx + z | t \in \R} $$

	Построим $ f_1 \in \mathscr Lin(L_1, \R) : \quad
	f_1 \bigr|_L = f, \quad f_1(y) \overset{L_1}\le p(y) $.
	$$ f_1(z) \coloneq c $$
	Позднее докажем, что можно выбрать такой $ c $.
	$$ f_1(x + tz) = f(x) + tc $$
	Далее продолжение тоже будем обозначать $ f $.
	$$ f(y) \overset{L_1}\le p(y) \iff f(x) + tc \overset{L}\le p(x + tc) \iff
	\begin{cases}
		f(x) + tc \le p(x + tz), \quad t > 0, \\
		f(x) - tc \le p(x - tz), \quad t > 0
	\end{cases} $$
	Разделим на $ t $:
	$$
	\begin{cases}
		f \bigl( \frac x t \bigr) + c \le p \bigl( \frac x t + z \bigr), \\
		f \bigl( \frac x t \bigr) - c \le p \bigl( \frac x t - z \bigr)
	\end{cases} $$
	Обозначим $ u = \frac x t, ~ v = \frac x t $ (никак друг с другом не связанные).
	$$
	\begin{cases}
		f(u) + c \le p(u + z) \quad \forall u \in L, \\
		f(v) - c \le p(v - z) \quad \forall v \in L
	\end{cases} $$
	$$ f(v) - p(v - z) \le c \le p(u + z) - f(u) $$
	$$ A \coloneq \Set{p(u + z) - f(u)}_{u \in L} \sub \R, \quad
	B \coloneq \Set{f(v) - p(v - z)}_{v \in L} \sub \R $$
	Проверим, что $ \forall a \in B \quad \forall b \in B \quad b \le a $.
	$$ b \le a \iff f(v) - p(v - z) \le p(u + z) - f(u) \iff
	\underbrace{f(u) + f(v)}_{f(u + v) \in L} \le p(u + z) + p(v - z) $$
	$ f $ подчинён $ p $, значит,
	$$ f(u + v) \le p(u + v) \underset{\text{выпуклость}}\le p(u + z) + p(v - z) \implies
	\exists c \in \R : \quad f_1(z) = c $$
\end{eproof}

\section{Лемма Цорна.
Доказательство теоремы Хана\texorpdfstring{"--~}{--}Банаха для вещественного
пространства (без доказательства возможности продолжения линейного функционала с
подпространства
на линейную оболочку подпространства и вектора)}

\begin{definition}
	$ P $ полуупорядочено

	$ x $ "--- \emph{максимальный элемент} $ P $, если
	$$ \forall y \in P \quad x \le y \implies x = y $$
\end{definition}

\begin{lemma}[Цорн]
	$ P $ частично упорядочено

	Если для любой цепи имеется верхняя грань, то во всём множестве $ P $ существует максимальный
	элемент.
\end{lemma}

\begin{theorem}[Хан"--~Банах]
	$ X $ "--- линейное над $ \R, \quad p : X \to \R $ "--- выпуклый функционал, \\
	$ L $ "--- подпространство $ X, \quad f \in \mathscr Lin(L, \R), \quad
	f(x) \overset L \le p(x) $ (говорят, что $ f $ \emph{подчинён} $ p $)

	$$ \exists g \in \mathscr Lin(X, \R) : \quad
	\begin{cases}
		g \bigr|_L = f \\
		g(x) \overset X \le p(x)
	\end{cases} $$
\end{theorem}

\begin{eproof}
\item[2.] $ \mathcal P \coloneq \Set{(M, h)}, \quad L \sub M \sub X, \quad
	M $ "--- подпр-во, $ \quad h \in \mathscr Lin(M, \R), \quad h \bigr|_L = f, \quad
	h(x) \overset M \le p(x) $

	Определим порядок:
	$$ (M, h) \le (M_1, h_1) \iff
	\begin{cases}
		M \sub M_1, \\
		h_1 \bigr|_M = h
	\end{cases} $$

	Пусть $ A $ "--- цепь в $ \mathcal P $, \ie
	$$ A = \Set{(M_\alpha, h_\alpha)}_{\alpha \in I} : \quad \forall \alpha, \beta \in I \quad
	\left[
	\begin{array}{l}
		(M_\alpha, h_\alpha) \le (M_\beta, h_\beta) \\
		(M_\beta, h_\beta) \le (M_\alpha, h_\alpha)
	\end{array} \right. $$

	Построим верхнюю грань для $ A $.
	$$ M_0 \coloneq \bigcup_{\alpha \in I} M_\alpha $$

	Проверим, что $ M_0 $ "--- подпространство.
	Пусть $ x, y \in M_0 $.
	$$ \exists \alpha, \beta : x \in M_\alpha, ~ y \in M_\beta \implies \left[
	\begin{array}{l}
		M_\alpha \sub M_\beta, \\
		M_\beta \sub M_\alpha
	\end{array} \right. $$
	Пусть выполняется первое.
	$$ \implies x, y \in M_\beta \text{ "--- подпространство } \implies ax + by \in M_\beta \subset
	M_0 \implies M_0 \text{ "--- подпространство} $$

	Определим $ h_0 : M_0 \to \R $.

	Возьмём $ x \in M_0 $.
	$$ \exists \alpha : x \in M_\alpha $$
	Положим $ h_0(x) = h_\alpha(x) $.

	Проверим корректность.
	Пусть $ x \in M_\beta $.
	$$ \left[
	\begin{array}{l}
		(M_\alpha, h_\alpha) \le (M_\beta, h_\beta), \\
		(M_\beta, h_\beta) \le (M_\alpha, h_\alpha)
	\end{array} \right. $$
	Пусть выполнено первое.
	$$ x \in M_\alpha, ~ x \in M_\beta \implies h_\beta(x) = h_\alpha(x) = h(x) $$

	Можно проверить, что $ h_0 \in \mathscr Lin(M_0, \R) $.
	$$ \forall \alpha \in I \quad (M_\alpha, h_\alpha) \le (M_0, h_0) \implies
	(M_0, h_0) \text{ "---  верхняя грань для } A $$
	По лемме Цорна, в $ \mathcal P $ существует максимальный элемент $ (M, h) $.
	$$ L \sub M, \quad h \bigr|_L = f, \quad h(x) \overset M \le p(x) $$

	Докажем, что $ M = X $.
	\textbf{Пусть} $ \exists z \in X \setminus M $.
	$$ M_1 \coloneq \mathscr Lin\set{M, z} $$
	По первой части доказательства
	$$ \exists h_1 \in \mathscr Lin(M_1, \R) : \quad h_1 \bigr|_M = h, \quad h_1(x) \overset{M_1}\le
	p(x) $$
	$$ \implies (M_1, h_1) \in \mathcal P $$
	$$ (M, h) \le (M_1, h_1), \quad M \subsetneq M_1 $$
	Это \textbf{противоречит} максимальности $ (M, h) $.
\end{eproof}

\begin{statement}\label{st:cont:1}
	$ X $ линейно над $ \R, \quad p $ "--- выпуклый функционал, $ \quad
	f \in \mathscr Lin(X, \R), \quad f(x) \overset X \le p(x) $

	$$ f(x) \ge -p(-x) $$
\end{statement}

\begin{proof}
	$ f(x) \le p(x) \implies f(-x) \le p(-x) \implies -f(x) \le p(-x) $
\end{proof}

\begin{statement}
	$ p $ "--- полунорма, $ \quad f(x) \overset X \le p(x) $

	$$ |f(x)| \le p(x) $$
\end{statement}

\begin{proof}
	$ p(-x) \undereq{p \text{ "--- полунорма}} p(x) \implies f(x) \ge -p(x) $
\end{proof}

\section{Обобщённый предел в пространстве ограниченных
последовательностей}

\begin{theorem}
	$ l_\R^\infty = \Set{x = \Set{x_n \in \R}_{n = 1}^\infty |
	\|x\| = \sup\limits_{n \in \N}|x_n| < +\infty} $

	$$ \exists L \in (l^\infty)^* : \quad \|L\|_{(l^\infty)^*} = 1, \quad
	\forall x = \Set{x_n}_{n = 1}^\infty \in l^\infty \quad
	\varliminf\limits_{n \to \infty} x_n \le L(x) \le \varlimsup\limits_{n \to \infty} x_n $$

	В частности, если $ \exists \lim x_n = x_0 $, то $ L(x) = x_0 $.
\end{theorem}

\begin{proof}
	Для $ x \in l^\infty $ положим $ p(x) = \varlimsup\limits_{n \to \infty} x_n $.

	Докажем, что $ p $ "--- выпуклый функционал:
	\begin{itemize}
		\item $ t > 0 \implies p(tx) = tp(x) $ "--- очевидно;
		\item $ p(x + y) \le p(x) + p(y) $
			$$ \varlimsup x_n + \varlimsup y_n \overset?\le \varlimsup(x_n + y_n) $$
			Вспомним определение верхнего предела:
			$$ a_n \coloneq \sup\limits_{m \ge n} \Set{x_m} \implies a_n \searrow \implies
			\exists \lim a_n = a, \quad \varlimsup x_n \coloneq a $$
			$$ b_n = \sup\limits_{m \ge n} \Set{y_m}, \quad b = \varlimsup y_n $$
			$$ c_n = \sup\Set{x_m + y_m}, \quad c = \varlimsup \Set{x_n + y_n} $$
			Зафиксируем $ n $.
			$$ \forall m \ge n \quad x_m + y_m \le a_n + b_n \implies c_n \le a_n + b_n $$
			Устремим $ n \to \infty $:
			$$ c \le a + b $$
	\end{itemize}

	$$ C \sub l^\infty, \quad
	C \coloneq \Set{x = \Set{x_n}_{n = 1}^\infty | \exists \lim x_n = x_0} $$

	Определим $ f : C \to \R : \quad f(x) = \lim x_n = x_0 $.
	$$ f(x) = \lim x_n \le \varlimsup x_n \implies f(x) \overset C \le p(x) $$
	Применим теорему Хана"--~Банаха:
	$$ \exists g \in \mathscr Lin(l^\infty, \R), \quad g(x) \overset C = f(x), \quad
	g(x) \overset{l^\infty}\le \varlimsup x_n $$
	По \autoref{st:cont:1}
	$$ g(x) \ge -p(-x) = -\varlimsup(-x_n) \undereq{\text{задача из Демидовича}} \varliminf x_n $$

	$ g = L $ "--- в теореме, $ \quad g \in \mathscr Lin(l^\infty, \R) $.
	$$
	\begin{rcases}
		g(x) \le \varlimsup x_n \le \sup|x_n| \\
		g(x) \ge \varliminf x_n \ge -\sup|x_n|
	\end{rcases} \implies |g(x)| \le \|x\|_\infty \implies \|g\| \le 1 \implies
	g \in (l^\infty)^* $$

	Возьмём $ x = \Set{1, 1, \dotsc} $.
	$$ \|x\| = 1, \quad |g(x)| = 1 \implies \|g\| \ge \frac{|g(x)|}{\|x\|} = 1 \implies \|g\| = 1 $$
\end{proof}

\section[Теорема Боненблюста\texorpdfstring{"--~}{--}Собчика о
продолжении линейного функционала для комплексного пространства]
{Теорема Боненблюста"--~Собчика о продолжении линейного \\
функционала для комплексного пространства}

\begin{theorem}[Боненблюст"--~Собчик]
	$ X $ линейно над $ \Co, \quad p $ "--- полунорма, $ \quad L $ "--- подпространство $ X, $ \\
	$ f \in \mathscr Lin(L, \Co), \quad |f(x)| \overset L \le p(x) $

	$$ \exists g \in \mathscr Lin(X, \Co) : \quad
	\begin{cases}
		g \bigr|_L = f, \\
		|g(x)| \overset X \le p(x)
	\end{cases} $$
\end{theorem}

\begin{proof}[овеществление]
	$ X $ линейно и над $ \R $, \ie
	$$
	\begin{rcases}
		a, b \in \R \\
		x, y \in X
	\end{rcases} \implies ax + by \in X $$
	Также, $ L $ "--- подпространство и в вещественном смысле.
	$$ f(x) = u(x) + \ii v(x), \quad u(x), v(x) \in \R $$

	Проверим, что $ u, v \in \mathscr Lin(L, \R) $.
	Пусть $ x, y \in L $.
	$$ f(x) = u(x) + \ii v(x), \quad f(y) = u(y) + \ii v(y) $$
	$$ f(x + y) = u(x + y) + \ii v(x + y) $$
	Сложим первые два равенства:
	$$ f(x + y) = \bigl( u(x) + u(y) \bigr) + \ii \bigl( v(x) + v(y) \bigr) $$
	Вещественные и мнимые части равны:
	$$ u(x+ y) = u(x) + u(y), \quad v(x + y) = v(x) + v(y) $$

	Возьмём $ x \in X, ~ a \in \R $.
	$$ f(x) = u(x) + \ii v(x) $$
	$$ f(ax) = u(ax) + \ii v(ax) $$
	$$ f(ax) = af(x) = a u(x) + \ii a v(x) \implies
	\begin{cases}
		u(ax) = au(x) \\
		v(ax) = av(x)
	\end{cases} $$
	$$ \implies u, v \in \mathscr Lin(L, \R) $$
	$$ f \in \mathscr Lin(X, \Co) \implies f(\ii x) = \ii f(x) \implies u(\ii x) + \ii v(\ii x) =
	\ii \bigl( u(x) + \ii v(x) \bigr) \implies
	u(\ii x) = -v(x) $$

	Применим теорему Хана"--~Банаха к $ u $, а $ v $ определим из этого тождества.
	\begin{multline*}
		u(x) \le |f(x)| \le p(x) \quad \forall x \in L \underimp{\text{т. Хана"--~Банаха}} \\
		\implies \exists \phi \in \mathscr Lin(X, \R) : \quad
		\begin{cases}
			\phi \bigr|_L = u, \\
			\phi(x) \overset X \le p(x)
		\end{cases} \underimp{p \text{ "--- полунорма}} |\phi(x)| \overset L \le p(x)
	\end{multline*}
	\begin{equ}{bone-sob:1}
		x \in X \quad \psi(x) \coloneq -\phi(\ii x) \implies \psi \in \mathscr Lin(X, \R)
	\end{equ}
	$$ g(x) \coloneq \phi(x) + \ii \psi(x) \implies g \in \mathscr Lin(X, \R), \quad
	g(x) \overset L = f(x) $$
	$$ g(\ii x) =
	\phi(\ii x) + \ii \psi(\ii x) \ii \bigl( -\ii \phi(\ii x) + \psi(\ii x) \bigr)
	\undereq{\eref{bone-sob:1}} \ii \bigl(\phi(x) + \ii \psi(x) \bigr) \implies
	g \in \mathscr Lin(X, \Co) $$

	Проверим подчинение.
	$$ x \in X, ~ g(x) \in \Co \quad g(x) = re^{\ii \theta}, \quad r \ge 0, ~ \theta \in \R $$
	$$ \implies |g(x)| = r = g \bigl( xe^{-\ii \theta} \bigr) =
	\phi \bigl( xe^{-\ii \theta} \bigr) + \ii \psi \bigl( xe^{-\ii \theta} \bigr) $$
	Слева вещественное число, справа "--- комплексное, значит, мнимая часть равна нулю:
	$$ |g(x)| = \phi \bigl( xe^{-\ii \theta} \bigr) \le
	p \bigl( xe^{-\ii \theta} \bigr) \undereq{p \text{ "--- полунорма}} p(x) $$
\end{proof}

\section[Теорема Хана\texorpdfstring{"--~}{--}Банаха для нормированного пространства.
Следствия: о достаточном числе линейных функционалов, формула для нормы элемента пространства,
формула для расстояния до подпространства, критерий полноты системы элементов]
{Теорема Хана"--~Банаха для нормированного пространства. \\
Следствия: о достаточном числе линейных функционалов, формула для нормы элемента пространства,
формула для расстояния до подпространства, критерий полноты системы элементов}

\begin{theorem}
	$ (X, \|\cdot\|) $ над $ \mathbb K, \quad
	L $ "--- подпространство $ X $ в алгебраическом смысле, $ \quad f \in L^* $

	$$ \exists g \in X^* : \quad
	\begin{cases}
		g \bigr|_L = f, \\
		\|g\|_{X^*} = \|f\|_{L^*}.
	\end{cases} $$
\end{theorem}

\begin{proof}
	Обозначим $ M = \|f\|_{L^*} $.
	$$ p(x) \coloneq M \cdot \|x\| \implies p \text{ "--- норма на } X $$

	Возьмём $ x \in L $.
	$$ |f(x)| \le \|f\|_{L^*}\|x\| = M \|x\| = p(x) \implies f \text{ подчинён } p $$
	Применяем теорему Хана"--~Банаха или Бонеблюста"--~Собчика:
	$$ \exists g \in \mathscr Lin(X, \mathbb K) : \quad
	\begin{cases}
		g \bigr|_L = f, \\
		|g(x)| \overset X \le p(x)
	\end{cases} $$
	$$ \implies |g(x)| \overset X \le M \cdot \|x\| \implies \|g\|_{X^*} \implies g \in X^* $$
	$$ \|g\|_{X^*} = \sup\limits_{
		\begin{subarray}{c}
			x \in X \\
			\|x\| \le 1
		\end{subarray}}|g(x)| \ge \sup\limits_{
		\begin{subarray}{c}
			x \in L \\
			\|x\| \ge 1
		\end{subarray}}|g(x)| = \|f\|_{L^*} $$
\end{proof}

\begin{implication}[о достаточном множестве линейных функционалов]
	$ (X, \|\cdot\|) $ над $ \mathbb K, \quad x_0 \in X $

	\begin{enumerate}
	\item
		$$ \exists g \in X^* : \quad
		\begin{cases}
			\|g\| = 1, \\
			g(x_0) = \|x_0\|;
		\end{cases} $$
	\item $ \|x_0\| = \max \{|h(x_0)| \mid \|h\|_{X^*} \le 1\} $.
	\end{enumerate}
\end{implication}

\begin{iproof}
\item $ x_0 \ne 0 $

	Рассмотрим $ L = \Set{\alpha x_0}_{\alpha \in \mathbb K} $.
	Определим $ f : L \to \mathbb K : \quad f(\alpha x_0) = \alpha \|x_0\| $. \\
	Понятно, что $ f \in \mathscr Lin(L, \mathbb K), \quad f(x_0) = \|x_0\|, \quad
	|f(\alpha x_0)| = |\alpha| \|x_0\| $.
	$$ \implies \|f\|_{L^*} = 1 $$

	По теореме
	$$ \exists g \in X^* : \quad
	\begin{cases}
		\|g\|_{X^*} = 1, \\
		g(x_0) = \|x_0\|.
	\end{cases} $$

	Рассмотрим $ h \in X^* : \quad \|h\| \le 1 $.
	$$ |h(x_0)| \le \|h\| \|x_0\| \le \|x_0\| \implies \sup\limits_{\|h\| \le 1}|h(x_0)| \le
	\|x_0\| $$
	Но $ \exists g : \quad
	\begin{cases}
		\|g\| = 1, \\
		|g(x_0)| = \|x_0\|.
	\end{cases} $
\item $ x_0 = 0 $

	Возьмём $ g \in X^* : \quad \|g\| = 1 $.
	$$ g(x_0) = 0 $$
\end{iproof}

\begin{implication}[расстояние до подпространства]
	$ (X, \|\cdot\|), \quad L $ "--- замкнутое подпространство $ X, \\ $
	$ x_0 \in X \setminus L, \quad d \coloneq \rho(x_0, L) $

	\begin{enumerate}
	\item
		$$ \exists g \in X^* : \quad
		\begin{cases}
			\|g\| = 1, \\
			g(x_0) = d, \\
			g \bigr|_L = 0;
		\end{cases} $$
	\item $ d = \min\left\{ |h(x_0)| \mid \|h\| \le 1, ~ h\bigr|_L = 0\right\} $.
	\end{enumerate}
\end{implication}

\begin{proof}
	$ M \coloneq \mathscr Lin(x_0, L) = \Set{\alpha x_0 + y | y \in L} $

	Определим $ f : M \to \mathbb K : \quad f(\alpha x_0 + y) = \alpha $.
	$$ \implies f(y) = 0 \quad \forall y \in L $$
	$$ f(x_0) = 1 $$
	Понятно, что $ f \in \mathscr Lin(M, \mathbb K) $.

	Воспользуемся теоремой о норме линейного функционала:
	$$ \|f\|_{M^*} = \frac1d $$
	$$ f_1 \coloneq df \implies \|f_1\| = 1, \quad f_1(x_0) = d $$
	$$ \exists g \in X^* : \quad
	\begin{cases}
		\|g\| = 1, \\
		g \bigr|_M = f_1
	\end{cases} \implies g \bigr|_L = 0, \quad g(x_0) = d $$

	Возьмём $ h \in X^* : \quad \|h\| \le 1, ~ h \bigr|_L = 0 $.
	$$ |h(x_0)| = |h(x_0 - y)| \le \|h\| \cdot \|x_0 - y\| \le \|x_0 - y\| \quad \forall y \in L $$
	$$ \implies |h(x_0)| \le d \implies \sup\limits_{
		\begin{subarray}{c}
			\|h\| \le 1 \\
			h\bigr|_L = 0
		\end{subarray}}|h(x_0)| \le d $$
	$$ \exists g : \quad
	\begin{cases}
		g(x_0) = d, \\
		g \bigr|_L = 0, \\
		\|g\| \le 1.
	\end{cases} $$
	$$ \implies d = \max\limits_{
		\begin{subarray}{c}
			\|h\| \le 1 \\
			h\bigr|_L = 0
		\end{subarray}}|h(x_0)| $$
\end{proof}

\begin{remark}
	Если $ L = \Set{0} $, то получим предыдущее следствие.
\end{remark}

\begin{implication}[критерий полноты семейства элементов в нормированном пространстве]
	\hfill \\
	$ (X, \|\cdot\|), \quad \Set{x_\alpha \in X}_{\alpha \in A} $

	$ \Set{x_\alpha} $ "--- полная система элементов \textbf{тогда и только тогда}, когда
	$$ \Bigl( f \in X^* \quad f(x_\alpha) = 0 \quad \forall \alpha \in A \Bigr) \implies f = \On[] $$
\end{implication}

\begin{proof}
	$ L \coloneq \ol{\mathscr L \Set{x_\alpha}}_{\alpha \in A} $
	\begin{itemize}
		\item $ \implies $

			$ \Set{x_\alpha} $ "--- полное, $ \quad f \in X^*, \quad f(x_\alpha) = 0 $
			$$ x = \sum_{k = 1}^n c_kx_{\alpha_k} \implies f(x) = 0 $$
			То есть, $ \forall x \in \mathscr L\Set{x_\alpha} \quad f(x) = 0 $.

			Возьмём $ x \in X $.
			$$ \exists \Set{x_n \in \mathscr L\Set{x_\alpha}}_{n = 1}^\infty : \quad \lim x_n = x $$
			$ f $ непрерывна $ \implies f(x) = \lim f(x) = 0 \implies f = \On[] $.
		\item $ \impliedby $

			\textbf{Пусть} $ L \subsetneq X $, \ie $ \exists x_0 \ne 0 \in X \setminus L $.
			По первому следствию
			$$ \exists g \in X^* : \quad
			\begin{cases}
				\|g\| = 1, \\
				g(x_0) = \|x_0\|
			\end{cases} \implies g \ne \On[] $$
			При этом, $ g \bigr|_L = 0 \implies g(x_\alpha) = 0 $ "--- \contra.
	\end{itemize}
\end{proof}

\section{Сепарабельность пространства, у которого сопряжённое пространство сепарабельно}

\begin{theorem}
	$ (X, \|\cdot\|) $

	Если $ X^* $ сепарабельно, то $ X $ сепарабельно.
\end{theorem}

\begin{proof}
	$ X^* $ сепарабельно $ \iff \exists \Set{f_n}_{n = 1}^\infty : \quad
	f_n $ плотны в $ X^* $, \ie
	$$ \forall f \in X^* \quad \exists \Set{f_{n_k}}_{k = 1}^\infty : \quad
	\lim \|f - f_{n_k}\| = 0 $$

	Пусть $ f \in X^* $.
	$$ \|f\| = \sup\limits_{\|x\| = 1} |f(x)| \implies \exists x :
	\begin{cases}
		\|x\| = 1, \\
		\|f\| \le 2|f(x)|
	\end{cases} $$

	Пусть теперь $ f \ne \On[] $.
	$$ \exists \Set{x_n} : \quad \|x_n\| = 1, \quad \|f_n\| \le 2|f_n(x_n)| $$
	Проверим, что $ \Set{x_n} $ полна в $ X $ (по следствию 3).
	Пусть $ f \in X^*, \quad f(x_n) = 0 \quad \forall n $.
	$$ \exists f_{n_k} \in X^* : \quad \lim \|f - f_{n_k}\| = 0 $$
	$$ \frac12 \|f_{n_k}\| \le |f_{n_k}(x_{n_k})| =
	|\underbrace{f(x_{n_k})}_0 - f_{n_k}(x_{n_k})| \le
	\|f - f_{n_k}\| \cdot \underbrace{\|x_{n_k}\|}_1 $$
	$$ \implies \lim \|f_{n_k}\| = 0 \implies f = \On[] \implies \Set{x_n} \text{ полна} $$
\end{proof}

\section
[Принцип равномерной ограниченности.
Сильная сходимость линейных операторов, \texorpdfstring{\\}{}
оценка нормы сильного предела]
{Принцип равномерной ограниченности.
Сильная сходимость линейных операторов, оценка нормы сильного предела}

\begin{lemma}
	$ (X, \|\cdot\|), ~ (Y, \|\cdot\|), \quad U \in \mathscr Lin(X, Y) $

	$$ \exists a \in X ~ \exists \eps > 0 ~ \exists R > 0 : \quad
	U \bigl( \mathtt B_\eps(a) \bigr) \sub \ol{\mathtt B_R(0)} \implies
	\begin{cases}
		U \in \mathscr B(X, Y), \\
		\|U\| \le \frac{2R}\eps
	\end{cases} $$
\end{lemma}

\begin{proof}
	Возьмём $ z \in X : \quad \|z\| < \eps $.
	\begin{multline*}
		a + z \in \mathtt B_\eps(a) \implies U (a + z) \sub \ol{B_R(0)} \implies
		\begin{cases}
			\|U(a + z)\| \le R, \\
			\|U(a)\| \le R
		\end{cases} \underimp{z = (a + z) - a} \\
		\implies \|Uz\| \le \|U(a + z)\| + \|U(a)\| \le 2R
	\end{multline*}

	Возьмём $ x \in X : \quad \|x\| < 1 $.
	$$ z = \eps x \implies \|\eps x\| < \eps \implies \|U(\eps x)\| \le 2R \implies
	\|Ux\| \le \frac{2R}\eps \underimp{\|U\| = \sup\limits_{\|x\| \le 1}\|Ux\|}
	\|U\| \le \frac{2R}\eps $$
\end{proof}

\begin{theorem}
	$ X $ "--- банахово, $ \quad (Y, \|\cdot\|), \quad
	\Set{U_\alpha \in \mathscr B(X, Y)}_{\alpha \in A}, \quad \forall x \quad
	\sup\limits_{\alpha \in A}\|U_\alpha x\| < +\infty $

	$$ \sup\limits_{\alpha \in A}\|U_\alpha\| < +\infty $$
\end{theorem}

\begin{proof}
	Нужно доказать, что $ \sup\limits_{\|x\| \le 1}\sup\limits_\alpha \|U_\alpha x\| < +\infty $.
	$$ Y_n \coloneq \Set{y | \|y\| \le n} \sub Y $$
	$ U_\alpha^{-1}(Y_n) $ замкнуто, \as $ U_\alpha $ непрерывен.
	$$ E_n = \bigcap_{\alpha \in A} U_\alpha^{-1}(Y_n) $$
	$ E_n $ замкнуты.

	Пусть $ x \in X $.
	$$ \exists n \in \N : \quad \sup\limits_{\alpha \in A}\|U_\alpha x\| < n \implies
	x \in U_\alpha^{-1}(Y_n) \underimp{\forall \alpha \in A} x \in E_n \implies
	X = \bigcup_{n = 1}^\infty E_n $$

	Применим теорему Бэра о категориях:
	$$ \exists n_0 : \quad \operatorname{Int}E_{n_0} \ne \O $$
	То есть,
	$$ \exists \mathtt B_\eps(a) \sub E_{n_0} \implies
	\mathtt B_\eps(a) \in U_\alpha^{-1}(Y_{n_0}) \quad \forall \alpha \in A \implies
	U_\alpha \bigl( \mathtt B_\eps(a) \bigr) \sub Y_{n_0} \underimp{\text{лемма}}
	\|U_\alpha\| \le \frac{2n_0}\eps \quad \forall \alpha \in A $$
\end{proof}

\begin{implication}[Принцип фиксации особенности]
	$ X $ "--- банахово, $ \quad (Y, \|\cdot\|), \quad \Set{U_\alpha \in \mathscr B(X, Y)}, $ \\
	$ \sup\limits_{\alpha \in A}\|U_\alpha\| = +\infty $

	$$ \exists x_0 \in X : \quad \sup\limits_{\alpha \in A}\|U_\alpha x_0\| = +\infty $$
\end{implication}

\begin{definition}
	$ (X, \|\cdot\|), ~ (Y, \|\cdot\|), \quad \Set{U_n \in \mathscr Lin(X, Y)}_{n = 1}^\infty, \quad
	\forall x \in X \quad \exists \lim\limits_{n \to \infty}U_n x \eqqcolon Ux $

	$ U = \slim U_n $ "--- \emph{сильный (поточечный) предел} $ U_n $.
\end{definition}

\begin{props}
	\item $ U_n \in \mathscr Lin(X, Y) $
		$$ U = \operatorname{s-lim} U_n \implies U \in \mathscr Lin(X, Y) $$
	\item $ U_n, U \in \mathscr B(X, Y) $
		$$ \lim \|U - U_n\| = 0 \implies Ux = \slim U_n x $$
\end{props}

\begin{eproof}
\item Очевидно.
\item $ x \in X $
	$$ \|Ux - U_nx\| \le \underbrace{\|U - U_n\|}_{\to 0} \cdot \|x\| \implies \lim U_nx = Ux $$
\end{eproof}

\begin{remark}
	Если $ U = \slim U_n, \quad U, U_n \in \mathscr B(X, Y) $, то \textbf{не обязательно}
	$ \lim \|U - U_n\| = 0 $.
\end{remark}

\begin{eg}
	$ X = l^1 $
	$$ x = \Set{x_n}_{n = 1}^\infty \in l^1 $$
	$$ f_n(x) \coloneq x_n, \quad \|f_n(x)\| = |x_n| \le \|x_n\|_1 \implies \|f_n\| \le 1 \implies
	f_n \in (l^1)^* $$

	Пусть $ e_n = (0, \dots, 0, \underset n 1, 0, \dots, 0) $.
	$$ f_n(e_n) = 1 \implies \|f_n\| \ge 1 $$

	Найдём сильный предел:
	$$ \lim x_n = 0 \implies \lim f_n(x) = 0 \quad \forall x \in l^1 \implies \On[] = \slim f_n $$
	$$ \|f - \On[]\| = 1, \quad \lim \|f_n - \On[]\| = 1 \ne 0 $$
\end{eg}

\begin{implication}
	$ X $ "--- банахово, $ \quad (Y, \|\cdot\|), \quad \Set{U_n}_{n = 1}^\infty, \quad
	U_n \in \mathscr B(X, Y), \quad U = \slim U_n $

	$$ U \in \mathscr B(X, Y), \quad \|U\| \le \varliminf \|U_n\| $$
\end{implication}

\begin{proof}
	$$ \exists \lim U_n x \implies \sup\limits_{n \in \N} \|U_nx\| < +\infty \underimp{\text{теорема}}
	\exists M > 0 : \quad \|U_n\| \le M \quad \forall n \in \N $$
	$$ b \coloneq \varliminf \|U_n\| \in \R, \quad b \le M $$
	$$ \exists \Set{n_k} : \quad b = \lim\limits_{k \to \infty} \|U_{n_k}\| \implies
	Ux = \lim U_{n_k} x \quad \forall x \in X $$
	$$ \|Ux\| = \lim\limits_{k \to \infty} \|U_{n_k}x\| \le \lim\|U_{n_k}\| \cdot \|x\| =
	b\|x\| \implies \|U\| \le b $$
\end{proof}

\begin{remark}
	$ U = \slim U_n $

	Возможно строгое неравенство (в примере так и было).
\end{remark}

\section{Теоремы Банаха\texorpdfstring{"--~}{--}Штейнгауза:
критерий сходимости операторов}

\begin{theorem}[Банах"--~Штейнгауз]
	$ X, Y $ "--- банаховы, $ \quad \Set{U_n \in \mathscr B(X, Y)}_{n = 1}^\infty $

	Существует сильный предел $ U_n $ \textbf{тогда и только тогда}, когда
	\begin{enumerate}
		\item $ \exists M > 0 : \quad \|U_n\| \le M \quad \forall n \in \N $;
		\item $ \Set{U_nx}_{n = 1}^\infty $ фундаментальна для $ \forall x \in E \sub X $, где
			$ E $ "--- полное в $ X $, \ie $ \ol{\mathscr L\Set{E}} = X $.
	\end{enumerate}
\end{theorem}

\begin{iproof}
\item $ \implies $
	\begin{enumerate}
		\item уже доказано;
		\item очевидно.
	\end{enumerate}
\item $ \impliedby $

	Проверим, что $ \forall x \in X \quad \Set{U_nx} $ фундаментальна.

	Обозначим $ L = \mathscr L\Set{E} $.
	Возьмём $ x \in L $.
	$$ x = \sum_{k = 1}^m c_kx_k, \quad x_k \in E $$
	Пусть $ p > q \in \N $.
	$$ \|U_px - U_qx\| \le
	\sum_{k = 1}^\infty |c_k| \underbrace{\|U_px_k - U_qx_k\|}_{\to 0} \underarr{p, q \to \infty}
	0 $$
	$ \implies \Set{U_nx} $ фундаментальна для $ x \in L $.

	Возьмём произвольный $ x \in X $.
	$$ \forall \eps > 0 \quad \exists z \in L : \quad \|x - z\| < \eps $$
	Воспользуемся тем, что $ \Set{U_nz} $ фундаментальна:
	$$ \exists N \in \N : \quad \forall n > m \ge N \quad \|U_nz - U_mz\| < \eps $$
	$$ \|Unx - U_mx\| \le
	\underbrace{\|U_nx - U_nz\|}_{\le \|U_m\| \cdot \|x - z\| \le M\|x - z\| < M\eps} +
	\underbrace{\|U_nz - U_mz\|}_{< \eps} + \underbrace{\|U_mz - U_mx\|}_{< M\eps} < \eps(2M + 1) $$
	$ \implies \Set{U_nx}_{n = 1}^\infty $ фундаментальна для $ \forall x \in X $.
	$$ \underimp{Y \text{ "--- банахово}} \exists \lim U_n x \eqqcolon U \quad
	\forall x \in X, \quad U = \slim U_n $$
\end{iproof}

\begin{remark}
	$ \implies $ верно для банахова $ X $ и нормированного $ Y $. \\
	$ \impliedby $ верно для нормированного $ X $ и банахова $ Y $.
\end{remark}

\begin{theorem}[Банах"--~Штейнгауз]
	$ X $ "--- банахово, $ \quad (Y, \|\cdot\|), \quad
	\Set{U_n \in \mathscr B(X, Y)}_{n = 1}^\infty, \quad U \in \mathscr B(X, Y) $

	$ U = \slim U_n $ \textbf{тогда и только тогда}, когда
	\begin{enumerate}
		\item $ \exists M > 0 : \quad \|U_n\| \le M \quad \forall n \in \N $;
		\item $ \exists \lim U_nx \quad \forall x \in E \sub X $, где $ E $ "--- полное в $ X $.
	\end{enumerate}
\end{theorem}

\begin{iproof}
\item $ \implies $ "--- уже доказано.
\item $ L \coloneq \mathscr L\Set{E}, \quad x \in L $
	$$ x = \sum_{k = 1}^m ck x_k, \quad x_k \in E $$
	$$ \exists \lim U_n x_k \quad \forall x_k \implies \exists \lim U_n x $$

	Проверим, что $ \forall x \in X \quad \exists \lim U_n x $.
	$$ \forall \eps > 0 \quad \exists z \in L : \quad \|x - z\| < \eps $$
	$$ \exists N : \quad \forall n \ge N \quad \|Uz - U_nz\| < \eps $$
	$$ \|Ux - U_nx\| \le \underbrace{\|Ux - Uz\|}_{\le \|U\| \cdot \|x - z\| < \|U\| \cdot \eps} +
	\underbrace{\|Uz - U_nz\|}_{< \eps} +
	\underbrace{\|U_nz - U_nx\|}_{\|U_n\| \cdot \|x - z\| < M\eps} < \eps(\|U\| + 1 + M) $$
	$$ \implies \exists U_nx = Ux $$
\end{iproof}

\section{Обратный оператор к \texorpdfstring{$ I - A $}{I - A} , где
\texorpdfstring{$ A $ "--- }{A~--- } сжатие.
Множество обратимых операторов открыто}

\begin{statement}\label{inv:st:1}
	$ A, B \in \mathscr B(X) $

	$$ \|AB\| \le \|A\| \cdot \|B\| $$
\end{statement}

\begin{proof}
	$$ \|A(Bx)\| \le \|A\| \cdot \|Bx\| \le \|A\| \cdot \|B\| \cdot \|x\| $$
\end{proof}

\begin{theorem}
	$ X $ "--- банахово, $ \quad I $ "--- тождественный, $ \quad A \in \mathscr B(X), \quad
	\|A\| < 1 $

	$$ \exists (I - A)^{-1}, \quad \|(I - A)^{-1}\| \le \frac1{1 - \|A\|}, \quad
	(I - A)^{-1} = \sum_{k = 0}^\infty A^k $$
\end{theorem}

\begin{proof}
	Проверим, что сумма норм операторов сходится.
	По \autoref{inv:st:1}
	$$ \|A^k\| \le \|A\|^k \implies \sum_{k = 0}^\infty \|A^k\| \le \sum_{k = 0}^\infty \|A\|^k =
	\frac1{1 - \|A\|} $$
	$ X $ "--- банахово $ \implies \mathscr B(X) $ "--- банахово.
	Воспользуемся критерием полноты:
	$$ \implies \exists S = \sum_{k = 0}^\infty A^k \in \mathscr B(X) $$
	$$ S_n \coloneq \sum_{k = 0}^n A^k $$
	$$ \|S_n\| \le \sum_{k = 0}^n \|A\|^k < \frac1{1 - \|A\|}, \quad
	\lim\limits_{n \to \infty} \|S - S_n\| = 0 $$
	$$ \implies \|S\| \le \frac1{1 - \|A\|} $$

	$$ (I - A)S_n = (I - A) \sum_{k = 0}^n A^k = \sum_{k = 0}^n A^k - \sum_{k = 0}^n A^{k + 1} =
	I - A^{n + 1} $$
	$$ \lim\limits_{n \to \infty} \|A^{n + 1}\| = 0 \implies \lim (I - A^{n + 1}) = I $$
	$$
	\begin{rcases}
		\begin{rcases}
			\lim(I - A)S_n = I \\
			\lim(I - A)S_n = (I - A)S
		\end{rcases} \implies (I - A)S = I \\
		S_n(I - A) = I - A^{n + 1} \implies S(I - A) = I
	\end{rcases} \implies S = (I - A)^{-1} $$
\end{proof}

\begin{definition}
	$ (X, \|\cdot\|), ~ (Y, \|\cdot\|) $

	Определим \emph{множество обратимых операторов}:
	$$ \mathrm{In}(X, Y) = \Set{A \in \mathscr B(X, Y) | \exists A^{-1} \in \mathscr B(Y, X)} $$
\end{definition}

\begin{theorem}
	$ X $ "--- банахово, $ \quad (Y, \|\cdot\|) $
	\begin{enumerate}
		\item $ U \in \mathrm{In}(X, Y), \quad V \in \mathscr B(X, Y), $
			$$ \|U - V\| < \frac1{\|U^{-1}\|} $$

			$$ \implies V \in \mathrm{In}(X, Y), $$
			\begin{equ}1
				\|V^{-1}\| \le \frac{\|U^{-1}\|}{1 - \|U - V\| \cdot \|U^{-1}\|}
			\end{equ}
			\begin{equ}2
				\|U^{-1} - V^{-1}\| \le \frac{\|U^{-1}\|^2 \|U - V\|}{1 - \|U - V\| \cdot \|U^{-1}\|}
			\end{equ}
		\item $ \phi : \mathrm{In}(X, Y) \to \mathrm{In}(Y, X) : \quad \phi(A) = A^{-1} $

			Тогда $ \phi $ непрерывно.
	\end{enumerate}
\end{theorem}

\begin{eproof}
\item Рассмотрим оператор $ W = U^{-1}(U - V) $.
	$$ \|W\| \le \|U^{-1}\| \cdot \|U - V\| < 1, \quad W \in \mathscr B(X) $$
	Можно применить предыдущую теорему:
	$$ \exists (I - W)^{-1}, \quad \|(I - W)^{-1}\| \le \frac1{1 - \|W\|} \le
	\frac1{1 - \|U^{-1}\| \|U - V\|} $$
	$$ W = U^{-1}(U - V) = I - U^{-1}V \implies I - W = U^{-1}V $$
	$$ V = U(U^{-1}V) = U(I - W) $$
	Каждый их них обратим, значит,
	$$ \exists V^{-1} = (I - W)^{-1}U^{-1} $$
	$$ \|V^{-1}\| \le \|U^{-1}\| \cdot \|(I - W)^{-1}\| \le
	\frac{\|U^{-1}\|}{1 - \|U^{-1}\| \cdot \|U - V\|} $$
	Получили неравенство \eref{1}.
	Получим из него неравенство \eref{2}:
	$$ U^{-1} - V^{-1} = U^{-1}(V - U)V^{-1} \implies
	\|U^{-1} - V^{-1}\| \le \|U^{-1}\| \|V - U\| \|V^{-1}\| \underset{\eref{1}}\le
	\frac{\|U^{-1}\|^2 \|U - V\|}{1 - \|U^{-1}\| \cdot \|U - V\|} $$
\item $ \phi(U) = U^{-1} $
	$$ \phi(U) - \phi(V) = U^{-1} - V^{-1} \underimp{\eref{2}}
	\lim\limits_{\|U - V\| \to 0}\|\phi(U) - \phi(V)\| = 0 $$
\end{eproof}

\section{Определение открытого отображения.
Критерий открытости для линейного оператора.
Условия существования непрерывного обратного оператора}

\begin{definition}
	$ (X, \tau_1), ~ (Y, \tau_2) $ "--- топологические пространства

	Отображение $ U : X \to Y $ \emph{открыто}, если оно переводит открытые множества в открытые:
	$$ G \sub X \text{ открыто } \implies U(G) \text{ открыто} $$
\end{definition}

\begin{remark}
	$ U $ непрерывно, если \textbf{прообраз} открытого множества открыт.
\end{remark}

\begin{statement}
	$ (X, \tau_1), ~ (Y, \tau_2) $ "--- топологические пространства, $ \quad
	U : X \to Y $ "--- биекция

	$$ U \text{ открыто } \iff U^{-1} \text{ непрерывно} $$
\end{statement}

\begin{statement}
	$ (X, \|\cdot\|), ~ (Y, \|\cdot\|), \quad U \in \mathscr Lin(X, Y) $

	$$ U \text{ открыт } \iff \exists r > 0 : \quad
	\mathtt B_r(0) \sub U \bigl( \mathtt B_1(0) \bigr) $$
\end{statement}

\begin{iproof}
\item $ \implies $
	$$
	\begin{rcases}
		U(0) = 0 \\
		U \text{ открыто } \implies U \bigl( \mathtt B_1(0) \bigr) \text{ открыто }
	\end{rcases} \implies 0 \in U \bigl( \mathtt B_1(0) \bigr) \implies
	\exists r > 0 : \quad \mathtt B_r(0) \sub U \bigl( \mathtt B_1(0) \bigr) $$
\item $ \impliedby $

	Возьмём $ G \sub X $ "--- открытое.
	Пусть $ y_0 \in U(G) $.
	$$ \exists x_0 \in G : \quad Ux_0 = y_0 $$
	$$ \exists R > 0 : \quad \mathtt B_R(x_0) \sub G $$
	$$ \mathtt B_r(0) \sub U \bigl( \mathtt B_1(0) \bigr) \underimp{\text{линейность}}
	\mathtt B_{rR}(0) \sub U \bigl( \mathtt B_R(0) \bigr) $$
	Сдвинем на $ U(x_0) $:
	$$ \underbrace{Ux_0 + \mathtt B_{rR}(0)}_{\mathtt B_{rR}(Ux_0) \sub
	U(\mathtt B_R(x_0)) \sub U(G)} \sub
	Ux_0 + U \bigl( \mathtt B_R(0) \bigr) = U \bigl( x_0 + \mathtt B_R(0) \bigr) =
	U \bigl( \mathtt B_R(x_0) \bigr) $$
	$ \implies Ux_0 $ "--- внутренняя точка $ U(G) $.
\end{iproof}

\begin{implication}
	$ (X, \|\cdot\|), ~ (Y, \|\cdot\|), \quad U \in \mathscr Lin(X, Y) $

	$$ U \text{ открыто } \implies U(X) = Y \text{, \ie $ U $ "--- сюръекция} $$
\end{implication}

\begin{proof}
	По критерию $ \exists r > 0 : \quad \mathtt B_r(0) \sub U \bigl( \mathtt B_1(0) \bigr) $.
	Возьмём $ y \in Y $.
	$$ \exists n \in \mathbb N : \quad \|y\| < nr \implies y \in \mathtt B_{nr}(0) \sub
	U \bigl( \mathtt B_n(0) \bigr) \implies y \in U(X) $$
\end{proof}

\section{Теорема Банаха об открытом отображении}

\begin{lemma}[редукция]
	$ X $ "--- банахово, $ \quad (Y, \|\cdot\|), \quad U \in \mathscr B(X, Y), \quad
	\exists r > 0 : \quad \mathtt B_r(0) \sub \ol{U \bigl( \mathtt B_1(0) \bigr)} $

	$$ \mathtt B_{\frac r2}(0) \sub U \bigl( \mathtt B_1(0) \bigr) $$
\end{lemma}

\begin{proof}
	Пусть $ y \in \mathtt B_{\frac r2}(0) $, \ie $ \|y\| < \frac r2 $.
	Докажем, что $ \exists x : \quad \|x\| < 1, \quad Ux = y $.
	$$ \mathtt B_r(0) \sub \ol{U \bigl( \mathtt B_1(0) \bigr)} \implies
	\forall k \in \mathbb N \quad \mathtt B_{\frac r {2^k}}(0) \sub
	\ol{U \bigl( \mathtt B_{\frac1{2^k}}(0) \bigr)} $$
	\begin{itemize}
		\item При $ k = 1 $
			$$ \|y\| < \frac r 2 \implies y \in \mathscr B_{\frac r 2}(0) \sub
			\ol{U \bigl( \mathtt B_{\frac12}(0) \bigr)} $$
			$$ \exists x_1 : \quad \|x_1\| < \frac12, \quad \|y - Ux_1\| < \frac r 4 $$
		\item $ k = 2 $
			$$ y - Ux_1 \in \mathtt B_{\frac r 4}(0) \sub
			\ol{U \bigl( \mathtt B_{\frac14}(0) \bigr)} \implies
			\exists x_2 : \quad \|x_2\| < \frac14, \quad \|y - Ux_1 - Ux_2\| < \frac r {2^3} $$
			$$ \dots $$
		\item Построим $ \Set{x_n}_{n = 1}^\infty $ такую, что
			$$ \|x_n\| < \frac1{2^n}, \quad \|y - Ux_1 - \dots - Ux_n\| < \frac r {2^{n + 1}} $$
	\end{itemize}
	$$ \sum_{n = 1}^\infty \|x_n\| < 1 \underimp{\text{критерий полноты}}
	\exists x = \sum_{n = 1}^\infty x_n \in X, \quad \|x\| < 1 $$
	$$ S_n \coloneq \sum_{k = 1}^n x_k $$
	$$
	\begin{rcases}
		\lim S_n = x \implies \lim US_n = Ux \\
		\lim \|y - US_n\| = 0 \implies \lim US_n = y
	\end{rcases} \implies Ux = y $$
\end{proof}

\begin{theorem}[Банах]
	$ X, Y $ "--- банаховы, $ \quad U \in \mathscr B(X, Y) $

	Если $ U(X) = Y $, то $ U $ открыто.
\end{theorem}

\begin{proof}
	Обозначим $ B_r = \mathtt B_r(0) $.
	$$ X = \bigcup_{n = 1}^\infty (B_n) \implies U(X) = Y = \bigcup_{n = 1}^\infty U(B_n) $$
	По теореме Бэра о категориях
	$$ \exists n_0 \in \mathbb N : \quad \operatorname{Int} \ol{U(B_{n_0})} \ne \emptyset $$
	При этом, $ U(B_{n_0}) = n_0U(B_1) $ (в $ n_0 $ раз увеличенное $ U(B_1) $).
	$$ \implies \operatorname{Int} \ol{U(B_1)} \ne \emptyset \implies
	\exists a \in Y ~ \exists y > 0 : \quad \mathtt B_r(a) \sub \ol{U(B_1)} $$

	Чтобы воспользоваться леммой, нужно сдвинуть точку $ a $ в 0.

	Возьмём $ z \in X : \quad \|z\| < r $.
	$$
	\begin{rcases}
		a + z \in \mathscr B_r(a) \sub \ol{U(B_1)} \\
		a \in \ol{U(B)} \implies -a \in \ol{U(B_1)}
	\end{rcases} \implies z = (a + z) + (-a) \in
	\ol{U(B)} + \ol{U(B_1)} \sub \ol{U(B_1)} $$
	Воспользуемся подобием:
	$$ \mathtt B_{\frac r2}(0) \sub \ol{U(B)} \underimp{\text{лемма}}
	\mathtt B_{\frac r4}(0) \sub U(B) $$
\end{proof}

\section[Теорема Банаха об обратном операторе.
Эквивалентность норм, в которых пространство банахово]
{Теорема Банаха об обратном операторе.
Эквивалентность \\
норм, в которых пространство банахово}

\begin{theorem}[Банах]
	$ X, Y $ "--- банаховы, $ \quad U \in \mathscr B(X, Y) $ "--- биекция

	$$ U^{-1} \in \mathscr B(Y, X) $$
\end{theorem}

\begin{proof}
	$$ U(X) = Y \underimp{\text{т. об откр. отобр.}} U \text{ открыто } \underimp{\text{утв.}}
	U^{-1} \text{непрерывно} $$
\end{proof}

\begin{theorem}[об эквивалентных нормах]
	$ (X, \|\cdot\|_1), ~ (X, \|\cdot\|_2) $ "--- банаховы, $ \quad
	\exists C > 0 : \quad \|x\|_2 \overset X\le C\|x\|_1 $

	$$ \exists A > 0 : \quad \|x\|_1 \overset X \le A\|x\|_2 $$
\end{theorem}

\begin{proof}
	Обозначим $ X = (X, \|\cdot\|_1), \quad Y = (X, \|\cdot\|_2) $.

	Определим оператор $ I : X \to Y : \quad Ix = x $.
	Понятно, что $ I $ "--- биекция и $ I \in \mathscr Lin(X, Y) $.
	$$ \|Ix\|_2 \le C\|x\|_1 \implies I \in \mathscr B(X, Y) $$

	По теореме Банаха об обратном операторе
	$$ I^{-1} \in \mathscr B(Y, X) \implies \|I^{-1}x\|_1 \le A\|x\|_2, \quad A = \|I^{-1}\| $$
	$$ \iff \|x\|_1 \le A\|x\|_2 $$
\end{proof}

\section{Замкнутый оператор.
Теорема о замкнутом графике.
Пример замкнутого, но не непрерывного оператора}

\begin{definition}
	$ (X, \|\cdot\|), ~ (Y, \|\cdot\|) $ над $ \mathbb K $

	$ X \times Y $ "--- \emph{линейное пространство}:
	$$ (x_1, y_1) + (x_2, y_2) \coloneq (x_1 + x_2, y_1 + y_2) $$
	$$ \alpha (x, y) = (\alpha x, \alpha y) $$

	Определим \emph{норму} на $ X \times Y $:
	$$ \|(x, y)\|_{X \times Y} = \|x\|_X + \|y\|_Y $$
\end{definition}

\begin{remark}
	Сходимость по такой норме "--- покоординатная:
	$$ \lim\limits_{n \to \infty}(x_n, y_n) = (x, y) \iff
	\begin{cases}
		\lim x_n = x \\
		\lim y_n = y
	\end{cases} $$

	Если $ X, Y $ "--- банаховы, то $ X \times Y $ "--- банахово.
\end{remark}

\begin{definition}
	$ (X, \|\cdot\|), ~ (Y, \|\cdot\|), \quad U \in \mathscr Lin(X, Y) $

	Определим \emph{график} $ U $:
	$$ G_U = \Set{(x, Ux)}_{x \in X} \sub X \times Y $$

	Оператор $ U $ называется \emph{замкнутым}, если $ G_U $ замкнуто в $ X \times Y $.
	$$ \iff \Bigl( \lim(x_n, Ux_n) = (x, y) \implies y = Ux \Bigr) $$
	$$ \iff \left(
		\begin{rcases}
			\lim x_n = x \\
			\lim Ux_n = y
		\end{rcases} \implies y = Ux \right) $$
\end{definition}

\begin{remark}
	Замкнутый оператор \textbf{не тот}, который переводит замкнутые множества в замкнутые.
\end{remark}

\begin{remark}
	$ (X, \|\cdot\|), ~ (Y, \|\cdot\|), \quad U \in \mathscr Lin(X, Y) $

	\begin{enumerate}
		\item $ \lim x_n = x $;
		\item $ \lim Ux_n = y $;
		\item $ y = Ux $.
	\end{enumerate}

	$$ U \text{ замкнут } \iff \Bigl( 1) + 2) \implies 3) \Bigr) $$
	$$ U \text{ непрерывен } \iff \Bigl( 1) \implies 2) + 3) \Bigr) $$

	Значит, если $ U $ непрерывен, то он замкнут.
\end{remark}

\begin{theorem}[о замкнутом графике]
	$ X, Y $ "--- банаховы, $ \quad U \in \mathscr Lin(X, Y) $ "--- замкнутый

	$$ U \text{ непрерывен} $$
\end{theorem}

\begin{proof}
	Определим новую норму пространства $ X $:
	$$ \|x\|_1 = \|x\|_X + \|Ux\|_Y $$

	Проверим, что $ (X, \|\cdot\|_1) $ банахово.
	Пусть $ \Set{x_n}_{n = 1}^\infty $ фундаментальна в $ (X, \|\cdot\|_1) $, \ie
	$$ \lim\limits_{n \to \infty}\|x_m - x_n\|_1 = 0 $$
	$$ \iff \lim(\|x_m - x_n\|_X + \|Ux_m + Ux_n\|_Y) = 0 $$
	Значит, $ \Set{x_n} $ фундаментальна в $ (X, \|\cdot\|_X) $, и $ \Set{Ux_n} $ фундаментальна в
	$ Y $.
	Оба эти пространства банаховы.
	$$ \implies \exists x \in X ~ \exists y \in Y : \quad \lim \|x_n - x\|_X = 0 $$
	$$ \implies
	\begin{cases}
		\lim x_n = x \text{ в } (X, \|\cdot\|_X) \\
		\lim Ux_n = y \text{ в } Y
	\end{cases} \underimp{U \text{ замкнут}} y = Ux \implies
	\lim\bigl(\|x_n - x\|_X + \|Ux_n - Ux\|_Y \bigr) = 0 $$
	\begin{multline*}
		\implies \lim\|x_n - x\|_1 = 0 \implies (X, \|\cdot\|_1) \text{ банахово}, \quad
		\|x\|_X \le \|x\|_X + \|Ux\|_Y = \|x\|_1 \underimp{\text{т. об экв. нормах}} \\
		\implies \exists A > 0 : \quad \|x\|_1 \le A\|x\|_X \implies
		\|x\|_X + \|Ux\|_Y \le A\|x\|_X \implies \|Ux\| \le A\|x\| \implies U \in \mathscr B(X, Y)
	\end{multline*}
\end{proof}

\begin{eg}[$ U $ замкнут, но не непрерывен]
	$ X = \Set{\exists f' \in \mathcal C[0, 1]}, \quad
	\|f\|_X = \max\limits_{[0, 1]}|f(x)| $ \\
	$ X \sub \mathcal C[0, 1] $ в алгебраическом смысле

	$ Y = \mathcal C[0, 1], \quad \|g\|_Y = \max\limits_{[0, 1]}|g(x)| = \|g\|_\infty $

	$$ \mathcal D(f) = f', \quad \mathcal D \in \mathscr Lin(X, Y), \quad
	\mathcal D \text{ замкнут} $$
	$$
	\begin{rcases}
		\Set{f_n \in X}_{n = 1}^\infty, \quad \lim f_n = f \iff f_n \uniarr{[0, 1]} f \\
		\mathcal D(f_n) = f_n', \quad \mathcal D(f_n) \to g
	\end{rcases} \underimp{\text{док. в анализе}} g = f' \implies g = \mathcal D(f) \iff
	f_n' \uniarr{[0, 1]} g $$
	$$ x^n \in X : \quad \|x^n\|_X = 1, \quad \mathcal D(x^n) = nx^{n - 1}, \quad
	\|\mathcal D(x^n)\| = n $$
	$$ \implies \sup\limits_{\|f\| \le 1} \|D(f)\| = +\infty \implies
	\mathcal D \not\in \mathscr B(X, Y) $$
\end{eg}

\section(Сопряжённые пространства к пространствам последовательностей)
{Сопряжённые пространства к пространствам последовательностей $ C_0 $, $ l^p $, пространству
$ \mathrm L^p $ для конечных $ p $}

\begin{theorem}[сопряжённые к $ l^p $]
	$ 1 \le p < +\infty, \quad \frac1p + \frac1q = 1 $
	\begin{enumerate}
		\item $ y = \Set{y_n}_{n = 1}^\infty \in l^q, \quad F_y : l^p \to \mathbb K, \quad
			x = \Set{x_n}_{n = 1}^\infty \in l^p, \quad
			F_y(x) \coloneq \sum_{n = 1}^\infty x_ny_n $

			$$ F_y \in \bigl( l^p \bigr)^*, \quad \|F_y\|_{(l^p)^*} = \|y\|_q $$
		\item $ F \in \bigl( l^p \bigr)^* $
			$$ \exists ! y : \quad F = F_y $$
	\end{enumerate}
\end{theorem}

\begin{eproof}
\item $ y \in l^q, \quad x \in l^p $
	$$ |F_y(x)| = \Bigl| \sum_{n = 1}^\infty x_ny_n \Bigr| \underset{\text{нер-во Гёльдера}}\le
	\|x\|_p \cdot \|y\|_q, \quad F_y \in \bigl( l^p, \mathbb K \bigr) $$
	$$ \implies F_y \in \mathscr B \bigl(l^p, \mathbb K \bigr) = \bigl( l^p \bigr)^*, \quad
	\|F_y\|_{(l^p)^*} \le \|y\|_q $$
\item $ F \in \bigl( l^p \bigr)^*, \quad
	e_n = (0, \dots, 0, \underset n 1, 0, \dots, 0) $ "--- базис $ l^p $

	Определим $ y_n = F(e_n) $.
	$$ x = \Set{x_n}_{n = 1}^\infty \in l^p \implies
	x = \sum_{n = 1}^\infty x_ne_n \text{ "--- сходится в } l^p $$
	\begin{multline*}
		S_n \coloneq \sum_{k = 1}^n x_ke_k \implies
		F(S_n) = \sum_{k = 1}^n x_ky_k \underimp{F \text{ непрерывен}} \\
		\implies \lim S_n = x \implies
		\lim F(S_n) = F(x) \implies F(x) = \sum_{k = 1}^\infty x_ky_k \implies
		F = F_y
	\end{multline*}

	Проверим, что $ y \in l^q $.
	\begin{itemize}
		\item $ p > 1 \implies q < +\infty $
			Рассмотрим пробные функции
			$$ x_k =
			\begin{cases}
				\frac{\ol y_k}{|y_k|} \cdot |y_k|^{q - 1}, \quad y_k \ne 0, \\
				0, \quad y_k = 0
			\end{cases} $$

			Рассмотрим последовательности
			$$ \nder x = \sum_{k = 1}^n x_ke_k = (x_1, \dots, x_n, 0, \dots) $$
			$$ F \bigl( \nder x \bigr) = \sum_{k = 1}^n x_ky_k = \sum |y_k|^q $$
			$$ \bigl\| \nder x \bigr\|_p^p = \sum_{k = 1}^n |x_k|^p =
			\sum |y_k|^{p(q - 1)} \undereq{\frac1p + \frac1q = 1 \implies p + q =
			pq \implies pq - p = q} \sum_{k = 1}^n |y_k|^q $$
			\begin{multline*}
				\|F\| = \sup\limits_{x \ne 0} \frac{|F(x)|}{\|x\|_p} \ge
				\frac{F(\nder x)}{\|\nder x\|} =
				\frac{\sum_{k = 1}^n |y_k|^q}{ \bigl( \sum_{k = 1}^n |y_k|^q \bigr)^{\frac 1p}} =
				\Bigl( \sum_{k = 1}^n |y_k|^q \Bigr)^{1 - \frac1p} =
				\Bigl( \sum |y_k|^q \Bigr)^{\frac1q} \quad \forall n \implies \\
				\implies y \in l^q, \quad \|F\| \ge \|y\|_q, \quad F = F_y
			\end{multline*}

			Докажем единственность: \\
			\textbf{Пусть} $ F = F_y, ~ F = F_z $.
			$$ F(e_n) = y_n, \quad F(e_n) = z_n \implies y_n = z_n $$
		\item $ p = 1 \implies q = \infty $
			$$ F(e_n) = y_n, \quad \|F\|_{(l^1)^*} \ge |y_n| \implies y \in l^\infty $$
			$$ \|F\| \ge \sup\limits_{n \in \mathbb N}|y_n| = \|y\|_\infty $$
	\end{itemize}
\end{eproof}

\begin{remark}
	$ T : (l^q) \to \bigl( l^p \bigr)^* : \quad T(y) = F_y $

	Мы доказали, что, при $ 1 \le p < +\infty $, $ T $ "--- линейный изометрический изоморфизм.
	Говорят, что $ \bigl( l^p \bigr)^* = l^q $.
\end{remark}

\begin{theorem}[сопряжённое к $ C_0 $]
	$ C_0 = \Set{ x = \Set{x_n \in \mathbb K}_{n = 1}^\infty |
	\exists \lim\limits_{n \to \infty} x_n = 0} $
	\begin{enumerate}
		\item $ y \in l^1, \quad x \in C_0, \quad F_y(x) \coloneq \sum_{n = 1}^\infty x_ny_n $
			$$ F_y \in C_0^*, \quad \|F_y\| = \|y\|_1 $$
		\item $ F \in C_0^* $
			$$ \exists ! y \in l^1 : \quad F = F_y $$
	\end{enumerate}
\end{theorem}

\begin{eproof}
\item $ y \in l^1, \quad x \in C_0 $
	$$ |F_y(x)| = \Bigl| \sum_{n = 1}^\infty x_ny_n \Bigr| \le
	\sup\limits_{n \in \mathbb N}|x_n| \cdot \sum_{n = 1}^\infty |y_n| =
	\|x\|_\infty \cdot \|y\|_1 $$
	$$ \implies F_y \in C_0^*, \quad \|F_y\| \le \|y\|_1 $$
\item $ \Set{e_n}_{n = 1}^\infty $ "--- базис в $ C_0 $.
	Рассмотрим $ F \in C_0^* $.
	$$ y_n \coloneq F(e_n) $$
	$$ x = \Set{x_n}_{n = 1}^\infty \in C_0 \implies x = \sum_{k = 1}^\infty x_ke_k $$
	$$ S_n \coloneq \sum_{k = 1}^n x_ke_k $$
	$$ F(S_n) = \sum_{k = 1}^n x_k y_k \underimp{F \text{ непрерывен}}
	\lim F(S_n) = F(x) \implies F(x) = \sum_{k = 1}^\infty x_ky_k \text{ сходится } \implies
	F = F_y $$

	Проверим, что $ y \in l^1 $.
	$$ \nder x \coloneq \sum_{k = 1}^n \frac{\ol{y_k}}{|y_k|} e_k \implies \nder x \in C_0, \quad
	\|\nder x\|_\infty = 1 $$
	$$ \implies F(\nder x) = \sum_{k = 1}^n y_k \frac{\ol{y_k}}{|y_k|} = \sum |y_k| $$
	$$ \|F\| \ge |F(\nder x)| = \sum_{k = 1}^n |y_k| \quad \forall n \in \N \implies y \in l^1 $$
\end{eproof}

\begin{theorem}[сопряжённое к $ \mathrm L^p $]
	\hfill
	\begin{enumerate}
		\item $ 1 \le p \le +\infty, \quad g \in \mathrm L^q(X, \mu), \quad \frac1p + \frac1q = 1,
			\quad g $ "--- фиксирована, $ \quad h \in \mathrm L^p $
			$$ F_g(h) \coloneq \int\limits_X h(x) g(x) \di \mu $$

			$$ \implies F_g \in (\mathrm L^p)^*, \quad \|F_g\| = \|g\|_{\mathrm L^p} $$
		\item $ 1 \le p < +\infty, \quad F \in (\mathrm L^p)^* $

			$$ \exists! g \in \mathrm L^p : \quad F = F_g $$
	\end{enumerate}
\end{theorem}

\begin{proof}
	Докажем только первое утверждение.

	Очевидно, что $ F_g \in \mathscr Lin(\mathrm L^p, \Co) $, так как интеграл линеен.

	Для $ f \in \mathrm L^q $, $ h \in \mathrm L^p $,
	$$ |F_g(h)| = \Bigl| \int\limits_X hg \di \mu \Bigr| \underset{\text{нер-во Гёльдера}}\le
	\|h\|_p \|g\|_q \quad \forall h \in \mathrm L^p \implies
	\|F_g\| \le \|g\|_q $$

	\begin{itemize}
		\item $ 1 < p \le +\infty $

			Рассмотрим пробные функции
			$$ U(x) =
			\begin{cases}
				\frac{\ol{g(x)}}{|g(x)|} |g(x)|^{q - 1}, \quad g(x) \ne 0, \\
				0, \quad g(x) = 0
			\end{cases} $$

			Проверим, что $ U \in \mathrm L^p $:
			$$ |U(x)|^p = |g(x)|^{p(q - 1)} \undereq{(q - 1)p = q(1 - \frac1q)p = q \frac1p p = q}
			|g|^q \implies \Bigl( \int\limits_X |U|^p \di \mu \Bigr)^{\frac1p} =
			\Bigl( \int\limits_X |g|^q \di \mu \Bigr)^{\frac1p} \implies U \in \mathrm L^p $$
			$$ F_g(U) = \int\limits_X g(x) \frac{\ol{g(x)}}{|g(x)|} |g(x)|^{q - 1}\di \mu =
			\int\limits_X |g|^q \di \mu = \|g\|^q $$
			$$ \|F_g\| = \sup\limits_{h \ne 0} \frac{\|F_g(h)\|}{\|h\|} \ge \frac{|F_g(U)|}{\|U\|} =
			\frac{\|g\|^q}{\|g\|^{\frac q p}} = \|g\| $$
		\item $ p = 1, ~ q = \infty $

			\begin{itemize}
				\item $ \|g\|_\infty = 0 \implies g = 0 $ \ale $ \implies F_g = \On, \quad
					\|F_g\| = 0 $;
				\item $ \|g\|_\infty > 0 $

					Возьмём $ c > 0: \quad \|g\| > c $.
					$$ A \coloneq \Set{x \in X | {} |g(x)| \ge c} \implies +\infty > \mu(A) > 0 $$

					Докажем, что такое $ A $ вообще существует.
					Возьмём $ e \sub A $.
					$$ X = \bigcup_{j = 1}^\infty X_j, \quad \mu(X_j) < +\infty \implies
					0 < \mu e < +\infty $$
					$$ \implies A = \bigcup_{j = 1}^\infty (A \cap X_j), \quad
					e_j = A \cap X_j \implies \mu e_j < +\infty $$
					$$ \implies e = e_j, \quad 0 < \mu e < +\infty, \quad e \sub A $$
					$$ U(x) = \frac{\ol{g(x)}}{|g(x)|} \chi_e(x) \implies \|U\|_\infty = 1 $$
					$$ F_g(U) = \int\limits_X g(x) \frac{\ol{g(x)}}{|g(x)|}\chi_e \di \mu =
					\int\limits_e |g(x)| \di \mu \ge c \mu(e) $$
					$$ U \in \mathrm L^1, \quad \|U\|_1 = \int\limits_X |U(x)| \di \mu =
					\int\limits_e \di \mu = \mu(e) $$
					$$ \|F_g\| \ge \frac{|F_g(U)|}{\|U\|_1} \ge \frac{c\mu(e)}{\mu(e)} = c \quad
					\forall 0 < c < \|g\|_\infty $$
					$$ \implies \|F_g\| \ge \|g\|_\infty $$
			\end{itemize}
	\end{itemize}
\end{proof}

\section{Второе сопряжённое пространство.
Каноническое вложение.
Рефлексивность.
Примеры}

\begin{definition}\label{def:canon-enclos}
	$ (X, \|\cdot\|), ~ (Y, \|\cdot\|), \quad x \in X, \quad f \in X^* $
	$$ \Braket{f, x} \coloneq f(x) $$

	$ T \in \mathscr B(X, Y), \quad T^* : Y^* \to X^* $
	$$ \Braket{T^*f, x} \coloneq \Braket{f, Tx} $$

	$ \pi : X \to X^{**}, \quad f \in X^*, \quad x \in X $
	$$ \Braket{\pi(x), f} \coloneq f(x) $$
	Можно отождествить $ \pi(X) $ с $ X $:
	$$ \Braket{x, f} = f(x) $$
\end{definition}

\begin{figure}[ht]
	\centering
	\begin{tikzpicture}[>=Stealth]
		\draw[->] (0, 0) -- (2, 0);
		\draw[->] (0, 0) -- (1, -1);
		\draw[->] (2, 0) -- (1, -1);

		\node at (0, 0) [anchor=east] {$ X $};
		\node at (2, 0) [anchor=west] {$ X^{**} $};
		\node at (1, -1) [anchor=north] {$ \Co $};

		\node at (1, 0) [anchor=south] {$ \pi $};
		\node at (0.5, -0.5) [anchor=east] {$ f(x) $};
		\node at (1.5, -0.5) [anchor=west] {$ \bigl( \pi(x) \bigr)(f) $};
	\end{tikzpicture}
	\caption{Иллюстрация к \autoref{def:canon-enclos}}
\end{figure}

\begin{eg}
	$ \mathrm L^p(\mu), \quad \mathrm L^q(\mu), \quad 1 < p < +\infty, \quad
	\frac1p + \frac1q = 1, \quad f \in \mathrm L^p, \quad g \in \mathrm L^q $

	$$ \Braket{f, g} = \int\limits_X f(x)g(x) \di \mu, \quad
	\Braket{g, f} = \int\limits_X f(x) g(x) \di \mu $$
\end{eg}

\begin{property}[канонического вложения]
	$ \pi : X \to X^{**} $

	$$ \pi \in \mathscr B(X, X^{**}), \quad \pi \text{ "--- изометрическое вложение} $$
	(\ie $ \|\pi(x)\| = \|x\| $)
\end{property}

\begin{proof}
	Зафиксируем $ x \in X $.
	\begin{itemize}
		\item Проверим, что $ \pi(x) \in \mathscr Lin(X^*, \Co) $

			Для $ \alpha \in \Co $ и $ f \in X^* $
			$$ \bigl( \pi(x) \bigr)(f) \bydef = (\alpha f)(x) = \alpha f(x) =
			\alpha \bigl(\pi(x) \bigr)(f) $$

			Для $ f, g \in X^* $ имеем по определению
			$$ \bigl(\pi(x) \bigr)(f + g) = (f + g)(x) = f(x) + g(x) =
			\bigl(\pi(x) \bigr)(f) + \bigl(\pi(x) \bigr)(g) $$
			Значит, $ \pi(x) \in \mathscr Lin(X^*, \Co) $.
		\item Проверим, что $ \|\pi(x)\| = \|x\| $
			$$ \bigl| \bigl( \pi(x) \bigr)(f) \bigr| =
			|f(x)| \le \|f\| \cdot \|x\| \quad \forall f \in X^* \implies
			\pi(x) \in (X^*)^*, \quad \|\pi(x)\| \le \|x\| $$

			Неравенство в другую сторону имеем по следствию о достаточном числе линейных
			функционалов:
			$$ \exists g \in X^* : \quad \|g\| = 1, \quad g(x) = \|x\| $$
			(для фиксированного $ x \in X $)
			$$ \|\pi(x)\| =
			\sup\limits_{f \in X^* : ~ \|f\| \le 1} \bigl| \bigl(\pi(x)\bigr)(f) \bigr|
			\underset{\text{можно взять такое } g}\ge
			\bigl| \bigl(\pi(x)\bigr)(g)\bigr| = |g(x)| = \|x\| $$
		\item Проверим, что $ \pi \in \mathscr Lin(X, X^{**}) $
			\begin{itemize}
				\item Возьмём $ \alpha \in \Co, ~ x \in X, ~ f \in X^* $.
					$$ \bigl(\pi(\alpha x) \bigr)(f) = f(\alpha x) = \alpha f(x) =
					\alpha \cdot \bigl(\pi(x)\bigr)(f) \quad \forall f \in X^* \implies
					\pi(\alpha x) = \alpha \pi(x) $$

				\item Возьмём $ x, y \in X, ~ f \in X^* $.
					\begin{multline*}
						\bigl(\pi(x + y)\bigr)(f) = f(x + y) = f(x) + f(y) =
						\bigl(\pi(x)\bigr)(f) + \bigl(\pi(y)\bigr)(f) \quad
						\forall f \in X^* \implies \\
						\implies \pi(x + y) = \pi(x) + \pi(y) \underimp{\|\pi(x)\| = \|x\|}
						\pi \in \mathscr B(X, X^{**}), \quad \|\pi\| = 1
					\end{multline*}
			\end{itemize}
	\end{itemize}
\end{proof}

\begin{implication}[как строить пополнение нормированного пространства]
	$ (X, \|\cdot\|), \quad Y = \ol{\pi(X)} $

	Тогда $ Y $ "--- пополнение $ X $.
\end{implication}

\begin{proof}
	$ X^* $ "--- банахово $ \implies X^{**} $ банахово $ \implies \pi : X \to X^{**} $ "---
	изометрическое вложение $ \implies \ol{\pi(X)} $ "--- пополнение $ X $.
\end{proof}

\begin{definition}
	Пространство $ X $ называется \emph{рефлексивным}, если $ \pi(X) = X^{**} $, \ie $ \pi $
	сюръективно, \ie
	$$ \forall G \in (X^*)^* \quad \exists x \in X : \quad G = G_x \quad
	\bigl( \text{или } G = \pi(x) \bigr) $$
\end{definition}

\begin{implication}
	$ X $ рефлексивно.

	Тогда $ X $ "--- банахово.
\end{implication}

\begin{implication}
	$ X $ рефлексивно, $ \quad f \in X^* $

	$$ \|f\| = \max\limits_{\|x\| \le 1}|f(x)| $$
	(в случае рефлексивного пространства в норме можно супремум заменить на максимум)
\end{implication}

\begin{proof}
	Вспомним, что для любого нормированного $ X $
	$$ \|f\| = \sup\limits_{\|x\| \le 1}|f(x)|, \quad
	\|x\| = \max\limits_{\|g\|_{X^*} \le 1}|g(x)| $$
	$$ f \in X^* \implies \|f\| = \max\limits_{\phi \in X^{**} : \|\phi\| = 1}|\phi(f)|
	\undereq{\text{рефлексивность}} \max\limits_{\|x\| = 1} \bigl| \bigl(\pi(x) \bigr)(f) \bigr| =
	\max\limits_{\|x\| = 1}|f(x)| $$
\end{proof}

\begin{exmpls}
\item $ 1 < p < +\infty $

	$ \mathrm L^p(X, \mathscr U, \mu) $ "--- рефлексивные.
	$$ (\mathrm L^p)^* \sim \mathrm L^q, \quad 1 < q < +\infty $$
	$$ (\mathrm L^q)^* \sim \mathrm L^p $$
\item $ \mathrm L^1 $ и $ \mathrm L^\infty $ не рефлексивны.
\item $ C_0 $ и $ l^1 $ не рефлексивны.
	$$ (C_0)^* \sim l^1 $$
	$$ (l^1)^* \sim l^\infty $$
\item $ H $ "--- гильбертово

	$ H $ рефлексивно.

	$ H^* $ сопряжённо линейно изоморфно $ H $.
	Тогда $ H^{**} $ сопряжённо линейно изоморфно $ H^* \implies H^{**} \sim H $.
\item $ \mathcal C(\mathbb K) $ не рефлексивно (без доказательства).
\end{exmpls}
