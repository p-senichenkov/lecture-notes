\part{Сопряжённые операторы}

\section{Существование и простейшие свойства сопряжённого оператора в нормированном пространстве}

\begin{definition}\label{def:conj-oper}
	$ (X, \|\cdot\|), ~ (Y, \|\cdot\|), \quad T \in \mathscr B(X, Y) $.

	Определим \emph{сопряжённый оператор} к $ T $:
	$$ T^* : Y^* \to X^*: \quad (T^*f)(x) = f(Tx) $$
\end{definition}

\begin{figure}[h]
	\centering
	\begin{tikzpicture}[>=Stealth]
		\draw[->] (0, 0) -- (2, 0);
		\draw[->] (0, 0) -- (1, -1);
		\draw[->] (2, 0) -- (1, -1);

		\node at (0, 0) [anchor=east] {$ X $};
		\node at (2, 0) [anchor=west] {$ Y $};
		\node at (1, -1) [anchor=north] {$ \Co $};

		\node at (1, 0) [anchor=south] {$ T $};
		\node at (0.5, -0.5) [anchor=east] {$ T^*(f) $};
		\node at (1.5, -0.5) [anchor=west] {$ f $};
	\end{tikzpicture}
	\caption{Иллюстрация к \autoref{def:conj-oper}}
\end{figure}

\begin{statement}
	$ T^* $ линеен
\end{statement}

\begin{proof}
	$ f $ линеен, ~ $ T $ линеен $ \implies T^* $ линеен.
\end{proof}

\begin{properties}
	$ (X, \|\cdot\|), ~ (Y, \|\cdot\|), \quad T \in \mathscr B(X, Y) $
	\begin{enumerate}
		\item $ T^* \in \mathscr B (Y^*, X^*), \quad \|T^*\| = \|T\| $;
		\item $ \alpha \in \Co $
			$$ (\alpha T)^* = \alpha T^* $$
		\item $ S, T \in \mathscr B(X, Y) $
			$$ (S + T)^* = S^* + T^* $$
		\item $ X \xrightarrow{T} Y \xrightarrow{S} Z $
			$$ (ST)^* = T^*S^* $$
	\end{enumerate}
\end{properties}

\begin{eproof}
\item
	\begin{itemize}
		\item Линейность

			Возьмём $ f, g \in Y^*, ~ \alpha \in \Co, ~ x \in X $.
			\begin{multline*}
				\bigl( T^*(\alpha f + g) \bigr)(x) = (\alpha f + g)(Tx) = \alpha f(Tx) + g(Tx) =
				\alpha (T^*f)(x) + (T^* g)(x) \quad \forall x \in X \implies \\
				\implies \alpha T^* f + T^* g = T^*(\alpha f + g)
			\end{multline*}
		\item Равенство норм
			$$ \|T^*\| = \sup\limits_{f \in Y^* : ~ \|f\| < 1} \|T^*f\| =
			\sup\limits_{\|f\| < 1} \bigl( \sup\limits_{\|x\| < 1}|(T^*f)(x)| \bigr) \bydef{T^*}
			\sup\limits_{\|x\| < 1} \bigl( \sup\limits_{\|f\| < 1}|f(Tx)| \bigr) $$
			Но, по следствию о достаточном числе линейных функционалов,
			$ \sup\limits_{\|f\| < 1}|f(Tx)| = \|Tx\| $.
			$$ \|T^*\| = \sup\limits_{\|x\| < 1}\|Tx\| = \|T\| $$
	\end{itemize}
\item $ \alpha \in \Co, \quad f \in Y^*, \quad x \in X $
	$$ \bigl( (\alpha T)^* f \bigr)(x) = f(\alpha Tx) = \alpha f(Tx) =
	\alpha \bigl( T^*(f) \bigr)(x) \quad \forall x \in X \quad \forall f \in Y^* \implies
	(\alpha T)^* = \alpha T^* $$
\item $ S, T \in \mathscr B(X, Y), \quad f \in Y^*, \quad x \in X $
	\begin{multline*}
		\bigl( (S + T)^* f \bigr)(x) = f \bigl( (S + T)x \bigr) = f(Sx) + f(Tx) =
		(S^*f)(x) + (T^*f)(x) \quad \forall x \in X \quad \forall f \in Y* \implies \\
		\implies (S + T)^* = S^* + T^*
	\end{multline*}
\item $ X \xrightarrow T Y \xrightarrow S Z $
	$$ ST : X \to Z, \quad (ST)^* : Z^* \to X^* $$

	Возьмём $ f \in Z^*, \quad x \in X $.
	\begin{multline*}
		\bigl( (ST)^*f \bigr)(x) = f \bigl( (ST)x \bigr) = f \bigl( S(Tx) \bigr) = (S^*f)(Tx) =
		\bigl( T^*(S^*f) \bigr)(x) \quad \forall x \in X \quad \forall z \in Z^* \implies \\
		\implies (ST)^* = T^* S^*
	\end{multline*}
\end{eproof}

\textcolor{gray}{Про существование найти не могу...}

\section{Теорема об интегральном операторе с ядром из
\texorpdfstring{$ \mathrm L^p $}{пространства Лебега} и сопряжённом к нему}

\begin{theorem}
	$ (\mathbb R^n, \lambda_n), ~ (\mathbb R^m, \lambda_m), \quad
	K(x, y) \in \mathrm L^p \bigl(\mathbb R^{n + m}, \lambda_{n + m} \bigr), \quad
	x \in \mathbb R^n, \quad y \in \mathbb R^m $
	$$ M \coloneq \Bigl( \int\limits_{\mathbb R^n} \int\limits_{\mathbb R^m}
	|K(x, y)|^p \di x \di y \Bigr)^{\frac1p} < +\infty $$
	$$ (\mathscr K f)(x) = \int\limits_{\mathbb R^m} K(x, y) f(y) \di y $$

	Тогда
	\begin{enumerate}
		\item $ \mathscr K \in
			\mathscr B \bigl( \mathrm L^q(\mathbb R^m), \mathrm L^p(\mathbb R^n) \bigr), \quad
			\|\mathscr K\| \le M $;
		\item $ \mathscr K^* \in
			\mathscr B \bigl( \mathrm L^q(\mathbb R^m), \mathrm L^p(\mathbb R^m) \bigr) $
			$$ (\mathscr K^* g)(y) = \int\limits_{\mathbb R^n} K^*(y, x) g(x) \di x $$
			$$ \implies K^*(y, x) = K(x, y) $$
	\end{enumerate}
\end{theorem}

\begin{eproof}
\item Воспользуемся теоремой Фубини:
	$$ \int\limits_{\R^n} \Bigl( \int\limits_{\R^m} |K(x, y)|^p \di y \Bigr) \di x < +\infty \implies
	\text{ для \ale \ } x \quad \int\limits_{\R^m} |K(x, y)|^p \di y < +\infty $$

	Возьмём $ f \in \mathrm L^q(\R^m) $ и $ x $ такой, что внутренний интеграл конечен.
	$$ |(\mathscr Kf)(x)| = \Bigl| \int\limits_{\R^m} K(x, y)f(y) \di y \Bigr|
	\underset{\text{нер-во Гёльдера}}\le
	\Bigl( \int\limits_{\R^m}|K(x, y)|^p \di y \Bigr)^{\frac1q} \cdot \|f\|_q $$
	Возведём в степень $ p $ и проинтегрируем по $ \R^n $:
	\begin{multline*}
		\int\limits_{\R^n}|(\mathscr Kf)(x)|^p \di x \le
		\|f\|_q^p \int\limits_{\R^n} \Bigl( \int\limits_{\R^m}|K(x, y)|^p \di y \Bigr) \di x =
		M^p \implies \\
		\implies \|\mathscr Kf\|_p \le M\|f\|_q \quad \forall f \in \mathrm L^q \implies
		\|\mathscr K\| \le M, \quad \mathscr K \in \mathscr B(\dots)
	\end{multline*}
\item $ \mathscr K^* \in \mathscr B \bigl( \mathrm L^q(\R^n), \mathrm L^p(\R^m) \bigr) $
	$$ g \in \mathrm L^q(\R^n), \quad f \in \mathrm L^q(\R^m) $$
	\begin{equ}{int_op:1}
		\Braket{\mathscr K^* g, f} = \Braket{g, \mathscr Kf} =
		\int\limits_{\R^n}g(x) \Bigl( \int\limits_{\R^m}K(x, y)f(y) \di y \Bigr) \di x
		\undereq{\text{т. Фубини}} \int\limits_{\R^m}
		\Bigl( \int\limits_{\R^n} K(x, y) g(x) \di x \Bigr) f(y) \di y
	\end{equ}
	$$ \Braket{ \mathscr K^* g, y} \bydef \int\limits_{\R^n} K^*(y, x) g(x) \di x $$
	$$ \Braket{\mathscr K^* g, f} = \int\limits_{\R^m}
	\Bigl( \int\limits_{\R^n} K^*(y, x) g(x) \Bigr) f(y) \di y \underimp{\eref{int_op:1}}
	K^*(y, x) = K(x, y) $$
\end{eproof}

\section{Существование и простейшие свойства эрмитово"=сопряжённого оператора в гильбертовом
пространстве}

$ H $ "--- гильбертово, $ \quad T \in \mathscr B(H), \quad y \in H $ "--- фиксирован,
$ \quad G_y : H \to \Co, \quad x \in H, \quad G_y(x) \coloneq (Tx, y) $

$$ \implies G_y \in \mathscr Lin(H, \Co) $$

Возьмём $ x \in H $.
$$ |G_y(x)| = \bigl| (Tx, y) \bigr| \le \|Tx\| \cdot \|y\| \le
\|T\| \cdot \|y\| \cdot \|x\| \implies
G_y \in H^*, \quad \|G_y\| \le \|T\| \cdot \|y\| $$

Воспользуемся теоремой Рисса:
$$ \exists ! z \in H : \quad G_y(x) = (x, z), \quad \|z\| = \|G_y\|_{H^*} $$

\begin{definition}
	$ T^*y = z $ будем называть \emph{эрмитово-сопряжённым} к $ T $.
\end{definition}

\begin{properties}
	$ H $ "--- гильбертово, $ \quad T \in \mathscr B(H) $

	\begin{enumerate}
		\item $ T^{**} = T $;
		\item $ T^* \in \mathscr B(H), \quad \|T^*\| = \|T\| $;
		\item $ \alpha \in \Co, \quad (\alpha T)^* = \ol \alpha T^* $;
		\item $ T, S \in \mathscr B(H) \implies (T + S)^* = T^* + S^* $;
		\item $ (TS)^* = S^* T^* $;
		\item если $ T $ "--- биекция, то $ \exists T^{-1} \in \mathscr B(H), \quad
			\exists (T^*)^{-1} $, при этом $ (T^*)^{-1} = (T^{-1})^* $.
	\end{enumerate}
\end{properties}

\begin{eproof}
\item $ (Tx, y) = (x, T^*y) $
	$$ (x, Ty) = \ol{(Ty, x)} = \ol{(y, T^*x)} = (T^*x, y) = (x, T^{**}y) \quad \forall x $$
	$$ \implies Ty = T^{**}y \quad \forall y \in H $$
\item $ T^* \in \mathscr B(H), \quad \|T^*\| \le \|T\| \implies
	\|T^{**}\| \le \|T^*\| \underimp{T^{**} = T} \|T^*\| = \|T\| $
\item $ \alpha \in \Co $
	$$ (Tx, y) = (x, T^*y) \implies \bigl( (\alpha T)x, y \bigr) = (x, \ol \alpha T^*y) \implies
	(\alpha T^*) = \ol \alpha T^* $$
\item Очевидно.
\item Очевидно.
\item $ \exists T^{-1} $
	$$
	\begin{rcases}
		T \cdot T^{-1} = I \underimp{5)} (T^{-1})^* T^* = (I^*) = I \\
		T^{-1} \cdot T = I \underimp{5)} (T^*) (T^{-1})^* = I
	\end{rcases} \implies \exists (T^*)^{-1} = (T^{-1})^* $$
\end{eproof}

\begin{remark}
	$ X, Y $ "--- банаховы, $ \quad T \in \mathscr B(X, Y), \quad
	\exists T^{-1} \in \mathscr B(Y, X) $

	$$ \exists (T^*)^{-1} = (T^{-1})^* $$
\end{remark}

\begin{noproof}
\end{noproof}

\section{Теорема об интегральном операторе с ядром из \texorpdfstring{$ \mathrm L^2 $}{пространства
Лебега} и эрмитово-сопряжённом к нему}

\begin{theorem}
	$ \mathrm L^2(\R^n, \lambda_n), \quad K(x, y) \in \mathrm L^2(\R^{2n}), $
	$$ M = \Bigl( \int\limits_{\R^n} \int\limits_{\R^n} |K(x, y)|^2 \di x \di y \Bigr)^{\frac12} $$
	$$ (\mathscr Kf)(x) = \int\limits_{\R^n} K(x, y) f(y) \di y $$

	Тогда
	\begin{enumerate}
		\item $ \mathscr K \in \mathscr B \bigl( \mathrm L^2(\R^n) \bigr), \quad
			\|\mathscr K\| \le M $
		\item $ \mathscr K^* \in \mathscr B \bigl( \mathrm L^2(\R^n) \bigr) $
			$$ (\mathscr K^*g)(y) = \int\limits_{\R^n} K^*(y, x) g(x) \di x $$
			$$ \implies K^*(y, x) = \ol{K(x, y)} $$
	\end{enumerate}
\end{theorem}

\begin{eproof}
\item Так же, как в предыдущей теореме.
\item Возьмём $ f, g \in \mathrm L^2(\R^n) $.
	$$ (f, g) = \int\limits_{\R^n} f(x)\ol{g(x)} \di x $$
	\begin{multline*}
		(\mathscr K^*g, f) = (g, \mathscr Kf) =
		\int\limits_{\R^n} g(x) \ol{\int\limits_{\R^n} K(x, y)f(y) \di y} \di x
		\undereq{\text{т. Фубини}}
		\int\limits_{\R^n} \Bigl( \int\limits_{\R^n} \ol{K(x, y)} g(x) \Bigr) \ol{f(x)} \di y = \\
		= \Bigl( \int\limits_{\R^n} \ol{K(x, y)} g(x) \di x, f \Bigr)
	\end{multline*}
	$$ \implies (\mathscr K^*g)(y) = \int\limits_{\R^n}\ol{K(x, y)}g(x) \di x \implies
	K^*(y, x) = \ol{K(x, y)} $$
\end{eproof}

\section{Теорема об образе и ядре оператора и его сопряжённого.
Теорема об ортогональном дополнении инвариантного подпространства.
Самосопряжённый оператор, примеры}

\begin{undefthm}{Общее наблюдение.}
	$ L $ "--- подпространство $ H $ в алгебраическом смысле,
	$$ M = L^\perp = \Set{x \in H | x \perp y \quad \forall y \in L} $$
	$$ (\ol L)^\perp = M $$
	$$ \implies H = M \oplus \ol L $$
\end{undefthm}

\begin{theorem}
	$ H $ "--- гильбертово, $ \quad T \in \mathscr B(H) $

	$$ H = (\operatorname{Ker} T) \oplus \ol{T^*(H)} = (\operatorname{Ker}T^*) \oplus \ol{T(H)} $$
\end{theorem}

\begin{proof}
	В качестве $ L $ возьмём $ L = T^*(H) $.
	Возьмём $ x \in L^\perp $.
	$$ (x, T^*y) = 0 \iff (Tx, y) = 0 \iff Ty = 0 \iff y \in \operatorname{Ker} M $$
	$$ \implies \bigl( T^*(H) \bigr)^\perp = \operatorname{Ker}T \implies H =
	\operatorname{Ker} T \oplus \ol{T^*(H)} $$

	Применим эту формулу к $ T^* $ (учитывая, что $ T^{**} = T $):
	$$ H = \operatorname{Ker} T^* \oplus \ol{T(H)} $$
\end{proof}

\begin{definition}
	$ (X, \|\cdot\|), \quad T \in \mathscr B(X), \quad Y \sub X $ "--- подпространство
	в алгебраическом смысле

	Будем говорить, что $ Y $ "--- \emph{инвариантное подпространство} для $ T $, если
	$ T(Y) \sub Y $ (\ie $ T\bigr|_Y \in \mathscr B(Y) $).
\end{definition}

\begin{theorem}
	$ H $ "--- гильбертово, $ \quad T \in \mathscr B(H), \quad Y $ "--- инвариантное
	подпространство для $ T $

	$$ Y^\perp \text{ "--- инвариантное подпространство для } T^* $$
\end{theorem}

\begin{proof}
	$ z \in Y^\perp, \quad y \in Y $
	$$ (y, T^*z) = (Ty, z) \undereq{
		\begin{subarray}{c}
			Ty \in Y \\
			z \in Y^\perp
		\end{subarray}
	} 0 \implies T^*z \in Y^\perp \implies T^*(Y^\perp) \sub Y^\perp $$
\end{proof}

\begin{definition}
	$ H $ "--- гильбертово, $ \quad T \in \mathscr B(H) $

	$ T $ называется \emph{самосопряжённым}, если $ T = T^* $, то есть
	$ (Tx, y) = (x, Ty) \quad \forall x \in H $.
\end{definition}

\begin{implication}
	$ H $ "--- гильбертово, $ \quad T \in \mathscr B(H), \quad T $ "--- самосопряжённый

	Если $ Y \sub H $ инвариантно для $ T $, то $ Y^\perp $ тоже инвариантно.
\end{implication}

\begin{eg}
	$ M = \ol M \sub H, \quad M $ "--- подпространство
	$$ P \text{ "--- ортопроектор на } M \implies P = P^* $$
\end{eg}

\section{Определение спектра и резольвенты оператора.
Теорема о свойствах резольвенты}

\begin{definition}
	$ X $ "--- банахово, $ \quad T \in \mathscr B(X), \quad I $ "--- тождественный оператор,
	$ \quad \lambda \in \Co $

	Будем говорить, что $ \lambda $ "--- \emph{регулярная точка}, если
	$ V(\lambda) \coloneq \lambda I - T $ "--- биекция.

	По теореме Банаха об обратном операторе,
	$$ V^{-1}(\lambda) \in \mathscr B(X) $$
	$$ R(\lambda, T) = R(\lambda) = V^{-1}(\lambda) $$
	$ R $ называется \emph{резольвентой}.
\end{definition}

\begin{definition}
	Множество всех регулярных точек называется \emph{резольвентным множеством}:
	$$ \rho(T) = \Set{\lambda \in \Co | \lambda \text{ "--- регулярная}} $$
\end{definition}

\begin{definition}
	Множество всех остальных точек назовём \emph{спектром} $ T $:
	$$ \sigma(T) = \Co \setminus \rho(T) $$
\end{definition}

\begin{definition}[части спектра]
	\hfill
	\begin{enumerate}
		\item $ \sigma_p(T) $ "--- \emph{точечный спектр}
			$$ \sigma_p(T) = \Set{\lambda \in \Co | \lambda T - I \text{ не инъекция}} $$
			Для линейного оператора это означает, что
			$ X_\lambda = \operatorname{Ker}(\lambda I - T) \ne \Set{0} $.
			$$ u \ne 0 \in X \implies Tu = \lambda u $$
			$ u $ "--- собственный вектор $ T $, соответствующий с. ч. $ \lambda $.
			$ X_\lambda $ "--- собственное подпространство.
		\item $ \sigma_c(T) $ "--- \emph{непрерывный спектр}
			$$ \sigma_c(T) = \Set{\lambda \in \Co | \operatorname{Ker}(\lambda I - T) = \Set{0},
			\quad \ol{V(\lambda)(X)} = X} $$
			(\ie образ $ V(\lambda) $ всюду плотен в $ X $).
		\item $ \sigma_r(T) $ "--- \emph{остаточный спектр}
			$$ \sigma_r(T) = \Set{\lambda \in \Co | \operatorname{Ker}(\lambda I - T) = \Set{0},
			\quad \ol{V(\lambda)(X)} \subsetneq X} = \sigma \setminus (\sigma_p \cup \sigma_c) $$
	\end{enumerate}
\end{definition}

\begin{remark}
	В конечномерном случае $ \sigma = \sigma_p $.
\end{remark}

\begin{properties}
	$ X $ "--- банахово, $ \quad T \in \mathscr B(H) $
	\begin{enumerate}
		\item $ \lambda, \mu \in \rho(T) $
			$$ \implies R(\lambda) R(\mu) = R(\mu) R(\lambda) $$
		\item \emph{Тождество Гильберта}: $ \lambda, \mu \in \rho(T) $
			$$ \implies R(\lambda) - R(\mu) = (\mu - \lambda) R(\lambda) R(\mu) $$
		\item $ \rho(T) $ открыто в $ \Co $

			Кроме того, если $ \mu \in \rho(T) $, то
			$$ |\lambda - \mu| < \frac1{\|R(\mu)\|} \implies \lambda \in \rho(T) $$
		\item $ \lambda \in \Co, \quad |\lambda| > \|T\| $
			$$ \implies \lambda \in \rho(T) $$
		\item $ R(\lambda) $ "--- непрерывная функция, \ie если $ \mu \in \rho(T) $, то
			$$ \lim\limits_{\lambda \to \mu}R(\lambda) = R(\mu), \quad
			\lim\limits_{\lambda \to \infty} R(\lambda) = 0 $$
		\item $ F \in \bigl( \mathscr B(X) \bigr)^*, \quad \lambda \in \rho(T), \quad
			g(\lambda) \coloneq F \bigl( R(\lambda) \bigr) $
			$$ \implies g(\lambda) \text{ аналитична в } \rho(T), \quad
			\lim\limits_{\lambda \to \infty} g(\lambda) = 0 $$
	\end{enumerate}
\end{properties}

\begin{eproof}
\item $ \lambda, \mu \in \rho(T) $
	$$ V(\lambda) V(\mu) = (\lambda I - T)(\mu I - T) = V(\mu) V(\lambda) $$
	$$ \exists \bigl( V(\lambda) \bigr)^{-1}, ~ \bigl( V(\mu) \bigr)^{-1}, \quad
	R(\lambda) = \bigl( V(\lambda) \bigr)^{-1}, \quad R(\mu) = \bigl( V(\mu) \bigr)^{-1} $$
	$$ \implies \bigl( V(\mu) \bigr)^{-1} \bigl( V(\lambda) \bigr)^{-1} =
	\bigl( V(\lambda) \bigr)^{-1} \bigl( V(\mu) \bigr)^{-1} $$
\item $ \lambda, \mu \in \rho(T) $
	$$ V(\lambda) - V(\mu) = (\lambda I - T) - (\mu I - T) = (\lambda - \mu) I $$

	Если $ A, B \in \mathscr B(X), \quad \exists A^{-1}, B^{-1} $, то
	$$ A^{-1} - B^{-1} = A^{-1}(B - A)B^{-1} $$

	Возьмём $ A = V(\lambda), ~ B = V(\mu) $.
	$$ R(\lambda) - R(\mu) = R(\lambda) \bigl( (\mu - \lambda) I \bigr) R(\mu) =
	(\mu - \lambda)R(\lambda)R(\mu) $$
\item Известно, что $ \operatorname{In}(X) $ (множество обратимых операторов) открыто:
	$$ \exists A^{-1} \quad \|A - B\| < \frac1{\|A^{-1}\|} \implies \exists B^{-1} $$
	$$ \mu \in \rho(T), \quad A = V(\mu) \implies \exists R(\mu) = \bigl( V(\mu) \bigr)^{-1} $$
	$$ V(\lambda) - V(\mu) = (\lambda - \mu)I \implies
	\|V(\lambda) - V(\mu)\| = |\lambda - \mu| $$
	$$ |\lambda - \mu| < \frac1{\|R(\mu)\|} \implies \|V(\lambda) - V(\mu) \| < \frac1{\|R(\mu)\|}
	\implies \exists \bigl( V(\lambda) \bigr)^{-1} \implies \lambda \in \rho(T) $$
\item $ \lambda \in \Co, \quad |\lambda| > \|T\| $

	Рассмотрим оператор
	\begin{multline*}
		\Bigl\| \frac1\lambda T \Bigr\| < 1 \underimp{\text{т. об обр. опер., близкого к тожд.}}
		\exists \Bigl( I - \frac1\lambda T \Bigr)^{-1} \implies \\
		\implies V(\lambda) = \lambda I - T = \lambda \Bigl( I - \frac1\lambda T \Bigr) \implies
		\exists R(\lambda) = \bigl( V(\lambda) \bigr)^{-1} =
		\frac1\lambda \Bigl( I - \frac1\lambda T \Bigr)^{-1}
	\end{multline*}
\item $ \mu \in \rho(T) $
	$$ \lim\limits_{\lambda \to \mu} \bigl( V(\mu) - V(\lambda) \bigr) =
	\lim\limits_{\lambda \to \mu} (\mu - \lambda) I = 0 $$
	По теореме об открытости $ \mathrm{In}(A) $,
	$$ \phi : A \to A^{-1}, \quad A \in \mathrm{In}(X) \implies \phi \text{ непрерывно} $$
	$$
	\begin{rcases}
		\phi \bigl( V(\lambda) \bigr) = V(\lambda) \\
		\lim\limits_{\lambda \to \mu} V(\lambda) = V(\mu)
	\end{rcases} \implies \lim\limits_{\lambda \to \mu} R(\lambda) = R(\mu) $$

	Пусть $ |\lambda| > \|T\| $.
	$$ \lim\limits_{\lambda \to \infty} \Bigl( I - \frac1\lambda T \Bigr) = I $$
	$$ R(\lambda) = \frac1\lambda \Bigl( I - \frac1\lambda T \Bigr)^{-1} $$
	По непрерывности,
	$$ \lim \Bigl( I - \frac1\lambda T \Bigr)^{-1} = I \implies
	\lim R(\lambda) = 0 $$
\item $ \mu \in \rho(T), \quad \lambda $ из некоторой окрестности $ \mu $
	$$ \frac{R(\lambda) - R(\mu)}{\lambda - \mu} \undereq{\text{т-во Гильберта}}
	\frac{(\mu - \lambda)R(\lambda)R(\mu)}{\lambda - \mu)} = -R(\lambda)R(\mu) \implies
	\exists \lim\limits_{\lambda \to \mu} \frac{R(\lambda) - R(\mu)}{\lambda - \mu} =
	- \bigl( R(\mu) \bigr)^2 $$

	Возьмём $ F \in \bigl( \mathscr B(X) \bigr)^*, \quad F : \mathscr B(X) \to \Co $.
	Рассмотрим функцию $ g(\lambda) = F \bigl( R(\lambda) \bigr) $ при $ \lambda \in \rho(T) $.
	$$ \lim\limits_{\lambda \to \mu} \frac{g(\lambda) - g(\mu)}{\lambda - \mu} =
	\lim\limits_{\lambda \to \mu} \frac{ F \bigl( R(\lambda) \bigr) - F \bigl( R(\mu) \bigr)}
	{\lambda - \mu} \undereq{\text{лин. } F}
	\lim F \Bigl( \frac{R(\lambda) - R(\mu)}{\lambda - \mu} \Bigr) \undereq{\text{непр. } F}
	-F \Bigl( \bigl( R(\mu) \bigr)^2 \Bigr) $$
	То есть, $ \exists g'(\mu) \quad \forall \mu \in \rho(T) $.

	$$ \lim\limits_{\mu \to \infty} R(\mu) = 0 \underimp{\text{непр. } F}
	\lim\limits_{\mu \to \infty} F \Bigl( \bigl( R(\mu) \bigr)^2 \Bigr) = 0 $$
\end{eproof}

\section{Теорема о компактности и непустоте спектра.
Формула для спектрального радиуса.
Следствие о спектре сопряжённого оператора}

\begin{implication}
	$ X $ "--- банахово, $ \quad T \in \mathscr B(X) $

	$$ \sigma(T) \text{ "--- компакт}, \quad \sigma(T) \ne \emptyset $$
\end{implication}

\begin{eproof}
\item Компактность
	$$ \rho(T) \text{ открыто } \implies \sigma(T) \text{ замкнуто } $$
	$$ |\lambda| > \|T\| \implies \lambda \in \rho(T) $$
	Значит,
	$$ \lambda \in \sigma(T) \implies |\lambda| \le \|T\| $$
	То есть,
	$$ \sigma(T) \sub \Set{\lambda \in \Co | {} |\lambda| \le \|T\|} \implies
	\sigma(T) \text{ ограничено } \implies \sigma(T) \text{ "--- компакт} $$
\item Непустота

	\textbf{Пусть} $ \sigma(T) $ пусто.
	Тогда $ \rho(T) = \Co $, то есть
	$$ \forall F \in \bigl( \mathscr B(X) \bigr)^* \quad g(\lambda) = F \bigl( R(\lambda) \bigr)
	\text{ "--- аналитическая в } \Co \text{ (целая)} $$

	$$ V(0) = -T \implies \exists T^{-1} \in \mathscr B(X) \underimp{\text{сл. из т. Хана"--~Банаха}}
	\exists F \in \bigl( \mathscr B(X) \bigr)^* : \quad g(0) \ne 0 $$
	Выберем $ g(\lambda) = F \bigl( R(\lambda) \bigr) $.

	Таким образом, $ g(z) $ "--- целая, $ g $ ограничена.
	Значит, по теореме Лиувилля, $ g \equiv \const $ "--- \contra с $ \lim\limits_{g \to 0} = 0 $.
\end{eproof}

\begin{eg}
	$ Ix = x $
	$$ \sigma(I) = \Set{1} = \sigma_p(I) $$

	$$ (\lambda I - I) = (\lambda - 1)I \implies R(\lambda) = \frac1{\lambda - 1}I \quad
	\forall \lambda \ne 1 $$
\end{eg}

\begin{theorem}[спектр и резольвента сопряжённого оператора]
	\hfill
	\begin{enumerate}
		\item $ X $ "--- банахово, $ \quad T \in \mathscr B(X) $

			$$ \implies \sigma(T^*) = \sigma(T) $$

			Если $ \lambda \in \rho(T) $, то
			$$ \bigl( R(\lambda, T) \bigr)^* = R(\lambda, T^*) $$
		\item $ H $ "--- гильбертово, $ \quad T \in \mathscr B(H), \quad
			T $ "--- эрмитово-сопряжённый

			$$ \implies \sigma(T^*) = \Set{\lambda | \ol \lambda \in \sigma(T)} $$

			Если $ \lambda \in \rho(T) $, то
			$$ R(\lambda, T^*) = \bigl( R(\ol \lambda, T) \bigr)^* $$
	\end{enumerate}
\end{theorem}

\begin{eproof}
\item $ X $ "--- банахово, $ \quad \lambda \in \rho(T) $
	$$ V(\lambda) = \lambda I - T \implies \bigl( V(\lambda) \bigr)^* = \lambda I - T^* $$
	$$ \Bigl( \bigl( V(\lambda) \bigr)^{-1} \Bigr)^* =
	\Bigl( \bigl( V(\lambda) \bigr)^* \Bigr)^{-1} $$
\item $ X $ "--- гильбертово
	$$ \bigl( V(\lambda) \bigr)^* = \ol \lambda I - T^* $$
\end{eproof}

\section{Компактные операторы.
Компактность оператора конечного ранга.
Размерность замкнутого подпространства образа компактного оператора.
Следствия}

\begin{definition}
	$ X, Y $ "--- банаховы, $ \quad T \in \mathscr Lin(X, Y) $

	$ T $ называется \emph{компактным}, если $ T \bigl( \mathtt B_1(0) \bigr) $ относительно
	компактен.
\end{definition}

\begin{notation}
	$ \mathrm{Com}(X, Y) $ "--- множество компактных операторов.
\end{notation}

\begin{props}
	\item $ \mathrm{Com}(X, Y) \sub \mathscr B(X, Y) $
	\item $ T \in \mathrm Com(X, Y), \quad A \sub X $ ограничено
		$$ T(A) \text{ относительно компактно} $$
	\item $ T \in \mathrm{Com}(X, Y) $ \textbf{тогда и только тогда}, когда
		$$ \Set{x_n \in X}_{n = 1}^\infty \text{ "--- ограниченная } \implies \exists
		\Set{n_k}_{k = 1}^\infty : \quad \exists \lim\limits_{k \to \infty}T_{n_k} \in Y $$
\end{props}

\begin{eproof}
\item $ T \in \mathrm{Com}(X, Y) $

	Обозначим $ B = \mathtt B_1^X(0) $.
	$$ \implies \ol{T(B)} \text{ "--- компакт } \implies T(B) \text{ ограничено } \implies T \in
	\mathscr B(X, Y) $$
\end{eproof}

\begin{definition}
	$ X, Y $ "--- банаховы, $ T \in \mathscr B(X, Y) $

	Если $ \dim T(X) < +\infty $, то $ T $ называется \emph{оператором конечного ранга}.
\end{definition}

\begin{eg}
	$ X, Y $ "--- банаховы, $ \quad f_1, \dots, f_n \in X^*, \quad y_1, \dots, y_n \in Y $

	Для $ x \in X $ определим
	$$ Tx = \sum_{j = 1}^n f_j(x) y_j $$

	$$ T(x) \sub \mathscr L \Set{y_j}_{j = 1}^n \implies \dim T(X) \le n \implies
	T \text{ "--- конечного ранга} $$

	Все операторы конечного ранга имеют такой вид.
\end{eg}

\begin{statement}
	$ X, Y $ "--- банаховы, $ \quad T $ "--- конечного ранга

	$$ T \in \mathrm{Com}(X, Y) $$
\end{statement}

\begin{proof}
	$ B \coloneq \mathtt B_1^X(0) $
	\begin{multline*}
		T(B) \sub T(X) \implies T(B) \text{ "--- ограниченное множество в конечномерном пр-ве }
		\implies \\
		\implies T(B) \text{ относительно компактно}
	\end{multline*}
\end{proof}

\begin{theorem}
	$ X, Y $ "--- банаховы, $ \quad T \in \mathrm{Com}(X, Y), \quad L \sub T(X) $ "--- замкнутое
	подпространство

	$$ \dim L < +\infty $$
\end{theorem}

\begin{iproof}
\item $ T(X) $ "--- замкнутое подпространство $ Y $
	\begin{multline*}
		T \in \mathscr B(X, T(X)), \quad T(X) \text{ "--- банахово } \underimp{\text{т. Банаха об
		откр. отобр.}} T \text{ открыто } \implies \\
		\exists r > 0 : \quad \mathtt B_r^{T(x)}(0) \sub \underbrace{T(B)}_{\text{отн. комп.}}
		\implies \mathtt B_r^{T(x)}(0) \text{ относительно компактно } \underimp{\text{т. Рисса}}
		\dim T(X) < +\infty
	\end{multline*}
\item $ L \sub T(X) $ "--- замкнутое подпространство

	Обозначим $ X_1 = T^{-1}(L) $ (прообраз)
	$$ T \in \mathscr B(X, Y) \underimp{L \text{ замкнуто}} T^{-1}(L) \text{ замкнуто в } X \implies
	X_1 \text{ замкнуто } \implies X_1 \text{ "--- банахово} $$
	По первой части доказательства
	$$ T(X_1) = L \implies \dim L < +\infty $$
\end{iproof}

\begin{implication}
	$ X $ "--- банахово, $ \quad T \in \mathrm{Com}(X) $

	\begin{enumerate}
		\item Если $ T(X) = X $, то $ \dim X < +\infty $.
		\item Если $ \dim X = +\infty $, то $ 0 \in \sigma(T) $.
	\end{enumerate}
\end{implication}

\begin{eproof}
\item Очевидно.
\item \textbf{Пусть} $ 0 \in \rho(T) $.
	$$ 0 \in \rho(T) \iff V(0) = 0 \cdot I - T = -T \underimp{\exists T^{-1}} T(X) = X
	\text{ "--- \contra с первым пунктом} $$
\end{eproof}

\section{Теорема об арифметических операциях с компактными операторами.
Предел компактных операторов.
Компактность эрмитово"=сопряжённого оператора}

\begin{theorem}
	\hfill
	\begin{enumerate}
		\item $ X, Y $ "--- банаховы пространства
			$$ \mathrm{Com}(X, Y) \text{ "--- замкнутое подпространство } \mathscr B(X, Y) $$
		\item $ X, Y, Z $ "--- банаховы, $ \quad X \xrightarrow T Y \xrightarrow S Z $
			\begin{enumerate}
				\item $ T \in \mathrm{Com}(X, Y), \quad S \in \mathscr B(Y, Z) $
					$$ ST \in \mathrm{Com}(X, Z) $$
				\item $ T \in \mathscr B(X, Y), \quad S \in \mathrm{Com}(Y, Z) $
					$$ ST \in \mathrm{Com}(X, Z) $$
			\end{enumerate}
	\end{enumerate}
\end{theorem}

\begin{proof}
	Обозначим $ B = \mathtt B_1^X(0) $.
	\begin{enumerate}
		\item
			\begin{itemize}
				\item $ \alpha \in \Co, \quad T \in \mathrm{Com}(X, Y) \implies
					\alpha T \in \mathrm{Com}(X, Y) $ "--- очевидно
				\item Возьмём $ T, S \in \mathrm{Com}(X, Y) $.

					$ T(B) $ относительно компактно $ \implies T(B) $ вполне ограничено.
					Возьмём $ \eps > 0 $ и рассмотрим $ \eps $-сети:
					$$ \exists E \sub Y \text{ "--- конечная $ \eps $-сеть для } T(B) $$
					$$ \exists F \sub Y \text{ "--- конечная $ \eps $-сеть для } S(B) $$
					\begin{multline*}
						E + F = \Set{e + f | e \in E, ~ f \in F} \text{ "--- конечное, $ 2\eps $-сеть для }
						T(B) + S(B) \implies \\
						\implies T(B) + S(B) \text{ относительно компактно}
					\end{multline*}
					$$ (T + S)(B) \sub T(B) + S(B) \implies (T + S)(B) \text{ относительно компактно}
					\implies T + S \in \mathrm{Com}(X, Y) $$
				\item $ \Set{T_n \in \mathrm{Com}(X, Y)}_{n = 1}^\infty, \quad
					\lim\limits_{n \to \infty}\|T_n - T\| = 0 $

					Возьмём $ \eps > 0 $.
					$$ \exists n \in \N : \quad \|T - T_n\| < \eps $$
					При этом, $ T_n(B) $ относительно компактно.
					$$ \implies \exists E \text{ "--- конечная $ \eps $-сеть для } T_n(B) $$

					Проверим, что $ E $ "--- $ \eps $-сеть для $ T(B) $.
					Возьмём $ x \in B $.
					$$ \exists e \in E : \quad \|T_nx - e\| < \eps $$
					$$ \|Tx - e\| \trile \|Tx - T_nx\| + \|T_nx - e\| \le
					\underbrace{\|T - T_n\|}_{< \eps} \cdot \underbrace{\|x\|}_{< 1} + \eps <
					2\eps $$
					Значит, $ T(B) $ вполне ограничено и относительно компактно.
			\end{itemize}
		\item $ X, Y, Z $
			\begin{enumerate}
				\item $ T \in \mathrm{Com}(X, Y) $
					$$ T(B) \text{ относительно компактно }, \quad S \in \mathscr B(Y, Z) $$
					$$ \underimp{\text{непр. } S} S \bigl( T(B) \bigr)
					\text{ относительно компактно} \implies ST \in \mathrm{Com}(X, Z) $$
				\item $ T \in \mathscr B(X, Y) $
					$$ \implies T(B) \text{ ограничено}, \quad S \in \mathrm{Com}(X, Y) $$
					$$ \implies S \bigl( T(B) \bigr) \text{ относительно компактно} $$
			\end{enumerate}
	\end{enumerate}
\end{proof}

\begin{implication}
	$ X $ "--- банахово

	Тогда $ \mathrm{Com}(X) $ "--- \emph{двусторонний замкнутый идеал алгебры } $ \mathscr B(X) $.
\end{implication}

\textcolor{gray}{Про предел компактных операторов тоже нигде найти не могу...}

\begin{theorem}
	$ H $ "--- гильбертово
	$$ T \in \mathrm{Com}(H) \iff T^* \in \mathrm{Com}(H) $$
	($ T^* $ "--- эрмитово сопряжённый)
\end{theorem}

\begin{iproof}
\item $ T \in \mathrm{Com}(H) $

	Возьмём $ x \in H $.
	\begin{equ}{comp_dual:1}
		\|T^*x\|^2 = (T^*x, T^*x) = (TT^*x, x) \underset{\text{К"--~Б}}\le \|TT^*x\| \cdot \|x\|
	\end{equ}

	Возьмём $ \Set{x_n \in H}_{n = 1}^\infty $ такую, что
	$$ \exists M > 0 : \quad \|x_n\| \le M \quad \forall n \in \N $$

	Проверим, что $ \exists \Set{n_k} : \quad \exists \lim T^*x_{n_k} $.
	\begin{multline*}
		\begin{rcases}
			T \in \mathrm{Com}(H) \\
			T^* \in \mathscr B(H)
		\end{rcases} \implies TT^* \in \mathrm{Com}(H) \implies \exists \Set{n_k} : \quad
		\exists \lim TT^*x_{n_k} \implies \\
		\implies \Set{TT^*x_{n_k}} \text{ фундаментальна}
	\end{multline*}
	\begin{multline*}
		\|T^*x_{n_k} - T^*x_{n_j}\|^2 = \bigl\| T^*(x_{n_k} - x_{n_j}) \bigr\|^2
		\underset{\eref{comp_dual:1}}\le
		\underbrace{\|(TT^*)(x_{n_k}) - TT^*x_{n_j}\|}_{\underarr{k, j \to \infty} 0} \cdot
		\underbrace{\|x_{n_k} - x_{n_j}\|}_{\le 2M} \implies \\
		\implies \Set{T^*x_{n_k}} \text{ фундаментальна } \implies \exists \lim T^*x_{n_k}
		\implies T^* \in \mathrm{Com}(H)
	\end{multline*}
\item $ T^* \in \mathrm{Com}(H) \implies T = T^{**} \in \mathrm{Com}(H) $
\end{iproof}

\begin{remark}
	$ X, Y $ "--- банаховы

	$$ T \in \mathrm{Com}(X, Y) \iff T^* \in \mathrm{Com}(Y^*, X^*) $$
\end{remark}

\begin{noproof}
\end{noproof}

\section{Конечность числа линейно-независимых собственных векторов компактного оператора,
соответствующих собственным числам, модули которых равномерно отделены от нуля.
Следствия}

\begin{remark}[воспоминания из алгебры]
	$ X $ "--- линейное пространство, $ \quad T \in \mathscr Lin(X), \quad
	\Set{\lambda_j}_{j = 1}^n $ "--- с. ч., \\
	$ Tx_j = \lambda_j x_j, \quad \lambda_j \ne \lambda_k, \quad x_j $ "--- с. в. ($ x_j \ne 0 $)

	$$ \implies \Set{x_j} \text{ ЛНЗ} $$
\end{remark}

\begin{theorem}
	$ X $ "--- банахово, $ \quad T \in \mathrm{Com}(X), \quad \lambda \in \sigma_p(T) $ "--- с. ч.,
	$ \quad X_\lambda = \operatorname{Ker}(\lambda I - T) $ "--- собств. подпр-во, \\
	$ \delta > 0 $

	$$ \sum_{
		\begin{subarray}{c}
			\lambda \in \sigma_p(T) \\
			|\lambda| \ge \delta
		\end{subarray}
	} \dim(X_\lambda) < +\infty $$

	То есть, число линейно-независимых собственных векторов $ T $, соответствующих собственным
	числам $ \lambda $, таких, что $ |\lambda| \ge \delta $, конечно.
\end{theorem}

\begin{proof}
	\textbf{Пусть} $ \set{x_n}_{n = 1}^\infty $ "--- ЛНЗ с. в.:
	$$ Tx_n = \lambda_n x_n, \quad |\lambda_n| \ge \delta $$

	Рассмотрим последовательность подпространств:
	$$ L_n = \mathscr L\Set{x_j}_{j = 1}^n, \quad L_n \subsetneq L_{n + 1} $$
	$$ \underimp{\text{лемма Рисса}} \exists \Set{y_n}_{n = 1}^\infty : \quad \|y_n\| = 1, \quad
	\rho(y_{n + 1}, L_n) = \inf\limits_{x \in L_n} \|y_{n + 1} - x\| \ge \frac12 $$
	(\as $ \dim L_n = n $, то $ \exists y_{n + 1} : \rho(y_{n + 1}, L_n) = 1 $)

	Проверим, что $ \|Ty_n - Ty_m\| \ge \frac\delta2 $.
	Тогда не будет существовать фундаментальной подпоследовательности $ \Set{Ty_n} $, а значит,
	и последовательности $ \Set{n_k} $ такой, что $ \exists \lim T y_{n_k} $ "--- \contra с $ T \in
	\mathrm{Com}(X) $.

	Пусть $ y_n \in L_n, \quad y_n \notin L_{n - 1} $.
	Тогда $ y_n = \alpha n x_n + u_n, \quad \alpha_n \ne 0, ~ u_n \in L_{n - 1} $.
	$$ Tx_n = \lambda_n x_n \implies Ty_n = \alpha_n \lambda_n x_n + Tu_n =
	\lambda_n(\alpha_n x_n + u_n) -
	\underbrace{\lambda_n u_n + Tu_n}_{ \eqqcolon v_n \in L_{n - 1}} = \lambda_ny_n + v_n $$

	Пусть $ n > m $.
	$$ Ty_m \in L_n \sub L_{n - 1} $$
	$$ \|Ty_n - Ty_m\| = \|\lambda_ny_n + v_n - Ty_m\| =
	|\lambda_n| \Bigl\| y_n - \underbrace{\frac1{\lambda_n}(-v_n + Ty_m)}_{\in L_{n - 1}} \Bigr\|
	\ge \frac \delta 2 $$
	Значит, последовательность $ y_n $ не содержит фундаментальных подпоследовательностей.
\end{proof}

\begin{implication}
	$ T \in \mathrm{Com} $
	\begin{enumerate}
		\item $ \delta > 0 $

			$$ \bigl|\Set{\lambda \in \sigma_p(T) | {} |\lambda| \ge \delta}\bigr| < +\infty $$
		\item $ \lambda \in \sigma_p(T), \quad \lambda \ne 0, \quad
			X_\lambda = \operatorname{Ker}(\lambda I - T) $

			$$ \dim X_\lambda <+\infty $$
		\item $ N $ "--- количество собственных чисел.

			Тогда
			\begin{enumerate}
				\item $ 0 \le N \le +\infty $ (\ie $ \sigma_p(T) $ не более, чем счётно);
				\item если $ N = +\infty $, то $ \lim\limits_{n \to \infty} \lambda_n = 0 $.
			\end{enumerate}

			Их можно занумеровать в порядке убывания модулей.
	\end{enumerate}
\end{implication}

\begin{eproof}
\item Очевидно.
\item Очевидно.
\item $ E_n \coloneq \Set{\lambda \in \sigma_n(T) | {} |\lambda| \ge \frac1n}, \quad |E_n| <+\infty $
	$$ \sigma_p(T) \setminus \Set{0} = \bigcup_{n = 1}^\infty E_n \implies
	\sigma_p(T) \text{ не более, чем счётно} $$
	$$ \forall \delta > 0 \quad \Set{\lambda \in \sigma_p(T) | {} |\lambda| > \delta}
	\text{ конечно } \implies \lim\limits_{n \to \infty}\lambda_n = 0 $$
\end{eproof}
