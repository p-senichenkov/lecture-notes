\part{Линейные пространства}

\section{Линейные операторы.
Примеры.
Простейшие свойства}

\begin{definition}
	$ X $ "--- \emph{линейное пространство} над $ K $ ($ K = \R $ или $ \Co $), если
	\begin{enumerate}
		\item $ x, y \in X \implies \alpha x + \beta y \in X \quad \forall \alpha, \beta \in K $;
		\item $ \O \in X $.
	\end{enumerate}
\end{definition}

\begin{definition}
	$ X, Y $ линейны над $ K, \quad A : X \to Y $.

	Будем $ A $ называть \emph{линейным оператором}, если
	\begin{enumerate}
		\item $ A(x + z) = Ax + Az $;
		\item $ A(\alpha x) = \alpha Ax $.
	\end{enumerate}
\end{definition}

\begin{remark}
	$ \mathscr Lin $ "--- линейное пространство над $ K $.
\end{remark}

\begin{exmpls}
\item $ K(s, t) \in \mathcal C \bigl( [a, b] \times [a, b]\bigr) $
	Определим \emph{интегральный оператор}:
	$$ ( \mathcal K f)(s) = \int_a^b K(s, t) f(t) \di t $$
	$ K(s, t) $ называется \emph{ядром интегрального оператора}.
	$$ \mathcal K \in \mathscr Lin \Bigl( \mathcal C[a, b] \Bigr) $$
\item $ X = \nder[1]{\mathcal C}[a, b] = \set{f | f' \in \mathcal C[a, b]}, \quad Y = \mathcal C[a, b] $
	$$ \mathcal D(f) \define f' $$
	$$ \mathcal D \in \mathscr Lin (X, Y) $$
\item $ l^1 \hookrightarrow l^2 $
	$$ \sum_{j = 1}^\infty |x_j| < +\infty \implies \sum_{j = 1}^\infty |x_j|^2 < +\infty $$
	$ Ax = x $ "--- \emph{оператор вложения}
\end{exmpls}

\begin{definition}
	$ X $ "--- линейное пространство над полем $ K, \quad B \sub X $.

	$ B $ называется \emph{выпуклым}, если
	$$ \forall x, y \in B \quad \forall t \in [0, 1] \quad tx + (1 - t)y \in B $$
\end{definition}

\begin{theorem}
	$ X, Y $ "--- линейные пространства, $ A \in \mathscr Lin(X, Y) $.

	\begin{enumerate}
		\item $ L \sub X $ "--- подпространство $ \implies A(L) $ "--- подпространство $ Y $;
		\item $ M \sub Y $ "--- подпространство $ \implies \underset{\text{(прообраз)}}{A^{-1}(Y)} $ "--- подпространство $ X $;
		\item $ B \sub X $ "--- выпуклое $ \implies A(B) $ "--- выпуклое в $ Y $;
		\item $ C \sub Y $ "--- выпуклое $ \implies \underset{\text{(прообраз)}}{A^{-1}(C)} $ "--- выпуклое в $ X $;
		\item $ A $ "--- биекция $ \implies \exists A^{-1} \in \mathscr Lin(Y, X) $.
	\end{enumerate}
\end{theorem}

\begin{noproof}
	Было в алгебре.
\end{noproof}

\begin{definition}
	$ A \in \mathscr Lin(X, Y), \quad X, Y $ линейны над $ K $.

	Определим \emph{ядро оператора}:
	$$ \operatorname{Ker} A = \set{x \in X | Ax = 0} $$
	и \emph{образ оператора}:
	$$ \operatorname{Im} A = A(X) = \set{y \in Y | \exists x \in X : \quad y = Ax} $$
\end{definition}

\begin{implication}
	$ A \in \mathscr Lin(X, Y) \implies \operatorname{Ker} A $ "--- подпространство $ X, \quad \operatorname{Im} A $ "--- подпространство $ Y $.
\end{implication}

\begin{definition}
	$ X, Y, Z $ линейны, $ \quad X \overarr A Y \overarr B Z $.

	$ C = B \cdot A $ называется \emph{произведением операторов}, если $ C(x) = B \bigl( A(x) \bigr) $.
\end{definition}

\begin{statement}
	Если $ A, B $ линейны, то $ A \cdot B $ линеен.
\end{statement}

\hyphenation{оп-ре-де-ле-ни-е}

\section{Непрерывность и ограниченность линейного оператора.
Определение нормы}

\begin{definition}
	$ (X, \|\cdot\|), ~ (Y, \|\cdot\|), \quad A \in \mathscr Lin(X, Y) $.

	$ A $ \emph{ограничен}, если
	$$ \forall \text{ ограниченного } B \sub X \quad A(B) \text{ ограничено} $$
\end{definition}

\begin{theorem}[экввивалентность непрерывности и ограниченности линейного оператора]
	\hfill \\
	$ (X, \|\cdot\|), ~ (Y, \|\cdot\|), \quad A \in \mathscr Lin(X, Y) $.

	Следующие условия равносильны:
	\begin{enumerate}
		\item $ A $ непрерывен в $ \On $;
		\item $ A $ непрерывен на $ X $;
		\item $ \exists c > 0 : \quad \|Ax\|_Y \le c\|x\|_X \quad \forall x \in X $;
		\item $ A $ ограничен;
		\item $ \exists r > 0 : \quad A \bigl( \mathtt B_r(\On) \bigr) $ ограничено.
	\end{enumerate}
\end{theorem}

\begin{iproof}
\item 1 $ \implies $ 2

	В силу линейности $ A(\On) = \On $.

	$ A $ непрерывен в $ \On \implies \exists \delta > 0 : \quad \forall \|x\| < \delta \quad \|Ax\| < \eps $.

	Зафиксируем $ x_0 \in X $, обозначим $ y_0 = Ax_0 $.
	$$ \|x - x_0\| < \delta \underimp{\text{непр. в } \On} \|A(x - x_0) \| < \eps \iff
	\|Ax - Ax_0\| < \eps \iff \text{ непр. в } x_0 $$
\item 2 $ \implies $ 1 "--- очевидно.
\item 1 $ \implies $ 3

	$$ \forall \eps > 0 \exists \delta > 0 : \quad \forall \|x\| \le \delta \quad \|Ax\| < \eps $$
	Возьмём $ z \in X, ~ z \ne \On $.
	$$ \Bigl\| \frac z{\|z\|} \Bigr\| = 1 $$
	\begin{multline*}
		x = \delta \cdot \frac z{\|z\|} \implies \|x\| = \delta \implies \|Ax\| < \eps \implies
		\Bigl\| A \Bigl(\delta \cdot \frac z{\|z\|} \Bigr) \Bigr\| < \eps \iff \\
		\iff \frac{\delta}{\|z\|} \cdot \| Az \| < \eps \iff
		\|Az\| < \frac\eps\delta \|z\|
	\end{multline*}
\item 3 $ \implies $ 4

	Возьмём $ B \sub X $ "--- ограниченное, \ie
	$$ \exists R > 0 : \quad \forall x \in B \quad \|x\| \le R \implies \|Ax\| \underset{3)}\le c\|x\| \le cR \implies
	A(B) \text{ ограничено} $$
\item 4 $ \implies $ 5

	$$ \mathtt B_r(\On) \text{ ограничено } \underimp{\text{очевидно}} A \bigl( \mathtt B_r(0) \bigr) $$
\item 5 $ \implies 1 $

	$$ \exists R > 0 : \quad A \bigl( \mathtt B_r(0) \bigr) \sub \mathtt B_R(\On) $$
	\ie $ \|x\| < r \implies \|Ax\| < R $.

	Пусть $ \|Ax\| < \eps $.
	Найдём $ \delta : \|x\| < \delta $.
	$$ \delta = \eps \cdot \frac r R $$
	Пусть $ \|x\| < \delta $
	$$ \implies \|x\| < \eps \cdot rR \implies \Bigl\| x \cdot \frac R\eps \Bigr\| < r \implies \Bigl\| A \Bigl( x \cdot \frac R\eps \Bigr) \Bigr\| < R \implies \|Ax\| \cdot \frac R\eps < R \iff
	\|Ax\| < \eps $$
\end{iproof}

\begin{definition}
	$ \mathscr B(X, Y) $ "--- множество всех ограниченных линейных операторов из $ X $ в $ Y $ \\
	($ \iff $ множество всех непрерывных).
\end{definition}

\begin{remark}
	$ \mathscr B(X, Y) $ "--- подпространство $ \mathscr Lin(X, Y) $.
\end{remark}

\begin{definition}
	$ A \sub \mathscr B(X, Y) $.

	Введём \emph{норму} на $ \mathscr B $:
	$$ \|A\| = \inf\set{c > 0 | \|Ax\| \le c\|x\| \quad \forall x \in X} $$
	(по свойству 3 она конечна).
\end{definition}

\section{Формула для вычисления нормы линейного оператора.
Примеры непрерывных и не непрерывных операторов}

\begin{statement}
	$ (X, \|\cdot\|), ~ (Y, \|\cdot\|) $

	\begin{enumerate}
		\item $ A \in \mathscr B(X, Y) \implies \|Ax\| \le \|A\| \cdot \|x\| \quad \forall x \in X $;
		\item $ \|\cdot\|_{\mathscr B(X, Y)} $ удовлетворяет аксиомам нормы.
	\end{enumerate}
\end{statement}

\begin{eproof}
\item Зафиксируем $ x \in X $, возьмём $ c > \|A\| $.
	$$ \implies \|Ax\| \le c\|X\| \implies \|Ax\| \le
	\inf\set{c > 0 | c > \|A\|}\|x\| \implies \|Ax\| \le \|A\| \cdot \|x\| $$
\item $ A \in \mathscr B(X, Y) $.

	\begin{enumerate}
		\item $ \lambda \in K, \quad x \in X $.
			$$ \bigl\| (\lambda A)x \bigr\| = \bigl\|\lambda \cdot (Ax) \bigr\| =
			|\lambda| \cdot \|Ax\| \le |\lambda| \cdot \|A\| \cdot \|x\| \quad \forall x $$
			$$ \implies \|\lambda A\| \le |\lambda| \cdot \|A\| $$
			Пусть $ \lambda \ne 0 $.
			$$ \Bigl\| \frac1\lambda (\lambda A) \Bigr\| \le \Bigl| \frac1\lambda \Bigr| \cdot \|\lambda A\| $$
			$$ \implies \|A\| \cdot |\lambda| \le \|\lambda A\| $$
		\item Неравенство треугольника

			Пусть $ x \in X $
			$$ \|(A + B)x\| =
			\|Ax + Bx\| \le \|Ax\| + \|Bx\| \le \|A\| \cdot \|x\| + \|B\| \cdot \|x\| =
			(\|A\| + \|B\|) \|x\| \quad \forall x $$
			$$ \implies \|A + B\| \le \|A\| + \|B\| $$
		\item $ \|A\| = 0 $

			$$ \implies \forall x \in X \quad \|Ax\| \le \|A\| \cdot \|x\| = 0 $$
	\end{enumerate}
\end{eproof}

\begin{theorem}
	$ A \in \mathscr B(X, Y) $

	$$ \|A\| = \sup\limits_{\|x\| \le 1}\|Ax\| =
	\sup\limits_{\|x\| < 1}\|Ax\| =
	\sup\limits_{\|x\| = 1}\|Ax\| =
	\sup\limits_{x \ne 0} \frac{\|Ax\|}{\|x\|} $$
\end{theorem}

\begin{proof}
	Обозначим $ a = \sup\limits_{\|x\| \le 1}, \quad b = \sup\limits_{\|x\| < 1}, \quad c = \sup\limits_{\|x\| = 1}, \quad
	d = \sup\limits_{x \ne 0} $. \\
	Очевидно, что $ a \le b, \quad d \ge c $.

	Докажем, что
	$$ \|A\| \ge a \ge b \ge \|A\|, \quad \|A\| \ge d \ge c \ge \|A\| $$

	\begin{enumerate}
		\item Пусть $ x : \|x\| \le 1 $.

			$$ \|Ax\| \le \|A\| \cdot \|x\| \le \|A\| \implies a \le \|A\| $$
		\item Пусть $ z \in X \ne 0, \quad \eps > 0 $.
			Рассмотрим
			$$ x = \frac z{\|z\| \cdot (1 + \eps)} \quad \implies \|x\| < 1 \implies \|Ax\| \le b $$
			$$ \implies \Bigl\| A \Bigl( \frac z{\|z\| (1 + \eps)} \Bigr) \Bigr\| \le b \iff
			\|Az\| \le (1 + \eps)b \cdot \|z\| \quad \forall z \implies
			\|A\| \le (1 + \eps)b \quad \forall \eps > 0 \implies
			\|A\| \le b $$
		\item Пусть $ x \in X \ne 0 $
			$$ \|Ax\| \le \|A\| \cdot \|x\| \implies \frac{\|Ax\|}{\|x\|} \le \|A\| \implies
			\sup\limits_{x \ne 0} \frac{\|Ax\|}{\|x\|} \le \|A\| \implies
			d \le \|A\| $$
		\item Пусть $ z \in X \ne 0 $
			$$ \Bigl\| \frac z{\|z\|} \Bigr\| = 1 \implies \Bigl\| A \Bigl( \frac z{\|z\|} \Bigr) \le c \implies
			\|Az\| \le c\|z\| \implies
			\|A\| \le c $$
	\end{enumerate}
\end{proof}

\begin{exmpls}
\item $ X = \mathcal C[a, b], \quad h(x) \in \mathcal C[a, b] $
	$ M_h(f) = h(x) \cdot f(x) $ "--- \emph{мультипликатор}.

	Проверим, что
	$$ M_h \in \mathscr B(\mathcal C[a, b]), \quad
	\|M_h\|_{\mathscr B(\mathcal C[a, b])} = \|h\|_\infty $$

	$$ x \in [a, b] \implies \bigl( M_h(f)(x) \bigr) = h(x)f(x) \implies
	\|h \cdot f\|_\infty \le \max\limits_{[a, b]} |h(x)| \cdot \max\limits_{[a, b]}|f(x)| =
	\|h\|_\infty \cdot \|f\|_\infty $$

	$$ \forall f \in \mathcal C[a, b] \quad \|M_h(f)\|_\infty \le \|h\|_\infty \|f\|_\infty \implies
	\forall M_h \in \mathscr B(\mathcal C[a, b]) \quad \|M_h\|_{\mathscr B(\mathcal C[a, b])} \le \|h\|_\infty $$

	$$ f(x) = \chi_{[a, b]}(x) \in \mathcal C[a, b], \quad \|\chi_{[a, b]}\|_\infty = 1 $$
	$$ \|M_h\| = \sup\limits_{\|f\| = 1} \|M_h(f)\|_\infty \ge \|M_h(\chi_{[a, b]})\|_\infty =
	\|h\|_\infty \implies
	\|M_h\|_{\mathscr B(\mathcal C[a, b])} \|h\|_\infty $$
\item $ Y = \mathcal C[0, 1], \quad X = \set{f | f' \in \mathcal [0, 1]} $

	$ X $ "--- подпространство $ Y $, \ie $ \|f\|_X = \max\limits_{[0, 1]}|f(x)| $.
	$$ f \in X, \quad \mathscr D(f) = f', \quad \mathscr D \in \mathscr Lin(X, Y) $$
	Проверим, что $ \mathscr D $ не является непрерывным.
	$$ x^n \in \mathcal C[0, 1], \quad \|x^n\|_\infty = 1 \implies
	\sup\limits_{\|f\| = 1} \|\mathscr D(f)\|_\infty \ge \sup\limits_{n \in \N} \|\mathscr D(x^n)\| $$
	$$ \mathscr D(x^n) = nx^{n - 1} \implies \|\mathscr D(x^n)\|_\infty = n $$
	$$ \implies \sup \|\mathscr D(f)\| \ge \sup \set{\mathscr D(x^n)} = +\infty $$
\item $ Y = \mathcal C[0, 1], \quad X = \nder[1]{\mathcal C}[0, 1] = \set{f | f' \in \mathcal C[0, 1]} $
	$$ \|f\|_X = \|f\|_{\nder[1]{\mathcal C}} = \max\set{\|f\|_\infty, \|f'\|_\infty} $$
	$$ \mathcal D(f) = f' $$
	Проверим, что $ \mathcal D \in \mathscr B(X, Y), ~ \|\mathcal D\|_{\mathscr B(X, Y)} = 1 $.

	\begin{itemize}
		\item Возьмём $ f \in X $.
			$$ \mathcal D(f) \in Y, \quad \mathcal D \in \mathscr Lin(X, Y) $$
			$$ \|\mathcal D(f)\| = \max\limits_{[0, 1]}|f(x)| = \|f\|_\infty \le \max\set{\|f\|_\infty, \|f'\|_\infty} $$
			$$ \implies \mathcal D \in \mathscr B(X, Y), \quad \|D\| \le 1 $$
		\item Пусть $ f(x) = x \in X = \nder[1]{\mathcal C[0, 1]} $
			$$ \|f\|_\infty = 1, \quad \|f'\| = 1 $$
			$$ \implies \|\mathcal D(f)\| = 1, \quad \|f\|_X = \max\set{\|f\|_\infty, \|f'\|_\infty} = 1 $$
			$$ \implies \|\mathcal D(f)\| = \sup\limits_{\|g\| = 1}\|\mathcal D(g)\| \ge
			\|\mathcal D(x)\|_\infty = 1 \implies
			\|\mathcal D \| = 1 $$
	\end{itemize}
\end{exmpls}

\section{Теоремы вложения для \texorpdfstring{$ l^p $}{...} и для
\texorpdfstring{$ \mathrm L^p $}{пространств Лебега}.
Полнота пространства операторов, действующих из нормированного пространства в банахово}

\begin{theorem}
	$ 1 \le p_1 < p_2 \le \infty, \quad x = \set{x_n}_{n = 1}^\infty \in l^{p_1}, \quad Ax = x $

	$$ \implies A \in \mathscr B(l^{p_1}, l^{p_2}), \quad \|A\|_{\mathscr B(l^{p_1}, l^{p_2})} = 1 $$
	$ A $ называется \emph{оператором вложения}.
\end{theorem}

\begin{proof}
	Пусть $ x = \set{x_n}_{n = 1}^\infty \in l^{p_1}, \quad \|x\|_{p_1} = 1 $, \ie
	$$ \Bigl( \sum_{n = 1}^\infty |x_n|^{p_1} \Bigr)^{\frac1{p_1}} = 1 $$

	\begin{itemize}
		\item $ p_2 < +\infty $
			$$ \frac{p_2}{p_1} > 1, \quad |x_n| \le 1 \quad \forall n \implies
			|x_n|^{p_2} \le |x_n|^{p_1} \implies
			\|x\|_{p_2} = \Bigl( \sum_{n = 1}^\infty |x_n|^{p_2} \Bigr)^{\frac1{p_2}} \le
			\|x\|_{p_1} = 1 $$
			$$ \|A\| = \sup\limits_{\|x\|_{p_1} = 1}\|Ax\|_{p_2} \le 1 $$
			$$ \implies A \in \mathscr B(l^{p_1}, l^{p_2}) $$
		\item $ p_2 = +\infty $
			$$ x \in l^{p_1}, ~ \|x\|_{p_1} = 1 \implies |x_n| \le 1 \implies
			\|x\|_\infty = \sup\limits_{n \in \N} |x_n| \le 1 \implies
			\|A\|_{\mathscr B(l^{p_1}, l^{p_2})} \le 1 $$

			Возьмём $ e = (1, 0, 0, \dotsc) $.
			$$ \|e\|_p = 1 \quad \forall p $$
			$$ \|Ae\|_{p_2} = \|e\|_{p_1} \implies \|A\| \ge 1, \quad
			\|A\|_{\mathscr B(l^{p_1}, l^{p_2})} = 1 $$
	\end{itemize}
\end{proof}

\begin{theorem}
	$ (T, \mathcal U, \mu), \quad \mu(T) < +\infty, \quad
	1 \le p_1 < p_2 \le +\infty, \quad
	f \in \mathrm L^{p_2}(\mu), \quad
	Af = f $

	$$ \implies A \in \mathscr B(\mathrm L^{p_2}, \mathrm L^{p_1}), \quad
	\|A\|_{\mathscr B(\mathrm L^{p_2}, \mathrm L^{p_1})} = \Bigl( \mu(T) \Bigr)^{\frac1{p_1} - \frac1{p_2}} $$
\end{theorem}

\begin{iproof}
\item $ p_2 = +\infty $

	Пусть $ f \in \mathrm L^\infty(T, \mu) $, \ie
	$$ |f(x)| \le \|f\|_\infty \text{ \ale по } \mu $$
	$$ \|Af\|_{p_1} = \Bigl( \int\limits_T |f(x)|^{p_1} \di \mu \Bigr)^{\frac1{p_1}} \le
	\|f\|_\infty \Bigl( \int\limits_T \di \mu \Bigr)^{\frac1{p_1}} =
	\|f\|_\infty \Bigl( \mu(T) \Bigr)^{\frac1{p_1}} $$
	$$ \implies A \in \mathscr B(\mathrm L^\infty, \mathrm L^{p_1}), \quad
	\|A\| \le \Bigl( \mu(T) \Bigr)^{\frac1{p_1}} $$
\item $ p_2 < +\infty $

	Пусть $ f \in \mathrm L^{p_2} $
	\begin{multline*}
		\|Af\|_{p_1} = \Bigl( \int\limits_T |f|^{p_1} \di \mu \Bigr)^{\frac1{p_1}} \underset{
			\begin{subarray}{c}
				\text{нер-во Гёльдера} \\
				p = \frac{p_2}{p_1} > 1 \\
				g = 1, ~
				f = |f|^{p_1}
			\end{subarray}}\le
		\Bigl\lgroup \Bigl( \int\limits_T (|f|^{p_1})^{\frac{p_2}{p_1}} \di \mu \Bigr)^{\frac{p_1}{p_2}}
		\Bigl( \int\limits_T \di \mu \Bigr)^{1 - \frac{p_1}{p_2}} \Bigr\rgroup^{\frac1{p_1}} = \\
		= \|f\|_{p_2} \cdot \Bigl( \mu(T) \Bigr)^{\frac1{p_1} - \frac1{p_2}} \implies
		A \in \mathscr B(\mathrm L^{p_2}, \mathrm L^{p_1}), \quad
		\|A\| \le \bigl( \mu(T) \bigr)^{\frac1{p_1} - \frac1{p_2}}
	\end{multline*}
\item $ f = \chi_T(x) \in \mathrm L^p, \quad 1 \le p \le +\infty $
	$$ \|\chi_T\|_{\mathrm L^p} = \Bigl( \int\limits_T \chi_T(x) \di \mu \Bigr)^{\frac1p} =
	\bigl( \mu(T) \bigr)^{\frac1p} $$
	$$ \|A\|_{\mathscr B(\mathrm L^{p_2}, \mathrm L^{p_1})} = \sup\limits_{
		\begin{subarray}{c}
			g \in \mathrm L^{p_2} \\
			g \ne \On
	\end{subarray}} \frac{\|Ag\|_{p_2}}{\|g\|_{p_2}} =
	\sup \frac{\|g\|_{p_1}}{\|g\|_{p_2}} \ge
	\frac{\|\chi_T\|_{p_1}}{\|\chi_T\|_{p_2}} =
	\bigl( \mu(T) \bigr)^{\frac1{p_1} - \frac1{p_2}} $$
\end{iproof}

\begin{remark}
	Если $ \mu(T) = +\infty $, то всё возможно.
\end{remark}

\begin{eg}
	$ T = [0, +\infty), \quad \lambda $ "--- мера Лебега.

	Докажем, что $ \mathrm L^1 \not\sub \mathrm L^2, \quad
	\mathrm L^2 \not\sub L^1 $.

	Возьмём $ f(x) = \frac1{1 + x} $
	$$ f \notin \mathrm L^1, \quad f \in \mathrm L^2 \quad
	\implies L^1 \not\sub L^2 $$
	Возьмём $ g(x) = \frac1{\sqrt x + x^2} $.
	$$ g \in \mathrm L^1, \quad f \notin \mathrm L^2 \quad
	\implies \mathrm L^1 \not\sub \mathrm L^2 $$
\end{eg}

\begin{theorem}\label{th:full:1}
	$ (X, \|\cdot\|), \quad (Y, \|\cdot\|) $ "--- банахово.

	Тогда $ \mathscr B(X, Y) $ "--- банахово.
\end{theorem}

\begin{proof}
	$ \set{A_n}_{n = 1}^\infty, \quad
	A_n \in \mathscr B(X, Y), \quad
	\set{A_n}_{n = 1}^\infty $ фундаментальна.

	$$ \forall \eps > 0 \quad \exists N \in \N : \quad \forall n, m > N \quad \|A_n - A_m\| < \eps $$
	Зафиксируем $ x \in X $.
	Проверим, что $ \set{A_nx} $ фундаментальна.
	$$ \|A_nx - A_mx\| < \|A_n - A_m\| \cdot \|x\| < \eps \|x\| \implies
	\set{A_nx}_{n = 1}^\infty \text{ фундаментальна в } Y \underimp{Y \text{ банахово}}
	\exists \lim\limits_{n \to \infty} A_nx $$
	Положим
	$$ Ax \define \lim\limits_{n \to \infty} A_nx $$
	$$ A_n \in \mathscr Lin(X, Y) \implies
	A \in \mathscr Lin(X, Y) $$
	Уже знаем, что $ \|A_nx - A_mx\| < \eps \|x\| $.
	Зафиксируем $ m $ и перейдём к пределу по $ n $:
	$$ \|Ax - A_mx\| \le \eps \|x\| \implies
	A - A_m \in \mathscr B(X, Y), \quad \|A - A_m\| \le \eps \implies
	A \in \mathscr B(X, Y), \quad \lim\limits_{m \to \infty} \|A - A_m\| = 0 $$
\end{proof}

\section{Линейные функционалы.
Примеры.
Вычисление норм интегрального функционала и интегрального оператора в
\texorpdfstring{$ \mathcal C[a, b] $}{пространстве непрерывных функций}}

\begin{definition}
	$ (X, \|\cdot\|) $ "--- нормированное над полем $ \mathbb K $.

	$ X^* $ называется \emph{сопряжённым пространством}, если
	$$ X^* = \mathscr B(X, \mathbb K) $$
	То есть $ X^* $ "--- множество линейных непрерывных функционалов.

	$$ f \in X^* \quad \|f\|_{X^*} = \inf\set{C > 0 | \ |f(x)| \le C\|x\|} $$
\end{definition}

\begin{implication}
	$ (X, \|\cdot\|) \implies X^* $ "--- банахово.
\end{implication}

\begin{proof}
	$ \R, \Co $ "--- полные.
	Можно воспользоваться \autoref{th:full:1}.
\end{proof}

\begin{implication}
	$ f \in X^* $

	$$ \|f\|_{X^*} = \sup\limits_{\|x\| \le 1}|f(x)| = \sup\limits_{\|x\| < 1} |f(x)| =
	\sup\limits_{\|x\| = 1} |f(x)| = \sup\limits_{x \ne 0} \frac{|f(x)|}{\|x\|} $$
\end{implication}

\begin{exmpls}
\item $ l^p, \quad $ зафиксируем $ i \in \N $
	$$ x = \set{x_n}_{n = 1}^\infty \in l^p, \quad f(x) \define x_i $$
	Докажем, что $ f \in (l^p)^*, \quad \|f\|_{(l^p)^*} = 1 $.
	$$ |f(x)| = |x_i| \le \Bigl( \sum_{n = 1}^\infty |x_n|^p \Bigr)^{\frac1p}, \quad 1 \le p < +\infty $$
	$$ |x_i| < \sup\limits_{n \ge 1}|x_n| = \|x\|_\infty, \quad p = +\infty $$
	$$ \implies f \in (l^p)^* \implies \|f\|_{(l^p)^*} \le 1 $$

	Возьмём $ e_i = (0, \dots, 0, \underset i 1, 0, \dots, 0) $.
	$$ \|e_i\|_p = 1, \quad |f(e_i)| = 1 \implies
	\|f\|_{(l^p)^*} \ge 1 $$
\item $ X = \mathcal C(K), \quad K $ "--- компакт, $ \quad x_0 \in K $ "--- фиксирована.
	$$ G : \mathcal C(K) \to \Co : \quad f \in \mathcal C(k) \quad G(f) = f(x_0) \text{ "--- \emph{функционал значения в точке}} $$
	Докажем, что $ G \in \bigl( \mathcal C(K) \bigr)^*, \quad \|G\| = 1 $.
	$$ f \in \mathcal C(K) \quad |G(f)| = |f(x_0)| \le \max\limits_{x \in K}|f(x)| =
	\|f\|_{\mathcal C(K)} $$
	$$ \implies G \in \bigl( \mathcal C(K) \bigr)^*, \quad \|G\|_{\bigl( \mathcal C(K) \bigr)^*} \le 1 $$

	Пусть $ f(x) = \chi_K $.
	$$ \chi_K(x) \in \mathcal C(K), \quad \|\chi_K\|_\infty = 1, \quad
	|G(\chi_K)| = |\chi_K(x_0)| = 1 \implies
	\|G\| \ge 1 $$
\end{exmpls}

\begin{theorem}
	$ \phi(x) \in \mathcal [a, b], \quad \phi(x) $ фиксирована, $ \quad f \in \mathcal C[a, b] $.
	$$ G(f) \define \int_a^b f(x) \cdot \phi(x) \di x $$

	$$ \implies G \in \bigl( C[a, b] \bigr)^*, \quad \|G\| = \int_a^b |\phi(x)| \di x $$
\end{theorem}

\begin{iproof}
\item $ f \in \mathcal C[a, b] $
	\begin{multline*}
		|G(f)| = \Bigl| \int_a^b \phi(x) \cdot f(x)\di x\Bigr| \le \int_a^b |f(x)| \cdot |\phi(x)| \di x \le
		\|f\|_\infty \cdot \int_a^b |\phi(x)| \di x \quad \forall x \implies \\
		\implies G \in \bigl( \mathcal C[a, b] \bigr)^*, \quad
		\|G\| \le \int_a^b |\phi(x)| \di x
	\end{multline*}
\item $ \phi(x) \overset{[a, b]}> 0 $

	Пусть $ f(x) = \chi_{[a, b]} $.
	$$ \|\chi_{[a, b]} = 1 $$
	$$ |G(\chi_{[a, b]})| = \Bigl| \int_a^b \phi(x) \di x \Bigr| \undereq{\phi(x) > 0}
	\int_a^b \phi(x) \di x $$
\item $ \phi(x) \overset{[a, b]}\le 0 $
	$$ |G(\chi_{[a, b]})| = \Bigl| \int_a^b \phi(x) \di x \Bigr| = \int_a^b |\phi(x)| \di x $$
\item $ \phi $ "--- произвольная
	$$ g(x) \define \operatorname{sign} \phi(x), \quad \|g\|_\infty = 1 $$
	$$ \int_a^b g\phi = \int_a^b |\phi(x)| \di x $$
	$ g $ не является непрерывной, поэтому подставить её мы не можем.
	Приблизим её непрерывными:
	$$ \eps > 0 \quad \phi(t) \text{ равномерно непрерывна } \implies \exists \delta > 0 : \quad
	\forall |t - s| < \delta \quad
	|\phi(t) - \phi(s)| < \eps $$
	$$ a = t_0 < t_1 < \dots < t_n = b \text{ "--- разбиение } [a, b] : \quad
	t_k - t_{k - 1} < \delta $$
	$$ x, t \in [t_{k - 1}, t_k] \implies |\phi(s) - \phi(t)| < \eps $$
	$$ \mathscr F \define \set{[t_{k - 1}, t_k]}_{k = 1}^n $$
	Разобьём отрезки на два множества:
	\begin{enumerate}
		\item $ \Delta_1, \dots, \Delta_2 $ "--- отрезки такие, что $ \phi(t) \overset{\Delta_j}> 0 $
			или $ \phi(t) \overset{\Delta_j} < 0 $;
		\item $ \Delta_{r + 1}, \dots, \Delta_n $ "--- отрезки, на которых
			$ \exists s \in \Delta_j : \quad \phi(s) = 0 $;
	\end{enumerate}

	Проверим, что вклад отрезков второго сорта в интеграл очень маленький.
	Пусть $ r + 1 \le j \le n, \quad t \in \Delta_j, \quad \exists s \in \Delta_j : \quad \phi(s) = 0 $.
	$$ |\phi(t)| = |\phi(t) - \phi(s)| < \eps \implies
	\int\limits_{\Delta_j} |\phi(t)| \di t \le \eps |\Delta_j| \implies
	\int\limits_{\bigcup_{j = r + 1}^n \Delta_j}|\phi(t)| \di t \le
	\eps \Bigl( \sum_{j = r + 1}^n |\Delta_j| \Bigr) \le
	\eps(b - a) $$
	$$ h(t) =
	\begin{cases}
		\operatorname{sign} \phi(t), \quad t \in \Delta_j, \quad 1 \le j \le r \\
		\text{линейная на } \Delta_j, \quad r + 1 \le j \le n \\
		\text{если } [a, t_1] = \Delta_j, \quad r + 1 \le j \le n, \text{ то } h(a) \define 0 \\
		\text{если } [t_{n - 1}, b] = \Delta_j, \quad r + 1 \le j \le n, \text{ то } h(b) \define 0
	\end{cases} $$
	Понятно, что
	$$ h(t) \in \mathcal C[a, b], \quad \|h\|_\infty \le 1, \quad
	f(t) \cdot \phi(t) = |\phi(t)|, \quad
	t \in \Delta_j, \quad 1 \le j \le r $$
	$$ |h(t) \cdot \phi(t)| \le |\phi(t)| \quad \forall t \in [a, b] $$
	\begin{multline*}
		\|G\| = \sup\limits_{\|f\| \le 1} |G(f)| \ge |G(h)| =
		\Bigl| \int_a^b \phi(t) \cdot h(t) \di t \Bigr| =
		\Bigl| \int\limits_{\bigcup_{j = 1}^r \Delta_j} |\phi(t)| \di t + \int\limits_{\bigcup_{j = r + 1}^n \Delta_j} \phi(t) \cdot h(t) \di t \Bigr| \ge \\
		\ge \int\limits_{\bigcup_{j = 1}^n \Delta_j} |\phi(t)| \di t - \Bigl| \int\limits_{\bigcup_{j = r + 1}^n \Delta_j} |\phi(t)| \cdot |h(t)| \di t \Bigr| \ge
		\int\limits_{\bigcup_{j = 1}^n \Delta_j} |\phi(t)| \di t - \int\limits_{\bigcup_{j = r + 1}^n \Delta_j} |\phi(t)| \di t = \\
		= \int_a^b |\phi(t)| \di t - 2 \int\limits_{\bigcup_{j = r + 1}^n \Delta_j} |\phi(t)| \di t \ge
		\int_a^b |\phi(t)| \di t - 2 \eps(b - a) \quad \forall \eps > 0
	\end{multline*}
	$$ \implies \|G\| \ge \int_a^b |\phi(t)| \di t \implies
	\|G\|_{\mathcal C^*[a, b]} =
	\int_a^b |\phi(t)| \di t $$
\end{iproof}

\begin{theorem}
	$ K(s, t) \in \mathcal C \bigl( [a, b] \times [a, b] \bigr), \quad f \in \mathcal C[a, b] $
	$$ (\mathscr Kf)(s) = \int_a^b K(s, t)f(t) \di t $$

	$$ \implies \mathscr K \in \mathscr B \bigl( \mathcal C[a, b] \bigr), \quad
	\|\mathscr K\|_{\mathscr B(\mathcal C[a, b])} =
	\underbrace{\max\limits_{a \le s \le b} \int_a^b |K(s, t)| \di t}_M $$
\end{theorem}

\begin{iproof}
\item $ \|\mathscr K\| \le M $

	Возьмём $ f \in \mathcal C[a, b], \quad s \in [a, b] $.
	\begin{multline*}
		|(\mathscr Kf)(s)| =
		\Bigl| \int_a^b K(s, t) f(t) \di t \Bigr| \le
		\|f\|_\infty \int_a^b |K(s, t) \di t \le
		M \cdot \|f\|_\infty \implies \\
		\implies \|\mathscr K f\|_\infty = \max\limits_{a \le s \le b} |(\mathscr Kf)(s)| \le
		M \cdot \|f\|_\infty \implies \mathscr K \in \mathscr B \bigl( \mathcal C[a, b] \bigr), \quad
		\|\mathscr K\| \le M
	\end{multline*}
\item $ \ge $

	$$ g(s) \define \int_a^b |K(s, t)| \di t \implies g \in \mathcal C[a, b] \implies
	\exists s_0 : \quad \max g(s) = g(s_0) = M $$
	$$ \phi(t) = K(s_0, t) $$
	$$ f \in \mathcal C[a, b], \quad \|f\|_\infty \le 1 $$
	$$ \|\mathscr Kf\| = \max\limits_{a \le s \le b} |(\mathscr Kf)(s)| \ge |(\mathscr Kf)(s_0)| =
	\Bigl| \int_a^b K(s_0, t)f(t) \di t \Bigr| =
	|G(f)|, \text{ где } G(f) = \int_a^b f(t) \cdot \phi(t) \di t $$
	По предыдущей теореме
	$$ \|G\|_{\mathcal C^* [a, b]} = \int_a^b |K(s_0, t)|\di t = M $$

	$$ \|K\| = \sup\limits_{\|f\| \le 1} \|\mathscr Kf\| \ge
	\sup\limits_{\|f\| \le 1} |G(f)| = \|G\| = M $$
\end{iproof}

\section{Изоморфизм пространств, эквивалентные нормы.
Свойства линейно-изоморфных пространств}

\begin{definition}
	$ (X, \|\cdot\|), ~ (Y, \|\cdot\|) $ над $ \R $ (или $ \Co $)

	$ (X, \|\cdot\|) $ \emph{линейно изоморфно} $ (Y, |\cdot\|) $, если
	$$ \exists A \in \mathscr B(X, Y), \quad \exists A^{-1} \in \mathscr B(Y, X) $$
	$ A $ называется \emph{линейным изоморфизмом}
\end{definition}

\begin{theorem}
	$ (X, \|\cdot\|), ~ (Y, \|\cdot\|), \quad A : X \to Y $

	$ A $ "--- линейный изоморфизм тогда и только тогда, когда
	\begin{enumerate}
		\item $ A \in \mathscr Lin(X, Y) $;
		\item $ A(X) = Y $ ($ A $ "--- сюръекция);
		\item $ \exists C_2 > C_1 > 0 : \quad C_1\|x\|_X \le \|Ax\|_Y \le C_2 \|x\|_X \quad \forall x \in X $.
	\end{enumerate}
\end{theorem}

\begin{iproof}
\item $ \implies $

	$$ A \in \mathscr B(X, Y) \implies A \in \mathscr Lin(X, Y) $$
	$$ \exists A^{-1} \implies A(X) = Y $$
	$$ A \in \mathscr B(X, Y) \implies
	\|Ax\| \le \|A\| \cdot \|x\| \quad \forall x \in X \implies C_2 = \|A\| $$
	$$ A^{-1} \in \mathscr B(Y, X) \implies \|A^{-1}y\| \le \|A^{-1}\| \cdot \|y\| $$
	Возьмём $ x \in X, \quad y \define Ax $.
	$$ A^{-1}y = x $$
	$$ \|x\| \le \|A^{-1}\| \cdot \|Ax\| \implies
	\frac1{\|A^{-1}\|} \cdot \|x\| \le \|Ax\| \implies
	C_1 = \frac1{\|A^{-1}\|} $$
\item $ \impliedby $

	$$ \|Ax\| \le C_2 \|x\| \implies A \in \mathscr B(X, Y), \quad \|A\| \le C_2 $$

	Проверим, что $ A $ "--- инъекция, \ie ядро состоит только из нуля:
	$$ Ax = 0 \implies C_1\|x\| \le \|Ax\| \implies \|x\| = 0 \implies x = 0 $$
	$$ \implies \exists A^{-1}, \quad A^{-1} \in \mathscr Lin (Y, X) $$

	Возьмём $ y \in Y $.
	$$ \exists! x : \quad Ax = y, \quad x = A^{-1}y $$
	$$ C_1\|x\| \le \|Ax\| \implies C_1\|A^{-1}y\| \le \|y\| \implies
	\|A^{-1}y\| \le \frac1{C_1}\|y\| \implies
	A^{-1} \in \mathscr B(Y, X), \quad \|A^{-1}\| \le \frac1{C_1} $$
\end{iproof}

\begin{implication}[критерий обратимости линейного оператора]
	$$ (X, \|\cdot\|), ~ (Y, \|\cdot\|), \quad A \in \mathscr Lin(X, Y), \quad
	A(X) = Y, \quad \exists C_1 > 0 : \quad C_1 \|x\| \le \|Ax\| $$

	$$ \implies \exists A^{-1} \in \mathscr B(Y, X) $$
\end{implication}

\begin{statement}
	$ (X, \|\cdot\|), ~ (Y, \|\cdot\|), \quad A $ "--- линейный изоморфизм.

	Тогда
	\begin{enumerate}
		\item $ (X, \|\cdot\|) $ банахово $ \iff (Y, \|\cdot\|) $ банахово;
		\item $ K \sub X $ "--- компакт $ \iff A(K) $ "--- компакт в $ Y $;
		\item $ K \sub X $ относительно компактно $ \iff A(K) $ относительно компактно в $ Y $.
	\end{enumerate}
\end{statement}

\begin{eproof}
\item $ (X, \|\cdot\|) $ "--- банахово.
	Проверим, что $ Y $ "--- банахово.

	Возьмём $ \set{y_n}_{n = 1}^\infty $ "--- фундаментальная в $ Y $.
	Докажем что она имеет предел.

	$$ x_n \define A^{-1}y_n $$
	$$ \lim\limits_{n, m \to \infty}\|y_n - y_m\| = 0 $$
	\begin{multline*}
		\|x_n - x_m\| = \|A^{-1}y_n - A^{-1}y_m\| =
		\|A^{-1}(y_n - y_m)\| \le \\
		\le \|A^{-1} \cdot \|y_n - y_m\| \underarr{n, m \to \infty} 0 \implies
		\set{x_n}_{n = 1}^\infty \text{ фундаментальна}
	\end{multline*}
	$ X $ "--- банахово $ \implies \exists \lim x_n = x_0 \implies \lim y_n = Ax_0 \implies
	Y $ "--- банахово.
\item $ K $ "--- компакт, $ A $ непрерывно $ \implies A(K) $ "--- компакт.
\item $ K $ относительно компактно, $ A $ непрерывно $ \implies A(K) $ относительно компактно.
\end{eproof}

\begin{definition}
	$ X $ "--- линейное пространство над $ \R $(или $ \Co $), $ \quad
	\|\cdot\|_1, ~ \|\cdot\|_2 $\ "--- различные нормы на $ X $.

	$ \|\cdot\|_1 $ \emph{эквивалентна} $ \|\cdot\|_2 $, если
	$$ \forall \set{x_n}_{n = 1}^\infty : x_n \in X, ~ x_0 \in X \quad
	\lim\limits_{n \to \infty}\|x_n - x_0\|_1 = 0 \iff \lim\limits_{n \to \infty} \|x_n - x_0\|_2 = 0, $$
	то есть, нормы порождают одну и ту же топологию.
\end{definition}

\begin{implication}[критерий эквивалентности норм]
	$ X $ "--- линейное, $ \quad \|\cdot\|_1, ~ \|\cdot\|_1 $ "--- нормы на $ X $.

	$$ \|\cdot\|_1 \text{ эквивалентна } \|\cdot\|_2 \iff
	\exists 0 < C_1 < C_2 < +\infty : \quad
	C_1\|x\|_1 \le \|x\|_2 \le C_2\|x\|_1 $$
\end{implication}

\begin{proof}
	Обозначим $ X = (X, \|\cdot\|_1), ~ Y = (X, \|\cdot\|_2) $.
	Определим $ I : X \to Y : \quad Ix = x $
	$$ \lim\limits_{n \to \infty} \|x_n - x_0\| = 0 \implies
	\lim\limits_{n \to \infty} \|Ix_n - Ix_0\|_2 = 0 \implies
	I \in \mathscr B(X, Y) $$
	\begin{multline*}
		I \text{ "--- биекция } \iff
		I \in \mathscr B(Y, X) \implies
		I \text{  "--- линейный изоморфизм } \underiff{\text{теорема}} \\
		\iff \exists 0 < C_1 < C_2 : \quad \|Ix\|_2 \le C_2\|x\|_1 \iff
		C_1\|x\|_1 \le \|x\|_ \le C_2\|x\|_1
	\end{multline*}
\end{proof}

\section{Изоморфизм конечномерных пространств, эквивалентность норм, полнота, характеристика
относительно компактных и компактных множеств, непрерывность линейных операторов}

\begin{definition}
	$ X $ "--- линейное пространство.

	Говорят, что $ \dim X = n \in \N $, если $ \exists \set{x_1, \dots, x_n} $ "--- ЛНЗ, при этом
	$ \forall \set{x_j}_{j = 1}^{n + 1} $ "--- ЛЗ.

	Если $ \forall n \in \N \quad \exists \set{x_j}_{j = 1}^n $ "--- ЛНЗ, то $ \dim X = \infty $.
\end{definition}

\begin{theorem}
	$ (X, \|\cdot\|), ~ (Y, \|\cdot\|) $ "--- линейные над $ \R $ (или $ \Co $), $ \quad
	\dim X = \dim Y = n \in \N $.

	Тогда $ X $ линейно изоморфно $ Y $.
\end{theorem}

\begin{proof}
	Рассмотрим $ Z = l_n^2 = (\R^n, \|\cdot\|_2) $.
	Докажем, что $ l_n^2 $ и $ X $ линейно изоморфны.
	Этого достаточно в силу транзитивности.

	Пусть $ \set{f_j}_{j = 1}^n $ "--- базис в $ X, \quad
	e_j = (0, \dots, 0, \underset j 1, 0, \dots, 0) $ "--- базис в $ l_n^2 $.

	Определим $ A : l_n^2 \to X : \quad Ae_j = f_j $.
	$$ A \Bigl( \sum_{j = 1}^n c_je_j \Bigr) \define \sum_{j = 1}^n c_jf_j \implies
	A \in \mathscr Lin(l_n^2, X) $$
	Понятно, что $ A $ "--- биекция.
	\begin{itemize}
		\item Проверим непрерывность
			$$ z \in l_n^2, ~ z = \sum_{j = 1}^n c_je_j \quad \|A(z)\| =
			\Bigl\| \sum_{j = 1}^n c_jf_j\|_X \le \sum_{j = 1}^n |c_j| \cdot \|f_j\| \underset{\text{КБ}}\le
			\Bigl( \sum_{j = 1}^n |c_j|^2 \Bigr)^{\frac12} \Bigl( \sum_{j = 1}^n \|f_j\|^2 \Bigr)^{\frac12} =
			M\|z\|_{l_n^2} $$
			$$ \implies A \in \mathscr B(l_n^2, X), \quad \|A\| \le M $$
		\item Найдём $ c > 0 : \quad \|Az\| \ge c\|z\| \quad \forall z \in l_n^2 $
			$$ g(z) \define \|Az\|, \quad z \in l_n^2 $$
			$ g(z) $ непрерывна на $ l_n^2 $.
			$$ S \define \set{z \in l_n^2 | \|z\| = 1} $$
			$ S $ "--- компакт в $ l_n^2 $.
			$$ \exists \min\limits_{z \in S} g(z) = g(z_0) = r > 0 \implies
			\forall z \in S \quad \|Az\| \ge r $$
			Возьмём $ u \in l_n^2 \ne 0 $.
			$$ \frac u{\|u\|} \in S \implies
			\Bigl\| A \Bigl( \frac u{\|u\|} \Bigr)\Bigr\| \ge r \implies
			\|Au\| \ge r\|u\| \underimp{\text{критерий лин. изоморфности}}
			l_n^2 \text{ линейно изоморфно } X $$
	\end{itemize}
\end{proof}

\begin{implication}
	$ (X, \|\cdot\|), \quad \dim X < +\infty $

	\begin{enumerate}
		\item $ X $ "--- банахово;
		\item $ K \sub X $ "--- компакт $ \iff K $ ограничено и замкнуто;
		\item $ K \sub X $ относительно компактно $ \iff K $ ограничено.
	\end{enumerate}
\end{implication}

\begin{eproof}
\item $ l_n^2 $ "--- банахово $ \implies X $ "--- банахово.
\item $ K \sub X, \quad A : X \to l_n^2 $ "--- линейный изоморфизм $ \implies
	A, A^{-1} $ ограничены.

	$ K $ "--- компакт $ \implies \Bigl( A(K) $ "--- компакт $ \iff
	K $ ограничено и замкнуто $ \Bigr) $.
	$$ \implies K = A^{-1} \bigl( A(K) \bigr), \quad
	K \text{ ограничено и замкнуто} $$
\end{eproof}

\begin{theorem}
	$ (X, \|\cdot\|), \quad \dim X < +\infty, \quad (Y, \|\cdot\|) $

	$$ \mathscr Lin(X, Y) = \mathscr B(X, Y) $$
\end{theorem}

\begin{iproof}
\item Пусть $ T \in \mathscr Lin(l_n^2, X), \quad z \in l_n^2 $.
	$$ z = \sum_{j = 1}^n c_je_j, \quad e_j = (0, \dots, 0, \underset j 1, 0 \dots, 0) $$
	\begin{multline*}
		\|Tz\| = \Bigl\| \sum_{j = 1}^n c_j Te_j \Bigr\| \le
		\sum_{j = 1}^n |c_j| \|Te_j\| \underset{\text{К"--~Б}}\le
		\Bigl( \sum_{j = 1}^n |c_j|^2 \Bigr)^{\frac12} \Bigl( \sum_{j = 1}^n \|Te_j\|^2 \Bigr)^{\frac12} \implies \\
		\implies \|Tz\| \le M \|z\| \implies
		T \in \mathscr B(l_n^2, X)
	\end{multline*}
\item $ U \in \mathscr Lin(X, Y) $

	Пусть $ l_n^2 \overarr A X \overarr U Y, \quad A $ "--- линейный изоморфизм.
	Положим $ T = UA $.
	$$ \implies T \in \mathscr Lin(l_n^2, Y) = \mathrm B(l_n^2, Y) \implies
	T \in \mathscr B(l_n^2, Y) $$
	$$ U = TA^{-1} \implies U \in \mathscr B(X, Y) $$
\end{iproof}

\begin{implication}
	$ \dim X = n \in \N, \quad \|\cdot\|_1, ~ \|\cdot\|_2 $ "--- нормы на $ X $.

	Тогда $ \|\cdot\|_1 $ и $ \|\cdot\|_2 $ эквивалентны.
\end{implication}

\section{Конечномерные подпространства: замкнутость, существование элемента наилучшего приближения.
Существование многочлена наилучшего приближения}

\begin{definition}
	$ (X, \rho) $ "--- метрическое пространство, $ \quad Y \sub X, \quad a \in X $

	$$ \rho(a, Y) = \inf\limits_{y \in Y} \rho(a, y) $$
	Если $ \exists y_0 \in Y : \quad \rho(a, Y) = \rho(a, y_0) $, то $ y_0 $ называется
	\emph{элементом наилучшего приближения}.
\end{definition}

\begin{remark}
	Если $ Y $ "--- компакт, то элемент наилучшего приближения существует
	(\as $ \rho(a, y) $ непрерывна).
\end{remark}

\begin{theorem}
	$ (X, \|\cdot\|) $
	\begin{enumerate}
		\item $ L \in X, \quad L $ "--- конечномерное подпространство в алгебраическом смысле
			$ \implies L $ замкнуто;
		\item $ a \in X \implies $ в $ L $ существует элемент наилучшего приближения.
	\end{enumerate}
\end{theorem}

\begin{eproof}
\item $ \dim L < +\infty \implies L $ "--- банахово $ \implies L $ замкнуто.
\item $ a \in X, \quad f = \rho(a, L) = \inf\|a - y\| $
	$$ \exists \set{y_n}_{n = 1}^\infty : \quad d \le \|a - y_n\| \le d + \frac1n $$
	Проверим, что $ \set{y_n} $ ограничена:
	$$ \|y_n\| \trile \|a\| + \|y_n - a\| \le \|a\| + d + 1 $$
	$$ \dim L < +\infty \implies
	\set{y_n}_{n = 1}^\infty \text{ относительно компактна } \implies
	\exists \set{y_{n_j}} : \quad \exists \lim\limits_{j \to \infty} y_{n_j} = y_0 \in L $$
	$$ d \le \|a - y_{n_j}\| \le d + \frac1{n_j} \implies \|a - y_0\| = d $$
\end{eproof}

\begin{remark}
	Элемент наилучшего приближения не обязательно единственен.
\end{remark}

\begin{exmpls}
\item $ l_2^\infty = \{(x, y), \|(x, y)\| = \max\{|x|, |y|\}\} $
	$ \mathtt B_1(0, 0) $ выглядит в этом пространстве как квадрат.

	$ L = \set{y = kx}, \quad k \ne 0 $.
	Для $ L $ существует единственный элемент наилучшего приближения.
\item $ L = \set{y = 0} $.
	Для $ L $ элемент наилучшего приближения не единственен.
\item $ l_2^1 = \{ (x, y), \|(x, y)\| = |x| + |y|\} $
	$ B_1(0, 0) $ выглядит как квадрат, повёрнутый на $ \frac\pi4 $.

	$ L = \set{y = kx}, \quad k \ne \pm 1 $.
	Для $ L $ существует единственный элемент наилучшего приближения.
\item $ L = \set{y = x} $.
	Для $ L $ элементов наилучшего приближения бесконечно много.
\end{exmpls}

\begin{implication}
	$ \mathcal C[a, b], \quad \|f\|_\infty = \max|f(x)|, \quad
	\mathscr P_n = \set{\sum_{j = 0}^n a_jx^j, \quad a_j \in \R} $

	$$ \exists p \in \mathscr P_n : \quad \inf\limits_{q \in \mathscr P_n}\|f - q\|_\infty = \|f - p\| $$
	$ p $ называется \emph{многочленом наилучшего приближения}
\end{implication}

\begin{remark}
	Для $ \mathscr P_n $ существует единственный элемент наилучшего приближения.
\end{remark}

\section{Лемма Рисса о почти перпендикуляре, следствия их неё.
Теорема Рисса: критерий конечномерности пространства}

\begin{lemma}
	$ (X, \|\cdot\|), \quad L \subsetneq X $ "--- подпространство, $ L = \ol L, \quad 0 < \eps < 1 $

	$$ \exists x_0 : \quad \|x_0\| = 1, \quad \rho(x_0, L) > 1 - \eps $$
\end{lemma}

\begin{proof}
	Возьмём $ z \in X \setminus L $.
	$$ \rho(z, L) = d > 0 \quad \text{ (\as $ L = \ol L $)} $$
	$$ \rho(z, L) = \inf\limits_{y \in L} \|z - y\| $$
	$$ \exists y \in L : \quad d \le \|z - y\| < \frac d{1 - \eps} $$
	Выберем $ x_0 = \frac{z - y}{\|z - y\|} $.
	Возьмём $ u \in L $.
	$$ \|x_0 - u\| = \Bigl\| \frac{z - y}{\|z - y\|} - u \Bigr\| =
	\frac{\bigl\|z - y - u \cdot \|z - y\| \bigr\|}{\|z - y\|} \ge
	\frac d {\frac d {1 - \eps}} = 1 - \eps $$
\end{proof}

\begin{remark}
	Если $ \exists y_0 : \quad \rho(z, y_0) = d $, то $ x_0 = \frac{z - y_0}{\|z - y_0\|} \implies
	\|x_0 - u\| \ge 1 $
\end{remark}

\begin{implication}
	$ (X, \|\cdot\|), \quad L $ "--- подпространство, $ \quad \dim L < +\infty $

	$$ \exists x_0 : \quad
	\begin{cases}
		\|x_0\| = 1 \\
		\rho(x_0, L) = 1
	\end{cases} $$
\end{implication}

\begin{implication}
	$ (X, \|\cdot\|), \quad \set{L_n}_{n = 1}^\infty, \quad L_n \subsetneq L_{n + 1}, \quad
	L_n $ "--- замкнутые подпространства, $ \quad L_1 \ne \emptyset $

	$$ \exists \set{y_n}_{n = 1}^\infty : \quad
	\begin{cases}
		\|y_n\| = 1, \\
		y_n \in L_n, \\
		\rho(y_{n + 1}, L_n) > \frac12
	\end{cases} $$
\end{implication}

\begin{proof}
	\textbf{Индукция}.
	Пусть $ y_1 \in L_n, \quad \|y_1\| = 1, \quad L_n \subseteq L_2 $.
	По лемме
	$$ \exists y_2 \in L_2 : \quad
	\begin{cases}
		\|y_2\| = 1, \\
		\rho(y_2, L_1) > \frac12
	\end{cases} $$
\end{proof}

\begin{theorem}
	$ \ol{\mathtt B} $ "--- замкнутый единичный шар в пространстве $ X $.

	$$ \ol{\mathtt B} \text{ "--- компакт } \iff \dim X < +\infty $$
\end{theorem}

\begin{iproof}
\item $ \impliedby $

	$ \dim X < +\infty \implies \ol{\mathtt B} $ ограничен и замкнут $ \iff $ компакт
\item $ \implies $

	\textbf{Пусть} $ \dim X = +\infty $.
	Существует ЛНЗ набор $ \set{x_n}_{n = 1}^\infty $.
	Положим $ L_n = \mathscr L \set{x_j}_{j = 1}^\infty $ "--- линейная оболочка.
	$$ L_n \subsetneq L_{n + 1} \underimp{\text{второе следствие}}
	\exists \set{y_n} : \quad
	\begin{cases}
		\|y_n\| = 1, \\
		\|y_{n + 1} - y_m\| > \frac12 \quad \forall m \ne (n + 1)
	\end{cases} $$
	Значит, не существует фундаментальной подпоследовательности $ \set{y_{n_j}} $.
	Значит, $ \not\exists \lim\limits_{j \to \infty} y_{n_j} $.
\end{iproof}

\section{Продолжение линейного оператора со всюду плотного множества}

\begin{theorem}
	$ (X, \|\cdot\|) $ "--- нормированное, $ \quad (Y, \|\cdot\|) $ "--- банахово, $ \quad
	L \sub X $ "--- подпространство в алгебраическом смысле, $ \quad
	L $ всюду плотно, $ \quad A \in \mathscr B(L, Y) $

	$$ \exists ! V \in \mathscr B(X, Y) : \quad
	\begin{cases}
		V\big|_L = A, \\
		\|V\|_{\mathscr B(X, Y)} = \|A\|_{\mathscr B(L, Y)}
	\end{cases} $$
\end{theorem}

\begin{iproof}
\item Существование

	Возьмём $ x \in X $.
	$$ \exists \set{x_n \in L}_{n = 1}^\infty : \quad \lim x_n = x $$
	$$ \|Ax_n - Ax_m\| =
	\|A(x_n - x_m)\| \le \|A\| \cdot \|x_n - x_m\| \underarr{n, m \to \infty} 0 $$
	Значит, $ \set{Ax_n}_{n = 1}^\infty $ фундаментальна в $ Y \implies
	\exists \lim Ax_n \in Y $.
	Положим $ V_x = \lim Ax_n $.

\item Корректность определения (не получится тот же предел для другой последовательности)

	Пусть $ \lim z_n = x, \quad z_n \in L \quad \implies \exists \lim Az_n $
	$$
	\begin{rcases}
		\exists \lim Ax_n \\
		\exists \lim Az_n
	\end{rcases} \implies \lim \bigl( A(x_n - z_n) \bigr) = 0 \implies
	\lim Ax_n = \lim Az_n $$

	Пусть $ x \in L, \quad x_n \equiv[n \in \N] x \quad \implies
	Ax_n = Ax \implies V_x = \lim Ax_n = Ax $.
	$$ \lim x_n = x, \quad \|Ax_n\| \le \|A\| \cdot \|x_n\|, \quad x_n \in L $$
	$$ \implies \|Vx\| = \lim \|Ax_n\| \le \|A\| \cdot \lim \|x_n\| =
	\|A\| \cdot \|x\| \implies \|V\| \le \|A\| $$

	В другую сторону верно по определению:
	$$ \|V\| = \sup\limits_{\set{x \in X | \|x\| \le 1}} \overset{\sup \text{ по большему мн-ву}}\ge
	\sup\limits_{\set{x \in L | \|x\| \le 1}} \|Vx\| = \|A\| $$

	В итоге $ \|V\| = \|A\| $.

\item Единственность

	Пусть $ V, W \in \mathscr B(X, Y), \quad V\big|_L = A, \quad W\big|_L = A, \quad
	x \in X $.

	$$ \exists \set{x_n \in L} : \quad \lim x_n = x \implies
	V_x = \lim Vx_n = \lim Ax_n = \lim Wx_n = Wx $$
\end{iproof}

\section{Фактор-пространство нормированного и банахова пространства}

\begin{definition}
	$ X $ "--- линейное, $ \quad Y \sub X $ "--- подпространство в алгебраическом смысле.

	\emph{Фактор-пространство} $ X $ по $ Y $ "--- это
	$$ \faktor X Y = \set{\ol x}_{x \in X}, \quad \ol x = \set{z \in X | (z - x) \in Y} $$
\end{definition}

\begin{remark}
	$ \faktor X Y $ линейно.
\end{remark}

\begin{proof}
	$ \lambda \in \mathbb K $
	$$ \lambda \ol x = \ol{\lambda x}, \quad \ol x + \ol y = \ol{x + y}, \quad \ol{\On} = Y $$
\end{proof}

\begin{definition}
	$ (X, \|\cdot\|), \quad Y \sub X $ "--- замкнутое подпространство, $ \quad
	\ol x \in \faktor X Y $.

	\emph{Норма} $ \ol x $ "--- это
	$$ \|\ol x\| = \inf\limits_{z \in X}\|z\| = \inf\limits_{y \in Y}\|x - y\| = \rho(x, Y) $$
\end{definition}

\begin{definition}
	$ \phi : X \to \faktor X Y, \quad \phi(x) = \ol x $

	$ \phi $ называется \emph{каноническим гомоморфизмом}.
\end{definition}

\begin{remark}
	$ \phi \in \mathscr Lin\bigl(X, \faktor X Y\bigr) $
\end{remark}

\begin{theorem}
	$ (X, \|\cdot\|), \quad Y \sub X $ "--- замкнутое подпространство
	\begin{enumerate}
		\item $ \|\ol x\| $ удовлетворяет аксиомам нормы;
		\item $ \phi \in \mathscr B \bigl( X, \faktor X Y \bigr) $,
			причём $ \|\phi\| = 1 $, если $ Y \subsetneq X $;
		\item если $ X $ "--- банахово, то $ \faktor X Y $ "--- банахово.
	\end{enumerate}
\end{theorem}

\begin{eproof}
\item
	\begin{itemize}
		\item $ \|\ol x\| = 0 \implies \rho(x, Y) = 0 $

			$ Y $ замкнуто $ \implies $ расстояние может быть равно нулю только при $ x \in Y \implies
			\ol x = \ol\On $
		\item $ \alpha \in \Co, \quad \alpha \ol x = \ol{\alpha x} $
			$$ \|\ol{\alpha x}\| = \inf\limits_{x \in X}\|\alpha x\| =
			|\alpha| \cdot \inf\limits_{x \in \ol x}\|x\| =
			|\alpha| \cdot \|\ol x\| $$
		\item $ \ol x, \ol z \in \faktor X Y, \quad u \in \ol x, \quad v \in \ol z $
			$$ \|\ol x + \ol z\| = \|\ol{x + z}\| \le \|u + v\| \le \|u\| + \|v\| $$
			В силу произвольности $ u $ и $ v $, можно взять $ \inf $ в неравенствах:
			$$ \|\ol x + \ol z\| \le
			\inf\limits{u \in \ol x}\|u\| + \inf\limits_{v \in \ol z}\|v\| =
			\|ol x\| + \|\ol z\| $$
	\end{itemize}
\item $ \phi(x) = \ol x $
	$$ \|\ol x\| = \inf\limits_{u \in \ol x}\|u\| \underset{x \in \ol x}\le \|x\| \implies
	\|\phi(x)\| \le \|x\| \implies \|\phi\| \le 1 $$

	Пусть $ Y \subsetneq X $.
	Возьмём $ \eps > 0 $.
	По лемме Рисса имеем
	$$ \exists x_0 : \rho(x_0, Y) > 1 - \eps $$
	$$ \implies \|\phi\| = \sup\limits_{\|x\| \le 1} |\phi(x)| \ge |\phi(x_0)| =
	\rho(x_0, Y) > 1 - \eps $$
	Значит, $ \|\phi\| \le 1 $ и $ \|\phi\| > 1 - \eps \implies \|\phi\| = 1 $.
\item $ X $ "--- банахово

	Воспользуемся критерием полноты:
	$$ \set{\ol x_n}_{n = 1}^\infty : \quad \sum_{n = 1}^\infty \|\ol x_n\| < +\infty \overset?\implies
	\sum_{n = 1}^\infty \ol x_n < +\infty $$
	$$ \exists z_n \in \ol x_n : \quad \|z_n\| \le 2\|\ol x_n\| \implies
	\sum_{n = 1}^\infty \|z_n\| < +\infty $$
	По критерию полноты к $ X $ имеем, что $ \exists z = \sum z_n, \quad z \in X $.
	$$ S_n = \sum_{j = 1}^n z_j, \quad \lim S_n = z $$
	$ \phi $ непрерывно $ \implies \lim \phi(S_n) = \phi(z) $
	$$ \phi(z) = \ol z, \quad \phi(S_n) = \sum_{j = 1}^n \ol z_j = \sum \ol x_j $$
\end{eproof}

\section{Гильбертово пространство.
Примеры.
Замкнутость ортогонального дополнения.
Непрерывность скалярного произведения, тождество параллелограмма}

\begin{definition}
	$ H $ "--- линейное пространство над $ \Co $ \emph{со скалярным произведением}, то есть имеется
	отображение $ H \times H \to \Co $, если
	\begin{enumerate}
		\item $ (\lambda x, y) = \lambda (x, y), \quad \lambda \in \Co, \quad x, y \in H $;
		\item $ (x + z, y) = (x, y) + (z, y), \quad x, y, z \in H $;
		\item $ (y, x) = \ol{(x, y)}, \quad x, y \in H $;
		\item $ (x, x) \ge 0, \quad (x, x) = 0 \iff x = \On[] $.
	\end{enumerate}
\end{definition}

\begin{definition}
	Норма, \emph{порождённая скалярным произведением}: $ \|x\| = \sqrt{(x, x)} $.
\end{definition}

\begin{definition}
	$ (H, \|\cdot\|), \quad \|\cdot\| $ порождена $ (\cdot, \cdot) $

	$ H $ называется \emph{предгильбертовым} пространством.
\end{definition}

\begin{definition}
	$ (H, \|\cdot\|) $ "--- предгильбертово.

	$ H $ называется \emph{гильбретовым}, если $ H $ полно.
\end{definition}

\begin{props}
\item Неравенство Коши"--~Буняковского
	$$ |(x, y)| \le \|x\| \cdot \|y\| $$
\item $ \|x\| = \sqrt{(x, x)} $ "--- норма
\item Тождество параллелограмма
	$$ x, y \in H \quad \|x + y\|^2 + \|x - y\|^2 = 2 \bigl( \|x\|^2 + \|y\|^2 \bigr) $$
\item Непрерывность скалярного произведения
	$$
	\begin{rcases}
		\lim\limits_{n \to \infty} x_n = x \\
		\lim\limits_{n \to \infty} y_n = y
	\end{rcases} \implies \lim\limits_{n \to \infty}(x_n, y_n) = (x, y) $$
\end{props}

\begin{figure}[!h]
	\begin{tikzpicture}[>=Stealth]
		\draw[->] (0, 0) -- (1, 1) node[anchor=west] {$ y $};
		\draw[->] (0, 0) -- (2, 0) node[anchor=west] {$ x $};
		\draw (1, 1) -- (1, 0);
		\node[anchor=west] at (1, 0.5) {$ h $};
	\end{tikzpicture}
	\caption{К доказательству неравенства Коши"--~Буняковского}
	\label{fig:Cauchy-B}
\end{figure}

\begin{eproof}
\item См. \autoref{fig:Cauchy-B}

	$ \|h\|^2 \ge 0 \iff $ Коши"--~Буняковского.
\item
	\begin{enumerate}
		\item $ \|x\| = 0 \iff x = 0 $
		\item $ \|\lambda x\| = \sqrt{(\lambda x, \lambda x)} =
			\sqrt{\lambda \ol \lambda (x, x)} = |\lambda| \cdot \|x\| $
		\item
			\begin{multline*}
				\|x + y\|^2 = (x + y, x + y) = \|x\|^2 + \|y\|^2 + (x, y) + (y, x) =
				\|x\|^2 + \|y\|^2 + 2\operatorname{Re}(x, y) \le \\
				\le \|x\|^2 + \|y\|^2 + 2\|x\| \cdot \|y\| =
				\bigl( \|x\| + \|y\| \bigr)^2
			\end{multline*}
	\end{enumerate}
\item Упражнение
\item
	\begin{multline*}
		|(x, y) - (x_n, y_n)| \le |(x, y) - (x_n, y_n)| + |(x_n, y) - (x_n, y_n)| =
		|(x - x_n, y) - (x_n, y - y_n)| \underset{\text{К"--~Б}}\le \\
		\le \|x - x_n\| \cdot \|y\| + \|x_n\| \cdot \|y - y_n\| \underset{\lim x_n = x \implies \exists M > 0 : \|x_n\| \le M}\le
		\underbrace{\|x - x_n\|}_{\underarr{n \to \infty} 0} \cdot \|y\| + M \cdot \underbrace{\|y - y_n\|}_{\underarr{n \to \infty} 0}
	\end{multline*}
\end{eproof}

\begin{exmpls}
\item $ l_n^2 = (\Co^n, \|\cdot\|_2) $
	$$ (x, y) = \sum_{j = 1}^n x_j\ol y_j $$
	$$ \|x\|^2 = (x, x) = \sum |x_j|^2 $$
	$ l_n^2 $ "--- полное $ \implies l_n^2 $ "--- гильбертово.
\item $ l^2 = \set{x = \set{x_j}_{j = 1}^\infty | \ \|x\|_2 = \sqrt{\sum_{j = 1}^\infty |x_j|^2}} $
	$$ (x, y) = \sum_{j = 1}^\infty x_j \ol y_j $$
	$ l^2 $ "--- полное $ \implies l^2 $ "--- гильбертово (с нормой,
	порождённой скалярным произведением)
\item $ F $ "--- финитные последовательности
	$$ (x, y) = \sum_{j = 1}^\infty x_j \ol y_j $$
	(на самом деле, число слагаемых конечно)
	$$ \|x\|^2 = \sum |x_j|^2 $$
	$ F $ не полно $ \implies F $ "--- предгильбертово.
	$ l^2 $ "--- пополнение $ (F, \|\cdot\|_2) $ до гильбертова.
\item $ (T, \mathcal U, \mu), \quad L^2(T, \mu) $ "--- гильбертово пространство.
	$$ (f, g) = \int\limits_T f(x) \ol{g(x)} \di \mu $$
	$$ \|f\|^2 = (f, f) $$
\item $ \mathcal C_\Co[a, b], \quad f : [a, b] \to \Co $
	$$ (f, g) = \sum_a^b f(x) \ol{g(x)} \di x $$
	$$ \|f\| = \Bigl( \int_a^b |f(x)|^2 \di x \Bigr)^{\frac12} $$
	$ \bigl( \mathcal C[a, b], \|\cdot\|_2 \bigr) $ не полно $ \implies $ оно предгильбертово.
	Его пополнение до гильбертова "--- $ L^2([a, b], \lambda) $.
\item $ \mathscr P = \set{p(x) = \sum_{n = 0}^N c_nx^n} $
	\begin{itemize}
		\item $ \mathscr P \sub \mathcal C[a, b] $
			$$ (p, q) = \int_a^b p(x) \ol{q(x)} \di x $$
			Такое пространство предгильбертово.
			Пополнение "--- $ (L^2[a, b], \lambda) $.
		\item $ \mathscr P \sub F $
			$$ (p, q) = \sum_{n = 0}^N a_n\ol{b_n} $$
			Предгильбертово пространство. Пополнение "--- $ l^2 $.
	\end{itemize}
\item $ H^2 = \set{f(z) = \sum_{n = 0}^\infty a_nz^n | \set{a_n}_{n = 0}^\infty \in l^2} $ "---
	\emph{пространство Харди}
	$$ (f, g) = \sum_{n = 0}^\infty a_n\ol{b_n}, \quad \|f\|_{H^2} = \|\set{a_n}\|_2 =
	\Bigl( \sum_{n = 0}^\infty |a_n|^2 \Bigr)^{\frac12} $$
	$ H^2 $ "--- гильбертово.
	$$ \set{a_n}_{n = 0}^\infty \in l^2 = \lim\limits_{n \to \infty} a_n = 0 \implies
	\ulim\limits_{n \to \infty} \sqrt[n]{|a_n|} \le 1 $$
	Радиус сходимости $ R = \frac1{\ulim \sqrt[n]{|a_n|}} \ge 1 \implies
	f $ аналитична в $ \set{z | \ |z| < 1} $.
\end{exmpls}

\begin{definition}
	$ H $ "--- предгильбертово, $ \quad x, y \in H $

	Будем говорить, что $ x $ \emph{ортогонально} $ y $ ($ x \perp y $), если $ (x, y) = 0 $.
\end{definition}

\begin{definition}
	$ M \sub H $

	$$ M^\perp = \set{y \in H | x \perp y \quad \forall x \in M} $$
	Если $ M $ "--- подпространство, то $ M^\perp $ называется \emph{ортогональным
	дополнением}.
\end{definition}

\begin{statement}
	$ M \sub H $ (подмножество)

	$ m^\perp $ "--- замкнутое подпространство $ H $.
\end{statement}

\begin{iproof}
\item Подпространство

	Возьмём $ y, z \in M^\perp, \quad \alpha, \beta \in \Co $
	$$ \forall x \in M \quad
	\begin{rcases}
		(y, x) = 0 \\
		z(x) = 0
	\end{rcases} \implies (\alpha y + \beta z, x) = \alpha(y, x) + \beta(z, x) = 0 \implies
	\alpha y + \beta z \in M^\perp $$
\item Замкнутость

	Возьмём $ \set{y_n \in M^\perp}_{n = 1}^\infty ; \quad \lim\limits_{n \to \infty} y_n = y_0 $.
	Возьмём $ x \in M $.
	$$ (y_n, x) = 0 $$
	$$ \lim (y_n, x) = (y_0, x) \implies
	(y_0, x) = 0 \implies
	y_0 \in M^\perp $$
\end{iproof}

\section{Существование и единственность ближайшего приближения в подпространстве гильбертова
пространства. Теорема о проекции на подпространство.
Следствия}

\begin{lemma}
	$ M $ "--- замкнутое подпространство $ H, \quad
	u, v \in M, \quad x \in H \setminus M, \quad
	d = \rho(x, M) = \inf\limits_{z \in M} \|x - z\| $
	$$ \|u - v\|^2 \le 2 \bigl( \|u - x\|^2 + \|v - x\|^2 \bigr) - 4d^2 $$
\end{lemma}

\begin{proof}
	Воспользуемся тождеством параллелограмма для $ (u - x), ~ (v - x) $.
	$$ 2 \bigl( \|u - x\|^2 + \|v - x\|^2 \bigr) =
	\|u - v\|^2 + \|u + v - 2x\|^2 $$
	$$ \|u + v - 2x\| = 2 \Bigl\|x - \frac{u + v}2 \Bigr\| $$
	$$ u, v \in M \implies \frac{u + v}2 \in M  \implies
	\Bigl\| x - \frac{u + v}2 \Bigr\| \ge d \implies
	\|u + v - 2x\|^2 \ge 4d^2 $$
	$$ \|u - v\|^2 \le \bigl( \|u - x\|^2 + \|v - x\|^2 \bigr)^2 - 4d^2 $$
\end{proof}

\begin{theorem}
	$ H $ "--- гильбертово, $ \quad M $ "--- замкнутое подпространство, $ \quad
	x \in H $

	$$ \exists ! y \in M : \quad \rho(x, M) = \|x - y\|, $$
	\ie $ y $ "--- элемент наилучшего приближения для $ H $.
\end{theorem}

\begin{iproof}
\item Существование

	$ x \in H, \quad M, \quad d = \rho(x, M) $
	$$ \exists \set{y_n \in M}_{n = 1}^\infty : \quad \lim\limits_{n \to \infty} \|x - y_n\| = d $$
	Проверим, что $ \set{y_n} $ фундаментальна.
	Применим лемму.
	$$ \|y_n - y_m\|^2 \le 2 \bigl( \underbrace{\|y_n - x\|^2}_{\underarr{n \to \infty} d^2}
	+ \underbrace{\|y_m - x\|^2}_{\underarr{m \to \infty} d^2} \bigr)
	- 4d^2 \underarr{n, m \to \infty} 0 $$
	$ \implies \set{y_n} $ фундаментальна, $ M $ замкнуто $ \implies M $ полно $ \implies
	\exists y \in M : \quad \lim\limits_{n \to \infty} y_n = y \implies
	\underbrace{\lim\|x - y_n\|}_{d} = \|x - y\| \implies
	\|x - y\| = d $
\item Единственность

	Пусть $ \|x - y\| = d, \quad \|x - z\| = d $.
	По лемме
	$$ \|y - z\|^2 = 2 \bigl( \|y - x\|^2 + \|z - x\|^2 \bigr) - 4d^2 = 0 \implies y = z $$
\end{iproof}

\begin{theorem}
	$ H $ "--- гильбертово, $ \quad M \sub H, \quad M $ "--- замкнутое подпространство, $ \quad
	x \in H $

	$$ \exists ! y \in M, ~ z \in M^\perp : \quad x = y + z $$
\end{theorem}

\begin{iproof}
\item Существование

	$ d = \rho(x, M) $

	По предыдущей теореме
	$$ \exists! y \in M : \quad \|x - y\| = d = \inf\limits_{u \in M}\|x - u\| $$
	$$ z \define x - y $$

	Проверим, что $ z \perp M $.
	Возьмём $ u \ne 0 \in M $.
	$$ \set{tu}_{t \in \R} \sub M \implies
	y + tu \in M \quad \forall t \in \R \implies
	\|x - (y + tu)\|^2 \ge d^2 \iff
	\|z - tu\|^2 \ge d^2 $$

	Докажем, что это возможно только при $ z \perp u $:
	$$ (z - tu, z - tu) = \|z\|^2 + t^2\|u\|^2 - t \bigl( (u, z) + (z, u) \bigr) =
	\underbrace{\|z\|^2}_{d^2} + t^2 \|u\|^2 - 2t \operatorname{Re} (z, u) \ge d^2 $$
	$$ \implies t^2\|u\|^2 \ge 2t \operatorname{Re}(z, u) \quad \forall t \in \R $$
	\begin{itemize}
		\item $ t > 0 $
			$$ t\|u\|^2 \ge 2\operatorname{Re}(z, u) \quad \forall t > 0 \implies
			0 \ge \operatorname{Re}(z, u) $$
		\item $ t < 0 $
			$$ t\|u\|^2 \le 2\operatorname{Re}(z, u) \quad \forall t < 0 \implies
			0 \le \operatorname{Re}(z, u) $$
	\end{itemize}
	$$ \implies \operatorname{Re}(z, u) = 0 $$

	Аналогично для мнимой части получаем
	$$ \operatorname{Im}(z, u) = 0 $$
\item Единственность
	$$ x = y + z, ~ x = y_1 + z_1, \quad y, y_1 \in M, ~ z, z_1 \in M^\perp $$
	$$ u = y - y_1 \implies u = z_1 - z $$
	$$ \implies u \in M \text{ и } u \in M^\perp \text{ "--- \contra } \implies u = 0 $$
\end{iproof}

\begin{definition}
	$ H $ "--- гильбертово пространство, $ \quad X, Y $ "--- замкнутые подпространства

	Будем говорить, что $ H $ "--- \emph{ортогональная сумма} ($ H = X \oplus Y $), если
	\begin{enumerate}
		\item $ \forall h \in H \quad \exists x \in X, ~ y \in Y : \quad h = x + y $;
		\item $ \forall x \in X, ~ y \in Y \quad (x, y) = 0 $.
	\end{enumerate}
\end{definition}

\begin{statement}
	Если $ X \perp Y, \quad X, Y $ "--- подпространства, то $ X \cap Y = \set{0} $
\end{statement}

\begin{proof}
	Пусть $ u \in X \cap Y \implies u \in X, ~ u \in Y \implies
	u \perp u \implies u = 0 $.
\end{proof}

\begin{remark}
	Если $ H = X \oplus Y $, то
	$$ \forall h \in H \quad \exists!x \in X, ~ y \in Y : \quad h = x + y $$
\end{remark}

\begin{proof}
	Пусть $ h = x + y, ~ h = x_1 + y_1 $.
	$$ u = x - x_1 \implies
	u \in X, ~ u \in Y \implies u = 0 $$
\end{proof}

\begin{implication}
	$ H $ "--- гильбертово, $ \quad M $ "--- замкнутое подпространство $ H $

	$$ H = M \oplus M^\perp $$
\end{implication}

\begin{implication}
	$ (M^\perp)^\perp = M $
\end{implication}

\begin{implication}
	Если $ H = X \oplus Y, \quad X, Y $ "--- замкнутые, то $ Y = X^\perp $.
\end{implication}

\begin{definition}
	$ H $ "--- гильбертово, $ \quad M $ "--- замкнутое подпространство

	$$ \forall h \in H \quad \exists !x \in M, ~ y \in M^\perp : \quad h = x + y $$
	Определим \emph{оператор ортогонального проектирования}:
	$$ \underset M P h \define x $$
	($ x $ "--- элемент наилучшего приближения для $ h $ в $ M $).
	$$ \underset{M^\perp}P h = y $$
\end{definition}

\section{Критерий принадлежности множеству ортогональных проекторов}

\TODO{Критерий принадлежности множеству ортогональных проекторов}

\section{Проектор на конечномерное пространство.
Критерий полноты семейства элементов.
Неравенство Бесселя}

\TODO{Проектор на конечномерное пространство}
