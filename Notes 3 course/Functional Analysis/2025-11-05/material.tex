\chapter{Линейные функционалы}

\section{Линейные функционалы в гильбертовом пространстве}

\subsection{Представление линейного непрерывного функционала в гильбертовом пространстве}

\begin{theorem}[Рисс]
	$ H $ "--- гильбертово пространство

	\begin{enumerate}
		\item $ y \in H $ "--- фиксирован, $ \quad f_y : H \to \Co : \quad f_y(x) = (x, y) $
			$$ f_y \in H^*, \quad \|f_y\|_{H^*} = \|y\|_H $$
		\item $ f \in H^* $
			$$ \exists ! y \in H : \quad f = f_y $$
	\end{enumerate}
\end{theorem}

\begin{eproof}
\item $ y \in H $
	\begin{itemize}
		\item Проверим, что $ f_y \in \mathscr Lin(H, \Co) $

			$ \alpha \in \Co, \quad x, y \in H $
			$$ f_y(\alpha x + z) = (\alpha x + z, y) =
			\alpha (x, y) + (z, y) = \alpha f(x) + f(z) $$
		\item Оценим $ \|f_y\| $
			$$ \|f_y\| = |(x, y)| \underset{\text{К"--~Б}}\le \|y\| \cdot \|x\| \implies
			\|f_y\| \le \|y\| \implies f \in H^* $$

			\begin{itemize}
				\item Если $ y = 0 $, то $ f_y(x) = 0 \implies f_y = \On[], \quad \|f_y\| = 0 $
				\item $ y \ne 0 $
					$$ \|f_y\| = \sup\limits_{x \ne 0} \frac{|f_y(x)|}{\|x\|} \ge
					\frac{|f_y(y)|}{\|y\|} = \frac{(y, y)}{\|y\|} = \|y\| $$
			\end{itemize}
	\end{itemize}
\item $ f \in H^* $
	\begin{itemize}
		\item Если $ f = \On[] $, \ie $ f(x) \equiv 0 $, то $ f = f_0 $
		\item $ f \ne \On[] $

			Рассмотрим $ N = \operatorname{Ker} f \ne H $
			$$ H = N \oplus N^\perp, \quad N^\perp \ne \set{0} $$
			Возьмём $ y_0 \in N^\perp $.
			Докажем, что $ \dim N^\perp = 1 $.

			$$ f(y_0) \ne 0 $$
			$$ v \coloneq \frac{y_0}{f(y_0)} \implies f(v) = 1 $$

			Возьмём $ z \in N^\perp $.
			Докажем, что $ z $ пропорционально $ v $.
			$$ u \coloneq z - f(z)v \implies u \in N^\perp $$
			$$ f(u) = f(z) - f(z)\underbrace{f(v)}_1 = 0 \implies u \in N $$
			Значит, $ u = 0 \implies z = f(z)v $.

			Будем искать $ y $ в виде $ y = \alpha v $, где $ \alpha \in \Co $.
			$$ 1 = f_y(v) = f_y(v) = (v, \alpha v) = \alpha \|v\|^2 \implies
			\alpha = \frac1{\|v\|^2} \implies y = \frac{v}{\|v\|^2} $$
		\item Проверим единственность

			Пусть $ f = f_y, ~ f = f_z $.
			$$ \forall x \in H \quad f_y(x) = f_z(x) \implies (x, y) = (x, z) \implies
			(x, y - z) = 0 \implies y - z \in H^\perp = \set{0} $$
	\end{itemize}
\end{eproof}

\begin{remark}
	$ C : H \to H^* : \quad C(y) = f_y $

	Знаем, что $ \|f_y\|_{H^*} = \|y\|_H $.

	$$ C(y + z) = f_{y + z} $$
	$$ \forall x \in H \quad f_{y + z}(x) = (x, y + z) = (x, y) + (x, z) = f_y + f_z $$
	$$ C(y + z) = C(y) + C(z) $$

	Возьмём $ \alpha \in \Co $.
	$$ C(\alpha y) = f_{\alpha y} $$
	$$ f_{\alpha y}(h) = (x, \alpha y) = \ol\alpha (x, y) $$
	$ C $ "--- сопряжённо-линейный изометрический изоморфизм между $ H $ и $ H^* $.
	Говорят, что $ H^* = H, $ при этом имеют в виду, что $ C(H) = H^* $.
\end{remark}

\begin{exmpls}
\item $ l^2 $
	$$ f \in (l^2)^* \iff \exists! y = \set{y_n}_{n = 1}^\infty \in l^2 : \quad
	\forall x \in l^2 \quad f(x) = f_y(x) = \sum_{n = 1}^\infty x_n \ol y_n $$
\item $ (T, \mathcal U, \mu), \quad \mathrm L^2(\mu) $
	$$ \Phi \in \bigl( \mathrm L^2 (T, \mu) \bigr)^* \iff \exists ! g \in \mathrm L^2(\mu) : \quad
	\forall f \in \mathrm L^2(\mu) \quad \Phi(f) = \int\limits_T f(x) \ol{g(x)} \di \mu $$
\end{exmpls}

\section{Геометрический смысл линейного функционала}

\begin{theorem}
	$ X $ "--- линейное пространство над $ \mathbb K $
	\begin{enumerate}
		\item $ f \in \mathscr Lin(X, \mathbb K), \quad
			f \ne \On[], \quad L = \operatorname{Ker} f $
			$$ \operatorname{codim} L = \dim \bigl( \faktor X L \bigr) = 1 $$
			Это называется \emph{коразмерность} $ L $.
		\item $ L \sub X, \quad \operatorname{codim} L = 1, \quad x_0 \in X \setminus L $
			$$ \exists ! f \in \mathscr Lin(X, \mathbb K) : \quad L = \operatorname{Ker} f, \quad
			f(x_0) = 1 $$
	\end{enumerate}
\end{theorem}

\begin{eproof}
\item $ f \ne \On[] \implies \exists y_0 \notin L $
	$$ v \coloneq \frac{y_0}{f(y_0)} \implies f(v) = 1 $$

	Возьмём $ x \in X $.
	\begin{multline*}
		u \coloneq x - f(x)v \implies f(u) = f(x) - f(x)f(v) = 0 \implies u \in L \implies
		\ol u = \ol 0 \implies \ol 0 = \ol x - f(x) \ol v \implies \\
		\implies \ol x = f(x) \ol v \implies \ol x = f(x) \ol v \implies
		\faktor X L = \set{\alpha \ol v | \alpha \in \mathbb K} \implies
		\dim \bigl( \faktor X L \bigr) = 1
	\end{multline*}
\item $ L : \quad \dim \bigl( \faktor X L \bigr) = 1 $

	Возьмём $ x \in X $.
	$$ \exists \alpha \in \mathbb K : \ol x = \alpha \cdot \ol x_0 $$
	Определим $ f(x) = \alpha $.
	$$ \ol x_0 = 1 \cdot \ol x_0 \implies f(x_0) = 1 $$

	Проверим, что $ f \in \mathscr Lin(X, \mathbb K) $.
	Возьмём $ \alpha \in \mathbb K, \quad x, y \in X $.
	$$
	\begin{rcases}
		\ol x = \beta \ol v \\
		\ol y = \gamma \ol v
	\end{rcases} \implies \alpha \ol x + \ol y = (\alpha \beta + \gamma) \ol v \implies
	f(\alpha x + y) = \alpha \beta + \gamma = \alpha f(x) + f(y) $$
	$$ f(x) = 0 \iff \ol x = 0 \cdot \ol v = \ol 0 \iff x \in L \implies \operatorname{Ker} f = L $$

	Проверим единственность.
	Пусть $ \exists g \in \mathscr Lin(X, \mathbb K), \quad \operatorname{Ker} g = L, \quad
	g(x_0) = 1 $.
	$$ \forall x \in X \quad x = \alpha \cdot \ol x_0 \implies
	x = \alpha x_0 + y, \quad y \in L \implies
	\begin{cases}
		f(x) = \alpha \\
		g(x) = \alpha
	\end{cases} $$
\end{eproof}

\begin{remark}
	В условиях второго пункта $ L = \operatorname{Ker} f, \quad f(x_0) = 1 $
	$$ f^{-1}(1) = x_0 + L = \set{x_0 + y | y \in L} $$
\end{remark}

\begin{proof}
	Обозначим $ M = f^{-1}(1) $.
	\begin{itemize}
		\item $ z \in x_0 + L $, \ie $ z = x_0 + y, \quad y \in L $
			$$ f(z) = \underbrace{f(x_0)}_1 + \underbrace{f(y)}_0 = 1 \implies z \in M \implies
			x_0 + L \sub M $$
		\item $ z \in M $
			$$ f(z) = 1 \implies f(z - x_0) = 0 \implies z - x_0 \in L \implies z - x_0 = y \implies
			z = x_0 + y \in x_0 + L \implies M \sub x_0 + L $$
	\end{itemize}
\end{proof}

\begin{theorem}[норма линейного функционала]
	$ (X, \|\cdot\|), \quad f \in X^*, \quad f \ne \On[], \quad f(x_0) = 1, \quad
	L = \operatorname{Ker} f $
	$$ \| f \| = \frac1{\rho(x_0, L)} $$
\end{theorem}

\begin{proof}
	Обозначим $ d = \rho(x_0, L) = \inf \|x_0 - y\| $.
	\begin{itemize}
		\item
			$$ 1 = f(x_0) = f(x_0 - y) \le \|f\| \cdot \|x_0 - y\| \quad \forall y \in L \implies
			1 \le \|f\| \inf \|x_0 - y\| = \|f\|d \implies \frac1d \le \|f\| $$
		\item $ x \in X \setminus L, \quad f(x) \ne 0 $
			\begin{multline*}
				f \Bigl( \frac{x}{f(x)} \Bigr) = 1 \implies
				f \Bigl( x_0 - \frac{x}{f(x)} \Bigr) \implies x_0 - \frac{x}{f(x)} \in L \implies
				\Bigl\| x_0 - \Bigl( x_0 - \frac{x}{f(x)} \Bigr) \Bigr\| \ge d \iff \\
				\iff \frac{\|x\|}{\|f(x)\|} \ge d \implies |f(x)| \le \frac1d \|x\| \implies
				\|f\| \le \frac1d
			\end{multline*}
	\end{itemize}
\end{proof}

\begin{remark}
	В обозначениях теоремы $ M = f^{-1}(1) $
	$$ \rho(x_0, L) = \rho(0, M) $$
\end{remark}

\begin{proof}
	$$ \rho(x_0, L) = \inf \|x_0 - y\| $$
	$$ \rho(0, M) = \rho(0, x_0 + L) = \inf \|x_0 + y\| = \inf \|x_0 - y\| = \rho(x_0, L) $$
\end{proof}

\section{Продолжение линейных функционалов}

\subsection{Небольшое отступление: трансфинитная индукция}

\begin{definition}
	$ \mathscr P $ "--- \emph{частично упорядоченное множество}, если для некоторых элементов
	$ a, b \in \mathscr P $ выполняется $ a \le b $, \ie выделено
	$$ R \sub \mathscr P \times \mathscr P : \quad (a, b) \in R \text{ если } a \le b $$
	и выполнены аксиомы:
	\begin{enumerate}
		\item Рефлексивность: $ \forall a \in \mathscr P \quad a \le a $.
		\item Транзитивность:
			$$
			\begin{rcases}
				a \le b \\
				b \le c
			\end{rcases} \implies a \le c $$
		\item Антисимметричность:
			$$
			\begin{rcases}
				a \le b \\
				b \le a
			\end{rcases} \implies a = b $$
	\end{enumerate}
\end{definition}

\begin{definition}
	$ A \sub \mathscr P $

	$ y $ "--- \emph{верхняя грань} для $ A $, если $ \forall a \in A \quad a \le y $.
\end{definition}
