\chapter{Вероятностное пространство}

\section{Формулы полной вероятности}

\begin{theorem}
	Пусть $ A_1, A_2, \dots $ "--- полная (конечная или бесконечная) группа несовместных событий и для $ P(A_i) > 0 $.

	Тогда для любого случайного события $ B \in F $ верно
	$$ P(B) = \sum_i P(B|A_i) \cdot P(A_i) $$
\end{theorem}

\subsection{Формула Байеса}

\begin{theorem}
	Пусть $ A_1, A_2, \dots $ "--- полная группа несовместных событий, имеющих ненулевые вероятности.
	Рассмотрим произвольное событие $ B $ с ненулевой вероятностью.

	Тогда для любого события $ A_i $ справедливо
	$$ P(A_i|B) = \frac{P(B|A_i) \cdot P(A_i)}{\sum_j P(B|A_j) \cdot P(A_j)} $$
\end{theorem}

\section{Независимость событий}

\begin{definition}
	Пусть $ P(B) \ne 0 $.

	События $ A $ и $ B $ называются \emph{независимыми}, если $ P(A|B) = P(A) $.
\end{definition}

\begin{definition}
	События $ A $ и $ B $ будем называть \emph{независимыми}, если при $ P(A)P(B) \ne 0 $ выполнены равенства
	$$ P(A|B) = P(A), \quad P(B|A) = P(B) $$
\end{definition}

\begin{definition}
	События $ A $ и $ B $ \emph{независимы}, если $ P(AB) = P(A)P(B) $.
\end{definition}
