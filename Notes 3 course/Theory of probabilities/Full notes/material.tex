\chapter{Вероятностное пространство}

\section{Основные определения}

\begin{definition}
	Результаты опытов или наблюдений будем называть \emph{событиями}.
\end{definition}

\begin{notation}
	$ \omega $ "--- исход, элементарное событие.
\end{notation}

\begin{notation}
	$ \Omega $ "--- множество всех элементарных событий.
\end{notation}

\begin{definition}
	Если $ A \cap B = \O $, то говорим, что $ A $ и $ B $ \emph{несовместны}.
\end{definition}

\begin{definition}
	Если $ A \sub B $, то говорят, что появление события $ A $ \emph{влечёт} появление события $ B $.
\end{definition}

\begin{notation}
	$ \sigma $-алгебру событий будем обозначать $ \Im $.
\end{notation}

\begin{definition}
	Пусть $ P : \Im $ удовлетворяет следующим условиям:
	\begin{enumerate}
		\item $ P(A) \ge 0 \quad \forall A \in \Im $;
		\item $ P(\Omega) = 1 $;
		\item $ A_1, A_2, \dotsc $ "--- не более чем счётный набор событий,
			$ A_i \cap A_j = \O $, то
			$$ P \Bigl( \bigcup_{i = 1}^\infty A_i \Bigr) = \sum_{i = 1}^\infty P(A_i) $$
	\end{enumerate}

	Будем говорить, что $ P $ является \emph{вероятностной (счётно-аддитивной) мерой} или
	\emph{вероятностью}.
\end{definition}

\begin{definition}
	Говорят, что $ A_1, A_2, \dotsc $ "--- \emph{полная группа несовместных событий},
	если их объединение представляет собой достоверное событие.
\end{definition}

\section{Свойства вероятностей}

\begin{props}
\item $ \ol A $ "--- событие, дополнительное к событию $ A $.
	Тогда $ P(\ol A) = 1 - P(A) $.
\item $ 0 \le P(A) \le 1 $.
\item $ P(\O) = 0 $.
\item $ P(A \cup B) = P(A) + P(B) - P(AB) $
	$$ P \Bigl( \bigcup_{i = 1}^n A_i \Bigr) =
	\sum_{i = 1}^n P(A_i) - \sum_{i < j} P(A_iA_j) + \sum_{i < j < k}P(A_iA_jA_k) -
	\dots + (-1)^{n + 1}P \Bigl( \bigcap_{i = 1}^n A_i \Bigr) $$
\item $ P(\bigcup A_i) \le \sum P(A_i) $.
\item $ A \sub B \implies P(A) \le P(B) $.
\item $ A_1 \sub A_2 \sub \dotso, \quad A = \bigcup A_n $
	$$ P(A) = \lim\limits_{n \to \infty} P(A_n) $$
\item $ B_1 \supset B_2 \supset \dotso, \quad B = \bigcap B_n $
	$$ P(B) = \lim\limits_{n \to \infty} P(B_n) $$
\item $ A_1, A_2, \dotsc $ "--- не более чем счётная полная группа событий
	$$ \sum P(A_n) = 1 $$
\end{props}

\begin{statement}[неравенства Бонферрони]
	$$ \sum_{k = 1}^n P(A_k) - \sum{1 \le k \le l \le n} P(A_kA_l) \le
	P \Bigl( \bigcup_{k = 1}^n A_k \Bigr) \le \sum_{k = 1}^n P(A_k) $$
\end{statement}

\section{Условные вероятности}

\begin{definition}
	$ A, B \in \Im, \quad P(B) > 0 $

	\emph{Условной вероятностью} $ P(A|B) $ называется отношение
	$$ P(A|B) = \frac{P(AB)}{P(B)} $$
\end{definition}

\begin{statement}
	Условная вероятность является вероятностью.
\end{statement}

\begin{eproof}
\item
	$$ P(A|B) = \frac{P(AB)}{P(B)} \ge 0 $$
\item
	$$ P(\Omega|B) = \frac{P(\Omega B)}{P(B)} = \frac{P(B)}{P(B)} = 1 $$
\item
	$$ P \Bigl( \bigcup_k A_k \Bigm| B \Bigr) =
	\frac{P \Bigl\lgroup B \cap \Bigl( \bigcup_k A_k \Bigr) \Bigr\rgroup}{P(B)} =
	\frac{P \Bigl( \bigcup_k A_kB \Bigr)}{P(B)} =
	\frac{\sum_k P(A_kB)}{P(B)} = \sum_k P(A_k|B) $$
\end{eproof}

\section{Формула полной вероятности}

\begin{theorem}
	$ A_1, A_2, \dotsc $ "--- полная группа несовместных событий, $ \quad P(A_i) > 0 $

	$$ P(B) = \sum_i P(B|A_i) \cdot P(A_i) $$
\end{theorem}

\begin{proof}
	Учитывая несовместность событий $ BA_1, BA_2, \dotsc $, получаем
	\begin{multline*}
		P(B) = P(B \cap \Omega) =
		p \Bigl\lgroup B \cap \Bigl( \bigcup_{i \ge 1}A_i \Bigr) \Bigr\rgroup =
		P \Bigl( \bigcup_{i \ge 1} BA_i \Bigr) = \sum_{i \ge 1} P(BA_i) = \\
		= \sum_{i \ge 1} \frac{P(BA_i)P(A_i)}{P(A_i)} = \sum_{i \ge 1} P(B|A_i) P(A_i)
	\end{multline*}
\end{proof}

\begin{theorem}[формула Байеса]
	$ A_1, A_2, \dotsc $ "--- полная группа несовместных событий \\
	$ P(A_i) > 0, \quad P(B) > 0 $

	$$ P(A_i|B) = \frac{P(B|A_i) \cdot P(A_i)}{\sum_j P(B|A_j) \cdot P(A_j)} $$
\end{theorem}

\begin{proof}
	$$ \frac{P(B|A_i)P(A_i)}{\sum_j P(B|A_j)P(A_j)} = \frac{P(B|A_i)P(A_i)}{P(B)} =
	\frac{P(A_iB)P(A_i)}{P(A_i)P(B)} = P(A_i|B) $$
\end{proof}

\section{Независимость случайных событий}

\begin{definition}
	$ P(B) \ne 0 $

	События $ A $ и $ B $ будем называть \emph{независимыми}, если $ P(A|B) = P(A) $.
\end{definition}

\begin{definition}
	События $ A $ и $ B $ будем называть \emph{независимыми}, если при $ P(A)P(B) \ne 0 $ выполнены
	равенства
	$$ P(A|B) = P(A), \quad P(B|A) = P(B) $$
\end{definition}

\begin{definition}
	События $ A $ и $ B $ называются \emph{независимыми}, если $ P(AB) = P(A)P(B) $.
\end{definition}

\begin{definition}
	События $ A_1, A_2, \dotsc $ \emph{попарно независимы}, если для любых двух событий $ A_i $ и
	$ A_j $ выполняется соотношение
	$$ P(A_iA_j) = P(A_i)P(A_j) $$
\end{definition}

\begin{definition}
	Говорят, что события $ A_1, A_2, \dots, A_n $ \emph{независимы в совокупности (взаимно
	независимы)}, если
	$$ \forall k \le n \quad \forall 1 \le i_1 < i_2 < \dots < i_k \le n \quad
	P \Bigl( \bigcap_{j = 1}^k A_{i_j} \Bigr) = \prod_{j = 1}^k P(A_{i_j}) $$
\end{definition}

\begin{theorem}
	$ A_1, A_2, \dotsc $ взаимно независимы.

	Тогда взаимно независимы события $ B_1, B_2, \dotsc $, где $ B_i $ "--- это $ A_i $ или его
	дополнение.
\end{theorem}

\section{Схемы испытаний Бернулли}

\begin{definition}
	Повторные независимые испытания называются \emph{испытаниями Бернулли}, если каждое такое
	испытание имеет только два возможных исхода и вероятности соответствующих исходов остаются
	неизменными для всех испытаний.
\end{definition}

\begin{statement}
	Вероятность из $ n $ испытаний получить ровно $ m $ успехов равна
	$$ P_n(m) = C_n^m p^mq^{n - m} $$
\end{statement}

\begin{theorem}[локальная предельная]
	$ 0 < p < 1 $

	Соотношение
	$$ \frac{\sqrt{npq}P_n(m)}{\phi \Bigl( \dfrac{m - np}{\sqrt{npq}} \Bigr)}
	\underarr{n \to \infty} 1, $$
	где
	$$ \phi(x) = \frac1{\sqrt{2\pi}} e^{-\frac{x^2}2}
	\text{ "--- плотность нормального распределения}, $$
	выполняется равномерно для значений $ m $ таких, что
	$$ -\infty < C_1 \le \frac{m - np}{\sqrt{npq}} \le C_2 < +\infty $$
\end{theorem}

\begin{notation}
	$ \mu_n $ "--- количество успехов в испытаниях Бернулли.
\end{notation}

\begin{notation}
	Определим \emph{функцию распределения нормального закона}:
	$$ \Phi(x) = \frac1{\sqrt{2\pi}} \int_{-\infty}^x e^{-\frac{t^2}2} \di t =
	\int_{-\infty}^x \phi(t) \di t $$
\end{notation}

\begin{theorem}[интегральная предельная]
	$ 0 < p < 1 $

	Равномерно по всем значениям $ -\infty \le a < b \le +\infty $ выполняется соотношение
	$$ P\set{a \le \mu_n \le b} \underarr{n \to \infty}
	\Phi \Bigl( \frac{b - np}{\sqrt{npq}} \Bigr) - \Phi \Bigl( \frac{a - np}{\sqrt{npq}} \Bigr) $$
\end{theorem}

Погрешность оценивается неравенством
$$ \Delta_n \le \frac c{\sqrt{npq}} $$
$ c $ "--- абсолютная константа.
Можно взять $ c = 1 $ или $ c = 0.7655 $.

\begin{theorem}
	При большом $ n $ рассмотрим такую схему серий независимых событий: \\
	\begin{tabular}{l | c}
		& p \\
		\hline
		$ A_{11} $ & \lambda \\
		$ A_{21} $ $ A_{22} $ & \lambda/2 \\
		$ A_{31} $ $ A_{32} $ $ A_{33} $ & \lambda/3 \\
		\vdots & \vdots \\
		$ A_{n1} $ $ A_{n2} $ \dots $ A_{nn} $ & \lambda/n
	\end{tabular}

	Для каждого фиксированного $ m $ справедливо
	$$ P_n(m) \underarr{n \to \infty} \frac{e^{-\lambda}\lambda^m}{m!} $$
\end{theorem}

\section{Полиномиальная схема}

\begin{definition}
	Пусть $ P\Set{A_i} = p_i, \quad p_1 + \dots + p_r = 1 $.

	Вероятность того, что в $ n $ испытаниях событие $ A_i $ произойдёт $ m_i $ раз, равна:
	$$ P_n(m_1, m_2, \dots, m_r) =
	\frac{n!}{m_! m_2! \cdots m_r!} p_1^{m_1} p_2^{m_2} \cdots p_r^{m_r}, \quad
	m_1 + m_2 + \dots + m_r = n $$
\end{definition}

\chapter{Случайные величины}

\section{Случайные величины и их распределения}

\begin{definition}
	$ (\Omega, \Im, P) $

	Говорим, что $ \xi : \Omega \to \R $ "--- \emph{случайная величина}, если $ \xi $ измерима.
\end{definition}

\begin{definition}
	\emph{Функцией распределения} случайной величины $ \xi $ будем называть функцию
	$$ F_\xi(x) = P_\xi \bigl( (-\infty, x) \bigr) = P \Set{\xi < x} $$
\end{definition}

\begin{remark}
	$ P_\xi \bigl( [x, y) \bigr) = P \set{x \le \xi \le y} = F_\xi(y) - F_\xi(x) $
\end{remark}

\begin{props}
\item $ 0 \le F_\xi(x) \le 1 $;
\item $ F_\xi $ "--- неубывающая функция;
\item $ F_\xi $ непрерывна слева во всех точках;
\item $ F_\xi(x) \underarr{x \to -\infty} 0, \quad F_\xi(x) \underarr{x \to +\infty} 1 $.
\end{props}

\begin{definition}
	Если для случайных величин функции распределения совпадают, то такие
	случайные величины называют \emph{одинаково распределёнными}.
\end{definition}

\subsection{Типы вероятностных распределений}

\subsubsection{Дискретный закон распределения}

\begin{definition}
	Говорим, что $ \xi $ "--- случайная величина с \emph{дискретным законом распределения}, если
	существует не более чем счётное множество точек $ A \sub \R $ такое, что $ P(\xi \in A) = 1 $.
\end{definition}

\begin{exmpls}
\item $ \xi $ "--- случайная величина с \emph{вырожденным распределением} в точке $ c $.
	$$ F_\xi(x) =
	\begin{cases}
		0, \quad x \le c, \\
		1, \quad x > c.
	\end{cases} $$
\item $ \xi $ "--- случайная величина с \emph{распределением Бернулли}.

	$ \xi $ принимает значения 0 и 1 с вероятностями $ 1 - p $ и $ p $.
	$$ F_\xi(x) =
	\begin{cases}
		0, \quad x \le 0, \\
		1 - p, \quad 0 < x \le 1, \\
		1, \quad x > 1.
	\end{cases} $$
\item $ \xi $ имеет \emph{биномиальное распределение} с параметрами $ n, p $.

	Будем использовать обозначение $ \xi \in B(n, p) $.
	$ \xi $ принимает значения 0, 1, \dots, n с вероятностями
	$$ P(\xi = k) = C_n^k p^k (1 - p)^{n - k} $$
\item $ \xi $ распределена по \emph{закону Пуассона} с параметром $ \lambda > 0 $.

	Обозначение: $ \xi \in \Pi(\lambda) $.
	$ \xi $ принимает целые неотрицательные значения с вероятностями
	$$ P_k = \frac{e^{-\lambda}\lambda^k}{k!} $$
\end{exmpls}

\subsubsection{Абсолютно непрерывное распределение}

\begin{definition}
	Говорим, что $ \xi $ "--- случайная величина с \emph{абсолютно непрерывным распределением}, если
	существует функция $ p_\xi(x) $ такая, что для любого борелевского множества
	$ B \in \mathcal B $
	$$ P( \xi \in B) = \int\limits_B p_\xi(x) \di x $$

	Функцию $ p_\xi $ будем называть \emph{плотностью распределения} случайной величины $ \xi $.
\end{definition}

\begin{remark}
	$ p_\xi(x) = F_\xi'(x) $ почти всюду.
\end{remark}

\begin{props}
\item $ p_\xi(x) \ge $ почти всюду;
\item
	$$ \int_{-\infty}^\infty p_\xi(x) \di x = 1 $$
\end{props}

\begin{exmpls}
\item $ \xi $ имеет \emph{равномерное распределение} на отрезке $ [a, b] $ ($ \xi \in U([a, b]) $),
	если существует плотность распределения
	$$ p_\xi(x) =
	\begin{cases}
		0, \quad x \notin [a, b], \\
		\frac1{b - a}, \quad x \in [a, b].
	\end{cases} $$
	$$ F_\xi(x) = \frac{x - a}{b - a} \quad \text{ при } a \le x \le b $$
\item $ \xi $ имеет \emph{экспоненциальное распределение} с параметром $ \lambda > 0 $, если
	$$ p_\xi(x) =
	\begin{cases}
		0, \quad x \le 0, \\
		\frac{e^{-\frac x \lambda}}\lambda, \quad x > 0.
	\end{cases} $$
	$$ F_\xi(x) =
	\begin{cases}
		0, \quad x \le 0, \\
		1 - e^{-\frac x \lambda}, \quad x > 0.
	\end{cases} $$
\item $ \xi $ имеет \emph{нормальное (гауссово)} распределение с параметрами $ a, \sigma^2 $ ($ \xi
	\in N(a, \sigma^2) $), если
	$$ p_\xi(x) = \frac1{\sqrt{2\pi \sigma^2}}e^{-\frac{(x - a)^2}{2\sigma^2}} $$

	Если $ a = 0, ~ \sigma^2 = 1 $, то говорят, что $ \xi $ имеет \emph{стандартное нормальное
	распределение}.
	При этом, обычно для плотности распределения и функции распределения используются обозначения
	$ \phi(x) $ и $ \Phi(x) $.
\item $ \xi $ имеет \emph{распределение Коши}, если
	$$ p_\xi(x) = \frac1{\pi(1 + x^2)} $$
	$$ F_\xi(x) = \frac12 + \frac{\arctg x}\pi $$
\end{exmpls}

\subsubsection{Сингулярная случайная величина}

\begin{definition}
	Говорим, что $ x $ "--- \emph{точка роста} функции распределения $ F(x) $, если
	$$ \forall \eps > 0 \quad F(x + \eps) > F(x - \eps) $$
\end{definition}

\begin{definition}
	$ \xi $ "--- случайная величина с \emph{сингулярным распределением}, если множество точек роста
	$ F_\xi(x) $ имеет меру 0.
\end{definition}

\subsubsection{Семейство вероятностных распределений}

\begin{definition}
	$ \xi $ "--- случайная величина

	К \emph{семейству}, включающему эту величину, относим все величины $ \eta $, получаемые линейным
	преобразованием
	$$ \eta = a + b \xi, $$
	в котором $ a $ "--- \emph{параметр сдвига}, $ b > 0 $ "--- \emph{параметр масштаба}.

	$$ F_\eta(x) = F_\xi \Bigl( \frac{x - a}b \Bigr) $$
	$$ p_\eta(x) = \frac{p_\xi \bigl( \frac{x - a}b \bigr)}b $$
\end{definition}

\section{Случайные векторы}

\begin{definition}
	$ \xi : \Omega \to \R^d $ "--- \emph{случайный вектор}, если $ \xi $ измерима, \ie
	$$ \forall B \in \mathcal B^d \quad \xi^{-1}(B) \in F $$
\end{definition}

\begin{definition}
	\emph{Функция распределения} $ F_\xi $ $ d $-мерного случайного вектора задаётся как
	$$ F_\xi(x_1, \dots, x_d) = P\Set{\xi_1 < x_1, \dots, \xi_d < x_d} $$
\end{definition}

\begin{props}
\item $ 0 \le F_\xi(x_1, \dots, x_d) $;
\item $ F_\xi $ непрерывна слева по каждой из координат;
\item $ F_\xi(x_1, \dots, x_d) \to 0 $ если хотя бы одна из координат $ x_k \to -\infty $;
\item $ F_\xi(x_1, \dots, x_k, \dots, x_d) \to
	F_{(\xi_1, \dots, x_{k - 1}, \xi_{k + 1}, \dots, \xi_d)}
	(x_1, \dots, x_{k - 1}, x_{k + 1}, \dots, x_d) $ если $ x_k \to +\infty $;
\item $ F_\xi(x_1, \dots, x_d) \to 1 $ если все координаты $ x_i \to +\infty $;
\item $ d = 2 $
	$$ P \bigl( (\xi_1, \xi_2) \in [a_1, b_1) \times [a_2, b_2) \bigr) =
	F_\xi(b_1, b_2) - F_\xi(a_1, b_2) - F_\xi(b_1, a_2) + F_\xi(a_1, a_2) $$
\end{props}

\subsection{Случайные векторы с дискретным распределением}

\begin{definition}
	Случайный вектор $ \xi $ имеет $ d $-мерное \emph{дискретное распределение}, если существует не
	более чем счётное множество точек $ A \sub \R^d $ такое, что $ P(\xi \in A) = 1 $.
\end{definition}

\begin{statement}
	Следующие утверждения равносильны:
	\begin{enumerate}
		\item $ \xi = (\xi_1, \dots, \xi_d) $ "--- случайный вектор с дискретным распределением;
		\item величины $ \xi_k $ имеют дискретное распределение для всех $ k = 1, 2, \dots, d $.
	\end{enumerate}
\end{statement}

\subsection{Случайные векторы с абсолютно непрерывным законом распределения}

\begin{definition}
	Случайный вектор $ \xi $ имеет $ d $-мерное \emph{абсолютно непрерывное распределение}, если
	существует функция $ p_\xi(x_1, \dots, x_d) $ такая, что
	$$ \forall y_1, \dots, y_d \quad F_\xi(y_1, \dots, y_d) = \int_{-\infty}^{y_1} \dots
	\int_{-\infty}^{y_d} p_\xi(x_1, \dots, x_d) \di x_1 \dots \di x_d $$

	Функция $ p_\xi(x_1, \dots, x_d) $ называется \emph{плотностью распределения} случайного
	вектора.
\end{definition}

\begin{props}
\item $ p_\xi(x_1, \dots, x_d) \ge 0 $;
\item
	$$ \iint\limits_{\R^d} p_\xi(x_1, \dots, x_d) \di x_1 \dots \di x_d = 1 $$
\end{props}

\begin{statement}
	Каждая компонента $ \xi_k $ случайного вектора с абсолютно непрерывным распределением
	представляет собой случайную величину с абсолютно непрерывным распределением.
\end{statement}

\section{Независимые случайные величины}

\begin{definition}
	Говорим, что $ \xi_1, \xi_2, \dots, \xi_n $ "--- \emph{независимые случайные величины}, если
	$$ \forall x_1, x_2, \dots, x_n \in \mathbb R \quad F_{(\xi_1, \dots, \xi_n)}(x_1, \dots, x_n) =
	\prod_{i = 1}^n F_{\xi_i}(x_i), $$
	то есть
	$$ P(\xi_1 < x_1, \dots, \xi_n < x_n) = \prod_{i = 1}^n P(\xi_i < x_i) $$
\end{definition}

\begin{definition}
	Говорим, что $ \xi_1, \xi_2, \dots, \xi_n $ "--- \emph{независимые случайные величины}, если для
	любых борелевских множеств $ B_1, \dots, B_n \in \mathcal B $ справедливо
	$$ P(\xi_1 \in B_1, \dots, \xi_n \in B_n) = \prod_{i = 1}^n P(B_i) $$
\end{definition}

\begin{enumerate}
	\item Для дискретных случайных величин $ \xi_1, \dots, \xi_n $ определение независимости
		сводится к проверке соотношений
		$$ P\Set{\xi_1 = a_1, \dots, \xi_n = a_n} = P\Set{\xi_1 = a_1} \cdots P\Set{\xi_n = a_n} $$
		для всех значений, принимаемых этими случайными величинами.
	\item Если случайные величины $ \xi_1, \dots, \xi_n $ имеют абсолютно непрерывные распределения,
		то их независимость сводится к равенствам
		$$ p_{(\xi_1, \dots, \xi_n)}(x_1, \dots, x_n) = \prod_{i = 1}^n p_{\xi_i}(x_i) $$
\end{enumerate}

\section{Формулы свёртки}

Пусть $ \xi $ и $ \eta $ "--- независимые случайные величины с функциями распределения $ F_\xi $,
$ F_\eta $, а $ \nu = \xi + \eta $.
Найдём функцию распределения $ H(x) = P\Set{\xi + \eta < x} $.

$$ H(x) = F_{(\xi + \eta)}(x) = \int_{-\infty}^\infty F_\xi(x - y) \di F_\eta(y) $$
$$ H(x) = F_{(\xi + \eta)}(x) = \int_{-\infty}^\infty F_\eta(x - y) \di F_\xi(y) $$

\begin{statement}
	Если величина $ \xi $ имеет плотность распределения $ p_\xi(x) $, то
	$$ F_\nu(x) = \int_{-\infty}^\infty F_\eta(x - y) p_\xi(y) \di y $$
\end{statement}

\begin{statement}
	Если обе величины $ \xi, \eta $ имеют абсолютно непрерывные распределения, то их сумма $ \nu $
	также имеет абсолютно непрерывное распределение, и
	$$ p_\nu(x) = \int_{-\infty}^\infty p_\xi(x - y) p_\eta(y) \di y =
	\int_{-\infty}^\infty p_\eta(x - y) p_\xi(y) \di y $$
\end{statement}

\chapter{Моменты случайных величин}

\section{Математические ожидания}

\begin{definition}
	\emph{Математическим ожиданием} случайной величины $ \xi $ называется её усреднённое значение,
	\ie среднее ожидаемое значение функции $ \xi(\omega) $, которое выражается в виде интеграла по
	вероятностной мере
	$$ \mathbb E \xi = \int\limits_\Omega \xi(\omega) P\Set{\di \omega} $$

	Если множество $ \Omega $ состоит из конечного или счётного набора элементарных исходов, то
	$$ \mathbb E \xi = \sum \xi(\omega_i)P(\omega_i) $$

	Для дискретного распределения
	$$ \mathbb E \xi = \sum_k x_k P\Set{\xi = x_k} $$

	Для абсолютно непрерывного распределения
	$$ \mathbb E \xi = \int_{-\infty}^\infty xp_\xi(x) \di x $$
\end{definition}

Если от величины $ \xi $ переходим к величине $ \eta = g(\xi) $, то математическое ожидание можно
выразить через распределение исходной величины:
$$ \mathbb E \eta = \sum_k g(x_k) P \Set{\xi = x_k}, \quad \text{дискретное распределение} $$
$$ \mathbb E \eta = \int_{-\infty}^\infty g(x) p_\xi(x) \di x, \quad \text{абсолютно непрерывное
распределение} $$

\begin{definition}
	Пусть $ g(x) = x^n $.
	$$ \mathbb E \eta = \mathbb E g(\xi) = \mathbb E \xi^n =
	\int_{-\infty}^\infty x^n \di F_\xi(x) =
	\begin{cases}
		\sum x_k^n P\Set{\xi = x_k} \\
		\int x^n p_\xi(x) \di x
	\end{cases} $$

	$ \mathbb E \xi^n $ называется \emph{начальным моментом} случайной величины
	$ \xi $ порядка $ n $.
\end{definition}

\begin{definition}
	Если возьмём $ g(x) = (x - \mathbb E \xi)^n $, то получим
	$ \mathbb E(\xi - \mathbb E \xi)^n $ "--- \emph{центральный момент} порядка $ n $.
\end{definition}

\begin{definition}
	При $ g(x) = |x|^n $ получаем $ \mathbb E|\xi|^n $ "--- \emph{абсолютный начальный момент}
	порядка $ n $.
\end{definition}

\begin{definition}
	Если $ g(x) = |x - \mathbb E \xi|^n $, то получаем $ \mathbb E|\xi - \mathbb E \xi|^n $ "---
	\emph{абсолютный центральный момент} порядка $ n $.
\end{definition}

\begin{definition}
	Второй центральный момент случайной величины $ D \xi = \mathbb E(\xi - \mathbb E \xi)^2 $
	называется \emph{дисперсией} этой случайной величины.
\end{definition}

\subsection{Свойства математических ожиданий}

\begin{props}
\item Если математические ожидания $ \xi $ и $ \eta $ существуют, то
	$ \mathbb E(\xi + \eta) = \mathbb E \xi + \mathbb E \eta $.
\item $ \mathbb E(\alpha + \beta \xi) = \alpha + \beta \mathbb E\xi $.
\item $ \mathbb E c = c $.
\item $ P\Set{\alpha \le \xi \le \beta} = 1 \implies \alpha \le \mathbb E \xi \le \beta $.
\item $ \xi \le \eta \implies \mathbb E \xi \le \mathbb E \eta $.
\item $ |\mathbb E \xi| \le \mathbb E |\xi| $ (\as $ -|\xi| \le \xi \le |\xi| $).
\item Если $ P \Set{\xi \ge 0} = 1 $ и $ \mathbb E \xi = 0 $, то $ P\Set{\xi = 0} = 1 $.
\item Если $ \xi $ и $ \eta $ независимы, то $ \mathbb \xi \eta =
	\mathbb E \xi \cdot \mathbb E \eta $.
\end{props}

\subsection{Свойства дисперсии}

\begin{props}
\item $ D\xi \ge 0 $.
\item Если $ D \xi = 0 $, то $ P\Set{\xi = C} = 1 $ в некоторой точке $ C $.
\item $ D(\alpha \xi + \beta) = \alpha^2 D\xi $.
\item Если $ \xi $ и $ \eta $ независимы, то $ D(\xi \pm \eta) = D\xi + D\eta $.
\item $ D(\xi + \eta) = D\xi + D\eta + 2 \operatorname{cov}(\xi, \eta) $.
\item $ D\xi \ge D|\xi| $.
\item $ D \xi = \mathbb E\xi^2 - (\mathbb \xi)^2 $.
\end{props}

\subsection{Вычисление математических ожиданий и дисперсий}

\begin{enumerate}
	\item \emph{Вырожденное распределение}
		$$ P\Set{\xi = C} = 1 $$
		$$ \mathbb E \xi = C, \quad D\xi = 0 $$
	\item \emph{Двухточечное распределение}
		$$ \xi =
		\begin{cases}
			0 \quad \text{ с вероятностью } q = 1 - p \\
			1 \quad \text{ с вероятностью } p
		\end{cases} $$
		$$ \mathbb E \xi = p, \quad D \xi = p(1 - p) $$
		$$ \eta =
		\begin{cases}
			a \quad \text{ с вероятностью } q = 1 - p \\
			b \quad \text{ с вероятностью } p
		\end{cases} $$
		$$ D \eta = (b - a)^2p(1 - p) $$
	\item \emph{Биномиальное распределение} ($ \xi \sim B(n, p) $)

		$ \xi $ принимает значения $ 0, 1, 2, \dots, n $ с вероятностями
		$$ p_k = C_n^kp^k(1 - p)^{n - k} $$
		$$ \mathbb E \xi = np, \quad D \xi = np(1 - p) $$
	\item \emph{Пуассоновское распределение} ($ \xi \sim \pi(\lambda) $)
		$$ P\Set{\xi = k} = e^{-\lambda}\frac{\lambda^k}{k!} $$
		$$ \mathbb \xi = \lambda, \quad D\xi = \lambda $$
	\item \emph{Геометрическое распределение} ($ \xi \sim \operatorname{Geom}(p) $)

		$ \xi $ принимает значения $ 0, 1, 2, \dotsc $ с вероятностями
		$$ p_k = (1 - p)p^k $$
		$$ \mathbb E\xi = \frac p{1 - p}, \quad D \xi = \frac p{(1 - p)^2} $$
	\item \emph{Равномерное распределение} ($ \xi \sim U[a, b] $)

		Плотность распределения имеет следующий вид:
		$$ p(x) =
		\begin{cases}
			\frac1{b - a}, \quad x \in [a, b], \\
			0, \quad x \not\in [a, b]
		\end{cases} $$
		$$ \mathbb E \xi = \frac{b + a}2, \quad D \xi = \frac{(b - a)^2}{12} $$
	\item \emph{Экспоненциальное распределение} ($ \xi \sim E(\lambda) $)

		Плотность $ \xi $ имеет вид
		$$ p(x) =
		\begin{cases}
			0, \quad x < 0, \\
			\frac1\lambda e^{-\frac x \lambda}, \quad x \ge 0
		\end{cases} $$
		$$ \mathbb E \xi = \lambda, \quad D\xi = \lambda^2 $$
	\item \emph{Распределение Коши}
		$$ p(x) = \frac1\pi (1 + x^2) $$
		$$ \int\limits_{-\infty}^\infty p(x) \di x = \infty $$
		Значит, математического ожидания и дисперсии $ \xi $ не существует.
		Более того, понятно, что и моментов более высокого порядка не существует.
	\item \emph{Нормальное распределение} ($ \xi \sim N(a, \sigma^2) $)
		$$ p(x) = (2\pi)^{-\frac12} \frac1\sigma e^{-\frac{(x - a)^2}{2\sigma^2}} $$
		$$ \mathbb \xi = a, \quad D \xi = \sigma^2 $$
\end{enumerate}

\section{Вероятностные неравенства}

\begin{lemma}[Чебышёва]
	$ \xi \ge 0, \quad \exists \mathbb E \xi, \quad t > 0 $

	$$ P\Set{\xi \ge t} \le \frac{\mathbb E \xi}t $$
\end{lemma}

\begin{proof}
	$$ P\Set{\xi \ge t} = \int_t^\infty \di F(x) \le \frac1t \int_t^\infty x \di F(x) \le
	\frac1t \int_0^\infty x \di F(x) = \frac{\mathbb E \xi}t $$
\end{proof}

\begin{remark}
	Аналогично,
	$$ P\Set{\xi \ge t} \le \frac{\mathbb E\xi^2}{t^n} $$
\end{remark}

Подставляя $ \xi = (\xi - \mathbb \xi)^2 $ и $ t = \eps^2 $, получаем следующий результат

\begin{statement}[неравенство Чебышёва]
	$ \exists \mathbb E \xi, ~ D \xi, \quad \eps > 0 $

	$$ P\{|\xi - \mathbb E\xi| > \eps\} \le \frac{D\xi}{\eps^2} $$
\end{statement}

\begin{statement}[неравенство Лярунова]
	$ \gamma_n \coloneq \mathbb E |\xi|^n $

	$$ \implies \gamma_1 \le \gamma_2^{\frac12} \le \gamma_3^{\frac13} \le \dots \le
	\gamma_{m - 1}^{\frac1{m - 1}} \le \gamma_m^{\frac1m} \le \dotso $$

	$$ \phi(a) = (\mathbb E|\xi|^\alpha)^{\frac1\alpha} $$

	Тогда $ \phi $ монотонно возрастает.
\end{statement}

\section{Смешанные моменты}

\begin{definition}
	Для $ n $-мерного случайного вектора $ \ol\xi = (\xi_1, \dots, \xi_n) $ рассмотрим
	\emph{математическое ожидание}
	$$ \mathbb E g(\xi_1, \dots, \xi_n) = \iiiint\limits_{-\infty}^\infty g(x_1, \dots, x_n)
	F(\di x_1, \dots, \di x_n) $$

	Для дискретных случайных векторов имеем
	$$ \mathbb Eg(\xi_1, \dots, \xi_n) = \sum P\Set{\xi_1 = x_1, \dots, \xi_n = x_n}
	g(x_1, \dots, x_n) $$

	Для абсолютно непрерывных случайных векторов
	$$ \mathbb Eg(\xi_1, \dots, \xi_n) = \iiint\limits_{-\infty}^\infty g(x_1, \dots, x_n)
	p(x_1, \dots, x_n) \di x_1 \dots \di x_n $$
\end{definition}

Пусть $ \xi, \eta $ "--- независимые случайные величины.
Тогда
$$ F_{\xi, \eta}(x, y) = F_\xi(x)F_\eta(y) $$

\begin{statement}
	Если $ g(x, y) = h(x)r(y) $, то
	$$ \mathbb Eg(\xi, \eta) = \mathbb E \bigl( h(\xi)r(\eta) \bigr) =
	\int\limits_{-\infty}^\infty \int\limits_{-\infty}^\infty h(x)r(y) F_{\xi, \eta}(\di x, \di y) =
	\int\limits_{-\infty}^\infty h(x)F_\xi(\di x) \int\limits_{-\infty}^\infty r(y) F_\eta(\di y) =
	\mathbb Eh(\xi) \mathbb E r(\eta) $$
\end{statement}

\begin{definition}
	\emph{Начальный смешанный момент} порядка $ n + m $ определяется для двумерных дискретных и
	двумерных абсолютно непрерывных вектором следующими соотношениями:
	$$ \mathbb E\xi^n \eta^m = \sum_k \sum_l x_k^n y^-l^m P\Set{\xi = x_k, ~ \eta = y_l} $$
	$$ \mathbb E\xi^n\eta^m = \int\limits_{-\infty}^\infty \int\limits_{-\infty}^\infty x^ny^m
	p_{\xi, \eta}(x, y) \di x \di y $$
\end{definition}

\begin{definition}
	Если взять $ g(x, y) = (x - \mathbb E\xi)^n (y - \mathbb E\eta)^m $, то математическое ожидание
	$ \mathbb E(\xi - \mathbb E\xi)^n(\eta - \mathbb \eta)^m $ называется
	\emph{центральным смешанным моментом} порядка $ n + m $.
\end{definition}

\begin{definition}
	Центральный смешанный момент $ \operatorname{cov}(\xi, \eta) =
	\mathbb (\xi - \mathbb E\xi)(\eta - \mathbb \eta) $ называется \emph{ковариацией}.
\end{definition}

\subsection{Свойства ковариаций}

\begin{props}
\item $ \mathbb E(\xi - \mathbb E\xi)(\eta - \mathbb E\eta) =
	\mathbb E(\xi\eta) - \mathbb E\xi \mathbb E \eta $.
\item $ \operatorname{cov}(\xi, \xi) = \mathbb E(\xi - \mathbb E \xi)^2 = D\xi $.
\item $ \operatorname{cov}(\alpha \xi + \beta, \gamma \eta + \delta) =
	\alpha \gamma \operatorname{cov}(\xi ,\eta) $.
\item Если $ \xi $ и $ \eta $ независимы, то $ \operatorname{cov}(\xi, \eta) = 0 $.
\item $ D(\xi \pm \eta) = D\xi + D\eta \pm 2\operatorname{cov}(\xi, \eta) $.
\item $ |\operatorname{cov}(\xi, \eta)| \le \sqrt{D\xi D\eta} $.
\end{props}

\begin{proof}[6]
	$$ D \Bigl( \frac\xi{\sqrt{D\xi}} + \frac\eta{\sqrt{D\eta}} \Bigr) =
	2 + 2 \frac{\operatorname{cov}(\xi, \eta)}{\sqrt{D\xi} \sqrt{D\eta}} \ge 0 $$
	$$ D \Bigl( \frac\xi{\sqrt{D\xi}} - \frac\eta{\sqrt{D\eta}} \Bigr) =
	2 - 2 \frac{\operatorname{cov}(\xi, \eta)}{\sqrt{D\xi} \sqrt{D \eta}} \ge 0 $$
\end{proof}

\subsection{Коэффициент корреляции}

Пусть $ \xi, \eta $ "--- невырожденные случайные величины, у которых существуют дисперсии.
Рассмотрим их центрированные и нормированные варианты
$$ \hat\xi = \frac{\xi - \mathbb \xi}{\sqrt{D \xi}}, \quad \hat \eta =
\frac{\eta - \mathbb E \eta}{\sqrt{D \eta}} $$

Имеем равенства
$$ \mathbb E \hat \xi = \mathbb E \hat \eta = 0, \quad D \hat \xi = D \hat \eta = 1 $$

\begin{definition}
	\emph{Коэффициент корреляции} $ \rho $ между $ \xi $ и $ \eta $ задаётся равенством
	$$ \rho(\xi, \eta) = \operatorname{cov}(\hat \xi, \hat \eta) =
	\frac{\operatorname{cov}(\xi, \eta)}{\sqrt{D \xi} \sqrt{D \eta}} $$
\end{definition}

\subsection{Свойства коэффициента корреляции}

\begin{props}
\item Если $ \xi, \eta $ независимы, то $ \rho(\xi, \eta) = 0 $.
	В общем случае, если $ \rho(\xi, \eta) = 0 $, то говорим, что $ \xi, \eta $ "---
	\emph{ортогональные (некоррелированные)} случайные величины.
\item $ |\rho| \le 1 $.
\item Если $ |\rho(\xi, \eta)| = 1 $, то $ \eta = \alpha\xi + \beta $, причём
	$ \sign(\rho) = \sign(\alpha) $.
\item $ \rho(\alpha \xi + \beta, \delta \eta + \gamma) = \sign(\alpha \delta) \rho(\xi, \eta) $.
\end{props}

\section{Производящие функции}

\begin{definition}
	$ \Set{a_n \in \mathbb R}_{k = 1}^\infty $
	$$ A(s) = \sum_{n = 0}^\infty a_ns^n $$

	Если ряд сходится на интервале $ [-s_0, s_0], ~ s_0 > 0 $, то функция $ A(s) $ называется
	\emph{производящей функцией} последовательности $ \Set{a_n} $.
\end{definition}

Рассмотрим в качестве $ \Set{a_n} $ распределения вероятностей.
Пусть случайная величина $ \xi $ принимает значения $ 0, 1, \dotsc $ с вероятностями
$ p_n = P\Set{\xi = n} $.

Построим производящую функцию для последовательности $ \Set{p_n} $:
$$ P(s) = \sum_{n = 0}^\infty p_ns^n $$
Ряд сходится абсолютно при $ -1 \le s \le 1 $.
$ P(s) $ "--- производящая функция $ \xi $.
Её можно представить в виде
$$ P(s) = \sum_{n = 0}^\infty p_ns^n = \mathbb Es^\xi $$

\begin{exmpls}
\item Биномиальное распределение $ \xi \sim B(n, p) $.

	Случайная величина принимает значения $ 0, 1, \dots, n $ с вероятностями
	$$ p_m = C_n^m p^m(1 - p)^{n - m} $$
	
	Её производящая функция имеет вид
	$$ P(s) = \sum_{m = 0}^n C_n^M p^m(1 - p)^{n - m}s^m =
	\sum_{m = 0}^n C_n^m (ps)^m (1 - p)^{n - m} = (1 - p + ps)^n $$
\item Пуассоновское распределение $ \xi \sim \pi(\lambda) $.

	Случайная величина принимает значения $ 0, 1, \dotsc $ с вероятностями
	$$ p_n = e^{-\lambda} \frac{\lambda^n}{n!} $$
	$$ P(s) = \sum_{n = 0}^\infty e^{-\lambda} \frac{(\lambda s)^n}{n!} =
	e^{-\lambda}e^{\lambda s} = e^{\lambda (s - 1)} $$
\item Геометрическое распределение
	$$ p_n = (1 - p)p^n, \quad n = 0, 1, \dots, \quad 0 < p < 1 $$
	$$ P(s) = \sum_{n = 0}^\infty (1 - p)(ps)^n = \frac{1 - p}{1 - ps} $$
\end{exmpls}

\subsection{Свойства производящих функций}

\begin{props}
\item $ P(1) = 1 $.
\item $ |P(s)| \le 1, \quad |s| \le 1 $.
\item Если $ \xi $ и $ \eta $ независимы, и $ \nu = \xi + \eta $, то
	$$ P_\nu(s) = \mathbb Es^\nu = \mathbb E s^\xi \cdot \mathbb E s^\eta = P_\xi(s)P_\eta(s) $$
\item $ 0 \le 1 $
	$$ P_\xi(s) = \sum_{n = 0}^\infty p_ns^n \ge 0 $$
	$$ \nder[k]P(s) = \sum_{n = 0}^\infty n (n - 1) \cdots (n - k + 1)p_ns^{n - k} $$
	$$ 0 \le \nder[k]P(s) \le \sum_{n = 0}^\infty n(n - 1) \cdots (n - k + 1)s^{n - k} $$
\item Если известна производящая функция, то можно восстановить вероятности:
	$$ \nder[k]P(0) = k!p_k $$
	$$ p_k = \frac{\nder[k]P(0)}{k!}, \quad p_0 = P(0) $$
\item $ P'(1) = \mathbb E \xi = \sum k p_k $

	Наряду с вероятностями $ p_k = P\Set{\xi = k} $ рассмотрим вероятности $ q_n = P\Set{\xi > n} $.
	\begin{multline*}
		(s) = \sum_{n = 0}^\infty q_ns^n =
		\sum_{n = 0}^\infty \Bigl( \sum_{k = n + 1}^\infty p_k \Bigr) s^n =
		\sum_{k = 1}^\infty p_k \sum_{n = 0}^{k - 1}s^n =
		\sum_{k = 1}^\infty p_k \cdot \frac{1 - s^k}{1 - s} = \\
		= \frac1{1 - s} \Bigl( \sum_{k = 1}^\infty p_k - \sum_{k = 1}^\infty p_ks^k \Bigr) =
		\frac1{1 - s} \bigl( 1 - P(s) \bigr)
	\end{multline*}

	При этом,
	$$ Q(s) = \frac{1 - P(s)}{1 - s} $$
	Переходя к пределу по $ s $, получаем
	$$ \lim\limits_{s \to 1} Q(s) = P'(1) = \mathbb E \xi $$

	Отсюда, в частности,
	$$ \mathbb E \xi = \sum_{n = 0}^\infty P\Set{\xi > n} $$
\end{props}

\section{Факториальные моменты}

\begin{definition}
	\emph{Факториальным моментом} порядка $ k $ случайной величины $ \xi $ называется величина
	$$ \mu_k = \mathbb E \bigl( \xi (\xi - 1)(\xi - 2) \cdots (\xi - k + 1) \bigr) $$	
\end{definition}

По начальным моментам однозначно находятся факториальные моменты:
$$ \mu_1 = \mathbb \xi = \alpha_1 $$
$$ \mu_2 = \mathbb E \xi (\xi - 1) = \alpha_2 - \alpha_1 $$
$$ \mu_3 = \mathbb E \xi (\xi - 1) (\xi - 2) = \alpha_3 - 3\alpha_2 + 2\alpha_1 $$

$$ \nder[k]P(s) = \sum_{n = 0}^\infty n (n - 1) \cdots (n - k + 1) p_ns^{n - k}, \quad 0 \le s < 1 $$
Так как $ P(s) $ и её производные "--- монотонные непрерывные функции,
$$ \exists \lim\limits_{s \to 1^-} \nder[k]P(s) = \nder[k]P(1) =
\sum_{n = 0}^\infty n (n - 1) \cdots (n - k + 1)p_n = \mu_k $$

Будем говорить, что существует \emph{конечный факториальный момент}, если существует конечный левый
предел $ \nder[k]P(s) $ при $ s \to 1^- $.

\begin{exmpls}
\item $ \xi \sim B(n, p) $
	$$ P(s) = (1 - p + ps)^n $$
	$$ \mu_k =
	\begin{cases}
		\nder[k]P(1) = n(n - 1) \cdots (n - k + 1)p^n, \quad k \le n, \\
		0, \quad k > n
	\end{cases} $$
\item $ \xi \sim \pi(\lambda) $
	$$ P(s) = e^{\lambda(s - 1)} $$
	$$ \mu_k = \nder[k]P(1) = \lambda^k $$
\item Геометрическое распределение
	$$ P(s) = \frac{1 - p}{1 - ps} $$
	$$ \mu_k = \nder[k]P(1) = \frac{(1 - p)p^k k!}{(1 - p)^{k + 1}} = p^k k! (1 - p)^{-k} =
	\Bigl( \frac{p}{1 - p} \Bigr)^k \cdot k! $$
\end{exmpls}
