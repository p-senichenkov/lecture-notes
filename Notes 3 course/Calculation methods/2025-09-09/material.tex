\chapter{Приближение функций}

\begin{note}
	В конспекте есть ещё несколько параграфов, которые на экзамен не попадут.
\end{note}

\section{Интерполирование как способ приближения функций}

\begin{problem}
	В дискретные моменты времени $ x_1, \dots, x_n $ наблюдаются значения функции $ f $.
	Требуется установить её значения при других $ x $.
\end{problem}

\begin{problem}
	Требуется вычислять одну и ту же сложную функцию в различных точках.
	Иногда целесообразно найти значения в некоторых точках, а в других точках вычислять по простым формулам, используя информацию о вычисленных точках.
\end{problem}

\section{Алгебраическое интерполирование}

\begin{definition}
	Пусть заданы значения функции в $ n + 1 $ попарно различных точках:
	$$ f(x_0), \dots, f(x_n), \quad x_i \ne x_j $$

	Требуется найти алгебраический многочлен степени не выше $ n $
	$$ P_n(x) : \quad P_n(x_j) = f(x_j) $$

	Эта задача называется \emph{задачей алгебраического интерполирования}.
\end{definition}

$$ P_n(x) = a_0 + a_1x + \dots + a_nx^n $$

\begin{statement}
	Задача алгебраического интерполирования однозначно разрешима при любом выборе попарно различных узлов.
\end{statement}

\subsection{Представление в форме Лагранжа}

\TODO{Начало представления в форме Лагранжа}

$$ \dots $$

$$ \omega_{n + 1}(x) = \prod_{k = 0}^n (x - x_k) $$
$$ l_{kn}(x) = \frac{(x - x_))(x - x_1) \cdots (x - x_{k - 1})\dots}{\dots} = \dots $$

\begin{note}
	В задании нужно пользоваться первым представлением из этой формулы.
\end{note}

Рассмотрим
\begin{equ}2
	L_n(x) = \sum_{k = 0}^n l_{kn}(x) f(x_k)
\end{equ}

Рассмотрим
$$ \frac{\omega_{n + 1}(x)}{x - x_k} = (x - x_0) \cdots (x - x_{k - 1})(x - x_{k + 1}) \cdots (x - x_n) $$
$$ \lim\limits_{x \to x_k} \frac{\omega_{n + 1}(x)}{x - x_k} = \omega_{n + 1}'(x_k) $$

Таким образом,
$$ L_n(x_j) = \sum_{k = 0}^n l_{kn}(x_j) f(x_k) = f(x_j) \quad \forall j = 0, \dots, n $$

Получили, что \eref2 "--- интерполяционный многочлен.
Он называется \emph{представлением в форме Лагранжа}.

\subsubsection{Устойчивость вычислений}

Ранее получили представление в форме Лагранжа для интерполяционного многочлена:
$$ P_n(x) = \sum_{k = 0}^n l_{kn}(x) f(x_k) $$

На практике оперируем приближёнными значениями $ \vawe f(x_k) = f(x_k) + \eps(x_k) $, где $ |\eps(x_k)| \le \eps_0 $.
Тогда
$$ \vawe P_n(x) = \sum_{k = 0}^n l_{kn}(x) \vawe f(x_k) = \dots = \sum l_{kn}(x) f(x_k) + \sum l_{kn}(x) \eps(x_k) = P_n(x) + \sum l_{kn}(x) \eps(x_k) $$

Оценим ошибку по абсолютной величине:
$$ \Bigl| \sum l_{kn}(x) \eps(x_k) \Bigr| \le \eps_0 \sum |l_{kn}(x)| $$
Следовательно, $ \lambda_n(x) = \sum |l_{kn}(x)| $ показывает, во сколько раз вырастет ошибка.
Этот коэффициент называется \emph{функцией Лебега}.

\begin{definition}
	\emph{Постоянной Лебега} назовём
	$$ \Lambda_n = \max\limits_{x \in [a, b]} \lambda_n(x) $$
\end{definition}

\begin{remark}
	Если рассмотреть интерполяционный многочлен для $ f \equiv 1 $, то $ P_n(x) \equiv f(x) \equiv 1 $.
	Тогда из представления в форме Лагранжа получаем
	$$ \sum l_{kn}(x) = 1 \implies \lambda_n(x) \ge 1 $$
\end{remark}

\subsubsection{Погрешность алгебраического интерполирования}

\begin{definition}
	\emph{Погрешностью (остаточным членом)} алгебраического интерполирования назовём
	$$ R_n(f, x) = f(x) - P_n(x) $$
\end{definition}

\begin{remark}
	Если функция $ f $ сама является многочленом степени не выше $ n $, то она совпадёт с $ P_n $ и $ R_n \equiv 0 $.

	Иначе $ R_n(f, x_j) = 0 $ (совпадает только в узлах интерполирования).
\end{remark}

\begin{theorem}[о представлении погрешности алгераического интерполирования]
	$ f \in \mathcal C^{n + 1}[A, B] $, где $ A = \min\set{x, x_0, \dots, x_n} $, $ B = \max\set{x, x_0, \dots, x_n} $.

	Тогда существует $ \xi = \xi(x) \in (A, B) $ такое, что погрешность в точке $ x $ допускает представление:
	$$ R_n(f, x) = f(x) - P_n(x) = \frac{\nder[n + 1]f(\xi)}{(n + 1)!}\omega_{n + 1}(x) $$
\end{theorem}
