\chapter{Равномерное приближение}

\section{Многочлены Чебышёва первого рода}

\begin{quote}
	\raggedleft
	Дальше "--- сакральное знание.
	В конспектах этого нет.
\end{quote}

\subsection{Свойство наименьшего уклонения от нуля}

\begin{definition}
	$ \Pi_n $ "--- множество приведённых многочленов степени ровно $ n $.
\end{definition}

\begin{definition}
	Рассмотрим приведённый многочлен Чебышёва:
	$$ \vawe T_n(x) = 2^{1 - n}T_n(x) = x^n + \dotsi $$
	Можно доказать, что $ \vawe T_n $ среди всех приведённых многочленов степени $ n $ имеет
	наименьшую норму в пространстве непрерывных функций $ \mathcal C[-1, 1] $.
	Такой приведённый многочлен будем называть \emph{многочленом, наименее уклоняющимся от нуля}.
\end{definition}

\subsubsection{Критерий многочлена наименьшего уклонения от нуля в
\texorpdfstring{$ \mathcal C[a, b] $}{C[a, b]}}

\begin{theorem}\label{th:crit}
	Для того, чтобы $ Q_n \in \Pi_n $ был МНУ от 0 в $ \mathcal C[-1, 1] $ \textbf{необходимо и
	достаточно}, чтобы
	$$ \exists -1 \le t_1 < \dots < t_{n + 1} \le 1 \quad
	\begin{cases}
		|Q_n(t_i)| = \|Q_n\|_{\mathcal C[-1, 1]} \\
		Q_n(t_i) = -Q_n(t_{i + 1})
	\end{cases} $$
\end{theorem}

\begin{proof}
	Рассмотрим $ \mathcal C[-1, 1] $.
	$$ \forall f \in \mathcal C[-1, 1] \quad \exists ! P_l^* \text{ "--- ПНРПр} $$
	Пусть $ f(x) = x^n $.
	$$ \exists ! P_{n - 1}^* \text{ "--- ПНРПр} \underiff{\text{т. Чебышёва}}
	\exists -1 \le t_1 < t_2 < \dots t_{n + 1} \le 1 $$

	$$ Q_n(x) \coloneq \frac{x^n - P_{n - 1}^*(x)}{|Q_n(t_i)|} = \|Q_n\|_{\mathcal C[-1, 1]} \quad
	\forall i = 1, \dots, n + 1 $$
	$$ Q_n(t_i) = -Q_n(t_{i + 1}) $$
\end{proof}

\begin{theorem}
	В $ \mathcal C[-1, 1] $ МНУ от 0 является
	$$ \vawe T_n(x) = \frac{T_n(x)}{2^{n - 1}} $$
\end{theorem}

\begin{proof}
	$ \vawe T_n \in \Pi_n $

	Точки экстремума $ T_n $:
	$$ x_l' \coloneq \cos \Bigl( \frac{\pi l}n \Bigr), \quad l = 0, 1, \dots, n $$

	Известно, что $ T_n(x_l') = (-1)^l $.
	$$ \|T_n\|_{\mathcal C[-1, 1]} = \max\limits_{x \in [-1, 1]}|T_n(x)| $$
	$$ \|\vawe T_n\|_{\mathcal C[-1, 1]} = \frac1{2^{n - 1}} $$

	Положим $ t_k = x_{n + 1 - k}, \quad k = 1, \dots, (n + 1) $.
	Дальше можно воспользоваться \autoref{th:crit}.
\end{proof}

Если требуется получить МНУ на конечном $ [a, b] $, то можно воспользоваться заменой
$$ z = \frac{b - a}2x + \frac{b + a}2, \quad x = \frac{2z - (a + b)}{b - a} $$
