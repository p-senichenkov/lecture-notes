\chapter{Численное дифференцирование}

\begin{definition}
	Задача называется \emph{корректной}, если её решение существует, единственно и непрерывно по исходным данным.
\end{definition}

$$ \nder[m]f(x) \simeq \nder[m]P_n(x) $$

\section{Формулы численного дифференцирования}

\subsection{Правая разностная производная}

Функция $ f $ задана в точках $ a $ и $ a + h $.

$$ f'(a) \simeq P_1'(a) = f(a, a + h) = \frac{f(a + h) - f(a)}h $$

\subsection{Левая разностная производная}

Функция $ f $ задана в точках $ a - h $ и $ a $.

$$ f'(a) \simeq P_1'(a) = \frac{f(a) - f(a - h)}h $$

\subsection{Центральная разностная производная}

$$ f'(a) \simeq f(a - h, a) + h f(a - h, a, a + h) = \frac{f(a + h) - f(a - h)}{2h} $$

\subsection{Точка в начале таблицы}

Известны значения функции $ f $ в точках $ a, a + h $ и $ a + 2h $.

$$ f'(a) \simeq \frac{-3f(a) + 4f(a + h) - f(a + 2h)}{2h} $$

\subsection{Точка в конце таблицы}

Известны значения в точках $ a - 2h, a - h $ и $ a $.

$$ f'(a) = \frac{3f(a) - 4f(a - h) + f(a - 2h)}{2h} $$

\section{Метод неопределённых коэффициентов построения формул численного дифференцирования}

Известны значения в точках $ x - h, x $ и $ x + h $.

Запишем разложение в ряд Тейлора:
$$ f(x + h) = f(x) + \frac{f'(x)}1 h + \dotsi $$
$$ f(x - h) = f(x) - \frac{f'(x)}1 h + \dotsi $$
$$ f(x) = f(x) $$

Домножим всё на $ \alpha, \beta, \gamma $ и сложим:
$$ \alpha f(x + h) + \beta f(x - h) + \gamma f(x) = (\alpha + \beta + \gamma) f(x) + h (\alpha - \beta)f'(x) + \dotsi $$

Подберём $ \alpha, \beta, \gamma $ так чтобы получалось значение второй производной и остаток:
$$ \alpha = \beta = \frac1{h^2}, \quad \gamma = -\frac2{h^2} $$
Получаем
$$ \frac{f(x + h) - 2f(x) + f(x - h)}{h^2} = f''(x) + \frac{h^2}{12}\nder[4]f(x) + \dotsi $$
