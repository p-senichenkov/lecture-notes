\chapter{Равномерное приближение}

\section{Многочлены Чебышёва первого рода}

\begin{definition}
	\emph{Многочленом Чебышёва первого рода} назовём функцию
	$$ T_n(x) = \cos(n \arccos x), \quad n = 0, 1, \dotsc \quad x \in [-1, 1] $$
\end{definition}

Можно доказать, что $ T_n $ "--- многочлен степени ровно $ n $.
$$ T_k(x) = 2xT_{k - 1}(x) - T_{k - 2}(x), \quad k = 2, 3, \dotsc $$

\subsection{Корни и точки экстремума многочленов Чебышёва}

Корни:
$$ x_k = \cos \Bigl( \frac{2k - 1}{2n} \pi \Bigr), \quad k = 1, 2, \dots, n $$

\subsubsection{Геометрическая интерпретация корней \texorpdfstring{$ T_n(x) $}{Tn(x)}}

$ x_k $ можно рассматривать как проекции на ось абсцисс $ 2n + 1 $ равноотстоящих точек единичной
окружности (\ie верхняя полуокружность разделена на $ 2n $ частей).

\subsubsection{Построение графика \texorpdfstring{$ T_n(x) $}{Tn(x)}}

Нужно разбить единичную полуокружность на $ 2n $ равных частей, занумеровать точки деления против
часовой стрелки, отметить корни (проекции точек с чётными номерами) и точки экстремума (проекции
точек с нечётными номерами).
Отметить экстремальные значения $ +1 $ и $ -1 $.

\subsection{Формы записи многочлена Чебышёва}

\begin{enumerate}
	\item $ T_n(x) = \cos (n \arccos x) $.
		Можно использовать только на $ [-1, 1] $.
	\item $ T_n(x) = 2xT_{n - 1}(x) - T_{n - 2}(x), \quad T_0(x), \quad T_1(x) $.
	\item $ T_n(x) = a_nx^n - a_{n - 2}x^{n - 2} + a_{n - 4}x^{n - 4} - \dotsb, \quad a_n = 2^{n - 1} $.
		Самая плохая форма.
	\item
		$$ T_n(x) = \prod \dots $$
	\item
		$$ T_n(x) = \frac12 \Bigl( \bigl(x + \sqrt{x^2 - 1} \bigr)^n + \bigl( x - \sqrt{x^2 - 1} \bigr)^n \Bigr) $$
		Удобна, когда $ x \in \R, ~ |x| \ge 1 $.
\end{enumerate}
