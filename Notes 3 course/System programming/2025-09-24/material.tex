\chapter{Логические вентили}

\section{Сумматоры}

\begin{figure}[!h]
	\centering
	\begin{subfigure}{0.32\textwidth}
		\centering
		\includegraphics[width=\textwidth]{quart_sum}
		\caption{Четвертьсумматор}
	\end{subfigure}
	\begin{subfigure}{0.32\textwidth}
		\centering
		\includegraphics[width=\textwidth]{half_sum}
		\caption{Полусумматор}
	\end{subfigure}
	\begin{subfigure}{0.32\textwidth}
		\centering
		\includegraphics[width=\textwidth]{full_sum}
		\caption{Полный сумматор}
	\end{subfigure}
\end{figure}

$ C_{out} $ "--- вычисляемый перенос, $ C_{in} $ "--- входной перенос.

\subsection{Длинные числа}

\begin{figure}[!h]
	\centering
	\includegraphics[width=0.3\textwidth]{ripple_carry}
	\caption{Схема с протаскиванием переноса (ripple-carry)}
\end{figure}

\subsection{Должок с прошлого раза}

Что такое \emph{комбинационная схема}:
\begin{enumerate}
	\item фиксированы входы и выходы;
	\item схема состоит из вентилей, на каждый из них идёт вход всей схемы или ровно одного вентиля;
	\item циклы запрещены.
\end{enumerate}

\begin{remark}
	Для того, чтобы использовать дерево отрезка достаточно ассоциативности операции.
\end{remark}

Для того, чтобы вычислять переносы, нужен нейтральный элемент (нулевой перенос).
Множество с ассоциативной операцией и нейтральным элементом "--- моноид.

\subsubsection{Как это применить в сумматорах?}

Единственное, что нужно научится вычислять быстро "--- это перенос.
Наша схема "--- это конвейер для переносов.

Схема работы каждого из сумматоров: \\
\begin{tabular}{c c | c}
	x & y & $ C_{out} $ ($ g_0 $) \\
	0 & 0 & 0 (p) \\
	0 & 1 & $ C_{in} $ (p) \\
	1 & 0 & $ C_{in} $ ($ g_1 $) \\
	1 & 1 & 1
\end{tabular}

$$ C_{out} = g_1(\dots(p(g_1(\dots(C_{in})\dots)))\dots) $$

Таблица умножения: \\
\begin{tabular}{r | c c c}
	& p & $ g_0 $ & $ g_1 $ \\
	\hline
	p & $ p $ & $ g_0 $ & $ g_1 $ \\
	$ g_0 $ & $ g_0 $ & $ g_0 $ & $ g_1 $ \\
	$ g_1 $ & $ g_1 $ & $ g_0 $ & $ g_1 $
\end{tabular}

Композиция этих операций ассоциативна.

\begin{figure}[!h]
	\centering
	\includegraphics[width=0.6\textwidth]{segment_tree_sum}
	\caption{Сумматор с использованием идеи дерева отрезков}
\end{figure}

Критический путь сократился до логарифма.
Элементов стало примерно в два раза больше (по аналогии с деревом отрезков).
