\chapter{Транзисторы}

\section{Технологический процесс}

Сейчас это не означает ровным счётом ничего.
Когда-то оно означало длину затвора транзистора.

Размер атома кремния "--- 130 пикометров.

\section{Про частоты}

Чем выше частота, тем чаще дёргаются транзисторы.
Транзистор больше нагревается за счёт того, что электроны бегают с гораздо большим энтузиазмом.

\chapter{Логические вентили}

\section{Базовые операции}

\begin{definition}
	$ f : \mathbb B^n \to \mathbb B^m $ "--- \emph{булева функция}.
\end{definition}

\begin{figure}[!ht]
	\centering
	\begin{subfigure}{0.499\textwidth}
		\centering
		\includegraphics[width=\textwidth]{not}
		\caption{NOT}
	\end{subfigure}
	\begin{subfigure}{0.19\textwidth}
		\centering
		\includegraphics[width=\textwidth]{nand}
		\caption{NAND}
	\end{subfigure}
	\begin{subfigure}{0.19\textwidth}
		\centering
		\includegraphics[width=\textwidth]{nor}
		\caption{NOR}
	\end{subfigure}
	\begin{subfigure}{0.19\textwidth}
		\centering
		\includegraphics[width=\textwidth]{or}
		\caption{OR}
	\end{subfigure}
	\begin{subfigure}{0.19\textwidth}
		\centering
		\includegraphics[width=\textwidth]{and}
		\caption{AND}
	\end{subfigure}
	\begin{subfigure}{0.19\textwidth}
		\centering
		\includegraphics[width=\textwidth]{xor}
		\caption{XOR}
	\end{subfigure}
\end{figure}

Остальные реализации долго рисовать.

$$ X \textsc{xor} Y = X\ol Y + \ol X Y $$
$$ X \to Y \sim \ol X \vee Y $$

\begin{remind}
	Транзистор с кружочком открывается при подаче 0 на затвор, транзистор без кружочка "--- при подаче 1.
\end{remind}

\section{Как построить произвольную булеву функцию?}

Строим AST, каждый узел заменяем на соответствующий вентиль.
Повторяющиеся блоки можно переиспользовать.

\section{Как построить минимальную схему для функции?}

Задача оптимизации схемы для функции (в каком-то смысле) алгоритмически \textbf{разрешима} (\as количество связок ограничено сверху, значит, все возможные схемы можно перебрать).
Доказано, что она принадлежит к классу $ \Sigma_2^p $ "--- разрешима на недетерминированной машине Тьюринга с оракулом (МТ может обращаться к оракулу, который умеет за один такт решать любую NP-задачу).

Будем пользоваться эвристиками, например Картами Карно.

\subsection{Карты Карно (Karnaugh)}

\begin{eg}
	Пусть есть булева функция четырёх аргументов, заданная своей таблицей истинности: \\
	\begin{tabular}{c c c c | c}
		$ x_1 $ & $ x_2 $ & $ x_3 $ & $ x_4 $ & $ F $ \\
		\hline
		0 & 0 & 0 & 0 & 1 \\
		. & . & . & . & .
	\end{tabular}

	Разбиваем аргументы на две группы (в нашем случае по два), рисуем квадратную табличку.
	Заголовки таблички записываем циклическим кодом Грея (то есть, в таком порядке, чтобы между двумя соседними расстояние Хэмминга было 1).
	Получаем таблицу истинности для пар битов: \\
	\begin{tabular}{c | c c c c}
		& 0 0 & 0 1 & 1 1 & 1 0 \\
		\hline
		0 0 & 1 & 1 & 0 & 0 \\
		0 1 & 0 & 0 & 1 & 1 \\
		1 1 & 1 & 0 & 1 & 1 \\
		1 0 & 1 & 1 & 0 & 0
	\end{tabular}

	Объединяем группы соседних единиц.
	При этом, группы могут пересекаться.

	$$ x_2 x_3 + \ol x_2 \ol x_3 + x_1 \ol x_3 \ol x_4 $$
\end{eg}

\section{Шифраторы}

\emph{Шифратор} (энекодером называть нельзя): \\
Есть несколько входов, на выходе число "--- номер входа, на который подана 1.
Если на двух ножках единица "--- это некорректный код.

\emph{Дешифратор} "--- наоборот.

\begin{figure}[!h]
	\centering
	\begin{subfigure}{0.49\textwidth}
		\centering
		\includegraphics[width=\textwidth]{encoder}
		\caption{Шифратор}
	\end{subfigure}
	\begin{subfigure}{0.49\textwidth}
		\centering
		\includegraphics[width=\textwidth]{decoder}
		\caption{Дешифратор}
	\end{subfigure}
\end{figure}

\section{Мультиплексоры}

\emph{Мультиплексор}: \\
Есть несколько входов, только один выход и два управляющих бита.
Управляющие сигналы показывают, какой вход слушать.

\emph{Демультиплексор} "--- наоборот.
