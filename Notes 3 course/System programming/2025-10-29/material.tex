\section{Архитектура процессора}

ISA "--- Instruction set Architecture

\begin{enumerate}
	\item Набор команд
		\begin{enumerate}
			\item привилегированные;
			\item непривилегированные.
		\end{enumerate}
	\item Видимое состояние
		\begin{enumerate}
			\item регистры;
			\item базовая периферия;
		\end{enumerate}
\end{enumerate}

\begin{exmpls}
\item RISC-V;
\item x86;
\item ARM;
\item MIPS;
\item PowerPC;
\item OpenRISC;
\item MICROBLAZE;
\item NIOS;
\item SPARC.
\end{exmpls}

Архитектуры бывают двух типов: CISC и RISC.
Основное различие: RISC "--- load-store architecture, CISC "--- нет.

\subsection{Микроархитектура (\texorpdfstring{\mu}{u}-arch)}

Intel Core "--- это не микроархитектура, а бренд.

Микроархитуктуры:
\begin{enumerate}
	\item SCR N;
	\item *lake (skylake, rocketlake, cometlake, \dots)
	\item broadwell, \dots
\end{enumerate}

Процессоры:
\begin{enumerate}
	\item Однотактовый.
	\item Многотактовый (loosely-coupled pipeline):
		\begin{enumerate}
			\item fetch;
			\item decode;
			\item execute;
			\item mem;
			\item write back.
		\end{enumerate}
	\item Конвейерная (tightly-coupled pipeline).
\end{enumerate}
