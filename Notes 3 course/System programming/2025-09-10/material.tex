\chapter{Введение}

Харриссы. Цифровая схемотехника и архитектура процессора.

Как будет строиться курс:
Транзисторы $ \to $ вентили $ \to $ комбинаторная логика $ \to $ последовательная логика $ \to $ CPU core $ \to $ SOC

\chapter{Транзисторы}

\section{Транзисторы p-типа и n-типа}

Транзисторы делаются из кремния.
У каждого из них 4 электрона, каждый кремний может соединиться с четырьмя кремниями.
Получается у каждого по 8 электронов.

Выкинем какой-нибудь кремний, на его место поставим мышьяк (у которого 5 электронов).
Получили 1 лишний электрон.
Это "--- \emph{полупроводник с электронной проводимостью}.
То есть, полупроводник \emph{n-типа} (n от слова negative).

Теперь заменим кремний на алюминий (у него 3 электрона).
Получилась дырка "--- $ sp^3 $-орбиталь, на которой нет электрона.
Это "--- \emph{полупроводник с дырочной проводимостью (p-типа)}.

\begin{remark}
	Ток течёт от + к -, дырки так же.
	Электроны бегут от - к +.
\end{remark}

\begin{figure}[!h]
	\centering
	\begin{subfigure}{0.49\textwidth}
		\centering
		\includegraphics[width=0.99\textwidth]{transistor-1}
		\caption{Транзистор p-типа}
	\end{subfigure}
	\begin{subfigure}{0.49\textwidth}
		\centering
		\includegraphics[width=0.99\textwidth]{transistor-2}
		\caption{Транзистор n-типа}
	\end{subfigure}
\end{figure}

Для p-типа: \\
Если к затвору подвести +, к базе - "--- ток (от истока к стоку) не течёт.
Наоборот "--- течёт.

Для n-типа "--- наоборот.

\begin{figure}[!h]
	\centering
	\begin{subfigure}{0.49\textwidth}
		\centering
		\includegraphics[height=2cm]{trans-icon-1}
		\caption{p-типа}
	\end{subfigure}
	\begin{subfigure}{0.49\textwidth}
		\centering
		\includegraphics[height=2cm]{trans-icon-2}
		\caption{n-типа}
	\end{subfigure}
	\caption{Обозначения в схемотехнике}
\end{figure}

FET "--- Field-effect transistor "--- полевой транзистор.

MOS "--- Metal-oxide semiconductor "--- МОП "--- роль изолятора играет оксид.

CMOS "--- КМОП "--- Complementary MOS "--- способ сделать на одном кристалле и p- и n-транзистор.

\section{Подключение на схемах}

- обозначается GND

+ обозначается $ V_{cc}, V_{ss}, V_{dd}, V_{ee}, V_c, V_s, V_d, V_e $.
\begin{itemize}
	\item s "--- source
	\item d "--- drain
	\item c "--- collector
	\item e "--- emitter
\end{itemize}
Буква удваивается, если этот уровень подключён к источнику напряжения.

Правильные варианты: $ V_{cc} $ (для биполярных транзисторов), $ V_{dd} $ (для полевых транзисторов).

К $ V_{dd} $ подключается исток p-типа или сток n-типа.
База подключается туда же.
